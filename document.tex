\documentclass[12pt,a4paper,oneside,cleardoublepage=empty,DIV12,leqno]{scrbook}

\setcounter{tocdepth}{3}
\setcounter{secnumdepth}{4}
\usepackage{pdfpages}
\usepackage{amsmath}
\usepackage{amsfonts}
\usepackage{amssymb}
\usepackage{textcomp}
\usepackage{graphicx}
\usepackage{floatflt}
\usepackage{microtype}
\usepackage{units}
\usepackage{tabularx}
\usepackage{wasysym}
\usepackage[normalem]{ulem}
\usepackage{enumitem}
\usepackage{tikz}
	\usetikzlibrary{trees,decorations.pathreplacing,decorations.markings,snakes}

\usepackage[english,ngerman]{babel}
% \usepackage[english,ngerman]{betababel}
\usepackage[T1]{fontenc}
\usepackage[utf8]{inputenc}
\usepackage[autostyle,german=guillemets,strict=true,english=british]{csquotes}
\newcommand{\threequote}[1]{\glq #1\grq}
\usepackage{makeidx}

\usepackage{oldstyle}
\usepackage{setspace}
\usepackage{soul}
\usepackage{verbatim}
\usepackage{xspace}
\usepackage{palatino}
\usepackage{hyperref}
\spacing{1.35}


\usepackage{fancyhdr}
\pagestyle{fancy}
\renewcommand{\chaptermark}[1]{%
        \markboth{\thechapter~#1}{}}
\renewcommand{\sectionmark}[1]{%
        \markright{\thesection~#1}}
\fancyhf{} % delete current header and footer
\fancyhead[LE,RO]{\textos{\thepage}}
\fancyhead[CO]{\textsc{\rightmark}}
\fancyhead[CE]{\textsc{\leftmark}}
\renewcommand{\headrulewidth}{0pt}
\renewcommand{\footrulewidth}{0pt}
\addtolength{\headheight}{0pt} % space for the rule
\fancypagestyle{plain}{%
   \fancyhead{} % get rid of headers on plain pages
   \renewcommand{\headrulewidth}{0pt} % and the line
}


\usepackage[style=authortitle-ibid,ibidpage=true,citetracker=true,backend=biber]{biblatex}
\renewbibmacro*{cite:title}{\indexnames[default]{author}%
	\printtext[bibhyperref]{%
		\printfield[citetitle]{labeltitle}}%
		\iffieldundef{year}{\iffieldundef{origyear}{}{%
		\printfield{origyear}}}{%
		~\mkbibparens{%
		\iffieldundef{origyear}{}{%
		\printfield{origyear}\addslash}\printfield{year}}}}

\renewbibmacro*{cite:shorthand}{\indexnames[default]{author}%
  \printtext[bibhyperref]{\printfield{shorthand}%
  }}%

\DeclareCiteCommand{\authorcite}
  {\boolfalse{citetracker}%
   \boolfalse{pagetracker}%
   \usebibmacro{prenote}}
  {\indexnames[default]{author}%
   \printnames{labelname}}
  {\multicitedelim}
  {\usebibmacro{postnote}}

\DeclareNameAlias{authorcite}{first-last}
\DeclareCiteCommand{\authorfullcite}
  {\boolfalse{citetracker}%
   \boolfalse{pagetracker}%
   \usebibmacro{prenote}}
  {\indexnames[default]{author}%
   \printnames[authorcite]{author}}
  {\multicitedelim}
  {\usebibmacro{postnote}}

\bibliography{/home/alexander/Dokumente/Bibliographien/Literatur.bib}
\DefineBibliographyStrings{german}{%
 shorthands = {Siglen},
}

% nach shorthand kein Komma (\postnotedelim), sondern nur Leerzeichen
\renewbibmacro*{postnote}{%
\iffieldundef{postnote}%
{}%
{\iffieldundef{shorthand}%
{\setunit{\postnotedelim}%
\printfield{postnote}}%
{\addspace\printfield{postnote}}%
}%
}%

\widowpenalty=10000
\clubpenalty=10000
\deffootnote[1em]{0em}{1em}{\textsuperscript{\thefootnotemark}}


 \renewenvironment{quote}{\list{}{\rightmargin1em\leftmargin1em}\item\relax
 \begin{singlespace}\begin{footnotesize}}{

\end{footnotesize}\end{singlespace}\endlist}

\newcommand{\vorschau}[1]{\phantomsection\addcontentsline{toc}{subsubsection}{#1}\textbf{#1}}

\makeatletter
\newcommand{\absatz}{%
   \refstepcounter{paragraph}%   Manually stepping paragraph counter without consulting secnumdepth
   \@startsection{subsection}{4}{\z@}%
     {3.25ex \@plus1ex \@minus.2ex}%
     {-0.5em}%
     {\normalfont\normalsize\bfseries\S~\thesubsection~}}%\space}} % Formatting the paragraph
\makeatother
\addtokomafont{sectioning}{\rmfamily}
\newcounter{ListCount}
\newcounter{BeispielCounter}

\sodef\so{}{.14em}{.4em plus.1em minus .1em}{.4em plus.1em minus .1em}
\newcommand{\beispielsatz}[1]{\begin{beispielsaetze}\addtocounter{BeispielCounter}{1}\item #1\end{beispielsaetze}}


\newcommand{\quine}[1]{\enquote{#1}}
\newcommand{\erwaehn}[1]{\enquote{#1}}
\newcommand{\distanz}[1]{\singlequote{#1}}
\newcommand{\verweis}[1]{\label{#1}\input{#1}}
\newcommand{\ori}[1]{\emph{#1}}
\newcommand{\oriII}[1]{\textsl{#1}}
\newcommand{\myemph}[1]{\emph{#1}}
\newcommand{\singlequote}[1]{\enquote*{#1}}
\newcommand{\leftquote}{\frqq}
\newcommand{\rightquote}{\flqq}
\newcommand{\punkt}{[\dots\unkern]\xspace}
\newcommand{\ohio}{o.\,H.\,i.\,O.}
\newcommand{\myherv}{Hervorhebung von mir}

\newcommand{\epoche}{Epoche}
\newcommand{\unterteilung}{\newpage}
\newcommand{\wolffsprech}[1]{\singlequote{#1}}
\newenvironment{these}{\begin{itemize}\item}{\end{itemize}}

\newenvironment{beispielsaetze}{\begin{list}{%
	(\thechapter{}.\arabic{BeispielCounter})}{\setlength{\listparindent}{\parindent}\parsep0em\leftmargin0em\addtolength{\itemindent}{\parindent}\addtolength{\itemindent}{\labelwidth}}%
		}{\end{list}}

\newenvironment{nummerierung}{\begin{list}{%
	\textos{\arabic{ListCount}})}{\usecounter{ListCount}\setlength{\listparindent}{\parindent}\parsep0em\leftmargin0em\addtolength{\itemindent}{\parindent}\addtolength{\itemindent}{\labelwidth}}%
		}{\end{list}}
\newcommand{\bemerkung}{\relax}


\newcommand{\randbemerkung}[1]{$^\ast$\marginpar{\footnotesize $^\ast$ #1}}
\newcommand{\name}[2][]{\index{#2, #1}#2}
\newcommand{\singlename}[1]{\index{#1}#1}
\newcommand{\titel}[1]{\enquote{\emph{#1}}}
\newcommand{\KapitelTitel}[1]{\emph{#1}}
\newcommand{\topos}[1]{\relax}
\newcommand{\luecke}{\textbf{LUECKE}}

\newtheorem{thesis}{These}
\newcommand{\jaeschelogik}{\singlequote{Jäsche-Logik}}
\newcommand{\kantcite}[4]{\cite[#2][#3]{#1},
\cite[][#4]{Kant:GesammelteWerke1900ff.}}


%%%% Revisionsbefehle

\newcommand{\Revision}[2][]{#2}

\author{Alexander Gunkel}
\title{Autonomie -- Metaphysik -- Endlichkeit}
\subtitle{Die Endlichkeit des menschlichen Denkens bei Kant}

\def\pfill{\unskip~\dotfill\penalty500
  \strut\nobreak\leaders\hbox to.6em{\hss.\hss}\hfill~\ignorespaces}
  
\newcommand{\keyw}[1]{\relax}
\renewcommand{\endinput}{\relax}
\makeindex
\begin{document}


\includepdf[pages={1}]{Titelblatt.pdf}
\mbox{ }
\thispagestyle{empty}
\newpage
\frontmatter
\pagestyle{fancy}
\tableofcontents
\newpage

\selectlanguage{english}
\section*{Abstract}\addcontentsline{toc}{chapter}{Abstract}\markboth{}{}
It is often maintained that German Enlightenment philosophy---and especially the
Enlightenment program that Immanuel \name[Immanuel]{Kant} articulated in his
famous article in the \titel{Berlinische Monatsschrift}---is not able to do
justice to the role of testimonial knowledge and hence is an obstacle to a down-to-earth
epistemology because of its essential individualistic stance. I
argue that this is a misunderstanding and that \name[Immanuel]{Kant} like other
Enlightenment thinkers actually recognizes these social factors and their impact on our
knowing and thinking. I show that \name[Immanuel]{Kant}'s philosophy can be
reconstructed as an attempt to show the compatibility of Enlightenment's demand
for independence and the essential dependency of finite cognizers. His mature philosophy aims at
developing a kind of Enlightenment Ethics of Belief that can be found in the
final sections of his \titel{Critique of Pure Reason}.

The central task of (a \name[Immanuel]{Kant}ian conception of) Enlightenment
points to intellectual independence, the \singlequote{thinking for oneself} and
the spontaneous activity of our own intellectual faculties that is not ruled by
others but only by ourselves. Now this activity should consist of the
actualization of the common faculty of reason and not in arbitrary
decision making to believe one thing over another. Thinking for oneself
thus seems to be the outcome of a competence rather than the result of a mere
decision. This is an insight which was famously defended by
\authorfullcite{Wolff:Discursuspraeliminarisdephilosophiaingenere1996}. While
\authorcite{Wolff:Discursuspraeliminarisdephilosophiaingenere1996} claims that
this competence is to be understood as being able to follow some special
method---the \emph{mathematical} method---\name[Immanuel]{Kant} points to the social
character of thinking. Thinking for oneself means competently reasoning in
accordance with rules that can only be followed and assessed through
conversational practice.

How is this demand for maturity affected by our finiteness? To answer this
question it is necessary to give a clear concept of our finiteness. According to
\name[Immanuel]{Kant}, the core of our finiteness as human beings  consists in
our dependency on being passively affected: on cognizing things with our senses
instead of through pure reason or by what we are told. We cannot cognize things
without being affected by them; all that we know by pure reason alone are very
general truths (for instance,  \enquote{things happen in accordance to natural
laws}). The \emph{terminus technicus} for this concept is
\singlequote{discursivity} which is the property of
concepts that distinguishes them from intuitions. A concept is a representation
that is discursive which means that it is mediately related to its objects via
universal marks and is hence a universal representation. Intuitions on the other
hand are immediately related to their objects and are hence singular
representations. Our understanding is discursive because it is a faculty of
thinking which cognizes objects through concepts that are dependent on a
faculty of intuitions to relate to objects. Without such
\singlequote{intuitive}, i.\,e. immediate relations our understanding would have
no content at all.

Most work on testimonial knowledge focuses on the trustworthiness and competence
of the speaker. Although \name[Immanuel]{Kant} addresses such topics,
these are rather secondary reflections unsuited to be an adequate foundation for
understanding of the compatibility of intellectual independence and the
dependency on others. In freely following Wolff Kant distinguishes in a first
step between rational and historical cognitions. A cognition is rational if it
is generated by the cognizer's own use of his pure reason, otherwise it is
historical, i.\,e.
it is given to him by his senses or the words of others. In a second step he
makes a difference between rational and empirical cognitions, the latter being
cognitions that can only be known historically. While the distinction between
rational and \emph{empirical} cognitions signifies a difference in their very
character, the difference between rational and \emph{historical}
cognitions means a difference in our own personal access to them. (In a
further step he distinguishes between \singlequote{discursive} rational
cognitions in philosophy and \singlequote{intuitive} rational cognitions in
mathematics, but this is of lesser importance than the first distinction.) The
point is that different kinds of cognitions are to be treated in different ways.
Having a critical stance toward testimonial knowledge means that we should not accept
rational cognitions on the grounds of being told.

Knowingly diverging from his predecessors \name[Immanuel]{Kant} defines
metaphysics as the system of rational cognitions, i\,e. the systematically
ordered body of knowledge we gained through pure reason. Thus metaphysics is the
discipline that does not allow for just historical cognitions. Now to
\name[Immanuel]{Kant} philosophy is not---as it has been for
\authorcite{Wolff:Discursuspraeliminarisdephilosophiaingenere1996} and
others---the whole body of science besides mathematics, but is rather identical
with metaphysics. Summarizing all these results we can assign a reason for why
we should learn philosophizing rather than philosophy: It is because metaphysics
is a rational discipline that we cannot learn by acquiring mere historical
knowledge but only by learning to use our own faculty of reason.
\selectlanguage{ngerman}

\mainmatter
\flushbottom



\chapter{Einleitung}

Unter dem Titel einer \emph{Sozialen Erkenntnistheorie} florieren seit einigen
Jahrzehnten Überlegungen dazu, wie wir in unserem je eigenen Wissen und Erkennen
von der Gemeinschaft abhängig sind, in der wir leben und aufgewachsen
sind.\footnote{Einen Überblick geben \cite{Schmitt:Introduction2010},
\cite{Wilholt:SozialeErkenntnistheorie2007}, sowie
\cite{Scholz:DasZeugnisanderer2001}.} Dabei handelt es sich -- allen
anders lautenden Bekundungen zum Trotz\footnote{Siehe etwa
\cite[][529--531]{Grundmann:AnalytischeEinfuehrungindieErkenntnistheorie2008},
sowie \cite[][46]{Wilholt:SozialeErkenntnistheorie2007}.} -- keineswegs um eine
neue Stoßrichtung der Erkenntnistheorie.\footnote{\cite[Vgl.][46]{Scholz:DasZeugnisanderer2001}.}
Nicht nur \authorcite{Hume:AnEnquiryConcerningHumanUnderstanding1964} und
\authorcite{Reid:EssaysontheIntellectualPowersofMan2002}, sondern bereits
\authorcite{Descartes:OeuvresdeDescartes1983},
\authorcite{Spinoza:EthikingeometrischerOrdnungdargestellt2007}, verschiedene
Autoren der deutschen Aufklärung und nicht zuletzt \name[Immanuel]{Kant} setzen
sich mit den sozialen Grundlagen unseres Denkens und Erkennens
auseinander. Das Wissen und Können anderer -- Eltern, Lehrer, Traditionen -- sind
die ersten Quellen unseres je eigenen Wissens und Könnens. Dass wir bei diesem
ursprünglichen Wissenserwerb alles andere als mündig und selbständig sind, ist
der Ausgangspunkt von \authorcite{Descartes:OeuvresdeDescartes1983}'
Überlegungen zur Verlässlichkeit unserer
Überzeugungen.\footnote{\cite[Vgl.][VI:
13.1--12]{Descartes:OeuvresdeDescartes1983}.
Zur Bedeutung für die deutsche Aufklärungsphilosophie siehe
\cite[][104]{Schneiders:AufklaerungundVorurteilskritik1983}.} Dessen Behauptung,
wir könnten versuchen, das \emph{corpus} unseres Wissens auf
individualistischer Grundlage neu zu errichten, ist  auch im 17. und 18.
Jahrhundert nicht mehrheitsfähig. Selbst unser alltäglichstes Wissen,
beispielsweise unser Wissen um das eigene Geburtsdatum und die eigene Herkunft
sind -- so erinnert uns \authorcite{Spinoza:SpinozaOpera1972} im \titel{Tractatus de intellectus
emendatione} -- Dinge, die wir ausschließlich von anderen erfahren
können.\footnote{\cite[Vgl.][II: 10.22--24]{Spinoza:SpinozaOpera1972}.} Erst
auf der Grundlage solchen Wissens und von anderen erlernter Fähigkeiten beginnen
wir, selbständig  Wissen zu generieren. Und dass wir irgendwann anfingen, nur
noch solche Überzeugungen neu zu erwerben, von deren Wahrheit wir uns selbst
überzeugen können, ist weder realistisch noch vernünftig.
\name[Immanuel]{Kant} sagt, wir könnten uns einen solchen
\singlequote{historischen Unglauben} -- den Verzicht darauf, Überzeugungen
auf der Grundlage von Mitteilungen zu bilden -- gar nicht als vorsätzlich
vorstellen; zu absurd und realitätsfern wäre diese
Haltung.\footnote{\cite[Vgl.][A
328]{Kant:Washeisst:SichimDenkenorientieren?1977}, \cite[][VIII:
146.8--11]{Kant:GesammelteWerke1900ff.}.}

In der Neuzeit haben (mitunter dieselben) Philosophen gefordert, sich von
den Einflüssen durch andere zu emanzipieren, selbst zu denken, statt Lehrern und
Traditionen zu folgen. Mündig sollen wir sein, nicht autoritätshörig, im Denken
wie im Handeln dem \emph{eigenen} Verstand folgen, statt zu tun und für wahr zu
halten, wozu andere uns anhalten -- das war und ist eine Forderung, die sich mit
dem Etikett \enquote{Aufklärung} schmückt. Doch sie scheint in
Konflikt zu geraten mit der Einsicht in die sozialen Grundlagen unseres
Wissens. Deswegen müssen wir fragen, was es eigentlich
heißen soll, wir sollten \emph{selbst} denken, unseren \emph{eigenen} Verstand
und die \emph{eigene} Vernunft gebrauchen. Solange wir diese Fragen nicht
hinreichend beantworten können, bleibt jede Anknüpfung an \singlequote{die}
Aufklärung -- zumindest in der Form, die \name[Immanuel]{Kant} ihr in dem
berühmten Aufsatz gab -- leer und unbestimmt.

Wir verstehen uns als Erben der Aufklärung, verwenden sorglos das Adjektiv
\enquote{aufgeklärt} und wenden es ohne Zögern auf uns selbst und
unseren Kulturkreis an. Doch nur selten  fragen wir, worin \singlequote{die} Aufklärung
eigentlich besteht. Dabei ist unklar, ob es einen einheitlichen
Sinn gibt, in dem wir von Aufklärung und Mündigkeit sprechen, und ob diese
Ausdrücke ein vernünftiges Projekt
beschreiben.\footnote{\cite[Vgl.][41]{Stekeler-Weithofer:Denken2012}:
\enquote{Doch es ist erst noch einmal zu bedenken, ob wir einer derartigen
Selbstdarstellung der Aufklärung überhaupt folgen können. Denn was heißt hier
Denken und Selbstdenken? Gegen wen oder was richtet sich das
\singlequote{Selbst}?} Siehe auch \cite[][42]{Stekeler-Weithofer:Denken2012}:
\enquote{Wenn sich Aufklärung dann aber auch gegen eine \singlequote{Disziplinlosigkeit des Denkens}
richtet und Autonomie damit offenbar ähnlich wie die Kompetenz zu herrschen erst
einmal Disziplin und Ordnung, vielleicht sogar Abrichtung, voraussetzt, bemerken
wir, dass der scheinbar klaren Entgegensetzung von Autonomie und der Autorität
der Tradition eine viel komplexere Struktur zugrunde liegt.} \authorfullcite{Stuke:Aufklaerung1972}
behauptet, dass \name[Immanuel]{Kant} den Begriff Aufklärung uneinheitlich
verwende und seine Aufklärungskonzeption in sich inkonsistent
sei \parencite[vgl.][265--272]{Stuke:Aufklaerung1972}.} Möglicherweise
beschreibt der Terminus \enquote{Aufklärung} nur eine zurückliegende Epoche der
europäischen Geistesgeschichte, aber kein in sich schlüssiges Programm, auf das wir
uns heute noch berufen könnten. Und vielleicht sind die Forderungen, die sich
hinter Begriffen wie \enquote{Mündigkeit} und \enquote{Selbstdenken} verbergen,
angesichts unserer Lebenswirklichkeit und kognitiven Ausstattung viel zu
anspruchsvoll. Insbesondere der Konflikt zwischen der Forderung nach
Selbständigkeit und (epistemischer) Unabhängigkeit auf der einen und der
eingangs beschriebenen Abhängigkeit von Anderen auf der anderen Seite lässt die
Forderung der Aufklärung als unrealistisch
erscheinen.\footnote{\phantomsection\label{Anmerkung:KantundderDoxastischeVoluntarimus}\authorfullcite{Scholz:KantsAufklaerungsprogramm2009}
diskutiert des Weiteren den Einwand, Aufklärung setze eine Form des
\emph{doxastischen Voluntarismus}
voraus \mkbibparens{\cite[vgl.][38--40]{Scholz:KantsAufklaerungsprogramm2009}.}
Unter doxastischem Voluntarismus verstehen wir die Position, dass wir
willentlich kontrollieren können, welche Behauptungen wir für wahr halten. Wenn
wir uns in der Annahme von Überzeugungen auf bestimmte Art und Weise
verhalten sollen, dann -- so scheint es -- muss es uns möglich sein, selbst zu
entscheiden, welche Überzeugungen wir annehmen und welche nicht. Wir müssten
frei sein in der Entscheidung, was wir für wahr halten. Nun scheint der
doxastische Voluntarismus falsch zu sein, denn er widerspricht unserer
alltäglich Erfahrung: Wenn wir wahrnehmen, dass es regnet, dann können wir
nicht \emph{ad libitum} eine andere Überzeugung annehmen, etwa die, dass es
schneit. Der doxastische Voluntarismus wird aus diesem Grund auch von
\name[Immanuel]{Kant} zurückgewiesen \mkbibparens{\cite[vgl.][\nopp
2508]{Kant:Reflexionen1900ff.}, \cite[][XVI:
398.11--14]{Kant:GesammelteWerke1900ff.}. In der {\jaeschelogik} steht: \enquote{Unmittelbar hat der Wille keinen Einfluß
auf das Fürwahrhalten; dies wäre auch sehr ungereimt} \mkbibparens{\cite[][A
113]{Kant:ImmanuelKantsLogik1977}, \cite[][IX:
73.33--34]{Kant:GesammelteWerke1900ff.}}. Siehe dazu auch
\cite[][]{Cohen:KantontheEthicsofBelief2014}, sowie
\cite{Cohen:KantonDoxasticVoluntarismanditsImplicationsfortheEthicsofBelief2013}.
Dagegen behauptet \authorfullcite{Chignell:KantsConceptsofJustification2007},
dass \name[Immanuel]{Kant} zumindest bezüglich mancher Arten von Überzeugungen
Voluntarist sei
\parencite[vgl.][36]{Chignell:KantsConceptsofJustification2007}.}}

Diese kritischen Nachfragen stellen keine Perspektive dar, mit der erst wir heutigen Interpreten
uns der Aufklärung und ihrem Programm zuwenden: Neben der Frage, was derjenige,
der Aufklärung einklagt, von seinen Mitmenschen eigentlich fordert, stritten
Philosophen des 18. Jahrhunderts auch darüber, ob der Mensch mit seinen
eingeschränkten intellektuellen Fähigkeiten der Aufklärung überhaupt fähig ist.
Für die Einschränkungen unserer intellektuellen oder kognitiven
Leistungsfähigkeit, insofern sie uns nicht zufällig als dieses oder jenes
Individuum betreffen, sondern die Beschränktheiten meinen, denen wir \emph{als Menschen} unterliegen, steht der Begriff der \emph{Endlichkeit}.
Zu zeigen, dass die Aufklärungsformel nicht leer ist, beinhaltet, ein kantisches
Aufklärungsprogramm zu rekonstruieren, das auf der Forderung nach Mündigkeit
(nach einer aktiven, selbständigen und vorurteilsfreien Art zu denken) gründet,
dabei aber auch unsere Endlichkeit als zentralen Bestandteil der
\emph{conditio humana} berücksichtigt. Es geht daher darum, eine
\singlequote{\emph{ethics of belief}}\footnote{Der Ausdruck stammt von
\authorfullcite{Clifford:TheEthicsofBelief1877}, der ihn 1877 als Titel eines
Textes wählt, in dem er dafür argumentiert, dass wir für unsere Überzeugungen
auch in ethischer Hinsicht verantwortlich sind
\parencite[siehe][\pno~189\,f.]{Clifford:TheEthicsofBelief1877}.} zu entwickeln,
die zwischen der Forderung der Aufklärung nach Unabhängigkeit und
Selbständigkeit auf der einen und unserer Endlichkeit auf der anderen Seite
vermittelt.  Ich möchte also zeigen, dass aus dem \enquote{sapere aude!}
konkrete Regeln erwachsen, die zu befolgen sind, um mündig zu sein.

Wenn hier von der Endlichkeit des Menschen gesprochen wird, dann geht es um die
Endlichkeit in der Ausübung seiner kognitiven Vermögen, also um die Endlichkeit
des \emph{Verstandes} bzw. der (theoretischen wie praktischen) \emph{Vernunft}. Es
wird sich zeigen, dass wir nach \name[Immanuel]{Kant} endlich sind,
insofern wir in uns in unserem Denken, Erkennen
und Handeln als \emph{abhängig} erweisen.\footnote{\cite[Vgl.][B
72]{Kant:KritikderreinenVernunft2003}, \cite[][III: 72.29--73.4]{Kant:GesammelteWerke1900ff.}. Siehe auch
\cite[][\S~10]{Kant:Demundisensibilisatqueintelligibilisformaetprincipiis1968},
\cite[][II: 396.19--397.4]{Kant:GesammelteWerke1900ff.}. Nach
\name[Immanuel]{Kant}s Auskunft in der \titel{Kritik der reinen Vernunft} ist
unser Verstand endlich, insofern er nur denkt, nicht aber anschaut.
Er ist in diesem Sinne ein \singlequote{diskursiver}, kein intuitiver
Verstand. Man beachte aber, dass \name[Immanuel]{Kant} in der \titel{Kritik
der reinen Vernunft} kein einziges Mal von einem diskursiven oder intuitiven
Verstand spricht; erst in \S~57 der \titel{Prolegomena zu einer jeden künftigen
Metaphysik, die als Wissenschaft wird auftreten können} und in \S~77 der
\titel{Kritik der Urteilskraft} findet sich diese Bezeichnungsweise. Bis dahin sind es
bewusste objektive Vorstellungen, die als diskursiv (Begriffe) oder intuitiv (Anschauungen)
bezeichnet werden, sowie Erkenntnisse \emph{ex principiis}, die sich in
diskursive Vernunfterkenntnisse (Philosophie) und intuitive Vernunfterkenntnisse
(Mathematik) unterteilen. Die Bezeichnung unseres Verstandes als diskursiv ist
eine spätere Übertragung der Bedeutung, die aussagt, dass der Verstand eben ein
Vermögen der Begriffe, nicht der Anschauungen ist. Siehe dazu Kap.
\ref{subsection:DiskursiverVerstandundsinnlicheAnschauung}.} Ein unendliches Wesen
wäre insofern vollkommen unabhängig oder zeichnete sich -- wie
\name[Immanuel]{Kant} 1763 schreibt -- durch \singlequote{Allgenugsamkeit}
aus.\footnote{\cite[Vgl.][A
186\,f.,]{Kant:DereinzigmoeglicheBeweisgrundvomDaseinGottes1977}
\cite[][II: 154.4--19]{Kant:GesammelteWerke1900ff.}.} Unser Verstand ist
endlich,\footnote{Man ist versucht, von der Endlichkeit unseres Verstandes oder
auch unseres Denkens zu sprechen und dagegen Gott ein unendliches Denken zuzusprechen, doch dies ist im Rahmen der
sprachlichen Gepflogenheiten \name[Immanuel]{Kant}s mindestens ungenau.
\name[Immanuel]{Kant} sagt, unser Verstand sei endlich, insofern er denkt und
nicht anschaut. Es ist also der Verstand -- und nicht das Denken -- welcher
endlich ist, insofern Denken die Tätigkeit ist, die er auszuführen
vermag. Ein unendlicher Verstand dächte nicht, sondern schaute; ein
unendliches Denken kann es nicht geben, weil Denken gerade die Tätigkeit ist,
die unserem Verstand als endlichem
zukommt \mkbibparens{\cite[vgl.][B 71]{Kant:KritikderreinenVernunft2003},
\cite[][III: 72.10--16]{Kant:GesammelteWerke1900ff.}, wonach Denken stets das
Vorliegen von Schranken beweise}. Denken -- so ließe sich auch sagen -- ist
\emph{per se} endlich, der Ausdruck \enquote{Endlichkeit des Denkens}
beschreibt eine Tautologie. Eine unendliche (das hieße: intellektuelle)
Anschauung wiederum wäre die Erkenntnis eines Vermögens der
Spontaneität, welches nicht der Rezeptivität unserer Sinne bedürfte.}
weil er als von Rezeptivität abhängiger Verstand ohne Sinnlichkeit nichts zu erkennen
vermag. Damit beschreibt die Endlichkeit eine Spannung innerhalb des oberen
Erkenntnisvermögens: Als Verstand ist dieses selbsttätig oder
\singlequote{spontan}, nur aus sich selbst heraus handelnd und unabhängig von
äußeren Bestimmungen. Als endlich ist unser Verstand wiederum abhängig. In
dieser Beschreibung der Endlichkeit kommt damit der Widerstreit zweier
Grundbestimmungen der Aufklärungsprogrammatik wieder zum Vorschein: Unserer
Unabhängigkeit und Selbständigkeit steht unsere Abhängigkeit oder Endlichkeit entgegen.

\Revision[Pelletier]{Die Tatsache, dass wir auf epistemische Vorarbeiten
Anderer angewiesen sind, stellt eine Ausprägung oder Instantiierung unserer
allgemeinen Endlichkeit dar. Die Forderung der Aufklärung, die eigene Vernunft
zum obersten Kriterium der Wahrheit zu machen, provoziert die Frage, wie diese
Forderungen mit der Tatsache kompatibel ist, dass wir die meisten Erkenntnisse
nicht aus unserer Vernunft, sondern empirisch -- aus dem Gebrauch der Sinn --
erlangen. Unser Wissen muss uns in aller Regel \emph{gegeben} werden. Um diese
Frage zu beantworten ist es sinnvoll, sich mit derjenigen Ausprägung zu
befassen, die das Projekt der Aufklärung am offensichtlichsten anzugreifen
scheint: der Angewiesenheit auf Erkenntnisse, die wir lediglich auf die
Autorität anderer hin als Wissen anerkennen. Es wird sich zeigen, dass die
Auflösung dieser Frage auch allgemein auf den scheinbaren Konflikt zwischen
Aufklärungsforderung und Abhängigkeit von empirischer Erkenntnis anwendbar ist.}

\section{Inhaltliches Vorgehen}
In Kapitel \ref{section:KantalsliberalerAufklaerer} expliziere ich die
Grundstrukturen des kantischen Aufklärungsbegriffs und erarbeite die
mit ihm verbundenen Schwierigkeiten und Herausforderungen. Die
Endlichkeit des Menschen ist Thema des \ref{chapter:endlichkeitmenschlichendenkens}.
Kapitels. Hier zeige ich, dass \name[Immanuel]{Kant} in einem einheitlichen Sinne
von der Endlichkeit unseres Verstandes spricht, die sich sowohl in der
Abhängigkeit von rezeptiv gewonnenen Informationen, als auch in dem nötigenden
Charakter der praktischen Vernunft und schließlich in der in \S~77 der
\titel{Kritik der Urteilskraft} beschriebenen Besonderheit zeigt, dass
wir vom Analytisch-Allgemeinen (von Begriffen) ausgehend   zum
Besonderen gehen müssen und nicht von einem
Synthetisch-Allgemeinen (der Anschauung des Ganzen als eines solchen)
ausgehen können. Kapitel \ref{chapter:AufklaerungundWissenschaft}
zeigt auf, dass Aufklärung und Mündigkeit primär (aber nicht
ausschließlich) unseren praktischen 
Vernunftgebrauch betreffen und auf die Endlichkeit des \emph{Willens} -- d.\,i. der
praktischen Vernunft -- verweisen. Die Kapitel
\ref{section:autonomieunddaszeugnisanderer},
\ref{chapter:MuendigerErwerbTestimonialenWissens} und
\ref{Chapter:KantsSocialEpistemology} behandeln das epistemische
Grundproblem des Aufklärungsprogramms: die Spannung zwischen
emanzipatorischem Anspruch und sozialer Wirklichkeit des Denkens und
Erkennens. Kapitel \ref{section:autonomieunddaszeugnisanderer} wird dieses
Grundproblem als solches ausarbeiten, wie es heute in der
\enquote{Soziale Erkenntnistheorie} unter dem Stichwort \enquote{testimoniales Wissen} diskutiert
wird: Während Aufklärung von uns Unabhängigkeit verlangt, sind wir
doch darin  abhängig, dass wir darauf angewiesen sind, Wissen auf
die Autorität anderer hin zu übernehmen. Die Vereinbarkeit von
aufklärerischer Forderung nach Selbständigkeit und Abhängigkeit von
testimonialem Wissen wird die Themen der folgenden Kapitel bestimmen: Kapitel
\ref{chapter:MuendigerErwerbTestimonialenWissens} zeigt auf, wie zwei
verschiedene Perspektiven auf diese Frage in der deutschen Aufklärung zu
unterscheiden sind. Dies dient der Unterscheidung einer bei
\name[Immanuel]{Kant} und anderen Aufklärern vorherrschenden Perspektive von
derjenigen, die ebenso \name[David]{Hume}s Herangehensweise leitet wie
diejenige heutiger Philosophen. Kapitel \ref{Chapter:KantsSocialEpistemology}
erläutert schließlich, wie \name[Immanuel]{Kant}s an
\authorcite{Wolff:Discursuspraeliminarisdephilosophiaingenere1996} anschließende
Herangehensweise und Lösung aussieht, der es um die Unterscheidung bloß
historischer Kenntnisse von mündigem Wissen geht. Darin wird Kapitel
\ref{Absatz:AufklaerungundZugangsInternalismus} zeigen, dass
\name[Immanuel]{Kant}s Position mitnichten individualistisch ist.
Kapitel \ref{section:MuendigkeitundPhilosophie} fragt danach, wie die Forderung
nach Unabhängigkeit im Denken und Erkennen dennoch ihren Gehalt und ihre
Berechtigung hat, und zeigt auf, welche zentrale Rolle der Metaphysik
zukommt. Kapitel \ref{section:MetaphysikausderPerspektivedesMenschen} verbindet
\name[Immanuel]{Kant}s Metaphysikbegriff mit dem Begriff der Autonomie
(als Charakteristikum oberer Erkenntnisvermögen) und findet auf diesem
Weg Anschluss an die Überlegungen zur
Aufklärung. In Kapitel \ref{section:KantsEthicsofBelief} wird
abschließend auf der Grundlage von \name[Immanuel]{Kant}s
Unterscheidungen zwischen Überredung und Überzeugung die Grundstruktur
einer kantischen \singlequote{\emph{Ethics of Belief}} erarbeitet. 


\section{Forschungsstand}
Der Konflikt zwischen Aufklärungsprogramm und Endlichkeit ist der Forschung
freilich längst bekannt. Erstaunlicherweise wurde er aber nicht zum Gegenstand
eigenständiger Untersuchungen.
Als offenes Problem der Aufklärungsforschung benannt, aber nicht ausgearbeitet
findet er sich beispielsweise bei
\authorfullcite{Engfer:ChristianThomasius1989}, der den Konflikt folgendermaßen
beschreibt: Auf der einen Seite stehe eine aufklärerische Tradition, die dem
Menschen Selbstaufklärung und Selbstbestimmung abverlange, auf der anderen Seite
die der protestantischen Tradition entstammende Überzeugung, dass dem Menschen
als endlichem Wesen gerade die Fähigkeit hierzu
fehle.\footnote{\cite[Vgl.][36]{Engfer:ChristianThomasius1989}.} Als Gegenpol zu
der von \authorcite{Engfer:ChristianThomasius1989} vertretenen
Problemschilderung tritt \authorfullcite{Schnaedelbach:WirKantianer2005} auf,
der die Aufklärung als geradezu in der Einsicht in unsere Endlichkeit fundiert
begreift. Er zählt das Programm der Aufklärung und die Einsicht in unsere
Endlichkeit zu den bewahrenswerten und grundlegenden Aspekten der Philosophie
Kants. Als Philosoph der Endlichkeit verkörpere Kant zugleich die Aufklärung und
die moderne Kultur, denn die Einsicht in unsere Endlichkeit unterminiere die
Ansprüche einer überkommenen Metaphysik und gerade dies mache das Wesen der Aufklärung
aus.\footnote{\cite[Vgl.][passim]{Schnaedelbach:WirKantianer2005}. Siehe
auch \cite[][100]{Schnaedelbach:Vernunft2007}, sowie
\cite[][976]{Schnaedelbach:PhilosophiealsGespraech2012}: \enquote{Die Frage ist
natürlich, was es heißt, jetzt Kantianer zu sein? Seine Raumkonzeption, die
Synthesistheorie des Urteils, das ganze Verhältnis von Sinnlichkeit und Verstand
kann man sich so nicht mehr zu Eigen machen. Kantianer zu sein, ist dann fast
nur noch eine Frage des Stils, -- eines Stils, der sich an der Endlichkeit
unserer Vernunft orientiert.} Die Deutung \name[Immanuel]{Kant}s als des Philosophen
der modernen Kultur orientiert sich an
\textcite[vgl.][]{Rickert:KantalsPhilosophdermodernenKultur1924}.} Auch
\authorfullcite{Hinske:KantalsHerausforderungandieGegenwart1980} sieht eine
klare Verbindung zwischen der Betonung unserer Endlichkeit und der
Aufklärung.\footnote{\enquote{Philosophie der Aufklärung ist demgemäß ihrem
eigenen Impuls zufolge Philosophie der endlichen Vernunft. Sie lebt und denkt
aus der Einsicht, daß das Ganze, die Totalität der Wahrheit, auf die Vernunft
aus ist, dem Menschen nicht gegeben, sondern nur aufgegeben ist. Sie steht im
Horizont des Unbedingten, aber sie befindet sich nicht in seinem Besitz. Sie
leidet an dieser ihrer Endlichkeit, aber sie lügt sich nicht über sie hinweg.
Eben das aber gibt ihr zugleich auch die Offenheit, die Vernunft des Anderen:
die Vernunft jedes Anderen, als ein Stück der allgemeinen Vernunft zu begreifen
und ernstzunehmen}
\parencite[][38]{Hinske:KantalsHerausforderungandieGegenwart1980}.} Die These von einem
inneren Widerstreit zwischen der Einsicht in unsere Endlichkeit und dem
Aufklärungsdenken wird also durchaus bestritten -- bis hin zur Behauptung,
Aufklärung sei Ausdruck der Einsicht in unsere Endlichkeit.

Gemeinsam ist den genannten Arbeiten, dass sie den Begriff der Endlichkeit nicht
explizit zum Thema machen. Bereits \authorfullcite{Heidegger:KantunddasProblemderMetaphysik1965}
macht darauf aufmerksam, dass die Bedeutung der Aussage, der Mensch
sei endlich, und damit auch \name[Immanuel]{Kant}s Verständnis
unserer Endlichkeit klärungsbedürftig ist.\footnote{\enquote{Wie soll nach der Endlichkeit im
Menschen gefragt werden? Ist das überhaupt ein ernsthaftes Problem? Liegt die
Endlichkeit des Menschen nicht allerorts und jederzeit tausendfältig zutage?\\
So mag es schon genügen, Endliches am Menschen zu nennen, aus seinen
Unvollkommenheiten beliebige anzuführen. Auf diesem Wege gewinnen
wir allenfalls Belege dafür, daß der Mensch ein endliches Wesen
ist. Wir erfahren aber weder, worin das Wesen seiner Endlichkeit
besteht, noch gar, wie diese Endlichkeit den Menschen als das
Seiende, das er ist, von Grund aus im ganzen
bestimmt} \parencite[][198]{Heidegger:KantunddasProblemderMetaphysik1965}.} Bis
heute gibt es jedoch keine eigenständige Untersuchung der Frage, was die
Endlichkeit des Menschen nach \name[Immanuel]{Kant} ausmacht und wie sie sich
zum Projekt der Aufklärung verhält.\footnote{Nicht einschlägig ist hier die
Auffassung \name[Immanuel]{Kant}s von einem mathematischen Endlichen und
Unendlichen, wie es etwa in der Antithetik diskutiert wird. Dies werde ich in
Kap. \ref{subsection:QuantitativeundqualitativeUnendlichkeit} darlegen.}
Mitunter wird sie eher beiläufig und ohne nähere Auseinandersetzung
beantwortet,\footnote{So behauptet etwa
  \authorfullcite{Sandkuehler:KantsenquoteRevolutionderDenkungsart2005}:
\enquote{Was Kant letztlich mit der systematischen Kritik der Erkenntnis
anstrebt, ist Aufklärung -- der \enquote{Ausgang aus selbstverschuldeter
Unmündigkeit}. Es geht ihm um die Ermöglichung von Selbstdenken ohne Widerspruch
und mit Rücksicht auf andere}
\parencite[][93]{Sandkuehler:KantsenquoteRevolutionderDenkungsart2005}. Gründe
für diese Interpretation liefert er jedoch nicht.} damit aber gerade nicht als
drängendes Problem wahrgenommen. Dabei ist die Frage nach unserer Endlichkeit
gerade im Zusammenhang mit dem Programm der Aufklärung nicht nur
philosophiegeschichtlich höchst interessant, sondern betrifft eine Grundfrage
von bleibender Aktualität. Denn zum einen möchte auch heute niemand hinter die
Aufklärung zurück. Selbst die vielfachen Forderungen nach einer
\enquote{Aufklärung der Aufklärung}\footcite[Vgl.][]{Gutschmidt:AufklaerungderAufklaerung2012} oder die
Behauptungen einer \enquote{Dialektik der
Aufklärung}\footcite[Vgl.][]{Horkheimer:DialektikderAufklaerung1997}
beanspruchen ja eine Weiterentwicklung und keine Zurücknahme der Aufklärung
\emph{toto genere}. Auf der anderen Seite bestreitet niemand unsere grundlegende
Endlichkeit. Auch Hegelianer wie \authorfullcite{Stekeler-Weithofer:TheQuestionofSystem2006}
behaupten, unsere Endlichkeit lediglich besser verstehen zu wollen. Sie
entlasten Hegel von Behauptungen, die das Unendliche betreffen,
oder unterstellen ihm einen unverfänglichen Sinn von
Unendlichkeit.\footnote{\authorfullcite{Stekeler-Weithofer:TheQuestionofSystem2006}
schreibt über \name[Georg Wilhelm Friedrich]{Hegel}: \enquote{Grob gesagt schlägt er vor, die Rede über
Unendlichkeit im allgemeinen, über ein absolutes oder unendliches Wissen im
besonderen als ideale Rede zu begreifen. Sie ist Rede über die invariante Form
einer guten Entwicklung der je in ihrer Endlichkeit relativen Wahrheiten, des je
bestmöglichen Wissens und Bereifens}
\parencite[][186]{Stekeler-Weithofer:PhilosophiedesSelbstbewusstseins2005}.
Siehe auch \cite[][98]{Stekeler-Weithofer:TheQuestionofSystem2006}.
Ähnliches behauptet
\cite{Chiereghin:WozuHegelineinemZeitalterderEndlichkeit?1998}; dieser
Interpretation widerspricht wiederum \cite{Philipsen:NichtsalsKontexte2000}.}
Und \authorfullcite{Foerster:Die25JahrederPhilosophie2011}, der zuletzt durch
die an \name[Johann Wolfgang von]{Goethe} orientierte Behauptung auffiel, wir
verfügten über einen intuitiven Verstand, stellt zugleich klar, dieser
sei nicht mit einem unendlichen, göttlichen
Verstand zu identifizieren; vielmehr handle es sich auch bei dem Verstand, den
\name[Immanuel]{Kant} selbst in der \titel{Kritik der Urteilskraft} unserem
diskursiven Verstand entgegenstellt, um einen \enquote{\ori{endlichen}
intuitiven
Verstand}\footnote{\cite[][256]{Foerster:Die25JahrederPhilosophie2011}.}.

Obwohl die Grundfrage dieser Arbeit bisher keine eigenständige Untersuchung zu
Tage förderte, sind zentrale Themen freilich bereits in der
\name[Immanuel]{Kant}forschung diskutiert worden.
\authorfullcite{Engfer:MenschlicheVernunft2002},
\authorfullcite{Foerster:DieBedeutungvonSS7677deremphKritikderUrteilskraftfuerdieEntwicklungdernachkantischenPhilosophieTeil12002}
und \authorfullcite{Nuzzo:KritikderUrteilskraftSS76--772009} diskutieren die
\S\S~76 und 77 der \titel{Kritik der Urteilskraft}.
\authorcite{Engfer:MenschlicheVernunft2002} argumentiert dafür, dass zentrale
Ansätze der kritischen Philosophie von Annahmen über die Beschaffenheit unseres
menschlichen Erkenntnisvermögens abhängig
sind.\footcite[Vgl.][]{Engfer:MenschlicheVernunft2002}
\authorcite{Foerster:Die25JahrederPhilosophie2011} möchte nachweisen, dass
\name[Immanuel]{Kant} keinen einheitlichen Begriff unserer Endlichkeit
verwendet, sondern auf verschiedene Besonderheiten unseres Erkenntnisvermögens
verweist.\footnote{Siehe
\cite{Foerster:DieBedeutungvonSS7677deremphKritikderUrteilskraftfuerdieEntwicklungdernachkantischenPhilosophieTeil12002,Foerster:DieBedeutungvonSS7677deremphKritikderUrteilskraftfuerdieEntwicklungdernachkantischenPhilosophieTeil22002},
und \cite{Foerster:Die25JahrederPhilosophie2011}. Dasselbe behauptete
\textcite[vgl.][153--159]{McLaughlin:KantsKritikderteleologischenUrteilskraft1989}.}
\authorcite{Nuzzo:KritikderUrteilskraftSS76--772009} wiederum behauptet, dass es
sich in den \S\S~76 und 77 der \titel{Kritik der Urteilskraft} stets um
Beschreibungen derselben Besonderheit unseres Verstandes handelt -- der
Diskursivität -- und dass \name[Immanuel]{Kant} in diesen Passagen die Abgeschlossenheit der
Transzendentalphilosophie aufzeige.\footnote{Siehe
\cite{Nuzzo:KritikderUrteilskraftSS76--772009}, sowie
\cite[][348--353]{Nuzzo:KantandtheUnityofReason2005}. \enquote{\S\S~76--77 prove that Kant's transcendental philosophy
is already completed, and that it cannot be corrected without inaugurating an
utterly different paradigm}
\parencite[][146]{Nuzzo:KritikderUrteilskraftSS76--772009}.}
\authorfullcite{Duesing:DieTeleologieinKantsWeltbegriff1968} diskutiert die
Begriffe von diskursivem und intuitivem Verstand im Rahmen seiner Untersuchungen
zur Teleologie.\footnote{Siehe
\cite[][66--74]{Duesing:DieTeleologieinKantsWeltbegriff1968}, sowie
\cite[][144--147]{Duesing:NaturteleologieundMetaphysikbeiKantundHegel1990}.}
\authorfullcite{Allison:KantsTranscendentalIdealism2004}\footcite[Vgl.][]{Quarfood:DiscursivityandTranscendentalIdealism2012}
und
\authorfullcite{Quarfood:DiscursivityandTranscendentalIdealism2012}\footnote{\cite[Vgl.][]{Allison:KantsTranscendentalIdealism2004}.
Siehe dazu auch \cite{Pippin:IdealismandFinitude2008}.} thematisieren die
Diskursivität des Verstandes als Fundament des transzendentalen Idealismus.
\authorcite{Engfer:MenschlicheVernunft2002} sieht vor allem das Eingeständnis
expliziert, dass die Transzendentalphilosophie auf Voraussetzungen beruhe, die
sie selbst nicht in der Lage sei einzuholen.\footnote{\enquote{Was bedeutet
diese absichtsvoll wiederholte Berufung auf die besondere Beschaffenheit des
menschlichen Erkenntnisvermögens am Ende der Kritik der Urteilskraft systematisch
für das Ganze der kritischen Philosophie? Sie bedeutet
offenbar, daß \name[Immanuel]{Kant} jedenfalls an dieser Stelle die These vertritt, daß bestimmte
zentrale Ansätze und Gelenkstellen der kritischen Philosophie sich als
Konsequenzen spezifischer Eigentümlichkeiten der menschlichen Vernunft erweisen
und insofern trotz der scheinbar dagegen sprechenden Äußerungen \name[Immanuel]{Kant}s von
spezifisch menschlichen Voraussetzungen abhängig zu sein scheinen, die dann
trotz der entgegengesetzten Beteuerungen \name[Immanuel]{Kant}s die Basis für wesentliche
Aussagen der kritischen Philosophie sein könnten. Denn hier werden ja nicht etwa
marginale Bestimmungen, sondern zentrale und wesentliche Begriffe und
systematische Voraussetzungen aller drei Hauptschriften der kritischen
Philosophie selbst thematisiert}
\parencite[][\pno~272\,f.]{Engfer:MenschlicheVernunft2002}.} Zugleich zeigten
die Überlegungen aber auf, dass die Endlichkeit des Menschen auch einen Vorzug des
Menschen bedinge: Die Unterscheidungen von Möglichkeit und Wirklichkeit sowie
von Sein und Sollen artikulierten nicht nur Unvollkommenheiten des menschlichen
Verstandes, sondern seien zugleich Zeichen für die Freiheit des
Menschen.\footnote{\cite[Vgl.][283]{Engfer:MenschlicheVernunft2002}.}

Ähnlich wie mit dem Begriff der Endlichkeit verhält es sich mit dem der
Aufklärung; wobei hier der Vorteil vorliegt, dass \name[Immanuel]{Kant} einen
eigenständigen Text dazu verfasst hat, der schon wegen seiner Bekanntheit
Interpretationen provoziert.\footnote{Die Masse an Publikationen zu \name[Immanuel]{Kant}s
Beitrag in der \titel{Berlinischen Monatsschrift} ist, wie kaum überrascht,
unüberschaubar. Interessant sind hier v.\,a. diejenigen Beiträge, die auf die
Verbindung zu weiteren Aspekten seiner Philosophie, insbesondere der
Vernunftkritik, eingehen. So zeigt
\authorfullcite{Allison:KantsConceptionofemphAufklaerung2012}, dass
\name[Immanuel]{Kant} eine komplexe und detaillierte Aufklärungskonzeption
entwirft, die zu den grundlegendsten seiner philosophischen Überzeugungen in Verbindung steht
\parencite[vgl.][]{Allison:KantsConceptionofemphAufklaerung2012}.
\authorfullcite{Schmidt:WhatEnlightenmentWas1992} interpretiert
\name[Immanuel]{Kant}s Aufsatz als Reinterpretation des Aufklärungsprogramms auf
der Grundlage der Vernunftkritik
\parencite[vgl.][]{Schmidt:WhatEnlightenmentWas1992}.
\authorfullcite{Scholz:KantsAufklaerungsprogramm2009} möchte schließlich zeigen,
dass \name[Immanuel]{Kant}s Aufklärungsprogramm als Fluchtpunkt seiner gesamten
Philosophie angesehen werden kann
\parencite[vgl.][]{Scholz:KantsAufklaerungsprogramm2009,Scholz:BeantwortungderFrage:WasisteinaufgeklaerteWeltbuerger2011}.
Vor allem in den Arbeiten \authorfullcite{Hinske:ArtikelAufklaerung1985}s finden sich Argumente
für die Verbindung von Aufklärungsprogramm und Vernunftkritik
\parencite{Hinske:KantsVernunftkritik--FruchtderAufklaerungundoderWurzeldesDeutschenIdealismus1993,Hinske:ZwischenAufklaerungundVernunftkritik1998,Hinske:ZwischenAufklaerungundVernunftkritik1993}.
\authorfullcite{LaRocca:WasAufklaerungseinwird2004} thematisiert
Implikationen des Aufklärungsprogramms für den Begriff der Rationalität
\parencite[vgl.][]{LaRocca:WasAufklaerungseinwird2004}. Verbindungen zwischen
\name[Immanuel]{Kant}s Aufklärungsbegriff und anderen Aspekten seiner
Philosophie behauptet des weiteren auch Robert
\textcite[vgl.][]{Theis:KantetlAufklaerung2012}.
\Revision[Theis,
Pelletier]{\authorfullcite{Foucault:DieRegierungdesSelbstundderanderen2009}
nimmt \name[Immanuel]{Kant}s Aufklärungsaufsatz vielfach auf; siehe bspw.
\cite{Foucault:WasistAufklaerung1990}, und
\cite{Foucault:DieRegierungdesSelbstundderanderen2009}. Dabei sieht er die
Themen der drei Kritiken bereits in der Beschreibung der Unmündigkeit
angesprochen
\parencite[vgl.][\pno~49\,f.]{Foucault:DieRegierungdesSelbstundderanderen2009}.
Siehe weiter \cite{Foucault:WasistAufklaerung1990}, sowie
\cite{Foucault:WasistKritik?1992}. Dabei betont er selbst die Notwendigkeit,
eine andere Form unserer Abhängigkeit zu thematisieren: Demzufolge gehe es
heute nicht mehr um eine transzendentale Analyse von formalen Strukturen mit
universeller Bedeutung, sondern um die archäologische Aufdeckung all derjenigen
geschichtlichen Ereignisse und Umstände, durch die wir uns selbst erst
konstituieren und als die Subjekte unseres Handelns und Denkens ansehen können
\parencite[vgl.][]{Foucault:WasistAufklaerung1990}. Offensichtlich setzt sich
\name[Immanuel]{Kant} mit dieser Form von Abhängigkeit nicht auseinander; ob
dies ein ernsthaftes Manko ist, wird in vorliegender Arbeit jedoch offen
bleiben.}} Jedoch wird zumeist die Frage ausgeklammert, was die Forderung nach
Selbstdenken und eigenem Vernunftgebrauch von uns konkret fordert. Dass es bei
\name[Immanuel]{Kant} in diesem Sinne eine \emph{Ethics of Belief} zu
rekonstruieren gilt, ist erst in den letzten Jahren bewusst
geworden.\footnote{Dass die meisten Arbeiten zu \name[Immanuel]{Kant}s
Erkenntnistheorie seine \emph{ethics of belief} ignorieren, bedauert
\authorfullcite{Chignell:BeliefinKant2007}, der im Begriff des Glaubens einen
tragfähigen Ansatz zu einer Theorie nicht-epistemischer Rechtfertigungen sieht
\parencite[vgl.][]{Chignell:KantsConceptsofJustification2007}.
\authorcite{Chignell:KantsConceptsofJustification2007} möchte den Begriff des
Glaubens auch für eine \singlequote{liberale Metaphysik} nutzbar
machen, die durch Abschwächung der Geltungsbedingungen auch Aussagen über Dinge an sich auf Grundlage des
theoretischen Vernunftgebrauchs erlaube
\parencite[vgl.][335--360]{Chignell:BeliefinKant2007}: \enquote{Given
  this situation, we can and should go ahead and build metaphysical
  arguments in all the usual ways, by appealing to
  \enquote{intuitions} (of the \name[George Edward]{Moore}an rather than the
  \name[Immanuel]{Kant}ian sort), reflective equilibrium, inference to
best explanation, simplicity, and so forth} \parencite[][360]{Chignell:BeliefinKant2007}.
Wie \authorcite{Chignell:BeliefinKant2007} rekurriert
\authorfullcite{Stevenson:OpinionBelieforFaithandKnowledge2003} primär auf die
Überlegungen des Kapitels \titel{Vom Meinen, Wissen und Glauben} der
\titel{Kritik der reinen Vernunft}
\parencite[vgl.][]{Stevenson:OpinionBelieforFaithandKnowledge2003}.
\authorfullcite{Cohen:KantontheEthicsofBelief2014} verlässt den engen Rahmen
der Überlegungen zu Meinen, Wissen und Glauben und greift auf Überlegungen der
Moralphilosophie zurück, um \name[Immanuel]{Kant}s \emph{ethics of belief} zu
interpretieren \parencite[vgl.][]{Cohen:KantontheEthicsofBelief2014}.} Dazu
gehört insbesondere die Frage nach konkreten Regeln im Umgang mit dem sozialen
Charakter unseres Wissens und Erkennens. Die Diskussion um
\name[Immanuel]{Kant}s Position zu testimonialem Wissen -- um die wichtigste
Frage in diesem Zusammenhang herauszugreifen -- ist in einem 
Aufsatz von \authorfullcite{Schmitt:JustificationSocialityandAutonomy1987}
erstmalig diskutiert worden.
\authorcite{Schmitt:JustificationSocialityandAutonomy1987} sieht
\name[Immanuel]{Kant}s Aufklärungsprogrammatik dabei als Grund an, ihm eine
individualistische Position zu unterstellen, die mit unserer epistemischen
Wirklichkeit nicht kompatibel
ist.\footnote{\cite[Vgl.][46]{Schmitt:JustificationSocialityandAutonomy1987}.
Prominenten Ausdruck findet die Behauptung eines \index{Kant,
Immanuel}kantischen Individualismus bei Karl-Otto
\textcite[][passim]{Apel:DasAprioriderKommunikationsgemeinschaft1976}.} In den
letzten Jahren wurde dem widersprochen, insbesondere von
\authorfullcite{Scholz:DasZeugnisanderer2001}\footnote{Siehe
\cite{Scholz:AutonomieangesichtsepistemischerAbhaengigkeiten2001},
\cite{Scholz:DasZeugnisanderer2001},
\cite{Scholz:enquotedotsdenoberstenProbiersteinderWahrheitinsichselbstd.i.inseinereigenenVernunftsuchen2004},
und\cite{Scholz:Aufklaerung:VonderErkenntnistheoriezurPolitik2006}.} und
\authorfullcite{Gelfert:KantonTestimony2006}\footnote{Siehe \cite{Gelfert:KantonTestimony2006}, sowie
\cite{Gelfert:KantandtheEnlightenmentsContributiontoSocialEpistemology2010}.},
aber auch von weiteren Autoren wie
\authorfullcite{Mikalsen:TestimonyandKantsIdeaofPublicReason2010}.\footnote{\cite[Vgl.][]{Mikalsen:TestimonyandKantsIdeaofPublicReason2010}.
Die offenere Fragestellung, inwieweit \name[Immanuel]{Kant} Öffentlichkeit als
Notwendigkeit für unseren Vernunftgebrauch ansieht, stößt freilich auf mehr
Interesse in der Forschungslandschaft. Siehe dazu etwa
\cite{Hoeffe:EinerepublikanischeVernunft1996},
\cite{Deligiorgi:UniversalisabilityPublicitaandCommunication2002}, sowie
\cite{Deligiorgi:KantandtheCultureofEnlightenment2005}.}

\section{Anmerkungen zum methodischen Vorgehen}

Die Arbeit versteht sich weder als ideengeschichtlicher Beitrag, der die Autoren
fernab jeder Bewertung aus heutiger Sicht lediglich in ihrem historischen
Kontext darstellen möchte, noch als rationalisierende Interpretation, die sich
der Untersuchung historischer Umstände und Bezüge enthält. Mir geht es um ein
Verständnis der Position \name[Immanuel]{Kant}s, was sowohl erfordert, ihn aus
heutiger Perspektive -- mit heutigem Wissen und heutiger Terminologie -- zu
beschreiben, als auch, ihn in seiner Zeit zu betrachten.
Dabei schien es bei der Erstellung der Arbeit oft wenig ergiebig zu sein, der
Genese der Überlegungen in \name[Immanuel]{Kant}s diachroner Entwicklung nachzuspüren. Eine
umfangreichere Entwicklungsgeschichte des Aufklärungsdenkens oder des
Endlichkeitsbegriffs ist im Werk \name[Immanuel]{Kant}s nicht zu finden; eine
Entwicklungsgeschichte verwandter Themen wie der Metaphysik würde den Rahmen
der vorliegenden Arbeit sprengen. Einen größeren Gewinn verspricht der Versuch,
systematische Zusammenhänge zwischen den speziellen Darlegungen zu den hier
anvisierten Fragen auf der einen und den Grundstrukturen der kritischen
Philosophie auf der anderen Seite zu rekonstruieren.


Die Authentizität der Interpretation ist dabei nur dadurch zu gewährleisten,
dass relevante und zuverlässige Quellen ausgewertet werden. Die von \name[Immanuel]{Kant} selbst
veröffentlichten Werke genießen hierbei Vorrang
gegenüber unveröffentlichten Schriften wie Briefen, Vorlesungsmitschriften,
Reflexionen und dem \titel{Opus postumum}. Dabei ist neben der zeitlichen
Einordnung der Vorlesung und der Zuverlässigkeit der Mitschriften, von denen mitunter weder
das Datum noch der Mitschreiber bekannt sind, fraglich, inwieweit \name[Immanuel]{Kant} in diesen
Vorlesungen überhaupt eigene Ansichten referiert. Schließlich lag z.\,B.
seinen Metaphysikvorlesungen kein eigenständiges Konzept und keine eigene
Schrift, sondern die Metaphysik
\authorcite{Baumgarten:Metaphysica---Metaphysik2011}s zugrunde. Deswegen eignen
sich die Vorlesungsmitschriften nur bedingt als Maßstab der
Interpretation,\footnote{Dass man auch die Vorlesungsmitschriften gewinnbringend
anführen kann, wenn die Schwierigkeiten stets bewusst bleiben, demonstrieren
Autoren wie
\textcite[vgl.][passim]{Hinske:ZwischenAufklaerungundVernunftkritik1998}.} zumal
\name[Immanuel]{Kant} selbst gerade in den drei Kritiken hinreichend Auskunft zu den hier
interessierenden Fragen erteilt. Damit ist der Nutzen von Vorlesungsmitschriften
und Notizen \name[Immanuel]{Kant}s nicht in Abrede gestellt. Sie sind in ihrer Bedeutung von solchen
Schriften, die \name[Immanuel]{Kant} selbst publiziert und autorisiert hat, dennoch
grundlegend verschieden. Als
Maßstab der Interpretation dienen in erster Linie die \titel{Werke}
\name[Immanuel]{Kant}s, wie sie sich in der ersten Abteilung der
Akademieausgabe finden. Der neunte und letzte Band dieser Abteilung, der die
\titel{Logik}, die \titel{physische Geographie} und die \titel{Pädagogik}
enthält, nimmt dabei eine Sonderrolle ein, insofern es sich letztlich gar nicht
um von \name[Immanuel]{Kant} verantwortete und autorisierte Publikationen
handelt.\footnote{Nach
\authorfullcite{Stark:DieKant-AusgabederBerlinerAkademie--EineMusterausgabe2000}
ist dieser Band \enquote{[g]egen den Stand der Wissenschaft {\punkt}
erschienen}
\parencite[][216]{Stark:DieKant-AusgabederBerlinerAkademie--EineMusterausgabe2000},
die physische Geographie sei sogar \enquote{wissenschaftlich wertlos}
\parencite[][214]{Stark:DieKant-AusgabederBerlinerAkademie--EineMusterausgabe2000}.}
Dabei hat insbesondere die von \name[Gottlob Benjamin]{Jäsche} auf Geheiß
\name[Immanuel]{Kant}s hin erstellte {\jaeschelogik} in den letzten Jahren
vermehrt das Interesse der \name[Immanuel]{Kant}forschung geweckt, obwohl schon
seit langer Zeit Bedenken bezüglich der Zuverlässigkeit als Quelle für die
Positionen und Gedanken artikuliert
werden.\footnote{\phantomsection\label{Anmerkung:Einleitung:VorbehaltegegenueberderJaescheLogik}Die
{\jaeschelogik} wurde zwar auf der Grundlage von \name[Immanuel]{Kant}s
Durchschussexemplar von \authorfullcite{Meier:Vernunftlehre1752}s
\titel{Auszug aus der Vernunftlehre} und im Auftrag \name[Immanuel]{Kant}s, aber
letztlich selbständig von Gottlob Benjamin \name[Gottlob Benjamin]{Jäsche}
ausgearbeitet \mkbibparens{vgl.\ hierzu \name[Gottlob Benjamin]{Jäsche}s eigene
Auskunft in \cite{Kant:ImmanuelKantsLogik1977}, \cite[][IX:
3.2--4.14]{Kant:GesammelteWerke1900ff.}; siehe auch
\cite[][52--55]{Rameil:KantueberLogikalsVernunftwissenschaft2004}, und
\cite{Hinske:DieemphJaesche-LogikundihrbesonderesSchicksalimRahmenderAkademie-Ausgabe2000}}.
\authorfullcite{Reich:DieVollstaendigkeitderkantischenUrteilstafel1932} lehnt
die Beachtung der {\jaeschelogik} rundweg ab, weil es ihr ersichtlich an
Qualität mangle
\parencite[vgl.][21--24]{Reich:DieVollstaendigkeitderkantischenUrteilstafel1932}.
Von erheblicher Bedeutung ist aus der Sicht der vorliegenden Untersuchung, dass
\authorcite{Reich:DieVollstaendigkeitderkantischenUrteilstafel1932} zur
Begründung seines Urteils auf Unstimmigkeiten im IX. Abschnitt der Einleitung verweist, in der
unterschiedliche Formen des Fürwahrhaltens thematisiert werden, die hier in
Kapitel \ref{section:KantsEthicsofBelief} untersucht werden. Dem Urteil
\authorcite{Reich:DieVollstaendigkeitderkantischenUrteilstafel1932}s schließt
sich \authorfullcite{Stuhlmann-Laeisz:KantsLogik1976} an, der zusätzlich die
Unverständlichkeit der Ausführungen \name[Gottlob Benjamin]{Jäsche}s  zum
Verhältnis von hypothetischen und kategorischen Urteilen bemängelt
\parencite[vgl.][1]{Stuhlmann-Laeisz:KantsLogik1976}. Bedauerlich ist die
mangelnde Authentizität des Textes v.\,a., weil sich dort zusammenhängende
Überlegungen zu Form und Bildung von Begriffen finden, die man in den
eigenhändig verfassten Werken \name[Immanuel]{Kant}s vergeblich sucht. Diese
Überlegungen sind insbesondere im Rahmen der Analyse der Diskursivität in
Kapitel \ref{chapter:endlichkeitmenschlichendenkens} relevant. Ähnliche
Bedenken wie gegenüber der {\jaeschelogik} sind bezüglich der von Friedrich Theodor
\name[Friedrich Theodor]{Rink} bearbeiteten \enquote{physische Geographie}
und \enquote{Pädagogik} zu berücksichtigen. Beide sind auf ähnliche
Weise wie die {\jaeschelogik} entstanden \mkbibparens{\cite[siehe][IX:
  155.12--24, 439.5--9]{Kant:GesammelteWerke1900ff.}}.} Diese
Schriften werden hier dennoch nicht einfach ignoriert -- wie dies
\authorfullcite{Stuhlmann-Laeisz:KantsLogik1976} im Anschluss an
\authorfullcite{Reich:DieVollstaendigkeitderkantischenUrteilstafel1932}
beschließt\footnote{\cite[Vgl.][1]{Stuhlmann-Laeisz:KantsLogik1976}, sowie
\cite[][\pno~24, Anm.
11]{Reich:DieVollstaendigkeitderkantischenUrteilstafel1932}. Wenn letzterer den
Wert der {\jaeschelogik} für das Studium \name[Immanuel]{Kant}s betont, sie
aber nicht als selbständiges Beweisstück heranzieht, dann ist dies weitgehend
im Sinne dieser Arbeit. Die {\jaeschelogik} hat -- wie Reflexionen und
Vorlesungsnachschriften -- mehr Indizien- als Beweischarakter. Allerdings kann ich
das Vertrauen, welches
\authorcite{Reich:DieVollstaendigkeitderkantischenUrteilstafel1932} und
\authorcite{Stuhlmann-Laeisz:KantsLogik1976} auf der anderen Seite in die Vorlesungsnachschriften
setzen, nicht nachvollziehen. Für sie gilt m.\,E. dasselbe, was auch für die
{\jaeschelogik} gilt.} --, sondern mit der gebotenen Vorsicht herangezogen.
Letztlich soll in den von \name[Immanuel]{Kant} selbst
publizierten und \emph{autorisierten} Schriften das wichtigste Kriterium gesehen
werden.\footnote{Siehe zu diesem Ansatz auch
\textcite[][21--23]{Schwaiger:KategorischeundandereImperative1999}, der sich
damit von \textcite[vgl.][6--8]{Schmitz:WaswollteKant1989} abgrenzt.} Briefe,
Notizen, Reflexionen und Vorlesungsmitschriften sind gute Hilfsmittel bei der
Erstellung von Interpretationen. Aber die autorisierten Publikationen sind
letztlich das Tribunal, vor dem jede Interpretation sich rechtfertigen muss.

\phantomsection\label{Einleitung:AbschnittIdealistennachKant}
\name[Immanuel]{Kant}s Überlegungen zu unserer Endlichkeit und seine Konzeption
eines nicht\-dis\-kur\-si\-ven Verstandes und einer intellektuellen statt
sinnlichen Anschauung haben auch weite Teile der an ihn anschließenden Philosophien
geprägt. Viele dieser Diskussionen fanden in kritischer Auseinandersetzung mit
\name[Immanuel]{Kant}s Positionen und seinem Verhältnis zu Vorläufern wie
\authorfullcite{Leibniz:Meditationesdecognitioneveritateetideis1999} statt. Es
ist verlockend, in einer Darstellung der Philosophie
\name[Immanuel]{Kant}s auf diese Autoren einzugehen, schon um deren Verständnis
nutzbar zu machen. Jedoch steht dem entgegen, dass Autoren wie
\name[Salomon]{Maimon}\footnote{Siehe dazu
\cite{Atlas:SolomonMaimonsDoctrineofInfiniteReasonanditsHistoricalRelations1952},
sowie \cite[][326--361]{Kroner:VonKantbisHegel2007}, und
\cite{Ehrensperger:WeltseeleundunendlicherVerstand2006}.},
\authorcite{Fichte:DieBestimmungdesMenschen1800},
\name[Friedrich Wilhelm Joseph]{Schelling} und
\authorcite{Hegel:GesammelteWerke} in jeweils eigenständiger und mitunter
origineller, dabei aber nicht immer fairer Auseinandersetzung an
\name[Immanuel]{Kant} anschlossen. Ihre Einwände und Entgegnungen wären jeweils
ein eigenständiges Thema von gehörigem Umfang, weswegen in dieser Arbeit auf die
Auseinandersetzung mit solchen Autoren verzichtet wird. Der zu erwartende
Mehrwert läge sicherlich unter den beachtlichen Verlusten an Genauigkeit und
Tiefe im Durchdringen der Texte, die eine weitere thematische Verbreiterung mit
sich brächte.

Abschließend seien noch ein paar allgemeine Angaben zur Zitierweise gemacht.
Die Schriften \name[Immanuel]{Kant}s werden nach den im Literaturverzeichnis angegebenen
Ausgaben zitiert und durch Angabe der jeweiligen Schrift (als Siglum) und des
Ortes in der Akademieausgabe angegeben. Dabei wird auf die Akademieausgabe
in der Form \enquote{\cite{Kant:GesammelteWerke1900ff.} [Band]:
[Seite].[Zeile(n)]} mit Bandangabe in römischen sowie Seiten- und Zeilenangaben
in arabischen Ziffern und vorhergehendes Siglum verwiesen. Somit
verweist die Angabe \enquote{\cite{Kant:KritikderreinenVernunft2003},
\cite[][III: 108.16]{Kant:GesammelteWerke1900ff.}} auf die Stelle in der Kritik
der reinen Vernunft, die in der Akademieausgabe in Band III auf Seite 108 in der
16. Zeile zu finden ist. Wenn bei anderen Autoren eine mehrbändige Werkausgabe
vorliegt, verfahre ich analog.


\chapter{Aufklärung als vernünftige Selbstbestimmung}\label{section:KantalsliberalerAufklaerer}
Es ist umstritten, in welchem Verhältnis \name[Immanuel]{Kant}s
Aufklärungsdenken zur Vernunftkritik und zur kritischen Philosophie insgesamt
steht.\footnote{\enquote{\name[Immanuel]{Kant}s Verhältnis zur Aufklärung
scheint also die verschiedensten Auslegungen zuzulassen, ja es scheint nicht weniger zu schillern
als der Begriff der Aufklärung selber}
\parencite[][31]{Hinske:KantalsHerausforderungandieGegenwart1980}.} Oliver
\authorcite{Scholz:DasZeugnisanderer2001} behauptet: \enquote{Kants gesamte
theoretische und praktische Philosophie dient direkt oder indirekt der
Begründung der Ideen der
Aufklärung.}\footnote{\cite[][30]{Scholz:KantsAufklaerungsprogramm2009}, siehe
auch
\cite{Scholz:enquotedotsdenoberstenProbiersteinderWahrheitinsichselbstd.i.inseinereigenenVernunftsuchen2004}
sowie \cite[][23]{Recki:KantunddieAufklaerung2006}: \enquote{Die in der
Aufforderung \enquote{Sapere aude} geforderte Selbstbefreiung des Menschen
dürfen wir als das Leitmotiv der gesamten -- theoretischen und praktischen --
Philosophie Kants ansehen.}. Auch
\authorfullcite{Wood:KantandtheProblemofHumanNature2003} scheint diese Ansicht
zu vertreten, wenn er -- m.\,E. stark übertreibend -- schreibt:
\enquote{Kant's critical philosophy as a whole is the greatest and most
characteristic product of the intellectual and social movement, known as
\enquote{the Enlightenment,} which remains the unique source in the world for
all progressive thought and action (at least insofar as it has its roots
anywhere in the Western tradition)}
(\cite[][56]{Wood:KantandtheProblemofHumanNature2003}).}
\Revision[Theis, Pelletier]{Ähnlich
\authorfullcite{Foucault:DieRegierungdesSelbstundderanderen2009} in der
Beschreibung der Unmündigkeit die Themen aller drei Kritiken angesprochen.
Wenngleich die Beziehung von \name[Immanuel]{Kant} selbst nicht explizit
angesprochen werde, wende er sich mit der \titel{Kritik der reinen Vernunft}
doch gerade gegen die Abhängigkeit von geistigen
Autoritäten.}\footnote{\Revision[Theis, Pelletier]{\enquote{Die kritischen
Grenzen zu überschreiten und sich der Autorität eines anderen zu unterstellen, das sind die beiden Seiten
dessen, wogegen \name[Immanuel]{Kant} sich in der \emph{Kritik} erhebt,
dasjenige, von dem der Prozeß der Aufklärung selbst uns befreien soll}
\parencite[][\pno~50\,f.]{Foucault:DieRegierungdesSelbstundderanderen2009}.
Siehe auch \cite[][41]{Foucault:WasistAufklaerung1990}: \enquote{In einem
gewissen Sinne ist die Kritik das Handbuch der in der \ori{Aufklärung} mündig
gewordenen Vernunft, und umgekehrt ist die \ori{Aufklärung} das Zeitalter der
Kritik.}}} Dagegen ist die Deutung verbreitet, dass
\name[Immanuel]{Kant} sich \emph{neben} seinen metaphysikkritischen
Untersuchungen von den Herausgebern der Berlinischen Monatsschrift \emph{auch
noch} zu einer Beschäftigung mit den Themen der Aufklärung hat hinreißen lassen.
Zumindest wird die Vernunftkritik oft genug außerhalb des Kontextes der
Aufklärungsphilosophie
gelesen.\footnote{\cite[Vgl.][47]{Scholz:BeantwortungderFrage:WasisteinaufgeklaerteWeltbuerger2011}.}
Mitunter wird die These vertreten, dass \name[Immanuel]{Kant}s Vernunftkritik
gerade den für die Aufklärung konstitutiven Erkenntnisoptimismus und damit die
Aufklärung selbst überwinde.\footnote{Siehe beispielsweise die allerdings stark
tendenziöse Betrachtung von Magnus
\textcite{Selling:DieUeberwindungderAufklaerung1942}.} Oder es wird auf den
synthetischen Charakter der Vernunftkritik gegenüber einer analytischen
Ausrichtung der Aufklärung als einer \enquote{Philosophie der Analysis}
verwiesen, um einen Bruch \name[Immanuel]{Kant}s mit dem Aufklärungsdenken durch
die Vernunftkritik aufzuweisen.\footnote{\enquote{[W]ährend die Philosophie der
Aufklärung, was \name[Immanuel]{Kant}s eigene Generation angeht, ihrer ganzen
Tendenz nach eine Philosophie der Analysis ist, analytische Philosophie,
versteht sich die \ori{Kritik der reinen
Vernunft} in bewußtem Gegensatz dazu als Philosophie der Synthesis}
\parencite[][34]{Hinske:KantalsHerausforderungandieGegenwart1980}.}
\authorfullcite{Schneiders:AufklaerungundVorurteilskritik1983} nennt das
Erscheinen der \titel{Kritik der reinen Vernunft} zusammen mit dem Ausgang der
Französischen Revolution als ein entscheidendes Ereignis, das den Übergang von
der Aufklärung zum Deutschen Idealismus eingeleitet
habe.\footnote{\enquote{Beide Ereignisse haben in Deutschland entscheidend zu
einem neuen, primär an der Idee der Freiheit orientierten Menschenbild
beigetragen und so u.\,a. zum sogenannten Deutschen Idealismus geführt}
\parencite[][263]{Schneiders:AufklaerungundVorurteilskritik1983}. An anderer
Stelle wiederum verzichtet er auf die Nennung der Vernunftkritik und sieht die
Französische Revolution als einzigen wichtigen Markstein auf dem Weg zu einem
Ende der Aufklärung
\parencite[vgl.][18]{Schneiders:DasZeitalterderAufklaerung2005}. Zur
\singlequote{Zwischenstellung} \name[Immanuel]{Kant}s und seiner Vernunftkritik
zwischen Aufklärung und Idealismus siehe auch
\cite{Hinske:KantsVernunftkritik--FruchtderAufklaerungundoderWurzeldesDeutschenIdealismus1993}.}
Der Bruch mit der Aufklärung wird so als ein solcher interpretiert, den
\name[Immanuel]{Kant} selbst weder intendierte noch bemerkte, den er selbst
möglicherweise nicht durchführte, zu dem er jedoch mit seiner kritischen
Philosophie den Grund
legte.\footnote{\cite[Vgl.][60]{Hinske:ZwischenAufklaerungundVernunftkritik1993}:
\enquote{Die Logikvorlesungen zeigen \name[Immanuel]{Kant} also als
entschiedenen Vertreter der Philosophie der deutschen Aufklärung und zugleich
als Urheber jener neuen, kritischen Philosophie, die den Bruch mit den
Grundideen und Grundüberzeugungen der deutschen Aufklärung ungewollt
vorbereitet.}}

Der Begriff \enquote{Aufklärung} birgt selbst schon erhebliche
Schwierigkeiten. Denn er beschreibt einerseits eine historische Epoche der
(europäischen) Geistesgeschichte, ist aber andererseits auch ein
Programmbegriff, also eine Bezeichnung für eine philosophische Herangehensweise,
für die sich Vertreter auch außerhalb der \enquote{Aufklärung} genannten Zeit
finden.\footnote{\phantomsection\label{Anmerkung:ScholzundderBegriffEinerBewegung}\cite[Vgl.][S.~28\,f.]{Scholz:KantsAufklaerungsprogramm2009}.
Nach \authorcite{Scholz:DasZeugnisanderer2001} bezeichnet der Terminus außerdem eine philosophische
\enquote{\emph{Bewegung}}, also ein Gesamt an philosophischen Denkern, Verbreitern und Förderern
des Aufklärungsdenkens innerhalb seiner Epoche. Unter einer \emph{Bewegung} ist
hier also das Gesamt an Bemühungen vieler Menschen in einem mehr oder weniger
langen zeitlichen Rahmen zu verstehen, ein bestimmtes Programm, das nicht immer klar umrissen sein muss,
umzusetzen. Beispiele für solche Bewegungen sind etwa die 68er, die
Friedensbewegung und die Widerstandsbewegung zur Zeit des Nationalsozialismus.}
Und wenngleich wir zwischen Aufklärung als historischer Epoche oder
Bewegung\footnote{Zum Begriff \enquote{Bewegung} siehe
Anm.~\ref{Anmerkung:ScholzundderBegriffEinerBewegung}.} und Aufklärung als noch
immer aktueller Programmatik unterscheiden können und sollten, sind beide Fragen
doch miteinander verwoben, was die Behandlung des Begriffs erneut
erschwert.\footnote{\cite[Vgl.][9]{Schneiders:HoffnungaufVernunft1990}:
\enquote{Aufklärung als Aktion oder Aktionsprogramm und Aufklärung als
geschichtliche Erscheinung oder Epoche scheinen nahezu unvermeidlich immer in
einem Atemzug genannt werden zu müssen, deskriptiv-historische und
philosophisch-systematische bzw.
normativ-programmatische Fragen scheinen sich unvermeidlich zu verknüpfen und zu
vermischen. Die historische Frage, was Aufklärung zu ihrer Zeit war bzw. bis
heute ist, präformiert die programmatische Frage, was Aufklärung auch heute und
in Zukunft noch sein könnte oder sein sollte.}} Auch
ein programmatischer Begriff ist nicht ohne historische Untersuchungen zu haben,
denn jeder Aufklärungsbegriff ist dem geschichtlichen Phänomen der Aufklärung
verpflichtet, wenn er nicht leer und beliebig werden
soll.\footnote{\cite[Vgl.][4]{Bubner:WaskannsollunddarfPhilosophie?1978}.} 
Umgekehrt muss jeder Versuch, einen historischen Aufklärungsbegriff zu
bestimmen, schon mit einem Vorverständnis beginnen, das sich aus systematischen Interessen
speist.\footnote{\textcite[][28]{Scholz:KantsAufklaerungsprogramm2009}
beschreibt die Bedeutungen von \enquote{Aufklärung} als Epoche oder Bewegung
gegenüber dem Programmbegriff als derivativ. Auch
\textcite[][\pno~9\,f.]{Schneiders:HoffnungaufVernunft1990} erklärt, dass
systematische und programmatische Fragen die historische Aufklärungsforschung
leiten, wenngleich er das Verhältnis eher als gegenseitig beschreibt.}

Wir können auf eine Definition \name[Immanuel]{Kant}s zurückgreifen, der den
Begriff der Aufklärung ausdrücklich zu explizieren versuchte, werden aber
feststellen müssen, dass \name[Immanuel]{Kant}s Bestimmung des
Aufklärungsbegriffs Fragen offen lässt. Besonders offensichtlich wird dies bei
dem Begriff des Selbstdenkens -- oder des \singlequote{eigenen
Verstandesgebrauchs}, den \name[Immanuel]{Kant} als Definiens des
Aufklärungsbegriffs heranzieht. So bemühen sich über das gesamte 18.
Jahrhundert hinweg (und auch bereits im 17. Jahrhundert) verschiedene Autoren
wie Christian \name[Christian]{Thomasius} und Christian
\authorcite{Wolff:Psychologiaempirica1968}, die darin schulbildend wirkten, mit
unterschiedlichen Ergebnissen um einen vernünftigen Begriff des Selbstdenkens --
wobei freilich der Ausdruck \enquote{Selbstdenken} erst spät entstand und zuvor
eher von \enquote{Eklektik} gesprochen
wurde.\footnote{\cite[Vgl.][92]{Albrecht:Thomasius--keinEklektiker?1989}, sowie
\cite[][241--243]{Albrecht:ChristianThomasius1999}.} Wir dürfen außerdem
nicht vergessen, dass in \name[Immanuel]{Kant}s \titel{Beantwortung der Frage:
was ist Aufklärung?} Beschreibungen eines übergreifenden Projekts des 18.
Jahrhunderts und seiner spezifisch eigenen Programmatik Hand in Hand gehen.

Zur \emph{historischen Epoche} der Aufklärung zählt man in Bezug auf den deutschen
Sprachraum für gewöhnlich die Zeit von 1690 bis etwa 1800, wobei es sich
anbietet, weiter zu unterteilen in (a) eine Frühaufklärung von 1690 bis 1720 mit
Christian \name[Christian]{Thomasius} als Leitfigur, (b) eine Hochaufklärung, welche
wiederum eine schulphilosophische Phase (1720 bis
1750) und eine popularphilosophische Phase (1750 bis 1780) durchläuft, und (c)
eine Spätaufklärung (1780 bis 1800), der sich auch Immanuel
\name[Immanuel]{Kant} zuordnen lässt.\footnote{\cite[Vgl.][33]{Schneiders:HoffnungaufVernunft1990}.
Siehe zur Epocheneinteilung auch
\cite[][311]{Hinske:WolffsStellunginderdeutschenAufklaerung1986}.} Darüber
hinaus bezieht sich der Begriff der Aufklärung natürlich auch auf philosophische
und wissenschaftliche Entwicklungen und Autoren außerhalb Deutschlands, etwa in
Frankreich, Großbritannien und den Niederlanden. Schon aufgrund der
unterschiedlichen Rahmenbedingungen musste sich die Aufklärung in den
verschiedenen Ländern und Zeiten unterschiedlich
entwickeln. Während -- wie
\authorfullcite{Schneiders:DasZeitalterderAufklaerung2005}
ausführt\footnote{Siehe dazu
\cite[][16--18]{Schneiders:DasZeitalterderAufklaerung2005} und passim.} -- für
alle Regionen dasselbe historische Ereignis (der Ausgang der Französischen
Revolution) als \emph{Endpunkt} der Aufklärung auszumachen sei,\footnote{An
anderer Stelle wiederum nennt
\authorcite{Schneiders:AufklaerungundVorurteilskritik1983} als weiteres
Ereignis neben dem Ausgang der Französischen Revolution, welches zur Auflösung
der Aufklärung beigetragen habe, das Erscheinen der \titel{Kritik der reinen Vernunft}
\parencite[vgl.][263]{Schneiders:AufklaerungundVorurteilskritik1983}.} begann
sie doch in Frankreich, Großbritannien und Deutschland zwar in etwa zeitgleich, aber doch unabhängig voneinander durch unterschiedliche historische Ereignisse: Die
\emph{Glorious Revolution} von 1688, der Aufhebung des \emph{Edikts von Nantes}
1685 und \name[Christian]{Thomasius}' Leipziger Vorlesungsankündigung von 1687.
In Großbritannien habe sich Aufklärung auch als Religionskritik weitgehend
ungestört entwickeln können, in Frankreich hingegen sei sie in Opposition zu
einem politisch starken Katholizismus geraten und in Deutschland sei sie
\enquote{im wesentlichen durch zwei Faktoren bestimmt: das weitgehend positive
Verhältnis zur christlichen Religion und zum absolutistischen Staat einerseits
sowie die institutionelle Bindung an die Universitäten
andererseits.}\footcite[][89]{Schneiders:DasZeitalterderAufklaerung2005}
Wie die Aufklärung sich in den verschiedenen Regionen entwickelte und in welchem
Verhältnis dann wiederum die deutsche Aufklärung zu den Entwicklungen in anderen
Gegenden stand, ist Gegenstand divergierender Deutungen und anhaltender
philosophiegeschichtlicher Forschung.\footnote{Zuletzt hat
\authorfullcite{CarbonciniGavanelli:DasParadoxderAufklaerung2007} sich gegen die
verbreitete Ansicht gewandt, eine Beeinflussung sei lediglich von Großbritannien
und Frankreich aus in Richtung Deutschland erfolgt.
\cite[Vgl.][\pno~73\,f.]{CarbonciniGavanelli:DasParadoxderAufklaerung2007}.}


Diese Schwierigkeiten, interne Differenzierungen der Epoche der Aufklärung
vorzunehmen und die verschiedenen Strömungen wiederum in Beziehung zueinander zu
setzen, erschwert dann natürlich auch das Verständnis des Aufklärungsdenkens
\name[Immanuel]{Kant}s.\footnote{Siehe hierzu die Beiträge in
\cite{Emundts:ImmanuelKantunddieBerlinerAufklaerung2000}.} Es gibt keine einheitliche
Definition des Aufklärungsbegriffs, die als Grundlage und Ausgangspunkt dienen
könnte; und es muss eine solche auch nicht geben. Es kann auch sein, dass wir
verschiedene Philosophen und Gedanken zur Aufklärung rechnen, die keine
gemeinsamen \distanz{Wesensmerkmale} besitzen, sondern eher in einer Weise
zusammengehalten werden, wie sie \name[Ludwig]{Wittgenstein} unter dem Titel
\enquote{Familienähnlichkeit}
beschreibt.\footcite[Vgl.][\S\S~65--67]{Wittgenstein:PhilosophischeUntersuchungen2003}
Oliver \authorcite{Scholz:DasZeugnisanderer2001} behauptet zwar bezüglich des
allgemeinen Programms der Aufklärung:
\enquote{Ohne die regionalen und nationalen Unterschiede herunterzuspielen, darf
man den Kern dieses Programms folgendermaßen fassen: Der Mensch soll sich
mittels des richtigen Gebrauchs seines Vernunftvermögens selbst befreien und
kognitiv, vor allem aber moralisch
vervollkommnen.}\footcite[][28]{Scholz:KantsAufklaerungsprogramm2009} Doch hier
hängt nicht nur alles an den Begriffen von Selbstbefreiung und Vervollkommnung,
die es zu klären gilt. Es lässt sich außerdem bezweifeln, erstens, dass diese
Beschreibung auf alle Aufklärer zutrifft, und zweitens, dass sie nicht auch ganz
andere philosophische Strömungen und Autoren ebenso treffend charakterisiert.
Man möchte bezweifeln, dass
\authorfullcite{Wolff:Discursuspraeliminarisdephilosophiaingenere1996}s
Philosophie durch den Begriff der Selbstbefreiung passend beschrieben ist. Und
\name[Immanuel]{Kant}s Philosophie in den Dienst der moralischen Vervollkommnung
zu stellen, ist zumindest stark erläuterungsbedürftig.\footnote{Damit ist nicht
gesagt, dass die moralische Vervollkommnung nicht als Ziel der
Aufklärungsbemühungen nach dem Verständnis \name[Immanuel]{Kant}s gesehen
werden kann. Zur Vervollkommnung äußert er sich in den \S\S~21\,f. der
\titel{Tugendlehre}
\mkbibparens{\cite[vgl.][A 113--115]{Kant:DieMetaphysikderSitten1977Tugendlehre},
\cite[][VI: 446.9--447.17]{Kant:GesammelteWerke1900ff.}}.
Siehe zu praktischen und moralischen Zielen der Aufklärung nach
\name[Immanuel]{Kant} auch Kapitel \ref{chapter:AufklaerungundWissenschaft}
dieser Arbeit.} Und wenn Norbert \name[Norbert]{Hinske} die Aufklärung mittels
ihrer Programmideen (Aufklärung, Selbstdenken, Mündigkeit), Kampfideen
(Vorurteile, Aberglaube, Schwärmerei) und Basisideen (Bestimmung des Menschen,
allgemeine Menschenvernunft)
charakterisiert,\footnote{\cite[Vgl.][392--398]{Hinske:ArtikelAufklaerung1985}.
Außerdem nennt \authorcite{Hinske:ArtikelAufklaerung1985} noch Toleranz und
Pressefreiheit als \singlequote{abgeleitete Ideen}
\parencite[vgl.][\pno~398\,f.]{Hinske:ArtikelAufklaerung1985}.} so ist damit das
Erläuterungsbedürfnis ebenfalls nur verschoben.\footnote{Damit soll der
Verdienst der Erläuterung nicht in Abrede gestellt werden.
\name[Norbert]{Hinske}s Unterscheidung zwischen drei Formen von Ideen der
Aufklärung erhellt durchaus die innere Struktur aufklärerischen Denkens,
freilich nur für denjenigen, der schon weiß, was Aufklärung ist.} Zusätzlich
bleibt natürlich fraglich, ob solche Charakterisierungen wirklich allgemein oder
auf bestimmte Teile der Aufklärung -- im Fall \name[Norbert]{Hinske}s: auf
\name[Immanuel]{Kant} -- zugeschnitten sind, und ob sie vielleicht im Dienste
bestimmter philosophischer Ansichten stehen. Dies ist zum Beispiel ganz klar der
Fall, wenn
\authorfullcite{Sternberg:AufklaerungKlassizismusundRomantikbeiKant1931} die
Aufklärung in offen hegelianisierender Manier als \enquote{Sache des
Verstandes}, als \enquote{Analysieren, \punkt{} Sichten und Auseinanderhalten}
beschreibt.\footcite[][\pno~31\,f.]{Sternberg:AufklaerungKlassizismusundRomantikbeiKant1931}
Dasselbe Vorgehen finden wir bei
\authorcite{Horkheimer:DialektikderAufklaerung1997}, wenn Aufklärung über die
Endlichkeit oder Diskursivität des Denkens definiert und dabei im Interesse des
Beweisziels verkürzt wird.\footnote{Der Begriff der Diskursivität kommt in der
Eindeutschung \enquote{fortschreitendes Denken} in der \titel{Dialektik der
Aufklärung} vor: \enquote{Seit je hat Aufklärung im umfassenden Sinn
fortschreitenden Denkens das Ziel verfolgt, von den Menschen die Furcht zu
nehmen und sie als Herren einzusetzen}
\parencite[][19]{Horkheimer:DialektikderAufklaerung1997}. Zur Einschätzung der
Arbeit von \authorcite{Horkheimer:DialektikderAufklaerung1997} siehe
\cite[][xiii--xv]{Hinske:Einleitung1990}.}
Die Heterogenität dessen, was wir heute als historisches Phänomen unter dem Namen
\enquote{Aufklärung} zusammenfassen, wird noch deutlicher, wenn wir
\authorfullcite{Israel:RadicalEnlightenment2001}s Unterscheidung von Radikalaufklärung und moderater
Aufklärung\footnote{\cite[Vgl.][3--13]{Israel:RadicalEnlightenment2001} und
passim. Die Radikalaufklärung \enquote{rejected all compromise with the past and
sought to sweep away existing structures entirely, rejecting the Creation as
traditionally understood in Judaeo-Christian civilization, and the intervention
of a providential God in human affairs, denying the possibility of miracles, and
reward and punishment in an afterlife, scorning all forms of ecclesiastical
authority, and refusing to accept that there is any God-ordained social
hierarchy, concentration of privilege or land-ownership in noble hands, or
religious sanction for monarchy}
\parencite[][\pno~11\,f.]{Israel:RadicalEnlightenment2001}. Dem stehe die
wirkmächtigere moderate Aufklärung gegenüber, der u.\,a.
Christian \name[Christian]{Thomasius}
und \authorfullcite{Wolff:Discursuspraeliminarisdephilosophiaingenere1996}
angehörten und die von zahlreichen Regierungen und Teilen der Kirchen
unterstützt wurde. \enquote{This was the Enlightenment which aspired to conquer
ignorance and superstition, establish toleration, and revolutionize ideas,
education, and attitudes by means of philosophy but in such a way as to preserve
and safeguard what were judged essential elements of the older structures,
effecting a viable synthesis of old and new, and of reason and faith}
\parencite[][11]{Israel:RadicalEnlightenment2001}.} oder
\authorfullcite{Hunter:RivalEnlightenments2001}s auf die deutsche Philosophie
bezogene Differenzierung von bürgerlicher und metaphysischer
Aufklärung\footnote{\cite[Vgl.][14--9]{Hunter:RivalEnlightenments2001} und
passim.} hinzu nehmen. Dies sollte uns davor warnen, allzu schnell
Verallgemeinerungen über eine ganze Epoche vorzunehmen und Zusammenhänge
zwischen Autoren zu postulieren, die erst noch zu erweisen sind.

Der Begriff von Aufklärung, den ich im folgenden entwickle und verwende, wird
somit nicht beanspruchen können, dem Gesamtphänomen dessen gerecht zu werden,
was sich retrospektiv \singlequote{die} Aufklärung nennen ließe, sondern primär
auf \name[Immanuel]{Kant} fokussieren. Ich werde dabei nicht auf solche Fragen
eingehen wie die, ob \name[Immanuel]{Kant} noch zur Aufklärung oder schon zu
einer \enquote{klassischen deutschen Philosophie} oder einem \enquote{deutschen
Idealismus} gehört, ob er einer \singlequote{Berliner Aufklärung}, einer
\singlequote{moderaten} oder einer \singlequote{metaphysischen} Aufklärung
zuzurechnen ist. Mich interessiert, ob sich die Endlichkeit des Denkens oder die
Vernunftkritik, die von der Endlichkeit menschlichen Denkens ausgeht, in
\name[Immanuel]{Kant}s Aufklärungsprogrammatik integrieren lässt, vielleicht
sogar einen wesentlichen Teil derselben ausmacht, oder ob die Vernunftkritik den
Rahmen aufklärerischen Denkens bereits verlässt.\footnote{Diese Frage werde ich
jedoch in diesem \ref{section:KantalsliberalerAufklaerer}. Kapitel mitnichten
abschließend beantworten. Hier soll zunächst der Aufklärungsbegriff entwickelt
werden, der einer Antwort zugrunde liegt.} Der zugrunde gelegte
Aufklärungsbegriff ist dabei der, den \name[Immanuel]{Kant} selbst entwickelt --
explizit etwa in der Berlinischen Monatsschrift.

\section{\enquote{Aufklärung} und \enquote{aufklären}}
Bereits im 18.\ Jahrhundert wurde darauf hingewiesen, dass der Begriff der
Aufklärung keinen klar umrissenen und allseits bekannten Gehalt hat. Man könnte
gar fragen, ob ihm \emph{überhaupt} ein Gehalt zukommt und ob er ein reales
Phänomen bezeichnet; schließlich kam er als Neologismus des 18.
Jahrhunderts recht spät zum
Vorschein\footnote{\phantomsection\label{Anmerkung:MosesMendelssohnZumNeologismus}So
schreibt Moses
\textcite[][3]{Mendelssohn:UeberdieFrage:washeisstaufklaeren?2008}: \enquote{Die
Worte \ori{Aufklärung, Kultur, Bildung} sind in unsrer Sprache noch neue
Ankömmlinge.} Dies verweise aber nicht darauf, dass entsprechende Phänomene erst
neu seien.} und wurde in den Jahrtausenden
zuvor allem Anschein nach auch nicht vermisst. Eine entsprechende Skepsis kommt in der Frage \authorfullcite{Zoellner:IstesrathsamdasEhebuendnissnichtfernerdurchdieReligionzusancieren?1783}s
zum Ausdruck, die schließlich die Antwortversuche \name[Immanuel]{Kant}s und
\name[Moses]{Mendelssohn}s provozierte: \enquote{Was ist Aufklärung? Diese
Frage, die beinahe so wichtig ist, als: was ist Wahrheit, sollte doch mal
beantwortet werden, ehe man aufzuklären anfinge! Und noch habe ich sie nirgends
beantwortet
gefunden}\footnote{\cite[][516]{Zoellner:IstesrathsamdasEhebuendnissnichtfernerdurchdieReligionzusancieren?1783}.
\authorcite{Zoellner:IstesrathsamdasEhebuendnissnichtfernerdurchdieReligionzusancieren?1783}
selbst bestimmte die Aufklärung später als Vorurteilskritik
\parencite[siehe
dazu][\pno~271\,f.]{Schneiders:AufklaerungundVorurteilskritik1983}.}.
Diese Frage explizit zu stellen ist wichtig, weil eine Position oder Behauptung
als unaufgeklärt zu bezeichnen einer finalen Zurückweisung gleichkommt, die
ebenso schlagend ist, wie der Nachweis der Falschheit einer Behauptung.
Damit dieses argumentative Vorgehen legitim ist, muss natürlich klar sein, worin
der enthaltene Vorwurf besteht, dem sich niemand ausgesetzt sehen möchte. Was --
so scheint
\authorcite{Zoellner:IstesrathsamdasEhebuendnissnichtfernerdurchdieReligionzusancieren?1783}
fragen zu wollen -- können wir von unserem Denken denn noch mehr oder anderes
verlangen, als dass es sich an der Wahrheit orientiere?

Man könnte versucht sein, auf
\authorcite{Zoellner:IstesrathsamdasEhebuendnissnichtfernerdurchdieReligionzusancieren?1783}s
Frage, was denn zur \emph{Orientierung} an der Wahrheit hinzukommen müsse, zu
antworten: ihr \emph{Besitz}! Im Sinne einer szientistisch verstandenen
Aufklärung ließe sich dann mutmaßen, es ginge um die Gewinnung und Verbreitung
sicheren, \enquote{wissenschaftlich fundierten} Wissens im Kontrast zu religiösen,
mystischen oder metaphysischen Erklärungen \distanz{unaufgeklärter} Zeitalter
wie des \distanz{dunklen
Mittelalters}. Man
könnte so den Begriff der Aufklärung in enger Anlehnung an die Entwicklung der
modernen Naturwissenschaften zu explizieren versuchen und Aufklärung in
Entgegensetzung gegen ein zuvor vorherrschendes religiöses oder metaphysisches
Weltbild verstehen. Unaufgeklärt wäre dann eine (falsche) Überzeugung, die im
Zuge der wissenschaftlichen Revolution und der Entwicklung eines
wissenschaftlichen Weltbildes widerlegt wurde.

Der Etymologie des Wortes \enquote{Aufklärung} gemäß -- und dies gilt ebenso für
die Entsprechungen \enquote{Enlightenment}, \enquote{Si{\`e}cle des
Lumi{\`e}res} und \enquote{illuminismo} -- soll \distanz{Licht in's Dunkel}
gebracht werden. Und dazu passt sicherlich die Rede von einem \distanz{dunklen
Mittelalter} mit seinen überkommenen Erkenntnissen. Diese Abgrenzungen gegen
eine scholastische Philosophie des Mittelalters mag verbreitet gewesen
sein.\footnote{\cite[Vgl.][42]{Stekeler-Weithofer:Denken2012}:
\enquote{Die Epoche der Aufklärung sieht in der Abhängigkeit von traditionalen
Vorbildern, wie sie etwa das \singlequote{Mittelalter} unter Einschluss der
Renaissance und Reformation angeblich oder wirklich prägen, eine Heteronomie,
die es aufzuheben gilt.}} Aber zur Beschreibung und Selbstabgrenzung gegen
falsche Erklärungen hätte der Begriff der Wahrheit durchaus zugereicht. Wozu
also dieser Neologismus des {18.} Jahrhunderts?

Nun lag die Unklarheit für
\authorcite{Zoellner:IstesrathsamdasEhebuendnissnichtfernerdurchdieReligionzusancieren?1783}
auch in dem Umstand begründet, dass es sich bei dem Wort \enquote{Aufklärung}
um einen Neuankömmling in der deutschen Sprache handelte.\footnote{Siehe
Anm. \ref{Anmerkung:MosesMendelssohnZumNeologismus} auf
S.~\pageref{Anmerkung:MosesMendelssohnZumNeologismus}.} Heute ist es eher der
inflationäre Gebrauch, der seinen Gehalt undeutlich werden lässt. Aber dieser häufige Gebrauch spricht doch auch dafür, dass er eine wichtige und einzigartige sprachliche Funktion
übernimmt, für die der Begriff der Wahrheit allein kein Äquivalent liefert.
Diese Funktion mag wiederum nicht einheitlich sein. Eine Vielfalt an Formen von
\enquote{Aufklärung} findet sich nicht nur historisch in der (zeitlichen und
regionalen) Heterogenität der Epoche, sondern auch systematisch in den
verschiedenen, noch immer verbreiteten Redeweisen von \enquote{aufklären} und
\enquote{aufgeklärt} wieder. Eltern klären ihre Kinder auf,
wenn sie sie mit der menschlichen Sexualität vertraut machen. Ärzte klären
Patienten auf, indem sie ihnen die Wirkungsweise, die Risiken und Nebenwirkungen
und auch die Chancen einer Therapie erläutern. Verbraucherschützer propagieren
Verbraucheraufklärung. Verbrechen werden aufgeklärt. Politiker sprechen mitunter
von \distanz{schonungsloser Aufklärung}. Das Militär kennt seine eigene Form der
\distanz{Aufklärung}, ebenso wie auch nichtmilitärische Geheimdienste
\distanz{Aufklärung} betreiben. All diese Redeweisen unterscheiden sich
erheblich von dem Projekt des 18.\ Jahrhunderts, sind aber doch durch dieses
geprägt.

Meines Erachtens sehen wir die Herkunft dieser Redeweisen am besten anhand einer
bestimmten Auswahl: Wo der Begriff im Sinne von \enquote{\emph{jemanden}
aufklären} (im Kontrast zu \enquote{\emph{etwas} aufklären}) verwendet wird,
geht es zwar \emph{auch} darum, jemanden zu informieren -- über den menschlichen
und damit auch den eigenen Körper, über eine Krankheit und was damit einhergeht,
über die In\-halts\-stof\-fe von Lebensmitteln oder die Rechte und Pflichten
eines Vertragspartners. Aber nicht jede Information zählt als Aufklärung,
sondern nur diejenige, die auf die Mündigkeit desjenigen abzielt, den sie
aufklärt. Wer jemanden aufklärt, möchte ihm ermöglichen, ein eigenes Urteil und
eigene Entscheidungen zu fällen -- der Aufgeklärte urteilt und handelt
selbstbestimmt.
So ist das Ziel der Sexualaufklärung der mündige und selbstbestimmte Umgang mit der eigenen
Sexualität. Und der Arzt, der einen Patienten aufklärt, versorgt ihn mit genau
den und so vielen Informationen, wie nötig sind, um den Patienten in die Lage zu
versetzen, sich \emph{selbst} für oder wider eine bestimmte Therapieform zu
entscheiden. Gerade darin erweist sich der Begriff der Aufklärung auch in seinen
derivativen Verwendungen als Abkömmling der europäischen Geistesgeschichte:
Aufklärung verbindet Information mit Freiheit, Selbstbestimmung und Mündigkeit.
Und gerade diese Verbindung macht ihn aus und sichert ihm seine Bedeutung unter
unseren Begriffen. Negierte man das zweite dieser Momente und setzte etwa
Aufklärung in die Verbreitung wahrer Erkenntnisse, so verlöre dieser Begriff
seinen spezifischen Reiz. Wir sollten dann lieber gleich von Wahrheit sprechen,
um Missverständnisse zu vermeiden.

\section{Selbstdenken als Autonomie der
Vernunft}\label{subsection:SelbstdenkenbeiKant}
In den 1780er Jahren artikuliert \name[Immanuel]{Kant} in Aufsätzen in der
Berlinischen Monatsschrift eine Form von Aufklärung, die sich nicht an
bestimmten Inhalten, sondern der Freiheit und Selbstbestimmung des einzelnen
Subjekts orientiert. Dies wird zunächst in der bekannten
Aufklärungsschrift\footnote{\cite[Vgl.][A~481]{Kant:BeantwortungderFrage:WasistAufklaerung?1977},
\cite[][VIII: 35.1--8]{Kant:GesammelteWerke1900ff.}:
\enquote{\ori{Aufklärung ist der Ausgang des Menschen aus seiner
selbstverschuldeten Unmündigkeit. Unmündigkeit} ist das Unvermögen, sich seines
Verstandes ohne Leitung eines anderen zu bedienen. \punkt\ Sapere aude! Habe
Mut, dich deines eigenen Verstandes zu bedienen! ist also der Wahlspruch der
Aufklärung.}} deutlich, die durchgängig die Relation der Erkenntnis zu
dem einzelnen Subjekt hervorhebt und keinerlei spezielle Inhalte als solche der
Aufklärung herausstellt.\footnote{Mit der Religion, die \name[Immanuel]{Kant}
als zentrales Thema der Aufklärung herausstellt \mkbibparens{\cite[vgl.][A
492]{Kant:BeantwortungderFrage:WasistAufklaerung?1977}; \cite[][VIII:
41.10--12]{Kant:GesammelteWerke1900ff.}}, ist eben bloß ein
\emph{Thema} als besonders relevant bezeichnet, aber keine bestimmte Behauptung.}
\name[Immanuel]{Kant} stellt durch Betonung des Selbstdenkens die intellektuelle
Freiheit in das Zentrum der Aufklärungsprogrammatik und verdeutlicht, dass es nicht um
den \emph{Inhalt} des Fürwahrgehaltenen geht, sondern um die
\enquote{Denkungsart}.\footnote{Siehe exemplarisch \cite[][A
484]{Kant:BeantwortungderFrage:WasistAufklaerung?1977},
\cite[][VIII: 36.28--33]{Kant:GesammelteWerke1900ff.}:
\enquote{Durch eine Revolution wird vielleicht wohl ein Abfall von
persönlichem Despotism und gewinnsüchtiger oder herrschsüchtiger Bedrückung,
aber niemals wahre Reform der Denkungsart zu Stande kommen; sondern neue
Vorurteile werden, eben sowohl als die alten, zum Leitbande des gedankenlosen
großen Haufens dienen.}} Insbesondere in der Schrift \titel{Was heißt:
sich im Denken orientieren?} betont er dies durch die Abgrenzung des eigenen
Aufklärungsverständnisses von Positionen, die Aufklärung an
\enquote{Kenntnissen}
ausrichten.\footnote{\cite[Vgl.][A~329]{Kant:Washeisst:SichimDenkenorientieren?1977},
\cite[][VIII: 146.31--32]{Kant:GesammelteWerke1900ff.}.} Dabei ist jedoch nicht
leicht zu erkennen, was \name[Immanuel]{Kant} hier als \enquote{Kenntnisse}
anspricht. Eine naheliegende Deutungsmöglichkeit scheint mir zu sein, dies als
Ausdruck für den \emph{Inhalt} unseres Wissens zu lesen, im Kontrast dazu, wie
er erworben und weiter gehandhabt wird, ob er Produkt des Selbstdenkens ist und
wie er in die Handlungsplanung des Subjekts eingeht. \enquote{Kenntnisse} steht
demnach für die Informationen, die jemand über die Welt haben oder nicht haben
kann, das gesamte Tatsachenwissen, wie wir es beispielsweise in Lexika
finden.\footnote{\cite[Vgl.][32]{Scholz:KantsAufklaerungsprogramm2009}:
\enquote{Aufklärung besteht also nicht in dem materialen Besitz von Kenntnissen,
in irgendwelchen spezifischen Denkinhalten, Lehren oder Kenntnissen, die man
etwa in einer Liste zusammenfassen und dann jedermann einpauken könnte}.
Mitunter scheinen mit \enquote{Kenntnisse} aber auch gewisse (intellektuelle) Fähigkeiten
angesprochen zu sein, die nicht hinlänglich zur Mündigkeit sind, weil sie der
notwendigen eigenen Einsicht in ihre Vernünftigkeit entbehren. Siehe
z.\,B. \cite[][A~26]{Kant:ImmanuelKantsLogik1977}, \cite[][IX:
25.21--26]{Kant:GesammelteWerke1900ff.}: \enquote{[O]hne Kenntnisse wird man
nie ein Philosoph werden, aber nie werden auch Kenntnisse allein den
Philosophen ausmachen, wofern nicht eine zweckmäßige Verbindung aller
Erkenntnisse und Geschicklichkeiten zur Einheit hinzukommt, und eine Einsicht
in die Übereinstimmung derselben mit den höchsten Zwecken der menschlichen
Vernunft.} Dieser Passus erinnert an die Unterscheidung zwischen Schul- und
Weltbegriff der Philosophie (siehe dazu weiter unten Kapitel
\ref{subsection:DieBestimmungdesMenschen}) und identifiziert, sollte diese
Analogie stimmig und intendiert sein, die \enquote{Kenntnisse} mit den
Fähigkeiten nach dem Schulbegriff.} In der \titel{Anthropologie in pragmatischer
Hinsicht} beschreibt \name[Immanuel]{Kant} die \enquote{Kenntnisse} als Ausdruck
von
\enquote{Büchergelehrsamkeit}\footnote{\cite[Vgl.][BA~166]{Kant:AnthropologieinpragmatischerHinsicht1977},
\cite[][VII: 228.14--16]{Kant:GesammelteWerke1900ff.}:
\enquote{Büchergelehrsamkeit vermehrt zwar die Kenntnisse, aber erweitert nicht
den Begriff und die Einsicht, wo nicht Vernunft dazu kommt.} Mir scheint, dass
\name[Immanuel]{Kant} implizit auf die \emph{nudae factorum notitiae} der
wolffschen Philosophie und
\authorcite{Wolff:Psychologiaempirica1968}s Warnung vor einer \emph{cognitio
cognitionis philosophicae historica} verweist, die ich nicht an dieser, sondern
einer späteren Stelle ausführlicher behandeln werde. Hier sei nur folgendes als
Erläuterung angeführt: Eine \emph{nuda facti notitia} oder historische Erkenntnis bei
\authorcite{Wolff:Psychologiaempirica1968} ist die Kenntnis einer Tatsache ohne Verständnis ihres Grundes
(\cite[vgl.][\S~7]{Wolff:Discursuspraeliminarisdephilosophiaingenere1996}) und
die Warnung gilt dementsprechend einem Wiedergebenkönnen von Wahrheiten ohne
philosophisches Verständnis derselben (\enquote{wo nicht Vernunft dazu kommt},
siehe Kapitel \ref{paragraph:wolffswarnung}).
Zur Differenz zwischen der Warnung in
\authorcite{Wolff:Psychologiaempirica1968}s und der in \name[Immanuel]{Kant}s
Fassung siehe Kapitel \ref{section:MuendigkeitundPhilosophie}.}, also einer Art
des unselbständigen Denkens, das sich von anderen leiten lässt. Gerade die
\emph{Art} des Denkens -- ob jemand selbst denkt oder sich von anderen leiten
lässt -- kann bei veränderten Inhalten unverändert bleiben (und umgekehrt kann
sich bei gleichbleibenden Inhalten die \emph{Art} des Denkens
ändern), weswegen nur
eine langsame Reform, aber keine Revolution im Denken der Aufklärung förderlich
sei.\footnote{\cite[Vgl.][A~484]{Kant:BeantwortungderFrage:WasistAufklaerung?1977},
\cite[][VIII: 36.28--33]{Kant:GesammelteWerke1900ff.}: \enquote{Durch eine
Revolution wird vielleicht wohl ein Abfall von persönlichem Despotismus und
gewinnsüchtiger oder herrschsüchtiger Bedrückung, aber niemals wahre Reform der
Denkungsart zustande kommen; sondern neue Vorurteile werden ebenso wohl als die
alten zum Leitbande des gedankenlosen großen Haufens dienen.}} Man könnte eine
solche Aufklärungskonzeption, die sich auf die \emph{Art} des Denkens, die
Relation der Inhalte des Denkens zu dem denkenden Subjekt bezieht, eine
\phantomsection\label{Benennung:LiberaleAufklaerung}\emph{liberale} Aufklärung
nennen und sie einer \emph{szientistischen} Aufklärung entgegensetzen, die bestimmte Inhalte -- etwa den Heliozentrismus
oder die Evolutionstheorie -- als solche der Aufklärung glaubt ausweisen zu
können.
\authorfullcite{Schneiders:DasZeitalterderAufklaerung2005} unterscheidet in eben
diesem Sinne einen rationalistischen Aufklärungsbegriff, dem zufolge es um die
Klärung von Begriffen und die Beseitigung von Unvernunft und Unwissenheit geht, von einem
emanzipatorischen Aufklärungsbegriff, der sich an der \enquote{Befreiung von Fesseln
aller Art}
orientiert.\footcite[Vgl.][7]{Schneiders:DasZeitalterderAufklaerung2005} Und
\authorfullcite{Stuke:Aufklaerung1972} spricht etwas allgemeiner von dem
Unterschied zwischen einem formalen Aufklärungsverständnis und einem solchen,
welches durch objektiv-materiale Kriterien konstituiert werde, und betont, dass
\name[Immanuel]{Kant} mit seiner Artikulation einer streng formalen (oder --wie
ich sie nenne -- liberalen) Aufklärung von dem vorherrschenden Sprachgebrauch
seiner Zeitgenossen
abweiche.\footnote{\cite[Vgl.][\pno~265\,f.]{Stuke:Aufklaerung1972} Dabei
verfahre \name[Immanuel]{Kant} jedoch inkonsequent, insofern mindestens in
seiner Religionsphilosophie bestimmte Wissensgehalte den Begründungszusammenhang
der Aufklärung umschrieben \parencite[vgl.][265--272]{Stuke:Aufklaerung1972}. Es
ist dabei sicherlich korrekt, dass \name[Immanuel]{Kant} mitunter auch bestimmte
Überzeugungen als solche voraussetzt, ohne die Aufklärung nicht möglich ist.
Warum er diese voraussetzen muss, wird erst auf Grundlage der Überlegungen in
Kapitel \ref{chapter:MuendigerErwerbTestimonialenWissens} zu beantworten sein.
Es läuft darauf hinaus, dass es von der Art der Erkenntnisse abhängt, was es je
konkret heißt, sie mündig zu erwerben. Ein Missverständnis bezüglich der Art
der vorliegenden Erkenntnisse macht es unmöglich, sich ihnen gegenüber kritisch zu
verhalten. Siehe dazu Kapitel \ref{section:KantsEthicsofBelief}.}
\name[Immanuel]{Kant} hat die liberale Aufklärung freilich nicht
erfunden, sondern in Anlehnung an Selbstdenker\footnote{Mitunter
findet sich statt des Ausdrucks \enquote{Selbstdenker} gerade in Bezug auf
\name[Christian]{Thomasius} auch die Bezeichnung \enquote{Eklektiker}. Siehe
dazu auch Anm. \ref{Anmerkung:BegriffderEklektik} auf S.
\pageref{Anmerkung:BegriffderEklektik}.} wie \name[Christian]{Thomasius} und
Autoren wie \authorcite{Lessing:EineDuplik1897} aufgegriffen.\footnote{Gerade
\authorcite{Lessing:EineDuplik1897} verleiht der geistigen Liberalität Ausdruck,
wenn er schreibt:
\phantomsection\label{Zitat:Lessing:EineDuplik}
  \enquote{Ein Mann, der Unwahrheit, unter entgegengesetzter Ueberzeugung, in
  guter Absicht, ebenso scharfsinnig als bescheiden durchzusetzen sucht, ist
  unendlich mehr werth, als ein Mann, der die beste edelste Wahrheit aus
  Vorurtheil, mit Verschreyung seiner Gegner, auf alltägliche Weise
  vertheidiget.
  \punkt\ Nicht die Wahrheit, in deren Besitz irgend ein Mensch ist, oder zu
  seyn vermeynet, sondern die aufrichtige Mühe, die er angewandt hat, hinter
  die Wahrheit zu kommen, macht den Werth des
  Menschen} \parencite[][S. 23\,f.]{Lessing:EineDuplik1897}.
Auf der anderen Seite reicht es nicht, Wahres zu denken, um frei von Vorurteilen
zu sein. Mit \authorcite{Meier:AuszugausderVernunftlehre1752} kommt die
Auffassung auf, dass auch \emph{wahre} Urteile \emph{Vor}urteile sein
können. \cite[Vgl.][89]{Schneiders:PraktischeLogik1980}: \enquote{Im
Gegensatz zu \name[Christian]{Thomasius}
und \authorcite{Wolff:Discursuspraeliminarisdephilosophiaingenere1996}
verzichtete \authorcite{Meier:AuszugausderVernunftlehre1752} ausdrücklich
darauf, die Vorurteile als falsche Urteile zu definieren; er betont vielmehr
nachdrücklich, daß sie als unbegründete Urteile dennoch wahre Urteile sein
könnten.}}

\subsection{Der Begriff des
Selbstdenkens}\label{subsection:DerBegriffdesSelbstdenkens}
In der Aufklärungsschrift skizziert \name[Immanuel]{Kant} nicht ein neues
Projekt, sondern bemüht sich um Artikulation eines Projekts, das aus seiner
Sicht längst besteht und dessen wichtigstes Merkmal das Selbstdenken ist. Doch
bleibt \name[Immanuel]{Kant} zunächst eine Bestimmung dessen schuldig, was es
heißt, selbst zu denken. Während er in der Aufklärungsschrift nur abstrakt von
dem Mut des eigenen Verstandesgebrauchs und der Mündigkeit als des Vermögens,
\enquote{sich seines Verstandes ohne Leitung eines anderen zu
bedienen}\footnote{\cite[][A~481]{Kant:BeantwortungderFrage:WasistAufklaerung?1977},
\cite[][VIII: 35.3]{Kant:GesammelteWerke1900ff.}.}, spricht, sieht er sich an
anderer Stelle genötigt, die Rede vom Selbstdenken näher zu erläutern. Er
schreibt:
\begin{quote}\label{def:selbstdenken}
 \ori{Selbstdenken} heißt den obersten Probirstein der Wahrheit in sich selbst
(d.\,i.\ in seiner eigenen Vernunft) suchen; und die Maxime, jederzeit selbst zu
denken, ist die \ori{Aufklärung}. Dazu gehört nun eben so viel nicht, als sich
diejenigen einbilden, welche Aufklärung in \ori{Kenntnisse} setzen: da sie
vielmehr ein negativer Grundsatz im Gebrauche seines Erkenntnißvermögens ist, und öfter
der, so an Kenntnissen überaus reich ist, im Gebrauche derselben am wenigsten
aufgeklärt
ist.\footnote{\cite[][A 329]{Kant:Washeisst:SichimDenkenorientieren?1977},
\cite[][VIII: 146.29--35]{Kant:GesammelteWerke1900ff.}.}
\end{quote}
\name[Immanuel]{Kant} erläutert nicht explizit, was er unter einem negativen
Grundsatz versteht. Nahe liegt der Gedanke, dass er glaubt, damit nur gesagt zu
haben, was der Selbstdenker nicht tut. Man denke etwa an den \enquote{negativen
Begriff} der Freiheit im dritten Abschnitt der \titel{Grundlegung}. Dort
expliziert \name[Immanuel]{Kant} die Freiheit des Willens als Unabhängigkeit von
bestimmenden Ursachen. Diese Unabhängigkeit reiche aber noch nicht aus, um
hinreichend zu sagen, was Freiheit ist,\footnote{\cite[Vgl.][BA~97]{Kant:GrundlegungzurMetaphysikderSitten1965},
\cite[IV: 446.13--15]{Kant:GesammelteWerke1900ff.}: \enquote{Die angeführte
Erklärung der Freiheit ist \ori{negativ}, und daher, um ihr Wesen einzusehen,
unfruchtbar; allein es fließt aus ihr ein \ori{positiver} Begriff derselben, der
desto reichhaltiger und fruchtbarer ist.}} es müsse noch ein positiver Begriff
hinzukommen: im Falle der Freiheit die \enquote{Autonomie, d.\,i.\ die
Eigenschaft des Willens, sich selbst ein Gesetz zu
sein}\footnote{\cite[][BA~98]{Kant:GrundlegungzurMetaphysikderSitten1965},
\cite[][IV: 447.1--2]{Kant:GesammelteWerke1900ff.}.}. Entsprechend lässt
\name[Immanuel]{Kant} auch in der Definition des Selbstdenkens einen entsprechenden
\phantomsection\label{positiverBegriffdesSelbstdenkens}\distanz{positiven}
Begriff folgen, freilich ohne ihn als solchen zu benennen:
\begin{quote}
  Sich seiner \ori{eigenen} Vernunft bedienen will nichts weiter sagen, als bei
  allem dem, was man annehmen soll, sich selbst fragen: ob man es wohl tunlich
  finde, den Grund, warum man etwas annimmt, oder auch die Regel, die aus dem,
  was man annimmt, folgt, zum allgemeinen Grundsatze seines Vernunftgebrauchs zu
  machen?\footnote{\phantomsection\label{Fussnote:positiverBegriffdesSelbstdenkens}\cite[A~329]{Kant:Washeisst:SichimDenkenorientieren?1977},
  \cite[VIII: 146.35--147.6]{Kant:GesammelteWerke1900ff.}.
  \authorfullcite{ONeill:AufgeklaerteVernunft1996} rechnet auch diese Bestimmung
  des Selbstdenkens zum negativen Grundsatz, obwohl sie die Parallele zur
  praktischen Philosophie erkennt
  (\cite[vgl.][218]{ONeill:AufgeklaerteVernunft1996}). Auch
  \authorfullcite{Deligiorgi:UniversalisabilityPublicitaandCommunication2002}
  bezeichnet den Grundsatz als negativ, weil er keine konkreten Inhalte benenne
  \parencite[vgl.][146]{Deligiorgi:UniversalisabilityPublicitaandCommunication2002}.
  Beide übersehen m.\,E.\ die Ähnlichkeit zum positiven Begriff der Freiheit.
  Siehe auch unten, Kapitel \ref{subsection:MetaphysikundAutonomie}.}
\end{quote}
Um selbständig zu denken sollen wir zweierlei beachten: Wir sollen erstens nach
\emph{Grundsätzen} urteilen und zweitens diese Grundsätze danach bewerten, ob sie als
\emph{allgemeine} Grundsätze akzeptiert werden können.
\authorfullcite{Cohen:KantontheEthicsofBelief2014} verweist auf die
Ähnlichkeit dieser Formulierung zur Universalisierungsformel des Kategorischen Imperativs --
und damit zu dem Gesetz, das der Wille als praktische Vernunft sich selbst
ist.\footnote{\cite[Vgl.][330]{Cohen:KantontheEthicsofBelief2014}.} Diese \enquote{Maxime
der \ori{Selbsterhaltung} der Vernunft} -- wie \name[Immanuel]{Kant} sie an
derselben Stelle bezeichnet -- werde ich weiter unten wieder aufgreifen.
Zunächst soll aber noch der \enquote{negative Grundsatz} weiter betrachtet
werden, der nach \name[Immanuel]{Kant}s Darstellung die Aufklärung
ausmacht.\footnote{\cite[Vgl.][\S~40]{Kant:KritikderUrteilskraft2009},
\cite[][V: 294.35--36]{Kant:GesammelteWerke1900ff.}.}

\phantomsection\label{falscheFaehrte:VernunftkritikalsWissensbegrenzung}
Innerhalb der theoretischen Philosophie verwendet \name[Immanuel]{Kant} das Wort
\enquote{positiv} dort, wo Wissen und Erkenntnis vermehrt werden,
\enquote{negativ} hingegen dort, wo es um die Vermeidung von Irrtümern geht.\footnote{Z.\,B.~in der \titel{Anthropologie}:
\enquote{Der Verstand ist positiv und vertreibt die Finsternis der Unwissenheit
-- die Urteilskraft mehr negativ zu Verhütung der Irrtümer aus dem dämmernden
 Lichte, darin die Gegenstände erscheinen.}
 (\cite[][BA~166]{Kant:AnthropologieinpragmatischerHinsicht1977},
 \cite[][VII: 228.10--12]{Kant:GesammelteWerke1900ff.}.)} In der Kritik der
 reinen Vernunft vergleicht er die Philosophie als negatives Instrument mit der
 Polizei, insofern sie die Menschen voreinander schütze und so den positiven
 Nutzen habe, dass \enquote{ein jeder seine Angelegenheit ruhig und sicher treiben
 könne.}\footnote{\cite[][B~xxv]{Kant:KritikderreinenVernunft2003}, \cite[][III:
16.29--30]{Kant:GesammelteWerke1900ff.}.} Sie ist \enquote{negativ}, weil sie
nicht (als \emph{Organon}) unser Wissen erweitert, sondern (als \emph{Kanon})
vermeintlich objektive Wissensansprüche zurückweist, um Freiräume für den
Glauben und je selbst zu verantwortende Überzeugungen zu gewähren. Der
\enquote{negative Grundsatz} scheint also deswegen die Aufklärung auszumachen,
weil sie einen Freiraum garantiert, innerhalb dessen jeder in dem, was er für
wahr hält und sagt, nur \enquote{seiner eigenen Vernunft} Rechenschaft schuldig
ist. Dies wird im allgemein  dahingehend gedeutet, dass \name[Immanuel]{Kant}
die je eigene Einsicht und das je eigene Sich-selbst-überzeugen anspricht.
\authorfullcite{Gerhardt:Selbstbestimmung1999} etwa interpretiert das
\enquote{sapere aude} sogar folgendermaßen:
\begin{quote}
\ori{Suche deine eigene Einsicht und folge ihr} -- das ist der Wahlspruch der
Philosophie. Jeder solle sich, so hat \name[Immanuel]{Kant} es für die
Aufklärung formuliert, \ori{seines eigenen Verstandes} bedienen. Wäre das Ich
nur insoweit gefragt, als es ohnehin bei jedem intellektuellen Akt beteiligt
ist, bedürfte es der Betonung des jeweiligen \ori{eigenen} Verstandes nicht.
Also lebt die Philosophie tatsächlich aus einer \ori{selbstbewußten Forcierung
der Individualität}, eine Besonderheit, die sie, wie \singlename{Platon} bereits
wußte, mit den Künsten
teilt.\footnote{\cite[][35]{Gerhardt:Selbstbestimmung1999}. Zur Verwendung des
\singlename{Horaz}-Zitats in der Aufklärung siehe
\cite{Venturi:ContributiadundizionariostoricoI:WasistAufklaerungSapereaude!1959,Firpo:Ancoraapropositiodienquotesapereaude!1960}.}
\end{quote}
Der negative Grundsatz des Selbstdenkens richtet sich somit gegen (die Berufung
auf) Autoritäten. Und dies wird in der \titel{Anthropologie} noch einmal
deutlich, wo \name[Immanuel]{Kant} der Beschreibung des Selbstdenkens als
negativ das \singlename{Horaz}-Zitat beigibt:
\enquote{nullius addictus iurare in verba
magistri}\footnote{\cite[][BA~167]{Kant:AnthropologieinpragmatischerHinsicht1977},
\cite[][VII: 228.35]{Kant:GesammelteWerke1900ff.}. Zum ursprünglichen Kontext
des Zitats in der antiken Eklektik im Umfeld von \singlename{Cicero} und
\singlename{Horaz} siehe \cite[][38--49]{Albrecht:Eklektik1994}.} -- auf das
Wort keines Lehrers sei man in seinem Urteil verpflichtet.

\phantomsection\label{Abschnitt:AufklaerungundMuendigkeitdurchKompetenz}
Man kann die Ablehnung von Autoritätshörigkeit gewiss als eine wichtige
Konstante in der deutschen Aufklärungsphilosophie bezeichnen, die zumindest viele Autoren
eint. Im Falle der Selbstdenker wie \name[Christian]{Thomasius} und der an ihn
anschließenden Strömungen der deutschen Frühaufklärung ist sie gar zum Programm erhoben und selbst
\authorcite{Wolff:Psychologiaempirica1968}, der der \name[Christian]{Thomasius}-Anhängerschaft vielleicht unverdächtigste
Philosoph der deutschen Aufklärung, lehnt die Berufung auf Autoritäten zumindest
für einen bestimmten Personenkreis (diejenigen, die Philosophie nach
philosophischer Methode lehren sollen) \emph{expressis verbis}
ab,\footcite[Vgl.][\S~156]{Wolff:Discursuspraeliminarisdephilosophiaingenere1996} wenngleich er natürlich weit davon entfernt ist, Gedankenfreiheit in einer Weise zu fordern, wie sie uns heute selbstverständlich ist und wenigstens
partiell \name[Immanuel]{Kant} und einigen anderen Aufklärern
vorschwebte.\footnote{\cite[Vgl.][S.~xliv]{Gawlick:Einleitung1996}.}
Gerade \authorcite{Wolff:Psychologiaempirica1968} erkennt aber auch die Gefahren eines unbestimmten Begriffs
des Selbstdenkens, wenn dieser unreflektiert zum Ideal und Leitbild erklärt
wird. Es ist nichts gewonnen, wenn Freiheit und Mündigkeit bloß zu
Oberflächlichkeit und Beliebigkeit
führen.\footcite[Vgl.][Dedicatio,
\pno~259]{Wolff:Discursuspraeliminarisdephilosophiaingenere1996}


Es ist der Geist der Gründlichkeit, die Vorrangstellung der (methodisch
disziplinierten) Vernunft und das Vertrauen auf ihre Wirksamkeit, welche
\name[Immanuel]{Kant} stets an \authorcite{Wolff:Psychologiaempirica1968}
rühmt,\footnote{\cite[Vgl.~z.\,B.][B~xxxv--xxxvii]{Kant:KritikderreinenVernunft2003},
\cite[][III: 21.24--22.23]{Kant:GesammelteWerke1900ff.}.} und welche dieser in
der Philosophie des {18.} Jahrhunderts
verankerte.\footnote{\cite[Vgl.][315]{Hinske:WolffsStellunginderdeutschenAufklaerung1986}:
\enquote{Wolff hat der deutschen Aufklärung nicht nur ihre methodische Strenge
und ihre systematische Weite geschenkt, sondern auch ihren Glauben an die Macht
der Vernunft, der durch ihn ein wesentliches Moment von Aufklärung geworden
ist.}} \authorcite{Wolff:Psychologiaempirica1968}s unbestrittenes Verdienst ist
es, die Forderung nach Gründlichkeit und den unhintergehbaren Führungsanspruch
der Vernunft innerhalb der Aufklärung gegen
alle Widersacher -- Fideisten wie Freigeister und Schwärmer -- erhoben und
letztlich durchgesetzt zu
haben.\footnote{\cite[Vgl.][242]{Kreimendahl:ChristianWolff:EinleitendeAbhandlungueberPhilosophieimallgemeinen1994}:
\enquote{An seiner Demonstrationswut und der penetranten Aufdringlichkeit seiner
Methode haben sich bereits die Zeitgenossen gestoßen, darunter auch der junge
Kant \punkt . Allein der damalige philosophische Betrieb der Universitäten
scheint einen Zuchtmeister wie Wolff nötig gehabt zu haben. Er insistiert auf der
Forderung, nur das anzuerkennen, von dessen Wahrheit man durch eigene Einsicht
überzeugt ist. Sein Pochen auf die Notwendigkeit eines vom Subjekt selbst zu
erbringenden Erweises einer Wahrheit leistet einen entscheidenden Beitrag zur
Emanzipation der Vernunft von der Vorherrschaft jedweder Autorität. Dadurch
wirkt er genuin aufklärerisch und prägt wesentlich das intellektuelle Klima, in
dem noch Kant aufwächst.} Seine Logik wollte allgemeiner Standard für alle
intellektuellen Unternehmungen sein; und als \authorcite{Baumgarten:Metaphysica---Metaphysik2011} und einige Andere
sich daran machten, diesen Standard über die Mathematik, Naturforschung und
Philosophie hinaus auch auf die Theologie anzuwenden, wurde die Aufklärung
brisant \parencite[vgl.][\pno~xvii\,f.]{Gawlick:Einleitung2011}.}
Der Genius des Freidenkers bleibt -- auch dank \authorcite{Wolff:Psychologiaempirica1968} -- die Ausnahme. Und
die Frage, wie Freiheit und Mündigkeit allgemein realisierbar sind, ohne in Beliebigkeit und
geistigen Niveauverlust zu münden, bleibt ein zentrales Thema der Aufklärung.
Sie soll nicht die Wahrheit zugunsten der Freiheit und Selbständigkeit
zurückstellen, sondern beide verbinden. Und genau hier beginnt die
philosophische Herausforderung einer Explikation von Aufklärung, Selbstdenken
und Mündigkeit.

Das Reden von einem  negativen Grundsatz, die Zurückweisung von Wissensansprüchen
und nicht zuletzt die Forderung, wir sollten den \enquote{Probirstein der
Wahrheit} in uns selbst suchen, suggerieren ein Aufklärungsverständnis,
welches sein Heil in Schwärmerei und Freigeisterei sucht. Schwärmerei ist laut
\name[Immanuel]{Kant} \enquote{die Maxime der Ungültigkeit einer zu oberst
gesetzgebenden
Vernunft}\footnote{\cite[][A~327]{Kant:Washeisst:SichimDenkenorientieren?1977},
\cite[VIII: 145.25--26]{Kant:GesammelteWerke1900ff.}.}, die schließlich in
Aberglaube und Unmündigkeit
umschlage.\footnote{\cite[Vgl.][A~327]{Kant:Washeisst:SichimDenkenorientieren?1977},
\cite[][VIII: 145.27--35]{Kant:GesammelteWerke1900ff.}.} Um Selbstdenken und
Mündigkeit vor dem Hintergrund eines möglichen Konflikts mit Wahrheit und
Vernunft zu explizieren, ist auf die bereits genannte positive Beschreibung zu
rekurrieren:
\begin{quote}
  Sich seiner \ori{eigenen} Vernunft bedienen will nichts weiter sagen, als bei
  allem dem, was man annehmen soll, sich selbst fragen: ob man es wohl tunlich
  finde, den Grund, warum man etwas annimmt, oder auch die Regel, die aus dem,
  was man annimmt, folgt, zum allgemeinen Grundsatze seines Vernunftgebrauchs zu
  machen?\footnote{\cite[A~329]{Kant:Washeisst:SichimDenkenorientieren?1977},
  \cite[VIII: 146.35--147.6]{Kant:GesammelteWerke1900ff.}.}
\end{quote}\enlargethispage{\baselineskip}
Wer sich daran halte, sei
nicht nur mündig, er werde auch \enquote{Aberglauben und Schwärmerei bei dieser
Prüfung alsbald verschwinden sehen, wenn er gleich bei weitem die Kenntnisse
nicht hat, beide aus objektiven Gründen zu
widerlegen.}\footnote{\cite[A~329]{Kant:Washeisst:SichimDenkenorientieren?1977},
  \cite[VIII: 147.6--9]{Kant:GesammelteWerke1900ff.}.} Im Falle der Freiheit
erklärt \name[Immanuel]{Kant} es zu einem Missverständnis anzunehmen, wer nicht durch
Gesetze der Natur bestimmt sei, unterliege \emph{keinen} Gesetzen. Wer frei ist,
unterliegt seinen \emph{eigenen} Gesetzen, und zwar nicht denen, die er sich aus
welchen Gründen auch immer selbst \emph{gibt}, sondern denen, die seinen
Willen als \emph{vernünftigen} Wille oder als \emph{praktische Vernunft}
allererst \emph{konstituieren}.\footnote{An dieser Stelle sei lediglich betont,
dass Autonomie der praktischen Vernunft nicht heißen kann, sich beliebige
Gesetze selbst zu geben. Wer autonom handelt, der folgt damit Gesetzen der
\emph{Vernunft}. Siehe dazu Kapitel \ref{subsection:MetaphysikundAutonomie}.}
Ebenso kann Selbstdenken nicht heißen, regellos zu verfahren oder sich die
Regeln des Denkens selbst nach Gutdünken zu geben, sondern den \emph{Gesetzen}
der \emph{Vernunft} gemäß zu denken.\footnote{Die Grundlage dieser Vorstellung
ist wohl bei Hermann Samuel
\authorcite{Reimarus:DieVernunftlehrealseineAnweisungzumrichtigenGebrauchderVernunftinderErkenntnisderWahrheit1756}
zu suchen. Dies behauptet jedenfalls
\textcite[][25]{Hinske:ReimaruszwischenWolffundKant1980}:
\enquote{Thema der Philosophie, Thema der Logik und Erkenntnistheorie sind nun
[d.\,i.\ bei
\authorcite{Reimarus:DieVernunftlehrealseineAnweisungzumrichtigenGebrauchderVernunftinderErkenntnisderWahrheit1756};
A.\,G.] nicht mehr die vorgegebenen Gesetze, an denen sich die Vernunft zu
orientieren hat, sondern eben jene Regeln, die sie von Hause aus selbst
mitbringt und in deren Rahmen sie sich daher auch zwangsläufig bewegt. Die
Vernunft wird damit zu einer Quelle eigener, in ihr selbst gründender,
apriorischer Gesetze, sie wird im wörtlichen Sinne autonom.} Siehe hierzu auch
\cite{Arndt:DieLogikvonReimarusimVerhaeltniszumRationalismusderAufklaerungsphilosophie1980}.}
Doch was sind diese Gesetze? Woher stammen sie? Solche Fragen sind naturgemäß
nicht leicht zu beantworten. Das 18.
Jahrhundert hält eine Fülle an verschiedenen Deutungen parat und auch
\name[Immanuel]{Kant}s Position ist deutlich komplexer, als die Auskünfte in der
Berlinischen Monatsschrift vermuten lassen. Und auch der soeben genannte
positive Grundsatz bedarf noch der Auslegung.

\subsection{Kompetenzen und Entscheidungen}\label{Abschnitt:WolffunddieWissenschaftlichkeitderPhilosophiemoregeometrico}
Für \authorcite{Wolff:Psychologiaempirica1968} ist die Vermeidung von Oberflächlichkeit und
Beliebigkeit eine Frage der Methode, wobei die \emph{mathematische} Methode auch
in der Philosophie anzuwenden
sei.\footnote{\cite[Vgl.][\S~139]{Wolff:Discursuspraeliminarisdephilosophiaingenere1996}:
\enquote{Identitas methodi philosophicae {\&} mathematicae}.} \Revision{Diese
Methode zeichne sich durch drei Merkmale aus:
\begin{nummerierung}
\item Alle Begriffe bedürfen einer deutlichen Explikation,
\item alle Behauptungen müssen gründlich bewiesen werden, und
\item alle Behauptungen müssen gemäß der logischen Struktur ordentlich
verknüpft werde.\footnote{\Revision{\enquote{Wenn ich alles auf das genaueste
    überlege, was in der mathematischen Lehr-Art vorkommet, so finde
    ich diese drey Haupt-Stücke, 1. daß alle \ori{Wörter, dadurch die
      Sachen angedeutet werden, davon man etwas erweiset, durch
      deutliche und ausführliche Begriffe erkläret werden; 2. daß alle
    Sätze durch ordentlich an einander hangende Schlüsse erwiesen
    werden; 3. daß kein Förder-Satz angenommen wird, der nicht vorher
    wäre ausgemacht worden, und solchergestalt die folgenden Sätze mit
  dem vorhergehenden verknüpfft
  werden}} \parencite[][\S~25]{Wolff:AusfuehrlicheNachrichtvonseineneigenenSchrifftendieerindeutscherSpracheherausgegeben1973}.
Die Darstellung des \titel{Discursus} ist ausführlicher, verzichtet
jedoch auf diese übersichtliche Zusammenfassung \parencite[vgl.][\S\S~115--139]{Wolff:Discursuspraeliminarisdephilosophiaingenere1996}.}}
\end{nummerierung}}
Die Orientierung an der Mathematik und ihrem disziplinierten Vorgehen soll den einzelnen Denker
in die Lage versetzen, allein auf sich gestellt der Wahrheit nachzuforschen,
ohne sich in ungewissen Meinungen zu verlieren. Die eigene methodische Kompetenz
ermöglicht so Selbständigkeit im Verhältnis zu
anderen.\footnote{\cite[Vgl.][\S\S~156--162]{Wolff:Discursuspraeliminarisdephilosophiaingenere1996}.}
Dabei sieht \authorcite{Wolff:Psychologiaempirica1968}, dass nicht auf allen
Gebieten des Wissens die Beachtung methodischer Vorgaben im gleichen Umfange
möglich und sinnvoll ist. Mitunter sei es nicht möglich, sich streng
an die mathematische Methode zu halten, und manchmal führte dies auch nicht zu
größerer Genauigkeit, sondern übermäßiger Weitläufigkeit des Gedankengangs.
Zumindest sei die Genauigkeit der mathematischen Methode in vielen anderen
Disziplinen \emph{noch nicht} realisierbar.\footnote{\cite[Vgl.][Cap. 7,
\S~2]{Wolff:VernuenftigeGedankenvondenKraeftendesmenschlichenVerstandesundihremrichtigenGebraucheinErkenntnisderWahrheit1978}.
Siehe ebenso
\cite[][\S~148]{Wolff:Cogitationesrationalesdeviribusintellectushumani1983}.
Leider gibt \authorcite{Wolff:Discursuspraeliminarisdephilosophiaingenere1996}
keine Beispiele für diese Behauptung an.}


Auch wenn die Genauigkeit, die wir aus der Mathematik kennen, nicht in allen
Bereichen realisierbar und wünschenswert sei, bleibe es doch die Disziplinierung
des Denkers durch Übung in der Mathematik, die ihn auch auf anderen Feldern zu
Vernunft und Erkenntnis der Wahrheit befähige -- selbst dort, wo die Anwendung
der Methode nicht vollständig erfolgen kann. Nun mag es sein, dass
\authorcite{Wolff:Psychologiaempirica1968} seine Betonung von Methodik und
Systematik in Abgrenzung gegen Bestrebungen auf Freiheit und Selbstdenken
entwickelt.\footnote{\cite[Vgl.][12]{Schneiders:Deusestphilosophusabsolutesummus1986}:
\enquote{Als Wolff 1707 nach Halle kam, fand er dort eine Denkweise vor, die
mehr an Freiheit und Selbstdenken, an Praxis und Popularität als an reiner
Erkenntnis und wissenschaftlicher Gewißheit (Richtigdenken) interessiert war.
Dies mußte seinen metaphysischen und methodischen Interessen zuwiderlaufen und
sie reaktiv verstärken.}} Es ist aber nicht \authorcite{Wolff:Psychologiaempirica1968}s Absicht, das Anliegen
der Eklektiker und Selbstdenker zurückzuweisen, sondern es in sein auf
Systematik ausgerichtetes Philosophieren zu integrieren.\footnote{Man beachte
auch, dass das \singlename{Horaz}-Zitat \enquote{Sapere aude} zunächst in der an
\authorcite{Wolff:Psychologiaempirica1968}s Philosophie orientierten Vereinigung der Alethophilen
Verwendung fand
\parencite[vgl.][\pno~255\,f.]{Bronisch:WasistAufklaerung?2011}.} Gründlichkeit
und Wissenschaftlichkeit gelten ihm nicht als Alternative zu Selbständigkeit und
Selbstdenken, sondern als deren Ausdruck. Wer über einen systematischen Verstand
verfügt, der kann selbst die Wahrheit oder Falschheit einer Behauptung begründet
einsehen und muss sich nicht auf die Autorität anderer verlassen. Der
Systematiker ist daher der wahre
Eklektiker.\footnote{\phantomsection\label{Stellenverweis:Wolff:SelbstaendigkeitnurdurchKompetenz}Vgl.
\cite[\S~16]{Wolff:Dedifferentiaintellectussystematici&nonsystematici2011}:
  \enquote{Qui intellectu systematico praediti sunt, ab autoritatis praejudicio
  immunes, {\&} eclecticos agere apti sunt. Qui enim intellectu polent
  systematico, iidem non admittunt, nisi quod per principia in
  systemate contenta demonstrari potest. Judicant adeo ex intrinsecis
  rationibus\punkt{} Enimvero autoritas inter rationes extrinsecas locum
  habet ad quas confugiunt, qui intrinsecas minime capiunt.}}

\authorcite{Wolff:Psychologiaempirica1968} ist zuversichtlich, den Forderungen der Eklektiker und Selbstdenker
-- soweit sie berechtigt sind -- durch Integration in sein systematisches
Philosophieren Genüge zu tun.\footcite[Vgl.][526--538]{Albrecht:Eklektik1994}
Gerade der Systematiker sei es, der den berechtigten Forderungen der Aufklärung
genüge, weil nur er die Kompetenzen hat, die ihn von Autoritäten emanzipieren.
Selbstdenken ist also etwas, dessen derjenige überhaupt erst fähig ist, der die
philosophische Methode beherrscht. Umgekehrt ist Autoritätsgläubigkeit keine
schlechte Angewohnheit oder Neigung, die sich einfach durch eine Entscheidung
ablegen ließe. Sie ist das notwendige Resultat einer Schwäche oder Inkompetenz:
Gerade derjenige, der eine Wahrheit nicht streng beweisen kann, muss bei
schlechten  Gründen wie Autoritäten seine Zuflucht suchen.
\authorcite{Wolff:Psychologiaempirica1968} spricht von \enquote{äußeren Gründen}, also solchen, die
nicht vom Ziel der Wahrheit
ausgehen.\footnote{\cite[Vgl.][\S~155]{Wolff:Discursuspraeliminarisdephilosophiaingenere1996}.}
Wer nicht über entsprechende Erkenntnisfähigkeiten verfügt, der versuche
vergeblich, Vorurteile zu
vermeiden.\footnote{\cite[Vgl.][28--31]{Wolff:OratiodeSinarumphilosophiapractica1988}.}
Und so kann \authorcite{Wolff:Psychologiaempirica1968} die \emph{libertas philosophandi} auch auf
diejenigen beschränken, die Philosophie nach philosophischer Methode zu lehren
haben.\footcite[Vgl.][\S~166]{Wolff:Discursuspraeliminarisdephilosophiaingenere1996}
Bei ihm findet sich nichts von der Liberalität, die dem Programm und der
didaktischen Selbstverpflichtung eines \name[Christian]{Thomasius} zumindest in
dessen frühen Jahren noch
anhing.\footnote{\cite[Vgl.][241--243]{Albrecht:ChristianThomasius1999}. Zur
Ausbildung eines zunehmend pessimistischen Bildes vom Menschen bei
\name[Christian]{Thomasius}, das dessen Liberalität Abbruch tat, siehe
\cite{Engfer:ChristianThomasius1989}.} Aus \name[Christian]{Thomasius}'
Sicht ist \emph{jeder} Mensch -- unabhängig von Herkunft und Geschlecht -- in
vergleichsweise kurzer Zeit fähig, soweit gelehrt zu werden, dass er selbständig
und ohne Anleitung weiter studieren oder sein Wissen nutzbringend anwenden
könne.\footnote{\cite[Vgl.][34--36]{Thomasius:ChristianThomasiuseroeffnetDerStudirendenJugendzuLeipzigineinemDiscoursWelcherGestaltmandenenFrantzoseningemeinemLebenundWandelnachahmensolle?1994}.}
\authorcite{Wolff:Psychologiaempirica1968} traut dies allem Anschein nach nur seinem eigenen Stande des
Universitätsprofessors zu. Wer die hehren Ansprüche eines
wolffschen Systematikers erfüllen soll, muss mehr als ein paar
Semester Philosophie studiert haben. Die methodische Strenge und Systematik
wissenschaftlichen Denkens -- und das heißt bei ihm: \Revision{nach
  dem Vorbild der Mathematik} -- ist es
also, was \authorcite{Wolff:Psychologiaempirica1968} zum Ausdruck von Selbstdenken und Mündigkeit erklärt und
welche wir als seine Konzeption der Aufklärung bezeichnen können.\footnote{Ich
werde \authorcite{Wolff:Psychologiaempirica1968}s Konzeption weiter unten noch in einigen Punkten erweitern
(siehe Kapitel \ref{paragraph:wolffswarnung}), um genauer angeben zu können,
wie \name[Immanuel]{Kant} sich ihm anschließt (Kapitel
\ref{section:MuendigkeitundPhilosophie}). Hier soll aber zunächst der
systematische Rahmen zu \authorcite{Wolff:Psychologiaempirica1968}s und v.\,a.
\name[Immanuel]{Kant}s Überlegungen in diese Richtung erarbeitet werden.}

An dieser Stelle wird ein systematisch relevanter Unterschied sichtbar: Man kann
Selbstdenken einerseits als Ausdruck einer \emph{Fähigkeit} oder aber andererseits als
Ausdruck eines \emph{Entschlusses} auffassen. \authorcite{Wolff:Psychologiaempirica1968} verteidigt die
erste Sichtweise, wonach derjenige ein Selbstdenker sei, der die nötigen
Kompetenzen hat. Weder bedarf es nach dieser Ansicht eines Entschlusses zur
Unabhängigkeit, noch kann ein solcher Entschluss zum Selbstdenken führen, wo die entsprechenden
Kompetenzen nicht vorhanden sind. \name[Immanuel]{Kant} hingegen scheint die
gegenteilige Ansicht zu befördern, wenn er Aufklärung als den Ausgang aus
\enquote{selbstverschuldete[r] Unmündigkeit} bestimmt, deren \enquote{Ursache
\punkt{} nicht am Mangel des Verstandes, sondern der Entschließung und des Mutes
liegt}\footnote{\cite[][A~481]{Kant:BeantwortungderFrage:WasistAufklaerung?1977},
\cite[][VIII: 35.4--5]{Kant:GesammelteWerke1900ff.}}. Einige Bemerkungen
erinnern aber auch in \name[Immanuel]{Kant}s Schriften an die Grundzüge der Sichtweise
\authorcite{Wolff:Psychologiaempirica1968}s, dass Kompetenzen die Grundlage des Selbstdenkens bilden.
Dazu gehört, dass \name[Immanuel]{Kant}  den Kreis der Adressaten des Aufklärungsprogramms auf die
\enquote{Gelehrten} einschränkt, denen allein der freie öffentliche
Vernunftgebrauch zuzugestehen sei.\footnote{\cite[Vgl.][A
485]{Kant:BeantwortungderFrage:WasistAufklaerung?1977}, \cite[][VIII:
37.11--13]{Kant:GesammelteWerke1900ff.}: \enquote{Ich verstehe aber unter dem
öffentlichen Gebrauche seiner eigenen Vernunft denjenigen, den jemand als
Gelehrter von ihr vor dem ganzen Publikum der Leserwelt macht.}} Und auch im
Aufklärungsaufsatz betont er die Entwicklung der intellektuellen Fähigkeiten,
die sich aber erst auf Grundlage des Entschlusses und Mutes durch Öffentlichkeit
erwerben
ließen.\footnote{\cite[Vgl.][]{Kant:BeantwortungderFrage:WasistAufklaerung?1977},
\cite[][VIII: 36.10--15]{Kant:GesammelteWerke1900ff.}: \enquote{Wer sie
[d.\,i. die Fußschellen der Unmündigkeit; A.\,G.] auch abwürfe, würde dennoch
auch über den schmalsten Graben einen nur unsicheren Sprung tun, weil er zu
dergleichen freier Bewegung nicht gewöhnt ist. Daher gibt es nur Wenige, denen
es gelungen ist, durch eigene Bearbeitung ihres Geistes sich aus der
Unmündigkeit heraus zu wickeln und dennoch einen sicheren Gang zu tun.}}
\authorcite{Wolff:Psychologiaempirica1968} und \name[Immanuel]{Kant} -- so wird sich zeigen lassen
-- unterscheiden sich nicht bezüglich der Frage, \emph{ob} Selbstdenken in
intellektuellen Kompetenzen gründet, sondern hinsichtlich der Frage, \emph{wie}
solche Kompetenzen \emph{zu erwerben} sind und um \emph{welche} Kompetenzen es
sich konkret handelt.

\phantomsection\label{Abschnitt:AufklaerungundMuendigkeitdurchKompetenz-Ende}
\name[Immanuel]{Kant} schätzt den Vorteil einer Orientierung an der äußeren Form
der Mathematik schon 1763 als gering
ein\footnote{\enquote{Der Gebrauch, den man in der Weltweisheit von der Mathematik machen kann, bestehet
entweder in der Nachahmung ihrer Methode, oder in der wirklichen Anwendung ihrer
Sätze auf die Gegenstände der Philosophie. Man siehet nicht, daß der erstere
bis daher von einigem Nutzen gewesen sei, so großen Vorteil man sich auch
anfänglich davon versprach}
(\cite[][A~i]{Kant:VersuchdenBegriffdernegativenGroessenindieWeltweisheiteinzufuehren1977},
\cite[][II: 167.2--6]{Kant:GesammelteWerke1900ff.}).} und führt dies 1764 weiter
aus\footnote{\cite[Vgl.][A
71--79]{Kant:UntersuchungueberdieDeutlichkeitderGrundsaetzedernatuerlichenTheologieundderMoral1977},
\cite[][II: 276.1--283.9]{Kant:GesammelteWerke1900ff.}.}. In der \titel{Kritik
der reinen Vernunft} widmet er der Kritik der Anwendung der mathematischen Methode ein Kapitel der Methodenlehre.\footnote{\cite[Vgl.][B~740-766]{Kant:KritikderreinenVernunft2003},
\cite[][III: 468.22--483.32]{Kant:GesammelteWerke1900ff.}: \enquote{Die
Disziplin der reinen Vernunft im dogmatischen Gebrauche}.} Die Methode der \Revision{Mathematik} sei
zwar der Mathematik angemessen, für die Philosophie aber völlig unbrauchbar,
weil diese nicht mit Definitionen beginnen könne, sondern bei ihnen erst ende;
außerdem seien Axiome nur durch die reine Anschauung in der Mathematik
erhältlich. Der endliche Verstand habe außerhalb der Mathematik gar keine
solchen Sätze zur Verfügung, mit denen er es dem mathematischen Denken gleich tun
könnte.\footnote{\name[Immanuel]{Kant}s Zurückweisung der Identifizierung von
philosophischer und mathematischer Methode ist gut erforscht. Siehe z.\,B.
\cite[][26--67]{Engfer:PhilosophiealsAnalysis1982}, sowie
\cite[][42--101]{Wolff-Metternich:DieUeberwindungdesmathematischenErkenntnisideals1995}.
Siehe zur Bedeutung der Differenz mathematischen und philosophischen Erkennens
für die Themen \enquote{Aufklärung} und \enquote{Endlichkeit des Denkens} auch
unten Kapitel \ref{subsubsection:EndlichesundUnendlichesErkennen}.}

Nun redet \name[Immanuel]{Kant} selbstverständlich nicht Disziplinlosigkeit,
Schwärmerei und Geniekult das Wort.\footnote{Man beachte etwa den emphatischen Appell, die
eigene Freiheit des Denkens und der Feder nicht zu missbrauchen, mit dem er den
Aufsatz \titel{Was heißt: sich im Denken Orientieren?} schließt: \enquote{Freunde des
Menschengeschlechts und dessen, was ihm
am heiligsten ist! Nehmt an, was euch nach sorgfältiger und aufrichtiger Prüfung
am glaubwürdigsten scheint, es mögen nun Facta, es mögen Vernunftgründe sein;
nur streitet der Vernunft nicht das, was sie zum höchsten Gut auf Erden macht,
nämlich das Vorrecht ab, der letzte Probierstein der Wahrheit zu sein.
Widrigenfalls werdet ihr, dieser Freiheit unwürdig, sie auch sicherlich
einbüßen, und dieses Unglück noch dazu dem übrigen schuldlosen Teile über den
Hals ziehen, der sonst wohl gesinnt gewesen wäre, sich seiner Freiheit
\ori{gesetz}mäßig und dadurch auch zweckmäßig zum Weltbesten zu bedienen!}
(\cite[][A~329\,f.]{Kant:Washeisst:SichimDenkenorientieren?1977},
\cite[][VIII: 146.23-147.4]{Kant:GesammelteWerke1900ff.})} Eine
\enquote{Freiheit im Denken, wenn sie so gar unabhängig von Gesetzen der
Vernunft verfahren will,} zerstöre \enquote{endlich sich
selbst.}\footnote{\cite[][A~328]{Kant:Washeisst:SichimDenkenorientieren?1977},
\cite[][VIII: 146.21--22]{Kant:GesammelteWerke1900ff.}.} Nur ist es bei
\name[Immanuel]{Kant} nicht mehr die eine Methode des \authorcite{Wolff:Psychologiaempirica1968}ianismus, die das Denken
bestimmen soll; stattdessen bleibt zunächst nur der abstrakte Appell an einen
durch \authorcite{Wolff:Psychologiaempirica1968} etablierten \enquote{bisher noch nicht
erloschenen Geist \punkt{} der
Gründlichkeit}\footnote{\cite[][B~xxxvi]{Kant:KritikderreinenVernunft2003},
\cite[III: 22.9]{Kant:GesammelteWerke1900ff.}.}.
\phantomsection\label{Terminus:methodischerNaturalismus}\name[Immanuel]{Kant}
verwirft die Ansicht, die er die \enquote{naturalistische Methode} nennt,
{d.\,i.} die Auffassung, jeder brächte (speziell in der Metaphysik) die nötigen
Kompetenzen in seinem Urteilen von Haus aus mit, ohne einer speziellen Ausbildung zu
bedürfen.\footnote{\cite[Vgl.][B~883\,f.]{Kant:KritikderreinenVernunft2003},
\cite[][III: 551.30--552.13]{Kant:GesammelteWerke1900ff.}.} Der mündige Denker
ist der -- zumindest in gewissen Ausmaßen -- methodisch ausgebildete Denker, der
in seiner Ausbildung zwar nicht notwendig bestimmte Inhalte gelernt, wohl aber
gewisse Kompetenzen erworben haben muss.\footnote{Gewisse Einschränkungen wird
diese Behauptung in Kapitel \ref{chapter:AufklaerungundWissenschaft} erfahren.
Speziell Kapitel \ref{subsection:DieBestimmungdesMenschen} wird sich mit
\singlequote{pragmatischem} Wissen auseinandersetzen, welches für Aufklärung und
Mündigkeit relevant ist. Es handelt sich dabei primär um Wissen bzgl. der
\emph{conditio humana}.} Dies wird dadurch gestützt, dass das Selbstdenken als
\enquote{Maxime der \ori{vorurteilsfreien} \punkt{} Denkungsart}, die sich an
den Verstand richte, in der \titel{Kritik der Urteilskraft} sowie in der
\titel{Anthropologie in pragmatischer Hinsicht} durch zwei weitere Maximen
flankiert wird: Einerseits durch die Maxime der \emph{erweiterten} oder
\emph{liberalen} \enquote{Denkungsart} der Urteilskraft, andererseits durch die
Maxime einer \enquote{\ori{konsequenten} Denkungsart} der
Vernunft.\footnote{\cite[Vgl.][\S~40]{Kant:KritikderUrteilskraft2009}, \cite[V:
294.14--295.19]{Kant:GesammelteWerke1900ff.};
\cite[BA~166\,f.]{Kant:AnthropologieinpragmatischerHinsicht1977}, \cite[VII:
228.27--229.2]{Kant:GesammelteWerke1900ff.}.} Es ist die dritte Maxime (der
konsequenten Art des Denkens), auf die \authorcite{Wolff:Psychologiaempirica1968} sich konzentrierte, indem er
sie in der Methodik der Mathematik paradigmatisch realisiert
sah.\footnote{Katerina
\textcite[vgl.][151]{Deligiorgi:UniversalisabilityPublicitaandCommunication2002}
sieht in der dritten Maxime nicht die Forderung nach Konsistenz oder
methodischer Stimmigkeit artikuliert, sondern speziell die Forderung danach, die
beiden ersten Maximen in Übereinstimmung zu bringen. Dennoch schließt sie die
Forderung nach der methodischen Stimmigkeit der Erkenntnisse selbst natürlich
mit ein.} Auch \name[Immanuel]{Kant} denkt an den freien Gebrauch einer
\emph{disziplinierten} Vernunft, aber er gibt sich
nicht der Illusion hin, dass hierzu schlicht bestimmte Regeln der Vernunft
befolgt werden müssten, die wir als solche in einem Lehrbuch niederschreiben und
lernen könnten. Die Orientierung an der
Vernunftform lasse sich nur auf der Grundlage der beiden anderen Maximen (der
vorurteilsfreien und der erweiterten Art des Denkens)
erreichen.\footnote{\cite[Vgl.][\S~40]{Kant:KritikderUrteilskraft2009}, \cite[V:
295.14--17]{Kant:GesammelteWerke1900ff.}: \enquote{Die dritte Maxime, nämlich
die der \ori{konsequenten} Denkungsart, ist am schwersten zu erreichen und kann auch
nur durch die Verbindung beider ersten und nach einer zur Fertigkeit gewordenen
öfteren Befolgung derselben erreicht werden.}}
Konsequent (der Form oder den Gesetzen der Vernunft gemäß) zu denken, setze also
voraus, selbst zu denken. Und selbst zu denken verlange, konsequent zu denken,
wie wir soeben dem Appell der Orientierungsschrift entnehmen konnten.
Beide Forderungen kommen darin überein, dass die Regeln, denen die Vernunft
unterworfen ist, gerade als Ausdruck und Konkretisierung der Freiheit
verstanden werden. Eine solche Konzeption, die Freiheit und Unterwerfung unter
Regeln nicht durch Entgegensetzung bestimmt, die die Gesetze nicht als äußere
Beschränkungen versteht, sondern als ihre innere Verwirklichung, bezeichnet man
mit dem Namen
\enquote{Autonomie}.\footnote{\cite[Vgl.][7]{Khurana:ParadoxienderAutonomie2011}.}
Und auch \name[Immanuel]{Kant} selbst nennt die Vernunft \enquote{das Vermögen, nach der
Autonomie, {d.\,i.} frei (Prinzipien des Denkens überhaupt gemäß) zu
urteilen}\footnote{\cite[A~25]{Kant:DerStreitderFakultaeten1977}, \cite[VII:
27.30--31]{Kant:GesammelteWerke1900ff.}.}. Ebenso wie ein freier Wille und ein
Wille unter sittlichen Gesetzen einerlei
sind\footnote{\cite[Vgl.][BA~98]{Kant:GrundlegungzurMetaphysikderSitten1965},
\cite[IV: 447.6--7]{Kant:GesammelteWerke1900ff.}.}, so ist auch eine freie
-- aufgeklärte oder mündige -- Vernunft eine Vernunft unter Vernunftgesetzen.
Und diese Idee der Autonomie gilt es als Autonomie des Denkens zu explizieren.

\subsection{Regeln mündigen Denkens}
Welche Regeln sind dies also, denen das Denken als freies Denken unterworfen
ist? Zumindest an einer Stelle -- am Schluss des Aufsatzes \titel{Was
heißt: sich im Denken orientieren?} -- gibt \name[Immanuel]{Kant} auf diese
Frage eine konkrete Antwort. Ich zitiere die Stelle erneut, um sogleich an einem Beispiel eine
Deutung zu versuchen:
\begin{quote}
  Sich seiner \ori{eigenen} Vernunft bedienen will nichts weiter sagen, als bei
  allem dem, was man annehmen soll, sich selbst fragen: ob man es wohl tunlich
  finde, den Grund, warum man etwas annimmt, oder auch die Regel, die aus dem,
  was man annimmt, folgt, zum allgemeinen Grundsatze seines Vernunftgebrauches
  zu machen?\footnote{\cite[][A
  329]{Kant:Washeisst:SichimDenkenorientieren?1977},
  \cite[][VIII: 146.35--37, 147.5--7]{Kant:GesammelteWerke1900ff.}.}
\end{quote}
\name[Immanuel]{Kant}s Formulierung ähnelt offenkundig der des kategorischen
Imperativs: \enquote{[H]andle nur nach derjenigen Maxime, durch die du
zugleich wollen kannst, daß sie ein allgemeines Gesetz
werde.}\footnote{\cite[][BA 52]{Kant:GrundlegungzurMetaphysikderSitten1965},
\cite[][IV:421.7--8]{Kant:GesammelteWerke1900ff.}. In der \titel{Kritik der
praktischen Vernunft} lautet die Formulierung: \enquote{Handle so, daß die
Maxime deines Willens jederzeit zugleich als Princip einer allgemeinen Gesetzgebung gelten könne}
(\cite[][\S~7]{Kant:KritikderpraktischenVernunft1974}, \cite[][V:
30.38--39]{Kant:GesammelteWerke1900ff.}).} Zunächst setzt dies voraus, dass wir
nicht willkürlich im Einzelfall ohne jede allgemeine Orientierung entscheiden,
was wir tun wollen, sondern nach \distanz{Maximen} oder
\singlequote{allgemeinen Grundsätzen} handeln.\footnote{\name[Immanuel]{Kant}
sagt nirgends ausdrücklich, dass wir nach Maximen handeln, aber er setzt dies erstens in seinen Formulierungen des KI
voraus, sagt zweitens ausdrücklich, dass Menschen \emph{nach Regeln} handeln,
und sagt außerdem, dass Maximen subjektive Regeln oder Prinzipien sind, also
Regeln, die je mir eigen sind.} Und entsprechend fordert die Explikation des
Begriffs des Selbstdenkens eine Orientierung an allgemeinen Grundsätzen des
Denkens: Selbstdenken heißt hier also, sich nach allgemeinen Grundsätzen des
Denkens zu richten, \emph{die man selbst als vernünftig ansieht}. Die letzte
Einschränkung ist nötig, weil letztlich jedes Denken -- \emph{qua} Denken --
sich an Grundsätzen orientieren muss. Auch und gerade das vorurteilsbehaftete
Denken folgt Grundsätzen, denn gerade in bestimmten Grundsätzen des Denkens
bestehen Vorurteile: \enquote{Vorurteile sind vorläufige Urteile, \ori{in so
ferne sie als Grundsätze angenommen werden}.}\footnote{\cite[][A
116]{Kant:ImmanuelKantsLogik1977}, \cite[][IX:
75.24--25]{Kant:GesammelteWerke1900ff.}. Siehe dazu die Parallelstelle in
\cite[][\nopp 2538]{Kant:Reflexionen1900ff.},
\cite[][XVI: 409.5]{Kant:GesammelteWerke1900ff.}.} Es wäre eine Illusion zu
glauben, wir könnten im Denken auf Grundsätze schlechthin verzichten; wir können
lediglich beeinflussen, \emph{welchen} Regeln wir unser Denken unterwerfen -- ob
wir eigenen (Autonomie) oder fremden (Heteronomie) Regeln folgen.\footnote{Die
ist die Pointe des Gedankengangs, dass ein Denken, welches sich \emph{keinen} Regeln zu
unterwerfen gedenkt, letztlich fremden Regeln unterworfen wird
\mkbibparens{\cite[vgl.][A
326\,f.,]{Kant:Washeisst:SichimDenkenorientieren?1977}, \cite[][VIII:
145.6--35]{Kant:GesammelteWerke1900ff.}}.}

\name[Immanuel]{Kant} spricht hier bewusst nicht von der Übereinstimmung der
Vernunft mit einem externen Maßstab, sondern mit Gesetzen, die sie sich selbst
gibt.\footnote{\cite[Vgl.][A 326]{Kant:Washeisst:SichimDenkenorientieren?1977},
\cite[][VIII: 145.6--7]{Kant:GesammelteWerke1900ff.}.}
Es geht also nicht darum, Standards vernünftigen Überlegens zu finden, die der Vernunft
von außen angetragen werden. Bezüglich der Moral fragt \name[Immanuel]{Kant},
wie es sein könne, dass unser Wille Gesetzen unterworfen ist, und beantwortet
dies damit, dass diese Gesetz eben nicht von außen an unseren Willen
herangetragen werden, sondern er nur der eigenen, aber allgemeinen Gesetzgebung
unterworfen sei. Nur durch die Vorstellung der Autonomie seien Pflicht und
Verbindlichkeit denkbar.\footnote{\cite[Vgl.][BA
73\,f.,]{Kant:GrundlegungzurMetaphysikderSitten1965} \cite[][IV:
432.25--433.11]{Kant:GesammelteWerke1900ff.}.} Ebenso wäre es ganz
verwunderlich, wenn unsere Vernunft Gesetzen unterworfen würde, die ihr
äußerlich sind. Wir müssten fragen -- und könnten diese Frage vermutlich nicht
beantworten --, was diesen Gesetzen die Autorität verleiht, unser Denken zu
reglementieren. Aber dadurch, dass es sich um \emph{interne} Normen handelt, die
unser Denken selbst ausmachen, verschwindet diese Merkwürdigkeit. Es ist die
Vernunft selbst, die sich die Gesetze des Denkens gibt und daher als Vernunft
ihnen auch unterworfen ist.


Wenn wir uns selbst fragen, ob unser Gedanke vernünftig
ist, sollten wir die Frage also in folgender Form stellen: Folge ich in meinem
Urteil einer epistemischen Regel, die ich nicht nur hier anwenden möchte,
sondern die ich \emph{stets} anzuwenden bereit bin, wenn sich ein Anwendungsfall
dieser Regel findet? Es mag beispielsweise die Frage im Raume
stehen, ob etwas für wahr gehalten werden sollte, weil es seit geraumer Zeit im
Dorf erzählt wird. Wer sich um Selbständigkeit und Mündigkeit bemüht, ist
gehalten, sich nun zu fragen, ob er es richtig fände, der allgemeinen Regel zu
folgen, für wahr zu halten, was im Dorf erzählt wird. Da dort so manche
Despektierlichkeit im Umlauf ist, wird der mündige Denker dies für keine gute
Idee halten und also sein Urteil auch in diesem konkreten Fall nicht auf das
Gerücht im Dorf gründen. Wir haben also zwar keine konkreten Regeln an der Hand,
wohl aber einen Verallgemeinerungstest für mögliche Grundsätze unseres Denkens,
der deutlich an die Universalisierungsformel des Kategorischen Imperativs
erinnert. Leider lässt diese Regel Vieles offen. Sollte ich es mir etwa zur
Regel machen, Informationen von Wikipedia zu vertrauen? Hier können verschiedene
Antworten gegeben werden, von denen sich keine unmittelbar als unvernünftig
erweist. Warum sollte jemand nicht den allgemeinen Grundsatz annehmen,
bestimmten Autoritäten (etwa seinem Pfarrer) \emph{alles} zu glauben? Es ist
nicht (zumindest nicht offensichtlich) widersprüchlich, es sich selbst zum
Grundsatz zu machen, unmündig zu bleiben. Und \emph{a fortiori} reicht es zum
Selbstdenken nicht, bloß logisch konsistent zu denken -- womit auch diese
Interpretation des \singlequote{positiven} Begriffs des Selbstdenkens ausgeschlossen ist.

Wie deutlich wird, erlaubt die Formulierung dessen, was Selbstdenken heißt, noch
eine Vielzahl möglicher Grundsätze des
Denkens.\footnote{\cite[Vgl.][224]{ONeill:AufgeklaerteVernunft1996}:
\enquote{Das Denkprinzip, im Denken nur solche Grundprinzipien zu verwenden, die
universell anerkannt werden können, beschränkt wohl die Kategorien des Denkens,
die Urteils- und Schlußweisen, die wir rationaliter anerkennen können, bestimmt
sie aber nicht vollständig. Kant bietet keinen Algorithmus für den vernünftigen
Erwerb von Überzeugungen.}} Dass wir nach universellen Grundsätzen urteilen
sollen, schließt somit nicht die Möglichkeit aus, dass die Beliebigkeit nun in
die Wahl der Grundsätze verlegt wird. \name[Immanuel]{Kant}s
Aufklärungsauffassung läuft somit trotz der eigenen Beschwichtigungen zumindest
\emph{prima facie} Gefahr, ein beliebiges Meinen und Fürwahrhalten auf Grund
persönlicher Neigungen und Vorlieben zu protegieren.
Er vertritt zwar gewiss nicht einen \enquote{gesetzlose[n]} Gebrauch
der Vernunft als \enquote{die Denkungsart \punkt , die man \ori{Freigeisterei}
nennt, d.\,i.\ den Grundsatz, gar keine Pflicht mehr zu
erkennen}\footnote{\cite[][A 328]{Kant:Washeisst:SichimDenkenorientieren?1977},
\cite[][VIII: 146.15--16]{Kant:GesammelteWerke1900ff.}.}. Aber ebenso wenig
schließt er sich der Konzeption
\authorcite{Wolff:Discursuspraeliminarisdephilosophiaingenere1996}s an, die
Freigeisterei und Beliebigkeit durch Rückbindung des Selbstdenkens an die
\Revision{mathematische} Methode zu verhindern sucht. Es gibt nach \name[Immanuel]{Kant}
keinen allgemeinen Standard, der vernünftige Überlegungen von unvernünftigen
unterscheidet. Dann jedoch gilt es, einen anderen Weg zu finden, die
Vernünftigkeit je eigenen Denkens sicherzustellen.


In der \titel{Kritik der Urteilskraft} im Kontext einer
Exposition des \enquote{\emph{sensus communis}}, der nicht nicht die
Berufung auf allgemein geteilte und als plausibel gewertete Vorannahmen eines
\distanz{common sense} oder einen als unkultiviert gedachten \distanz{gemeinen
Menschenverstand} bezeichnet, sondern thematisch werden lässt, dass Denken die
Tätigkeit sozial lebender Wesen ist, behandelt \name[Immanuel]{Kant} drei
Maximen eines an der Aufklärung orientierten Denkens.
Nur wenn wir in der Lage sind, in unseren Urteilsakten unser Denken nicht an
subjektiven Privatbedingungen und beliebigen Wünschen und Einfällen, sondern an
der allgemein Menschenvernunft zu orientieren, können wir überhaupt objektive
Urteile fällen.\footnote{\cite[Vgl.][\S~40]{Kant:KritikderUrteilskraft2009}, \cite[][V:
293.30--36]{Kant:GesammelteWerke1900ff.}: \enquote{Unter dem \ori{sensus
communis} aber muß man die Idee eines \ori{gemeinschaftlichen} Sinnes, d.\,i.\
eines Beurteilungsvermögens verstehen, welches in seiner Reflexion auf die
Vorstellungsart jedes anderen in Gedanken (a priori) Rücksicht nimmt, um
gleichsam an die gesamte Menschenvernunft sein Urteil zu halten und dadurch der
Illusion zu entgehen, die aus subjektiven Privatbedingungen, welche leicht für
objektiv gehalten werden könnten, auf das Urteil nachteiligen Einfluß haben
würde.}} Dieser \emph{sensus communis}, der auf die gemeinschaftliche Grundlage und
Ausrichtung des Denkens verweist, konkretisiert sich also in drei Maximen des
gemeinen Menschenverstandes, die angeben, worin die \emph{Kultivierung} des
Denkens im Sinne einer liberalen Aufklärung besteht. Die Maximen des gemeinen Menschenverstandes
sind folgende\footnote{\cite[Vgl.][\S~40]{Kant:KritikderUrteilskraft2009},
\cite[][V: 294.14--295.19]{Kant:GesammelteWerke1900ff.}. Nach
\authorfullcite{Cohen:KantontheEthicsofBelief2014} handelt es sich um Analoga zu
den drei Formulierungen des \emph{Kategorischen Imperativs}
\parencite[vgl.][330]{Cohen:KantontheEthicsofBelief2014}.}:
\begin{nummerierung}
 \item Die \emph{Maxime der vorurteilsfreien Denkungsart} ist die Maxime des
 Verstandes und einer niemals passiven Vernunft, welche wir bereits in der
 Form des \enquote{Sapere aude! Habe Mut, dich deines \ori{eigenen} Verstandes zu
 bedienen!}\footnote{\cite[][A
 481]{Kant:BeantwortungderFrage:WasistAufklaerung?1977},
 \cite[][VIII: 35.6--7]{Kant:GesammelteWerke1900ff.}.} kennen.
 \item Die \emph{Maxime der erweiterten Denkungsart} ist die Maxime der
 Urteilskraft. Sie fordert von uns einen \emph{Perspektivenwechsel} im Denken
 und steht im Kontrast zur bornierten Art zu denken. Sie fordert uns dazu auf,
 an der Stelle jedes anderen zu denken, also bei unseren Urteilen und
 Gedankengängen zu berücksichtigen, ob andere diese ebenfalls als vernünftig
 ansehen können.
 \item Die \emph{Maxime der konsequenten Denkungsart} ist die Maxime der
 Vernunft. Sie fordert von uns, dass wir mit uns selbst einstimmig denken. Das
 heißt, wir sollen nicht beliebige Gedanken und Behauptungen aneinander reihen,
 sondern uns in unserem Denken an den Regeln unserer eigenen Vernunft
 orientieren. Es ist also gerade die Maxime, die zu bewahren zentrales Anliegen
 \name[Immanuel]{Kant}s im Anschluss an
 \authorcite{Wolff:Discursuspraeliminarisdephilosophiaingenere1996} ist. Diese
 Maxime sei \enquote{am schwersten zu erreichen und} könne \enquote{auch nur
 durch die Verbindung beider ersten und nach einer zur Fertigkeit gewordenen
 öfteren Befolgung derselben erreicht
 werden.}\footnote{\cite[][\S~40]{Kant:KritikderUrteilskraft2009}, \cite[][V:
 295.15--17]{Kant:GesammelteWerke1900ff.}.}
\end{nummerierung}
Interessant ist hier die Angabe, dass sich Vernunft im Denken nur durch die
Verbindung der beiden anderen Maximen erreichen lasse. Selbstdenken und
erweiterte Denkungsart scheinen also konstitutiv zu sein für ein Denken, das
weder der Beliebigkeit anheim fällt noch heteronom wird, sondern sich einzig
und allein denjenigen Regeln unterwirft, die unsere Vernunft ausmachen.

\section{Selbstdenken als Denken in Gemeinschaft}\label{section:sensuscommunis}
Dass wir selbst denken, unsere eigene Vernunft als obersten Probierstein der
Wahrheit ansehen und uns keinen Autoritäten beugen sollen, ist die
zentrale Maxime der Aufklärung. Aber Aufklärung fordert von uns auch, uns am
Maßstab der Vernunft zu orientieren; und dies ist nicht einfach eine
weitere Maxime, sondern eine direkte Folgerung der ersten Maxime, wie
\authorfullcite{Wolff:Discursuspraeliminarisdephilosophiaingenere1996} betont:
Selbst denken kann nur, wer sich selbst kompetent am Maßstab der Vernunft
orientiert. Deswegen fallen Unabhängigkeit und Vernunft wesentlich zusammen.
Nach \authorcite{Wolff:Discursuspraeliminarisdephilosophiaingenere1996} besteht der
Maßstab der Vernunft in der mathematischen Methode, die auch die Methode der
Philosophie sei, gerade weil sie nichts anderes als das Wesen der Vernunft zum
Ausdruck
bringe.\footnote{\cite[Vgl.][\S~161]{Wolff:Discursuspraeliminarisdephilosophiaingenere1996}.}
Nun verneint \name[Immanuel]{Kant} die
Möglichkeit, einen solchen allgemeinen Standard für vernünftiges Denken im Sinne
einer einheitlichen Methode anzugeben. Wie ist es dann aber möglich zu
gewährleisten, dass die Befolgung der ersten Maxime nicht in die Vernachlässigung der zweiten Maxime
mündet -- dass Selbstdenken nicht zu \singlequote{Freidenkerei} und Beliebigkeit
führt?

\subsection{Logischer Egoismus versus Pluralismus}
Wir haben bereits gesehen, dass sich die Vernunft an Grundsätzen orientieren
muss und dass diese Grundsätze wiederum auf ihre Vernünftigkeit hin befragt
werden müssen. Grundsätze des Vernunftgebrauchs sind entweder solche, die der
Vernunft selbst entstammen (Autonomie) oder solche, die ihr von außen angetragen
werden (Heteronomie). Wie entscheiden wir aber, ob ein Grundsatz ein solcher der
Vernunft oder ein Vorurteil ist?

In der \titel{Kritik der Urteilskraft} ebenso wie in der \titel{Anthropologie in
pragmatischer Hinsicht}\footnote{\cite[Siehe][B
166\,f.,]{Kant:AnthropologieinpragmatischerHinsicht1977} \cite[][VII:
228.10--229.2]{Kant:GesammelteWerke1900ff.}.} verweist \name[Immanuel]{Kant} auf
die Maxime der erweiterten oder liberalen Art zu denken. Er versteht darunter
die Maxime, \enquote{sich über die subjektiven
Privatbedingungen des Urteils, wozwischen so viele andere wie eingeklammert
sind, weg[zu]setzen \punkt{} und aus einem \ori{allgemeinen Standpunkte} (den
[man] nur dadurch erreichen kann, daß [man] sich in den Standpunkt anderer
versetzt) über sein eigenes Urteil [zu]
reflektier[en].}\footnote{\cite[\S~40]{Kant:KritikderUrteilskraft2009},
\cite[][V: 295.10--14]{Kant:GesammelteWerke1900ff.}.} Hiernach sollen wir nicht
nur unsere Grundsätze einem Verallgemeinerungstest unterstellen, sondern unsere
Urteile auch aus einer anderen Perspektive (aus der Sicht eines anderen)
betrachten. In der Anmerkung am Ende von \titel{Was heißt: sich im Denken
orientieren?} übergeht \name[Immanuel]{Kant} zwar diese auf die Gemeinschaft mit
anderen verweisende \distanz{mittlere} Maxime zwischen dem Selbstdenken und der Unterwerfung unter
Gesetze der Vernunft.\footnote{Claudio
\textcite[][133]{LaRocca:WasAufklaerungseinwird2004} weist darauf hin, dass die
zitierte Anmerkung die Differenz zwischen der ersten und der dritten Maxime
quasi mit einem Schlag überbrücke.} Er insistiert aber kurz zuvor im Haupttext
auf die soziale Eingebundenheit unseres Denkens.\footnote{\cite[Vgl.][A
325]{Kant:Washeisst:SichimDenkenorientieren?1977}, \cite[][VIII:
144.17--22]{Kant:GesammelteWerke1900ff.}: \enquote{Zwar sagt man:
die Freiheit zu sprechen, oder zu schreiben, könne uns zwar durch obere Gewalt,
aber die Freiheit zu denken durch sie gar nicht genommen werden. Allein, wie
viel und mit welcher Richtigkeit würden wir wohl denken, wenn wir nicht
gleichsam in Gemeinschaft mit anderen, denen wir unsere und die uns ihre
Gedanken mitteilen, dächten!}} Er sieht Intersubjektivität für
einen wesentlichen Gesichtspunkt bei der Vereinbarkeit von Selbstdenken und
Richtigdenken an, der sich daraus ergibt, dass Vernunftmaßstäbe nicht in der
Form einer allgemeinen Methode wie der mathematischen zu haben
sind.\footcite[Vgl.][\pno~214\,f.]{ONeill:AufgeklaerteVernunft1996}

Man kann an dieser Stelle noch mit Gründen bezweifeln, dass dies bereits das
allgemeine Vorurteil über die \index{Kant, Immanuel}kantische Philosophie aus den Angeln
hebt, diese sei in ihrem Kern solipsistisch und trage einen monologischen
Charakter. Denn \name[Immanuel]{Kant} behauptet, dass man sich \emph{dadurch}
von den subjektiven Privatbedingungen emanzipiere und aus einem allgemeinen
Standpunkt reflektiere, dass
\begin{quote}
  man sein Urteil an anderer nicht sowohl wirkliche, als vielmehr bloß mögliche
  Urteile hält und sich in die Stelle jedes anderen versetzt, indem man bloß von
  den Beschränkungen, die unserer eigenen Beurteilung zufälligerweise anhängen,
  abstrahiert; welches wiederum dadurch bewirkt wird, daß man das, was in dem
  Vorstellungszustande Materie, {d.\,i.} Empfindung ist, soviel möglich wegläßt
  und lediglich auf die formalen Eigentümlichkeiten seiner Vorstellungen oder
  seines Vorstellungszustandes
  achthat.\footnote{\cite[\S~40]{Kant:KritikderUrteilskraft2009}, \cite[V:
  294.1--8]{Kant:GesammelteWerke1900ff.}.}
\end{quote}
Auch wenn man einmal von den Schwierigkeiten absieht, die die Forderung mit sich
bringt, von der Materie eines Urteils abzusehen und nur auf das Formale zu
achten, drängt sich doch eine Beobachtung auf: \name[Immanuel]{Kant} betrachtet
zumindest an dieser Stelle die gedankliche Operation, in der wir uns in den Standpunkt anderer
versetzen, nicht als Akt der Kommunikation und des gedanklichen Austausches, sondern als die einsame
Überlegung eines Denkers, der bloß von seinen eigenen Eigentümlichkeiten
absieht. Er verweist nicht auf den öffentlichen Vernunftgebrauch der
Aufklärungsschrift, sondern auf das Abstraktum einer allgemeinen
Menschenvernunft.\footnote{\Revision[Thei, Pelletier]{Dass wir auch abhängig
sind von einer Vorgeschichte, die es erst ermöglicht, dass wir
uns als autonome Subjekte konstituieren, ist eine Überlegung, die sich bei
\name[Immanuel]{Kant} noch nicht findet (und sich bei ihm vielleicht noch gar
nicht finden kann). \authorfullcite{Foucault:WasistAufklaerung1990} wird aus
solchen Gründen fordern, philosophische Kritik müsse in ihrem Bemühen um
Aufklärung heute nicht mehr transzendental, sondern archäologisch verfahren und
statt vermeintlich zeitloser formaler Strukturen die für uns (für unser Denken
und Handeln) konstitutiven Ereignisse und Umstände aufsuchen
\parencite[vgl.][]{Foucault:WasistAufklaerung1990}.}} Schließlich -- so ließe
sich eine entsprechende Position im Sinne der Aufklärung verteidigen -- muss die
Vernünftigkeit von Grundsätzen des Denkens von jeder faktischen Gemeinschaft und
ihren geteilten Überzeugungen in gewisser Hinsicht unabhängig sein, wenn der
Vernunft nicht doch der Beigeschmack von Heteronomie anhaften soll. Jedenfalls
kann ein Grundsatz nicht allein darum vernünftig oder unvernünftig sein, weil
die existierenden Mitglieder einer Kommunikationsgemeinschaft ihn als solchen
anerkennen oder ablehnen.\footnote{Siehe dazu auch
\cite[][20]{ONeill:ConstructionsofReason1989}: \enquote{At first thought the
idea of modeling reason on free debate may seem to add nothing. First, we may
suspect, this account too will only be negative instruction: Debates do not
usually produce agreement; hence this image adds nothing to that of the
\enquote{tribunal} of reason. Second, debates presuppose reason, so we cannot
draw on the notion of debate to explain the authority of reason. Third, we may doubt
that the prospects for uncoerced debate are any rosier that those for tribunals
that do not rest on power relations.}} Rassistische Vorurteile bleiben ja auch
dann Vorurteile, wenn wir ausschließlich von Rassisten umgeben sind. Dies mag
wie eine Leerstelle in \name[Immanuel]{Kant}s System aussehen, sich letztlich
jedoch als korrekte Einsicht erweisen. Es bleibt daher mit Birgit
\name[Birgit]{Recki} erst zu fragen: \enquote{Gibt es bei Kant Anknüpfungspunkte
für die Einsicht in die Angewiesenheit der menschlichen Erkenntnis auf Sprache,
in die Notwendigkeit der Mitteilung, des Austausches und der Auseinandersetzung in dem, worauf sich das
Interesse vernünftiger Wesen richtet -- in den dialogischen Charakter von
Vernunftleistungen?}\footnote{\cite[114]{Recki:enquoteAnderStelle[je]desanderendenken2006}.
Sie stellt daher explizit die Frage, ob diese zweite Maxime des gemeinen
Menschenverstandes überhaupt etwas mit Kommunikation zu tun hat
\parencite[vgl.][116]{Recki:enquoteAnderStelle[je]desanderendenken2006}.} Sollte
die Maxime der erweiterten Denkungsart nicht auf tatsächliche Akte der
Kommunikation, sondern auf das Abstraktum einer allgemeinen Menschenvernunft
verweisen, wäre sie mit einer individualistischen Hintergrundtheorie vereinbar.

\phantomsection\label{Abschnitt:KantunddieOeffentlichkeitderVernunft}Der
Interpretation \name[Immanuel]{Kant}s als eines solipsistischen und
monologischen Denkers ist bereits mehrfach ausführlich und mit überzeugenden
Argumenten widersprochen
worden.\footnote{\cite[Siehe][]{Hinske:PluralismusundPublikationsfreiheitimDenkenKants1986},
\cite{Hoeffe:EinerepublikanischeVernunft1996},
\cite{Recki:enquoteAnderStelle[je]desanderendenken2006} sowie \cite[][41,
56]{Wood:KantandtheProblemofHumanNature2003}, und
\cite[][325--328]{Pieper:EthikalsVerhaeltnisvonMoralphilosophieundAnthropologie1978},
und ausführlich
\cite{Keienburg:ImmanuelKantunddieOeffentlichkeitderVernunft2011}.} Es
gibt viele aussagekräftige Textbelege gegen diese Interpretation, von denen mir
eine Passage aus der Anthropologie am deutlichsten zu sein scheint. Die liberale oder
erweiterte Art des Denkens richtet sich nicht gegen den Selbstdenker überhaupt,
sondern gegen den \enquote{logische[n] Egoist[en]}, wie \name[Immanuel]{Kant} im
Anschluss an Georg
\authorcite{Meier:Vernunftlehre1752}\footnote{\cite[Vgl.][\S~202]{Meier:Vernunftlehre1752}
und \cite[][\S~170]{Meier:AuszugausderVernunftlehre1752} (\cite[][XVI:
413.29]{Kant:GesammelteWerke1900ff.}).
Nach \textcite[64]{Hinske:ZwischenAufklaerungundVernunftkritik1993} liegt der
Ursprung dieses Ausdrucks bei \authorcite{Meier:Vernunftlehre1752}.} denjenigen
nennt, der \enquote{es für unnötig [erachtet], sein Urteil auch am Verstande
anderer zu
prüfen}\footnote{\cite[][BA~6]{Kant:AnthropologieinpragmatischerHinsicht1977},
\cite[VII: 128.31--32]{Kant:GesammelteWerke1900ff.}. Zum Begriff des Egoismus in
der Philosophie des 18.\ Jahrhunderts siehe
\cite[][200--227]{Halbfass:DescartesFragenachderExistenzderWelt1968}. Einen
 Überblick über die Verwendung von \enquote{Egoismus} als Vorlage für
 \name[Immanuel]{Kant} liefert
 \cite[][15--46]{Heidemann:KantunddasProblemdesmetaphysischenIdealismus1998},
 der dabei jedoch ebenso wie \authorcite{Halbfass:DescartesFragenachderExistenzderWelt1968} \name[Immanuel]{Kant}s
 Vorstellungen von dogmatischem und problematischem Idealismus fokussiert und sich
weniger für den \emph{logischen} Egoismus interessiert.}. Dabei sei gerade aus
diesem Umstand die Freiheit öffentlichen Meinungsaustauschs so nötig, weil
Selbstdenken nur demjenigen möglich sei, der in Gemeinschaft mit anderen
denkt.\footnote{\cite[Vgl.][BA~6]{Kant:AnthropologieinpragmatischerHinsicht1977},
\cite[VII: 128.33--129.3]{Kant:GesammelteWerke1900ff.}: \enquote{Es ist aber so
gewiß, daß wir dieses Mittel, uns der Wahrheit unseres Urteils zu versichern,
nicht entbehren können, daß es vielleicht der wichtigste Grund ist, warum das
gelehrte Volk so dringend nach der \ori{Freiheit der Feder} schreit; weil, wenn
diese verweigert wird, uns zugleich ein großes \ori{Mittel} entzogen wird, die
Richtigkeit unserer eigenen Urteile zu prüfen, und wir dem Irrtum preis gegeben
werden.}} Die Möglichkeit der Mitteilung ist auch nach der \titel{Kritik der
reinen Vernunft} das entscheidende äußere Indiz der Wahrheit und Vernünftigkeit
einer Überzeugung, nämlich ob es auf einem objektiven Grund oder bloß auf der
subjektiven Beschaffenheit des Subjekts
beruht.\footnote{\cite[Vgl.][B 848\,f.,]{Kant:KritikderreinenVernunft2003}
\cite[][III: 532.3--16]{Kant:GesammelteWerke1900ff.}.} In der Aufklärungsschrift
erklärt er, dass der Ausgang aus selbst verschuldeter Unmündigkeit nur schwerlich den
Einzelnen, sondern nur einem Publikum möglich sei, dem ein Mindestmaß an Presse-
und Meinungsfreiheit gegeben
ist.\footnote{\cite[Vgl.][A~482--484]{Kant:BeantwortungderFrage:WasistAufklaerung?1977},
\cite[][VIII: 36.4--37]{Kant:GesammelteWerke1900ff.}.}


Den Ausdruck \enquote{logischer Egoismus} musste \name[Immanuel]{Kant} nicht neu
erfinden; er konnte auf eine breite Verwendungsweise in der Philosophie des 18.
Jahrhunderts zurückgreifen. Nach
\authorcite{Hinske:PluralismusundPublikationsfreiheitimDenkenKants1986} kam der
Begriff des \enquote{Egoismus} in der Bedeutung von \enquote{Solipsismus} durch
\authorcite{Wolff:Discursuspraeliminarisdephilosophiaingenere1996} in die
deutsche
Sprache.\footnote{\cite[Vgl.][39]{Hinske:PluralismusundPublikationsfreiheitimDenkenKants1986}.
In der Tat scheinen Ausdrücke \enquote{Egoismus} und \enquote{Solipsismus} ihre
Bedeutung seitdem vertauscht zu haben, insofern \enquote{Solipsismus} zunächst
in der Bedeutung von Selbstsucht und Selbstgefälligkeit verwendet wird
\parencite[vgl.][224--227]{Halbfass:DescartesFragenachderExistenzderWelt1968}.}
Dieser schreibt:
\begin{quote}
 Die Monisten sind abermahl von zweyerley Gattung, entweder \ori{Idealisten}
 oder \ori{Materialisten}. Jene geben blosse Geister oder auch solche Dinge zu,
 welche nicht aus Materie bestehen\punkt ; halten aber die Welt und die darinnen
 befindelichen Cörper für blosse Einbildungen \punkt\ und sehen sie nicht anders
 als einen regulirten Traum an. \punkt\ Endlich die Idealisten geben entweder
 mehr als ein Wesen zu, oder halten sich für das einige würckliche Wesen. Jene
 werden \ori{Pluralisten}; diese hingegen \ori{Egoisten}
 genennet.\footnote{\cite[][Vorrede zu der andern Auflage (nicht
 paginiert)]{Wolff:VernuenftigeGedankenvondenKraeftendesmenschlichenVerstandesundihremrichtigenGebraucheinErkenntnisderWahrheit1978}.}
\end{quote}
Nach der Position des Egoismus gibt es nur mich und nichts und niemanden sonst,
keine körperliche Welt, keine anderen \distanz{Geister}, weder die Leiber anderer Menschen, noch meinen
eigenen. Ich muss in der Lage sein, mein Denken und Erkennen
vollständig selbst zu generieren -- es ist in der Folge somit eine Vorstellung
von geistiger Autarkie, nicht nur von Autonomie vonnöten. Und dieses Moment der geistigen
Autarkie findet sich auch im Begriff des  des \emph{logischen} Egoisten, der
wohl auf Georg \authorcite{Meier:Vernunftlehre1752}
zurückgeht\footnote{\cite[Vgl.][64]{Hinske:ZwischenAufklaerungundVernunftkritik1993}.
Nach \authorcite{Hinske:PluralismusundPublikationsfreiheitimDenkenKants1986}
beginnt sich der Begriff des Egoismus genau an dieser Stelle entscheidend in
Richtung seiner heutigen Bedeutung zu entwickeln
\parencite[vgl.][39]{Hinske:PluralismusundPublikationsfreiheitimDenkenKants1986}.}.
\authorcite{Meier:Vernunftlehre1752} schreibt in der \titel{Vernunflehre}:
\begin{quote}
Die Egoisten glaubten, daß sie allein würklich wären, und sie führten sich
selbst beständig im Munde. Weil nun die logischen Egoisten allemal sich selbst zum
Grunde anführen, warum ist das oder das wahr? weil ichs sage; warum ist das oder
das falsch? weil ichs sage: so haben sie daher diesen Namen bekommen. Dieses
Vorurteil ist so unverschämt und pedantisch, daß es keiner Widerlegung bedarf,
und gleichwohl werden die meisten Gelehrten durch dieses Vorurteil
regieret.\footcite[][\S~202]{Meier:Vernunftlehre1752}
\end{quote}
\authorcite{Meier:Vernunftlehre1752} zählt \enquote{die \ori{logische Egoisterey}} zu den logischen
Vorurteilen, die in einem jeweils zu großen Zutrauen oder Misstrauen bestehen,
hier also in einem \enquote{gar zu grosse[n] Vertrauen, durch Hochmuth und
Eigenliebe verblendet, auf sich selbst und auf die Stärke seines Verstandes}%
\footnote{\Cite[][\S~202]{Meier:Vernunftlehre1752}.}. Das Bemühen um vernünftige
Selbständigkeit im Denken ist dabei immer die Suche nach einem Mittelweg
zwischen der blinden Autoritätshörigkeit und dem Aufgehen in einer
philosophischen \distanz{Schule} auf der einen und dem ebenso unvernünftigen
Vertrauen auf die eigenen Meinungen und das eigene Fürwahrhalten auf der anderen
Seite. Und so stehen die erste und die zweite Maxime des Verstandes in einem
Spannungsverhältnis.

Wir benötigen den Austausch mit anderen, um selbst denken zu
können. Ohne diesen Austausch fehlte uns mindestens die Selbstsicherheit in der
Ausübung der Vernunft. Darauf macht \name[Immanuel]{Kant} wiederholt aufmerksam. Und in
der Vorrede zur ersten Auflage der \titel{Kritik der reinen Vernunft} verweist
er auf die öffentliche Prüfung als Kriterium der Wissenschaftlichkeit von
Erkenntnissen: nur was einer öffentlichen Prüfung standhält, das ist zumindest
\emph{prima facie} vernünftig und vertrauenswürdig.\footnote{\cite[Vgl.][A
xi]{Kant:KritikderreinenVernunft2003}, \cite[][IV:
9.33--38]{Kant:GesammelteWerke1900ff.}: \enquote{Unser Zeitalter ist das
eigentliche Zeitalter der Kritik, der sich alles unterwerfen muss.
\ori{Religion} durch ihre \ori{Heiligkeit} und \ori{Gesetzgebung} durch ihre
\ori{Majestät} wollen sich gemeiniglich derselben entziehen. Aber alsdann
erregten sie gerechten Verdacht wider sich, und können auf unverstellte Achtung
nicht Anspruch machen, die die Vernunft nur demjenigen bewilligt, was ihre freie
und öffentliche Prüfung hat aushalten können.}}
Der logische Egoist in \name[Immanuel]{Kant}s Darstellung glaubt, er könne die Richtigkeit
und Qualität seines Denkens und Erkennens selbst kontrollieren. Urteilen ist
jedoch eine Tätigkeit, die aus der gemeinsamen Praxis unter gegenseitiger
Kontrolle hervorgeht und erst danach zurückgezogen und privat betrieben werden
kann. Wobei wir aber doch die prinzipielle Möglichkeit haben müssen, unsere
Urteile an denen anderer zu vergleichen. Dies trifft sogar -- wie
\name[Immanuel]{Kant} explizit
betont\footnote{\cite[Vgl.][\S~2]{Kant:AnthropologieinpragmatischerHinsicht1977},
\cite[][VII: 129.3--8]{Kant:GesammelteWerke1900ff.}.} -- auch und gerade auf die
Mathematik zu, die wegen ihrer methodischen Strenge und Exaktheit ein Absehen
von intersubjektiver Kontrolle bei geübten Menschen in konkreten Einzelfällen
noch am ehesten erlaubt und daher die Illusion erzeugt, sie sei von Natur aus privat
zu betreiben. Eine solche Illusion verkennt nicht nur die Mittel, die zu
vernünftigen Urteilen führen, sondern das Wesen der Vernunft selbst. Denn
Vernunft kann es nur geben, wenn es eine gemeinsam kontrollierte Praxis des
Bewertens gibt.\footnote{\cite[Vgl.][B
766f.,]{Kant:KritikderreinenVernunft2003} \cite[][III:
484.10--14]{Kant:GesammelteWerke1900ff.}: \enquote{Auf dieser Freiheit beruht
sogar die Existenz der Vernunft, die kein diktatorisches Ansehen hat, sondern
deren Ausspruch jederzeit nichts als die Einstimmung freier Bürger ist, deren
jeglicher seine Bedenklichkeiten, ja sogar sein veto, ohne Zurückhaltung muß
äußern können.}} Möglicherweise ließe sich sogar
eine weitaus stärkere These vertreten:
Zumindest auf den ersten Blick klingt es nach einer provokanten Behauptung, wenn
\name[Immanuel]{Kant} in der \titel{Kritik der reinen Vernunft} über die Meinungs- und
Publikationsfreiheit schreibt:
\begin{quote}
  Auf dieser Freiheit beruht sogar die \myemph{Existenz} der Vernunft, die kein
  diktatorisches Ansehen hat, sondern deren Ausspruch jederzeit nichts als die
  Einstimmung freier Bürger ist, deren jeglicher seine Bedenklichkeit, ja sogar
  sein veto, ohne Zurückhalten muß äußern
  können.\footnote{\cite[][B~766\,f.,]{Kant:KritikderreinenVernunft2003}
  \cite[III: 484.10--14]{Kant:GesammelteWerke1900ff.}, \myherv .}
\end{quote}
Dieser Textpassage zufolge geht es nicht nur darum, dass wir durch den Austausch
mit anderen in unserem Urteilen und Schließen selbstsicherer werden und \emph{besser} zu denken
lernen. Die Vernunft selbst ist grundsätzlich kommunikativ
verfasst.\footnote{\cite[Vgl.][120]{Recki:enquoteAnderStelle[je]desanderendenken2006}:
\enquote{Wir dürfen daraufhin behaupten, daß die Vernunft, auch wenn Kant nicht
ausdrücklich sagt, sie vollziehe sich im stetigen Austausch und Abgleich mit
anderen Vernünften, intern kommunikativ, ja: dialogisch konzipiert ist.} Mir
hingegen scheint \name[Immanuel]{Kant} dies an der zitierten Stelle sehr ausdrücklich zu
sagen.}

\name[Immanuel]{Kant} fordert nicht nur eine in gewisser Hinsicht offene,
nämlich den freien Meinungsaustausch zulassende bürgerliche Gesellschaft; er
fordert auch von jedem Einzelnen eine \enquote{pluralistische} Haltung,
\enquote{d.\,i. die Denkungsart: sich nicht als die ganze Welt in seinem Selbst
befassend, sondern als einen bloßen Weltbürger zu betrachten und zu
verhalten.}\footnote{\cite[BA 8]{Kant:AnthropologieinpragmatischerHinsicht1977},
\cite[VII: 130.12--14]{Kant:GesammelteWerke1900ff.}.} \singlequote{Pluralismus}
meint bei \name[Immanuel]{Kant} jedoch nicht das Lob von
Meinungs\emph{vielfalt}, sondern das Bemühen, in freier Kommunikation zu
\emph{einer gemeinsamen} Überzeugung zu gelangen.
\begin{comment}
Unser \distanz{moderner} Pluralismus mit seiner Forderung nach
Toleranz\footnote{\name[Immanuel]{Kant} nennt den Namen der Toleranz
\enquote{hochmütig}
\mkbibparens{\cite[][A 491]{Kant:BeantwortungderFrage:WasistAufklaerung?1977},
\cite[][VIII: 40.30]{Kant:GesammelteWerke1900ff.}}. Siehe dazu
\cite{Weidemann:VonenquotebisweilenunvermeidlicherGeringschaetzung2010}.} und
Offenheit gründet in der Annahme, dass entsprechende Urteile -- bezüglich des
\distanz{richtigen} Lebensstils, der Religion etc.\ -- keine epistemisch
fundierte Antwort erlauben, sei es, weil sie nicht wahrheitsfähig sind
(\enquote{de gustibus non est disputandum}), sei es, weil wir die Wahrheit nicht
ausmachen (Gott nicht erkennen) können. Es gehört zu unserem modernen,
aufgeklärten Selbstverständnis, dass die Wahl des religiösen Bekenntnisses und
die Ansichten über den \distanz{Sinn des Lebens} und die \distanz{Bestimmung des
Menschen} dem subjektiven Belieben anheim gestellt sind. Religiöse Toleranz,
sexuelle Selbstbestimmung und Gleichwertigkeit unterschiedlicher Lebensentwürfe
gehören zu dem, was wir ganz selbstverständlich als Erbe der Aufklärung und
Errungenschaften der Moderne ansehen. Dieser Begriff von Pluralismus gründet in
einer (heute vielleicht dominanten) Konzeption von Autonomie, die diese als
wesentlich individualistisch und egozentrisch begreift. Diese Konzeption trennt
Autonomie und Vernunft, insofern Autonomie darin besteht, unsere je eigenen
Präferenzen zu verfolgen, und der Vernunft die nachrangige Aufgabe zukommt, das
Verfolgen dieser Präferenzen effizient zu gestalten und mit den Präferenzen
anderer zu koordinieren.\footnote{Siehe dazu auch
\cite[][\pno~216\,f.]{ONeill:AufgeklaerteVernunft1996}.} Ein Dialog als
Grundlage unserer Vernunft und vernünftiger Einsichten und Entscheidungen kann
sich in \name[Immanuel]{Kant}s Konzeption hingegen nicht mit dem Anspruch
begnügen, einen \emph{modus vivendi} zur friedlichen Koexistenz verschiedener Gruppen auszuhandeln. Stattdessen erweist
sich die Vernünftigkeit der beteiligten Überzeugungen selbst erst im
Versuch ihrer Mitteilung, also der Überzeugung anderer. Auch die von uns
verfolgten Ziele sind dabei Gegenstand vernünftiger Erwägung und damit der
Mitteilbarkeit zugänglich.\footnote{Ich werde später zeigen, inwiefern
\name[Immanuel]{Kant} dennoch der Tatsache Rechnung trägt, dass es eine
Vielfalt an unterschiedlichen Präferenzen gibt, die je individuell zu verfolgen
ebenso zu unseren vernünftigen Rechten gehört. Das schließt nicht aus, dass wir
sie als vernünftig oder unvernünftig bewerten können und dass dies auch auf der
Möglichkeit der Kommunikation mit anderen beruht. Siehe dazu die Überlegungen
zur \enquote{Klugheit} in Kapitel \ref{subsection:aufklaerungundpraxis}.}
\end{comment}
Im Gegensatz zu neueren um Forderungen nach
Toleranz\footnote{\name[Immanuel]{Kant} nennt den Namen der Toleranz
\enquote{hochmütig}
\mkbibparens{\cite[][A 491]{Kant:BeantwortungderFrage:WasistAufklaerung?1977},
\cite[][VIII: 40.30]{Kant:GesammelteWerke1900ff.}}. Siehe dazu
\cite{Weidemann:VonenquotebisweilenunvermeidlicherGeringschaetzung2010}.} und
Offenheit zentrierten Pluralismusbegriffen wie
\authorcite{Rawls:TheLawofPeoples1999}' \enquote{\emph{reasonable pluralism}} meint Pluralismus bei
\name[Immanuel]{Kant} nicht die Anerkennung unüberwindlicher Differenzen
hinsichtlich religiöser und philosophischer Überzeugungen oder -- wie
\authorcite{Rawls:TheLawofPeoples1999} sagt -- \enquote{comprehensive
doctrines}.\footnote{\cite[Vgl.][136]{Rawls:TheLawofPeoples1999}: \enquote{The
fact of reasonable pluralism {\punkt} means that the differences between
citizens arising from their comprehensive doctrines, religious and nonreligious,
may be irreconcilable.} \cite[Siehe auch][\pno~11, 12, 31, 124,
131\,f.]{Rawls:TheLawofPeoples1999}.
\authorcite{Rawls:TheLawofPeoples1999} wertet den vernünftigen Pluralismus als zentralen
Bestandteil liberaler Demokratien; \cite[vgl.][124]{Rawls:TheLawofPeoples1999}: \enquote{A
basic feature of liberal democracy is the fact of reasonable pluralism---the
fact that a plurality of conflicting reasonable comprehensive doctrines, both
religious and nonreligious (or secular), is the normal result of the culture of
its free institutions.}} \name[Immanuel]{Kant}s Pluralismuskonzeption beschreibt
anderes als der \emph{reasonable pluralism} des 20. Jahrhunderts; er bezieht sich direkt auf die
erweiterte Denkungsart der zweiten Maxime des gemeinen Menschenverstandes. Diese
erweiterte Denkungsart nennt \name[Immanuel]{Kant} \enquote{Pluralismus} und setzt sie dem \enquote{logischen Egoismus}
entgegen.\footnote{\cite[Vgl.][\S~2]{Kant:AnthropologieinpragmatischerHinsicht1977},
\cite[][VII: 128.21--130.21]{Kant:GesammelteWerke1900ff.}. Zum Begriff des
Egoismus in der Philosophie des 18.\ Jahrhunderts siehe
\cite[][200--227]{Halbfass:DescartesFragenachderExistenzderWelt1968}. Einen
 Überblick über die Verwendung von \enquote{Egoismus} als Vorlage für
 \name[Immanuel]{Kant} liefert
 \textcite[vgl.][15--46]{Heidemann:KantunddasProblemdesmetaphysischenIdealismus1998},
 der dabei jedoch ebenso wie \authorcite{Halbfass:DescartesFragenachderExistenzderWelt1968} \name[Immanuel]{Kant}s
 Vorstellungen von dogmatischem und problematischem Idealismus -- den
 sogenannten \emph{metaphysischen} Egoismus -- fokussiert und sich weniger für
 den \emph{logischen} Egoismus interessiert.}

\subsection{Aufklärung der
Urteilskraft}\label{subsection:AufklaerungderUrteilskraft}
Dass es keinen einfachen Algorithmus gibt, der vernünftige
Prinzipien von Vorurteilen unterscheidet, ist vielleicht einfach eine Tatsache,
mit der wir immer umgehen müssen. Sie macht es so schwer, sich gänzlich von
Vorurteilen zu befreien, und verhindert, dass aus dem Zeitalter der Aufklärung
ein aufgeklärtes Zeitalter wird.\footnote{\authorfullcite{ONeill:AufgeklaerteVernunft1996} schlägt vor,
\name[Immanuel]{Kant} so zu lesen, dass zwar die Prinzipien der Vernunft nicht
sozial konstruiert sind, sie aber so verfasst sein müssen, dass sie die
\enquote{Bedingungen für die \ori{Konstruktion} von Intersubjektivität
erfüllen} \parencite[][213]{ONeill:AufgeklaerteVernunft1996}. Insofern sind sie
nicht davon abhängig, welche Überzeugungen und Prinzipien faktisch Zustimmung
finden; sie seien aber daran gebunden, Kommunikation möglich zu
machen. \cite[Vgl.][219]{ONeill:AufgeklaerteVernunft1996}:
\enquote{Kant gründet Vernunft nicht auf tatsächlichen Konsens oder die
Übereinstimmung und die Standards irgendeiner historischen Gemeinschaft; er
gründet sie auf die Zurückweisung aller Prinzipien, die die Möglichkeit von
strukturell unbegrenzt offener Kommunikation und Interaktion ausschließen.}
Auch dies stelle freilich nur eine Einschränkung der Menge möglicher Grundsätze
dar, ohne uns auf bestimmte Grundsätze festzulegen.}
Es gibt keinen Algorithmus, kein einfach anzuwendendes
Entscheidungsverfahren zur Vermeidung von Vorurteilen. Es bedarf zum
Selbstdenken nicht einfach einer an der Mathematik orientierten
Methodik\footnote{Wie \name[Immanuel]{Kant} nachdrücklich hervorhebt, ist die
Mathematik und ihre Methodik selbst in der \emph{gemeinsamen} Vernunftausübung
gegründet (\cite[vgl.][BA~6]{Kant:AnthropologieinpragmatischerHinsicht1977},
\cite[VII: 129.3--8]{Kant:GesammelteWerke1900ff.}).}, sondern einer gereiften
Urteilskraft,\footnote{In diese Richtung geht auch die Interpretation in
\cite{Enskat:BedingungenderAufklaerung2008}.} die sich naturgemäß nicht mit ein
paar Regeln hinreichend beschreiben lässt, sondern im \emph{gemeinsamen} Umgang
mit unseren Erkenntnissen (mit Aussagen und Begriffen) eingeübt und ausgebildet
werden muss. Deswegen ist Aufklärung nicht einzelnen Denkern, sondern nur einem
Publikum möglich. Somit unterscheidet sich \name[Immanuel]{Kant} von
\authorcite{Wolff:Psychologiaempirica1968} zwar nicht hinsichtlich der Ansicht,
dass freies Denken ein kompetentes Denken ist, wohl aber hinsichtlich des
Erwerbs der Kompetenzen: Während \authorcite{Wolff:Psychologiaempirica1968} den
Mathematikunterricht in seiner Methodik zum Vorbild nehmen kann und muss, steht
aus \name[Immanuel]{Kant}s Sicht die gemeinsame Ausbildung der Urteilskraft
Pate.\phantomsection\label{Abschnitt:KantunddieOeffentlichkeitderVernunft-Ende}

Die aufgeklärte \emph{Urteilskraft} ist es dann auch, die zur Krise der
Metaphysik führt, und zwar gerade weil das Kriterium der freien Einstimmung zu
dem Schluss führt, dass Metaphysik auf keinem vernünftigen Fundament erbaut sein
kann. Die der Vernunftkritik vorgängige Gleichgültigkeit in Fragen der
Metaphysik sei \enquote{offenbar die Wirkung nicht des Leichtsinns, sondern der
gereiften \ori{Urteilskraft} des Zeitalters, welches sich nicht länger durch
Scheinwissen hinhalten läßt}\footnote{\cite[][A
xi]{Kant:KritikderreinenVernunft2003}, \cite[][IV:
9.2--4]{Kant:GesammelteWerke1900ff.}.}. Die Vorreden zur \titel{Kritik der
reinen Vernunft} weisen die Notwendigkeit einer Vernunftkritik als
Anwendungsfall des Pluralismus aus, denn nur auf Grundlage der Maxime der
erweiterten Denkungsart lässt sich von der Tatsache fehlender Einhelligkeit
unter Metaphysikern auf die Unwissenschaftlichkeit der Disziplin schließen.

Dass Wissen mitteilbar ist, ist ihm nicht äußerlich. Es ist keine Forderung, die
wir von außen als plausibel oder für unsere epistemische Praxis notwendig dem
Wissen erst nachträglich abverlangen. Möglicherweise sagt sie uns sogar
etwas darüber, was Wissen ist; sie formuliert eine Beschränkung dessen, was wir
als Wissen akzeptieren können. Die Mitteilbarkeit ist als Prüfstein für die
Wahrheit eines Urteils aber auch bedeutsam für die (aus Sicht der Aufklärung
enorm wichtige) Unterscheidung von Überzeugung und Überredung.\footnote{\enquote{Der Probierstein des Fürwahrhaltens, ob es Überzeugung oder bloße Überredung
  sei, ist also äußerlich die Möglichkeit, dasselbe mitzuteilen und das
  Fürwahrhalten für jedes Menschen Vernunft gültig zu befinden; denn alsdenn ist
  wenigstens eine Vermutung, der Grund der Einstimmung aller Urteile, unerachtet
  der Verschiedenheit der Subjekte untereinander, werde auf dem
  gemeinschaftlichen Grunde, nämlich dem Objekte beruhen, mit welchem sie daher
  alle zusammenstimmen und dadurch die Wahrheit des Urteils beweisen
  werden} \mkbibparens{\cite[][B 488\,f.,]{Kant:KritikderreinenVernunft2003}
  \cite[][III: 532.9--16]{Kant:GesammelteWerke1900ff.}}.}
Überzeugungen machen die aktive, aufgeklärte Art des
Überzeugungserwerbs aus, Überredungen die passive und unmündige Art, etwas von
anderen zu übernehmen. Ich werde hierauf in Kapitel
\ref{section:KantsEthicsofBelief} eingehen.

Wir verfügen über Wissen, wenn unser Urteil mit dem Objekt, welches wir
beurteilen, übereinstimmt. Dies ist zunächst die \enquote{Namenerklärung der
Wahrheit}\footnote{\cite[][B 82]{Kant:KritikderreinenVernunft2003}, \cite[][III:
79.9]{Kant:GesammelteWerke1900ff.}.}. Aber sie verschafft uns kein einheitliches
Verfahren, die Wahrheit eines Urteil zu beurteilen. Die einzige sinnvolle
Prüfung, der wir unsere Urteile unterwerfen können, ist der Vergleich mit den
Urteilen anderer.\footnote{\enquote{Erkenntnisse und Urteile müssen sich, samt
der Überzeugung, die sie
  begleitet, allgemein mitteilen lassen; denn sonst käme ihnen keine
  Übereinstimmung mit dem Objekt zu; sie wären insgesamt ein bloß subjektives
  Spiel der Vorstellungskräfte, gerade so wie es der Skeptizism verlangt}
  \mkbibparens{\cite[][\S~21]{Kant:KritikderUrteilskraft2009}, \cite[][V:
  238.19--23]{Kant:GesammelteWerke1900ff.}.}.} Wenn wir Wissen zu haben
  beanspruchen, dann sollte es möglich sein, dieses Wissen mitzuteilen. Wissen
  zu haben ist eine genuin soziale Angelegenheit. Um
Wissen kann es sich daher nur bei \emph{mitteilbaren} Erkenntnissen handeln;
wenn wir etwas wissen, so ist es möglich, dass jemand anderes dieses Wissen
erlangt, indem wir es ihm mitteilen.

Eine weitere Bedeutung der pluralistischen Denkweise findet sich in
\name[Immanuel]{Kant}s Anmerkungen in seinem Handexemplar von
\authorcite{Baumgarten:Metaphysica---Metaphysik2011}s \titel{Metaphysica}:
\begin{quote}
 Allein in ansehung des bescheidnen Urtheils über den Werth seiner eignen
 wissenschaft und der Mäßigung des Eigendünkels und \ori{egoismus}, den eine
 Wissenschaft giebt, wenn sie allein im Menschen residirt, ist etwas nöthig, was
 dem gelehrten humanitaet gebe, damit er nicht sich selbst verkenne und seinen
 Kräften zu viel Zutraue.\\ Ich nenne einen solchen Gelehrten einen Cyclopen. Er
 ist ein egoist der Wissenschaft, und es ist ihm noch ein Auge nöthig, welches
 macht, daß er seinen Gegenstand noch aus dem Gesichtspunkte anderer Menschen
 ansieht. Hierauf gründet sich die humanitaet der Wissenschaften, d.\,i.\ die
 Leutseeligkeit des Urtheils, dadurch man es andrer Urtheil mit unterwirft, zu
 geben. \punkt\ Nicht die Stärke, sondern das einäugigte macht hier den Cyclop.
 Es ist auch nicht gnug, viel andre Wissenschaften zu wissen, sondern die
 Selbsterkenntnis des Verstandes und der Vernunft. \ori{Anthropologia
 transscendentalis}.\footnote{\cite[][\nopp 903]{Kant:Reflexionen1900ff.},
 \cite[][XV: 394.23--395.8,395.29--32]{Kant:GesammelteWerke1900ff.}. Nach
 \name[Erich]{Adickes} stammt dieser Eintrag aus den Jahren 1776-1778.}
\end{quote}
Es geht an dieser Stelle nicht um die Gewissheit und Verlässlichkeit einer
Erkenntnis, sondern um die korrekte Bewertung ihrer Bedeutung. Über eine reife
Urteilskraft verfügt, wer weiß, worauf es ankommt, wie \name[Immanuel]{Kant} an
anderer Stelle notiert.\footnote{\enquote{Die obere Erkentniskrafte. 1.) Er
[versteht] weiß, was er will; 2) weiß, worauf es ankommt; 3. er sieht ein,
worauf es hinausläuft. 1.) richtiger Verstand. 2) reife Urtheilskraft. 3.
geläuterte Vernunft} \mkbibparens{\cite[][\nopp 455]{Kant:Reflexionen1900ff.},
\cite[][XV: 188.6--8]{Kant:GesammelteWerke1900ff.}}.}
Wir neigen dazu, die Erkenntnisse derjenigen Disziplin, an deren Förderung wir
jeweils selbst mitwirken, für bedeutsamer zu halten, als sie es tatsächlich aus
einer allgemeinen Perspektive sind. Dies ist einerseits verständlich und fast
unvermeidlich, andererseits aber auch ganz einfach zu korrigieren: Wir müssen
lediglich die von unserer Einschätzung unterschiedene Bewertung durch andere
ernst nehmen. Dann können wir die Fehleinschätzung bereits ohne tiefgehende
Auseinandersetzung mit dem Wert wissenschaftlicher Forschung vermeiden. Der
Gelehrte dürfe dazu auch nicht zu eitel sein, sein Urteil auch an dem Urteil von
Menschen zu prüfen, die nicht seinem Stand angehören. Der Laie habe nicht
\emph{per se} ein geringeres Vermögen, die Relevanz wissenschaftlicher
Erkenntnisse zu bewerten. Er hat sogar einen Vorteil, und zwar den der
Unparteilichkeit.\footnote{\cite[Vgl.][A 66]{Kant:ImmanuelKantsLogik1977},
\cite[][IX: 48.13--16]{Kant:GesammelteWerke1900ff.}: \enquote{Die Schule hat
ihre Vorurteile so wie der gemeine Verstand. Eines verbessert hier das andre. Es ist daher
wichtig, ein Erkenntnis an Menschen zu prüfen, deren Verstand an keiner Schule
hängt.}}

Das Unternehmen einer \emph{anthropologia transscendentalis} als einer
Selbsterkenntnis des Verstandes und der Vernunft kann einerseits auf solche
Überlegungen, andererseits natürlich auch auf das Unternehmen einer Kritik der
reinen Vernunft verweisen. Letztlich fallen beide Sichtweisen zusammen, wenn man
die Vernunftkritik vor eben diesem Hintergrund eines
\singlequote{Weltbegriffs}\footnote{Siehe hierzu
Kap.~\ref{chapter:AufklaerungundWissenschaft} dieser Arbeit.} der Philosophie
liest, dem es um die Humanität der Wissenschaften geht.
In einer Erläuterung der zitierten Stelle jedenfalls klingt der Verweis auf die
Vernunftkritik deutlich an:
\begin{quote}
 Das zweyte Auge ist also das der Selbsterkentnis der Menschlichen Vernunft,
 ohne welches wir kein Augenmaas der Größe unserer Erkentnis haben. \punkt\ Der
 \ori{egoismus} rührt daher, weil sie den Gebrauch, welchen sie von der Vernunft
 in ihrer Wissenschaft machen, weiter ausdehnen und auch in anderen Feldern vor
 hinreichend halten.\footnote{\cite[][\nopp 903]{Kant:Reflexionen1900ff.},
 \cite[][XV: 395.17--19,24--27]{Kant:GesammelteWerke1900ff.}.}
\end{quote}
Wer ernsthaft auf die Einschätzungen anderer hört, der stelle fest, dass mit
aller Wissenschaft die wahren Fragen aus der Perspektive der Humanität noch
nicht beantwortet sind. Dies scheint mir das Ergebnis zu sein, das sich nach
\name[Immanuel]{Kant} einstellt, wenn der Metaphysiker auf das Urteil anderer --
in diesem Fall der Indifferentisten -- achtet.

\section{Zusammenfassung}

\name[Immanuel]{Kant}s Aufklärungsprogramm mit seiner Forderung, sich aus
selbst verschuldeter Unmündigkeit zu befreien, betont die intellektuelle Freiheit
und Selbständigkeit im Gebrauch des oberen Erkenntnisvermögens. Selbstdenken
bedeutet dabei den je eigenen Gebrauch der Vernunft und ist daher an den Erwerb
intellektueller \emph{Kompetenzen} gebunden, wie
\authorcite{Wolff:Discursuspraeliminarisdephilosophiaingenere1996} unermüdlich
hervorhob. Wer diese Kompetenzen nicht hat, der wird in seinem Urteil keine
andere Chance haben, als sich fremden Autoritäten zu unterwerfen, denen er ein
kompetentes Urteil zutraut. Wer hingegen selbst sicher Urteilen kann, wird auch
seine Vernunft selbst gebrauchen. Während
\authorcite{Wolff:Discursuspraeliminarisdephilosophiaingenere1996} jedoch die
Beherrschung einer bestimmten Methodik in das Zentrum seiner Überlegungen schob,
verwirft \name[Immanuel]{Kant} den Gedanken, es gebe eine solche einheitliche
Methode der Vernunft und verweist stattdessen auf den allgemeinen Standpunkt
einer \singlequote{erweiterten Denkungsart} und die damit verbundene
intersubjektive Kontrolle des je eigenen Denkens.


Aufklärung ist in weiten Teilen eine Frage der Ausbildung der Urteilskraft
im Austausch mit anderen.
Daher ist intellektuelle Selbständigkeit nicht als Autarkie misszuverstehen, da
vernünftige Freiheit nur in Gemeinschaft möglich ist, für die uns
\name[Immanuel]{Kant} selbst Metaphern aus dem Bereich der Politik anbietet,
wenn er vom Ausspruch der Vernunft als der \enquote{Einstimmung freier
Bürger}\footnote{\cite[][B 766]{Kant:KritikderreinenVernunft2003}, \cite[][III:
484.12]{Kant:GesammelteWerke1900ff.}.} spricht. Sie lässt sich mit Adjektiven
wie \enquote{republikanisch} oder \enquote{demokratisch}
umschreiben,\footnote{\cite[Vgl.][]{Hoeffe:EinerepublikanischeVernunft1996}.}
wenngleich dies die Gefahr von Missverständnissen fördern kann, wenn man die
Metaphern allzu wörtlich nimmt. Der Selbstdenker ist nicht derjenige, der sich
um die Urteile seiner Mitmenschen nicht schert, sondern derjenige, der sich
selbst als gleichberechtigtes Mitglied einer Kommunikationsgemeinschaft ansieht
und verhält. Das wiederum heißt nach der Maxime des Selbstdenkens, wir sollen in
unserem Denken und Erkennen nicht passiv, sondern aktiv sein. Statt Erkenntnisse
von außen aufzunehmen, sollen wir sie selbst generieren. In welchem Umfange wir
dies können, wird nun im \ref{chapter:endlichkeitmenschlichendenkens}. Kapitel
zu untersuchen sein.





\chapter{Die Endlichkeit des menschlichen
Verstandes}\label{chapter:endlichkeitmenschlichendenkens}
% \section{Endliche Vernunft}\label{section:EndlicheVernunft}
Es ist eine beiläufige Bemerkung aus dem Jahr 1763, in der \name[Immanuel]{Kant}
ein naheliegendes und oft angeführtes\footnote{So versteht bspw.
\authorfullcite{Kern:QuellendesWissens2006} die Endlichkeit unseres Verstandes:
\enquote{Wir sagen damit, daß es ein definierendes Merkmal desjenigen Wissens
ist, das wir zu verstehen versuchen, daß das Verhältnis zwischen dem Subjekt
dieses Wissens und dem Inhalt dieses Wissens auf eine Weise charakterisiert ist,
die Raum für die Möglichkeit des Irrtums hat. Dieses Wissen nennen wir endliches
Wissen im Kontrast zu unendlichem Wissen, welches genau diese Möglichkeit des
Irrtums nicht enthält} \parencite[][23]{Kern:QuellendesWissens2006}.} Merkmal
unserer Endlichkeit als solches erwähnt: unsere Fehlbarkeit. In der Schrift
\titel{Versuch, den Begriff der negativen Größen in die Weltweisheit
einzuführen} schreibt er:
\begin{quote}
 Der Mensch kann fehlen; der Grund dieser Fehlbarkeit liegt in der Endlichkeit
 seiner Natur, denn wenn ich den Begriff eines endlichen Geistes auflöse, so
 sehe ich, daß die Fehlbarkeit in demselben liege, das ist, einerlei sei mit
 demjenigen, was in dem Begriffe eines [endlichen; A.\,G.] Geistes enthalten
 ist.\footnote{\cite[][A 68]{Kant:VersuchdenBegriffdernegativenGroessenindieWeltweisheiteinzufuehren1977},
 \cite[][II: 202.23--27]{Kant:GesammelteWerke1900ff.}.}
\end{quote}
Die Fehlbarkeit endlicher Wesen können wir sowohl als epistemische Fehlbarkeit
verstehen -- wir können uns im Erkennen auch \emph{irren} --, als auch als
praktische Fehlbarkeit -- wir können unserer vernünftigen (insbesondere unserer
moralischen) Einsicht \emph{zuwider handeln}\footnote{Dies wird in Kapitel
\ref{chapter:AufklaerungundWissenschaft} als ein wesentlicher Gedanke aus der
Sicht der Forderung nach Mündigkeit herausgearbeitet werden (siehe
insbesondere das Ende von Kapitel \ref{section:MuendigeLebensfuehrung}). In
diesem (praktischen) Sinne ist unsere Endlichkeit ein Hindernis der Aufklärung oder -- vielleicht
treffender -- der Grund dafür, dass wir Aufklärung nötig haben und dass
Aufklärung an kein Ende gelangt.}.

Irrtumsanfälligkeit oder Fehlbarkeit ist dabei gewiss nicht die
einzige Möglichkeit, die Rede von unserer Endlichkeit in theoretischer
Hinsicht zu verstehen. Wir können die Endlichkeit des menschlichen
Verstandes als \textit{Begrenztheit} (in seinen praktischen und kognitiven
Fähigkeiten, räumlich, zeitlich etc.), als \textit{Abhängigkeit} (von Anderen,
von tradierten Wissensbeständen, von gesellschaftlichen Praxisformen, von der
Natur, der Welt, von Gott etc.) und auf manch andere Weise\footnote{Die
neuzeitliche Logik spricht beispielsweise von \emph{unendlichen Urteilen}
(\emph{iudicia infinita, indefinita, limitativa}), welche
oberflächlich betrachtet bejahende Urteile sind, deren Prädikate
jedoch Verneinungen enthalten und die daher
nicht angeben, was das Subjekt ist, sondern bloß kategorial bestimmen, was es nicht ist. Während dies
in der allgemeinen Logik vernachlässigt werden kann, interessiert es innerhalb
einer transzendentalen Logik; {\cite[vgl.][B
97\,f.,]{Kant:KritikderreinenVernunft2003} \cite[][III:
88.3--32]{Kant:GesammelteWerke1900ff.}; vgl. außerdem \cite[][A
160--162]{Kant:ImmanuelKantsLogik1977}, \cite[][IX:
103.23--104.24]{Kant:GesammelteWerke1900ff.}, und dessen Ursprung in
\cite[][\nopp 3065]{Kant:Reflexionen1900ff.}, \cite[][XVI:
639.2--5]{Kant:GesammelteWerke1900ff.}, als Anmerkung zu
\textcite[][\S~294]{Meier:AuszugausderVernunftlehre1752},
\cite[][XVI: 635.19--22, 636.14--22]{Kant:GesammelteWerke1900ff.}}.}
beschreiben. Ein Autor kann eine oder mehrere dieser Formen der Endlichkeit
beschreiben und er kann behaupten, dass sie untereinander in einem systematischen Zusammenhang
stehen oder nicht. \name[Immanuel]{Kant} behauptet -- wie ich zeigen werde --
einen solchen systematischen Zusammenhang.

\phantomsection\label{Absatz:DescarteszuEndlichkeitundVerhaeltnisvonWilleundVerstand}
Nun ist neben der Deutung der Endlichkeit unseres Verstandes als Fehlbarkeit
die Deutung als Begrenztheit vielleicht die naheliegendste.
\authorcite{Descartes:OeuvresdeDescartes1983} etwa interpretiert die
Endlichkeit unseres Verstandes in den \titel{Meditationes de prima philosophia}
dahingehend, dass wir nur einen Teil aller Wahrheiten kennen, während uns die
Einsicht in die Wahrheit (oder Falschheit) anderer Gedanken verborgen bleibe.
An Urteilen ist nach \authorcite{Descartes:OeuvresdeDescartes1983}
nun nicht nur der Verstand (\emph{intellectus}), sondern auch der Wille (\emph{voluntas})
beteiligt, insofern ein Urteil einen Akt der Zustimmung zu einer Behauptung
beinhaltet; und dieser Wille sei gerade nicht endlich. Der Wille sei unendlich, insofern es nichts gebe,
dem wir nicht unsere Zustimmung (oder Ablehnung) geben können. Auch Aussagen,
über deren Wahrheit unser Verstand wegen seiner Endlichkeit nicht entscheiden
könne, seien mögliche Gegenstände unserer Zustimmung. Doch wenn wir über Dinge,
die über die Reichweite unseres Verstandes hinausgehen, urteilen, dann unterliefen uns
möglicherweise Irrtümer.\footnote{\enquote{Unde ergo nascuntur mei errores?
Nempe ex hoc uno qu{\`o}d, c{\`u}m latius pateat voluntas qu{\`a}m intellectus, illam non intra
eosdem limites contineo, sed etiam ad illa qu{\ae} non intelligo extendo}
\parencite[][VII: 58.20--23]{Descartes:OeuvresdeDescartes1983}.} Unser
Verstand als solcher sei uns von Gott gegeben und könne als ein Vermögen,
welches uns von Gott, der kein Betrüger sein kann, gegeben wurde, auch nicht
täuschen. Nicht unsere Vermögen seien Ausgangspunkt von Irrtümern und
anderem Schlechten, sondern nur der Gebrauch, den wir endliche Wesen von
ihnen machen. Verstand und Wille seien jeweils ohne Einschränkung gut, denn sie
seien göttlichen Ursprungs; aber ein falscher, unvorsichtiger Gebrauch führe zu
Fehlern und Täuschungen -- wenn wir über Dinge urteilen, die wir nicht klar und
deutlich einsehen. Auf diese Weise kann
\authorcite{Descartes:OeuvresdeDescartes1983} erläutern, wie es zu Irrtümern
kommt, ohne behaupten zu müssen, eines unserer (von Gott verliehenen) Vermögen
sei fehleranfällig. Weder die \emph{voluntas} noch der \emph{intellectus} ist
für Fehler verantwortlich.\footnote{\enquote{Ex his autem percipio nec vim
volendi, quam a Deo habeo, per se spectatam, causam esse errorum meorum, est
enim amplissima, atque in suo genere perfecta; neque etiam vim intelligendi, nam
quidquid intelligo, c{\`u}m a Deo habeam ut intelligam, procul dubio recte
intelligo nec in eo fieri potest ut fallar} \parencite[][VII:
58.14--19]{Descartes:OeuvresdeDescartes1983}.}
Unser Verstand ist laut \authorcite{Descartes:OeuvresdeDescartes1983} also
endlich im Sinne von \emph{begrenzt}, denn er erkennt weniger Wahrheiten als ein unendlicher
Verstand; während unser Wille keinen solchen Begrenzungen unterliegt und somit
\enquote{unendlich} genannt werden kann. Unsere Irrtumsanfälligkeit ergibt sich
dann aber erst aus der Kombination eines endlichen Verstandes und eines
unendlichen Willens. Sie ist damit Folge, nicht Grund oder Wesen unserer
Endlichkeit.

Diese Konzeption hat \emph{prima facie} einige Ähnlichkeit mit
\name[Immanuel]{Kant}s Darstellung, denn auch dieser betont die Begrenztheit
unseres Verstandes und sagt, dass wir uns unvermeidlich in Irrtümer und
Scheinwahrheiten verlieren, wenn wir uns über diese Grenzen
hinauswagen.\footnote{Man vergleiche etwa das Bild von der Insel der Wahrheit
in \cite[][B 294\,f.,]{Kant:KritikderreinenVernunft2003}
\cite[][III: 202.12--203.3]{Kant:GesammelteWerke1900ff.}.}
Und die Unterscheidung zwischen einer Unbegrenzten \emph{voluntas} und einem
begrenzten \emph{intellectus} lässt sich leicht in Analogie bringen zur
kantianischen Unterscheidung von Denken und Erkennen. Doch
wie ich zeigen identifiziert \name[Immanuel]{Kant} die Endlichkeit des
Verstandes begrifflich gerade nicht mit
der Begrenztheit, sondern zunächst mit der Abhängigkeit des
Verstandes als eines Vermögens der Spontaneität von unserer Rezeptivität
(Sinnlichkeit). 
\phantomsection\label{Abschnitt:VertrauteBestimmungenergebensichausGrundbestimmungunsererEndlichkeit}
Die Begrenztheit hingegen ergibt sich bei ihm erst in Folge der Abhängigkeit.
Dasselbe gilt für die Diskursivität des Verstandes; auch sie folgt ebenso wie
seine Fehlbarkeit aus der Abhängigkeit, die sich als Abhängigkeit von der
Möglichkeit, von Gegenständen affiziert zu werden, erweist.

\section{Die vielen Bedeutungsfacetten des
Endlichkeitsbegriffs}\label{subsection:DievielenFacettendesEndlichkeitsbegriffs}
Bevor ich darauf eingehe, wie \name[Immanuel]{Kant} Diskursivität, Begrenztheit
und Fehlbarkeit unseres Verstandes als Resultat der
Abhängigkeit rekonstruiert, möchte ich auf die Notwendigkeit hinweisen, zwischen
verschiedenen Bedeutungen (oder Bedeutungsnuancen) des Unendlichkeits- wie des
Endlichkeitsbegriffs zu differenzieren. Diese Notwendigkeit
ist spätestens \authorcite{Spinoza:EthikingeometrischerOrdnungdargestellt2007}
bewusst gewesen und wird auch von \name[Immanuel]{Kant} beachtet. Aus ihr ergibt
sich eine Einschränkung der für das hier zu verhandelnde Thema einschlägigen
Literaturgrundlage: Es geht explizit nicht um Endlichkeit und Unendlichkeit
in einem quantitativen oder mathematischen Sinne (und darum sind auch
die Antithetik ebenso wie Überlegungen zur zeitlichen Begrenztheit
unseres Lebens hier nicht einschlägig).

\subsection{Quantitative und qualitative
Unendlichkeit}\label{subsection:QuantitativeundqualitativeUnendlichkeit}
Im zwölften Brief, in dem er ausführlich auf den Begriff des Unendlichen eingeht,
betont \authorcite{Spinoza:EthikingeometrischerOrdnungdargestellt2007}
drei wichtige Unterscheidung, deren Nichtbeachtung unweigerlich in
Konfusionen führe:
\begin{quote}
Die Frage nach dem Unendlichen erschien allen immer sehr schwierig, sogar
unlösbar, weil sie nicht unterschieden haben (i) zwischen (a\textsubscript{i})
dem, bei dem aus seiner Natur (\emph{sua natura}) oder kraft seiner Definition
folgt, dass es unendlich ist; und (b\textsubscript{i}) dem, was keine Grenzen
hat, aber nicht kraft seines Wesens (\emph{essentia}), sondern durch seine Ursachen (\emph{vi
suae causae}). (ii) Und auch weil sie nicht unterschieden haben zwischen
(a\textsubscript{ii}) dem, was unendlich heißt, weil es keine Grenzen hat; und
(b\textsubscript{ii}) dem, dessen Teile wir, auch wenn wir dessen Maximum und
Minimum haben, dennoch keiner Zahl angleichen (\emph{adaequare}) und durch sie
erläutern (\emph{explicare}) können. (iii) Schließlich weil sie nicht
unterschieden haben zwischen (a\textsubscript{iii}) dem, was wir ausschließlich
mit dem Verstand erkennen (\emph{intelligere}), uns aber nicht bildlich
vorstellen (\emph{imaginari}) können; und (b\textsubscript{iii}) dem, was wir
uns auch bildlich vorstellen können.\footnote{Brief an Ludwig \name[Ludwig]{Meyer} vom
20. April 1663: \enquote{Qu{\ae}stio de Infinito omnibus semper difficillima,
im{\`o} inextricabilis visa fuit, propterea qu{\`o}d non distinxerunt [(i)] inter
[(a\textsubscript{i})] id, quod su{\^a} natur{\^a}, sive vi su{\ae} definitionis
sequitur esse infinitum; {\&} [(b\textsubscript{i})] id, quod nullos fines
habet, non quidem vi su{\ae} essenti{\ae}; sed vi su{\ae} caus{\ae}. [(ii)] Ac etiam,
quia non distinxerunt inter [(a\textsubscript{ii})] id, quod infinitum dicitur,
quia nullos habet fines; {\&} [(b\textsubscript{ii})] id, cujus partes, quamvis
ejus maximum {\&} minimum habeamus, nullo tamen numero ad{\ae}quare, {\&}
explicare possumus. [(iii)] Denique quia non distinxerunt inter
[(a\textsubscript{iii})] id, quod solummod{\`o} intelligere, non ver{\`o}
imaginari; {\&} inter [(b\textsubscript{iii})] id, quod etiam imaginari
possumus} \parencite[][IV:
53.1--10]{Spinoza:SpinozaOpera1972}.}
\end{quote}
Hier beschreibt immer die durch \emph{a} gekennzeichnete Auffassung
das korrekte, hier relevante Verständnis von Unendlichkeit, während
die durch \emph{b} gekennzeichnete Auffassung ein Missverständnis
evoziert, wenn sie auf die Unendlichkeit Gottes angewendet oder auf
ihrer Grundlage die Endlichkeit des Menschen verstanden wird.
Die zweite Unterscheidung grenzt den Begriff der Unendlichkeit als
Unbegrenztheit (a\textsubscript{ii}) ab gegen die unendliche Mächtigkeit eines (nichtsdestotrotz
begrenzten) Kontinuums (b\textsubscript{ii}).\footnote{\authorcite{Spinoza:SpinozaOpera1972}
expliziert sie durch das Beispiel zweier Kreise unterschiedlicher Größe, die nicht
konzentrisch sind und von denen der kleinere zur Gänze innerhalb des größeren
liegt. Der Abstand zwischen beiden Kreisen hat ein Minimum und ein Maximum und
nimmt dazwischen unendlich viele verschiedene Werte an -- die Klasse der
Abstände bildet ein Kontinuum, das freilich durch Minimum und Maximum begrenzt
ist \mkbibparens{siehe \cite[][IV: 58.33--59.26]{Spinoza:SpinozaOpera1972}}.
\authorcite{Spinoza:SpinozaOpera1972} diskutiert dieses Beispiel, um auf
notwendige Unterscheidungen hinzuweisen, deren Nichtbeachtung in Antinomien
führt, wie sie \name[Immanuel]{Kant} als Antinomie der Teilung. Die
entsprechende Antinomie betrifft die \enquote{\ori{absolute Vollständigkeit} der
\ori{Teilung} eines gegebenen Ganzen in der Erscheinung} \mkbibparens{\cite[][B
443]{Kant:KritikderreinenVernunft2003}, \cite[][III:
287.7--10]{Kant:GesammelteWerke1900ff.}}. diskutiert. Wir dürfen die
Möglichkeit der beliebig feinen gedanklichen Unterteilung nicht mit der Existenz
entsprechender unendlich kleiner Dinge verwechseln, wenn wir uns bei der
geometrischen Beschreibung der Wirklichkeit nicht in Paradoxa wie dem von
Achilles und der Schildkröte verfangen wollen.}
Die Vorstellung eines Kontinuums hilft uns bei dem Verständnis unserer eigenen
Endlichkeit nicht weiter, weil dabei ein ganz anderer Unendlichkeitsbegriff
zugrunde gelegt wird. Es geht uns um das Unendliche, welches -- wie
\authorcite{Spinoza:SpinozaOpera1972} sagt -- keine äußeren Grenzen
hat.

Die erste Unterscheidung ist hier die wichtigste, weil sie im ersten
Teil (a\textsubscript{i}) die Definition des gesuchten Unendlichkeitsbegriffs enthält; sie reflektiert einen doppelten
Unendlichkeitsbegriff, den \authorcite{Spinoza:SpinozaOpera1972} im ersten Buch
der \titel{Ethica} als Unterscheidung von \enquote{absolut unendlich} (a\textsubscript{i}) und
\enquote{in seiner Art unendlich} (b\textsubscript{i}) entwickelt und der eine Unterscheidung
zwischen einer quantitativen und einer qualitativen Unendlichkeitsauffassung zur
Folge hat: Etwas sei in seiner Art endlich (\emph{res in suo genere finita}),
wenn es von etwas anderem von derselben Natur begrenzt werden
kann.\footnote{\enquote{Ea res dicitur in suo genere finita, quae alia ejusdem
naturae terminari potest} \parencite[][\nopp
1d2]{Spinoza:EthikingeometrischerOrdnungdargestellt2007}.}
Ein Körper sei beispielsweise endlich, weil wir uns einen größeren Körper denken
können. Und ein Gedanke werde von einem anderen Gedanken begrenzt,
weswegen er in seiner Art endlich sei.\footnote{Der Sinn der Behauptung, ein
Gedanke werde von einem anderem Gedanken \singlequote{begrenzt}, erschließt
sich sicherlich nicht unmittelbar. Denkbar ist, dass ein Gedanke endlich ist,
insofern er von anderen Gedanken unterschieden ist. Zu sagen, dass Katzen
Säugetiere sind, ist eben etwas anderen als zu sagen, dass Katzen Karnivore
sind. Zugleich ist ein Gedanke durch durch seine Negation begrenzt; dass Katzen
Säugetiere sind, heißt in diesem Sinne auch, dass sie keine Fische sind. Nur
durch solche Ausschlüsse sind Gedanken bestimmt gemäß dem Prinzip \emph{omnis
determinatio est negatio}, welches sich so bei
\authorcite{Spinoza:SpinozaOpera1972} jedoch nicht findet
\parencite[vgl.][\pno~50\,f.]{Schnepf:MetaphysikimerstenTeilderEthikSpinozas1996}.}
Ein Körper, der wegen seiner Größe von keinem anderen Körper begrenzt werden
kann, wäre in seiner Art unendlich. Und ebenso wäre ein Gedanke, den kein
anderer Gedanke begrenzen kann, \emph{in suo genere} unendlich.

Nun könne etwas auch in seiner Art unendlich sein, ohne doch schlechthin
Unendlichkeit beanspruchen zu können. Seine Unbegrenztheit dann liege nicht in seiner
Natur, sondern sei quasi von außen bedingt, sie sei Folge einer äußeren Ursache.
Hingegen sei Gott \emph{absolute infinitus}, insofern er aus unendlich vielen
Attributen bestehe, von denen jedes ein \singlequote{ewiges} und unendliches
Wesen (\emph{essentia})
ausdrückt.\footnote{\cite[Vgl.][\nopp
1d6]{Spinoza:EthikingeometrischerOrdnungdargestellt2007}.} Zu seinem
Wesen gehöre alles, was keine Verneinung in sich schließt, sondern ein
Wesen oder eine \singlequote{Essenz} ausdrückt (\emph{essentiam exprimit}). Die
absolute Unendlichkeit ist eine solche, bei der die Unendlichkeit aus dem Wesen
(und damit aus der korrekten Definition) der Sache selbst folgt. Gott ist etwa
unendlich, insofern -- gemäß \singlequote{ontologischem}
Gottesbeweis\footnote{Von einem \singlequote{ontologischen Gottesbeweis} wird
erst im Anschluss an \name[Immanuel]{Kant}s Bezeichnung
\mkbibparens{\cite[vgl.][B 619]{Kant:KritikderreinenVernunft2003},
\cite[][III: 396.29--31]{Kant:GesammelteWerke1900ff.}} gesprochen.
Ich gehe auf Fragen rund um diese Bezeichnung nicht ein, da es hier nicht weiter
relevant sein wird.} -- aus dem Begriff Gottes bereits dessen Existenz folgt.
Aus der absoluten Unendlichkeit folgt die Unendlichkeit \emph{in suo
  genere}: Wenn etwas absolut unendlich ist, dann kann es nicht in seiner
Art endlich sein, denn was über alle Realität verfügt, kann von nichts
begrenzt werden.\footnote{Außerdem folge, dass die Substanz, als das, was in
sich selbst ist und durch sich selbst verstanden
wird \mkbibparens{\enquote{\emph{quod in se est et per se concipitur}},
\cite[][\nopp 1d3]{Spinoza:EthikingeometrischerOrdnungdargestellt2007}}
und die bereits ihrer Natur nach existiert
\parencite[Vgl.][1p7]{Spinoza:EthikingeometrischerOrdnungdargestellt2007},
unendlich ist \parencite[vgl.][\nopp
1p8]{Spinoza:EthikingeometrischerOrdnungdargestellt2007}, während alle endlichen
Dinge in der einen unendlichen Substanz sind
\parencite[vgl.][1p15]{Spinoza:EthikingeometrischerOrdnungdargestellt2007}.}


Der Unterschied zwischen einer quantitativen und einer qualitativen
Unendlichkeitsauffassung -- und wie er aus dem Gesagten resultiert --
verdeutlicht sich an den Begriffen der Dauer und der Ewigkeit, die zugleich auf
\authorcite{Spinoza:SpinozaOpera1972}s dritte Unterscheidung verweisen. Es gebe
ein Unendliches, das wir uns vorstellen können (b\textsubscript{iii}); und es gebe ein Unendliches, das
wir zwar erkennen können, das aber nicht Gegenstand unserer Vorstellung sein
kann (a\textsubscript{iii}). Maß, Zeit und Zahl seien Modi des Denkens und auch des Vorstellens.
Eine Dauer, die keine Grenzen, kein Davor und Danach kennt,  sei denkbar und
auch vorstellbar, nicht aber die Ewigkeit, die wir zwar denken, uns aber nicht
vorstellen können. Und deshalb verfälschen wir jede Einsicht in die
Unendlichkeit Gottes, wenn wir ihn zum Gegenstand unseres Vorstellens zu machen
versuchen. Um die Substanz oder die Ewigkeit fassen zu können, müssen wir uns
auf den Verstand, nicht aber die Vorstellung berufen, denn etwas Unendliches vorzustellen heißt, es
nur als \singlequote{in seiner Art} unendlich zu verstehen, nicht aber als
\emph{absolute infinitus}.

Der zentrale Ausgangspunkt, der für ein Verständnis des absolut Unendlichen
unentbehrlich ist, ist also der, dass es sich hier nicht um einen mathematischen
Begriff von Endlichkeit respective Unendlichkeit handelt, sondern um Begriffe,
die wir \singlequote{\emph{metaphysisch}} nennen könnten. Etwas ist in diesem
Sinne nicht dann unendlich, wenn es eine Eigenschaft von entsprechender
Quantität hat, sondern genau dann, wenn seine Existenz bereits
aus seinem Wesen folgt. Freilich folgt die mathematische Unendlichkeit in
gewisser Hinsicht aus der metaphysischen Unendlichkeit: Was \emph{ewig} ist,
weil es als nicht-existierend gar nicht gedacht werden könnte, das existiert auch
über eine Dauer von unendlicher Ausdehnung. Aber darin geht der Begriff seiner
Unendlichkeit eben nicht auf.

Dies lässt sich auch auf den unendlichen Verstand übertragen:
Dieser kennt unendlich viele Wahrheiten, aber darin besteht nicht seine
Unendlichkeit. Denn die allgemeine metaphysische Position wirkt sich auch auf
den unendlichen Verstand (\emph{intellectus infinitus})\footnote{Dieser Ausdruck
findet sich erstmals in einem Folgesatz zur
\emph{propositio} 16 des ersten Teils der \titel{Ethica}
\parencite[siehe][1p16c1]{Spinoza:EthikingeometrischerOrdnungdargestellt2007}.}
aus, von dem der endliche menschliche Verstand ein Teil sei\footnote{\enquote{Hinc
sequitur mentem humanam partem esse infiniti intellectus Dei}
\parencite[][2p11c]{Spinoza:EthikingeometrischerOrdnungdargestellt2007}. Nun ist
der menschliche Geist freilich endlich, aber es gehört dennoch zu seiner Natur,
Dinge \emph{sub specie aeternitatis} zu erkennen
\mkbibparens{\cite[Vgl.][2p44c2]{Spinoza:EthikingeometrischerOrdnungdargestellt2007}}}.
Auch der menschliche Verstand steht nicht primär in einem quantitativen
Verhältnis zum unendlichen Verstand Gottes. Und entsprechend lässt sich die Unendlichkeit des
göttlichen Verstandes nicht mathematisch verstehen, wenngleich sie quantitative
Unendlichkeiten zur Folge hat. Der unendliche Verstand erkennt nicht mehr oder
mit größerer Gewissheit, sondern auf eine andere Art. (Er erkennt
ausschließlich auf die Art der \emph{scientia intuitiva} und nicht wie wir auch
in den Erkenntnisgattungen des Meinens und des logischen
Schlussfolgerns.\footnote{Siehe dazu Kapitel
\ref{subsection:IntuitiverVerstandunddasSynthetischAllgemeine}.}) Entsprechend
ist es Folge unserer Endlichkeit, dass unsere Existenz und unsere Kenntnis
wahrer Aussagen auch quantitativ begrenzt ist. Aber dies ist nicht Wesen unserer
Endlichkeit im metaphysischen Sinne, sondern eine ihrer Auswirkungen.

Es ist nicht nötig vorauszusetzen, dass \name[Immanuel]{Kant}
\authorcite{Spinoza:SpinozaOpera1972}s Überlegungen im Original
rezipiert hätte. Es handelt sich um Überlegungen, die sich auch an anderen
Stellen finden, mit denen \name[Immanuel]{Kant} ganz sicher vertraut war, etwa
in \authorcite{Baumgarten:Metaphysica---Metaphysik2011}s
\titel{Metaphysica}.\footnote{Nach \authorcite{Engelhard:DasEinfacheunddieMaterie2005}
greift \name[Immanuel]{Kant} die Unterscheidung von \emph{infinitum} und
\emph{indefinitum} vielfach auf
\parencite[vgl.][\pno~354\,f.]{Engelhard:DasEinfacheunddieMaterie2005}. In
vielen Punkten ist zu vermuten, dass \name[Immanuel]{Kant} die Position
\authorcite{Spinoza:SpinozaOpera1972}s nicht aus dessen Schriften selbst kannte,
aber doch mit Darstellungen und Anknüpfungen vertraut war.} Dieser bezeichnet
das mathematische Unendliche -- welches er auch \enquote{unendlichscheinend}
nennt -- als \emph{indefinitum} (oder \emph{infinitum imaginarium}) und
unterscheidet es von einem \emph{infinitum} (dem \enquote{uneingeschränkten}),
dessen \singlequote{Realitätsgrad} (\enquote{gradus realitatis}) keine Grenzen
habe.\footnote{Dem Kontext nach geht es \authorcite{Baumgarten:Metaphysica---Metaphysik2011}
ausschließlich um den Begriff des Unbeschränkten -- die Unterscheidung zwischen
\emph{infinitum} und \emph{indefinitum} scheint lediglich das Missverständnis
verhüten zu sollen, das aus einer Verwechslung resultierte, wenn die Bestimmung
des \emph{indefinitum} (dass wir die Grenzen nicht \emph{kennen} können oder
wollen) als ausreichend für den Begriff des \emph{infinitum} angesehen wird. Er
möchte sicherstellen, dass nur das als (wahrhaft) unendlich bezeichnet wird, was
tatsächlich keinerlei Beschränkungen hat. In einem uneigentlichen Sinne nämlich
können wir dasjenige als unendlich groß bezeichnet, von dem wir keine Größe
angeben können, obwohl es möglicherweise eine uns unbekannte Grenze
gibt.} Ein \emph{indefinitum} hingegen sei etwas, dessen Grenzen wir
nicht kennen können oder nicht kennen
wollen, unabhängig davon, ob es \emph{de facto} eine Grenze hat. Wir könnten es
auch als \enquote{unbestimmt} bezeichnen. Nur das Uneingeschränkte oder
\emph{infinitum} sei wirklich unendlich; und dieses Unendliche sei als
dasjenige, dessen \singlequote{Realitätsgrad} (\emph{gradus realitatis}) keine
Beschränkung habe, das notwendig Seiende (\emph{ens
necessarium}), welches keinen Veränderungen unterliege und solchen auch gar
nicht unterliegen
könne.\footnote{\cite[Vgl.][\S\S~253--258]{Baumgarten:Metaphysica---Metaphysik2011},
\cite[][XVII: 82.2--21]{Kant:GesammelteWerke1900ff.}.} Unendlichkeit im Sinne
des \emph{infinitum} sei nichts anderes als Realität,\footnote{\cite[Vgl.][\S~261]{Baumgarten:Metaphysica---Metaphysik2011}, \cite[][XVII: 82.34--35]{Kant:GesammelteWerke1900ff.}.} während das Endliche über Realitäten, aber auch über Negationen, also \emph{modi}
verfüge.\footnote{\cite[Vgl.][\S~263]{Baumgarten:Metaphysica---Metaphysik2011},
\cite[][XVII: 83.5--8]{Kant:GesammelteWerke1900ff.}} Gerade in dieser Anbindung
des \emph{infinitum} an die \emph{realitas} ähnelt
\authorcite{Baumgarten:Metaphysica---Metaphysik2011}s Differenzierung
derjenigen, die sich paradigmatisch und auf höchstem Reflexionsniveau bei \authorcite{Spinoza:SpinozaOpera1972}
findet.


Nach \name[Immanuel]{Kant}s Auskunft ist es nun zwar kein Fehler, von der
Unendlichkeit Gottes und unserer Endlichkeit zu sprechen, da man die Freiheit besitze, beide Ausdrücke
entsprechend auszulegen. Man müsse sich jedoch klar von einem mathematischen
Verständnis abgrenzen. Deshalb bevorzugt er eigenen Angaben aus dem Jahre 1763
zufolge den Ausdruck
\phantomsection\label{Allgenugsamkeit}\enquote{Allgenugsamkeit} gegenüber dem
der Unendlichkeit für die Charakterisierung göttlicher Vollkommenheit,
weil es ein Fehler sei, das Verhältnis des göttlichen zu unserem
menschlichen Denken als quantitativen Unterschied begreifen zu wollen.
Denn dies setzte voraus, dass beide ihrer Beschaffenheit nach grundsätzlich
gleich und nur von verschiedener Stärke seien.\footnote{\cite[Vgl.][A
186\,f.,]{Kant:DereinzigmoeglicheBeweisgrundvomDaseinGottes1977} \cite[][II:
154.4--19]{Kant:GesammelteWerke1900ff.}.}
Verstehen wir den Unterschied zwischen Gott und uns als lediglich graduell --
und sei die Differenz zahlenmäßig noch so groß -- dann befinden wir uns zwar auf
vertrautem Terrain, weil wir uns einen besseren Verstand leicht vorstellen
können (insofern hat diese Konzeption den Vorteil guter Verständlichkeit). Aber
wir bilden damit doch nur eine weitere anthropomorphe Gottesvorstellung, die dem
Begriff des wirklich Unendlichen nicht gerecht wird. Bei der Unterscheidung von
menschlichem und göttlichem Denken interessiert gerade ein qualitativ
Unendliches, nicht das mathematische oder quantitative Unendliche.\footnote{In
\titel{Was heißt: sich im Denken orientieren?} spricht \name[Immanuel]{Kant}
hingegen Gott \enquote{\ori{Unendlichkeit} der Größe nach
zur Unterscheidung von allem Geschöpfe} zu \mkbibparens{\cite[][A
322]{Kant:Washeisst:SichimDenkenorientieren?1977}, \cite[][VIII:
142.30]{Kant:GesammelteWerke1900ff.}}. Dort wird jedoch nicht der göttliche
Verstand beschrieben, sondern begründet, warum Gott in unserer Wahrnehmung nicht
als solcher identifizierbar sei.}

Die genannte Ablehnung eines mathematischen Verständnisses von Unendlichkeit bei
der Explikation göttlicher Eigenschaften stammt aus dem Jahr 1763; und
sicherlich ist denkbar, dass \name[Immanuel]{Kant} seine Ansicht später revidiert und doch
einen quantitativen Unterschied herausstellen möchte. Dies wird bezüglich des
Begriffs des intuitiven und diskursiven Verstandes in \S~77 der \titel{Kritik
der Urteilskraft} von
\authorfullcite{McLaughlin:KantsKritikderteleologischenUrteilskraft1989}
behauptet.\footnote{\enquote{Ein unendlicher Verstand könnte durch Kenntnis
sämtlicher empirischer Gesetze den Begriff eines jeden Besonderen durchgehend
bestimmen, so daß nichts Zufälliges übrig bleibt. Daß wir dies nicht können,
sondern auf die Urteilskraft angewiesen sind, hängt von der Endlichkeit
(Schranken) unseres Verstandes, nicht von seiner Qualität (Beschaffenheit) ab}
\parencite[][148]{McLaughlin:KantsKritikderteleologischenUrteilskraft1989}.
\name[Immanuel]{Kant}s Behauptung ist aber gerade, dass die empirischen Gesetze
als solche aus Sicht des Verstandes zufällig sind (sonst bräuchten wir keine
Erfahrung, um sie zu erkennen). Ein Gesetz ist eben empirisch, wenn wir es nur
historisch wissen \emph{können} (siehe Kapitel
\ref{section:MuendigkeitundPhilosophie}). Auch dann, wenn wir \emph{alle}
empirischen Gesetze kennten, wären diese in dieser Hinsicht zufällig und somit
die Natur zweckmäßig (siehe Kapitel \ref{subsection:MetaphysikundAutonomie}).
Siehe dazu auch weiter unten Kapitel
\ref{subsection:IntuitiverVerstandunddasSynthetischAllgemeine}.} Dennoch gilt
mindestens \emph{prima facie} die Interpretationsmaxime, dass die Endlichkeit
unseres Denkens bei \name[Immanuel]{Kant} nicht als quantitative Einschränkung
aufzufassen ist. Dafür sprechen die frühe Auskunft \name[Immanuel]{Kant}s von
1763 und die Tatsache, dass \name[Immanuel]{Kant} stets von der besonderen
\emph{Beschaffenheit} unseres Verstandes spricht und auch in der \titel{Kritik
der Urteilskraft} noch davon spricht, dass \enquote{keine menschliche Vernunft
(auch keine endliche, die der Qualität nach der unsrigen ähnlich wäre, sie aber
dem Grade nach noch so sehr überstiege) die Erzeugung auch nur eines Gräschens
aus bloß mechanischen Ursachen zu verstehen
hoffen}\footnote{\cite[][\S~77]{Kant:KritikderUrteilskraft2009}; \cite[][V:
409.33--37]{Kant:GesammelteWerke1900ff.}. Siehe auch
\cite[][\S~75]{Kant:KritikderUrteilskraft2009},
\cite[][V: 400.13--21]{Kant:GesammelteWerke1900ff.}.} könne.\footnote{Diese
Sichtweise teilt auch \authorfullcite{Allison:KantsTranscendentalIdealism2004}:
\enquote{[A]s discursive, human knowledge differs in kind, not merely in degree,
from that which might be had by a putative pure understanding}
\parencite[][17]{Allison:KantsTranscendentalIdealism2004}.} Ich gehe also davon
aus, dass \name[Immanuel]{Kant} in diesem Zusammenhang einen qualitativen
Unendlichkeitsbegriff vor Augen hat. Entsprechend hilft es auch nicht, die
Antithetik der \titel{Kritik der reinen Vernunft} zu konsultieren, denn die dort
thematisierte Totalität der \enquote{Reihe in der Synthesis des
Mannigfaltigen}\footnote{\cite[][B 442]{Kant:KritikderreinenVernunft2003},
\cite[][III: 286.37]{Kant:GesammelteWerke1900ff.}.} diskutiert nur mathematische
Unendlichkeitsvorstellungen, die vielleicht in aktual und potentiell Unendliches
differenziert werden können. Doch beides entspricht nicht dem Begriff des
Unendlichen, der bei der Explikation des Unterschiedes zwischen unserem
endlichen und einem göttlichen unendlichen Verstand hilfreich sein
könnte. Und der Begriff \enquote{Allgenugsamkeit} lässt vermuten, dass
\name[Immanuel]{Kant} bereits 1763 an die Unabhängigkeit als entscheidendes
Merkmal eines unendlichen Verstandes gedacht haben mag, ebenso wie er später die
Abhängigkeit unseres Verstandes als Charakteristikum unserer Endlichkeit
herausstellt.\footnote{Siehe dazu \cite[][B
72]{Kant:KritikderreinenVernunft2003}, \cite[][III:
72.29--73.4]{Kant:GesammelteWerke1900ff.}. Siehe auch
\cite[][\S~10]{Kant:Demundisensibilisatqueintelligibilisformaetprincipiis1968},
\cite[][II: 396.19--397.4]{Kant:GesammelteWerke1900ff.}.}

\subsection{Der Ursprung unserer Endlichkeitsvorstellung}
Es stellt sich nun aus methodischen Gründen die Frage, welcher Begriff der
ursprüngliche ist: der des endlichen oder der des unendlichen Verstandes. Mit
diesem Begriff müsste dann die Analyse begonnen werden. Autoren wie
\authorcite{Descartes:OeuvresdeDescartes1983}, \authorcite{Hobbes:Leviathan1962}
und \authorcite{Locke:TheWorksofJohnLocke1963} wenden sich dieser Frage
explizit zu und auch bei \name[Immanuel]{Kant} findet sich eine Antwort, die es
zu berücksichtigen gilt. Woher also haben wir unsere Begriffe von endlichem und unendlichem Verstand?
Können wir einen Begriff eines göttlichen Verstandes ausgehend von einem Begriff unseres endlichen
Denkens bilden? Oder liegt er dem Begriff unseres eigenen Verstandes sogar
zugrunde?


\authorcite{Descartes:OeuvresdeDescartes1983} sagt, der menschliche endliche Verstand sei
lediglich ein schwächeres Abbild des unendlichen göttlichen Verstandes und wer versteht,
was der Ausdruck \enquote{Verstand} bedeutet, wer also eine Vorstellung von
unserem endlichen Verstand hat, der müsse bereits eine Vorstellung von einem
nicht-endlichen, göttlichen Verstand haben. Denn die Vorstellung eines endlichen
Verstandes sei lediglich die Idee eines perfekten Verstandes verbunden mit einer
Privation: So wie wir die Idee einer defekten Maschine nur bilden können, wenn
wir zunächst eine Vorstellung von einer nicht-defekten Maschine besitzen und
diese mit der Vorstellung eines Mangels verbinden, ebenso könnten wir nur
dadurch eine Vorstellung von unserem Denken erwerben, dass wir die Idee des
Verstandes mit der Vorstellung eines Mangels verbinden. So lautet das Argument
der dritten Meditation, mit dem nachgewiesen werden soll, dass wir über eine
Idee des göttlichen Verstandes verfügen.\footnote{\enquote{Nec putare debeo me
non percipere infinitum per veram ideam, sed tant{\`u}m per negationem finiti,
ut percipio quietem {\&} tenebras per negationem mot{\^u}s {\&} lucis; nam
contr{\`a} manifeste intelligo plus realitatis esse in substanti{\^a}
infinit{\^a} qu{\`a}m in finit{\^a}, ac proinde priorem quodammodo in me esse
perceptionem infiniti qu{\`a}m finiti, hoc est Dei qu{\`a}m me{\^i} ipsius}
\parencite[][VII: 45.23--29]{Descartes:OeuvresdeDescartes1983}. Diese
Vorstellung eines göttlichen Verstandes können wir nicht aufbauend auf anderen
Ideen erworben haben, sie sei uns somit angeboren \parencite[vgl.][VII:
51.6--14]{Descartes:OeuvresdeDescartes1983}.}

Bei \authorfullcite{Locke:TheWorksofJohnLocke1963} findet sich die klassische
empiristische Gegenposition zu \authorcite{Descartes:OeuvresdeDescartes1983}'
Annahme einer ursprünglichen Gottesvorstellung.\footnote{Schon
\authorcite{Hobbes:Leviathan1962} behauptet gegen
\authorcite{Descartes:OeuvresdeDescartes1983}, dass dessen Behauptung, wir
verfügten über eine solche angeborene Idee Gottes, die als Ursprung unserer Idee
von unserem endlichen Verstand anzusehen ist, ungereimt sei. Vielmehr besäßen
wir \emph{keine} Vorstellung von einem göttlichen Verstand, da dieser als
unendlicher Verstand gar nicht geeignet sei, Gegenstand eines mentalen
\singlequote{Bildes} zu sein \parencite[vgl.][VII:
183.4--19]{Descartes:OeuvresdeDescartes1983}.
\authorcite{Descartes:OeuvresdeDescartes1983} erwidert wiederum, dass
\authorcite{Hobbes:Leviathan1962} über eine falsche Vorstellung darüber verfüge,
was eine Vorstellung (\enquote{\emph{idea}}) ist, und deshalb irrtümlich
annehme, es könne keine solche Vorstellung von einem göttlichen Verstand
geben \parencite[Vgl.][VII: 183.22--25]{Descartes:OeuvresdeDescartes1983}.}
Grundlage ist die Annahme, unser Geist gleiche einer \emph{tabula rasa}, der erst durch die sinnliche Wahrnehmung
-- äußere \emph{sensation} und innere \emph{reflection} -- Inhalte gegeben
werden. Unsere Vorstellungen erhalten wir dann entweder unmittelbar durch
entsprechende Wahrnehmungen oder mittelbar durch Modifikation solcher
Wahrnehmungen. Da wir keine unmittelbare Bekanntschaft mit einem unendlichen
Denken machen können, können wir unseren Begriff eines solchen Denkens nur
mittelbar bilden, indem wir von einem Denken ausgehen, das uns bekannt ist --
dem endlichen --, und diese Idee mit einer anderen Idee kombinieren. Nach
\authorcite{Locke:TheWorksofJohnLocke1963} bilden wir die Vorstellung eines
unendlichen Verstandes ausgehend von der durch Reflexion gewonnenen Vorstellung
von den Eigenschaften und Tätigkeiten unseres eigenen endlichen Verstandes durch
Steigerung mittels der Idee des Unendlichen.\footnote{\cite[Vgl.][II:
31]{Locke:TheWorksofJohnLocke1963}: \enquote{[H]aving, from what we experiment
in ourselves, got the ideas of existence and duration; of knowledge and power;
of pleasure and happiness; and of serveral other qualities and powers, which it
is better to have than to be without: when we would frame an idea the most
suitable we can to the Supreme Being, we enlarge every one of these with our
idea of infinity; and so putting them together, make our complex idea of God.}}


Verfügen wir also zunächst über eine Vorstellung von einem endlichen Verstand
und bilden auf dieser Grundlage die Vorstellung von einem unendlichen
Erkenntnisvermögen, wie dies die Empiristen
\authorcite{Hobbes:Leviathan1962} und \authorcite{Locke:TheWorksofJohnLocke1963}
behaupten? Oder ist der Begriff des unendlichen Verstandes ursprünglich gegeben
und der des endlichen Verstandes davon abgeleitet, wie
\authorcite{Descartes:OeuvresdeDescartes1983} nachweisen zu können glaubt?
Welche Annahme muss einer Interpretation \name[Immanuel]{Kant}s zugrunde gelegt
werden? \name[Immanuel]{Kant} kommentiert diese Auseinandersetzung nicht
explizit. Um seine Vorgehensweise zu verstehen, sollten wir jedoch
eine dritte Möglichkeit erwägen, die sich im Laufe der Explikation als
korrekt erweisen wird: So verfügen wir zunächst
über den Begriff des Verstandes, ohne zwischen einem endlichen und
einem unendlichen Verstand zu differenzieren. Ein Verstand wiederum ist entweder
diskursiv oder intuitiv, je nachdem, ob er lediglich denken kann oder anschaut.
Es handelt sich also weder bei der Vorstellung eines
intuitiven Verstandes, noch bei der Vorstellung einer intellektuellen Anschauung um Derivate ausgehend
von unseren Vorstellungen von einem diskursiven Verstand oder einer sinnlichen
Anschauung. Und wir benötigen auch keinen Begriff nicht-endlicher
Erkenntnisvermögen als Ursprung der Vorstellung unserer endlichen
Vermögen.\footnote{Es ist sicherlich korrekt, dass ein Verständnis einer
Fähigkeit und einer Tätigkeit dem Verständnis der fehlerhaften Ausübung vorausgeht.
Der Begriff \enquote{sich verrechnen} ist derivativ zu dem Begriff
\enquote{rechnen}: Nur wer (richtig) rechnen kann, kann
sich auch verrechnen; und nur wer weiß, was \enquote{rechnen} heißt,
kann auch wissen, was heißt, sich zu verrechnen. Wir können den Ausdruck \enquote{sich verrechnen} nur
verstehen als Beschreibung der fehlerhaften Ausführung der Operation des
Rechnens. Um es mit \authorfullcite{Ryle:TheConceptofMind2002} zu sagen: Dass jemand rechnen kann, beschreibt eine Kompetenz; dass er
sich auch verrechnen kann, beschreibt eine Anfälligkeit.
Kompetenzen gehen Anfälligkeiten logisch voraus: Um eine Anfälligkeit zu haben,
muss man zunächst über eine entsprechende Kompetenz verfügen. Wer nie rechnen
lernte, kann sich auch nicht verrechnen \parencite[vgl. hierzu][\pno~60, 130\,f.]{Ryle:TheConceptofMind2002}.
Aber daraus folgt gerade nicht, dass wir einen Begriff von Unfehlbarkeit
bräuchten, um unsere fehlbaren Vermögen zu verstehen.
\authorfullcite{Ryle:TheConceptofMind2002} behauptet, es gehöre zum
Verfügen über eine Fähigkeit dazu, auch für Fehler anfällig zu sein
\parencite[vgl.][130]{Ryle:TheConceptofMind2002}. Sollte das stimmen,
dann ist ein entsprechendes Vermögen überhaupt nur als endliches denkbar.}

Nun schreibt \name[Immanuel]{Kant} des öfteren, wir könnten die Möglichkeit
eines \singlequote{anderen} Verstandes oder einer \singlequote{anderen}
Anschauung nicht einsehen.\footnote{Vgl. \cite[][B
213]{Kant:KritikderreinenVernunft2003}, \cite[][III:
212.20--21]{Kant:GesammelteWerke1900ff.}} Ist es vielleicht nach
\name[Immanuel]{Kant} der Fall, dass wir
über gar keinen Begriff und keine Vorstellung eines unendlichen Verstandes
verfügen? Haben wir vielleicht nur eine Vorstellung von unserem eigenen,
endlichen Verstand. Es wäre merkwürdig, wenn \name[Immanuel]{Kant} sagen wollte,
dass wir über keinen Begriff eines nicht-endlichen Verstandes verfügen, denn
dann verstünden wir auch das nicht, was er mit Hilfe dieses Begriffs zu
erläutern versucht.  Denn nur durch die Konzeption eines
\singlequote{anderen} Verstandes und einer \singlequote{anderen} Anschauung sei
es möglich zu erkennen, dass wir die Dinge lediglich so erkennen, wie sie uns
erscheinen, nicht so, wie sie an sich sind.\footnote{\authorfullcite{Allison:KantsTranscendentalIdealism2004} betont,
dass der transzendentale Idealismus als Resultat der Analyse unserer
Endlichkeit -- \authorcite{Allison:KantsTranscendentalIdealism2004} schreibt
\enquote{discursivity} -- zu interpretieren sei
\parencite[vgl.][1--73]{Allison:KantsTranscendentalIdealism2004}.} Was wir nicht einsehen, ist vielmehr dieses: Wir
können die Wirkungsweise eines nicht-endlichen Verstandes
nicht näher erläutern. Die Wirkungsweise unseres eigenen Verstandes können wir analysieren,
die Wirkungsweise eines gänzlich anderen Verstandes ist eine \emph{terra
incognita}, über die wir nichts weiter sagen können, als diesen allgemeinen Begriff anzugeben,
dass er nicht denkt, sondern anschaut. Dieser ist konsistent, stellt
aber keine reale Möglichkeit dar, weil wir nicht wissen, wie er realisierbar
sein könnte.\footnote{Ganz ähnlich sagt
\authorcite{Descartes:OeuvresdeDescartes1983}, ein unendlicher
Verstand könne von uns nicht begriffen werden, weil wir selbst endlich sind.
\enquote{[E]st {\punkt} de ratione infiniti, ut a me, qui sum finitus, non
comprehendatur} \mkbibparens{\cite[][VII:
46.21--23]{Descartes:OeuvresdeDescartes1983}}.
Dennoch verfügten wir über eine Idee eines solchen Verstandes.
Zu beachten ist, dass \authorcite{Descartes:OeuvresdeDescartes1983} für das, was wir
\emph{nicht} können, den Ausdruck \enquote{comprehendere}
(\singlequote{begreifen}) verwendet, für das, was wir können, hingegen den
Ausdruck \enquote{intendere} (verstehen) bzw. \enquote{percipere}. Siehe dazu
auch die Stufen des Erkennens nach \name[Immanuel]{Kant} in \cite[][A
96\,f.,]{Kant:ImmanuelKantsLogik1977} \cite[][IX:
64.33--65.24]{Kant:GesammelteWerke1900ff.}. Siehe zur logischen im Gegensatz
zur realen Möglichkeit auch folgende Anmerkung zur Vorrede der zweiten Auflage
der \titel{Kritik der reinen Vernunft}: \cite[][B
xxvi]{Kant:KritikderreinenVernunft2003},
\cite[][III: 17.29--38]{Kant:GesammelteWerke1900ff.}.}

Ich gehe also davon aus, dass wir (mindestens) einen Begriff von einem
\singlequote{anderen} Verstand haben und dass dieser sich von unserem Verstand
nicht nur graduell, sondern qualitativ unterscheidet. Der Begriff dieses
Verstandes wird wie der Begriff unseres eigenen Verstandes im Ausgang des
Begriffs eines Verstandes überhaupt gebildet, unter den unserer ebenso
wie der \singlequote{andere} Verstand fallen, indem ihm bestimmte
Eigenschaft jeweils zu- oder abgesprochen werden. (Dasselbe wird auch
für den Begriff einer
sinnlichen \emph{respective} einer intellektuellen Anschauung gelten.) Im
weitere Verlauf wird es darum gehen, zunächst den Begriff des Verstandes unabhängig
davon, ob es sich um einen endlichen oder unendlichen Verstand handelt, zu
explizieren und anschließend herauszuarbeiten, was es ist, das dem
endlichen und unendlichen Verstand jeweils zu- oder abgesprochen wird.


\subsection{Eine mögliche
Vieldeutigkeit}\label{subsection:EineMoeglicheVieldeutigkeit}
Während die Wörter \enquote{endlich} und \enquote{unendlich} bei
\name[Immanuel]{Kant} eher selten zur Charakterisierung von Erkenntnisvermögen
herangezogen werden, finden sich sehr unterschiedliche Bezeichnungen, deren
systematische Zusammenhänge sich nicht von selbst verstehen. An vielen Stellen
der Schriften \name[Immanuel]{Kant}s kommt die Beschreibung eines Verstandes
vor, der nicht wie der unsrige nur denken kann und zur Gewährleistung des
Gehalts seines Denkens (zur Gewinnung von Anschauungen) auf die Sinne
rekurrieren muss, sondern der selbst anschaut.\footnote{\enquote{Ein Verstand,
in welchem durch das Selbstbewußtsein zugleich alles Mannigfaltige gegeben
würde, würde \ori{anschauen}; der unsere kann nur \ori{denken} und muß in den
Sinnen die Anschauung suchen} \mkbibparens{\cite[][B
135]{Kant:KritikderreinenVernunft2003}, \cite[][III:
110.26--29]{Kant:GesammelteWerke1900ff.}}. Ohne das Mannigfaltige der Anschauung
wäre unser Denken, wie \name[Immanuel]{Kant} durchgängig betont, leer.
\cite[Siehe z.\,B. den bekannten Ausspruch in][B
75]{Kant:KritikderreinenVernunft2003}, \cite[][III: 75.14--15]{Kant:GesammelteWerke1900ff.}:
\enquote{Gedanken ohne Inhalt sind leer, Anschauungen ohne Begriffe sind
blind.} An anderer Stelle schreibt er: \enquote{\ori{Verstand} und
\ori{Sinnlichkeit} können bei uns \ori{nur in Verbindung} Gegenstände
bestimmen. Wenn wir sie trennen, so haben wir Anschauungen ohne Begriffe, oder
Begriffe ohne Anschauungen, in beiden Fällen aber Vorstellungen, die wir auf
keinen bestimmten Gegenstand beziehen können}
\mkbibparens{\cite[][B 314]{Kant:KritikderreinenVernunft2003},
\cite[][III: 213.32--36]{Kant:GesammelteWerke1900ff.}}.} Er beschreibt einen
solchen Verstand auch als einen Verstand, der durch eine nicht-sinnliche
Anschauung seinen Gegenstand \singlequote{intuitiv} erkennt.\footnote{\cite[][B
311\,f.,]{Kant:KritikderreinenVernunft2003} \cite[][III:
212.16--21]{Kant:GesammelteWerke1900ff.}.} Ebenfalls in der \titel{Kritik der
reinen Vernunft} spricht \name[Immanuel]{Kant} von einer
\enquote{gesetzgebende[n] Vernunft (intellectus archetypus) {\punkt}, von der
alle systematische Einheit der Natur, als dem Gegenstande unserer Vernunft,
abzuleiten sei.}\footnote{\cite[][B 723]{Kant:KritikderreinenVernunft2003},
\cite[][III: 456.37--457.2]{Kant:GesammelteWerke1900ff.}.} Der Ausdruck
\enquote{\emph{intellectus archetypus}} findet sich auch in der \titel{Kritik
der Urteilskraft} zur Bezeichnung des \emph{intuitiven Verstandes} als
Gegenbegriff zu unserem diskursiven Verstand, den \name[Immanuel]{Kant}
\enquote{intellectus ectypus}
nennt.\footnote{\cite[Vgl.][\S~77]{Kant:KritikderUrteilskraft2009},
\cite[][V: 408.18--23]{Kant:GesammelteWerke1900ff.}.}


In einer \enquote{nur episodisch, zur
Erläuterung}\footnote{\cite[][\S~76]{Kant:KritikderUrteilskraft2009}, \cite[][V:
401.6--7]{Kant:GesammelteWerke1900ff.}.} angeführten Anmerkung in der
\titel{Kritik der Urteilskraft} finden wir die systematischste Ausarbeitung von
\name[Immanuel]{Kant}s Überlegungen zu einem menschlichen Denken:
\begin{nummerierung}
 \item Unser menschlicher Verstand müsse Möglichkeit und Wirklichkeit der Dinge
 unterscheiden, während ein anschauender Verstand nur das Wirkliche zum
 Gegenstand hätte.\footnote{\cite[Vgl.][\S~76]{Kant:KritikderUrteilskraft2009};
 \cite[][V: 401.31--403.19]{Kant:GesammelteWerke1900ff.}.} Unser
 Verstand sei ein Vermögen der Begriffe, das nur denken, nicht aber anschauen
 kann und daher auf Anschauungen angewiesen ist, die ihm von den Sinnen
 gegeben werden.\footnote{\cite[Vgl.][B
 92\,f.,]{Kant:KritikderreinenVernunft2003} \cite[][III:
 85.10--16]{Kant:GesammelteWerke1900ff.}. Dort sagt \name[Immanuel]{Kant}, dass
 die \emph{Erkenntnis} des menschlichen Verstandes diskursiv und nicht intuitiv
 sei \mkbibparens{ebenso \cite[][B 311\,f.,]{Kant:KritikderreinenVernunft2003}
 \cite[][III: 166.37--167.5]{Kant:GesammelteWerke1900ff.}}.
 Siehe dazu auch \cite[][B 135]{Kant:KritikderreinenVernunft2003}, \cite[][III:
 110.26--29]{Kant:GesammelteWerke1900ff.}, wo unser endlicher Verstand als
 Vermögen zu \emph{denken} beschrieben wird.}
 
 \item Unsere praktische \emph{Vernunft} ist endlich, insofern ihre Gesetze uns
 als Imperative entgegentreten, weil bei endlichen Wesen wie uns erstens die
 Einsicht in Gesetze, Regeln und Ratschläge nicht automatisch die
 Handlungsausführung herbeiführt, wir eben auch anders handeln können, und
 wir uns zweitens als endliche Wesen mit antagonistischen Neigungen
 konfrontiert sehen, die uns  gerade zum Zuwiderhandeln \emph{gegen}
 die Imperative der Vernunft verleiten.

 \item\label{IntuitiverVerstandvomAllgemeinenzumBesonderen} Nach Auskunft von
 \S~77 der \titel{Kritik der Urteilskraft} \enquote{können wir uns {\punkt}
 einen Verstand denken, der, weil er nicht wie der unsrige diskursiv, sondern
 intuitiv ist, vom Synthetisch-Allgemeinen (der Anschauung eines Ganzen als
 eines solchen) zum Besonderen geht, d.\,i. vom Ganzen zu den
 Teilen}\footnote{\cite[][\S~77]{Kant:KritikderUrteilskraft2009}, \cite[][V:
 407.19--23]{Kant:GesammelteWerke1900ff.}.}. Eben die entgegengesetzte
 Eigentümlichkeit unseres Verstandes, vom Analytisch-Allgemeinen (von Begriffen)
 zum Besonderen gehen zu müssen, sei verantwortlich dafür, dass die mechanische und
 die teleologische Erklärungsart von Naturprodukten \emph{aus unserer Sicht} als
 unterschiedliche Arten der Erklärung nebeneinander bestünden. Einen intuitiven
 Verstand nennt \name[Immanuel]{Kant} hier auch \enquote{\emph{intellectus
 archetypus}} im Gegensatz zu unserem \enquote{\emph{intellectus
 ectypus}}\footnote{\cite[Siehe][\S~77]{Kant:KritikderUrteilskraft2009},
 \cite[][V: 408.18--23]{Kant:GesammelteWerke1900ff.}.}, wobei offen bleibt, ob
 dieser Begriff identisch ist mit dem eines \emph{intellectus archetypus} als
 \singlequote{gesetzgebender Vernunft} in der \titel{Kritik der reinen
 Vernunft}\footnote{\cite[Siehe][B 723]{Kant:KritikderreinenVernunft2003},
 \cite[][III: 456.37]{Kant:GesammelteWerke1900ff.}.}.
\end{nummerierung}


Es liegt \emph{prima facie} nahe davon auszugehen, dass \name[Immanuel]{Kant} in
einem einheitlichen Sinn von unserem menschlichen, endlichen Verstand, einem
diskursiven Verstand oder einem \emph{intellectus ectypus} und entsprechend einheitlich von
einem nicht-endlichen oder intuitiven Verstand oder einem \emph{intellectus
archetypus} spricht. Nicht ganz so nah liegt es, den intuitiven Verstand zugleich
mit der intellektuellen Anschauung zu identifizieren. Andererseits drängt sich
folgender Gedankengang auf: Wenn ein Verstand diskursiv ist, der denkt und nicht
anschaut, ein intuitiver Verstand aber anschaut, dann muss man doch vermuten,
dass die Anschauungen eines solchen Verstandes eben intellektuelle im Gegensatz
zu sinnlichen Anschauungen sind; schließlich entstammen sie dem Verstand
(\emph{intellectus}) und nicht den Sinnen. Es ist nichts natürlicher, als die
intellektuellen Anschauungen eben als die Produkte eines anschauenden Verstandes
anzusehen.


Nun kann man dennoch mit
\authorfullcite{Foerster:DieBedeutungvonSS7677deremphKritikderUrteilskraftfuerdieEntwicklungdernachkantischenPhilosophieTeil12002}
die Auffassung vertreten, dass solche Annahmen zumindest
begründungsbedürftig sind. Es handelt sich schließlich um gehaltvolle
philosophiehistorische Behauptungen mit erheblicher Tragweite für unser
Verständnis der Philosophie \name[Immanuel]{Kant}s, wie auch der
nach-kantischen Philosophie von \name[Salomon]{Maimon} über
\authorcite{Fichte:DieBestimmungdesMenschen1800} und
\authorcite{Schelling:Historisch-kritischeAusgabe1976-} bis
\authorcite{Hegel:GesammelteWerke}.\footnote{Insbesondere
\authorcite{Hegel:GesammelteWerke} und
\authorcite{Schelling:Historisch-kritischeAusgabe1976-} wird vorgeworfen
vorauszusetzen, dass die intellektuelle Anschauung der
intuitive Verstand sei. \enquote{Dazu ist festzuhalten, dass Hegel sich bis 1803
hinsichtlich des Absoluten genau wie Schelling durchgängig am Urgrundgedanken
des \S~76 der Kritik der Urteilskraft orientiert und, genau wie dieser,
intellektuelle Anschauung und intuitiven Verstand identifiziert bzw.
nicht zwischen ihnen unterscheidet. Der wesentliche Unterschied zwischen den
\S\S~76 und 77, also zwischen dem, was nach Kant zwei je verschiedene Grenzen
des menschlichen Erkenntnisvermögens bezeichnet, wird von beiden gleichermaßen
übersehen}
\parencite[][325]{Foerster:DieBedeutungvonSS7677deremphKritikderUrteilskraftfuerdieEntwicklungdernachkantischenPhilosophieTeil22002}.
Dabei identifiziert \authorcite{Hegel:GesammelteWerke}
den intuitiven Verstand und damit zugleich die intellektuelle Anschauung
mit der transzendentalen Einbildungskraft, um zu zeigen, dass
\name[Immanuel]{Kant} inkonsequent verfahre, wenn er die Möglichkeit eines
intuitiven Verstandes leugnet: \enquote{Von dieser Idee erkennt Kant zugleich,
daß wir nothwendig auf sie getrieben werden, und die \ori{Idee} dieses
urbildlichen, \ori{intuitiven Verstandes} ist im Grunde
durchaus nichts anders als \ori{dieselbe Idee der transcendentalen
Einbildungskraft}, die wir oben betrachteten, denn sie ist
anschauende Thätigkeit, und zugleich ist ihre innere Einheit gar keine andere,
als die Einheit des Verstandes selbst, die Kategorie in die Ausdehnung versenkt, die
erst Verstand und Kategorie wird, insofern sie sich von der Ausdehnung
absondert; die transcendentale Einbildungskraft ist also selbst anschauender
Verstand}
\mkbibparens{\cite[][IV: 341.1--8]{Hegel:GesammelteWerke}}. Während es mir korrekt zu sein
scheint, intellektuelle Anschauung und intuitiven Verstand zwar nicht zu
identifizieren, aber doch zu sagen, dass intellektuelle Anschauungen eben die
Erkenntnisse sind, die ein Verstand generiert, der nicht denkt, sondern
anschaut, also ein intuitiver Verstand, ist die Identifizierung von intuitivem
Verstand und transzendentaler Einbildungskraft freilich zurückzuweisen.}
In Opposition zu leichtfertigen Identifizierungen ließe sich eben auch denken,
dass nur die \emph{Funktion} der beschriebenen Verstandesarten gleich
sei.\footnote{So behauptet auch
\authorfullcite{McLaughlin:KantsKritikderteleologischenUrteilskraft1989}, es
gebe bei \name[Immanuel]{Kant} eine ganze Reihe \singlequote{intuitiver}
Verstandesarten, die keine gemeinsamen Eigenschaften haben, aber alle eine
gemeinsame Funktion als Vergleichsverstand
\parencite[vgl.][\pno~153\,f.]{McLaughlin:KantsKritikderteleologischenUrteilskraft1989}.}
Sie werden konstruiert, um Eigentümlichkeiten unseres endlichen Denkens vor Augen zu führen.
Nichts garantiert dabei, dass es sich stets um dieselbe Eigentümlichkeit handelt. 
Und so differenziert \authorcite{Foerster:Die25JahrederPhilosophie2011} streng
zwischen den drei Charakterisierungen unserer Endlichkeit:
\begin{quote}
\ori{Weil} wir in Verstand und Sinnlichkeit zwei voneinander unabhängige Stämme der
Erkenntnis haben, müssen wir zwischen Möglichkeit und Wirklichkeit unterscheiden
(anders: eine intellektuelle Anschauung); \ori{weil} wir sowohl Sinnen- als auch
Vernunftwesen sind, erscheint uns das Sittengesetz als ein Sollen, nicht als ein
Sein oder Wollen (anders: ein heiliger Wille); \ori{weil} unser Verstand diskursiv
ist, beurteilen wir Organismen unweigerlich als Naturzwecke (anders: ein
intuitiver Verstand).\footcite[][153]{Foerster:Die25JahrederPhilosophie2011}
\end{quote}
In der \titel{Kritik der reinen Vernunft} gehe es um die Besonderheit unserer
\emph{Anschauung} als sinnlicher und intellektueller; die \titel{Grundlegung
zur Metaphysik der Sitten} und die \titel{Kritik der praktischen Vernunft}
thematisierten hingegen die Besonderheit unseres endlichen im Vergleich zu einem
heiligen \emph{Willen} und schließlich die \titel{Kritik der Urteilskraft} die
Besonderheit unseres diskursiven im Kontrast zu einem intuitiven
\emph{Verstand}.\footnote{\authorcite{Foerster:DieBedeutungvonSS7677deremphKritikderUrteilskraftfuerdieEntwicklungdernachkantischenPhilosophieTeil12002}
geht es dabei vor allem um den Unterschied zwischen der in der \titel{Kritik der
reinen Vernunft} thematisierten Besonderheit unseres Erkenntnisvermögens und der
in der \titel{Kritik der Urteilskraft} diskutierten Diskursivität unseres
Verstandes. Die Einordnung der Endlichkeit unseres Willens wird nicht eigens
thematisiert.}
\begin{comment}
\begin{figure}[htb]
\begin{minipage}[t]{\textwidth}
\centering
\begin{tikzpicture}[edge from parent fork down,
level 1/.style={sibling distance=6cm, level distance=0cm},
level 2/.style={sibling distance=4cm, level distance=0cm},
level 3/.style={sibling distance=3cm, level distance=1.5cm},
level 4/.style={sibling distance=3cm, level distance=1.5cm},
level 5/.style={sibling distance=3cm, level distance=1.5cm},
every node/.style={rectangle,draw=black,fill=gray!25, thin, inner sep=0.5em, minimum size=0.5em, align=center},
edge from parent/.style={draw=none},
mylabel/.style={draw=none, fill=none, text width=5cm,text centered, inner sep=0.5em, anchor=base} ]
\node[draw=none,fill=none] {}
child {node[draw=none,fill=none] {}
	child {node[draw=none,fill=none] {}
		child {node[text width=3cm,rounded corners,thick] (KrV) {\emph{KrV}}
		child {node[text width=3cm,rounded corners,thick] (KpV) {\emph{KpV}, \emph{GMS}}
		child {node[text width=3cm,rounded corners,thick] (KU) {\emph{KU}}}}}}}
child {node[draw=none,fill=none] {}
	child {node[text width=3cm,rounded corners,thick] (endlichkeit) {Unser
	endliches Vermögen} child {node[fill=none,draw=none,text width=3cm]
	(Anschauung) {sinnliche Anschauung} child {node[fill=none,draw=none,text width=3cm] (Wille) {endlicher Wille}
		child {node[fill=none,draw=none,text width=3cm] (Verstand) {diskursiver
		Verstand}}}}} child {node[text width=3cm,rounded corners,thick] (gegenbegriff)
		{\singlequote{Anderes} Vermögen} child {node[fill=none,text
		width=3cm,draw=none] (iA) {intellektuelle Anschauung}
		child {node[fill=none,text width=3cm,draw=none] (hW) {heiliger
		Wille}
		child {node[fill=none,text width=3cm,draw=none] (iV) {intuitiver
		Verstand}}}}}}; \draw[<->] (Anschauung.east) to (iA.west);
\draw[<->] (Wille.east) to (hW.west);
\draw[<->] (Verstand.east) to (iV.west);
% \draw[->] (KrV.east) to (Anschauung.west);
% \draw[->] (KpV.east) to (Wille.west);
% \draw[->] (KU.east) to (Verstand.west);
\end{tikzpicture}
  \caption{Gegenüberstellungen nach
  \authorcite{Foerster:Die25JahrederPhilosophie2011}}\label{abbildung:Foerster:Gegenueberstellungen}
\end{minipage}
\end{figure}
\end{comment}


\phantomsection\label{Abschnitt:FoerstersDifferenzierungDerArtenIntuivenVerstandes}

\authorcite{Foerster:Die25JahrederPhilosophie2011} warnt vor allem davor, die
intellektuelle Anschauung der ersten Kritik mit dem intuitiven Verstand der
dritten Kritik zu identifizieren. Zwar habe auch \name[Immanuel]{Kant} zunächst
nicht klar zwischen beiden differenziert, doch spätestens in den Ausführungen der \titel{Kritik der
Urteilskraft} seien beide systematisch unterschieden. Während die intellektuelle
Anschauung, die \name[Immanuel]{Kant} in \S~76 der \titel{Kritik der
Urteilskraft} thematisiert, Überlegungen zum Gegensatz von Rezeptivität und
Spontaneität zu explizieren helfe, betreffe die Konzeption
eines intuitiven Verstandes in \S~77 den Unterschied \emph{diskursiver} und
\emph{intuitiver} Erkenntnisse.\footnote{\cite[Vgl.][177]{Foerster:DieBedeutungvonSS7677deremphKritikderUrteilskraftfuerdieEntwicklungdernachkantischenPhilosophieTeil12002}.}

Nun spricht auch der \S~77 der \titel{Kritik der Urteilskraft} von einer
intellektuellen Anschauung, wie
\authorcite{Foerster:DieBedeutungvonSS7677deremphKritikderUrteilskraftfuerdieEntwicklungdernachkantischenPhilosophieTeil12002}
einräumen
muss.\footcite[Vgl.][178]{Foerster:DieBedeutungvonSS7677deremphKritikderUrteilskraftfuerdieEntwicklungdernachkantischenPhilosophieTeil12002}
Somit ist die ursprüngliche Einteilung in dieser Einfachheit letztlich nicht
haltbar; stattdessen sieht sich \authorcite{Foerster:Die25JahrederPhilosophie2011} gezwungen, eine
Mehrdeutigkeit auch des Begriff \enquote{intellektuelle Anschauung} und mehrere
Konzeptionen eines \singlequote{anderen} Verstandes anzunehmen.
Somit seien verschiedene Konzeptionen von intellektueller Anschauung und intuitivem Verstand
zu unterscheiden.\footnote{Siehe zu diesen Unterscheidungen bei
\authorcite{Foerster:DieBedeutungvonSS7677deremphKritikderUrteilskraftfuerdieEntwicklungdernachkantischenPhilosophieTeil12002}
auch \cite[][151]{Quarfood:DiscursivityandTranscendentalIdealism2012}.} Manche
dieser Konzeptionen seien als unendlich oder \singlequote{göttlich} anzusehen.
Andere wiederum seien nicht als unendlich konzipiert, sondern stellten andere
endliche Formen des Verstandes und des Anschauens dar. Zunächst sind nach
\authorcite{Foerster:DieBedeutungvonSS7677deremphKritikderUrteilskraftfuerdieEntwicklungdernachkantischenPhilosophieTeil12002}
folgende zwei Verständnisse von \enquote{intellektuelle Anschauung} zu
unterscheiden:
\begin{nummerierung}
  \item Die intellektuelle Anschauung, die in der Deduktion der reinen
  Verstandesbegriffe in der \titel{Kritik der reinen Vernunft} zum Vergleich
  herangezogen wird, sei eine \emph{produktive Anschauung}, die ihre
  Gegenstände selbst hervorbringe. Sie sei \enquote{produktive Einheit von
  Möglichkeit (Denken) und Wirklichkeit
  (Sein)}\footcite[][179]{Foerster:DieBedeutungvonSS7677deremphKritikderUrteilskraftfuerdieEntwicklungdernachkantischenPhilosophieTeil12002}.
  Das \emph{könne} heißen, dass die produktive Anschauung das Ganze der Welt
  hervorbringt, aber die Konzeption einer produktiven Anschauung verpflichte
  doch nicht darauf und sei daher nicht \emph{per se} überschwenglich.
  \authorcite{Fichte:DieBestimmungdesMenschen1800} und
  \name[Friedrich Wilhelm Joseph]{Schelling} hätten die
  Realisierbarkeit einer solchen Anschauung behauptet und für erfahrbar
  gehalten.
  \item Eine andere Form der intellektuellen Anschauung bringe ihre
  Gegenstände nicht selbst hervor, erkenne aber die Dinge an sich, da sie keine
  sinnliche Anschauung und daher nicht den Bedingungen unserer Sinnlichkeit
  unterworfen sei. (Die Abhängigkeit unserer Wahrnehmung von der Form unserer
  Sinnlichkeit macht \name[Immanuel]{Kant} dafür verantwortlich, dass wir die
  Dinge nur so wahrnehmen, wie sie uns \emph{erscheinen}, nicht so, wie sie
  \emph{an sich} sind.\footnote{\cite[Vgl.][B 59]{Kant:KritikderreinenVernunft2003},
  \cite[][III: 65.9--22]{Kant:GesammelteWerke1900ff.}.}) Als Beispiele für
  Konzeptionen nicht-sinnlicher Anschauungen verweist
  \authorcite{Foerster:Die25JahrederPhilosophie2011} auf antike
  \singlequote{Sehstrahltheorien}; in einer solchen Theorie werde unsere
  Anschauung beschrieben als nicht rezeptiv und daher fähig, die Dinge als Dinge
  an sich zu erkennen und nicht bloß, wie sie uns erscheinen.\footnote{Vgl.
  \cite[][178]{Foerster:DieBedeutungvonSS7677deremphKritikderUrteilskraftfuerdieEntwicklungdernachkantischenPhilosophieTeil12002}.
  Warum solche Anschauungen dies leisten können sollten, erläutert
  \authorcite{Foerster:Die25JahrederPhilosophie2011} jedoch nicht.} Es sei diese
  Form der intellektuellen Anschauung, die in \S~77 der \titel{Kritik der
  Urteilskraft} bemüht werde und die sich zuvor im
  \titel{Noumena und Phaenomena}-Kapitel der \titel{Kritik der reinen Vernunft}
  finde. Unter den Philosophen, die im Anschluss an \name[Immanuel]{Kant}
  philosophierten, habe keiner eine solche übersinnliche Anschauung für möglich
  gehalten.
\end{nummerierung}
\authorfullcite{Prien:KantsLogikderBegriffe2006} betont, dass nur die
produktive intellektuelle Anschauung die Anschauung eines
göttlichen Verstandes sei, während derselbe Ausdruck oft auch lediglich auf eine
\singlequote{übersinnliche Anschauung} verweise, die unserem Verstand ebenso ein Mannigfaltiges darbiete, das
zwar nicht von den Bedingungen unserer Sinnlichkeit abhänge, aber dennoch die
Gegenstände der Erkenntnis nicht hervorbringe. Dabei denkt
\authorcite{Prien:KantsLogikderBegriffe2006} nicht wie
\authorcite{Foerster:Die25JahrederPhilosophie2011} an antike Sehstrahltheorien,
sondern an eine \singlequote{Ideenschau} im Sinne der Mathematikauffassung
\singlename{Platon}s.\footnote{\cite[Vgl.][96]{Prien:KantsLogikderBegriffe2006}.
Auch \authorfullcite{Duesing:NaturteleologieundMetaphysikbeiKantundHegel1990} sieht
\name[Immanuel]{Kant}s intellektuelle Anschauung in der
platonisch-neuplatonischen Tradition der Ideenschau; die Ideen seien mit den
intellektuellen Anschauungen sogar zu identifizieren
\mkbibparens{\cite[vgl.][\pno~144\,f.,]{Duesing:NaturteleologieundMetaphysikbeiKantundHegel1990},
siehe außerdem
\cite[][\pno~72\,f.]{Duesing:DieTeleologieinKantsWeltbegriff1968}}.} Durch eine
solche übersinnliche Anschauung wäre uns -- wenn wir über sie verfügten --
ebenso wie durch die sinnliche Anschauung ein Mannigfaltiges gegeben, welches
der Synthesis durch unseren Verstand bedürfte. Eine übersinnliche Anschauung
vertrage sich daher mit der Endlichkeit unseres
Verstandes.\footnote{\cite[Vgl.][97]{Prien:KantsLogikderBegriffe2006}.}

Nun ist es hier nicht von weiterer Relevanz, ob es sich bei der nicht-sinnlichen
Anschauung, die jedoch nicht produktiv sein soll, eher um eine Art Sehstrahl
oder um eine Art Ideenschau handelt. Ich werde sie der Kürze halber im folgenden
schlicht als \emph{übersinnliche} Anschauung bezeichnen. Es ist somit nach
\authorcite{Prien:KantsLogikderBegriffe2006} wie nach \authorcite{Foerster:Die25JahrederPhilosophie2011}
zwischen zwei Arten einer nicht-sinnlichen Anschauung zu unterscheiden:
der übersinnlichen und der produktiven Anschauung. Beide Konzeptionen
wären je nach konkreter Ausgestaltung auch mit einem endlichen
Verstand kombinierbar.

Eine ähnliche Unterscheidung wie diejenige zwischen produktiver und
übersinnlicher Anschauung lässt sich nach
\authorcite{Foerster:DieBedeutungvonSS7677deremphKritikderUrteilskraftfuerdieEntwicklungdernachkantischenPhilosophieTeil12002}
zwischen zwei Arten, den Terminus \enquote{intuitiver Verstand} zu explizieren,
rekonstruieren.\footcite[Vgl.][\pno~178\,f.]{Foerster:DieBedeutungvonSS7677deremphKritikderUrteilskraftfuerdieEntwicklungdernachkantischenPhilosophieTeil12002}
Beide Verwendungsweisen finden sich in \S~77 der
\titel{Kritik der Urteilskraft} und entsprechen einer bekannten
\singlequote{Doppelfunktion} des unserem endlichen kontrastierten Verstandes\footnote{\phantomsection\label{FussnoteDoppelfunktion}Dieser dient einerseits als
gedanklicher Kontrast zu Beschreibung der Besonderheiten unseres Verstandes, andererseits aber auch als
übersinnlicher Grund der Natureinheit \mkbibparens{vgl.
\cite[][68]{Duesing:DieTeleologieinKantsWeltbegriff1968}}.
\authorfullcite{Duesing:DieTeleologieinKantsWeltbegriff1968} geht jedoch
offensichtlich davon aus, dass es sich stets um die Konzeption desselben
(göttlichen) Verstandes handelt. Siehe dazu
\cite[][66--74]{Duesing:DieTeleologieinKantsWeltbegriff1968}.} in
der Argumentation dieses Paragraphen:
\begin{nummerierung}
  \item Einerseits werde der intuitive Verstand am Ende von \S~77 als
  \emph{ursprünglicher Verstand} beschrieben, der als \emph{Weltursache} zu
  denken wäre.\footnote{Siehe \cite[][\S~77]{Kant:KritikderUrteilskraft2009},
  \cite[][V: 410.11]{Kant:GesammelteWerke1900ff.}. Auch wenn
  \authorcite{Foerster:Die25JahrederPhilosophie2011} dies an dieser Stelle
  nicht ausführt, könnte man in diesem Kontext auch den \emph{intellectus
  archetypus} der \titel{Kritik der reinen Vernunft} einordnen als
  \enquote{gesetzgebende Vernunft (intellectus archetypus), {\punkt}, von der
  alle systematische Einheit der Natur, als dem Gegenstande unserer Vernunft,
  abzuleiten sei} \mkbibparens{\cite[][B 723]{Kant:KritikderreinenVernunft2003},
  \cite[][III: 456.37--457.2]{Kant:GesammelteWerke1900ff.}}.} Ein solcher
  Verstand erkennte nicht nur anders als wir, er brächte sogar den Gegenstand
  seiner Erkenntnis hervor. \enquote{Hier besteht zweifellos die größte Nähe
  zwischen intuitivem Verstand und produktiver, intellektueller Anschauung,
  obwohl letztere natürlich nicht als Ursache gleich des Ganzen der Welt gedacht
  werden muß, sondern diese Möglichkeit nur
  zulässt.}\footnote{\cite[][179]{Foerster:DieBedeutungvonSS7677deremphKritikderUrteilskraftfuerdieEntwicklungdernachkantischenPhilosophieTeil12002}.
  Die Realisierbarkeit eines solchen ursprünglichen Verstandes sei im Anschluss
  an \name[Immanuel]{Kant} von niemandem behauptet
  worden \parencite[vgl.][179]{Foerster:DieBedeutungvonSS7677deremphKritikderUrteilskraftfuerdieEntwicklungdernachkantischenPhilosophieTeil12002}.}
    \item Andererseits charakterisiere Kant den Verstand als
  \emph{synthetisch-allgemeinen Verstand}, also als einen solchen, der nicht wie
  der unsrige vom Analytisch-All\-ge\-mei\-nen, sondern vom
  \enquote{\ori{Synthetisch-Allgemeinen} (der Anschauung eines Ganzen als
  eines solchen)}\footnote{\cite[][\S~77]{Kant:KritikderUrteilskraft2009},
  \cite[][V: 407.21--22]{Kant:GesammelteWerke1900ff.}.} aus zum Besonderen gehe.
  Er könne auch bei einzelnen Naturprodukten das Besondere aus dem
  Synthetisch-All\-ge\-mei\-nen bestimmen. Damit beschreibe
  \name[Immanuel]{Kant} einen Verstand, der der \emph{scientia intuitiva} bei
  \authorcite{Spinoza:EthikingeometrischerOrdnungdargestellt2007}
  entspreche.\footnote{\cite[Vgl.][189]{Foerster:DieBedeutungvonSS7677deremphKritikderUrteilskraftfuerdieEntwicklungdernachkantischenPhilosophieTeil12002}.
  Es sei \name[Johann Wolfgang von]{Goethe} gewesen, der die Möglichkeit eines
  solchen Verstandes als erster in Erwägung gezogen und damit
  \authorcite{Hegel:GesammelteWerke}s Bruch mit
  \authorcite{Schelling:Historisch-kritischeAusgabe1976-} eingeleitet
  habe \parencite[vgl.][180--190]{Foerster:DieBedeutungvonSS7677deremphKritikderUrteilskraftfuerdieEntwicklungdernachkantischenPhilosophieTeil12002}.}
\end{nummerierung}
\phantomsection\label{Absatz:IntuitiverVerstandIntellektuelleAnschauungEndlicherWesen}
Ein synthetisch-allgemeiner Verstand müsse nun als solcher noch kein
ursprünglicher Verstand sein, der als Ursache der Welt zu denken wäre -- genau
so wenig, wie eine übersinnliche Anschauung als produktiv oder eine produktive
Anschauung als Anschauung von Dingen an sich selbst zu denken wäre.
Synthetisch-allgemeiner Verstand und übersinnliche wie produktive Anschauung
mögen zwar uns Menschen möglicherweise verschlossen sein, aber es sei doch denkbar, dass sie
Wesen zukommen, die zwar anders als wir, aber dennoch \emph{endliche} Wesen
sind. Und so konnten \authorcite{Fichte:DieBestimmungdesMenschen1800} und
\name[Friedrich Wilhelm Joseph]{Schelling} die Möglichkeit einer
produktiven Anschauung und \name[Johann Wolfgang von]{Goethe} die
Realisierbarkeit eines synthetisch-allgemeinen Verstandes erwägen, ohne damit Ungeheuerliches zu
behaupten. Nur wenn die verschiedenen zum Vergleich konzipierten
Erkenntnisvermögen nicht auseinandergehalten werden,
ergebe sich der Anschein, die genannten Autoren müssten die Realisierbarkeit
von Fähigkeiten behaupten, die lediglich Göttern
zukommen.\footnote{\cite[Vgl.][175--180]{Foerster:DieBedeutungvonSS7677deremphKritikderUrteilskraftfuerdieEntwicklungdernachkantischenPhilosophieTeil12002}.
Eine ähnliche Differenzierung lässt sich freilich auch vornehmen, ohne auf der
strikten Trennung von intellektueller Anschauung und intuitivem Verstand zu
bestehen. Nach \authorfullcite{Gram:IntellectualIntuition1981} finden sich bei
\name[Immanuel]{Kant} drei verschiedene Konzeptionen einer intellektuellen
Anschauung, die jeweils unterschiedliche Begriffe eines korrelierten Verstandes
voraussetzen, über den wir gerade nicht verfügen.
Auch wenn er dies nicht explizit schreibt, setzt \authorcite{Gram:IntellectualIntuition1981}
offensichtlich voraus, dass intellektuelle Anschauung gerade die Erkenntnis ist, die der intuitive Verstand
hervorbringt; er trennt also nicht wie
\authorcite{Foerster:DieBedeutungvonSS7677deremphKritikderUrteilskraftfuerdieEntwicklungdernachkantischenPhilosophieTeil12002}
zwischen der intellektuellen Anschauung und dem intuitiven Verstand als
unterschiedlichen Themenfeldern, sondern sieht mit beiden Ausdrücken jeweils
dieselben Sachverhalte angesprochen. Nun beschreibe
\name[Immanuel]{Kant}
\begin{nummerierung}
  \item einen Verstand, der seinen eigenen Gegenstand
  hervorbringt -- dies  entspricht dem Konzept einer produktiven Anschauung --,
  \item einen Verstand, der Dinge an sich unabhängig von Bedingungen
  unserer Sinnlichkeit erkennt, -- was dem Konzept einer übersinnlichen
  Anschauung entspricht --, sowie
  \item  einen Verstand, der das Ganze aller \emph{Phänomena} anzuschauen
  vermag -- dies entspricht dem vom Synthetisch-Allgemeinen ausgehenden
  intuitiven Verstand, insofern es sich um einen Verstand handelt, der das Ganze
  (als Ganzes) anzuschauen
  vermag \parencite[vgl.][288]{Gram:IntellectualIntuition1981}.
\end{nummerierung}
\authorfullcite{Leech:MakingModalDistinctions2014} erwähnt außerdem die
Idealisten nach \name[Immanuel]{Kant} mit Theorien eines \singlequote{sich
selbst setzenden Ich} \parencite[vgl.][\pno~348,
353\,f.]{Leech:MakingModalDistinctions2014}. Diese Überlegungen bleiben hier
aus Gründen, die in der Einleitung ausgeführt  wurden (siehe
S.~\pageref{Einleitung:AbschnittIdealistennachKant}), unberücksichtigt.}



Das Argument, welches Autoren wie \authorcite{Foerster:Die25JahrederPhilosophie2011} vorbringen, die zwischen
den verschiedenen Beschreibungen eines nicht-endlichen Denkens lediglich einen funktionalen
Zusammenhang sehen, lautet: Mag auch der Name \enquote{intuitiver Verstand}
oder \enquote{\emph{intellectus archetypus}} gleich oder die Darstellung ähnlich
sein, so wechseln doch die Beschreibungen und Charakterisierungen. Diese seien aber logisch voneinander
unabhängig. Ob ein Verstand anschaut oder nur denkt, ob er seinen Gegenstand
selbst hervorbringt oder auf sinnlich Gegebenes angewiesen bleibt, das seien
ganz andere Fragen als die, ob er von den Teilen zum Ganzen oder
vom Ganzen zu den Teilen geht. Und letztlich gehe es \name[Immanuel]{Kant} auch
nicht darum, eine kohärente Konzeption eines nicht-endlichen Erkenntnisvermögens
zu beschreiben, sondern darum, verschiedene Besonderheiten unseres Verstandes
herauszustellen. Insofern sei zwar die \emph{Funktion} der Beschreibung eines
\singlequote{anderen} Verstandes stets dieselbe. Aber insofern an verschiedenen
Stellen seines Werks verschiedene Eigenschaften unseres Verstandes zu
beschreiben seien, variiere eben auch die Beschreibung der
Eigenschaften eines nicht-endlichen Verstandes.

Wenn das stimmen sollte, dann kann auch von einem einheitlichen Begriff
menschlicher Endlichkeit bei \name[Immanuel]{Kant} nicht gesprochen werden. Es
gäbe möglicherweise verschiedene Aspekte unserer Endlichkeit, die durch kein
systematisches Band zusammengehalten werden. Und dann wäre es auch falsch zu
fragen, was denn die \emph{grundlegende} Bestimmung unserer Endlichkeit ist. Es
wäre bei den Charekterisierungen jeweils fraglich, ob überhaupt eine Konzeption von \emph{Endlichkeit}
vorliegt. Schließlich wären zwar (möglicher- aber nicht notwendigerweise)
\emph{einige} Beschreibungen eines \singlequote{anderen} Verstandes
Beschreibungen eines unendlichen, andere aber wiederum bloß Beschreibungen eines
zwar anderen, aber nichtsdestotrotz ebenso endlichen Verstandes. Ob bei
\name[Immanuel]{Kant} überhaupt systematisch von Endlichkeit gesprochen wird,
wäre somit eine offene Frage.

Vorausgesetzt werden muss dabei freilich, dass die Bestimmungen tatsächlich
voneinander unabhängig sind, dass also beispielsweise ein Verstand, der seinen
Gegenstand selbst hervorbringt, ebenso von den Teilen zum Ganzen gehen kann, und
dass ein Verstand, der selbst anschaut, seinen Gegenstand damit nicht selbst
hervorbringt. Dass sich bei \name[Immanuel]{Kant} unterschiedliche
Beschreibungen eines \singlequote{anderen} Verstandes und einer \singlequote{anderen} Anschauung
finden lassen, ist nicht zu bestreiten. Eine Anschauung als produktiv zu
beschreiben ist etwas anderes als zu sagen, sie sei eine Anschauung von Dingen
an sich selbst. Und einen Verstand als anschauend zu bezeichnen scheint etwas
ganz anderes zu sein als zu sagen, er gehe vom Allgemeinen zum Besonderen.
Und dies alles unterscheidet sich weiterhin von der Aussage, eine Anschauung sei
nicht unseren Formen der Sinnlichkeit unterworfen. Insofern ist den genannten
Autoren dem ersten Anschein nach Recht zu geben. Aber aus der
Unterschiedlichkeit der Beschreibungen lässt sich nicht darauf schließen, dass es möglich ist, eine
Anschauung als nicht sinnlich und dennoch als nicht-produktiv zu konzipieren
oder einen Verstand, der zwar denkt und nicht anschaut, dabei aber nicht vom
Synthetisch-Allgemeinen, sondern vom Analytisch-Allgemeinen zum Besonderen
geht.\footnote{Lediglich \authorfullcite{Gram:IntellectualIntuition1981}
argumentiert ausführlich für die These, die Beschreibungen des intuitiven
Verstandes seien unterschiedlich, sondern auch wechselseitig logisch
inkompatibel \parencite[vgl.][287--296]{Gram:IntellectualIntuition1981}.}
Möglicherweise lassen sich zwischen den verschiedenen Charakterisierungen logische
Zusammenhänge entwickeln, die zeigen, dass \name[Immanuel]{Kant}s
unterschiedliche Beschreibungen eines Verstandes und einer Anschauung, die nicht
den unseren entsprechen, letztlich auf denselben Begriff eines intuitiven
Verstandes und seiner intellektuellen Anschauung hinauslaufen.\footnote{So
argumentiert auch Jessica
\textcite[vgl.][348--356]{Leech:MakingModalDistinctions2014}.}

\section{Abhängigkeit als Grundbestimmung unserer
Endlichkeit}\label{subsection:DiskursiverVerstandundsinnlicheAnschauung}
\name[Immanuel]{Kant} sagt, dass unser Verstand nur zusammen mit unserer
Sinnlichkeit Gegenstände bestimmen kann, weil er nur
\emph{denken}, nicht aber anschauen kann.\footnote{\enquote{Ein Verstand, in
welchem durch das Selbstbewußtsein zugleich alles Mannigfaltige gegeben würde, würde \ori{anschauen}; der unsere kann nur
\ori{denken} und muß in den Sinnen die Anschauung
suchen.} \mkbibparens{\cite[][\S~16]{Kant:KritikderreinenVernunft2003},
\cite[][III: 110.26--29]{Kant:GesammelteWerke1900ff.}}.
\cite[Vgl.\ außerdem][\S\S~17, 21, B~311f.]{Kant:KritikderreinenVernunft2003},
\cite[][III: 112.20--33, 116.13--18, 212.16--21]{Kant:GesammelteWerke1900ff.}.
\name[Immanuel]{Kant} vermeidet das Prädikat \enquote{endlich} und spricht eher
von \emph{unserem} Verstand, den er auch
\enquote{intellectus ectypus} im Unterschied zu einem \enquote{intellectus archetypus}
nennt, oder einer nicht-sinnlichen Anschauung und einem Verstand, der die
Objekte selbst hervorbringt.} Doch der Versuch zu
explizieren, was es heißt, dass unser Verstand nicht
anschaut, sondern denkt, erweist sich als schwieriger als zunächst angenommen.
\name[Immanuel]{Kant} spricht an entsprechenden Stellen davon, dass die
Erkenntnis unseres Verstandes nicht intuitiv, sondern diskursiv sei; doch macht
dies die Interpretation nicht leichter, denn den Begriff der Diskursivität
finden wir nirgends explizit erläutert. Hinzu kommt eben, dass dies nicht die
einzige Beschreibung der Besonderheit unseres endlichen Denkens und Erkennens
ist.

Ich möchte im folgenden zeigen, dass es eine
grundlegende Bestimmung unserer Endlichkeit gibt und dass diese in der
Abhängigkeit unseres Verstandes von der Sinnlichkeit liegt und er \emph{a
fortiori} diskursiv oder ein Vermögen der Begriffe ist. Dafür werde ich
argumentieren, dass \name[Immanuel]{Kant} über genau \emph{eine} Konzeption
eines intuitiven Verstandes als eines alternativ zu unserem konzipierten
Erkenntnisvermögen und einer intellektuellen Anschauung als der
korrespondierenden Erkenntnisart verfügt. Zunächst gilt es jedoch, die
beteiligten Begriffe genau zu differenzieren. Ohne Klarheit darüber, welche
Merkmale grundlegend für Begriffe wie \enquote{Verstand}, \enquote{Anschauung}
oder \enquote{diskursiv} sind, sind seriöse Antworten nicht zu erwarten. Daher
soll die nötige Begriffsklärung im folgenden Teil geleistet werden, wobei sich
insbesondere der Begriff der Diskursivität als klärungsbedürftig erweist; Kapitel
\ref{subsubsection:BegriffderDiskursivitaet} widmet sich entsprechend dem
Begriffspaar \enquote{diskursiv} und \enquote{intuitiv} und wird damit die
Begriffe \enquote{Begriff} und \enquote{Anschauung} zu differenzieren haben.
Erst im Anschluss an diese Klärungen ist es sinnvoll, sich über \name[Immanuel]{Kant}s Angaben zu den
Begriffen einer intellektuellen, sinnlichen oder nicht-sinnlichen Anschauung und
eines intuitiven, anschauenden oder diskursiven Verstandes zu verständigen.
Kapitel \ref{subsection:VerstandundRezeptivitaet} wird
sich dieser Aufgabe widmen, die Begriffe \enquote{Sinnlichkeit} und \enquote{Verstand} und
darin inbegriffen die Begriffe \enquote{intellektuell}, \enquote{sinnlich},
\enquote{spontan} und \enquote{rezeptiv} zu bestimmen.

\subsection[Begriffe und Anschauungen]{Begriffe und Anschauungen:
\enquote{diskursiv} und
\enquote{intuitiv}}\label{subsubsection:BegriffderDiskursivitaet}
Zunächst gilt es, die Adjektive \enquote{intuitiv} und \enquote{diskursiv} zu
analysieren; \name[Immanuel]{Kant} gibt dafür keine eigenständige Definition,
verknüpft sie aber mit den Begriffen \enquote{Anschauung} und \enquote{Begriff}.
Bevor jedoch auf genauere Zusammenhänge eingegangen werden kann, ist zu fragen,
worauf \name[Immanuel]{Kant} diese Adjektive überhaupt anwendet; denn nicht nur
den Verstand nennt er diskursiv oder intuitiv.
Es finden sich hier mindestens drei Ansätze, wovon es sinnvoll ist zu behaupten,
es sei diskursiv oder intuitiv:
\begin{nummerierung}
\item Fast ausschließlich in \S~77 der \titel{Kritik der Urteilskraft} ist es
der \emph{Verstand}, der als diskursiv oder intuitiv beschrieben
wird.\footnote{\cite[Vgl.][\S~77]{Kant:KritikderUrteilskraft2009}, \cite[][V:
407.19--21]{Kant:GesammelteWerke1900ff.}.} In der \titel{Kritik der reinen
Vernunft} werden die Adjektive \enquote{diskursiv} und \enquote{intuitiv} nicht ein
einziges Mal auf den Verstand selbst bezogen, sondern ausschließlich auf
Erkenntnisse des Verstandes.\footnote{In der \titel{Kritik der reinen Vernunft} nennt er die
\enquote{Erkenntnis eines jeden, jedenfalls des menschlichen, Verstandes}
diskursiv \mkbibparens{\cite[][B 93]{Kant:KritikderreinenVernunft2003},
\cite[][III: 85.14--15]{Kant:GesammelteWerke1900ff.}}. Eine
Erkenntnis durch diskursive Vorstellungen, also Begriffe, heißt nach
der \titel{Logik} eine \enquote{cognitio
discursiva} \mkbibparens{\cite[][\S~1]{Kant:ImmanuelKantsLogik1977}, \cite[][IX:
91.11]{Kant:GesammelteWerke1900ff.}}, wobei unausgemacht bleibt, ob es sich
hier um die Tätigkeit des Verstandes handelt oder um deren Resultate.} In den
\titel{Prolegomena} hingegen findet sich die Attribuierung der Diskursivität von
unserem Verstand in \S~57 \titel{Von der Grenzbestimmung der reinen
Vernunft}.\footnote{\cite[Siehe][A 163\,f.,
172]{Kant:ProlegomenazueinerjedenkuenftigenMetaphysikdiealsWissenschaftwirdauftretenkoennen1977},
\cite[][IV: 351.2--3, 355.7]{Kant:GesammelteWerke1900ff.}. Eine eher beiläufige
Bezeichnung des Verstandes als diskursiv findet sich außerdem in der
\titel{Kritik der praktischen Vernunft}
\mkbibparens{\cite[siehe][A 247]{Kant:KritikderpraktischenVernunft1974}, \cite[][V:
137.12]{Kant:GesammelteWerke1900ff.}}.} Wenn
\name[Immanuel]{Kant} nun die Erkenntnis des Verstandes diskursiv (oder
intuitiv) nennt, dann kann damit wegen der Vieldeutigkeit von
\enquote{Erkenntnis} immer noch dreierlei gemeint
sein: Zum einen die Tätigkeit des Verstandes und zum anderen die Erkenntnisse,
die wir mittels des Verstandes erwerben oder besitzen, also -- wiederum doppelt
-- Erkenntnisse im Sinne objektiv gültiger Urteile (Erkenntnisse im eigentlichen
Sinne) und Erkenntnisse im Sinne objektbezogener Vorstellungen (Erkenntnisse im
uneigentlichen
Sinne).\footnote{\phantomsection\label{Anmerkung:ErkenntnisInZweierleiSinn}Es
ist zu beachten, dass \name[Immanuel]{Kant} verschiedene Erkenntnisbegriffe hat;
in einem Sinne nennt er ausschließlich Urteile \enquote{Erkenntnisse}, in einem
anderen Sinne ist \enquote{Erkenntnis} der Oberbegriff zu Anschauungen und
Begriffen. \authorfullcite{Gruene:BlindeAnschauung2009} nennt die Erkenntnisse,
die sich in Urteilen äußern, deren Begriffe objektive Realität besitzen, \enquote{Erkenntnis im engen Sinn} und unterscheidet sie von
solchen im weiten Sinne, als bewusster Vorstellungen, die sich auf Gegenstände
beziehen und in Anschauungen und Begriffe unterteilt werden
\parencite[vgl.][29]{Gruene:BlindeAnschauung2009}.
\authorfullcite{Prien:KantsLogikderBegriffe2006} spricht von Erkenntnissen im
eigentlichen und im uneigentlichen Sinn. \enquote{Erkenntnis im eigentlichen
Sinne ist immer ein Urteil, denn nur in Urteilen kann man Gegenstände erkennen,
da man nur in Urteilen etwas von ihnen aussagt}
\parencite[][7]{Prien:KantsLogikderBegriffe2006}.} Die Tätigkeit des Erkennens
durch den Verstand oder dessen \singlequote{Form}\footnote{\cite[][B
283]{Kant:KritikderreinenVernunft2003}, \cite[][III:
195.37--196.1]{Kant:GesammelteWerke1900ff.}.} als diskursiv zu bezeichnen, kann
noch damit identifiziert werden, den Verstand selbst als diskursiv zu
bezeichnen; die anderen Möglichkeiten hingegen sind zweifellos gesondert zu betrachten:

\item In einem wichtigen Sinn sind objektive Vorstellungen diskursiv oder
intuitiv: Diskursivität ist in diesem Sinne eine Eigenschaft von Begriffen im
Gegensatz zu Anschauungen, die nicht diskursiv sind, sondern intuitiv. Hier bestimmt das
Adjektiv \enquote{diskursiv} \emph{Erkenntnisse im uneigentlichen
Sinne}, also Vorstellungen, die
bewusst und objektiv sind, und unterteilt sie in Begriffe (diskursive objektive
Vorstellungen, \enquote{representatio[nes]
discursiva[e]}\footnote{\cite[][\S~1]{Kant:ImmanuelKantsLogik1977}, \cite[][IX:
91.10]{Kant:GesammelteWerke1900ff.}.}) und Anschauungen (intuitive objektive
Vorstellungen).

\item\label{Aufzaehlung:Vernunftgebrauchintuitivoderdiskursiv} Schließlich
findet sich aber auch die Redeweise von diskursiven oder intuitiven
Erkenntnissen im eigentlichen Sinne, also objektiv gültigen Urteilen. In der
\titel{Kritik der reinen Vernunft} heißt es etwa: \enquote{Also ist ein
transzendentaler Satz ein synthetisches Vernunfterkenntnis nach bloßen
Begriffen, und mithin diskursiv}\footnote{\cite[][B
750]{Kant:KritikderreinenVernunft2003}, \cite[][III:
474.21--23]{Kant:GesammelteWerke1900ff.}.}. So unterscheidet Kant zumindest bei
rationalen Erkenntnissen -- also solchen \emph{a priori} oder \emph{ex principiis}\footnote{Siehe
hierzu Kapitel \ref{section:MuendigkeitundPhilosophie} dieser Arbeit.} --
zwischen intuitiven rationalen Erkenntnissen (aus der Konstruktion von Begriffen
in der reinen Anschauung) in der Mathematik und diskursiven rationalen
Erkenntnissen (aus Begriffen) in der Philosophie. Einen Beweis nennt
\name[Immanuel]{Kant} in der Methodenlehre der \titel{Kritik der reinen
Vernunft} diskursiv oder intuitiv je nach Art der Beweisführung (ob sie
mathematisch oder philosophisch ist).\footnote{\cite[][B
762\,f.,]{Kant:KritikderreinenVernunft2003} \cite[][III:
481.15--482.2]{Kant:GesammelteWerke1900ff.}.} Ganz analog dazu unterteilt die
\titel{Logik} die Grundsätze von Beweisen in diskursive und
intuitive\footnote{\cite[Vgl.][\S~35]{Kant:ImmanuelKantsLogik1977}, \cite[][IX:
110.24--28]{Kant:GesammelteWerke1900ff.}.} und ebenso die resultierende
\singlequote{Gewissheit} in intuitive Gewissheit (Evidenz in der Mathematik) und
diskursive Gewissheit (in der Philosophie)\footnote{\cite[Vgl.][A
107]{Kant:ImmanuelKantsLogik1977}, \cite[][IX:
70.34--37]{Kant:GesammelteWerke1900ff.}.}. In der Kritik der reinen Vernunft
wird in diesem Zusammenhang der \emph{Vernunftgebrauch} als diskursiv oder
intuitiv beschrieben.\footnote{\cite[Vgl.][B
747]{Kant:KritikderreinenVernunft2003}, \cite[][III:
472.25--28]{Kant:GesammelteWerke1900ff.}.}
Diese Verwendungsweise von \enquote{diskursiv} und \enquote{intuitiv} ist in der Philosophie
vor \name[Immanuel]{Kant} die gebräuchliche. So unterscheiden
\authorcite{Wolff:Discursuspraeliminarisdephilosophiaingenere1996},
\authorcite{Baumgarten:Metaphysica---Metaphysik2011} und
\authorcite{Meier:AuszugausderVernunftlehre1752} zwischen \emph{iudicia
intuitiva} und \emph{iudicia discursiva} (beziehungsweise
zwischen \emph{propositiones intuitivae} und \emph{discursivae}).\footnote{Siehe
\cite[][\S~51]{Wolff:PhilosophiarationalissiveLogica1740},
\cite[][\S~166]{Baumgarten:AcroasislogicainChristianumL.B.deWolff1983}, sowie
\cite[][\S~319]{Meier:AuszugausderVernunftlehre1752}
\parencite[][XVI: 674.24--28]{Kant:GesammelteWerke1900ff.}. Bzgl.
\authorcite{Wolff:Discursuspraeliminarisdephilosophiaingenere1996} siehe
auch
\cite{Ecole:Duroledelentendementintuitifdanslaconceptionwolffiennedelaconnaissance1986}.}
\end{nummerierung}

Daneben gibt es einige Zuschreibungen von Diskursivität, die selten
vorkommen und eindeutig als derivativ aufgefasst werden können: In der
Anthropologie in pragmatischer Hinsicht findet sich die Zuschreibung der
Adjektive \enquote{diskursiv} und \enquote{intuitiv} bezüglich des
Ausdrucks \enquote{Bewusstsein}.\footnote{\enquote{Weil Erfahrung empirisches
Erkenntnis ist, zum Erkenntnis aber (da es auf Urteilen beruht) Überlegung
(reflexio), mithin Bewußtsein, d.\,i. Tätigkeit in Zusammenstellung des
Mannigfaltigen der Vorstellung nach einer Regel der Einheit desselben, d.\,i.
\ori{Begriff} und (vom Anschauen unterschiedenes) Denken überhaupt erfordert
wird: so wird das Bewußtsein in das \ori{diskursive} (welches, als logisch, weil
es die Regel gibt, vorangehen muß), und das \ori{intuitive} Bewußtsein
eingeteilt werden} \mkbibparens{\cite[][BA
27]{Kant:AnthropologieinpragmatischerHinsicht1977}, \cite[][VII:
141.21--27]{Kant:GesammelteWerke1900ff.}}.} Dabei identifiziert
\name[Immanuel]{Kant} das diskursive Bewusstsein mit der reinen Apperzeption,
das intuitive Bewusstsein hingegen mit dem inneren Sinn. Der Ausdruck \enquote{intuitiv}
meint an dieser Stelle \enquote{empirisch} und verweist auf den inneren Sinn der
\titel{Kritik der reinen Vernunft}, während \enquote{diskursiv} hier mit
\enquote{rein} zu übersetzen ist und auf das \enquote{Ich denke} verweist, das
alle meine Vorstellungen muss begleiten können, welches aber nur die Einheit
enthält, durch die noch nichts Mannigfaltiges gegeben
ist.\footnote{Siehe \cite[][\S~16]{Kant:KritikderreinenVernunft2003},
\cite[][III: 108.16--110.35]{Kant:GesammelteWerke1900ff.}.} Ebenfalls in der
\titel{Anthropologie} spricht \name[Immanuel]{Kant} von der \enquote{diskursive[n] Vorstellungsart
durch laute Sprache oder durch Schrift}\footnote{\cite[][BA
192]{Kant:AnthropologieinpragmatischerHinsicht1977}, \cite[][VII:
244.36--245.1]{Kant:GesammelteWerke1900ff.}.} in Beredsamkeit und Dichtkunst im
Unterschied zur intuitiven Vorstellungsart in Musik und bildender Kunst.
Offensichtlich ist dies darin fundiert, dass Dicht- und Redekunst
(diskursive) Begriffe verwenden, während Musik und bildende Kunst die Anschauung
ansprechen (die -- wie hier deutlich wird -- nicht ausschließlich visuell
verstanden werden darf).

Oft wird angenommen, dass Diskursivität eine Eigenschaft unseres
\emph{Verstandes} ist und als solche expliziert zu werden
verlangt. \authorfullcite{Quarfood:DiscursivityandTranscendentalIdealism2012} sagt, Kant
verwende den Ausdruck \enquote{diskursiv}, um auf die Tatsache zu verweisen,
dass unser Erkenntnisvermögen über zwei Stämme verfügt -- Sinnlichkeit  und
Verstand --, die nur zusammen Erkenntnisse zu generieren vermögen. Der Verstand
sei diskursiv, insofern unser kognitives Vermögen seine eigenen Gegenstände
nicht durch Denken bereitzustellen vermag, sondern auf die Sinnlichkeit angewiesen
bleibt.\footnote{\enquote{Kant uses the term \enquote{discursivity} to refer to
this fundamental fact about our cognitive capacity, the fact that it doesn't
provide its own objects by thinking but must rely on sensibility's reception of
objects} \parencite[][143]{Quarfood:DiscursivityandTranscendentalIdealism2012}.}
Nach \authorcite{Allison:KantsTranscendentalIdealism2004} ist die Annahme der
Diskursivität menschlichen Erkennens (\emph{cognition}) die entscheidende
Voraussetzung der Vernunftkritik.\footnote{Kant’s \enquote{idealism is more
properly seen as epistemological or perhaps \enquote{metaepistemological} than
as metaphysical in nature, since it is grounded in an analysis of the discursive
nature of human cognition}
\parencite[][4]{Allison:KantsTranscendentalIdealism2004}.} Dass menschliches
Erkennen diskursiv ist, bedeute, dass es sowohl Begriffe als auch sinnliche
Anschauungen erfordere -- \enquote{cognition} meint hier die
\emph{Tätigkeit} des Verstandes. Und die Annahme, dass der transzendentale
Idealismus vom Verständnis menschlichen Erkennens als diskursiv abhänge, nennt
\authorcite{Allison:KantsTranscendentalIdealism2004} die
\enquote{\emph{discursivity
thesis}}.\footnote{\cite[Siehe][12]{Allison:KantsTranscendentalIdealism2004}.
Er schreibt weiter: \enquote{[W]e understand discursivity in the
\name[Immanuel]{Kant}ian sense, as requiring the joint contribution of
sensibility and understanding}
\parencite[][13]{Allison:KantsTranscendentalIdealism2004}. Dabei stellt er sich
-- soweit ich sehe -- nirgends explizit die Frage, wovon Diskursivität
eigentlich primär ausgesagt wird. Auch eine explizite Klärung dieses Begriffs
sucht man vergeblich.} Nach
\authorfullcite{Duesing:SpontanediskursiveSynthesis2004} wiederum bedeutet die
Lehre von der Diskursivität, \enquote{daß im schrittweisen Durchgehen durch
Mannigfaltiges Begriffe als Allgemeinheitsvorstellungen sowie
Begriffsverhältnisse gebildet werden, für die die Reziprozität von Umfang und
Inhalt gilt}\footnote{\cite[][103]{Duesing:SpontanediskursiveSynthesis2004}.}.

Doch in der \titel{Kritik der reinen Vernunft} wird Diskursivität
von Begriffen und Urteilen und auch von der Tätigkeit des Erkennens (durch
Begriffe), nicht aber von unserem Verstand ausgesagt. Dieser wird stattdessen
als Vermögen zu denken beschrieben, das nicht anzuschauen vermag. Lediglich
seine \emph{Tätigkeit} heißt in einem übertragenen Sinne diskursiv, insofern sie
durch Begriffe geschieht.\footnote{\enquote{Es gibt aber, außer der Anschauung,
keine andere Art, zu erkennen, als durch Begriffe. Also ist die Erkenntnis eines
jeden, wenigstens des menschlichen Verstandes, eine Erkenntnis durch Begriffe,
nicht intuitiv, sondern diskursiv}
\mkbibparens{\cite[][B 92\,f.,]{Kant:KritikderreinenVernunft2003}
\cite[][III: 85.13--16]{Kant:GesammelteWerke1900ff.}}.} Insofern der Verstand als diskursiv beschrieben wird, stellt
\name[Immanuel]{Kant} heraus, dass es sich um ein Vermögen der Begriffe
handelt.\footnote{So schreibt \name[Immanuel]{Kant} in der \titel{Kritik der
Urteilskraft}: \enquote{Unser Verstand ist ein Vermögen der Begriffe,
\myemph{d.\,i.} ein diskursiver Verstand}
\mkbibparens{\cite[][\S~77]{Kant:KritikderUrteilskraft2009}, \cite[][V:
406.16--17]{Kant:GesammelteWerke1900ff.}}} Und auch bei diskursiven
Erkenntnissen im engeren Sinne oder dem diskursiven Vernunftgebrauch besteht die Grundlage der
Zuschreibung von Diskursivität darin, dass ein Urteil auf der Grundlage
\emph{von Begriffen} gefällt wird. So heißen die
rationalen Erkenntnisse diskursiv, wenn es sich um rationale Erkenntnisse
aus Begriffen (statt aus der Konstruktion von Begriffe in reiner
Anschauung) handelt.\footnote{\enquote{Also ist ein
transzendentaler Satz ein synthetisches Vernunfterkenntnis nach bloßen
Begriffen, und mithin diskursiv} \mkbibparens{\cite[][B
750]{Kant:KritikderreinenVernunft2003}, \cite[][III:
474.21--23]{Kant:GesammelteWerke1900ff.}}. Siehe auch \cite[][B
747]{Kant:KritikderreinenVernunft2003}, \cite[][III:
472.25--28]{Kant:GesammelteWerke1900ff.}. Zu den unterschiedlichen Arten von
Erkenntnissen siehe Kapitel \ref{section:MuendigkeitundPhilosophie} dieser
Arbeit.} Es sind also zunächst Begriffe, von denen
wir Diskursivität im Unterschied zur Intuitivität von Anschauungen aussagen.

Eine Explikation der Unterscheidung von Anschauungen und Begriffen, die sich für
die Frage nach dem Begriff der Diskursivität als Ausgangspunkt anbietet, findet
sich in der {\jaeschelogik}. Dort heißt es:
\begin{quote}
Alle Erkenntnisse, das heißt: alle mit Bewußtsein auf ein Objekt bezogene
Vorstellungen sind entweder \ori{Anschauungen} oder \ori{Begriffe}. -- Die
Anschauung ist eine \ori{einzelne} Vorstellung (repraesentat. singularis), der
Begriff eine \ori{allgemeine} (repraesentat. per notas communes) oder
\ori{reflektierte} Vorstellung (repraesentat. discursiva).\\ Die Erkenntnis
durch Begriffe heißt das \ori{Denken} (cognitio
discursiva).\footnote{\cite[][\S~1]{Kant:ImmanuelKantsLogik1977}, \cite[][IX:
91.6--11]{Kant:GesammelteWerke1900ff.}. \name[Gottlob Benjamin]{Jäsche} hat
folgende Vorlagen aus \name[Immanuel]{Kant}s Handexemplar des Logikkompendiums
\authorcite{Meier:AuszugausderVernunftlehre1752}s nutzen können:
\enquote{\ori{cognitio est vel intuitus vel conceptus} (\ori{repraesentatio
discursiva}), Beym ersteren bin ich leidend (receptivitaet), beym zweyten
handelnd (spontaneitaet). \ori{intuitus} ist einzeln, \ori{conceptus} ist
\ori{repraesentatio per notam communem}}
\mkbibparens{\cite[][\nopp 2836]{Kant:Reflexionen1900ff.}, \cite[][XVI:
538.22--25]{Kant:GesammelteWerke1900ff.}}. \enquote{Denken ist
\ori{repraesentare per conceptus}: \ori{cognitio discursiva}}
\mkbibparens{\cite[][\nopp 2841]{Kant:Reflexionen1900ff.},
\cite[][XVI: 541.5]{Kant:GesammelteWerke1900ff.}}.}
\end{quote}
Die Bestimmung der {\jaeschelogik} ist somit sehr einfach:
Erkenntnisse unterteilen sich in Anschauungen und Begriffe;
Anschauungen sind einzelne, Begriffe allgemeine
Vorstellungen. Lediglich die lateinischen Verweise signalisieren
weitere Zusammenhänge: Begriffe sind reflektierte Vorstellungen oder
\emph{repraesentationes discursivae} und das Denken heißt als Erkennen durch Begriffe
\singlequote{diskursiv} (\emph{cognitio discursiva}). Danach scheint
\enquote{diskursiv} ein Synonym zu \enquote{reflektiert} zu sein und den Begriff
gegenüber der Anschauung durch die Eigenschaft auszuzeichnen, dass der Begriff
durch allgemeine Merkmale (\emph{per notas communes}) repräsentiert.
Hierin liegt -- wie ich zeigen werde -- die korrekte Deutung der Diskursivität:
Ein Begriff ist diskursiv, insofern er sich als reflektierte Vorstellung durch
allgemeine Merkmale auf Gegenstände bezieht.

In der \titel{Kritik der reinen Vernunft} ist die Unterscheidung zwischen
Anschauung und Begriff Teil der \singlequote{\emph{Stufenleiter}}, die freilich
primär der Frage nach der Bedeutung von \enquote{Idee} nachgeht und die
Ausdrücke \enquote{Anschauung} und \enquote{Begriff} nur in Vorbereitung mit
expliziert.\footnote{Auf diesen Punkt machte mich
\authorfullcite{Heidemann:AnschauungundBegriff2002} aufmerksam.} Dennoch handelt
es sich um die beste Grundlage, die sich in \name[Immanuel]{Kant}s selbst
publizierten Schriften findet. Sie lautet ähnlich wie die Unterscheidung in der
{\jaeschelogik}, verzichtet aber leider auf die Ausdrücke \enquote{diskursiv} und
\enquote{reflektiert}. Dafür geht sie auf die Rolle der allgemeinen Merkmale
(\emph{notae communes}) stärker ein:
\begin{quote}
Die Gattung ist \ori{Vorstellung} überhaupt (repraesentatio). Unter ihr steht
die Vorstellung mit Bewußtsein (perceptio). Eine \ori{Perzeption}, die sich
lediglich auf das Subjekt, als die Modifikation seines Zustandes bezieht, ist
\ori{Empfindung} (sensatio), eine objektive Perzeption ist \ori{Erkenntnis}
(cognitio). Diese ist entweder \ori{Anschauung} oder \ori{Begriff} (intuitus
vel conceptus). Jene bezieht sich unmittelbar auf den Gegenstand und ist
einzeln; dieser mittelbar, vermittelst eines Merkmals, was mehreren Dingen
gemein sein kann.\footnote{\cite[][B
376\,f.,]{Kant:KritikderreinenVernunft2003} \cite[][III:
249.37--250.7]{Kant:GesammelteWerke1900ff.}. Ich sehe hier ab von einer
Diskussion der Frage, ob die gesamte Einteilung schlüssig ist.
Insbesondere die Einteilung der Perzeptionen in Empfindungen und Erkenntnisse
erregt Kritik, insofern es sich doch bei der Empfindung als Wahrnehmung des
inneren Sinnes ebenfalls um eine Erkenntnis zu handeln scheint
\parencite[vgl.][79--81]{Heidemann:AnschauungundBegriff2002}.}
\end{quote}
Die Definitionen der Begriffe \enquote{Anschauung} und \enquote{Begriff} lauten
nach diesem Zitat: (a) Eine Anschauung ist eine Erkenntnis (im
\singlequote{uneigentlichen} Sinne), die (a\textsubscript{i}) \emph{einzeln} ist
und sich (a\textsubscript{ii}) \emph{unmittelbar auf einen Gegenstand bezieht}.
(b) Ein Begriff ist eine \singlequote{uneigentliche} Erkenntnis, die
(b\textsubscript{i}) \emph{allgemein} ist und sich (b\textsubscript{ii})
\emph{über allgemeine Merkmale vermittelt} auf Gegenstände bezieht. An dieser
Definition lassen sich zunächst \emph{genus} und \emph{differentia specifica}
herausstellen.

Das \emph{genus} zu Anschauungen und Begriffen lautet \enquote{Erkenntnis}: Eine
Erkenntnis ist eine bewusste Vorstellung, die sich auf einen oder mehrere Gegenstände bezieht; sie ist
objektiv, insofern sie nicht bloß eine Modifikation des Zustandes eines Subjekts
beschreibt, sondern geeignet ist, von Gegenständen etwas auszusagen.
Anschauungen und Begriffe, nicht aber Urteile sind die beiden Arten von
Erkenntnissen in diesem Sinne, so dass der Anblick von Peter, der gerade zur Tür
hereinkommt, ebenso wie die Begriffe \enquote{Pferd} oder \enquote{Säugetier}
Beispiele für Erkenntnisse in diesem Sinne sind.

Schwieriger gestaltet sich die Erläuterung der \emph{differentia specifica},
denn hier werden gleich zwei Unterschiede genannt. Das Merkmal von Anschauungen, sich unmittelbar
auf Gegenstände zu beziehen (a\textsubscript{ii}), ist von
\authorfullcite{Parsons:KantsPhilosophyofArithmetic1992} als \enquote{\emph{immediacy condition}} bezeichnet worden, die
Bedingung, sich auf einzelne Gegenstände zu beziehen (a\textsubscript{i}), als
\enquote{\emph{singularity condition}}.\footnote{\cite[Vgl.][\pno~43\,f.]{Parsons:KantsPhilosophyofArithmetic1992}.}
Analog ließen sich die Merkmale des Begriffs \enquote{Begriff} als
\enquote{\emph{universality condition}} (b\textsubscript{i}) und
\enquote{\emph{mediacy condition}} (b\textsubscript{ii}) bezeichnen.
\name[Immanuel]{Kant} erwähnt also gleich zwei Charakteristika von
Anschauungen gegenüber Begriffen (und \emph{vice versa}).\footnote{\authorfullcite{Hanna:KantandtheFoundationsofAnalyticPhilosophy2001}
zählt insgesamt sogar fünf verschiedene Merkmale auf, die \name[Immanuel]{Kant}
an verschiedenen Stellen zur Bestimmung des Begriffs \enquote{Anschauung}
anführe. Neben den Merkmalen der Unmittelbarkeit und der Singularität verweise
\name[Immanuel]{Kant} darauf, dass Anschauungen stets sinnlich seien, dass sie
vor allem Denken gegeben werden können, und dass sie abhängig sind von der
Gegenwart des Gegenstandes
\parencite[Vgl.][195]{Hanna:KantandtheFoundationsofAnalyticPhilosophy2001}.
Damit nennt er freilich auch Eigenschaften, die ausschließlich auf
\emph{sinnliche} Anschauungen zutreffen, und nicht nur solche, die den Begriff
\enquote{Anschauung} zu explizieren helfen.}

Die Frage, wie sich die verschiedenen Bestimmungen der Begriffe
\enquote{Anschauung} und \enquote{Begriff} zueinander verhalten, wird in der
\name[Immanuel]{Kant}forschung breit diskutiert.\footnote{Siehe etwa
\cite{Hintikka:OnKantsNotionofIntuition1969},
\cite{Hintikka:KantianIntuitions1972},
\cite{Thompson:SingularTermsandIntuitioninKantsEpistemology1972},
\cite{Howell:IntuitionSynthesisandIndividuationintheCritiqueofPureReason1973},
\cite[][194--211]{Hanna:KantandtheFoundationsofAnalyticPhilosophy2001}, sowie
zuletzt \cite[][35--53]{Gruene:BlindeAnschauung2009}.} Da die diesbezügliche
Forschung sich vor allem auf \name[Immanuel]{Kant}s Philosophie der Mathematik
und die Frage, wie Erkenntnis durch reine Anschauung möglich ist, bezieht, steht
dabei der Begriff der Anschauung mit seinen beiden Merkmalen im Zentrum der
Aufmerksamkeit.\footnote{Eine Ausnahme bezüglich der Fokussierung der
Mathematik stellt \textcite[vgl.][35--53]{Gruene:BlindeAnschauung2009} dar, die
jedoch ohnehin den Begriff der Anschauung selbst thematisiert und den des
Begriffs daher nur nebenbei mit thematisiert. Des weiteren bespricht
\authorfullcite{Thompson:SingularTermsandIntuitioninKantsEpistemology1972} den
Begriff der Anschauung mit Blick auf \emph{empirische} Anschauungen
\parencite[vgl.][314]{Thompson:SingularTermsandIntuitioninKantsEpistemology1972}.}
Ist eine Anschauung einzeln, weil sie sich unmittelbar auf den Gegenstand
bezieht? Oder sind es zwei verschiedene Bestimmungen, so dass es einzelne
Vorstellungen gibt, die sich mittelbar auf den Gegenstand beziehen und \emph{a
fortiori} keine Anschauungen sind? Analog lässt sich freilich fragen: Ist der
Begriff eine allgemeine Vorstellung, weil er sich vermittelt über allgemeine
Merkmale auf Gegenstände bezieht? Oder bezieht er sich über allgemeine Merkmale
auf Gegenstände, weil er eine allgemeine Vorstellung ist? Kann eine allgemeine
Vorstellung sich auch unmittelbar auf Gegenstände beziehen und somit kein Begriff sein? Können sich
Erkenntnisse vermittelt über Merkmale auf Gegenstände beziehen, ohne allgemein
zu sein?\footnote{Weiter ließe sich fragen, welchen Status die Formulierung hat, dass sich
Begriffe vermittelt über \emph{allgemeine} Merkmale auf Gegenstände beziehen? Beziehen sich nur
Begriffe durch Merkmale auf Gegenstände? Oder haben nur Begriffe Merkmale, die
\enquote{mehreren Dingen gemein} sind, Anschauungen aber andere Merkmale?
\authorfullcite{Gruene:BlindeAnschauung2009} legt dar, dass auch Anschauungen
Merkmale haben, durch die sie sich auf Gegenstände beziehen. Wenn Ingrid Max am
Gang erkennt, dann ist Max' Gang ein Merkmal, das als Erkenntnisgrund fungiert.
Aber nur Begriffe verfügen ausschließlich über Merkmale, die mehreren Dingen
zukommen, während Anschauung sich durch Merkmale auf Gegenstände beziehen, die
diesen exklusiv zukommen. Max' Gang ist eben deshalb ein Merkmal von Max, weil
er selbst singulär ist. Es ist kein Begriff -- etwa des Gehens -- anhand dessen
Ingrid Max erkennt. \name[Immanuel]{Kant} spreche hier von \emph{intuitiven}
Merkmalen als Inhalten von Anschauungen, während Begriffe \emph{diskursive}
Merkmale enthielten \parencite[vgl.][65--71]{Gruene:BlindeAnschauung2009}. Dabei
fällt jedoch auf, dass eine solche Unterscheidung nicht in Schriften vorkommt,
die \name[Immanuel]{Kant} zur Publikation freigegeben hat
\parencite[vgl.][62]{Prien:KantsLogikderBegriffe2006}. Da es mir hier nicht
primär um den Begriff der Anschauung, sondern um den des Begriffs -- resp. der
Diskursivität -- geht, kann ich auf eine abschließende Beantwortung solcher
Fragen an dieser Stelle verzichten.}


Dem ersten Anschein nach handelt es sich nun bei der \emph{immediacy condition}
um eine von der \emph{singularity condition} unabhängige Bedingung. Das wäre
natürlich problematisch, denn dann definierte \name[Immanuel]{Kant} den
Unterschied zwischen Begriffen und Anschauungen durch zwei voneinander
unabhängige \emph{definientia}. Nun gilt als unstrittig, dass aus der
Unmittelbarkeit einer Vorstellung folgt, dass sie sich auf Einzelnes bezieht;
denn zumindest die menschliche Anschauung ist stets eine Anschauung einzelner
Dinge, wir schauen keine \emph{universalia}
an.\footnote{Siehe etwa \cite[][45]{Parsons:KantsPhilosophyofArithmetic1992}.
Dies konstatiert auch
\authorcite{Wolff:Discursuspraeliminarisdephilosophiaingenere1996} (siehe dazu
das Zitat in Kapitel \ref{Zitat:Wolff:ErfahrungnurvoneinzelnenDingen} auf
S.~\pageref{Zitat:Wolff:ErfahrungnurvoneinzelnenDingen} dieser Arbeit).} Also
bleiben nur zwei Optionen: Entweder beide Bestimmungen sind identisch -- die \emph{singularity
condition} ist genau dann erfüllt, wenn die \emph{immediacy condition} erfüllt ist --, oder
die \emph{immediacy condition} stellt eine Spezifizierung der \emph{singularity
condition} dar, weil \name[Immanuel]{Kant} denkt, dass auch Begriffe sich
mitunter auf einzelne Gegenstände beziehen können, jedoch nicht unmittelbar. Die
erste Position wird von
\authorfullcite{Hintikka:KantontheMathematicalMethod1992}\footnote{Die
einschlägigen Publikationen sind: \cite{Hintikka:OnKantsNotionofIntuition1969},
\cite{Hintikka:KantianIntuitions1972}, sowie \cite{Hintikka:KantontheMathematicalMethod1992}.} und
\authorfullcite{Thompson:SingularTermsandIntuitioninKantsEpistemology1972}\footnote{Siehe
\cite{Thompson:SingularTermsandIntuitioninKantsEpistemology1972}.} vertreten,
die zweite von
\authorfullcite{Parsons:KantsPhilosophyofArithmetic1992}\footnote{Vgl. 
\cite{Parsons:KantsPhilosophyofArithmetic1992}. Siehe hierzu insgesamt
\cite[Vgl.][237]{VanCleve:ProblemsfromKant1999}.}.

Das Problem der Lesart, wonach \emph{singularity condition} und \emph{immediacy
condition} auf dasselbe hinauslaufen, liegt darin, dass damit jede singuläre
Repräsentation von Individuen als Anschauung zählte. Die Bezeichnung
\enquote{höchster Berg der Alpen} bezieht sich eindeutig auf einen einzelnen
Gegenstand (den Mont Blanc), aber ebenso eindeutig handelt es sich nicht um eine
Anschauung, denn definite Kennzeichnungen sind keine Anschauungen. Dies wiederum
wird auch von den meisten Interpreten so gesehen, die die These von der
Identität beider Bestimmungen  zurückweisen.\footnote{Siehe etwa
\cite[][\pno~207\,f.,]{Howell:IntuitionSynthesisandIndividuationintheCritiqueofPureReason1973}
sowie \cite[][70]{Parsons:KantsPhilosophyofArithmetic1992}.} Lediglich
\authorfullcite{Hintikka:KantianIntuitions1972} akzeptiert die
Schlussfolgerung, dass nach \name[Immanuel]{Kant} jede singuläre Bezugnahme als Anschauung zähle.
Aber das ist ein hoher Preis, der außerdem mit \name[Immanuel]{Kant}s Angabe
konfligiert, Begriffe könnten auch einzeln \singlequote{gebraucht}
werden.\footnote{\enquote{Es ist eine bloße Tautologie, von
allgemeinen oder gemeinsamen Begriffen zu reden; -- ein Fehler, der sich auf
eine unrichtige Einteilung der Begriffe in \ori{allgemeine}, \ori{besonderen}
und \ori{einzelne} gründet. Nicht die Begriffe selbst -- nur \ori{ihr Gebrauch}
kann so eingeteilt werden}
\mkbibparens{\cite[][\S~1]{Kant:ImmanuelKantsLogik1977}, \cite[][IX:
91.17--20]{Kant:GesammelteWerke1900ff.}}; siehe dazu auch
\cite[][B 96\,f.,]{Kant:KritikderreinenVernunft2003}
\cite[][III: 87.20--88.2]{Kant:GesammelteWerke1900ff.}.} Für den
einzelnen Gebrauch von (allgemeinen) Begriffen scheint gerade die
Bezugnahme auf einzelne Gegenstände in (gelingenden\footnote{Eine definite Kennzeichnung
  \singlequote{gelingt}, wenn die Beschreibung tatsächlich auf genau
  einen Gegenstand zutrifft und nicht wie \enquote{der gegenwärtige
    König von Frankreich} gar keinen Gegenstand herausgreift oder wie
  \enquote{Donald Ducks Neffe} auf zu viele Gegenstände zutrifft.})
definiten Kennzeichnungen paradigmatisch zu sein.

Die zweite Lesart, wonach die \emph{immediacy condition} als Spezifizierung der
\emph{singularity condition} aufzufassen ist, führt zu einer anderen
Überlegung: Wir müssten dann sagen, dass nur die Konjunktion aus beiden
Bedingungen den Begriff der Anschauung definiert. Aber wie verhält es sich dann
mit dem Begriff des Begriffs und seinen beiden Merkmalen? \emph{Prima facie}
müssten \emph{universality condition} und \emph{mediacy condition} ihn
disjunktiv definieren. Aber wenn aus der Unmittelbarkeit einer Bezugnahme folgen
soll, dass sie auf einen einzelnen Gegenstand geht, aus dem Erfülltsein der
\emph{immediacy condition} also folgt, dass auch die \emph{singularity
condition} erfüllt ist, dann gilt umgekehrt, dass aus der \emph{universality condition} auf
die \emph{mediacy condition} geschlossen werden kann.  Es reicht dann aber
völlig aus, Begriffe mittels der
\emph{mediacy condition} und Anschauung mittels der \emph{immediacy condition}
zu bestimmen. \emph{Singularity condition} und \emph{universality condition}
wären verzichtbar und nicht als Bestandteil der Definition, sondern als Folge
derselben aufzufassen. Dass Anschauungen singuläre Vorstellungen sind, Begriffe hingegen
allgemeines repräsentieren, folgt aus dieser Explikation (wenn auch nur mit
Einschränkungen).

Dass es die Bedingungen der Unmittelbarkeit \emph{respective} der Mittelbarkeit
sind, die objektive Vorstellungen zu Anschauungen \emph{respective} Begriffen
machen, bestätigt auch die Vermutung, dass der Ausdruck \enquote{diskursiv} die
Eigenschaft von Begriffen beschreibt, sich mittels allgemeiner Merkmale auf
Gegenstände zu beziehen. Ein Textbeleg findet sich in der Angabe, dass es die
Erkenntnis durch Merkmale ist, die unser Erkennen zu einem diskursiven mache:
\begin{quote}
Das menschliche Erkenntnis ist von Seiten des Verstandes \ori{diskursiv}; d.\,h.
es geschieht durch Vorstellungen, die das, was mehreren Dingen gemein ist, zum
Erkenntnisgrunde machen, mithin durch \ori{Merkmale}, als
solche.\footnote{\cite[][A 84\,f.,]{Kant:ImmanuelKantsLogik1977}
\cite[][IX: 58.9--12]{Kant:GesammelteWerke1900ff.}. Dabei handelt es sich um
die wörtliche Übernahme einer Anmerkung \name[Immanuel]{Kant}s im
\titel{Auszug aus der Vernunftlehre}
\mkbibparens{\cite[siehe][\nopp 2288]{Kant:Reflexionen1900ff.},
\cite[][XVI: 300.8--10]{Kant:GesammelteWerke1900ff.}}. Siehe auch
\cite[][\nopp 2281]{Kant:Reflexionen1900ff.}, \cite[][XVI:
298.7--10]{Kant:GesammelteWerke1900ff.}:
\enquote{Wir erkennen Dinge nur durch Merkmale; das heißt eben erkennen, welches
von kennen herkommt. Denn der Verstand ist ein Vermögen zu denken, d.\,i.
discursiv durch Begriffe zu erkennen; Begriffe aber sind Merkmale von
allgemeinem Gebrauche.}}
\end{quote}
Der Begriff der Intuitivität verweist entsprechend auf die Unmittelbarkeit, der
der Diskursivität auf die Mittelbarkeit, die sich auf allgemeine Merkmale
stützt. Dass unser Verstand diskursiv und nicht intuitiv ist (dass er ein
Vermögen zu denken oder der Begriffe, nicht aber ein Vermögen anzuschauen ist),
heißt \emph{a fortiori}: Unser Verstand kann sich nicht unmittelbar, sondern nur
vermittelt auf Gegenstände beziehen. Die Vermittlung geschieht dabei über
allgemeine Merkmale. Damit sich der Verstand jedoch überhaupt auf Gegenstände
beziehen kann, muss es Vorstellungen geben, die sich unmittelbar auf Gegenstände
beziehen. Sonst gerieten wir in einen Regress, der uns nur zu immer weiteren
Vorstellungen, niemals aber zu Gegenständen führte. Daher kann es kein Denken
ohne Anschauungen geben. Wir hätten nur Begriffe ohne Inhalt, das hieße ohne
Gegenstandsbezug.\footnote{Eine Schwierigkeit der Darstellung
  \name[Immanuel]{Kant}s besteht freilich darin, dass sich Begriffe
  zunächst auf \emph{allgemeine} Merkmale und damit wiederum auf
  andere Begriffe beziehen, die wiederum derselben Bedingung
  unterliegen. Somit ließe sich der infinite Regress allem Anschein
  nach nicht vermeiden. \name[Immanuel]{Kant} müsste m.\,E. einräumen,
  dass sich Begriffe auch vermittelt über Anschauungen auf Gegenstände
  beziehen können. Siehe hierzu auch weiter unten
  Anm. \ref{Fussnote:InfiniterRegressBegriffeueberMerkmale} auf Seite
  \pageref{Fussnote:InfiniterRegressBegriffeueberMerkmale}.}

Statt der \emph{singularity} und \emph{universality
condition} müssen wir also die Mittelbarkeit von Begriffen und Unmittelbarkeit
von Anschauungen verstehen, um Einsicht in das Wesen von Anschauungen und
Begriffen zu erhalten.\footnote{Die \emph{singularity condition} ist in der Tat
wenig problematisch. Eine Anschauung ist stets die Anschauung von diesem oder jenem
Gegenstand, eine Anschauung von Max bezieht sich eben auf Max und auf niemanden
sonst, während Begriffe wie \enquote{Junggeselle} sich nicht auf diesen oder
jenen Junggesellen -- etwa auf Max -- beziehen, sondern allgemein auf
Junggesellen. Das scheinbare Problem, dass eine Anschauung ja auch eine
Anschauung von mehrere Gegenständen sein kann (z.\,B. eine Anschauung von Ingrid und Max) lässt sich
einfach dadurch lösen, dass wir sagen, sie beziehe sich in solchen Fällen auf
die mereologische Summe dieser Gegenstände
\parencite[vgl.][47]{Gruene:BlindeAnschauung2009}. Schwieriger als
\emph{singularity condition} und \emph{universality condition} sind die
\emph{immediacy condition} und die \emph{mediacy condition} zu verstehen, da
hier notwendiger Weise zu klären ist, \emph{durch was} Begriffe als solche
vermittelt sind.}
Was also besagen \emph{mediacy} und \emph{immediacy condition}?
\authorfullcite{Hanna:KantandtheFoundationsofAnalyticPhilosophy2001} sagt,
Anschauungen seien nach \name[Immanuel]{Kant} di\-rekt-re\-feren\-tiell, während
sich Begriffe vermittelt über \emph{Beschreibungen} auf ihre Gegenstände
beziehen.\footnote{\enquote{So the Kantian distinction between conceptual
(mediate) reference and intuitive (immediate) reference is most accurately
construed as the difference between, on the one hand, indirect or
description-determined reference to an object, and, on the other, direct or
non-description-determined reference to an object. More plainly put, intuitional
reference is \ori{direct reference}}
\parencite[][197]{Hanna:KantandtheFoundationsofAnalyticPhilosophy2001}.}
Wenn wir von dem höchsten Berg der Alpen sprechen, dann
bezieht sich dieser Ausdruck mittels einer Beschreibung auf den Mont Blanc. Die
Bezugnahme ist vermittelt über die Merkmale, dass es sich um einen Berg handelt,
der zu den Alpen gehört und höher ist, als alle anderen Berge der Alpen. Eine
Anschauung des Mont Blanc, die wir haben, während wir direkt davor stehen,
bezieht sich hingegen ohne solche Merkmale direkt auf den Berg. Es ist dann
naheliegend, den Begriff der Anschauung über die bei endlichen Wesen stets involvierte
Sinnlichkeit bestimmen zu wollen. Ein solches Vorgehen findet sich bei
\authorfullcite{Willaschek:DertranszendentaleIdealismusunddieIdealitaetvonRaumundZeit1997}:
\enquote{Anschauungen beziehen sich nicht durch Merkmale auf ihren Gegenstand,
sondern kausal und insofern
unmittelbar.}\footnote{\cite[][548]{Willaschek:DertranszendentaleIdealismusunddieIdealitaetvonRaumundZeit1997}.
\enquote{Anschauungen, so kann man zusammenfassen, sind Vorstellungen, die auf eine
Affektion der Sinnlichkeit durch einen Gegenstand zurückgehen, die über einen
qualitativen Gehalt verfügen und die sich unmittelbar auf einen einzelnen
Gegenstand beziehen}
\parencite[][\pno~545\,f.]{Willaschek:DertranszendentaleIdealismusunddieIdealitaetvonRaumundZeit1997}.}

Nun fällt auf, dass \name[Immanuel]{Kant} in der Stufenleiter der \titel{Kritik
der reinen Vernunft} wie auch in der von \name[Gottlob Benjamin]{Jäsche} übernommenen
Passage darauf verzichtet, die Sinnlichkeit für die Definition
heranzuziehen.\footnote{Dies fiel schon \authorcite{Frege:DieGrundlagenderArithmetik1988} auf, der jedoch
anzunehmen scheint, es fänden sich in {\jaeschelogik} und
\titel{Kritik der reinen Vernunft} unterschiedliche Anschauungsbegriffe
\parencite[vgl.][27]{Frege:DieGrundlagenderArithmetik1988}.}
Dies ist an sich schon Grund genug, einer definitorischen Verbindung von
Anschauungen und Sinnlichkeit skeptisch gegenüber zu stehen. Aber es gibt
weitere Gründe, die eine solche Verbindung explizit ausschließen: Wären
Anschauungen \emph{per definitionem} sinnlich, dann könnte es weder reine
Anschauungen noch intellektuelle Anschauungen geben -- sie wäre in sich
widersprüchlich und schon begrifflich
ausgeschlossen.\footnote{\authorfullcite{Willaschek:DertranszendentaleIdealismusunddieIdealitaetvonRaumundZeit1997}
beschränkt seine Überlegungen bewusst auf das endliche Denken
\parencite[vgl.][\pno~547,
Anm.]{Willaschek:DertranszendentaleIdealismusunddieIdealitaetvonRaumundZeit1997},
mir scheint damit aber auch die begriffliche Durchdringung des endlichen Denkens
behindert zu werden, indem das, was zum Begriff der Anschauung gehört, und das,
was die weitere Bestimmung derselben als sinnlicher ausmacht, nicht eigens
herausgearbeitet wird.}

Das Wort \enquote{Anschauung} suggeriert eine Verbindung mit dem Begriff der
Sinnlichkeit, insofern \enquote{schauen} die sinnliche Wahrnehmung mittels des
Gesichtssinns
bezeichnet.\footnote{\cite[Vgl.][38]{Hintikka:OnKantsNotionofIntuition1969}.}
Aber zumindest in \name[Immanuel]{Kant}s philosophischer Terminologie ist die
Verbindung nicht so eindeutig.
\enquote{Anschauung} und \enquote{intuitiv} haben stattdessen die
Unmittelbarkeit zu ihrem zentralen Merkmal, was an die noch heute erhaltene
Bedeutung von \enquote{intuitiv} verweist, die sich etwa in der
Gegenüberstellung von Intuition und Demonstration bei
\authorcite{Descartes:OeuvresdeDescartes1983} und anderen Autoren der Neuzeit
zeigt.\footnote{Auch \authorfullcite{Hintikka:OnKantsNotionofIntuition1969}
bemerkt, dass im relevanten Zeitraum der Begriff der Anschauung
(\enquote{\emph{intuitus}}) nicht zwingend mit Sinnlichkeit, sondern eher mit
Unmittelbarkeit verbunden wurde. Die Verbindung zum Begriff der Sinnlichkeit sei
über das Merkmal der Individualität zustande gekommen
\parencite[vgl.][40--44]{Hintikka:OnKantsNotionofIntuition1969}.}
Bei \emph{uns} beziehen sich Anschauungen auf ihre Gegenstände mittels einer
kausalen Verknüpfung. Wir werden von Gegenständen nur dann affiziert, wenn diese
in bestimmter Hinsicht kausal auf uns einwirken. Es ist naheliegend davon
auszugehen, dass jeder Form des Affiziertwerdens eine solche Kausalrelation
zugrunde liegt. Aber nicht jeder denkbaren Anschauung liegt eine solche
Relation zugrunde; sonst wäre eine intellektuelle Anschauung nicht
denkbar.\footnote{Denkbar wäre es natürlich, einer intellektuellen Anschauung
läge ebenfalls eine Kausalrelation zugrunde, nur eben in die andere Richtung. Eine Anschauung
bezöge sich dann auf Gegenstände, weil sie die Gegenstände hervorbringt; sie
hätte einen unmittelbaren Gegenstandsbezug als produktive Anschauung.
Dies passt insbesondere auch zu den Überlegungen in \S~14 der \titel{Kritik der
reinen Vernunft} \mkbibparens{\cite[vgl.][B
124\,f.,]{Kant:KritikderreinenVernunft2003} \cite[][III:
104.6--17]{Kant:GesammelteWerke1900ff.}} sowie mit explizitem Bezug auf eine
vorgestellte göttliche Erkenntnis im Brief an \name[Marcus]{Herz} vom 21.
Februar 1772 \mkbibparens{\cite[vgl.][X:
130.6--21]{Kant:GesammelteWerke1900ff.}}.}

Es muss keine Gemeinsamkeit aller denkbaren unmittelbaren Bezugnahmen
geben. Die einzige garantierte Gemeinsamkeit von Anschauungen besteht darin,
dass ihr Bezug auf Gegenstände nicht über allgemeine Merkmale vermittelt ist. Hilfreicher als der Versuch, die
Unmittelbarkeit von Anschauungen über kausale Verbindungen zu erläutern, ist
daher die Untersuchung der Mittelbarkeit des Gegenstandsbezugs von Begriffen. Die Unmittelbarkeit oder Mittelbarkeit von etwas auszusagen,
bedarf immer der Qualifizierung, denn nichts ist \emph{schlechthin} unmittelbar
oder vermittelt, sondern immer nur in dieser oder jener Hinsicht.\footnote{Siehe
dazu \cite{Sellars:EmpiricismandthePhilosophyofMind1997}.} Anschauungen beziehen sich
beispielsweise vermittelst der Sinne auf Gegenstände. Hier aber geht es darum,
ob eine Vorstellung sich nur \enquote{vermittelst eines Merkmals, was mehreren
Dingen gemein sein kann,} auf Gegenstände bezieht.
\authorfullcite{Allison:KantsTranscendentalIdealism2004} sagt unter Rekurs auf
eine Beobachtung von \authorfullcite{Longuenesse:KantandtheCapacitytoJudge1998}, dass Begriffe als diskursive
Regeln gelten, weil sie begriffliche Zusammenhänge
behaupten.\footnote{\cite[Vgl.][79]{Allison:KantsTranscendentalIdealism2004}:
\enquote{On the other hand, concepts also serve as discursive rules affirming
conceptual connections.} \enquote{[I]t is the function of concepts as discursive rules
that accounts for their role in judgment. To form the concept of body as a
discursive representation is to think together the features of extension,
impenetrability, figure, and so forth, as marks or components of the concept
that are in some sense \punkt{} necessarily connected to it}
\parencite[][79]{Allison:KantsTranscendentalIdealism2004}.}
Nach \authorfullcite{Longuenesse:KantandtheCapacitytoJudge1998} gibt
es zwei Arten, \name[Immanuel]{Kant}s Redeweise von Begriffen als
Regeln zu verstehen: Einerseits verstehe \name[Immanuel]{Kant}
darunter Regeln der Synthesis des Mannigfaltigen in der Anschauung,
andererseits aber auch \emph{diskursive Regeln}, die die Merkmale des
Begriffs angeben und sagen, unter welchen Begriffen ein Gegenstand des weiteren
fällt.\footnote{\cite[Vgl.][48--50]{Longuenesse:KantandtheCapacitytoJudge1998}.
\enquote{The concept is a rule insofar as it is the consciousness of
  the unity of an act of sensible synthesis or the consciousness of
  the procedure for generating a sensible intuition. This first sense
  of rule anticipates what Kant, in the Schematism of the Pure
  Concepts of the Understanding, calls a schema. But the concept is a
  rule also in another, discursive sense. It is a rule in that
  thinking an object under a concept provides a reason to predicate of
  this object the marks that define the
  concept} \parencite[][50]{Longuenesse:KantandtheCapacitytoJudge1998}.}


Nach \S~1 der {\jaeschelogik} ist eine Erkenntnis diskursiv, insofern
sie \emph{reflektiert} ist. Denn diskursiv ist eine Erkenntnis durch Begriffe
und diese wiederum sind zunächst als reflektierte Vorstellungen zu verstehen.
Belege dafür finden sich auch im Handexemplar des
\authorcite{Meier:AuszugausderVernunftlehre1752}schen
Logiklehrbuchs: Für seine Logikvorlesungen notiert er: \enquote{Ein
  Begrif ist eine reflectierte 
Vorstellung.}\footnote{\cite[][\nopp 2834]{Kant:Reflexionen1900ff.},
\cite[][XVI: 536.2]{Kant:GesammelteWerke1900ff.}. Danach besteht die logische
Form eines Begriffs in der Reflexion, wodurch er eine allgemeine Vorstellung
werde \mkbibparens{\cite[vgl.][\nopp 2851]{Kant:Reflexionen1900ff.},
\cite[][XVI: 546.14--16]{Kant:GesammelteWerke1900ff.}}.} Und gegen
\authorcite{Meier:AuszugausderVernunftlehre1752}, der (allgemeine) Begriffe als
Ergebnis der Abstraktion ansieht,\footnote{\enquote{Alle Begriffe, welche
durch die logische Absonderung gemacht werden, sind \ori{abgesonderte} oder
\ori{abstracte Begriffe} (conceptus abstractus, notio). Begriffe, die nicht
abgesondert sind, heissen \ori{einzelne Begriffe} (conceptus singularis, idea)}
\mkbibparens{\cite[][\S~260]{Meier:AuszugausderVernunftlehre1752},
\cite[][XVI: 551.25--28]{Kant:GesammelteWerke1900ff.}}.} wendet er ein:
\enquote{Durch abstraction werden keine Begriffe, sondern durch reflexion:
entweder, wenn der Begrif gegeben ist, nur die Form und heißt reflectirter, oder
selbst der Begrif: reflectirender.}\footnote{\cite[][\nopp 2865]{Kant:Reflexionen1900ff.},
\cite[][XVI: 552.9--11]{Kant:GesammelteWerke1900ff.}.} Wir finden den Begriff
der Reflexion in der Redeweise von der \emph{reflektierenden} (im Gegensatz zur
bestimmenden) Urteilskraft, die nicht gegebene Begriffe anwendet, sondern
Begriffe bildet.\footnote{\enquote{Urteilskraft überhaupt ist das Vermögen, das Besondere als enthalten unter dem
Allgemeinen zu denken. Ist das Allgemeine (die Regel, das Prinzip, das Gesetz)
gegeben, so ist die Urteilskraft, welche das Besondere darunter subsumiert,
{\punkt} \ori{bestimmend}. Ist aber nur das Besondere gegeben, wozu sie das
Allgemeine finden soll, so ist die Urteilskraft bloß
\ori{reflektierend}} \mkbibparens{\cite[][B
xxv\,f.,]{Kant:KritikderUrteilskraft2009}
\cite[][V: 179.19--26]{Kant:GesammelteWerke1900ff.}}. In der \titel{Ersten
Einleitung} lesen wir: \enquote{\ori{Reflectiren} (Überlegen) aber ist:
gegebene Vorstellungen entweder mit andern, oder mit seinem
Erkenntnißvermögen, in Beziehung auf einen dadurch möglichen Begrif, zu
vergleichen und zusammen zu halten}
\mkbibparens{\cite[][16]{Kant:ErsteEinleitungindieenquoteKritikderUrteilskraft2009},
\cite[][XX: 211.14--16]{Kant:GesammelteWerke1900ff.}}. In der \titel{Kritik der
reinen Vernunft} spricht \name[Immanuel]{Kant} des weiteren von \enquote{transzendentaler} und
\enquote{logischer Reflexion}. Reflexion wird hier zunächst allgemein mit
\enquote{Überlegung} übersetzt, um dann zu sagen, dass die transzendentale
Überlegung darin besteht zu unterscheiden, ob die Vorstellungen
(\emph{respective} ihre \singlequote{Vergleichung}) zum reinen Verstand oder zur
sinnlichen Anschauung gehört. Dagegen betrachte die logische Reflexion lediglich
die Vorstellungen selbst, ohne darauf zu achten, ob sie zum reinen Verstand oder
zur sinnlichen Anschauung gehören \mkbibparens{\cite[vgl.][B
316--319]{Kant:KritikderreinenVernunft2003}, \cite[][III:
214.33--216.28]{Kant:GesammelteWerke1900ff.}}. Im Falle der Begriffsbildung
haben wir es -- wenn überhaupt mit einer Reflexion im Sinne der \titel{Kritik
der reinen Vernunft} -- mit einer logischen Überlegung zu tun.}
In der {\jaeschelogik} sowie in \singlequote{Reflexionen} und
Vorlesungsnachschriften, nicht aber in eigenständigen Publikationen finden sich
konkretere Ausführungen zu einer Theorie der Begriffsbildung, die auch den
Begriff der Reflexion weiter explizieren. Danach gehören drei Operationen des
Verstandes dazu, einen (empirischen) Begriff zu bilden:
\begin{quote}
Die logischen Verstandes-Actus, wodurch Begriffe ihrer Form nach erzeugt werden,
sind:
\begin{nummerierung}
\item die \ori{Komparation}, d.\,i. die Vergleichung der Vorstellungen unter
einander im Verhältnisse zur Einheit des Bewußtseins;
\item die \ori{Reflexion}, d.\,i. die Überlegung, wie verschiedene Vorstellungen
in Einem Bewußtsein begriffen sein können; und endlich
\item die \ori{Abstraktion} oder die Absonderung alles übrigen, worin die
gegebenen Vorstellungen sich
unterscheiden.\footnote{\cite[][\S~6]{Kant:ImmanuelKantsLogik1977},
\cite[][IX: 94.20--27]{Kant:GesammelteWerke1900ff.}. Siehe auch
\cite[][\nopp 2854]{Kant:Reflexionen1900ff.}, \cite[][XVI:
547.8--13]{Kant:GesammelteWerke1900ff.}. Es ergäbe eine krude Vorstellung,
verstünden wir solche Überlegungen als (quasi psychologische) Beschreibung unseres tatsächlichen
Begriffserwerbs. Doch das ist nicht nötig; \name[Immanuel]{Kant} beschreibt
hier nicht den Vorgang des Erwerbs eines Begriffs, sondern gedankliche
Operationen, derer ein Subjekt, das über Begriffe verfügt, \emph{fähig} sein
muss. \cite[Vgl.][\pno~82\,f.:]{Stuhlmann-Laeisz:KantsLogik1976} \enquote{Diese
Erklärung des Ursprungs von Begriffen will \name[Immanuel]{Kant} nicht verstanden wissen als eine
Beschreibung des wirklichen Vorgangs bei der Erwerbung eines Begriffs durch ein
denkendes Subjekt. Eine solche Beschreibung würde ja in die empirische
(Denk-)Psychologie, nicht aber in die formale Logik gehören. Der Anspruch dieser
Theorie ist vielmehr, die Bedingungen der Möglichkeit, Begriffe zu besitzen,
aufgezeigt zu haben: Ein Verstand kann genau dann Begriffe besitzen, wenn er der
drei betreffenden logischen Akte fähig ist. Nicht behauptet wird, daß wir de
facto auf die beschriebene Weise in den Besitz jedes unserer Begriffe
gelangen.}}
\end{nummerierung}
\end{quote}
Das Beispiel, das zur Erläuterung dient, lautet folgendermaßen: Wir sehen drei
verschiedene Bäume, beispielsweise eine Linde, eine Weide und eine Fichte. Wenn
wir diese vergleichen (komparieren), dann finden wir Gemeinsamkeiten und
Unterschiede. Um einen allgemeinen Begriff zu bilden, betrachten wir die
Gemeinsamkeiten: Sie alle verfügen über einen Stamm, Äste und Blätter. Diesen
Vorgang nennt \name[Immanuel]{Kant} Reflexion. Außerdem abstrahieren wir von den
Unterschieden in Größe und Gestalt. Was wir erhalten ist der Begriff des Baumes
mit seinen Merkmalen des Stammes, der Äste und der Blätter.

Dabei richten sich solche Ausführungen primär gegen die Auffassung,
(allgemeine) Begriffe seien \singlequote{abstrakte} Vorstellungen, die aus einer
Operation des Abstrahierens hervorgehen.
\authorfullcite{Meier:AuszugausderVernunftlehre1752} artikuliert eine solche
Auffassung, die schon daran scheitern muss, dass sie voraussetzt, Anschauungen
seien einzelne Begriffe. Wir müssen zwar auch in der Lage sein, von
Unterschieden zu abstrahieren. Aber diese Operation liefert uns noch keine
allgemeinen Merkmale. Dadurch, dass ich bei der Anschauung einer Fichte von der
Form der Äste und vielem anderen absehe, erhalte ich noch lange nicht den
Begriff eines Baumes. Dazu muss ich aktiv nach Gemeinsamkeiten Ausschau halten,
also reflektieren: überlegen, wie die Vorstellungen von Fichten, Linden und
Weiden in einem einzigen Bewusstsein vereinigt sein können.

Doch warum heißen Begriffe \enquote{\emph{diskursive}} Vorstellungen, wenn ihr
Wesen doch darin besteht, sich als \emph{reflektierte} Vorstellungen mittels
allgemeiner Merkmale auf ihre Gegenstände zu beziehen?
Nach der {\jaeschelogik} ist der Begriff als \emph{reflektierte} Vorstellung
eine \emph{repraesentatio
discursiva},\footnote{\cite[Vgl.][\S~1]{Kant:ImmanuelKantsLogik1977},
\cite[][IX: 91.10]{Kant:GesammelteWerke1900ff.}. Allerdings findet sich hierfür
keine direkte Vorlage in \name[Immanuel]{Kant}s Handexemplar des
\titel{Auszugs aus der Vernunftlehre}.} seine Diskursivität ist also daran
gebunden, dass er -- neben Komparation und Abstraktion -- auf der geistigen
Operation der \emph{Reflexion} beruht. Doch warum soll \enquote{reflektierte
Vorstellung} auf Latein als \enquote{\emph{representatio discursiva}} übersetzt
werden?

Ein Begriff ist reflektiert, weil er sich über allgemeine
Merkmale auf Gegenstände bezieht. Solche Merkmale sind Eigenschaften, die allen
Gegenständen zukommen, die unter den Begriff fallen. Sie entstammen der
Operation der Reflexion. Nun nennt \name[Immanuel]{Kant} Begriffe auch
\emph{Regeln}, worunter -- wie
\authorfullcite{Longuenesse:KantandtheCapacitytoJudge1998} feststellt --
zweierlei gemeint sein kann: Ein Begriff kann als Regel der Synthesis des
Mannigfaltigen der Anschauung oder als \singlequote{diskursive Regel} aufgefasst
werden.\footnote{\enquote{The concept is a rule insofar as it is the
consciousness of the unity of an act of sensible synthesis or the consciousness
of the procedure for generating a sensible intuition. This first sense of rule
anticipates what Kant, in the Schematism of the Pure Concepts of the
Understanding, calls a schema. But the concept is a rule also in another,
discursive sense. It is a rule in that thinking an object under a concept
provides a reason to predicate of this object the marks that define the concept}
\parencite[][50]{Longuenesse:KantandtheCapacitytoJudge1998}.} Dass ein Begriff
als diskursive Regel aufgefasst werden kann, heißt, dass er durch
Merkmale bestimmt ist, deren Angabe als Merkmale des Begriffs
Prinzipien an die Hand geben, die 
als Obersätze in Syllogismen fungieren (können).\footnote{\enquote{Every concept
is a rule insofar as its explication (e.\,g., a body is extended, limited in
space and impenetrable) can function as the major premise in a syllogism whose
conclusion would be the attribution of the marks belonging to this concept to an
object of sensible intuition}
\parencite[][50]{Longuenesse:KantandtheCapacitytoJudge1998}.}
Wenn ich vor mir eine bellende Dogge sehe und das Bellen ein
hinreichendes Merkmal dafür ist, dass etwas in Hund ist, dann kann ich
darauf schließen, dass die Dogge vor mir ein Hund ist. Ich erkenne die Dogge als Hund mittels des
allgemeinen Merkmals des Bellens. Um dies bewusst zu tun -- um das Bellen
\emph{als} Merkmal zu erkennen --, muss ich einen solchen Vernunftschluss
ausführen. Und einen solchen Vernunftschluss wiederum nennt die philosophische
Tradition des 18. Jahrhunderts einen \enquote{\emph{discursus}}. In
diesem Sinne ist es dann naheliegend, auch Begriffe \emph{diskursiv}
zu nennen, insofern sie sich vermittelt über allgemeine Merkmale auf
Gegenstände beziehen.

\phantomsection\label{Abschnitt:judiciumintuitivumdiscursivum}
Die historische Vorlage dieser Verwendungsweise der Begriffe \enquote{diskursiv}
und \enquote{intuitiv} findet sich in der Gegenüberstellung intuitiver und
diskursiver \emph{Urteile}, wie wir sie etwa bei
\authorfullcite{Wolff:Discursuspraeliminarisdephilosophiaingenere1996}
finden. \authorcite{Wolff:PhilosophiarationalissiveLogica1740}
unterscheidet mit den Attributen \enquote{intuitiv} und \enquote{diskursiv}
Urteile als je konkrete Akte des menschlichen Geistes. Nach
\authorcite{Wolff:Discursuspraeliminarisdephilosophiaingenere1996}
heißen diejenigen Urteile
diskursiv, die wir erst durch einen Syllogismus aus anderen Urteilen gewinnen,
während alle Urteile intuitiv genannt werden, deren Wahrheit wir ohne einen solchen
Vernunftschluss erkennen.\footnote{\enquote{\ori{Judicium} istud dicimus
\ori{intuitivum}, quo enti cuidam tribuimus, qu{\ae} in ipsius notione
comprehensa intuemur. Istud autem \ori{judicium discursivum} appellamus, quod
per ratiocinium {\punkt} elicitur.
Posset quoque dici \ori{diano{"e}ticum}}
\parencite[][\S~51]{Wolff:PhilosophiarationalissiveLogica1740}.} Der
\emph{discursus} ist die Tätigkeit des logischen
Schließens.\footnote{\cite[Vgl.][\S~52\,f.]{Wolff:PhilosophiarationalissiveLogica1740}.
Mitunter scheint er auf Wortgebräuche bei anderen verweisen zu wollen:
\enquote{Alii ratiocinationem \ori{Dian{\oe}am}, \ori{Discursum},
\ori{Argumentationem} vocant}
\parencite[][\S~50]{Wolff:PhilosophiarationalissiveLogica1740}. Siehe auch
\cite[][\S~204]{Baumgarten:AcroasislogicainChristianumL.B.deWolff1983}.}
Diskursive Urteile sind also -- in neuerer Sprache -- inferentiell, während
intuitive Urteile nicht-inferentielle Erkenntnisse sind.\footnote{Dies passt
zunächst natürlich zu \name[Immanuel]{Kant}s Unterscheidung diskursiver und
intuitiver rationaler Erkenntnisse im eigentlichen Sinne, also der
Unterscheidung des Vorgehens von Philosophie und Mathematik als
unterschiedlicher Vernunftwissenschaften. Die Mathematik verfügt über Axiome,
also unmittelbar einsichtige Vernunftwahrheiten, die nach
\authorcite{Wolff:Discursuspraeliminarisdephilosophiaingenere1996} als intuitive
Urteil gelten können (wobei
\authorcite{Wolff:Discursuspraeliminarisdephilosophiaingenere1996} primär an
sinnliche Anschauungen denkt).}
\authorcite{Baumgarten:AcroasislogicainChristianumL.B.deWolff1983} konkretisiert
dies dahingehend, dass die intuitiven Urteile (\emph{propositiones intuitivae})
diejenigen sind, derer wir aus unserer Erfahrung gewiss sind. Die
Gesamtmenge unserer Urteile unterteilt er also in
diskursive Urteile, die wir auf der Grundlage anderer Wahrheiten erschlossen
haben, auf der einen Seite und intuitive Erfahrungsurteile auf der
anderen Seite.\footnote{\enquote{\ori{Propositio} per experientiam
    nobis complete certa, est \ori{intuitiva}, ex aliis vero cognita,
    \ori{discursiva}} \parencite[][\S~166]{Baumgarten:AcroasislogicainChristianumL.B.deWolff1983}.}
Das intuitive Urteil nennt er auch einen \enquote{Erfahrungssatz}, das
diskursive Urteil eine
\enquote{Folgerung}.\footcite[Vgl.][\S~166]{Baumgarten:AcroasislogicainChristianumL.B.deWolff1983}
\authorcite{Meier:AuszugausderVernunftlehre1752} schließt sich
\authorcite{Baumgarten:Metaphysica---Metaphysik2011} an und nennt das
\emph{iudicium discursivum} ein \enquote{Nachurteil}, das \emph{iudicium
intuitivum} hingegen ein \enquote{anschauendes Urteil}, welches eine
unmittelbare Erfahrung darstelle (und \emph{a fortiori} singulär
sei).\footnote{\enquote{Die erweislichen Urtheile sind entweder bloss durch die
Erfahrung gewiss, oder nicht. Jene sind \ori{anschauende Urtheile} (iudicium
intuitivum), diese aber \ori{Nachurtheile} (iudicium discursivum). Das
\ori{anschauende Urtheil besteht aus lauter Erfahrungsbegriffen}, und ist eine
unmittelbare Erfahrung {\punkt}, und ein einzelnes Urtheil}
\mkbibparens{\cite[][\S~319]{Meier:AuszugausderVernunftlehre1752},
\cite[][XVI: 674.24--28]{Kant:GesammelteWerke1900ff.}}.}


\subsection[Verstand und Sinnlichkeit]{Verstand und Sinnlichkeit:
\enquote{intellektuell} und
\enquote{sinnlich}}\label{subsection:VerstandundRezeptivitaet}
Ein offensichtliches Problem bringt die Vielfalt der Charakterisierungen des
Verstandes mit sich, wenngleich \name[Immanuel]{Kant} behauptet, sie liefen
letztlich alle auf dasselbe hinaus\footnote{\cite[Vgl.][A
126]{Kant:KritikderreinenVernunft2003}, \cite[][IV:
92.25--29]{Kant:GesammelteWerke1900ff.}.}. Allein in der \titel{Kritik der
reinen Vernunft} finden sich folgende Charakterisierungen:
In der Einleitung sagt \name[Immanuel]{Kant}, \enquote{daß es zwei Stämme der
menschlichen Erkenntnis gebe, die vielleicht aus einer gemeinschaftlichen, aber
uns unbekannten Wurzel entspringen, nämlich Sinnlichkeit und Verstand, durch
deren ersteren uns Gegenstände \ori{gegeben}, durch den zweiten aber
\ori{gedacht} werden.}\footnote{\cite[][B 29]{Kant:KritikderreinenVernunft2003},
\cite[][III: 46.7--11]{Kant:GesammelteWerke1900ff.}.} Der Verstand ist danach also das
\emph{Vermögen, zu denken}, und in dieser Funktion ist er das Vermögen der
\enquote{Spontaneität der Begriffe}\footnote{\cite[][B
74]{Kant:KritikderreinenVernunft2003}, \cite[][III:
74.12]{Kant:GesammelteWerke1900ff.}.}.
Direkt im Anschluss wird der Verstand als \enquote{das Vermögen,
Vorstellungen selbst hervorzubringen, oder die \ori{Spontaneität} des
Erkenntnisses}\footnote{\cite[][B 75]{Kant:KritikderreinenVernunft2003},
\cite[][III: 75.7--8]{Kant:GesammelteWerke1900ff.}.} charakterisiert. Und später
wird der Verstand auch als das \emph{Vermögen der Begriffe}
beschrieben.\footnote{\cite[Vgl.][B 199]{Kant:KritikderreinenVernunft2003},
\cite[][III: 146.29--30]{Kant:GesammelteWerke1900ff.}.} An wiederum anderer
Stelle findet sich die Charakterisierung des Verstandes als des \emph{Vermögens
zu urteilen}\footnote{\cite[Vgl.][B 94]{Kant:KritikderreinenVernunft2003},
\cite[][III: 86.12]{Kant:GesammelteWerke1900ff.}.}. In einer
Anmerkung zu \S~16 der \titel{Kritik der reinen Vernunft} wiederum identifiziert
\name[Immanuel]{Kant} den Verstand mit dem \emph{Vermögen der synthetischen
Einheit der Apperzeption}: \enquote{Und so ist die
synthetische Einheit der Apperzeption der höchste Punkt, an dem man allen
Verstandesgebrauch, selbst die ganze Logik, und, nach ihr, die
Transzendental-Philosophie heften muß, ja dieses Vermögen ist der Verstand
selbst}\footnote{\cite[][\S~16]{Kant:KritikderreinenVernunft2003}, \cite[][III:
109.35--38]{Kant:GesammelteWerke1900ff.}}. Er ist dadurch das \emph{Vermögen,
\emph{a priori} zu
verbinden}\footnote{\cite[Vgl.][\S~16]{Kant:KritikderreinenVernunft2003},
\cite[][III: 110.15]{Kant:GesammelteWerke1900ff.}.}.
Wenige Seiten später heißt es, der Verstand sei schlicht das
\emph{Vermögen der Erkenntnisse}.\footnote{\enquote{\ori{Verstand} ist,
allgemein zu reden, das Vermögen der
\ori{Erkenntnisse}}
\mkbibparens{\cite[][\S~17]{Kant:KritikderreinenVernunft2003}, \cite[][III:
111.16]{Kant:GesammelteWerke1900ff.}}.}
In der \titel{Transzendentalen Dialektik} taucht außerdem die Charakterisierung
des Verstandes als \enquote{Vermögen der Einheit der Erscheinungen vermittelst
der Regeln}\footnote{\cite[][B 359]{Kant:KritikderreinenVernunft2003},
\cite[][III: 239.27--28]{Kant:GesammelteWerke1900ff.}.
Diese Definition liegt auch der Unterscheidung von Urteils\emph{vermögen}
(Verstand) und Urteils\emph{kraft} zugrunde:
\enquote{Wenn der Verstand überhaupt als das Vermögen der Regeln erklärt wird,
so ist Urteilskraft das Vermögen unter Regeln zu \ori{subsumieren}}
\mkbibparens{\cite[][B 171]{Kant:KritikderreinenVernunft2003}, \cite[][III:
131.13--14]{Kant:GesammelteWerke1900ff.}}. Siehe auch \cite[][B 197f., 356, A
126f.,]{Kant:KritikderreinenVernunft2003} \cite[][III:
146.6--12, 238.8--11, IV: 92.29--30]{Kant:GesammelteWerke1900ff.}.} auf.


Es hat zunächst den Anschein, als brächte \name[Immanuel]{Kant} etliche
verschiedene Definitionen des Verstandesbegriffs, ohne deren Zusammenhang zu
verdeutlichen. Es liegt nahe davon auszugehen, dass all diese
Bestimmungen letztlich übereinkommen, dass es verschiedene
Beschreibungen desselben Vermögens -- des Verstandes -- sind. 
Allerdings wird sich zeigen, dass dies nur gilt, solange es sich um einen
\emph{endlichen} Verstand handelt. Die Endlichkeit des Verstandes ist eine
Voraussetzung, auf deren Grundlage sich erst die Äquivalenz der
genannten Definitionen ergibt. Einige der Beschreibungen treffen auf
jeden denkbaren Verstand zu; sie bilden die Bestimmung des Begriffs
\enquote{Verstand}. Andere wiederum treffen nur auf einen endlichen
Verstand, nicht aber auf einen unendlichen oder
\singlequote{intuitiven} Verstand zu. Sie bestimmen den Begriff des
diskursiven Verstandes.

Eine Bestimmung des Verstandesbegriffs, die neutral ist gegenüber der Frage, ob
der Verstand diskursiv oder intuitiv ist und ob es sich um unseren endlichen
oder einen göttlichen (oder irgendwie \singlequote{anderen}) Verstand handelt,
findet sich in den Anfangsabschnitten der \titel{Transzendentalen Logik} der
\titel{Kritik der reinen Vernunft}:
\begin{quote}
Wollen wir die \ori{Rezeptivität} unseres Gemüts, Vorstellungen zu empfangen, so
fern es auf irgend eine Weise affiziert wird, \ori{Sinnlichkeit} nennen; so ist
dagegen das Vermögen, Vorstellungen selbst hervorzubringen, oder die
\ori{Spontaneität} des Erkenntnisses, der
\ori{Verstand}.\footnote{\cite[][B 75]{Kant:KritikderreinenVernunft2003},
\cite[][III: 75.5--8]{Kant:GesammelteWerke1900ff.}.}
\end{quote}
Diese Bestimmung baut unmittelbar auf der vorausgesetzten Dualität der
Erkenntnisstämme auf: \name[Immanuel]{Kant} spricht von Rezeptivität und
Spontaneität als den beiden \enquote{Grund\-quel\-len des
Gemüts}\footnote{\cite[][B 74]{Kant:KritikderreinenVernunft2003},
\cite[][III: 74.9]{Kant:GesammelteWerke1900ff.}.} oder den beiden
\enquote{Stämme[n] der menschlichen Erkenntnis}\footnote{\cite[][B
29]{Kant:KritikderreinenVernunft2003}, \cite[][III:
46.7]{Kant:GesammelteWerke1900ff.}.}. Er sagt damit, dass unser Gemüt einerseits
aus sich selbst heraus tätig und andererseits bloß leidend ist und insofern über
zwei Stämme oder Grundquellen verfügt: Spontaneität und Rezeptivität.

Die Dualität der Erkenntnisstämme oder Grundquellen des Gemüts -- Spontaneität
und Rezeptivität oder Verstand und Sinnlichkeit -- ist eine entscheidende
Voraussetzung der \titel{Kritik der reinen Vernunft}. In der Einleitung schreibt
\name[Immanuel]{Kant}:
\begin{quote}
Nur so viel scheint zur Einleitung, oder Vorerinnerung, nötig zu sein, daß es
zwei Stämme der menschlichen Erkenntnis gebe, die vielleicht aus einer
gemeinschaftlichen, aber uns unbekannten Wurzel entspringen, nämlich
Sinnlichkeit und Verstand, durch deren ersteren uns Gegenstände \ori{gegeben},
durch den zweiten aber \ori{gedacht}
werden.\footnote{\cite[][B 29]{Kant:KritikderreinenVernunft2003},
\cite[][III: 46.6--11]{Kant:GesammelteWerke1900ff.}.}
\end{quote}
Dieses Zitat versammelt die entscheidenden Voraussetzungen der Vernunftkritik, insofern es
die Endlichkeit unseres Verstandes als etwas ansieht, das keiner Begründung und
keines Nachweises, sondern lediglich einer \enquote{Vorerinnerung} fähig und
bedürftig sei. \authorfullcite{Gloy:DieKantischeDifferenzvonBegriffundAnschauungundihreBegruendung1984}
hat \name[Immanuel]{Kant}s Dualismus kritisiert, weil sie eine Begründung für
dieses Vorgehen
vermisst.\footnote{\cite[Vgl.][\pno~1\,f.]{Gloy:DieKantischeDifferenzvonBegriffundAnschauungundihreBegruendung1984}.
Zu weiteren Vertretern dieser Kritik an \name[Immanuel]{Kant} siehe die von
\textcite[vgl.][Anm. 3 auf S.~66\,f.]{Heidemann:AnschauungundBegriff2002}
angeführte Literatur.} Es ist ein alter Vorwurf an \name[Immanuel]{Kant}, dass seine
Philosophie auf unbegründeten Annahmen errichtet sei, zu denen insbesondere die
Endlichkeit des Menschen und seines Verstandes gehöre, über deren Momente er
sich und dem Leser keine Rechenschaft ablege.\footnote{So schreibt
\authorcite{Hegel:GesammelteWerke} in \titel{Glauben und Wissen}: \enquote{Kant
hat keinen andern Grund als schlechthin die Erfahrung und die empirische
Psychologie, daß das menschliche Erkenntnißvermögen seinem Wesen nach in dem
bestehe, wie es erscheint, nemlich in jenem Fortgehen vom Allgemeinen zum
Besondern oder rückwärts vom Besondern zum Allgemeinen; aber indem er selbst
einen intuitiven Verstand denkt, auf ihn als absolut nothwendige Idee geführt
wird, stellt er selbst die entgegengesetzte Erfahrung von dem Denken eines nicht
discursiven Verstandes auf, und erweist, daß sein Erkenntnißvermögen erkennt,
nicht nur die Erscheinung und die Trennung des Möglichen und Wirklichen in
derselben, sondern die Vernunft und das An-sich} \parencite[][IV:
341.21--29]{Hegel:GesammelteWerke}.} In der Tat gehen auch die meisten
Interpreten davon aus, dass sich eine Begründung dieser zentralen Annahme in der
\titel{Kritik der reinen Vernunft} wie in anderen Schriften
\name[Immanuel]{Kant}s nicht finden lässt.\footnote{Eine Ausnahme bildet
\authorfullcite{Heidemann:AnschauungundBegriff2002}, der behauptet, die
Unterscheidung zweier Stämme der Erkenntnis folge aus der Trennung von
Anschauung und Begriff, welche \name[Immanuel]{Kant} zunächst voraussetze, um
sie dann \enquote{im Nachhinein rekonstruktiv zu beweisen}
\parencite[][69]{Heidemann:AnschauungundBegriff2002}.}

\name[Immanuel]{Kant} sagt an der genannten Stelle aber nicht nur, dass es zwei
Erkenntnisstämme gebe und dass sie vielleicht einer uns unbekannten gemeinsamen
Wurzel entspringen. Er sagt außerdem, dass durch Sinnlichkeit Gegenstände
\emph{gegeben}, durch den Verstand dieselben jedoch \emph{gedacht} werden. Und
\emph{prima facie} könnte es scheinen, als definierte \name[Immanuel]{Kant} hier
erstmalig die Begriffe \enquote{Sinnlichkeit} und \enquote{Verstand}, die später
dann als Rezeptivität und Spontaneität beschrieben werden.
\name[Immanuel]{Kant} sagt also, (a) \emph{Sinnlichkeit} sei
(a\textsubscript{i}) die \emph{Rezeptivität} des Gemüts, durch die uns
(a\textsubscript{ii}) Gegenstände \emph{gegeben} werden; und weiter, (b) der
\emph{Verstand} sei (b\textsubscript{i}) ein Vermögen der \emph{Spontaneität},
durch das (b\textsubscript{ii}) Gegenstände \emph{gedacht} werden. Wie im Falle
von Begriffen und Anschauungen konfrontiert uns \name[Immanuel]{Kant} mit
verschiedenen Merkmalen, deren Zusammenhang erst zu explizieren ist. Die beiden
jeweils angegebenen Merkmale stimmen jedoch auch nach \name[Immanuel]{Kant}s
eigener Auffassung nicht \emph{per se} überein.
Daran ändert auch nicht, dass sich bei \name[Immanuel]{Kant} mitunter
Formulierungen finden, die den Anschein erwecken, als stelle der Verweis auf
Anschauungen und Begriffe eine ebenso legitime Möglichkeit dar, die Begriffe
\enquote{Sinnlichkeit} und \enquote{Verstand} zu definieren, wie der Verweis auf
Rezeptivität und Spontaneität. In der \jaeschelogik{} steht:
\begin{quote}
Reflektieren wir auf unsre Erkenntnisse in Ansehung der beiden wesentlich
verschiedenen Grundvermögen der Sinnlichkeit und des Verstandes, woraus sie
entspringen: so treffen wir auf den Unterschied zwischen Anschauungen und
Begriffen. Alle unsre Erkenntnisse nämlich sind, in dieser Rücksicht betrachtet,
entweder \ori{Anschauungen} oder \ori{Begriffe}. Die erstern haben ihre Quelle
in der \ori{Sinnlichkeit} -- dem Vermögen der Anschauungen; die letztern im
\ori{Verstande} -- dem Vermögen der Begriffe. Dieses ist der \ori{logische}
Unterschied zwischen Verstand und Sinnlichkeit, nach welchem diese nichts als
Anschauungen, jener hingegen nichts als Begriffe
liefert.\footnote{\cite[][A 45]{Kant:ImmanuelKantsLogik1977},
\cite[][IX: 35.33--36.8]{Kant:GesammelteWerke1900ff.}.}
\end{quote}
Neben diesem \singlequote{logischen Unterschied} gebe es jedoch auch einen
\emph{metaphysischen} Unterschied beider \singlequote{Grundvermögen}, wonach die Sinnlichkeit
ein \singlequote{Vermögen} der Rezeptivität, der Verstand eines der Spontaneität
sei.\footnote{\cite[Vgl.][A 45]{Kant:ImmanuelKantsLogik1977},
\cite[][IX: 36.8--12]{Kant:GesammelteWerke1900ff.}.
\authorfullcite{Heidemann:AnschauungundBegriff2002} spricht von einer
vermögenstheoretischen Unterscheidung im Gegensatz zur erkenntnistheoretischen
Unterscheidung zwischen dem Vermögen der Anschauungen und dem der Begriffe
\parencite[vgl.][84]{Heidemann:AnschauungundBegriff2002}.} Doch auch wenn diese
Stelle \name[Immanuel]{Kant}s Auffassung treffen sollte,\footnote{Diese Stelle
der {\jaeschelogik} scheint mir wenig verlässlich zu sein, insofern die Nennung
von logischem und metaphysischem Gesichtspunkt an keiner anderen Stelle im Werk
\name[Immanuel]{Kant}s vorzukommen scheint
\parencite[vgl.][83]{Heidemann:AnschauungundBegriff2002}. Wegen der in Anm.
\ref{Anmerkung:Einleitung:VorbehaltegegenueberderJaescheLogik} auf
S.~\pageref{Anmerkung:Einleitung:VorbehaltegegenueberderJaescheLogik}
angesprochenen Vorbehalte sollte diese Passage daher nur unter Vorbehalt
herangezogen werden.} gilt: Nur der metaphysische Unterschied markiert eine
grundlegende Definition von Verstand und Sinnlichkeit, die sich unabhängig von
weiteren Voraussetzungen anwenden lässt; der logische Unterschied ist nur bei
endlichen Wesen anwendbar. Das entscheidende Argument ist hier:
Wären beide gleich zulässig, könnten wir weder den Begriff einer intellektuellen
Anschauung noch den eines anschauenden Verstandes bilden.

Eine Darlegung, wie sich die Charakterisierung des Verstandes als des
Vermögens der Begriffe aus der grundlegenden Charakterisierung als Vermögen der
Spontaneität unter Zugrundelegung seiner Endlichkeit ergibt, bestätigt die hier
vorgelegte Interpretation.
\begin{quote}
Der Verstand wurde oben bloß negativ erklärt: durch ein nicht-sinnliches
Erkenntnisvermögen. Nun können wir, unabhängig von der Sinnlichkeit, keiner
Anschauung teilhaftig werden. Also ist der Verstand kein Vermögen der
Anschauung. Es gibt aber, außer der Anschauung, keine andere Art, zu erkennen,
als durch Begriffe. Also ist die Erkenntnis eines jeden, wenigstens des
menschlichen, Verstandes, eine Erkenntnis durch Begriffe, nicht intuitiv,
sondern diskursiv.\footnote{\cite[][B
92\,f.,]{Kant:KritikderreinenVernunft2003} \cite[][III:
85.10--16]{Kant:GesammelteWerke1900ff.}.}
\end{quote}
Wenn \name[Immanuel]{Kant} den Verstand hier zunächst nur negativ, als ein
nicht-sinnliches Erkenntnisvermögen bestimmt, dann ist damit seine grundlegende
Bestimmung als Verstand bereits vollständig angegeben. Die weitere Bestimmung
als ein Vermögen der Begriffe bestimmt ihn bereits als einen endlichen Verstand.
Denn \emph{wir} verfügen über keine Anschauung unabhängig von unserer
Sinnlichkeit. Anschauung ist also an Rezeptivität gebunden -- aber nur für
unseren endlichen Verstand. Da unser Verstand ein Erkenntnisvermögen ist,
welches nicht anzuschauen vermag, bleibt nur, dass er ein Vermögen der Begriffe
ist.

Dass der Verstand ein Vermögen der Spontaneität ist, dies ist die primäre
Bestimmung seines Begriffs, denn es trifft auch auf den intuitiven Verstand zu.
Daher kann \name[Immanuel]{Kant} in der \titel{Kritik der Urteilskraft}
schreiben, dass \enquote{ein Vermögen einer \ori{völligen Spontaneität der
Anschauung} {\punkt} Verstand in der allgemeinsten Bedeutung sein
würde}\footnote{\cite[][\S~77]{Kant:KritikderUrteilskraft2009},
\cite[][V: 406.21--24]{Kant:GesammelteWerke1900ff.}. Dieses Zitat missversteht
\textcite[vgl.][285]{Westphal:KantHegelandtheFateofenquotetheIntuitiveIntellect2000},
der nicht erkennt, dass \enquote{Verstand in der allgemeinsten Bedeutung} auf
die ursprüngliche Definition als Vermögen der Spontaneität verweist. Diese
Bedeutung ist allgemeiner als die Bestimmung des Verstandes als Vermögen der
Spontaneität der Begriffe, insofern sie auch den anschauenden Verstand mit
einschließt.
\authorcite{Westphal:KantHegelandtheFateofenquotetheIntuitiveIntellect2000}s
Schlussfolgerung, dass auch der anschauende Verstand ein Vermögen der Begriffe
sei, übersieht genau dies.}.
Ich behaupte also, dass für den Begriff der Sinnlichkeit das Merkmal der
Rezeptivität (a\textsubscript{i}) und für den Begriff des Verstandes das
Merkmal der Spontaneität (b\textsubscript{i}) grundlegend ist und sich andere
Charakterisierungen -- insbesondere (a\textsubscript{ii}) und
(b\textsubscript{ii}) -- erst auf seiner Grundlage ergeben. Für viele
Beschreibungen ist dabei die Endlichkeit des Verstandes bereits vorausgesetzt,
denn nur unter dieser Annahme lässt sich der Verstand auch als Vermögen der
Begriffe, des Urteilens, des Denkens oder der Regeln beschreiben. Es sind dies
keine Merkmale des Verstandes als eines solchen, sondern Merkmale des
\emph{endlichen} Verstandes.

\subsection{Endlichkeit des
Verstandes}\label{subsection:EndlichkeitdesVerstandes}
Begriffe sind diskursive Vorstellungen, weil sie sich nur \emph{vermittelt über
allgemeine Merkmale} auf Gegenstände beziehen. Anschauungen hingegen sind als
Vorstellungen mit unmittelbarem Gegenstandsbezug intuitiv. Dass Begriffe
allgemeine und Anschauungen einzelne Vorstellungen sind, ist eine nachrangige
Bestimmung, die aus Mittelbarkeit und Unmittelbarkeit folgt. Sinnlichkeit ist
nach \name[Immanuel]{Kant} definiert durch die Rezeptivität, der Verstand
hingegen durch Spontaneität. Andere Charakterisierungen setzen in der Regel
voraus, dass es sich um einen \emph{endlichen} Verstand handelt. Was ist nun die
Endlichkeit oder \singlequote{Diskursivität} des Verstandes, die als
zentrale Voraussetzung anzusehen ist? Wodurch wird sie bestimmt?

\authorfullcite{Foerster:Die25JahrederPhilosophie2011} behauptet,
\name[Immanuel]{Kant} betrachte im Themenbereich \enquote{intellektuelle
Anschauung} den Unterschied zwischen Rezeptivität und Spontaneität (der
Anschauung), im Themenbereich \enquote{intuitiver Verstand} aber den Unterschied
von Diskursivität und Intuitivität (des
Verstandes).\footcite[Vgl.][177]{Foerster:DieBedeutungvonSS7677deremphKritikderUrteilskraftfuerdieEntwicklungdernachkantischenPhilosophieTeil12002}
Er hat dabei sicherlich vollkommen Recht, wenn er darauf pocht, dass
\name[Immanuel]{Kant} einerseits zwischen \enquote{Rezeptivität} und
\enquote{Spontaneität}, andererseits aber zwischen \enquote{intuitiv} und
\enquote{diskursiv} unterscheidet. Aber daraus folgt gerade nicht, dass es sich
bei dem Vergleich unserer sinnlichen mit einer denkbaren intellektuellen
Anschauung und dem Vergleich unseres diskursiven Verstandes mit einem intuitiven
Verstand um völlig heterogene Ansätze handelt.

Wie so oft führt \name[Immanuel]{Kant} verschiedene Unterscheidungen ein, die
leicht konfundiert werden können, aber nicht verwechselt werden sollten, weil sie
jeweils ganz unterschiedliches unterscheiden: Das Begriffspaar
\enquote{Anschauung}/\enquote{Begriff} differenziert \emph{Erkenntnisse}, das
Begriffspaar \enquote{Verstand}/\enquote{Sinnlichkeit} differenziert
Erkenntnis\emph{vermögen}. Erkenntnisvermögen können spontan oder rezeptiv sein,
Erkenntnisse unmittelbar oder vermittelt, einzeln oder allgemein. Und die Frage nach einer intellektuellen
Anschauung geht ebenso wie die Frage nach der Möglichkeit eines anschauenden
oder intuitiven Verstandes darauf, welche Kombinationsmöglichkeit von
Erkenntnissen und Erkenntnisvermögen es gibt. Konkret geht es in beiden Fällen
darum, ob ein Vermögen der Spontaneität Anschauungen generieren kann, ohne auf
Rezeptivität als die andere Grundquelle unseres Gemüts zurückzugreifen.

\authorfullcite{Foerster:Die25JahrederPhilosophie2011} scheint mit dem Dualismus
von Rezeptivität und Spontaneität bereits unsere Endlichkeit -- zumindest
hinsichtlich der \titel{Kritik der reinen Vernunft} --
charakterisiert zu sehen, denn er kontrastiert dem
Stämmedualismus die intellektuelle Anschauung.\footnote{\enquote{Weil wir in
Verstand und Sinnlichkeit zwei voneinander unabhängige Stämme der Erkenntnis
haben, müssen wir zwischen Möglichkeit und Wirklichkeit unterscheiden (anders:
eine intellektuelle Anschauung)}
\parencite[][153]{Foerster:Die25JahrederPhilosophie2011}.}
\authorfullcite{Leech:MakingModalDistinctions2014} sagt, ein Verstand
sei unendlich, bei dem die beiden Vermögen zu einem Vermögen
verschmelzen. Ihre Darstellung beruht jedoch darauf, den Verstand mit dem
Vermögen zu denken zu identifizieren.\footnote{\enquote{My suggestion is that
the best way to understand these capacities is primarily as the product of collapsing our two
distinct capacities for thought and intuition into one. If understanding is a
capacity for thought, then \ori{intuitive understanding} is a capacity for
thought which can provide its own intuitions -- a capacity for thoughts
directly (immediately) about individuals. Likewise, if intuition is a capacity
for immediate, singular representation of individuals, then \ori{intellectual
intuition} is a capacity for such representations of individuals through an
intellectual capacity for thinking (not a capacity for sensing)}
\parencite[][345]{Leech:MakingModalDistinctions2014}.}
Ein unendlicher Verstand, bei dem beide Vermögen vereinigt sind, verfügte allem
Anschein nach über Rezeptivität wie Spontaneität und wäre ein Vermögen der
Anschauung ebenso wie ein Vermögen der Begriffe. Das jedoch scheint mir falsch
zu sein: Ein unendlicher Verstand verfügte nicht über Rezeptivität -- er wäre
ausschließlich aus sich selbst heraus tätig, also ganz Spontaneität -- und er
wäre auch kein Vermögen der Begriffe, sondern ausschließlich ein Vermögen der
Anschauung.

Dass ein unendlicher Verstand keine Rezeptivität beinhaltet, scheint mir
offensichtlich zu sein, denn Rezeptivität ist abhängig von äußeren
Einflüssen. Fraglich ist lediglich, ob ein unendlicher Verstand über
Begriffe verfügt. \authorfullcite{Westphal:KantHegelandtheFateofenquotetheIntuitiveIntellect2000}
bejaht dies: Es sei zwar üblich, den intuitiven Verstand als ein
nicht-begriffliches Vermögen zu beschreiben, doch bestimme \name[Immanuel]{Kant}
den Verstand als Vermögen der Begriffe und bezeichne auch den anschauenden
Verstand als Verstand in der allgemeinsten
Bedeutung.\footnote{\cite[Vgl.][\pno~284\,f.]{Westphal:KantHegelandtheFateofenquotetheIntuitiveIntellect2000}.}
\authorcite{Westphal:KantHegelandtheFateofenquotetheIntuitiveIntellect2000}s
Fehler liegt aber darin, nicht zu sehen, dass \name[Immanuel]{Kant} unseren
Verstand in zwei Stufen bestimmt: Als \emph{Verstand} ist er das Vermögen der
Spontaneität, als \emph{endlicher} Verstand ein Vermögen der Spontaneität der Begriffe.
\name[Immanuel]{Kant} selbst sagt explizit, dass ein unendlicher Verstand nicht
über Begriffe verfügte; denn eine Erkenntnis durch Begriffe heißt Denken, und
Denken wiederum beweise Schranken und könne daher dem göttlichen Verstand nicht
zugesprochen werden.\footnote{\cite[Vgl.][B
71]{Kant:KritikderreinenVernunft2003}, \cite[][III:
72.10--16]{Kant:GesammelteWerke1900ff.}. Siehe auch
\cite[][\nopp 6050]{Kant:Reflexionen1900ff.}, \cite[][XVIII:
434.22--24]{Kant:GesammelteWerke1900ff.}: \enquote{Die Ideen aber dieses
Ursprünglichen Verstandes können nicht Begriffe, sondern nur Anschauungen, aber
intellectuelle, seyn.}} Es spricht auch die Bestimmung von Begriffen als
mittelbaren Vorstellungen dafür, sie einem unendlichen Verstand nicht
zuzusprechen. Denn dieser erkennte allem Vermuten nach ausschließlich
unmittelbar, also anschaulich. Von Begriffen und von Denken kann daher bei einem
unendlichen Verstand gar nicht gesprochen werden. Stattdessen geht es darum, einem Vermögen der Spontaneität
Erkenntnisse zuzuschreiben, die sich unmittelbar auf ihre Gegenstände
beziehen.\footnote{Auch \authorfullcite{Leech:MakingModalDistinctions2014} tendiert an einer Stelle in
diese Richtung, bringt aber sofort das Vermögen des Denkens wieder ins Spiel,
weil sie den Verstand primär als Denkvermögen und erst \emph{a fortiori} als
Spontaneität bestimmt sieht
\parencite[vgl.][\pno~345\,f.]{Leech:MakingModalDistinctions2014}.
Dadurch muss sie wiederum annehmen, in der Tätigkeit des anschauenden Verstandes
fielen Denken und Anschauen in eins, was hieße, dass Mittelbarkeit (Begriffe)
und Unmittelbarkeit (Anschauungen) identifiziert würden. Dies
scheint mir jedoch keinen vernünftigen Sinn zu ergeben.}

Ebenso wäre freilich ein Wesen denkbar, dass zwar über beide Stämme verfügt,
dessen Verstand aber zumindest manchmal Dinge anschaut, ohne auf Sinnlichkeit
angewiesen zu sein; er könnte \emph{manche} Gegenstände sinnlich, andere rein
intellektuell anschauen. \authorcite{Fichte:DieBestimmungdesMenschen1800}s
intellektuelle Anschauung des Ich -- \name[Immanuel]{Kant} hat in den 1770er
Jahren möglicherweise eine ähnliche Konzeption
vertreten\footnote{\cite[Vgl.][78]{Duesing:SpontanediskursiveSynthesis2004}.} --
wäre ein Beispiel für eine solche Konzeption.
Anders verhält es sich bei unserem Verstand: Abstrahieren wir von der
Sinnlichkeit, so bleibt \enquote{nichts als die bloße Form des Denkens ohne
Anschauung übrig, wodurch allein ich nichts Bestimmtes, also keinen Gegenstand
erkennen kann. Ich müßte mir zu dem Ende einen andern Verstand denken, der die
Gegenstände anschauete, wovon ich aber nicht den mindesten Begriff habe, weil
der menschliche diskursiv ist, und nur durch allgemeine Begriffe erkennen
kann.}\footnote{\cite[][\S~57]{Kant:ProlegomenazueinerjedenkuenftigenMetaphysikdiealsWissenschaftwirdauftretenkoennen1977},
\cite[][IV: 355.33--356.1]{Kant:GesammelteWerke1900ff.}.}
Ein endlicher Verstand kann ohne Sinnlichkeit gar keine Erkenntnis
erwerben. Begriffe ohne Inhalt sind leer; und ihren Inhalt erhalten sie
durch den Bezug auf Gegenstände. Auf diese beziehen sie sich aber nur mittelbar
über allgemeine Merkmale, also über weitere Begriffe, die sich ebenso mittelbar auf
Gegenstände beziehen. Gäbe es nun keine Vorstellungen, die sich unmittelbar auf
ihre Gegenstände beziehen -- Anschauungen --, so käme gar kein Gegenstandsbezug
zustande.\footnote{\enquote{Auf welche Art und durch welche Mittel sich auch
immer eine Erkenntnis auf Gegenstände beziehen mag, so ist doch diejenige,
wodurch sie sich auf dieselbe[n] unmittelbar bezieht, und worauf alles Denken
als Mittel abzweckt, die \ori{Anschauung}}
\mkbibparens{\cite[][B 33]{Kant:KritikderreinenVernunft2003},
\cite[][III: 49.6--9]{Kant:GesammelteWerke1900ff.}}.} Der systematische
Zusammenhang der Begriffe untereinander und unser Denken in Begriffen verkämen
zu einem leeren \enquote{frictionless spinning in the void}, wie
\authorfullcite{McDowell:MindandWorld1994} schreibt.\footnote{\enquote{We need
to conceive this expansive spontaneity as subject to controll from outside our
thinking, on pain of representing the operations of spontaneity as a
frictionless spinning in the void} \parencite[][11]{McDowell:MindandWorld1994}.}
\name[Immanuel]{Kant} sagt, weil Gedanken ohne Inhalt
leer seien, deshalb müssten wir Begriffe sinnlich machen und ihnen den
Gegenstand in der Anschauung
beifügen.\footnote{\phantomsection\label{Fussnote:InfiniterRegressBegriffeueberMerkmale}\enquote{Gedanken
    ohne Inhalt sind leer, Anschauungen ohne Begriffe sind
    blind. Daher ist es eben so notwendig, seine Begriffe sinnlich zu
    machen, (d.\,i. ihnen den Gegenstand in der Anschauung
    beizufügen,) als seine Anschauungen sich verständlich zu machen
    (d.\,i. sie unter Begriffe zu bringen)} \mkbibparens{\cite[][B
    75]{Kant:KritikderreinenVernunft2003}, \cite[][III:
    75.14--18]{Kant:GesammelteWerke1900ff.}}. \name[Immanuel]{Kant}s
  Konzeption von Begriffen als Merkmalskomplexionen führt dabei auf
  zwei Schwierigkeiten: Zum einen scheint es, als könnten sich
  Begriffe immer nur auf weitere Begriffe, niemals aber auf
  Gegenstände beziehen. Denn wenn er sagt, dass sie sich mittels
  allgemeiner Merkmale auf Gegenstände beziehen, die selbst wiederum
  Begriffe sind, dann bezieht sich zunächst ein Begriff $ P_1 $ auf
  weitere Begriffe $ P_2^1 , P_2^2 \dots $, die sich wiederum auf
  Begriffe $P_3^1, P_3^2 \dots$ als ihre Merkmale beziehen. Der Bezug
  auf Gegenstände muss letztlich über Anschauungen geschehen, doch
  enthält die Begriffstheorie keinerlei Hinweise, welcher
  systematische Ort den Anschauungen hier zukommt. Zum anderen ist
  unklar, wie die Theorie von Begriffen als Merkmalskomplexionen sich auf
  \singlequote{höchste Gattungen} oder allgemeinste Begriffe anwenden
  lässt. Wenn $P_2$ Merkmal von $P_1$ ist, dann gilt dass alle unter
  $P_1$ fallenden Gegenstände auch unter $P_2$ fallen: $ \forall x
  (P_1(x) \supset P_2(x)) $. $P_2$ ist dann allgemeiner als $P_1$. Es
  muss aber eine oder mehrere allgemeinste Begriffe geben, auf die die
  Theorie von Begriffen als Merkmalskomplexionen \emph{a fortiori} gar
  nicht mehr anwendbar ist. Es ist hier nicht der Ort, diese
  Schwierigkeiten aufzulösen.}

Letztlich gilt dies auch für die reinen Begriffe, die -- obwohl gänzlich frei
von sinnlichem Gehalt -- ebenso nur dadurch gehaltvoll sind, dass sie sich auf
mögliche Erfahrungen beziehen und entsprechend schematisieren lassen. Die Frage
nach der realen Möglichkeit eines Begriffs wird von \name[Immanuel]{Kant} so
beantwortet, dass real möglich ist, was sich auf eine mögliche Erfahrung
bezieht.\footnote{\enquote{Ein Begriff, der eine Synthesis in sich faßt, ist
für leer zu halten, und bezieht sich auf keinen Gegenstand, wenn diese
Synthesis nicht zur Erfahrung gehört, entweder als von ihr erborgt, und dann
heißt er ein \ori{empirischer Begriff}, oder als eine solche, auf der, als
Bedingung a priori, Erfahrung überhaupt (die Form derselben) beruht, und denn
ist es ein \ori{reiner Begriff}, der dennoch zur Erfahrung gehört, weil sein
Objekt nur in dieser angetroffen werden kann}
\mkbibparens{\cite[][B 267]{Kant:KritikderreinenVernunft2003},
\cite[][III: 186.29--35]{Kant:GesammelteWerke1900ff.}}.}
Besonders deutlich wird dies unter Bezug auf die Mathematik, von der
\name[Immanuel]{Kant} in \S~22 der \titel{Kritik der reinen Vernunft} sagt, sie enthalte letztlich nur
deswegen Erkenntnisse, weil wir voraussetzen können, dass es Dinge gibt, von
denen wir \emph{empirische} Anschauungen haben, von denen die reine Anschauung
der Mathematik wiederum die Form angibt. Und dasselbe gilt dann natürlich von
den reinen
Verstandesbegriffen.\footnote{\enquote{Folglich verschaffen die reinen Verstandesbegriffe, selbst wenn sie auf
Anschauungen a priori (wie in der Mathematik) angewandt werden, nur so fern
Erkenntnis, als diese, mithin auch die Verstandesbegriffe vermittelst ihrer, auf
empirische Anschauung angewandt werden können}
\mkbibparens{\cite[][\S~22]{Kant:KritikderreinenVernunft2003},
\cite[][III: 117.22--26]{Kant:GesammelteWerke1900ff.}}.}
Der endliche Verstand kann ohne Sinnlichkeit keine objektiven Erkenntnisse
generieren, weder Begriffe noch Urteile. Auch reine Begriffe und transzendentale
Grundsätze sind nur dadurch Erkenntnisse, denen objektive Realität oder
objektive Gültigkeit zukommt, insofern sie sich auf mögliche sinnliche Erfahrungen beziehen.
Der unendliche Verstand, der selbst anschaute, unterläge einer solchen
Einschränkung nicht. Er wäre in seinem Erkennen nicht von Rezeptivität
abhängig, sondern gewährleistete aus reiner Selbsttätigkeit die
objektive Gültigkeit seiner Erkenntnis.

Der Verstand ist zunächst das Vermögen geistiger
Selbsttätigkeit.\footnote{\enquote{Selbsttätigkeit} ist bei
\name[Immanuel]{Kant} wie schon bei
\authorcite{Baumgarten:Metaphysica---Metaphysik2011} die Übersetzung des
lateinischen \enquote{spontaneitas}
\mkbibparens{\cite[vgl.][\S~15]{Kant:KritikderreinenVernunft2003}, \cite[][III:
107.7--25]{Kant:GesammelteWerke1900ff.};
\cite[][\S\,704]{Baumgarten:Metaphysica---Metaphysik2011}, \cite[][XVII:
131.26, 33]{Kant:GesammelteWerke1900ff.}}. Noch \authorcite{Wolff:Discursuspraeliminarisdephilosophiaingenere1996} übersetzt
\enquote{spontaneitas} hingegen als \enquote{Willkür}, wie er im ersten Register der
Deutschen Metaphysik vermerkt \parencite[vgl.][\pno~677
\mkbibparens{n.\,p.}]{Wolff:VernuenftigeGedanckenvonGottderWeltundderSeeledesMenschenauchallenDingenueberhauptDeutscheMetaphysik1983}.}
Als solches heißt er \enquote{oberes Erkenntnisvermögen}, welches dann auch mit
der Vernunft als dem Vermögen der Prinzipien identifiziert
wird.\footnote{\phantomsection\label{Anmerkung:ObereErkenntnisvermoegenSingularPlural}\name[Immanuel]{Kant}
kennt sowohl die Ausdrucksweise von dem einen oberen Erkenntnisvermögen, welches
Vernunft, Verstand und Urteilskraft umfasst \mkbibparens{\cite[siehe z.\,B.][BA
115]{Kant:AnthropologieinpragmatischerHinsicht1977},
\cite[][VII: 196.17--19]{Kant:GesammelteWerke1900ff.}}, als auch die
Ausdrucksweise von den oberen Erkenntnisvermögen, der zufolge jedes einzelne
dieser Vermögen als ein oberes Erkenntnisvermögen zählt
\mkbibparens{\cite[siehe z.\,B.][B 169]{Kant:KritikderreinenVernunft2003},
\cite[][III: 130.7--8]{Kant:GesammelteWerke1900ff.}}. Das obere
Erkenntnisvermögen als Ganzes nennt er mitunter \enquote{Verstand}
\mkbibparens{\cite[siehe z.\,B.][BA
115\,f.,]{Kant:AnthropologieinpragmatischerHinsicht1977} \cite[][VII:
196.17--197.3]{Kant:GesammelteWerke1900ff.}}
und manchmal \enquote{Vernunft} \mkbibparens{\cite[z.\,B.][B
863]{Kant:KritikderreinenVernunft2003}, \cite[][III:
540.28]{Kant:GesammelteWerke1900ff.}}. Hieraus erklärt sich
möglicherweise auch, warum in seiner Konzeption des Selbstdenkens -- etwa auch in dem Aufklärungsaufsatz -- von
Verstand und Vernunft teilweise synonym gesprochen zu werden scheint
\mkbibparens{\cite[siehe
etwa][13]{Bartuschat:KantueberPhilosophieundAufklaerung2009}, unter Verweis auf
\cite[][\pno~125\,ff.]{LaRocca:WasAufklaerungseinwird2004}}.} Unter
dem Titel \enquote{Verstand} wird das obere Erkenntnisvermögen als
Vermögen der Spontaneität oder Selbsttätigkeit beschrieben; unter der Überschrift
\enquote{Vernunft} als Vermögen der Erkenntnisse \emph{a priori} oder der
Autonomie.\footnote{Darauf werde ich in Kapitel
\ref{subsection:MetaphysikundAutonomie} dieser Arbeit näher eingehen.} Beides
muss nach der Systematik \name[Immanuel]{Kant}s koinzidieren, soll diese
terminologische Fixierung haltbar sein; es entspräche dann der Rede von einem
negativen und einem positiven Freiheitsbegriff oder -- noch näher -- einem
negativen und einem positiven Begriff des Selbstdenkens.\footnote{Zum negativen
und positiven Begriff des Selbstdenkens siehe Kapitel
\ref{subsection:DerBegriffdesSelbstdenkens}.} Insofern wir über ein solches
Vermögen verfügen, können wir selbständig denken und handeln.
Als endliche Wesen sind wir aber nicht nur selbständig, sondern in unserer
Selbständigkeit zugleich \emph{abhängig}. Diese Abhängigkeit betrifft die
Sinnlichkeit oder Rezeptivität, also die Tatsache, dass wir affiziert werden von
Gegenständen unserer Erkenntnis. Ein unendliches Wesen bräuchte keine äußeren
Einflüsse, denen gegenüber es sich leidend oder passiv verhielte. Es wäre sich
in seinem Denken selbst genug; daher entspricht dem Begriff eines nicht-endlichen
Verstandes der Begriff der \singlequote{Allgenugsamkeit}, den
\name[Immanuel]{Kant} als besseren Ausdruck für die göttliche Unendlichkeit
vorschlägt.\footnote{Siehe oben Seite \pageref{Allgenugsamkeit}.}

\section{Drei Vergleiche und ein Ursprung}\label{subsection:Unendlichkeiten}
Es gibt nach \name[Immanuel]{Kant} eine einheitliche Grundlage unserer
Endlichkeit: Die Abhängigkeit unseres Verstandes, der als Vermögen der
\singlequote{Spontaneität} oder \singlequote{Selbsttätigkeit} das obere
Erkenntnisvermögen ausmacht, von der Sinnlichkeit oder \emph{Rezeptivität} als
der Möglichkeit, von Dingen affiziert zu werden. Unser Verstand ist abhängig,
weil er die Gegenstände seines Erkennens nicht selbst hervorbringen kann,
sondern darauf angewiesen ist, dass sie ihm \emph{sinnlich gegeben} werden. Im
zurückliegenden Kapitel
\ref{subsection:DiskursiverVerstandundsinnlicheAnschauung} habe ich die
Konzeptionen eines in dieser Hinsicht endlichen und eines solcherart unendlichen
Verstandes herausgearbeitet. Im folgenden werde ich diese Darstellung in einigen
Punkten konkretisieren und dabei zeigen, dass dies tatsächlich die allgemeine
Grundlage der Vergleiche unseres endlichen mit einem \singlequote{anderen}
Erkenntnisvermögen ist. Dazu betrachte ich die drei Instanzen unserer
Endlichkeit, wie \name[Immanuel]{Kant} sie in der \titel{Kritik der
Urteilskraft} aufzählt, um zu zeigen, dass alle drei ihre Grundlage in dieser
allgemeinen Charakterisierung unserer Endlichkeit haben. Es gibt also eine
einheitliche Beschreibung dessen, was es heißt, ein endliches Vernunftwesen zu
sein: Es heißt, ein von Rezeptivität \emph{abhängiges} Wesen zu sein.

\begin{comment}
Bevor \name[Immanuel]{Kant} auf die jeweiligen Besonderheiten in Bezug zu den
drei oberen Erkenntnisvermögen \emph{en detail} eingeht, macht er eine
Bemerkung, die sich auf alle drei Betrachtungen erstreckt:
\begin{quote}
Man wird bald inne, daß, wo der Verstand nicht folgen kann, die Vernunft
überschwänglich wird, und in zwar gegründeten Ideen (als regulativen
Prinzipien), aber nicht objektiv gültigen Begriffen sich hervortut; der Verstand
aber, der mit ihr nicht Schritt halten kann, aber doch zur Gültigkeit für
Objekte nötig sein würde, die Gültigkeit jener Ideen der Vernunft nur auf das
Subjekt, aber doch allgemein für alle von dieser Gattung, d.\.i. auf die
Bedingung einschränke, daß nach der Natur unseres (menschlichen)
Erkenntnisvermögens, oder gar überhaupt nach dem Begriffe, \ori{den wir uns} von
dem Vermögen eines endlichen Vernünftigen Wesens überhaupt \ori{machen} können,
nicht anders als so könne und müsse gedacht werden: ohne doch zu behaupten, daß
der Grund eines solchen Urteils im Objekt
liegt.\footnote{\cite[][\S~76]{Kant:KritikderUrteilskraft2009},
\cite[][V: 401.14--26]{Kant:GesammelteWerke1900ff.}.}
\end{quote}
Es gibt Begriffe und Urteile, die ihren Ursprung nicht in ihren Gegenständen
haben, sondern in dem erkennenden Subjekt, und die dennoch nicht beliebig sind.
Um zu zeigen, dass der Grund für solche Begriffe und Urteile nicht in ihren
Objekten liegt, muss \name[Immanuel]{Kant} einen denkbaren Vergleichsverstand
konzipieren, der sich von jedem endlichen qualitativ Verstand unterscheidet. Der
Grund liegt aber in einer Besonderheit unseres Verstandes, die ihn als
\emph{endlichen} Verstand notwendigerweise auszeichnet; handelte es sich um eine
zufällige Besonderheit, die wir empirisch bei uns feststellten, dann wäre es gar
kein hinreichend begründetes Urteil. Wäre ein Urteil so beschaffen, dass es für
jeden denkbaren Verstand zwingend wäre, dann müssten wir eingestehen, dass es
eine objektive Eigenschaft der Gegenstände und nicht eine notwendige subjektive
Beschaffenheit unseres Erkenntnisvermögens betrifft.
\end{comment}

Meine Position lässt sich dabei unter Berücksichtigung der Tatsache, dass
\name[Immanuel]{Kant} einmal von dem Verstand (im weiteren Sinne) als \emph{dem}
oberen Erkenntnisvermögen, daneben aber auch von Verstand (im engeren Sinne),
Vernunft und Urteilskraft als \emph{den} oberen Erkenntnisvermögen
spricht,\footnote{Siehe oben Anm.
\ref{Anmerkung:ObereErkenntnisvermoegenSingularPlural} auf
S.~\pageref{Anmerkung:ObereErkenntnisvermoegenSingularPlural}.} folgendermaßen
artikulieren: Die Dreiteilung in \S~76 der \titel{Kritik der Urteilskraft} nennt
drei Besonderheiten unseres Verstandes als des gesamten oberen Erkenntnisvermögens
\emph{in Ansehung} jeweils eines der drei besonderen oberen Erkenntnisvermögen 
Verstand, Vernunft und Urteilskraft. Es bleibt dabei dieselbe grundlegende
Charakteristik eines Verstandes im weiteren Sinne als eines endlichen Vermögens
(als auf Rezeptivität angewiesene Spontaneität), die sich aber in Ansehung der
drei oberen Erkenntnisvermögen unterschiedlich äußert.

Ich werde im folgenden auch auf
\authorcite{Foerster:Die25JahrederPhilosophie2011}s Behauptung eingehen,
\name[Immanuel]{Kant} verfüge über jeweils mehrere Konzeptionen einer
intellektuellen Anschauung und eines intuitiven Verstandes.\footnote{Siehe
Kapitel \ref{subsection:EineMoeglicheVieldeutigkeit}.} Da
\authorcite{Foerster:Die25JahrederPhilosophie2011} die intellektuelle Anschauung
primär der Unterscheidung von Wirklichkeit und Möglichkeit zuordnet, werde ich
ihre vermeintliche Mehrdeutigkeit in Kapitel
\ref{subsubsection:UnterscheidungvonDenkenundErkennen} besprechen. Die
Mehrdeutigkeit des Ausdruck \enquote{intuitiver Verstand} ist Thema in Kapitel
\ref{subsection:IntuitiverVerstandunddasSynthetischAllgemeine}, da
\authorcite{Foerster:Die25JahrederPhilosophie2011} den intuitiven Verstand
primär den Überlegungen in \S~77 der \titel{Kritik der Urteilskraft} zuordnet.
Im letzten Teil komme ich auf die Endlichkeit des \emph{Willens} zu sprechen
(Kapitel \ref{subsubsection:DieEndlichkeitdesWillens}). Dass diese eine
Sonderform der Endlichkeit des Verstandes darstellt, mag \emph{prima facie}
überraschen; es wird sich aber aus \name[Immanuel]{Kant}s Verständnis beider
Vermögen leicht erhellen lassen.

\subsection{Die Unterscheidung von Denken und
Erkennen}\label{subsubsection:UnterscheidungvonDenkenundErkennen}

Die erste in \S~76 der \titel{Kritik der Urteilskraft} beschriebene Besonderheit
unseres Erkenntnisvermögens betrifft die Unterscheidung von Möglichkeit und
Wirklichkeit.\footnote{In der \titel{Kritik der reinen Vernunft} entspricht dem die
Unterscheidung von \emph{Denken} und \emph{Erkennen}:
\enquote{Sich einen Gegenstand \ori{denken}, und einen Gegenstand \ori{erkennen}, ist
also nicht einerlei. Zum Erkenntnisse gehören nämlich zwei Stücke: erstlich der
Begriff, dadurch überhaupt ein Gegenstand gedacht wird (die Kategorie), und
zweitens die Anschauung, dadurch er gegeben
wird} \mkbibparens{\cite[][\S~22]{Kant:KritikderreinenVernunft2003},
\cite[][III: 116.34--117.2]{Kant:GesammelteWerke1900ff.}}.
Diese Unterscheidung fundiert auch \name[Immanuel]{Kant}s Begriffe von Glauben
und Wissen.} Weil unser Erkenntnisvermögen über \enquote{Verstand für
Begriffe und sinnliche Anschauung für Objekte, die ihnen korrespondieren,} als
\enquote{ganz heterogene
Stücke}\footnote{\cite[][\S~76]{Kant:KritikderUrteilskraft2009}, \cite[][V:
401.34--36]{Kant:GesammelteWerke1900ff.}.} verfügt, die beide zur Ausübung
unserer Erkenntnisvermögen notwendig sind, können und müssen wir zwischen dem
bloß Möglichen und dem Wirklichen unterscheiden.\footnote{\enquote{Es
    ist dem menschlichen Verstande unumgänglich notwendig, Möglichkeit
    und Wirklichkeit der Dinge zu unterscheiden. Der Grund davon liegt
    im Subjekte und der Natur seiner Erkenntnisvermögen. Denn wären zu
    dieser ihrer Ausübung nicht zwei ganz heterogene Stücke, Verstand
    für Begriffe und sinnliche Anschauung für Objekte, die ihnen
    korrespondieren, erforderlich, so würde es keine solche
    Unterscheidung (zwischen dem Möglichen und dem Wirklichen) geben}
  \mkbibparens{\cite[][\S~76]{Kant:KritikderUrteilskraft2009},
    \cite[][V: 401.31--402.1]{Kant:GesammelteWerke1900ff.}}.}
\enquote{Wäre nämlich unser Verstand anschauend, so hätte er keine Gegenstände als das
Wirkliche.}\footnote{\cite[][\S~76]{Kant:KritikderUrteilskraft2009}, \cite[][V:
402.1--2]{Kant:GesammelteWerke1900ff.}.} Die Unterscheidung von
Möglichkeit und Wirklichkeit ist also dem Dualismus der
Grundkräfte unseres Gemüts und der Notwendigkeit der Rezeptivität für unser
Erkennen geschuldet.

Nun ist es zunächst der Besonderheit unseres Erkenntnisvermögens geschuldet,
dass wir überhaupt über Kategorien und damit über die Begriffe der
Möglichkeit und der Wirklichkeit, die ja selbst zu den Kategorien
gehören, verfügen. Aber das kann nicht  der Hintergrund sein, wenn
\name[Immanuel]{Kant} feststellt, dass die Unterscheidung \emph{von 
Wirklichkeit und Möglichkeit} Folge der Endlichkeit unseres Verstandes
ist.\footnote{Ein Problem  dieser Überlegungen liegt darin, dass
\name[Immanuel]{Kant} sagt, die Kategorien seien ohnehin nur Begriffe des
endlichen Verstandes; ein unendlicher Verstand -- ein Verstand, der selbst anschaut --
verfügte nicht über sie. Auch die übrigen Kategorien gelten nur aus der Sicht
eines endlichen Verstandes, wie \name[Immanuel]{Kant} in der \titel{Kritik der reinen Vernunft} erläutert:
\enquote{Sie sind nur Regeln für einen Verstand, dessen ganzes Vermögen im Denken
besteht, d.\,i. in der Handlung, die Synthesis des Mannigfaltigen, welches ihm
anderweitig in der Anschauung gegeben worden, zur Einheit der Apperzeption zu
bringen, der also für sich gar nichts erkennt, sondern nur den Stoff zum
Erkenntnis, die Anschauung, die ihm durchs Objekt gegeben werden muß, verbindet
und ordnet} \mkbibparens{\cite[][\S~21]{Kant:KritikderreinenVernunft2003},
\cite[][III: 116.18--23]{Kant:GesammelteWerke1900ff.}}. Dann aber scheint die
Bemerkung zu den Kategorien der Modalität gänzlich überflüssig zu sein. Siehe
hierzu auch \cite[][\pno~356\,f.]{Leech:MakingModalDistinctions2014}.}
Stattdessen informiert \name[Immanuel]{Kant} über eine Besonderheit der Kategorien der
Modalität, die diese von den anderen neun Kategorien unterscheidet. In der
\titel{Kritik der reinen Vernunft} lesen wir:
\begin{quote}
Die Kategorien der Modalität haben das Besondere an sich: daß sie den Begriff,
dem sie als Prädikate beigefüget werden, als Bestimmung des Objekts nicht im
mindesten vermehren, sondern nur das Verhältnis zum Erkenntnisvermögen
ausdrücken. {\punkt} Hiedurch werden keine Bestimmungen mehr im Objekte selbst
gedacht, sondern es frägt sich nur, wie es sich (samt allen seinen Bestimmungen)
zum Verstande und dessen empirischen Gebrauche, zur empirischen Urteilskraft,
und zur Vernunft (in ihrer Anwendung auf Erfahrung)
verhalte?\footnote{\cite[][B 266]{Kant:KritikderreinenVernunft2003},
\cite[][III: 186.4--14]{Kant:GesammelteWerke1900ff.}.}
\end{quote}
Die übrigen neun Kategorien beschreiben Eigenschaften des Objekts, sie drücken
beispielsweise dessen Substantialität (Kategorie der Substanz) aus oder
artikulieren eine objektive Verbindung zwischen Gegenständen oder Ereignissen
(Kategorien der Ursache und der Wirkung). Was sie gegenüber den Kategorien der
Modalität auszeichnet ist folgendes: Sie sind zwar reine Begriffe -- sie sind also
nicht der Erfahrung entlehnt, sondern entstammen der Selbsttätigkeit
(Spontaneität) des Verstandes --, aber sie drücken doch objektive Eigenschaften \emph{des
Gegenstandes} aus.\footnote{Aus diesem Grund behauptet
\authorcite{Hegel:GesammelteWerke}, dass es in \name[Immanuel]{Kant}s
Darstellung nur neun statt zwölf Kategorien gebe:
\enquote{Die Identität des Subjekts und Objekts schränkt sich auf zwölf oder
vielmehr nur auf neun reine Denkthätigkeiten ein -- denn die Modalität giebt
keine wahrhaft objektive Bestimmung, es besteht in ihr wesentlich die
Nichtidentität des Subjekts und Objekts} \mkbibparens{\cite[][IV: 6.
8--11]{Hegel:GesammelteWerke}}.} Wenn wir sagen, dass $A$ die Ursache von $B$
sei, dann sagt dies etwas über $A$ und $B$ aus, nicht jedoch über uns
als erkennende Subjekte oder über unser Verhältnis zu $A$ und $B$. Bei den Kategorien der Modalität
verhalte es sich nun anders, sie bestimmen laut \name[Immanuel]{Kant} nicht das
Objekt, sondern unser Verhältnis zu dem Objekt. Wenn wir also sagen, $A$ sei
möglich (oder wirklich), dann sagen wir gar nichts über $A$ aus, sondern
darüber, in welcher Beziehung $A$ zu uns als erkennenden Subjekten steht. Die
Erläuterung in der \titel{Analytik der Grundsätze} gibt nähere Auskunft, warum
dies so ist:
\begin{quote}
\begin{nummerierung}
\item Was mit den formalen Bedingungen der Erfahrung (der Anschauung und den
Begriffen nach) übereinkommt, ist \ori{möglich}.
\item Was mit den materialen Bedingungen der Erfahrung (der Empfindung)
zusammenhängt, ist \ori{wirklich}.
\item Dessen Zusammenhang mit dem Wirklichen nach allgemeinen Bedingungen der
Erfahrung bestimmt ist, ist (existiert)
\ori{notwendig}.\footnote{\cite[][B 265\,f.,]{Kant:KritikderreinenVernunft2003}
\cite[][III: 185.22--186.2]{Kant:GesammelteWerke1900ff.}.}
\end{nummerierung}
\end{quote}
Sowohl das Wirkliche als auch das Mögliche verträgt sich zunächst mit den formalen Bedingungen der
Erfahrung, also mit den metaphysischen Erkenntnissen, die wir von der Natur
haben.\footnote{Ich fokussiere hier die Kategorien Wirklichkeit und Möglichkeit unter
Auslassung der Kategorie der Notwendigkeit. Diese Entscheidung
gründet in der Tatsache, dass \name[Immanuel]{Kant} in der Rekapitulation in
\S~76 der \titel{Kritik der Urteilskraft} nicht auf die Notwendigkeit eingeht.
Sie lässt sich nicht durch Verweis auf eine Definierbar der Notwendigkeit durch
die anderen Modalbegriffe begründen, da \name[Immanuel]{Kant} alle drei für
jeweils ursprünglich hält. \cite[Vgl.][196]{Poser:DieStufenderModalitaet1981},
sowie
\cite[][42--45]{Kamlah:KantsAntwortaufHumeundeinelinguistischeAnalyseseinerModalbegriffe2009}.
\authorfullcite{Poser:DieStufenderModalitaet1981} verweist bei
\name[Immanuel]{Kant} insb. auf \cite[][B 111]{Kant:KritikderreinenVernunft2003},
\cite[][III: 96.8--17]{Kant:GesammelteWerke1900ff.}.} Schon
\authorcite{Leibniz:Meditationesdecognitioneveritateetideis1999} ringt
in den \titel{Meditationes de veritate, cognitione et ideis} mit der
Frage, wann etwas (\emph{realiter} und nicht nur logisch) möglich ist. Ihm folgen
\authorcite{Wolff:Discursuspraeliminarisdephilosophiaingenere1996} und andere
schließlich darin, die reale Möglichkeit letztlich in der logischen Möglichkeit
fundieren zu wollen; und gerade hierin weicht \name[Immanuel]{Kant} von der
Schule \authorcite{Wolff:Discursuspraeliminarisdephilosophiaingenere1996}s
ab,\footnote{\cite[Vgl.][195]{Poser:DieStufenderModalitaet1981}.} wenn er
schreibt:
\begin{quote}
Das Postulat der \ori{Möglichkeit} der Dinge fordert also, daß der Begriff
derselben mit den formalen Bedingungen einer Erfahrung überhaupt
zusammenstimme.\footnote{\cite[][B 267]{Kant:KritikderreinenVernunft2003},
\cite[][III: 186.25--27]{Kant:GesammelteWerke1900ff.}.}
\end{quote}
Die formalen Bedingungen der Erfahrung entstammen der Konstitution des Subjekts,
sie sind daher Thema der Transzendentalphilosophie. Während
\authorcite{Leibniz:Meditationesdecognitioneveritateetideis1999} behauptet, dass die reale
Möglichkeit auf der Grundlage einer vollständigen Analyse eines Begriffs und der
darauf fundierten Einsicht in seine Widerspruchsfreiheit erkannt werden
kann,\footnote{\cite[Vgl.][589]{Leibniz:Meditationesdecognitioneveritateetideis1999}.}
fordert \name[Immanuel]{Kant}: Eine reale Möglichkeit ist nur, was Gegenstand in
einer möglichen Erfahrung sein kann; und Gegenstand einer möglichen Erfahrung
ist, was den formalen Bestimmungen entspricht, die uns die transzendentale
Ästhetik und die transzendentale Logik aufdecken. Insofern ist die reale
Möglichkeit an die (formalen) Bedingungen unserer sinnlichen Wahrnehmung
gebunden, an Raum, Zeit und die Kategorien sowie Grundsätze des reinen
Verstandes. Wir erkennen sie \emph{a priori}.\footnote{Natürlich
können wir außerdem auch aufgrund einer tatsächlichen sinnlichen Wahrnehmung auf
die Möglichkeit des bereits als wirklich erkannten schließen.}

Das Wirkliche hat darüber hinaus tatsächlich eine Verbindung zu den
\singlequote{materialen} Bedingungen der Erfahrung, die \name[Immanuel]{Kant} hier \emph{Empfindung}
nennt und die offensichtlich auf unsere Rezeptivität verweisen:
\begin{quote}
Das Postulat, die \ori{Wirklichkeit} der Dinge zu erkennen, fordert
\ori{Wahrnehmung}, mithin Empfindung, deren man sich bewußt ist, zwar nicht
eben unmittelbar, von dem Gegenstande selbst, dessen Dasein erkannt werden
soll, aber doch Zusammenhang desselben mit irgend einer wirklichen
Wahrnehmung, nach den Analogien der Erfahrung, welche alle reale Verknüpfung
in einer Erfahrung überhaupt darlegen.\footnote{\cite[][B
272]{Kant:KritikderreinenVernunft2003}, \cite[][III:
189.23--28]{Kant:GesammelteWerke1900ff.}.}
\end{quote}
Zum einen erkennen wir etwas als wirklich, wenn wir es sinnlich wahrnehmen. Den
Baum vor dem Fenster, den Lärm der Straße oder den Duft des Apfelkuchens
erkennen wir auf diese Art als wirklich. Zum anderen erkennen wir als wirklich,
was wir zwar nicht direkt sinnlich wahrnehmen können, was aber in einer kausalen
Verbindung zu etwas von uns sinnlich wahrgenommenem steht. Atome, Moleküle,
Elektronen und viele andere Dinge erkennen wir in diesem Sinne zwar nicht durch
direkte sinnliche Wahrnehmung als wirklich, aber doch indirekt durch
beobachtbare Wirkungen. Wir können sagen, dass wir indirekt die Wirklichkeit
erkennen, wenn wir \emph{inferentielles} Wissen generieren; direkt erkennen wir
dann, wenn es sich um eine nicht-inferentielle Erkenntnis handelt.\footnote{Ob
wir Teilchen in der Nebelkammer nun direkt oder indirekt als wirklich erkennen, solche Fragen sollen hier offen bleiben.
Ein ähnlicher Topos findet sich bei
\authorfullcite{Wolff:Discursuspraeliminarisdephilosophiaingenere1996} (siehe
Anm.~\ref{Anmerkung:CognitioArcanaCognitioCommunis} auf
S.~\pageref{Anmerkung:CognitioArcanaCognitioCommunis}). Ebenso setze ich mich
hier nicht mit der Frage auseinander, ob es ausreicht, den Begriff der
Wirklichkeit über eine \emph{tatsächliche} kausale Verbindung zum erkennenden
Subjekt zu erläutern, oder ob stärker berücksichtigt werden muss, dass auch
Dinge wirklich sein können, die in keiner kausalen Verbindung zu uns stehen
(etwa Ereignisse, die jetzt in sehr großer Entfernung von uns geschehen).
\name[Immanuel]{Kant} spricht jedenfalls nicht davon, dass nur wirklich
\emph{ist}, was uns affiziert, sondern dass wir nur dieses als wirklich
\emph{erkennen}.} Was wir so indirekt erkennen, das erkennen wir vermittelt über
Merkmale, die als Erkenntnisgründe dienen. Aber dies setzt doch voraus, dass es
etwas gibt, das wir unmittelbar erkennen. Und diese unmittelbare -- in diesem
Sinne \singlequote{intuitive} -- Erkenntnis kann nur sinnlich sein. Wir erkennen
die Wirklichkeit nicht aus reiner Vernunft.

Die Unterscheidung von Möglichkeit und Wirklichkeit setzt voraus, dass es mit
Sinnlichkeit und Verstand zwei Stämme menschlicher Erkenntnis gibt.
Der Verstand zeichnet für die formalen Bedingungen der Erkenntnis verantwortlich, die
Sinnlichkeit für die materialen. Die Unterscheidung von Wirklichkeit und
Möglichkeit artikuliert gerade die Differenz zwischen diesen Quellen, insofern
etwas möglich ist, sobald die formalen Bedingungen erfüllt sind. Über die
Erfüllung der materialen Bedingungen ist damit noch nichts ausgemacht. Wäre unser
Verstand anschauend, könnte er die materialen Bedingungen unserer Erkenntnis
selbst garantieren. Damit artikulieren die Kategorien der Modalität
keine Eigenschaften der Gegenstände selbst, sondern ihr Verhältnis zu den beiden
\singlequote{Grundquellen unseres Gemüts}. Das ist die besondere Beziehung, die
sie zu der besonderen Beschaffenheit unseres Erkenntnisvermögens haben.

\authorfullcite{Foerster:Die25JahrederPhilosophie2011} stellt die Unterscheidung
von Möglichkeit und Wirklichkeit als Folge der Besonderheit unserer
\emph{Anschauung} und nicht unseres Verstandes
dar.\footnote{\cite[Vgl.][153]{Foerster:Die25JahrederPhilosophie2011}.} Nach
meiner Interpretation ist es hingegen irrelevant, ob wir etwas der Besonderheit
unseres Verstandes oder unserer Anschauung zuschreiben, denn intellektuelle
Anschauungen sind Erkenntnisse eines anschauenden Verstandes. Wichtiger
ist daher \authorcite{Foerster:Die25JahrederPhilosophie2011}s zweiter Schritt,
in welchem er verschiedene Arten intellektueller Anschauung differenziert, denen
unterschiedliche Arten eines anschauenden Verstandes zuzuordnen wären.
Der Einheit von Möglichkeit und Wirklichkeit entspräche eine produktive
Anschauung und dieser wiederum ein Verstand, der als Weltursache zu denken wäre.
Aber in der \titel{Kritik der reinen Vernunft} gibt es eben auch die Konzeption
einer im Vergleich zur unsrigen alternativen Anschauung, mit deren Hilfe die Unterscheidung von Dingen an sich
und Erscheinungen artikuliert wird. Dass wir die Dinge nur als Erscheinungen
erkennen und dass wir zwischen Möglichkeit und Wirklichkeit der Dinge in unserer
Erfahrungserkenntnis unterscheiden, das sind aber allem Anschein nach
unterschiedliche Themen. Und so ließe sich vermuten, dass auch
unterschiedliche Formen unserer Endlichkeit darin zur Sprache kommen.

Nach \authorcite{Foerster:Die25JahrederPhilosophie2011} ist es die
\emph{produktive} intellektuelle Anschauung, die in der Deduktion der Kategorien
bemüht wird und für die in den \singlequote{Postulaten des empirischen Denkens
überhaupt} und in \S~76 der \titel{Kritik der Urteilskraft} Wirklichkeit und
Möglichkeit in eins fallen \emph{respective} diese Unterscheidung verschwindet.
Eine andere Form der intellektuellen Anschauung werde hingegen im Anhang zur
transzendentalen Analytik der \titel{Kritik der reinen Vernunft} und in \S~77
der \titel{Kritik der Urteilskraft} herangezogen, welche nicht den Bedingungen
unserer Sinnlichkeit unterworfen sei und daher die Dinge an sich
erkenne, \emph{ohne} dass sie die Gegenstände ihres Erkennens
hervorbringt, also produktiv ist.\footnote{Siehe oben, Kapitel
\ref{subsection:EineMoeglicheVieldeutigkeit}.}

Ich möchte im folgenden darlegen, dass es sich bei der Anschauung eines
intuitiven Verstandes generell um eine produktive Anschauung handelt und dass die
Konzeption einer nicht produktiven, sondern \singlequote{übersinnlichen}
intellektuellen Anschauung gänzlich unverständlich ist. Unsere Anschauung heißt
sinnlich, \enquote{weil sie nicht ursprünglich, d.\,i. eine solche ist, durch
die selbst das Dasein des Objektes der Anschauung gegeben wird {\punkt}, sondern
von dem Dasein des Objektes abhängig, mithin nur dadurch, daß die
Vorstellungsfähigkeit des Subjekts durch dasselbe affiziert wird, möglich
ist.}\footnote{\cite[][B 72]{Kant:KritikderreinenVernunft2003}, \cite[][III:
72.24--28]{Kant:GesammelteWerke1900ff.}.} Eine Anschauung heißt also
ursprünglich, wenn das \emph{Dasein} des Gegenstandes durch sie gegeben wird.
Dies heißt allem Anschein nach, dass eine Anschauung als ursprünglich gilt, wenn
sie \emph{produktiv} ist, also den Gegenstand hervorbringt. Deswegen kommt eine
solche Anschauung auch nur Gott, niemals aber einem endlichen Wesen zu:
\begin{quote}
Es ist auch nicht nötig, daß wir die Anschauungsart in Raum und Zeit auf die
Sinnlichkeit des Menschen einschränken; es mag sein, daß alles endliche denkende
Wesen hierin mit dem Menschen notwendig übereinkommen müsse, (wiewohl wir dieses
nicht entscheiden können,) so hört sie um dieser Allgemeingültigkeit willen doch
nicht auf Sinnlichkeit zu sein, eben darum, weil sie abgeleitet (intuitus
derivatus), nicht ursprünglich (intuitus originarius), mithin nicht
intellektuelle Anschauung ist, als welche aus dem eben angeführten Grunde allein
dem Urwesen, niemals aber einem, seinem Dasein sowohl als seiner Anschauung nach
(die sein Dasein in Beziehung auf gegebene Objekte bestimmt), abhängigen Wesen
zuzukommen scheint[.]\footnote{\cite[][B 72]{Kant:KritikderreinenVernunft2003},
\cite[][III: 72.29--73.2]{Kant:GesammelteWerke1900ff.}. Schon in seiner
Inauguraldissertation unterscheidet er in ähnlicher Manier unsere Anschauung als
leidend von einer urbildlichen göttlichen Anschauung: \enquote{\ori{Intuitus}
nempe mentis nostrae semper est \ori{passivus}; adeoque eatenus tantum, quatenus
aliquid sensus nostros afficere potest, possibilis. Divinus autem intuitus, qui
obiectorum est principium, non principiatum, cum sit independens, est archetypus
et propterea perfecte intellectualis}
\mkbibparens{\cite[][\S~10]{Kant:Demundisensibilisatqueintelligibilisformaetprincipiis1968},
\cite[][II: 396.31--397.4]{Kant:GesammelteWerke1900ff.}}.}
\end{quote}
Nach diesem Zitat sind unterschiedliche Formen sinnlicher Anschauung möglich,
die von der Form, die uns Menschen eigen ist, abweichen können. Aber eine
Anschauung, die nicht sinnlich ist, sei doch nur als ursprüngliche Anschauung
denkbar, die ausschließlich Gott zugeschrieben werden könne.
\name[Immanuel]{Kant} sagt damit, dass es nur eine einzige Konzeption einer
nicht-sinnlichen Anschauung geben könne: die ursprüngliche, also produktive
Anschauung Gottes.


Doch zunächst ist zu konzidieren, dass es tatsächlich Textbelege für die
gegenteilige Interpretation gibt. Worauf
\authorcite{Foerster:Die25JahrederPhilosophie2011} sich bezieht ist die
nicht-sinnliche Anschauung, die laut \titel{Phaenomena und Noumena}-Kapitel am
Begriff eines \emph{Noumenon} in positiver Bedeutung beteiligt ist:
\begin{quote}
Wenn wir unter Noumenon ein Ding verstehen, \ori{so fern es nicht Objekt
unserer sinnlichen Anschauung ist}, indem wir von unserer Anschauungsart
desselben abstrahieren; so ist dieses ein Noumenon im \ori{negativen} Verstande.
Verstehen wir aber darunter ein Objekt einer \ori{nicht-sinnlichen Anschauung},
so nehmen wir eine besondere Anschauungsart an, nämlich die intellektuelle, die
aber nicht die unsrige ist, von welcher wir auch die Möglichkeit nicht einsehen
können, und das wäre das Noumenon in \ori{positiver}
Bedeutung.\footnote{\cite[][B 307]{Kant:KritikderreinenVernunft2003},
\cite[][III: 209.32--210.2]{Kant:GesammelteWerke1900ff.}.}
\end{quote}
Die Lehre von der Sinnlichkeit, die die Erscheinungen von den Dingen an sich
abgrenzt, rekurriert nun lediglich auf \emph{Noumena} in negativer
Bedeutung.\footnote{\cite[Vgl.][B 307--309]{Kant:KritikderreinenVernunft2003},
\cite[][III: 210.3--34]{Kant:GesammelteWerke1900ff.}.} Um nun auch nur zu
beweisen, dass ein solches \emph{Noumenon} in positiver Bedeutung \emph{möglich} ist,
bedürfte es einer Anschauung, die nicht sinnlich und \emph{a fortiori}
intellektuell ist. Eine solche Anschauung scheint nicht produktiv sein zu
müssen; sie gäbe einem zugehörigen (anschauenden) Verstand lediglich ein
Mannigfaltiges, ohne dafür der Rezeptivität zu bedürfen.


\name[Immanuel]{Kant} spricht selbst von
anderen denkbaren Formen der Anschauung, auf die ebenso Kategorien anwendbar
wären, insofern sie uns ein Mannigfaltiges darböten, das erst zu synthetisieren
wäre.\footnote{\cite[Vgl.][\S~23]{Kant:KritikderreinenVernunft2003},
\cite[][III: 118.7--10]{Kant:GesammelteWerke1900ff.}.} Eine solche Anschauung
scheint daher ebenso wie unsere menschliche Anschauung ein Mannigfaltiges zu
geben und keineswegs produktiv zu sein. Allerdings handelt es sich nach seiner
Auskunft dabei eben um eine andere Form \emph{sinnlicher} Anschauung. Zu Beginn
der transzendentalen Ästhetik findet sich jedoch eine Stelle, nach der in einer
intellektuellen Anschauung Gegenstände (nicht-sinnlich) \emph{gegeben} werden
können:
\begin{quote}
Auf welche Art und durch welche Mittel sich auch immer eine Erkenntnis auf
Gegenstände beziehen mag, so ist doch diejenige, wodurch sie sich auf dieselbe
unmittelbar bezieht, und worauf alles Denken als Mittel abzweckt, die
\ori{Anschauung}. Diese findet aber nur statt, so fern uns der Gegenstand
gegeben wird; dieses aber ist wiederum, uns Menschen wenigstens, nur dadurch
möglich, daß er das Gemüt auf gewisse Weise
affiziere.\footnote{\cite[][B 33]{Kant:KritikderreinenVernunft2003},
\cite[][III: 49.6--11]{Kant:GesammelteWerke1900ff.}. Der Einschub \enquote{uns
Menschen wenigstens} ist ein Zusatz der Auflage von 1787.}
\end{quote}
Dieses Zitat enthält zwei wichtige Aussagen: (i) Dass uns durch sie ein
Gegenstand \emph{gegeben} wird, gehört zur Anschauung \emph{per se} (also unabhängig
davon, ob es sich um eine sinnliche oder eine intellektuelle Anschauung
handelt). (ii)  Uns Menschen kann ein Gegenstand nur dadurch gegeben werden,
dass er uns affiziert, also nur mittels unserer Sinnlichkeit. Demnach ist denkbar, dass
einem Wesen ein Gegenstand durch intellektuelle Anschauung gegeben wird. Also --
so scheint es -- kann es eine intellektuelle Anschauung geben, die ich oben
\singlequote{übersinnlich} nannte, also eine solche, durch die ein Gegenstand
nicht hervorgebracht, sondern auf nicht-sinnliche Weise gegeben wird.


Dass eine Anschauung sinnlich ist, heißt, dass sie dadurch entsteht, dass
unser Erkenntnisvermögen von etwas affiziert wird. Während im Fall der
ursprünglichen Anschauung der Gegenstand der Anschauung seinem Dasein nach von
der Anschauung abhängt, hängt die sinnliche Anschauung von dem durch sie
repräsentierten Gegenstand ab. Die wichtige Frage ist nun, was mit einer
Anschauung gemeint sein soll, die weder sinnlich, noch produktiv, sondern
\singlequote{übersinnlich} ist, und ob es einen Begriff einer solchen
übersinnlichen Anschauung im Werk \name[Immanuel]{Kant}s
gibt.\footnote{Dass es ein drittes neben sinnlicher und übersinnlicher
  Anschauung gibt, behauptet auch
  \authorfullcite{Baum:DeduktionundBeweisinKantsTranszendentalphilosophie1986}, 
der nicht wie \authorcite{Foerster:Die25JahrederPhilosophie2011} und
\authorcite{Prien:KantsLogikderBegriffe2006} zwischen verschiedenen Formen
intellektueller Anschauung differenziert, sondern diese mit der produktiven
Anschauung identifiziert und der \singlequote{übersinnlichen} Anschauung
gegenüberstellt -- freilich ohne zu sagen, wie eine solche Anschauung denkbar
sein soll: \enquote{Diese Anschauung ist bei uns Menschen nicht intellektuell,
könnte aber immer noch entweder sinnlich oder auf andere als auf intellektuelle
Weise nicht-sinnlich sein, denn es besteht kein Widerspruch zwischen der Leugnung
einer intellektuellen Anschauung und der gegen sinnliche und nicht-sinnliche
Anschauung indifferenten Verstandesdefinition, weil zwischen sinnlicher und
intellektueller Anschauung kein kontradiktorischer, sondern ein konträrer
Gegensatz vorliegt}
\parencite[][83--85]{Baum:DeduktionundBeweisinKantsTranszendentalphilosophie1986}.
Mir scheinen \enquote{sinnlich} und \enquote{intellektuell} kontradiktorische
Gegensätze zu bezeichnen, insofern sie den beiden Erkenntnisquellen
\emph{Rezeptivität} und \emph{Spontaneität} entsprechen, wie ich in Kapitel
\ref{subsubsection:BegriffderDiskursivitaet} weiter ausgeführt habe.} 
Rezeptive Vermögen nennt \name[Immanuel]{Kant} Sinnlichkeit, unabhängig davon,
ob sie \singlequote{unserer} Sinnlichkeit
entsprechen.\footnote{\cite[Vgl.][B 75]{Kant:KritikderreinenVernunft2003},
\cite[][III: 75.5--7]{Kant:GesammelteWerke1900ff.}, und ausführlicher oben in
Kapitel \ref{subsection:VerstandundRezeptivitaet}.} Vielleicht ist ein Wesen denkbar,
das ebenso über Sinnlichkeit verfügt, deren Form aber nicht der unsrigen
entspricht und nicht auf Zeit und Raum beruht.\footnote{Dafür, dass Zeit und
Raum die Formen unserer Anschauung sind, können keine weiteren Gründe angegeben
werden, es ist folglich kontingent
\mkbibparens{\cite[vgl.][\S~21]{Kant:KritikderreinenVernunft2003}, \cite[][III:
116.23--29]{Kant:GesammelteWerke1900ff.}}.} Zu beachten ist, dass eine
Anschauung in \name[Immanuel]{Kant}s Sprachgebrauch auch
dann als sinnlich gilt, wenn sie nicht den Formen \emph{unserer} Sinnlichkeit
(Zeit und Raum) unterliegt. Es reicht, dass das Mannigfaltige dadurch gegeben
wird, dass es uns in irgendeiner Art und Weise \emph{affiziert}. Was es
heißen soll, dass es weder sinnlich gegeben, noch von uns hervorgebracht wird,
scheint mir mehr als fraglich zu sein. Es müsste heißen, dass weder
die Anschauung von dem Dasein des Gegenstandes noch dessen Dasein von
der Anschauung abhängt. Dennoch soll es freilich -- so ist wenigstens
anzunehmen -- eine notwendige Verbindung zwischen beiden geben. Auch nach
\authorfullcite{Duesing:NaturteleologieundMetaphysikbeiKantundHegel1990}
folgt daher aus der Spontaneität einer (intellektuellen) Anschauung ihre
Produktivität.\footnote{\cite[Vgl.][144]{Duesing:NaturteleologieundMetaphysikbeiKantundHegel1990}.}
\name[Immanuel]{Kant} selbst sagt, es sei kaum einsehbar, wie sich
Vorstellungen anders auf Gegenstände beziehen können, als dadurch, dass (a) der
Gegenstand die Vorstellung möglich macht, indem er das Subjekt affiziert, oder
(b) die Vorstellung den Gegenstand möglich macht, indem sie ihn
(b\textsubscript{1}) wie im Falle von Handlungen hervorbringt oder
(b\textsubscript{2}) indem sie ihn \emph{als Gegenstand} möglich macht, wie die die Kategorien
leisten.\footnote{\cite[Vgl.][\S~14]{Kant:KritikderreinenVernunft2003},
\cite[][III: 104.5--17]{Kant:GesammelteWerke1900ff.}. Siehe auch den Brief an
Markus \name[Marcus]{Herz} vom 21. Februar 1772
\parencite[][X: 124\,f.]{Kant:GesammelteWerke1900ff.} sowie meine Überlegungen
in Kap. \ref{subsection:MetaphysikundAutonomie} ab S.
\pageref{AutonomiedesVerstandes}.} Wenn wir den Sonderfall (b\textsubscript{2})
einmal außen vor lassen, bleibt nur, dass eine Anschauung sich entweder sinnlich
oder produktiv auf einen Gegenstand beziehen kann. In einer Anmerkung in seinem
Handexemplar der \titel{Metaphysica}
\authorcite{Baumgarten:Metaphysica---Metaphysik2011}s bemerkt \name[Immanuel]{Kant} ebenfalls, dass
nur Gott ein intuitiver Verstand zugeschrieben werden könne, weil dessen
Erkenntnisart nicht anders als produktiv gedacht werden
könne.\footnote{\enquote{Es ist schwerlich zu begreifen, wie ein anderer
intuitiver Verstand statt finden solte als der gottliche. Denn der erkennet in
sich als Urgrunde (und archetypo) aller Dinge Moglichkeit; aber endliche Wesen
können nicht aus sich selbst andere Dinge erkennen, weil sie nicht ihre Urheber
sind, es sey denn die bloße Erscheinungen, die sie a priori erkennen könen. 	  	  	 
Daher können wir die Dinge an sich selbst nur in Gott
erkennen} \mkbibparens{\cite[][\nopp 6048]{Kant:Reflexionen1900ff.},
\cite[][XVII: 433.16--21]{Kant:GesammelteWerke1900ff.}}. Nach
\name[Erich]{Adickes} stammt sie aus den Jahren 1783--1784. Damit lässt sich
natürlich anknüpfend an \authorcite{Foerster:Die25JahrederPhilosophie2011} auch
hier einwenden, dass \name[Immanuel]{Kant} sich die relevanten Unterschiede eben
erst später -- zwischen der zweiten Auflage der \titel{Kritik der reinen
Vernunft} 1787 und der \titel{Kritik der Urteilskraft} 1790 --  klar machen
konnte \parencite[vgl.][\pno~177,
Anm.]{Foerster:DieBedeutungvonSS7677deremphKritikderUrteilskraftfuerdieEntwicklungdernachkantischenPhilosophieTeil12002}.}

An der zitierten Stelle spricht \name[Immanuel]{Kant} davon,
dass \emph{jede} Anschauung nur stattfinden kann, insofern ihr Gegenstand
gegeben wird. Dies trifft also auch für die produktive Anschauung
zu. Damit dies nicht widersprüchlich ist, muss \name[Immanuel]{Kant}
auch dort davon sprechen, dass der Gegenstand gegeben wird, wo er hervorgebracht
wird. Der Gegenstand kann also dadurch gegeben werden, dass er von dem
erkennenden (unendlichen) Subjekt hervorgebracht wird, oder dadurch, dass er ein
erkennendes (endliches) Subjekt affiziert. In dieser Deutung entfiele wiederum
die Notwendigkeit, eine Doppeldeutigkeit des Begriffs \enquote{intellektuelle
Anschauung} anzunehmen. Die Argumente, die für die Annahme einer
Doppeldeutigkeit des Ausdrucks \enquote{intellektuelle Anschauung} sprechen,
lassen sich somit entkräften, während etliche Gründe exegetischer wie
philologischer Art für die Annahme sprechen, dass es nur einen Sinn von
\enquote{intellektuelle Anschauung} gibt und dass eine intellektuelle Anschauung
in diesem Sinne immer produktiv ist.

\subsection{Intuitiver Verstand und das
Synthetisch-Allgemeine}\label{subsection:IntuitiverVerstandunddasSynthetischAllgemeine}
Die Diskursivität des Verstandes und die Sinnlichkeit unserer Anschauung wurden
mit einem intuitiven Verstand und einer intellektuellen Anschauung verglichen, um zu
zeigen, dass die Unterscheidung von Möglichkeit und Wirklichkeit nur für
endliche Wesen gilt (Kapitel
\ref{subsubsection:UnterscheidungvonDenkenundErkennen}). Im nun zu besprechenden
Fall möchte \name[Immanuel]{Kant} zeigen, dass es der besonderen Beschaffenheit
unseres Erkenntnisvermögens geschuldet ist, dass wir Naturprodukte als Wirkungen
einer nach Zwecken handelnden Kausalität denken müssen. Er stellt dabei unseren diskursiven
einem als möglich angenommenen intuitiven Verstand gegenüber. Ziel der
Gegenüberstellung ist es zu zeigen, dass die Notwendigkeit, Naturprodukte als
Produkte einer nach Zwecken und Endursachen wirkenden Kausalität anzusehen,
lediglich der Endlichkeit unseres Verstandes geschuldet ist und nicht in den
Dingen selbst liegt.\footnote{\cite[][\S~77]{Kant:KritikderUrteilskraft2009},
\cite[][V: 408.2--13]{Kant:GesammelteWerke1900ff.}.} Außerdem verweist
\name[Immanuel]{Kant} auf einen
\singlequote{anderen} Verstand, weil die Besonderheit unseres Verstandes
es nötig mache, \enquote{den obersten Grund dazu in einem ursprünglichen Verstande als
Weltursache zu suchen.}\footnote{\cite[][\S~77]{Kant:KritikderUrteilskraft2009},
\cite[][V: 410.10--11]{Kant:GesammelteWerke1900ff.}.} Die Konzeption eines
anderen Verstandes kommt hier also in zwei verschiedenen Funktionen vor,\footnote{Zur Doppelfunktion des Verstandes siehe bspw.
\cite[][68]{Duesing:DieTeleologieinKantsWeltbegriff1968}, und oben
Anm. \ref{FussnoteDoppelfunktion} auf S. \pageref{FussnoteDoppelfunktion}.} von denen primär die
erste interessiert.\footnote{Die zweite Funktion beschreibt den
\singlequote{anderen} Verstand als produktiv und stellt definitiv keinerlei
Abweichung von der Interpretation in Kap.
\ref{subsection:DiskursiverVerstandundsinnlicheAnschauung} dar.}

Für die folgenden Rekonstruktionen sind einige Vorbemerkungen nötig:
Argumentation und Darstellung in \S~77 sind von vielen Vorannahmen abhängig, für
die \name[Immanuel]{Kant} in den vorherigen Abschnitten der \titel{Analytik der
teleologischen Urteilskraft} Beweise vorzulegen beansprucht. Die dort
entwickelten Positionen sind bekanntlich höchst umstritten. Dies liegt zumindest teilweise in den
Fortschritten begründet, die insbesondere die Biologie in den an
\name[Immanuel]{Kant} anschließenden beiden Jahrhunderten machte. Ich kann hier
weder den gesamten systematischen Rahmen nachzeichnen, noch eine Verteidigung
und Rechtfertigung der kantischen Position vornehmen. Ich werde beispielsweise
nicht fragen, ob \name[Charles]{Darwin} der \singlequote{\name[Isaac]{Newton}
des Grashalms} ist. Mir geht es ausschließlich um die Beantwortung der Frage, ob
es sich bei der Darstellung der Besonderheit unseres Verstandes in \S~77 um
dieselbe Besonderheit handelt, die bereits als Endlichkeit unseres
Erkenntnisvermögens ausgemacht wurde.

Zu Beginn des \S~77 der \titel{Kritik der Urteilskraft} stellt
\name[Immanuel]{Kant} fest, dass \enquote{die Ursache der Möglichkeit} des
\enquote{Begriff[s] eines Naturzwecks {\punkt} nur in der Idee liegen
kann}\footnote{\cite[][\S~77]{Kant:KritikderUrteilskraft2009}, \cite[][V:
405.9--12]{Kant:GesammelteWerke1900ff.}.}, wenngleich das
Naturprodukt selbst doch in der Natur gegeben sei; darin unterscheide sich die
Idee eines Naturzwecks von anderen Ideen.
Nach \name[Immanuel]{Kant} ist etwas ein Naturzweck, wenn es von sich selbst
sowohl Ursache als auch Wirkung
ist.\footnote{\cite[Vgl.][\S~64]{Kant:KritikderUrteilskraft2009},
\cite[][V: 370.35--37]{Kant:GesammelteWerke1900ff.}. \name[Immanuel]{Kant}
bezeichnet diese Angabe zunächst als \singlequote{vorläufig}.}
Und der bisherige Textverlauf präsentierte bereits die Annahme, dass der Begriff
eines organisierten Wesens, das als Naturzweck betrachtet wird, nicht
konstitutiv, sondern lediglich regulativ
ist und auf der besonderen Beschaffenheit unseres Erkenntnisvermögens
beruht.\footnote{\enquote{Es ist doch etwas ganz anderes, ob ich sage: die
Erzeugung gewisser Dinge der Natur, oder auch der gesamten Natur, ist nur durch
eine Ursache, die sich nach Absichten zum Handeln bestimmt, möglich, oder: ich
kann \ori{nach der eigentümlichen Beschaffenheit meiner Erkenntnisvermögen}
über die Möglichkeit jener Dinge und ihre Erzeugung nicht anders urteilen, als
wenn ich mir zu dieser eine Ursache, die nach Absichten wirkt, mithin ein Wesen
denke, welches nach der Analogie mit der Kausalität eines Verstandes produktiv
ist} \mkbibparens{\cite[][\S~75]{Kant:KritikderUrteilskraft2009},
\cite[][V: 397.31--398.3]{Kant:GesammelteWerke1900ff.}}.
\cite[Vgl.][\pno~385\,f.]{Guyer:FromNaturetoMorality2001}.
Siehe dazu auch \cite{Stolzenberg:OrganismusundUrteilskraft2001}.}
Darin stimmt die Idee eines Naturzwecks mit den Ideen einer
unbedingten Notwendigkeit des Urgrundes der Natur und einer unbedingten
Kausalität überein. Doch es besteht ein wichtiger Unterschied: Die Idee eines
Naturzwecks scheint ein konstitutives Prinzip zu sein, insofern Naturprodukte
doch in der Erfahrung gegeben sind. Dieser Unterschied habe seinen Ursprung
darin, dass die Idee ein Vernunftprinzip für die reflektierende Urteilskraft --
für \enquote{die Anwendung eines Verstandes überhaupt auf mögliche Gegenstände
der Erfahrung} -- sei und somit \enquote{eine Eigentümlichkeit \ori{unseres}
(menschlichen) Verstandes in Ansehung der Urteilskraft, in der Reflexion
derselben über Dinge der
Natur}\footnote{\cite[][\S~77]{Kant:KritikderUrteilskraft2009}, \cite[][V:
405.25--27]{Kant:GesammelteWerke1900ff.}.} betreffe. Wenn dies aber stimme
(wenn die Idee eines Naturzwecks der Beschaffenheit unseres Verstandes
geschuldet ist), dann müsse \enquote{die Idee von einem anderen möglichen
Verstande als dem menschlichen zum Grunde
liegen}\footnote{\cite[][\S~77]{Kant:KritikderUrteilskraft2009},
\cite[][V: 405.27--28]{Kant:GesammelteWerke1900ff.}.} und dann sei zu fragen, was
denn die Besonderheit unseres Verstandes ist und worin er sich von dem gedachten
Vergleichsverstand unterscheidet.

Nun sagt \name[Immanuel]{Kant}, wir müssten eine \emph{Zufälligkeit} der
Beschaffenheit unseres Verstandes aufsuchen. Und diese
fänden wir \enquote{ganz natürlich in dem \ori{Besonderen}, welches die
Urteilskraft unter das \ori{Allgemeine} der Verstandesbegriffe bringen soll; denn durch das Allgemeine
\ori{unseres} (menschlichen) Verstandes ist das Besondere nicht
bestimmt}\footnote{\cite[][\S~77]{Kant:KritikderUrteilskraft2009},
\cite[][V: 406.11--14]{Kant:GesammelteWerke1900ff.}.}. Weil unser Verstand ein
\emph{diskursiver} Verstand ist, sei es für ihn
zufällig, was er in der Natur an Besonderem vorfindet. Die Verstandesbegriffe,
unter die das Besondere der Natur zu bringen ist, sind die Kategorien; in Bezug
auf diese verhält sich unsere Urteilskraft bestimmend. Da ein endlicher Verstand
aber entsprechend \emph{a priori} zwar das Mögliche, nicht aber das Wirkliche
erkennen kann, sondern dafür auf die Sinne angewiesen ist, ist doch alles
Wirkliche in der Natur aus der Sicht des endlichen Verstandes
zufällig.

Bis hierher handelt es sich um \emph{dieselbe} Besonderheit unseres
Verstandes, die auch schon im vorigen Kapitel
\ref{subsubsection:UnterscheidungvonDenkenundErkennen} beschrieben
wurde. Ein intuitiver Verstand, der selbst anschaute, müsste nicht auf
die Sinne rekurrieren, da ihm das Besondere durch die eigene Tätigkeit
gegeben wäre -- er wäre \enquote{ein Vermögen einer \ori{völligen}
  Spontaneität der Anschauung}\footnote{\cite[][\S~77]{Kant:KritikderUrteilskraft2009},
  \cite[][V: 406.21--22]{Kant:GesammelteWerke1900ff.}.}. Es gäbe für
ihn keine Differenz von Möglichkeit und Wirklichkeit und \emph{a
  fortiori} keine Zufälligkeit. Wenn \name[Immanuel]{Kant} sagt, dass
\emph{unser} Verstand die Besonderheit an sich habe, \enquote{daß er
  in seinem Erkenntnisse z.\,B. der Ursache eines 
Produkts, vom \ori{Analytisch-Allgemeinen} (von Begriffen) zum Besonderen (der
gegebenen empirischen Anschauung) gehen}\footnote{\cite[][\S~77]{Kant:KritikderUrteilskraft2009},
\cite[][V: 407.13--16]{Kant:GesammelteWerke1900ff.}.} müsse, dann drückt dies also
zunächst keine neue Besonderheit, sondern die uns längst bekannte Endlichkeit unseres oberen
Erkenntnisvermögens aus. Bei den angesprochenen Begriffen handelt es sich um die
Kategorien, in dem genannten Beispiel geht es also um die Kategorie der Ursache. Von diesen
Begriffen geht der Verstand aus, um sinnliche Wahrnehmungen mittels der Urteilskraft darunter
zu subsumieren. Anders verführe ein intuitiver Verstand, der nicht von
allgemeinen Begriffen, sondern von Anschauungen ausginge:
\begin{quote}
Nun können wir uns aber auch einen Verstand denken, der, weil er nicht wie der
unsrige diskursiv, sondern intuitiv ist, vom \ori{Synthetisch-Allgemeinen} (der
Anschauung eines Ganzen als eines solchen) zum Besonderen geht, d.\,i. vom
Ganzen zu den Teilen; der also und dessen Vorstellung des Ganzen die
\ori{Zufälligkeit} der Verbindung der Teile nicht in sich enthält, um eine
bestimmte Form des Ganzen möglich zu machen, die unser Verstand bedarf, welcher
von den Teilen als allgemein gedachten Gründen zu verschiedenen darunter zu
subsumierenden möglichen Formen als Folgen fortgehen
muss.\footnote{\cite[][\S~77]{Kant:KritikderUrteilskraft2009},
\cite[][V: 407.19--28]{Kant:GesammelteWerke1900ff.}.}
\end{quote}
Die Anschauungen des intuitiven Verstandes wären -- zumindest nach meiner Lesart
-- keine sinnlichen, sondern intellektuelle Anschauungen, also Ausdruck der
Spontaneität unseres Erkenntnisvermögens, welches das Besondere aus sich selbst
heraus zu generieren verstünde. Für ein solches \enquote{Vermögen einer
\ori{völligen Spontaneität der
Anschauung}}\footnote{\cite[][\S~77]{Kant:KritikderUrteilskraft2009},
\cite[][V: 406.21--22]{Kant:GesammelteWerke1900ff.}.} gäbe es keinen Unterschied
zwischen den Modalitäten der Möglichkeit, Wirklichkeit und Notwendigkeit und
\emph{a fortiori} keine Zufälligkeit. Letztere ist nur Ausdruck der Tatsache,
dass unser Erkennen auf zwei grundlegenden Erkenntnisquellen beruht und das
Besondere nur durch die sinnliche Anschauung gegeben werden kann.

Diese Lesart bestätigt sich auch  bei Betrachtung der Redeweise von analytisch-
und synthetisch-allgemeinem. Der diskursive Verstand geht von Begriffen, der
intuitive Verstand von Anschauungen aus zum Besonderen. Die Begriffe des
diskursiven Verstandes sind analytisch-allgemein, die Anschauungen des
intuitiven Verstandes sind synthetisch-allgemein. Doch was bedeuten die
Ausdrücke \enquote{ana\-ly\-tisch-all\-ge\-mein} und
\enquote{syn\-the\-tisch-all\-ge\-mein}? In der {\jaeschelogik} ist hierzu zu lesen:
\begin{quote}
Allgemeine Regeln sind entweder \ori{analytisch} oder \ori{synthetisch}
allgemein. \ori{Jene} abstrahieren von den Verschiedenheiten; \ori{diese}
attendieren auf die Unterschiede und bestimmen folglich doch auch in Ansehung
ihrer.\footnote{\cite[][\S~21]{Kant:ImmanuelKantsLogik1977},
\cite[][IX: 102.31--33]{Kant:GesammelteWerke1900ff.}. Die Vorlage zu dieser
Textstelle in \name[Immanuel]{Kant}s Ausgabe des
\authorcite{Meier:AuszugausderVernunftlehre1752}schen Logikbuchs lautet:
\enquote{Allgemeine Regeln sind entweder
\ori{analytisch allgemein}: indem sie von den Verschiedenheiten
abstrahieren, oder \ori{synthetisch allgemein}: und diese attendieren
auf die Unterschiede, bestimmen doch auch in Ansehung ihrer}
\mkbibparens{\cite[][\nopp 3086]{Kant:Reflexionen1900ff.},
\cite[][XVI: 651.5--8]{Kant:GesammelteWerke1900ff.}}.}
\end{quote}
Begriffe sind allgemeine Regeln; und dass sie analytisch-allgemein sind, heißt,
dass sie von von den Verschiedenheiten der unter sie fallenden Gegenstände
abstrahieren. Der Begriff kann als Regel aufgefasst
werden, insofern er Merkmale enthält, unter die alle Gegenstände fallen, die
unter den Begriff fallen. Wenn Begriffe Merkmalskomplexionen sind, dann müssen wir
in entsprechenden Urteilen nicht über die Begriffe hinausgehen, um ihre Wahrheit
einzusehen. Es handelt sich daher um \emph{analytische} Urteile (was erklärt,
warum \name[Immanuel]{Kant} von analytisch-allgemeinen Regeln spricht). Der
Begriff Junggeselle hat als Merkmal beispielsweise \enquote{unverheiratet} --
dass Junggesellen unverheiratet sind ist ein analytisches Urteil. Merkmale wie
dieses treffen auf jeden Junggesellen zu, in ihrer Hinsicht sind sie alle
gleich. Die Verschiedenheiten unter den Junggesellen kommen unter diesen
Merkmalen nicht vor. Begriffe -- sagt \name[Immanuel]{Kant} in der \titel{Kritik
der reinen Vernunft} -- stehen unter dem Grundsatz der Bestimmbarkeit, ein
Begriff ist aber \enquote{in Ansehung dessen, was in ihm selbst nicht enthalten
ist, unbestimmt}\footnote{\cite[][B 599]{Kant:KritikderreinenVernunft2003},
\cite[][III: 385.18--19]{Kant:GesammelteWerke1900ff.}.}. Eine
synthetisch-allgemein Regel müsste alle Gegenstände, auf die sie anwendbar ist,
auch in ihren Besonderheiten bestimmen. Sie entspräche dem Grundsatz der durchgängigen
Bestimmung, unter dem Gegenstände nach Auskunft der \titel{Kritik der reinen
Vernunft} stehen.\footnote{\cite[Vgl.][B 599]{Kant:KritikderreinenVernunft2003},
\cite[][III: 385.25--28]{Kant:GesammelteWerke1900ff.}.} Eine
synthetisch-allgemeine Regel wäre also eine solche Regel, die alle in ihren
Anwendungsbereich fallenden Gegenstände vollständig bestimmt.

Während analytische Allgemeinheit durch Begriffe zustande
kommt,\footnote{\enquote{Je einfacher ein obiect gedacht wird, desto eher ist
analytische allgemeinheit zufolge eines Begrifs moglich}
\mkbibparens{\cite[][\nopp 3086]{Kant:Reflexionen1900ff.},
\cite[][XVI: 651.8--9]{Kant:GesammelteWerke1900ff.}}. Siehe auch die
Parallelstelle in \cite[][\S~21]{Kant:ImmanuelKantsLogik1977},
\cite[][IX: 102.33--103.2]{Kant:GesammelteWerke1900ff.}.} ist die Möglichkeit
synthetischer Allgemeinheit fraglich. Da es sich nicht um Begriffe handeln kann,
schlägt \authorfullcite{Nuzzo:KantandtheUnityofReason2005} vor, das
Synthetisch-Allgemeine als Anschauung etwa in den reinen Anschauungsformen Raum
und Zeit zu sehen. Während alle Gegenstände \emph{unter} den reinen
Verstandesbegriffen enthalten sind, sind sie zugleich \emph{in} Raum und Zeit
enthalten.\footnote{\cite[Vgl.][351]{Nuzzo:KantandtheUnityofReason2005}.} Dies
wiederum harmoniert mit einer anderen Textstelle:
In einer Anmerkung in seinem Exemplar der \titel{Metaphysica}
\authorcite{Baumgarten:Metaphysica---Metaphysik2011}s notiert
\name[Immanuel]{Kant}, dass zwischen einer \singlequote{intuitiven
Allgemeinheit} und einer \singlequote{diskursiven Allgemeinheit} zu
unterscheiden sei und dass das Besondere \singlequote{\emph{unter}} dem
dis\-kur\-siv-All\-ge\-mei\-nen, aber \singlequote{\emph{in}} dem
in\-tui\-tiv-All\-ge\-mei\-nen enthalten sei.\footnote{\enquote{Die intuitive
Allgemeinheit ist von der discursiven zu unterscheiden.
In der letzteren ist das besondere nicht im allgemeinen, sondern unter ihm 	  	  	 
enthalten, wohl aber in der
ersteren} \mkbibparens{\cite[][\nopp 6178]{Kant:Reflexionen1900ff.},
\cite[][XVIII: 481.4--6]{Kant:GesammelteWerke1900ff.}.}.}
Die Relationen \enquote{ist enthalten in} und \enquote{ist enthalten unter}
bestimmen Inhalt und Umfang von
Begriffen;\footnote{Vgl. \cite[][87]{Stuhlmann-Laeisz:KantsLogik1976}. Siehe
etwa \cite[][\S~7]{Kant:ImmanuelKantsLogik1977},
\cite[][IX: 95.27--30]{Kant:GesammelteWerke1900ff.}.} den Inhalt stellen
seine Merkmale dar, den Umfang all diejenigen Vorstellungen, von denen er
korrekt ausgesagt wird.\footnote{Nach dieser Redeweise sind Anschauungen $A^1, A^2$ und Begriffe $P^1, P^2$
\emph{unter} einem Begriff $P$ enthalten, wenn gilt:
$A^1$ ist $P$, $A^2$ ist $P$, $P^1$ ist $P$ und $P^2$ ist $P$. Umgekehrt ist $P$
in diesem Fall \emph{in} den Vorstellungen $A^1, A^2, P^1, P^2$ enthalten (Er
ist ein \emph{Merkmal} dieser Vorstellungen).} Wir betrachten als Beispiel den
Begriff des Hundes; \emph{unter} diesem Begriff finden sich sich etliche weitere
Vorstellungen wie die Begriffe des Schäferhundes, des Dackels und des Pudels,
aber auch Anschauungen wie die von Bello oder
Snoopy.\footnote{\singlequote{Unter} einem Begriff enthalten sind also
Anschauungen und Begriffe, nicht nur Begriffe -- wie in der traditionellen Logik
des 18. Jahrhunderts -- und nicht Gegenstände -- wie in der Logik ab
\authorcite{Frege:DieGrundlagenderArithmetik1988}
\parencite[vgl.][\pno~87\,f.]{Stuhlmann-Laeisz:KantsLogik1976}.} Es handelt sich
hier um das Besondere, das unter dem Begriff des Hundes enthalten ist. Der
Begriff ist in dieser Hinsicht diskursiv-allgemein.

Damit lässt sich auch die Gegenüberstellung von ana\-ly\-tisch-all\-ge\-mei\-nem
und syn\-the\-tisch-all\-ge\-mei\-nem als Ausgangspunkt des diskursiven
\emph{respective} intuitiven Verstandes im von mir angenommenen Paradigma des endlichen Verstandes
interpretieren. Warum sollte es einen Grund geben zu behaupten, dass hier eine
ganz andere Konzeption eines \singlequote{anderen} Verstandes zugrunde liege, die
nicht einmal fordern muss, dass dieser andere Verstand kein endlicher Verstand
ist?

Nach \authorcite{Foerster:Die25JahrederPhilosophie2011}\footnote{\authorcite{Foerster:Die25JahrederPhilosophie2011}
schreibt die folgenden Überlegungen ihrem Ursprung nach \name[Johann Wolfgang
von]{Goethe} zu, scheint sie aber auch selbst für zutreffend zu halten. Ich
werde mich nicht mit der Frage befassen, inwieweit es sich um eine authentische
\name[Johann Wolfgang von]{Goethe}-Interpretation handelt, sondern
die Überlegungen \authorcite{Foerster:Die25JahrederPhilosophie2011} zuschreiben.} scheitert
\name[Immanuel]{Kant} bei dem Versuch, das Wachstum und die Metamorphosen von
Organismen zu beschreiben. Denn Begriffe seien in dessen Theorie durch Merkmale
bestimmt, die auf alle unter den Begriff fallenden Gegenstände zutreffen und ihn
damit gegen andere abgrenzen. Im Ergebnis handle es sich dann aber um statische
Begriffe, die den Veränderungen nicht gerecht werden, die die unter sie
fallenden Gegenstände erleiden. Um metamorphosierende Organismen zu verstehen,
bräuchten wir Begriffe, die sich quasi mit den Gegenständen
wandeln.\footnote{\enquote{Das heißt, ich muss eine Möglichkeit finden, den
Begriff selbst so beweglich und veränderlich zu machen, dass er die Entwicklung
seines Gegenstandes mitvollziehen kann. Genauer gesagt, muss ich das Denken so
in die Anschauung versenken, den Begriff, mit dem der erste Zustand gedacht
wird, so plastisch oder flüssig machen, dass er sich mit dem metamorphosierenden
Gegenstand entwickelt}
\parencite[][183]{Foerster:DieBedeutungvonSS7677deremphKritikderUrteilskraftfuerdieEntwicklungdernachkantischenPhilosophieTeil12002}.}
Dies lässt sich an einem Beispiel illustrieren:
Der Begriff des Schmetterlings bezeichnet ein Insekt, dass sich in mehreren
Stadien -- Ei, Raupe, Puppe, Imago -- entwickelt, deren Übergänge wir als
Metamorphosen beschreiben können. Offensichtlich bestehen aber keinerlei
sichtbare Gemeinsamkeiten zwischen diesen verschiedenen Stadien: Die Raupe hat
mir dem Imago so wenig Ähnlichkeit wie mit dem Ei, aus dem sie schlüpfte, oder der
Puppe, die sie wird. Dennoch handelt es sich in jedem Stadium um denselben
Organismus, der in allen seinen Stadien unter den Begriff des Schmetterlings
fällt, wenngleich es keine gemeinsamen Merkmale zu geben scheint. Es gibt
\emph{prima facie} keine Begriffe $P$, $Q$\ldots , für die gilt, dass jeder
Gegenstand, der unter den Begriff des Schmetterlings fällt, \emph{eo ipso} unter
die Begriffe $P$, $Q$\ldots fällt. \name[Immanuel]{Kant}s Theorie des Begriffs, der zufolge Begriffe
Merkmalskomplexionen sind und aus Teilbegriffen bestehen, scheint auf Lebewesen,
die Metamorphosen durchmachen, nicht anwendbar zu sein.

In Wahrheit aber ist \name[Immanuel]{Kant}s Begriffstheorie natürlich
auch auf metamorphosierende Lebewesen anwendbar: Die Tatsache dieser
Metamorphosen ist einfach selbst als Merkmal des jeweiligen Begriffs
aufzufassen. Diese Metamorphosen wiederum werden durch Merkmale des Begriffs
\enquote{Schmetterling} bezeichnet; der Begriff selbst \emph{handelt von}
Veränderungen, aber er unterliegt ihnen freilich nicht selbst. Begriffe
beschreiben die Eigenschaften des von ihnen bezeichneten, aber sie teilen diese
Eigenschaften nicht. Und entsprechend bestehen Begriffe aus Merkmalen,
Gegenstände hingegen aus Teilen.

Es ist auch behauptet worden, dass \name[Immanuel]{Kant} das Verhältnis von
Begriffen und Anschauungen mit dem Verhältnis von Ganzem und Teil durcheinander
bringe und seine Gegenüberstellung daher fehlerhaft
sei, insofern sie eine begriffliche mit einer
physikalischen Relation konfundiere.\footnote{Vgl.
\cite[][62--67]{Rang:ZweckmaessigkeitZweckursaechlichkeitundGanzheitlichkeitinderorganischenNatur1993}.}
\authorfullcite{Quarfood:DiscursivityandTranscendentalIdealism2012} wendet
dagegen ein, dass der diskursive Verstand seinen Gegenstand
durch sukzessive Synthesis des in der Anschauung gegebenen Mannigfaltigen erst herstelle.\footnote{\enquote{However, if the
discursive understanding has to build its cognized objects by successive syntheses of data
given in intuition, this at least suggests that the syntheses will determine an
object little by little.}
\parencite[][147]{Quarfood:DiscursivityandTranscendentalIdealism2012}} Doch
\authorcite{Quarfood:DiscursivityandTranscendentalIdealism2012}s Rekonstruktion
ist unbefriedigend, schon weil die Theorie von der Synthesis des
Mannigfaltigen in der Anschauung, so wie sie hier angewandt wird, nicht
zufrieden stellt. Nach seiner Lesart möchte \name[Immanuel]{Kant} sagen, dass
wir von den Teilen eines Organismus zum Ganzen fortschreiten, weil unsere
Sinnlichkeit zunächst die Teile darbietet, die wir dann durch die Synthesis des
Verstandes erst zu einem Ganzen zusammensetzen. Aber \emph{de facto} sehen wir
nicht erst Beine, Ohren, Nase und andere Körperteile, die wir dann zu einem
Hamster zusammensetzen, sondern wir sehen den Hamster, den wir erst durch eine
gedankliche Leistung in seine Teile \singlequote{zerlegen} können.

Warum spricht \name[Immanuel]{Kant} davon, dass der intuitive Verstand im
Unterschied zum diskursiven Denken vom Ganzen zu den Teilen gehe? Inwiefern muss
unser Verstand \enquote{von den Teilen als allgemein gedachten Gründen
zu verschiedenen darunter zu subsumierenden möglichen Formen als
Folgen}\footnote{\cite[][\S~77]{Kant:KritikderUrteilskraft2009},
\cite[][V: 407.26--27]{Kant:GesammelteWerke1900ff.}.} fortschreiten?
In welchem Verhältnis steht das Verhältnis von Begriffen und Anschauungen zum
Verhältnis von Teilen und Ganzem? Diese Fragen führen auf das Problem, dass in
\name[Immanuel]{Kant}s Ausführungen nicht leicht zu erkennen ist, inwiefern die
Eigenschaft unseres Verstandes, sich in Erklärungen nach mechanischen Kausalgesetzen zu
richten, mit der Eigenschaft in Verbindung steht, ein Vermögen der Begriffe und
nicht der Anschauungen zu sein. Es ist mehrfach behauptet worden, dass hier gar
kein Zusammenhang
besteht.\footnote{\cite[Vgl.][153--159]{McLaughlin:KantsKritikderteleologischenUrteilskraft1989}.
Nach \authorcite{McLaughlin:KantsKritikderteleologischenUrteilskraft1989} stellt
\name[Immanuel]{Kant} \enquote{die mechanistische Eigentümlichkeit unseres
Verstandes als ein Faktum hin und unternimmt keinen Versuch zu erklären, worin
sie besteht und begründet sei}
\parencite[][157]{McLaughlin:KantsKritikderteleologischenUrteilskraft1989}.
Seinen Ursprung habe dieses Faktum in der \singlequote{analytisch-synthetischen}
oder \singlequote{resolutiv-kompositiven} Methode der Naturwissenschaft des 17.
und 18. Jahrhunderts
\parencite[vgl.][\pno~157\,f.]{McLaughlin:KantsKritikderteleologischenUrteilskraft1989}.}
Dennoch gilt es zu beachten, dass \name[Immanuel]{Kant} in diesem Abschnitt
stets auf eben diese Eigenschaft unseres Verstandes rekurriert, ein Vermögen des
Denkens zu sein, das selbst nicht anzuschauen vermag und für welches das
Besondere zufällig ist.

Die Überlegungen zur Diskursivität unseres Verstandes in der \titel{Kritik der
Urteilskraft} stehen im Kontext der Dialektik der teleologischen Urteilskraft
und sollen helfen, die Antinomie aufzulösen, die durch den Gegensatz von
mechanischer und teleologischer Erklärungsart ausgelöst wird.\footnote{Vgl.
\cite[][146]{McLaughlin:KantsKritikderteleologischenUrteilskraft1989}.} Die
Darstellung eines anderen Verstandes ist als Vorbereitung der
\enquote{Vereinigung des Prinzips des allgemeinen Mechanismus der Materie mit
dem teleologischen in der Technik der
Natur}\footnote{So lautet der Titel von
\S~78 der \cite{Kant:KritikderUrteilskraft2009}
\parencite[][V: 410.13--15]{Kant:GesammelteWerke1900ff.}.} in \S~78 zu
verstehen, nicht als eigenständige Abhandlung. Letztlich löst
\name[Immanuel]{Kant} die Antinomie dadurch, dass er zeigt, dass sie aus der
Sicht eines intuitiven Verstandes nicht
besteht.\footnote{Vgl. \cite[][351]{Nuzzo:KantandtheUnityofReason2005}.} Aus
unserer Sicht -- so lautet die zentrale These -- lassen sich Organismen nur
unter Rückgriff auf eine zweckmäßig wirkende Ursache verstehen. Um zu zeigen,
dass dies aber nicht beweist, dass sie nur auf diese Art möglich sind (was
bewiese, dass sie tatsächlich Wirkungen einer absichtlich handelnden Ursache
sind), rekurriert \name[Immanuel]{Kant} auf die mögliche Konzeption
eines \singlequote{anderen} Verstandes, der dieser Notwendigkeit nicht unterläge.
\begin{quote}
[M]an kann an einem Dinge, welches wir als Naturzweck beurteilen müssen (einem
organisierten Wesen), zwar alle bekannten und noch zu entdeckenden Gesetze der
mechanischen Erzeugung versuchen und auch hoffen dürfen, damit guten Fortgang zu
haben, niemals aber der Berufung auf einen davon ganz unterschiedenen
Erzeugungsgrund, nämlich der Kausalität durch Zwecke, für die Möglichkeit eines
solchen Produkts überhoben sein, und schlechterdings kann keine menschliche
Vernunft (auch keine endliche, die der Qualität nach der unsrigen ähnlich wäre,
sie aber dem Grade nach noch so sehr überstiege) die Erzeugung auch nur eines
Gräschens aus bloß mechanischen Ursachen zu verstehen
hoffen.\footnote{\cite[][\S~77]{Kant:KritikderUrteilskraft2009},
\cite[][V: 409.27--37]{Kant:GesammelteWerke1900ff.}.}
\end{quote}
Nur wegen der Beschaffenheit unseren menschlichen Erkenntnisvermögens sei es
daher notwendig, einen sich auf Zwecke berufenden Grund für die äußeren
Gegenstände im übersinnlichen Substrat der Natur und den obersten Grund
in einem \emph{intellectus archetypus} zu
suchen.\footnote{Vgl. \cite[][\S~77]{Kant:KritikderUrteilskraft2009},
\cite[][V: 410-3--11]{Kant:GesammelteWerke1900ff.}.}

Ein Organismus besteht aus Teilen, deren Anordnung zweckmäßig
ist. Die einzelnen Teile verhalten sich in einer Weise zueinander und zum
Ganzen, dass sie sich sowohl wechselseitig als auch das Ganze erhalten und
reproduzieren.\footnote{\cite[Vgl.][\S~64]{Kant:KritikderUrteilskraft2009},
\cite[][V: 371.13--372.11]{Kant:GesammelteWerke1900ff.}.}
\begin{quote}
Wenn wir nun ein Ganzes der Materie seiner Form nach als ein Produkt der Teile
und ihrer Kräfte und Vermögen, sich von selbst zu verbinden {\punkt} betrachten,
so stellen wir uns eine mechanische Erzeugungsart desselben vor: Aber es kommt
auf solche Art kein Begriff von einem Ganzen als Zweck heraus, dessen innere
Möglichkeit durchaus die Idee von einem Ganzen voraussetzt, von der selbst die
Beschaffenheit und Wirkungsart der Teile abhängt, wie wir uns doch einen
organisierten Körper vorstellen
müssen.\footnote{\cite[][\S~77]{Kant:KritikderUrteilskraft2009},
\cite[][V: 408.24--31]{Kant:GesammelteWerke1900ff.}.}
\end{quote}
Unser diskursiver Verstand kann die Entstehung des Ganzen und seine
Eigenschaften nur als Wirkungen der Beschaffenheit und Anordnung der Teile
erklären. Wir können die Wirkungsweise einer Uhr beispielsweise verstehen, wenn
wir verstehen, wie die einzelnen Teile in ihr angeordnet sind und
zusammenwirken. So erklären wir ihr Funktionieren auf mechanische Art. Die
Anordnung der Teile wiederum verstehen wir nur, wenn wir sie als Ergebnis der
Herstellung nach einem vorher gemachten Plan (einer \singlequote{Idee}) ansehen.
Auf diese Weise erklären wir das Zustandekommen der Uhr teleologisch (was
solchen Artefakten adäquat ist). Auch bei Organismen liegt es nahe,
die Eigenschaften des Ganzen mechanisch aus Beschaffenheit und Anordnung der
Teile zu verstehen, aber die Anordnung der Teile selbst scheint doch von dem
Ganzen abzuhängen und einem Plan zu entsprechen. Deshalb scheint es notwendig zu
sein, sie in Analogie zu künstlich hergestellten Gegenständen zu denken, obwohl
wir keinen Beleg dafür haben, dieses Urteil als objektiv gültig zu betrachten.

Die Vorstellung, dass das Ganze die Ursache der Teile ist, sei aus der Sicht
unseres diskursiven Verstandes ein
Widerspruch.\footnote{\cite[Vgl.][\S~77]{Kant:KritikderUrteilskraft2009},
\cite[][V: 407.34--37]{Kant:GesammelteWerke1900ff.}.} Das heißt zunächst, dass
wir nur zwei Optionen bei der Erklärung organisierter Wesen haben: Wir können
einerseits mechanisch erklären, warum etwas auf eine bestimmte Weise beschaffen
ist oder bestimmte Eigenschaften hat, indem wir dies auf äußere Ursachen oder
die Wirkung seiner Teile zurückführen. Andererseits können wir seine Form
teleologisch erklären, insofern wir die Anordnung und Beschaffenheit der Teile
als Wirkung einer Vorstellung auffassen. Wir können eben die Eigenschaften einer
Uhr als Wirkung des Zusammenspiels der Teile verstehen oder den Aufbau der Uhr
als Wirkung des zweckmäßigen Handelns eines Uhrmachers. Aber wir können die
Anordnung der Teile nicht als Wirkung des Ganzen der Uhr verstehen. Und dies
wiederum liege an der Diskursivität unseres Verstandes. Der intuitive
Verstand betrachte das Ganze des Organismus als \enquote{den Grund der
Möglichkeit der Verknüpfung der
Teile}\footnote{\cite[][\S~77]{Kant:KritikderUrteilskraft2009},
\cite[][V: 407.36]{Kant:GesammelteWerke1900ff.}.}, doch dies sei aus der Sicht
des diskursiven Verstandes ein Widerspruch. Unser endlicher Verstand könne sich nur
denken, dass \enquote{die \ori{Vorstellung} eines Ganzen den Grund der Möglichkeit der Form
desselben und der dazu gehörigen Verknüpfung der Teile
enthalte.}\footnote{\cite[][\S~77]{Kant:KritikderUrteilskraft2009},
\cite[][V: 407.37--408.2]{Kant:GesammelteWerke1900ff.}.} Eine solche Vorstellung
ist aber ein \emph{Zweck} und \emph{a fortiori} können wir uns Naturprodukte lediglich in
Analogie zu zweckgerichteter Produktion vorstellen.

Nach \authorcite{Foerster:Die25JahrederPhilosophie2011} ist
\name[Immanuel]{Kant}s Exposition eines Verstandes, der nicht vom
Ana\-ly\-tisch-All\-ge\-mei\-nen (von Begriffen), sondern vom
Synthetisch-Allgemeinen (von der Anschauung eines Ganzen) ausgeht, nicht mit der
produktiven intellektuellen Anschauung, für die Möglichkeit und Wirklichkeit
zusammenfallen, oder mit einem intuitiven Verstand, der die Gegenstände des
Erkennens selbst hervorbringt, sondern mit
\authorcite{Spinoza:EthikingeometrischerOrdnungdargestellt2007}s Konzeption
einer \emph{scientia intuitiva} in Verbindung zu bringen, die \enquote{alle
Eigenschaften [ihres] Gegenstandes aus dessen Wesenheit herzuleiten
erlaubt}\footcite[][187]{Foerster:DieBedeutungvonSS7677deremphKritikderUrteilskraftfuerdieEntwicklungdernachkantischenPhilosophieTeil12002},
insofern sie im \enquote{Besonderen zugleich das Allgemeine} sieht --
\enquote{Anschauung und Begriff fallen
zusammen.}\footcite[][188]{Foerster:DieBedeutungvonSS7677deremphKritikderUrteilskraftfuerdieEntwicklungdernachkantischenPhilosophieTeil12002}
Der Anknüpfungspunkt besteht gerade darin, dass wir versuchen wollen, uns die
Möglichkeit der Teile als vom Ganzen abhängig vorzustellen, insofern der
intuitive Verstand von dem Wesen des Gegenstandes ausgeht.

\authorcite{Spinoza:EthikingeometrischerOrdnungdargestellt2007} erwähnt die
\emph{scientia intuitiva} oder Erkenntnis der dritten Gattung (\emph{tertii
generis cognitio}) in einem \emph{scholium} zur 40. \emph{propositio} des
zweiten Teils der \titel{Ethica}. Sie gehe von der adäquaten Idee der Wesenheit
(\emph{essentia}) göttlicher Attribute aus und gelange dann zur ebenfalls
adäquaten Erkenntnis der Wesenheit einzelner Dinge.\footnote{\enquote{Atque
hoc cognoscendi genus procedit ab adaequata idea essentiae formalis quorundam
Dei attributorum ad adaequatam cognitionem essentiae rerum}
\parencite[][2p40s2]{Spinoza:EthikingeometrischerOrdnungdargestellt2007}. Diese
Gattung wird außerdem im \titel{Tractatus de intellectus emendatione}
beschrieben
\parencite[vgl.][II: 10.20--21, 11.13--19]{Spinoza:SpinozaOpera1972}.} Zur
Erläuterung stützt sich
\authorcite{Spinoza:EthikingeometrischerOrdnungdargestellt2007} vornehmlich auf
Beispiele aus der Mathematik. So lautet sein bekanntes Beispiel aus der
\titel{Ethica}: Es seien drei Zahlen $a$, $b$ und $c$ gegeben und dazu eine
vierte Zahl $d$ aufzufinden, so dass gilt: $b:a = d:c$. Zur Lösung ist $c$ mit
$b$ zu multiplizieren und das Ergebnis wiederum durch $a$ zu dividieren.
Nun gebe es verschiedene Arten, wie man dies wissen kann. Wir können es etwa von
anderen gehört haben oder wir
haben es bei einfachen Zahlen ausprobiert und dann verallgemeinert. Dies nennt
\authorcite{Spinoza:EthikingeometrischerOrdnungdargestellt2007} Erkenntnis der
ersten Gattung oder auch Meinung (\emph{opinio}) oder Vorstellung
(\emph{imaginatio}). Wir können aber auch einen mathematischen Beweis
nachvollzogen und Vernunfterkenntnis (\emph{ratio}) erworben haben, die er
Erkenntnis der zweiten Gattung nennt. Eine Erkenntnis der dritten Gattung läge
hingegen dann vor, wenn wir den Zusammenhang der ersten beiden Zahlen mit einem
Blick (\emph{uno intuitu}) erfassen könnten, um dann auf die vierte Zahl zu
schließen, wie uns dies bei ganz einfachen Zahlen möglich
sei.\footnote{\enquote{Ex.\,gr. datis numeris 1, 2, 3 nemo non videt quartum
numerum proportionalem esse 6, atque hoc multo clarius, quia ex ipsa ratione,
quam primum ad secundum habere uno intuitu videmus, ipsum quartum concludimus}
\parencite[][2p40s2]{Spinoza:EthikingeometrischerOrdnungdargestellt2007}.
\authorfullcite{Matheron:SpinozaandEuclideanArithmetic1986} betont zu Recht,
dass nach \authorcite{Spinoza:EthikingeometrischerOrdnungdargestellt2007}s
\titel{Ethica} das Verhältnis der ersten beiden Zahlen \emph{uno intuito}
erfasst werde, wir dann aber darauf \emph{schließen}, welche Zahl die vierte sein muss
\parencite[vgl.][\pno~144\,f.]{Matheron:SpinozaandEuclideanArithmetic1986}.}

Welches Merkmal grenzt die \emph{scientia intuitiva} von den anderen
Erkenntnisgattungen ab? Nach
\authorfullcite{Bartuschat:SpinozasTheoriedesMenschen1992} ist es die Tatsache,
dass es sich um eine rationale Erkenntnis von \emph{Einzelnem} handelt, während
die Vernunft (\emph{ratio}) als zweite Gattung nur Allgemeines
erkenne.\footnote{\enquote{Die Erkenntnis der Essenz von einzelnem und damit
Gottes im einzelnen ist die scientia intuitiva. Sie ist mit dem Konzept Gottes
genau dann verbunden, wenn Gott als ein Wesen konzipiert ist, das die Ursache
von Individuellem ist, das in seiner Endlichkeit von der Unendlichkeit Gottes
verschieden ist und deshalb sein Verursachtsein an sich selber erweisen muß}
\parencite[][121]{Bartuschat:SpinozasTheoriedesMenschen1992}.} Nach
\authorfullcite{Roed:SpinozasIdeederScientiaintuitivaunddieSpinozanischeWissenschaftskonzeption1977}
ist \enquote{die intuitive Einsicht primär Erkenntnis Gottes und sekundär
Erkenntnis endlicher Dinge in Abhängigkeit von der Erkenntnis
Gottes}\footcite[][\pno~497\,f.]{Roed:SpinozasIdeederScientiaintuitivaunddieSpinozanischeWissenschaftskonzeption1977}
und damit \enquote{Wissen vom absoluten Ganzen in der Einheit seiner
Momente}\footcite[][498]{Roed:SpinozasIdeederScientiaintuitivaunddieSpinozanischeWissenschaftskonzeption1977}.
Dabei habe der Begriff der \emph{scientia intuitiva} jedoch verschiedene
Bedeutungskomponenten, die nicht alle mit dieser Deutung in Einklang zu bringen
seien.\footcite[Vgl.][498]{Roed:SpinozasIdeederScientiaintuitivaunddieSpinozanischeWissenschaftskonzeption1977}
\authorfullcite{Foerster:Die25JahrederPhilosophie2011} wiederum verweist darauf,
dass die Vernunfterkenntnis die Dinge aus äußeren Ursachen erkenne, während die
\emph{scientia intuitiva} die Dinge erkenne, insofern sie Ursachen ihrer selbst
sind.\footnote{\cite[Siehe][91]{Foerster:Die25JahrederPhilosophie2011}:
\enquote{Bereits in seiner Frühschrift \ori{Abhandlung über die Berichtigung des Verstandes} hatte er
darauf insistiert, dass zur adäquaten Erkenntnis einer Sache diese entweder bloß
durch ihre Wesenheit oder durch ihre nächste Ursache begriffen werden müsse:
wenn eine Sache an und für sich besteht oder Ursache ihrer selbst ist, so muss
sie bloß durch ihre Wesenheit erkannt werden; wenn sie zu ihrem Dasein aber eine
Ursache braucht, dann muss sie durch ihre nächste Ursache erkannt werden (TIE
§92). Soll die Erkenntnis die Form eines Systems haben, dann muss freilich mit
dem der Anfang gemacht werden, dessen Begriff den Begriff keiner anderen Sache
voraussetzt.}} Und dies entspricht dem Begriff des Naturzwecks
bei \name[Immanuel]{Kant}. Der intuitive Verstand erkennt die Dinge als
Naturzwecke, die von sich selbst Ursache und Wirkung sind, ohne ihnen eine
äußere, nach Zwecken handelnde Ursache zuschreiben zu
müssen. \name[Immanuel]{Kant} wiederum behauptet, dass uns endlichen
Subjekten ein solches Vorgehen nicht möglich sei. \emph{Wir} können
die Eigenschaften eines Organismus und die Anordnung seiner Teile
nicht dadurch erklären, dass wir auf sein Wesen rekurrieren und so das
Ganze -- \emph{respective} dessen Essenz -- zur Ursache der Organisation der Teile machen.

Nach \authorcite{Foerster:Die25JahrederPhilosophie2011} besteht der wesentliche
Grund, warum \name[Immanuel]{Kant} die von
\authorcite{Spinoza:EthikingeometrischerOrdnungdargestellt2007} behauptete
Möglichkeit der \emph{scientia intuitiva} verneint, darin, dass er nicht wie
dieser von der Mathematik, sondern von einem göttlichen Verstand
ausgehe.\footnote{Vgl. \cite[][255]{Foerster:Die25JahrederPhilosophie2011}.}
Die Idee eines intuitiven Verstandes schwebe ihm ja selbst vor, er sehe nur
nicht, wie eine entsprechende Art zu erkennen möglich sein könnte. Auch \authorcite{Spinoza:EthikingeometrischerOrdnungdargestellt2007} bedauert, dass er so
wenige Beispiele für die \emph{scientia intuitiva}
finde.\footnote{\cite[Vgl.][II: 11.18--19]{Spinoza:SpinozaOpera1972}.} In der
\titel{Ethica} finden sich kaum Beispiele für das Vorliegen der
dritten Erkenntnisgattung. Stattdessen wird behauptet, dass wir über
solche intuitiven Erkenntnisse verfügen, weil wir eine adäquate Idee
Gottes besitzen und aus dieser Idee schlussfolgern können.
\name[Immanuel]{Kant} bestreitet gerade dies, und zwar vor dem Hintergrund der
Überlegungen zum Verhältnis von Anschauung und Begriff, wie sie oben betrachtet
wurden.\footnote{Siehe Kapitel
\ref{subsubsection:UnterscheidungvonDenkenundErkennen}.} Gerade weil uns als
endlichen Wesen die Unterscheidung von Möglichkeit und Wirklichkeit aufgezwungen
sei, könnten wir ein solches Wesen, wie es
\authorcite{Spinoza:EthikingeometrischerOrdnungdargestellt2007} als die eine
Substanz konzipiert, nicht denken und seine Existenz nicht gemäß dem
\singlequote{ontologischen} Gottesbeweis
einsehen.\footnote{\cite[Vgl.][\S~76]{Kant:KritikderUrteilskraft2009},
\cite[][V: 402.18--32]{Kant:GesammelteWerke1900ff.}.}

\authorcite{Foerster:Die25JahrederPhilosophie2011} zufolge hat erst \name[Johann
Wolfgang von]{Goethe} gesehen, wie die Idee der \emph{scientia intuitiva} von
der Mathematik auf die Naturerkenntnis ausgeweitet werden kann.\footnote{Vgl.
\cite[][188]{Foerster:DieBedeutungvonSS7677deremphKritikderUrteilskraftfuerdieEntwicklungdernachkantischenPhilosophieTeil12002}.}
Dabei bleibe diese Erkenntnisart esoterisch: Sie sei nur durch Übung zu erlangen
und darum wenigen Menschen vorbehalten.\footnote{\enquote{Dass eine solche
Verbindung von diskursivem und intuitivem Denken nur unter Anstrengungen und als
Ergebnis wiederholter Übung möglich ist, war Goethe durchaus klar}
\parencite[][185]{Foerster:DieBedeutungvonSS7677deremphKritikderUrteilskraftfuerdieEntwicklungdernachkantischenPhilosophieTeil12002}.}
\name[Immanuel]{Kant}s Fehler sei es gewesen, die Möglichkeit einer solchen --
doch immerhin denkbaren -- Erkenntnisart grundlos abzuleugnen, statt zumindest
den Versuch dazu zu unternehmen.\footnote{\enquote{Es dürfte klar sein, dass sich ein begründetes Urteil über die Möglichkeit oder
Unmöglichkeit eines solchen Verfahrens (und damit über die Möglichkeit eines
intuitiven Verstandes) nicht fällen lässt, ohne dass diese Schritte erprobt und
nachvollzogen werden. Ein ausschließlich diskursives Denken, das aus sich selbst
heraus seine Alternativenlosigkeit glaubt wissen zu können, erweist seine
philosophische Naivität gerade dadurch, dass es dogmatisch selbst noch hinter
die Kantische Forderung zurückfällt, eine Alternative zu unserem gegenwärtigen
Erkenntnisvermögen zumindest versuchsweise zu denken, um es nicht für das einzig
mögliche zu halten}
\parencite[][187]{Foerster:DieBedeutungvonSS7677deremphKritikderUrteilskraftfuerdieEntwicklungdernachkantischenPhilosophieTeil12002}.
Es ist mehrfach argumentiert worden, \name[Immanuel]{Kant} zeige doch selbst
die Möglichkeit eines \singlequote{anderen} Erkennens auf, indem er -- speziell
in den \S\S~76 und 77 der \titel{Kritik der Urteilskraft} ein solches
konzipiert, bevor er es grundlos bzw. aufgrund nicht explizit gemachter
empirischer Beobachtung verwirft. Ein solcher Vorwurf findet sich etwa bei
\authorcite{Hegel:GesammelteWerke} in \titel{Glauben und Wissen}
\parencite[vgl.][IV: 335,9..11, 338.35--343.17]{Hegel:GesammelteWerke}.}
Um intuitiv zu erkennen sei zunächst durch eine kontinuierlich Reihe von
Beobachtungen ein Ganzes in den Blick zu bekommen, von welchem ausgehend ein intuitiver Verstand erst zu
den Teilen gehen könne. Im Anschluss daran müsse man sich die Sache durch Denken
aneignen. Und dazu bedürfe es einer Verbindung von anschauender und diskursiver
Erkenntnis. Was soll das aber heißen?
\authorcite{Foerster:Die25JahrederPhilosophie2011} schreibt:
\begin{quote}
Ich kann die Pflanze nicht anders zeichnen, als Stück für Stück und
nacheinander. Genauso kann ich scheinbar deren Entwicklung nicht anders denken
als diskursiv und nacheinander. Aber so entwickelt sich kein Organismus. Er
wächst in allen Teilen zugleich. {\punkt} Die Pflanze bildet sich in allen ihren
Teilen zugleich. Um diese Prozesse nachzuvollziehen, muss ich in Gedanken also
auch an allen ihren Stellen zugleich sein; mit anderen Worten: das Denken muss
intuitiv, d.\,h. anschauend werden. Der Gedanke eines gleichzeitigen Ganzen von
Teilen und der einer Abfolge von Veränderungen der Teile muss ein einzelner,
selbst lebendiger Gedanke
werden.\footnote{\cite[][\pno~184\,f.]{Foerster:DieBedeutungvonSS7677deremphKritikderUrteilskraftfuerdieEntwicklungdernachkantischenPhilosophieTeil12002}.}
\end{quote}
Doch diese Überlegung ist nicht verlockender als die Überlegung zur Lebendigkeit
von Begriffen: Ebenso wie ein Begriff die Metamorphosen seines Gegenstands durch
Merkmale beinhaltet, ohne dass er selbst metamorphosieren müsste, kann ein Gedanke von
der sukzessiven Entwicklung eines Gegenstandes handeln, ohne sich dabei selbst
zu verändern.\footnote{\authorcite{Foerster:Die25JahrederPhilosophie2011} verwendet hier m.\,E.
einen anderen Diskursivitätsbegriff als denjenigen, den ich in Kapitel
\ref{subsection:DiskursiverVerstandundsinnlicheAnschauung} herausstellte. Ein
Denken ist danach diskursiv, wenn es Schritt für Schritt vorgeht (was der
Bedeutung des \enquote{\emph{discurrere}} natürlich nahe kommt). Intuitiv wäre
hingegen ein Denken, das das Ganze simultan in den Blick bekommt. Dies wiederum
erinnert eher an den Begriff der \emph{cognitio intuitiva} (im Kontrast zu
einer \emph{cognitio symbolica}) bei
\textcite[vgl.][585--588]{Leibniz:Meditationesdecognitioneveritateetideis1999}, als an
\authorcite{Spinoza:SpinozaOpera1972} und \name[Immanuel]{Kant}.}

\name[Immanuel]{Kant} betont auch in \S~77 der \titel{Kritik der Urteilskraft}
die Beschaffenheit unseres Verstandes als eines Vermögens der
\emph{Begriffe}\footnote{\enquote{Unser Verstand ist ein Vermögen der Begriffe,
d.\,i. ein diskursiver Verstand}
\mkbibparens{\cite[][\S~77]{Kant:KritikderUrteilskraft2009}, \cite[][V:
406.16--17]{Kant:GesammelteWerke1900ff.}}.} und die Tatsache, dass für einen
solchen \emph{diskursiven} Verstand die Verbindung der Teile eines Ganzen
zufällig sein muss. Es ist zunächst dieselbe
Zufälligkeit, die uns bei den besonderen Gesetzen der Natur begegnet, die auch
in der Konstitution lebender Wesen vorliegt. Ein intuitiver Verstand, für den die
Unterscheidung von Möglichkeit und Wirklichkeit nicht besteht, denkt nicht den
Grund für die Wirklichkeit eines Gegenstandes als äußere Ursache, weil der
Gegenstand mit seiner Möglichkeit bereits als wirklich gegeben wäre. Unser
diskursiver Verstand, der Möglichkeit und Wirklichkeit unterscheidet, kann sich
die Möglichkeit eines Gegenstandes nicht als dessen innere hinreichende Ursache
vorstellen, sondern bedarf der Annahme einer äußeren Ursache. Der intuitive
Verstand hingegen ginge genau deshalb vom Ganzen zu den Teilen, weil für ihn mit
der Möglichkeit des Ganzen dessen Wirklichkeit und damit auch die Wirklichkeit
seiner Teile und damit deren Anordnung gegeben wäre. Demnach ist es dieselbe Besonderheit
unseres Verstandes, die schon für die Unterscheidung von Möglichkeit und Wirklichkeit
verantwortlich zeichnete, die auch die Besonderheit in Ansehung der
reflektierenden Urteilskraft hervorbringt, Naturprodukte nur in Analogie zu
Zwecken denken zu können.

\subsection{Die Endlichkeit des
Willens}\label{subsubsection:DieEndlichkeitdesWillens}
Ebenso wie die Begriffe der Möglichkeit und Wirklichkeit nach den Überlegungen
aus Kapitel \ref{subsubsection:UnterscheidungvonDenkenundErkennen} und der
Begriff eines Naturzwecks nach Kapitel
\ref{subsection:IntuitiverVerstandunddasSynthetischAllgemeine} nur Bedeutung für
einen endlichen, diskursiven Verstand haben, weil sie die Trennung von Begriff
und Anschauung, von Verstand und Sinnlichkeit voraussetzen, für einen denkbaren
\singlequote{anderen}, nämlich intuitiven Verstand jedoch völlig ohne Bedeutung
wären, so liege es an \enquote{der subjektiven
Beschaffenheit unseres praktischen Vermögens {\punkt}, daß die moralischen
Gesetze als Gebote} \enquote{vorgestellt werden
müssen}\footnote{\cite[][\S~76]{Kant:KritikderUrteilskraft2009}, \cite[][V:
403.30--32]{Kant:GesammelteWerke1900ff.}.}. Für einen unendlichen Willen sind
keine Gebote und Imperative denkbar, denn es fehlt die Möglichkeit des Handelns
wider die Vernunft. Unser Wille ist endlich, insofern wir von den Gesetzen der
Autonomie abweichen können. Nur ein endlicher Wille unterliegt
Verbindlichkeiten.\footnote{\enquote{Der Wille, dessen Maximen notwendig mit den Gesetzen der Autonomie
zusammenstimmen, ist ein \ori{heiliger}, schlechterdings guter Wille. Die
Abhängigkeit eines nicht schlechterdings guten Willens vom Prinzip der Autonomie
(die moralische Nötigung) ist \ori{Verbindlichkeit}. Diese kann also auf ein
heiliges Wesen nicht bezogen werden} \mkbibparens{\cite[][BA
86]{Kant:GrundlegungzurMetaphysikderSitten1965}, \cite[][IV:
439.28--33]{Kant:GesammelteWerke1900ff.}}.}
Für einen heiligen Willen kann es keine Verbindlichkeiten und damit
keine Imperative geben, denn es fehlt die Möglichkeit des
Zuwiderhandelns.\footnote{\enquote{Ein vollkommen guter Wille würde also eben sowohl unter objektiven Gesetzen
(des Guten) stehen, aber nicht dadurch als zu gesetzmäßigen Handlungen
\ori{genötigt} vorgestellt werden können, weil er von selbst, nach seiner
subjektiven Beschaffenheit, nur durch die Vorstellung des Guten bestimmt werden
kann. Daher gelten für den \ori{göttlichen} und überhaupt für einen
\ori{heiligen} Willen keine Imperativen; das \ori{Sollen} ist hier am unrechten
Orte, weil das \ori{Wollen} schon von selbst mit dem Gesetz notwendig einstimmig
ist} \mkbibparens{\cite[][BA 39]{Kant:GrundlegungzurMetaphysikderSitten1965},
\cite[][IV: 414.1--8]{Kant:GesammelteWerke1900ff.}}. Siehe auch
\cite[][A 57]{Kant:KritikderpraktischenVernunft1974},
\cite[][V: 32.15--21]{Kant:GesammelteWerke1900ff.}: Das moralische Gesetz
\enquote{schränkt sich also nicht bloß auf Menschen ein, sondern geht auf alle endlichen Wesen, die Vernunft und Willen
haben, ja schließt sogar das unendliche Wesen, als oberste Intelligenz, mit ein.
Im ersteren Falle aber hat das Gesetz die Form eines Imperativs, weil man an
jenem zwar als vernünftigem Wesen einen reinen, aber als mit Bedürfnissen und
sinnlichen Bewegursachen affiziertem Wesen keinen heiligen Willen, d.\,i. einen
solchen, der keiner dem moralischen Gesetze widerstreitenden Maxime fähig wäre,
voraussetzen kann.}}
Deswegen ist das moralische Gesetz für endliche Wesen ein Gesetz der Pflicht,
das mit Nötigung und dem Gefühl der Achtung verbunden ist, für einen unendlichen
Willen hingegen ist es ein Gesetz der Heiligkeit.\footnote{\cite[][A
146]{Kant:KritikderpraktischenVernunft1974}, \cite[][V:
82.8--12]{Kant:GesammelteWerke1900ff.}. Außerdem kommen nur einem endlichen Willen Neigungen und Interessen zu: \enquote{Die Abhängigkeit des
Begehrungsvermögens von Empfindungen heißt Neigung, und diese beweist also jederzeit ein Bedürfnis. Die Abhängigkeit eines zufällig
bestimmbaren Willens aber von Prinzipien der Vernunft heißt ein Interesse.
Dieses findet also nur bei einem abhängigen Willen statt, der nicht von selbst
jederzeit der Vernunft gemäß ist; beim göttlichen Willen kann man sich kein
Interesse gedenken} \mkbibparens{\cite[][BA
38]{Kant:GrundlegungzurMetaphysikderSitten1965}, \cite[][IV:
413.26--31]{Kant:GesammelteWerke1900ff.}}.}

Dies ist die Charakterisierung der Endlichkeit unseres Willens. Die Frage, die nun zu beantworten ist, lautet:
Handelt es sich auch der Endlichkeit des menschlichen Willens um eine Ausprägung oder Folge der
Endlichkeit unseres Verstandes? Oder steht sie getrennt neben dieser
Endlichkeit? Letzteres scheint naheliegend zu sein, insofern die Endlichkeit eines konativen
statt eines kognitiven Vermögens angesprochen ist. Unterschiede
\name[Immanuel]{Kant} zwischen Verstand und Willen als zwei unterschiedlichen
Vermögen, dann läge es nahe davon auszugehen, dass beide Ausprägungen unserer Endlichkeit unverbunden
nebeneinander stehen. Geht man hingegen -- wie
\authorcite{Spinoza:EthikingeometrischerOrdnungdargestellt2007} gegen
\authorcite{Descartes:OeuvresdeDescartes1983} betont\footnote{Zum Verhältnis von
Wille und Verstand bei \authorcite{Descartes:OeuvresdeDescartes1983} siehe oben,
S. \pageref{Absatz:DescarteszuEndlichkeitundVerhaeltnisvonWilleundVerstand}\,f.
Spinoza kritisiert dies in der \titel{Ethica}:
\enquote{Voluntas, et intellectus unum, et idem sunt} \parencite[][\nopp
2p49c]{Spinoza:EthikingeometrischerOrdnungdargestellt2007}.} -- davon aus, dass
der Wille selbst ein intellektuelles Vermögen ist, dann gilt es, den möglichen
Zusammenhang genauer in's Auge zu fassen.

Die Eigenschaft des moralischen Gesetzes, uns in der Form von Imperativen zu
begegnen, beschreibt die Endlichkeit unserer Vernunft, insofern sie praktische
Vernunft ist. Vernunft (im engeren Sinne) ist zunächst das Vermögen, zu
schließen. Vernunft ist praktisch, wenn sie sich nicht in Ableitung von
Erkenntnissen äußert, sondern von Handlungen -- die praktische Vernunft heißt
darum auch \enquote{Wille}.\footnote{\enquote{Ein jedes Ding der Natur wirkt nach Gesetzen. Nur ein vernünftiges Wesen hat
  das Vermögen, \ori{nach der Vorstellung} der Gesetze, d.\,i. nach Prinzipien, zu
  handeln, oder einen \ori{Willen}. Da zur Ableitung der Handlungen von Gesetzen
  \ori{Vernunft} erfodert wird, so ist der Wille nichts anders, als praktische
  Vernunft} \mkbibparens{\cite[][BA 36]{Kant:GrundlegungzurMetaphysikderSitten1965},
  \cite[][IV: 412.26--30]{Kant:GesammelteWerke1900ff.}}.}
Wir können also gleichbedeutend von der Endlichkeit des Willens und der
praktischen Vernunft sprechen, denn der Wille ist \emph{nichts anderes} als
praktische Vernunft. Wie aus dem Wortlaut dieses Zitats deutlich wird, äußert
sich die praktische Vernunft oder der Wille in praktischen Syllogismen. Ein
Syllogismus ist praktisch, wenn er als Konklusion keine Aussage hat -- auch
keine imperativische, die zur Handlung auffordert --, sondern eine
Handlung. Als Prinzip ist beispielsweise die Maxime gegeben \enquote{Ich will
stets ehrlich sein.}.
Sie fungiert als Obersatz in einem Syllogismus und erlaubt in einer Situation,
in der ich etwas gefragt werde, den Schluss auf die
\emph{Handlung}, in der ich die Wahrheit sage (oder die Antwort verweigere,
damit aber zumindest nichts falsches sage).
Die Konklusion besteht wohlgemerkt nicht in einer Erkenntnis der Art \enquote{Ich sollte jetzt
das-und-das sagen!}, sondern direkt in der entsprechenden
Handlung.\footnote{Dem widersprechen jedoch einige Textbelege, an denen
\name[Immanuel]{Kant} entgegen dem Wortlaut der \titel{Grundlegung} als
Resultat der praktischen Vernunft keine Handlungen, sondern Erkenntnisse
bezüglich eines Sollens ansieht. Siehe z.\,B.
\cite[][B 661]{Kant:KritikderreinenVernunft2003},
\cite[][III: 421.17--19]{Kant:GesammelteWerke1900ff.}:
\enquote{Ich begnüge mich hier, die theoretische Erkenntnis durch eine solche zu
erklären, wodurch ich erkenne, was da \ori{ist}, die praktische aber, dadurch
ich mir vorstelle, was da \ori{sein soll}.} Siehe zu dieser Frage
\cite{Engstrom:KantsDistinctionbetweenTheoreticalandPracticalKnowledge2002}.}
Der Wille oder die praktische Vernunft ist somit ein Vermögen, Handlungen als
Ausdruck allgemeiner Grundsätze auszuführen.

\begin{comment}
Zunächst scheint unser Wille die Diskursivität des Verstandes und die
Sinnlichkeit der Anschauung gerade nicht zu teilen. Es ließe sich gar vermuten,
dass die praktische Vernunft der intellektuellen Anschauung oder dem intuitiven
Verstand entspricht, insofern sie die Gegenstände ihrer Erkenntnis -- die
ausgeführten Handlungen -- selbst hervorbringt. Nun ist -- wie
\authorfullcite{Engstrom:KantsDistinctionbetweenTheoreticalandPracticalKnowledge2002}
anführt --  auch die Erkenntnis der praktischen Vernunft eine Handlung des
\emph{diskursiven}
Verstandes.\footnote{\cite[Vgl.][\pno~59\,f.:]{Engstrom:KantsDistinctionbetweenTheoreticalandPracticalKnowledge2002}
\enquote{practical knowledge is productive of its object not only with respect to the latter's form, but even with respect to its existence. Now this description of how practical knowledge
is related to its object might appear to make practical knowledge
indistinguishable from intellectual intuition, the divine cognition with which
human discursive cognition is contrasted, and to which Kant tacitly alludes in
the passage from §14. But the distinction between these two can be duly
maintained if we bear in mind that practical knowledge, like theoretical
knowledge, is the product of discursive intellect, whose knowledge always
proceeds from concepts, or general representations, rather than from intuitions,
or singular representations, and further that, as a result, in the case of
practical knowledge, the production of the object is always the arrengement, in
accordance with a general form, of presupposed matter, whereas, in the case of
intellectual intuition, there is no matter requisite as a condition under which
the production is possible.}}
\end{comment}

Der Wille ist Bestandteil des oberen Erkenntnisvermögens und somit dessen
Endlichkeit unterworfen. In der \titel{Kritik der Urteilskraft} heißt es
entsprechend, dass die Endlichkeit des Willens aufzuheben hieße, \enquote{die Vernunft
ohne Sinnlichkeit (als subjektiver Bedingung ihrer Anwendung auf Gegenstände
der Natur)}\footnote{\cite[][\S~76]{Kant:KritikderUrteilskraft2009},
\cite[][V: 403.34--36]{Kant:GesammelteWerke1900ff.}.} und damit als Ursache in
einer intelligiblen Welt zu betrachten. Unser Wille ist endlich, weil
unsere Vernunft nur mit unserer Sinnlichkeit praktische Vernunft sein
kann. Dem stimmt auch
\authorfullcite{Foerster:Die25JahrederPhilosophie2011} zu:
\enquote{Für eine Vernunft, die ohne Sinnlichkeit als subjektive Bedingung ihrer
Anwendung wirksam sein könnte, fiele dieser Unterschied fort. Der Gegensatz
{\punkt} ist also nur gültig für ein praktisches Vernunftwesen, das zugleich
sinnlich ist und dessen Kausalität mit derjenigen der Sinnenwelt nicht
zusammenfällt.}\footnote{\cite[][151]{Foerster:Die25JahrederPhilosophie2011}.}
Da die Vernunft (im engeren Sinne) Teil des oberen Erkenntnisvermögens -- des
Verstandes als des Vermögens der Spontaneität -- ist und ihre Endlichkeit darin
besteht, auf Sinnlichkeit angewiesen zu sein, handelt es sich bei der
Endlichkeit unseres Willens also um eine Ausformung der Endlichkeit des
Verstandes.

\section{Zusammenfassung und Ausblick}
In Kapitel \ref{section:KantalsliberalerAufklaerer} erarbeitete ich einen
Aufklärungsbegriff, der die intellektuelle Freiheit und
Unabhängigkeit fokussiert und sich gegen Passivität und Abhängigkeit wendet.
Ihm zufolge sind wir zwar in unserem Verstandesgebrauch davon abhängig, dass wir uns
in einer Gemeinschaft mit anderen befinden. Aber wir verstehen uns innerhalb
dieser Gemeinschaft doch als gleichwertige Mitglieder, die jedes Urteil selbst
\mbox{(mit-)} kontrollieren, statt es passiv in den eigenen
Überzeugungsvorrat aufzunehmen. Selbstdenken ist zunächst abstrakt als
Selbsttätigkeit (Spontaneität) beschrieben und dem passiven Rezipieren entgegen
gestellt.

In diesem \ref{chapter:endlichkeitmenschlichendenkens}. Kapitel zeigte sich
aber, dass unsere Selbsttätigkeit in jedem Fall von Rezeptivität abhängig ist.
Dies macht gerade die Endlichkeit des Menschen aus:
dass er bei jeder kognitiven Operation von seiner Rezeptivität abhängig bleibt.
Insofern besteht hier ein klarer Gegensatz von Aufklärung und Endlichkeit, den
es im folgenden aufzulösen gilt. Es geht darum zu zeigen, wie angesichts der
vielfältigen Abhängigkeiten, denen wir als endliche Wesen unterliegen, dennoch
sinnvoll von intellektueller Selbständigkeit und Unabhängigkeit gesprochen
werden kann. In welchen Formen können uns Erkenntnisse \singlequote{gegeben}
werden, ohne dass dies einen Verlust an Mündigkeit mit sich bringt?



\chapter{Die Endlichkeit praktischer
Vernunft}\label{chapter:AufklaerungundWissenschaft}
\section{Aufklärung und Wissenschaft}\label{Abschnitt:EpistemischeArbeitsteilung}
Aufklärung ist eng mit der Entwicklung der modernen Wissenschaften verbunden.
Zwischen diesen Wissenschaften und ihren Organisationsformen auf der einen und der
Forderung der Aufklärung nach Mündigkeit und epistemischer Selbstbestimmung auf
der anderen Seite besteht jedoch ein kompliziertes Spannungsverhältnis. Die
unüberschaubar große Masse an wissenschaftlichen Erkenntnissen und die große
Bandbreite an universitären Disziplinen mit ihrer Vielfalt an Methoden und
zugehörigen Kompetenzen erlaubt es nicht, Experte auf jedem Gebiet zu sein.
Ebenso wenig können wir auf diese Bandbreite unseres Wissens verzichten
und uns auf das Wissen beschränken, welches uns ohne Rückgriff auf das
Wissen anderer zugänglich ist.\footnote{In der \titel{Idee zu einer
allgemeinen Geschichte in weltbürgerlicher Absicht} betont
\name[Immanuel]{Kant}, die Vernunft wirke \enquote{selbst nicht instinktmäßig,
sondern} bedürfe \enquote{Versuche, Übung und Unterricht, um von einer Stufe der
Einsicht zur andern allmählich fortzuschreiten.} Sie bedürfe gar \enquote{einer
vielleicht unabsehlichen Reihe von Zeugungen, deren eine der andern ihre
Aufklärung überliefert} \mkbibparens{\cite[][A
389]{Kant:IdeezueinerallgemeinenGeschichteinweltbuergerlicherAbsicht1977},
\cite[][VIII: 19.7--8]{Kant:GesammelteWerke1900ff.}}.} Dass wir auf unsere
arbeitsteilige Wissensgesellschaft gerade auch in unserem Bemühen um Aufklärung und Mündigkeit
angewiesen sind, zeigt sich ja darin, dass Aufklärung mit der Entwicklung der
modernen, kooperativen und arbeitsteiligen Wissenschaften entstand und durch
diese vorangetrieben wurde.


Wissenschaft ist nach \name[Immanuel]{Kant} ein Generationen übergreifendes
Gemeinschaftsunternehmen, das ein Einzelner in Zusammenarbeit mit anderen immer nur ein kleines Stück voranbringt,
nachdem er aufgreift und verinnerlicht, was andere vor ihm
bewerkstelligten.\footnote{\cite[Vgl.][A
388\,f.,]{Kant:IdeezueinerallgemeinenGeschichteinweltbuergerlicherAbsicht1977}
\cite[VIII: 18.29--19.16]{Kant:GesammelteWerke1900ff.}, sowie \cite[A
322\,f.,]{Kant:AnthropologieinpragmatischerHinsicht1977} \cite[VII:
325.30--326.9]{Kant:GesammelteWerke1900ff.}.} Er
beschreibt die Wissenschaften in Analogie zu Gewerbe und Handwerk
\emph{expressis verbis} als
arbeitsteilig.\footnote{\phantomsection\label{Fussnote:EpistemischeArbeitsteilungGMS}\cite[Vgl.][BA
v--vii]{Kant:GrundlegungzurMetaphysikderSitten1965}, \cite[][IV:
388.15--389.4]{Kant:GesammelteWerke1900ff.}.
\cite[Vgl.][67]{Brandt:UniversitaetzwischenSelbst-undFremdbestimmung2003}.} Der
Versuch, diese Aufteilung in Disziplinen mit ihren jeweiligen Experten
rückgängig zu machen, könne nur Stümperei (\name[Immanuel]{Kant} sagt: \enquote{Barbarei})
hervorbringen.\footnote{\cite[Vgl.][BA
vi]{Kant:GrundlegungzurMetaphysikderSitten1965}, \cite[][IV:
388.19--21]{Kant:GesammelteWerke1900ff.}: \enquote{Wo die Arbeiten so nicht
unterschieden und verteilt werden, wo jeder ein Tausendkünstler ist, da liegen
die Gewerbe noch in der größten Barbarei.}} Schließlich ermöglicht erst die
Aufteilung der epistemischen Arbeit in sich gegenseitig ergänzende Teilschritte
die Entstehung entsprechender Spezialisten, die für die nötige Qualität der
einzelnen Arbeitsschritte sorgen können. Eine solche
arbeitsteilige Wissenschaft und mit ihr die Qualität und Quantität ihrer
Ergebnisse gibt es aber nur durch das Vertrauen auf das Wissen von
Experten auf verschiedenen Gebieten. Ein naives Verständnis der Mündigkeits- und
Selbständigkeitsforderung, wie sie sich in dem Bemühen zeigte, \emph{jede}
Behauptung nur dann zu akzeptieren, wenn wir ihre Wahrheit unabhängig von
anderen kontrollieren können, brächte uns um die Früchte der modernen
Wissenschaften. Es kann also nicht um die \emph{Vermeidung} epistemischer
Abhängigkeiten gehen, sondern nur um deren \emph{Vereinbarung} mit der Forderung
nach epistemischer Selbständigkeit: Es gilt die Frage aufzuwerfen, wie
\enquote{Autonomie angesichts epistemischer Abhängigkeiten}\footnote{So der
Titel von \cite{Scholz:AutonomieangesichtsepistemischerAbhaengigkeiten2001}.} aussehen
kann. \name[Immanuel]{Kant} sagt nicht, sich seiner eigenen Vernunft zu bedienen heiße, alles
selbst erkennen zu wollen und sich generell nicht auf das Wissen anderer zu
berufen. Wir müssen, sollen und wollen Erkenntnisse auch von Anderen übernehmen,
aber wir müssen uns dabei mündig verhalten.

Die genannten Spannungen hinter dem Streben nach Autonomie und Authentizität im
Denken betreffen ausschließlich die Bestrebungen, die ich vorhin als
\emph{liberale} Aufklärung bezeichnete,\footnote{Siehe Kapitel
\ref{Benennung:LiberaleAufklaerung},
S.~\pageref{Benennung:LiberaleAufklaerung}.} nicht aber eine
\distanz{szientistische} Aufklärung, die die Ersetzung eines veralteten
(religiösen oder metaphysischen) Weltbildes durch ein neues, wissenschaftlich
fundiertes Weltbild anstrebt. Die szientistische Aufklärung fordert
Wissenschaftlichkeit und Objektivität, nicht Autonomie und Authentizität. Kritik
und Selbstdenken gelten ihr zwar als Forschungsmaximen \emph{innerhalb} der
Wissenschaften, die beispielsweise in der Forderung nach Überprüfbarkeit der
Ergebnisse zum Ausdruck kommen. Sie sollen aber nicht das Verhältnis des Laien
zu den Wissenschaften bestimmen. Und dies scheint sinnvoll und geboten, gerade
weil -- wie vorhin unter Rekurs auf
\authorfullcite{Wolff:Discursuspraeliminarisdephilosophiaingenere1996} betont
wurde -- Selbstdenken und Mündigkeit Kompetenzen voraussetzt, die der Laie
\emph{per definitionem} nicht besitzt. Weil liberale Aufklärung ganz wesentlich
darin besteht, das einzelne Subjekt zu Mündigkeit und Selbständigkeit, zu
Unabhängigkeit von Autoritäten und Traditionen zu erziehen, gibt es einen
Konflikt, der sich auch als Konflikt zwischen Aufklärung und Wissenschaft
beschreiben lässt. In offener Konfrontation gegen Wissenschaftlichkeit und
Objektivität wird Aufklärung freilich nicht bestehen können. Auch \name[Immanuel]{Kant}
verteidigt die Vorrangstellung der Wissenschaft gegenüber dem, was man einen
\enquote{gesunden Menschenverstand} nennen könnte.\footnote{Siehe oben, Kap.
\ref{subsection:SelbstdenkenbeiKant} dieser Arbeit.} Und deswegen ist liberale Aufklärung nur
dann eine ernstzunehmende Option, wenn es gelingt, zu einer Vermittlung von Aufklärung und Wissenschaft zu
gelangen.\footnote{\cite[Vgl.][837]{Schnaedelbach:WirKantianer2005}:
\enquote{Aufklärung durch Wissenschaft allein verfehlt ihr Ziel und endet
notwendig im Dogmatismus, wenn sich die Wissenschaft nicht über
sich selbst aufklärt; umgekehrt bekommt die Aufklärung, wenn sie sich von der
Wissenschaft fernzuhalten sucht, keinen Boden unter die Füße und verheddert
sich unvermeidlich in ihren skeptischen Argumenten.}}

Möglicherweise soll die Vernunftkritik den Weg weisen, Aufklärung und Wissenschaft in ein
zuträgliches Verhältnis zueinander zu setzen. Dies ist der Vorschlag
\authorfullcite{Schnaedelbach:WirKantianer2005}s, der auf den in der \titel{Geschichte der
reinen Vernunft} als einzigen Ausweg genannten kritischen
Weg\footnote{\cite[Vgl.][B~884]{Kant:KritikderreinenVernunft2003},
\cite[III: 552.19]{Kant:GesammelteWerke1900ff.}.} verweist, den \name[Immanuel]{Kant}, vor
genau dieses Dilemma gestellt,
propagiere.\footnote{\cite[Vgl.][837]{Schnaedelbach:WirKantianer2005}:
  \enquote{Das ist dann auch der Grund, warum alle ernst zu nehmenden
    Philosophen seit Kant zunächst einmal den kritischen Weg
    einschlagen mussten, und dies auch und gerade dann, wenn sie
    vorhatten, sich von Kant abzuwenden.}} Die vom Ausgangspunkt der
Endlichkeit menschlichen Denkens aus verfasste \titel{Kritik der reinen
Vernunft} sei gerade deswegen Kernstück der Aufklärung, weil die Einsicht in
unsere Endlichkeit und deren Konsequenzen die Vermittlung von Wissenschaft auf
der einen und Selbständigkeit auf der anderen Seite ermögliche. Aber auch wenn
dem so sein sollte, bleibt eine große Bandbreite an Möglichkeiten, dieses
Verhältnis von Aufklärung und Wissenschaft und die Rolle der Vernunftkritik bei
diesem Prozess zu interpretieren. Aus der Sicht \name[Herbert]{Schnädelbach}s geht es
primär um Metaphysikkritik; und eine Passage aus der Vorrede zur ersten Auflage
der \titel{Kritik der reinen Vernunft}, die eine Verbindung zwischen
Aufklärungsdenken und Vernunftkritik zumindest zu erahnen erlaubt, mag diese
Vermutung stützen. \name[Immanuel]{Kant} bezieht sich darin auf die Gleichgültigkeit, die
sein Zeitalter der Metaphysik entgegenbringe, und schreibt:
\begin{quote}
  Sie [die Gleichgültigkeit gegenüber Fragen der Metaphysik; A.\,G.] ist
  offenbar die Wirkung nicht des Leichtsinns, sondern der gereiften
  \ori{Urteilskraft} des Zeitalters, welches sich nicht länger durch
  Scheinwissen hinhalten läßt und eine Auffoderung an die Vernunft, das
  beschwerlichste aller ihrer Geschäfte, nämlich das der Selbsterkenntnis aufs
  neue zu übernehmen und einen Gerichtshof einzusetzen, der sie bei ihren
  gerechten Ansprüchen sichere, dagegen aber alle grundlose Anmaßungen, nicht
  durch Machtsprüche, sondern nach ihren ewigen und unwandelbaren Gesetzen,
  abfertigen könne und dieser ist kein anderer als die \ori{Kritik der reinen
  Vernunft} selbst.\footnote{\cite[][A xi-xii]{Kant:KritikderreinenVernunft2003},
  \cite[][IV: 9.1--10]{Kant:GesammelteWerke1900ff.}.}
\end{quote}
Ob etwas als Wissenschaft taugt, erkennt man nach \name[Immanuel]{Kant} daran,
ob es als \emph{gemeinsames} Erkenntnisprojekt betrieben werden kann. Denn Vernunft
ist an die Möglichkeit eines allgemeinen Standpunkts und des
intellektuellen Austauschs gebunden, wie sich in Kapitel
\ref{section:sensuscommunis} bei der Betrachtung der Maxime der erweiterten
Denkungsart ergab. Was als Kampfplatz einzelner Denker auftritt, die sich nicht
auf eine gemeinsame Grundlage und einen gemeinsamen Forschungsstand verständigen
können, ist (allem Vermuten nach) keine Wissenschaft.\footnote{\enquote{Ob die
Bearbeitung der Erkenntnisse, die zum Vernunftgeschäfte gehören, den sicheren
Gang einer Wissenschaft gehe oder nicht, das läßt sich bald aus dem Erfolg beurteilen.
Wenn \punkt{} es nicht möglich ist, die verschiedenen Mitarbeiter in der Art,
wie die gemeinschaftliche Absicht erfolgt werden soll, einhellig zu machen: so
kann man immer überzeugt sein, daß ein solches Studium bei weitem noch nicht den
sicheren Gang einer Wissenschaft eingeschlagen, sondern ein bloßes Herumtappen
sei} (\cite[][B vii]{Kant:KritikderreinenVernunft2003},
\cite[][III: 7.2--4, 7--11]{Kant:GesammelteWerke1900ff.}).} Die Vernunftkritik hat
die Aufgabe, echte Wissenschaft von bloß angemaßter Wissenschaftlichkeit
zu scheiden, wobei sie dem Bereich dessen, was nicht zur Wissenschaft taugt,
durchaus Achtung zollt und seine Bedeutung betont. Schließlich soll sie
\enquote{das \ori{Wissen} aufheben, um zum \ori{Glauben Platz} zu
bekommen}\footnote{\cite[B xxx]{Kant:KritikderreinenVernunft2003},
\cite[III: 19.6]{Kant:GesammelteWerke1900ff.}.}. Schafft Vernunftkritik also
einen Freiraum für aufgeklärte Religion, indem sie einer aufgeklärten
Selbständigkeit individueller Denker auf der einen und gemeinschaftlich
betriebener Wissenschaft auf der anderen Seite je eigene Bereiche zuweist?

Diese erste und naheliegende, dennoch aber falsche Deutung verweist
auf die Begrenzung von
Geltungsansprüchen.\footnote{\authorfullcite{LaRocca:WasAufklaerungseinwird2004}
spricht hingegen statt von \emph{Begrenzung} von der \emph{Differenzierung} von
Geltungsansprüchen: \enquote{Für das Projekt Aufklärung ist die kritische
Differenzierung von Geltungsbereichen wesentlicher als die bloße Gegenüberstellung
Vernunft/Aberglaube} \parencite[135]{LaRocca:WasAufklaerungseinwird2004}. Dies
entspricht der Deutungsrichtung, die \name[Immanuel]{Kant} als Philosophen der modernen Kultur liest; so z.\,B.
Heinrich \textcite[][141]{Rickert:KantalsPhilosophdermodernenKultur1924}:
\enquote{Kant hat als erster Denker in Europa die \ori{allgemeinsten theoretischen Grundlagen}
geschaffen, die wissenschaftliche Antworten auf spezifisch moderne
Kulturprobleme überhaupt \ori{möglich} machen, und insbesondere läßt sich dartun: sein
Denken, wie es sich in seinen drei großen Kritiken darstellt, ist in dem Sinn
\enquote{kritisch}, d.\,h. \ori{scheidend} und \ori{Grenzen ziehend} gewesen,
daß es dadurch im Prinzip dem Prozeß der \ori{Verselbständigung} und \ori{Differenzierung} der Kultur
entspricht, wie er sich seit dem Beginn der Neuzeit faktisch vollzogen, aber in
der Philosophie vor Kant noch keinen theoretischen Ausdruck gefunden hatte.}
Ähnliches behauptet im Anschluss an \authorcite{Rickert:KantalsPhilosophdermodernenKultur1924} auch
\textcite[vgl.][838]{Schnaedelbach:WirKantianer2005}.} Ihre Vertreter könnten
etwa folgendes anführen:
\begin{quote}
\enquote{Wir können, dürfen und sollen uns auf Autoritäten, Traditionen und unsere
Mitmenschen verlassen, aber nur dort, wo objektiv gültiges Wissen vorliegt oder
zumindest nach unserem Kenntnisstand vorliegen könnte. Bei den
Naturwissenschaften und der Mathematik ist dies der Fall -- deswegen ist niemand
unmündig, wenn er den Lehrbüchern von Physik, Chemie, Biologie oder Algebra
Glauben schenkt. Die Vernunftkritik beschränkt den Bereich möglichen Wissens auf das, was Gegenstand
möglicher Erfahrung ist. Wissenschaft im eigentlichen Sinne ist eine
mathematisch fundierte\footnote{\cite[Vgl.][A
viii]{Kant:MetaphysischeAnfangsgruendederNaturwissenschaften1977}, \cite[IV:
470.13--15]{Kant:GesammelteWerke1900ff.}: \enquote{Ich behaupte aber, daß in
jeder besonderen Naturlehre nur so viel eigentliche Wissenschaft angetroffen
werden könne, als darin Mathematik anzutreffen ist.}} Erfahrungserkenntnis nach
dem Vorbild \name[Isaac]{Newton}s. In der Metaphysik -- und damit auch in der
natürlichen Theologie -- gibt es daher keine gültigen Wissensansprüche, also ist
unmündig, wer Aussagen zur Metaphysik Glauben schenkt.}
\end{quote}
Eine solche Konzeption versucht, die von der Aufklärung geforderte Selbständigkeit
dadurch mit der Autorität der Wissenschaft zu vereinbaren, dass sie beiden
Seiten jeweils eigene Bereiche zuweist.


\phantomsection\label{ThomasiusZuPrivatheitreligioesenBekenntnisses}\name[Christian]{Thomasius}
begründet die Tatsache, dass Häresie nicht justiziabel ist, mit
dem Mangel an Wissenschaftlichkeit religiöser Behauptungen. Gerade weil es bei
vielen Fragen im Bereich der Religion keine allgemein verbindlichen und
wissenschaftlich beglaubigten Wahrheiten gebe, könne, dürfe und solle jeder
seinem je eigenen Urteil in Fragen der Religion
folgen.\footnote{\cite[Vgl.][\pno~244\,f.]{Albrecht:ChristianThomasius1999}.}
Und in dieselbe Richtung scheint es zu weisen, wenn \name[Immanuel]{Kant} der
Vernunftkritik die Aufgabe zuweist, das Wissen zu begrenzen, um dem Glauben
Platz zu verschaffen: Weil eben niemand ein fundiertes Expertenvotum bezüglich
der Wahrheit in Religionsfragen, die er zum Zentrum seines Aufklärungsdenkens
erklärt\footnote{\cite[Vgl.][A
492\,f.,]{Kant:BeantwortungderFrage:WasistAufklaerung?1977} \cite[][VIII:
41.10--22]{Kant:GesammelteWerke1900ff.}.}, abgeben könne, deshalb solle jeder
sich ein eigenes Urteil bilden.\footnote{Belege \emph{für} diese Deutung liefern
Stellen wie die folgende: \enquote{Ich kann also nur sagen: \ori{Ich} sehe mich
durch meinen Zweck nach Gesetzen der Freiheit genötiget, ein höchstes Gut in der
Welt als möglich anzunehmen, aber \ori{ich kann keinen andern durch Gründe
nötigen} (der Glaube ist \ori{frei})} (\cite[][A
104]{Kant:ImmanuelKantsLogik1977}, \cite[][IX:
69.22--25]{Kant:GesammelteWerke1900ff.}).} Denn hörten wir bei solchen Dingen
auf andere, so billigten wir ihnen eine Autorität zu, die sich nicht durch ihr
Wissen als legitim ausweisen lässt. Wir vertrauten auf unechte, nur angemaßte
Experten. Und dasselbe gelte schließlich auch für die Philosophie, die (noch)
keine Wissenschaft sei und die man deshalb nicht lernen
könne.\footnote{\cite[Vgl.][B 866]{Kant:KritikderreinenVernunft2003}, \cite[III:
542.12--14]{Kant:GesammelteWerke1900ff.}: \enquote{Bis dahin kann man keine
Philosophie lernen; denn, wo ist sie, wer hat sie im Besitze, und woran läßt sie
sich erkennen?}}

Eine solche Lösung scheint auf den ersten Blick naheliegend zu sein und dabei
zugleich einen möglichen Zusammenhang von Endlichkeit und
Aufklärungsprogrammatik aufzuzeigen: Wie in Kapitel
\ref{subsubsection:UnterscheidungvonDenkenundErkennen} gezeigt, ist die
Unterscheidung von Möglichkeit und Wirklichkeit der zentrale Ausdruck unserer
Endlichkeit. Und diese Unterscheidung wird mitunter als Grundlage unserer
(individuellen) Freiheit und der Aufklärung
interpretiert.\footnote{\cite[Vgl.][283]{Engfer:MenschlicheVernunft2002}.} Im Denken frei wären wir dann dort, wo wir
verschiedene Möglichkeiten \emph{denken}, aber ihre Wahrheit oder Falschheit
nicht \emph{erkennen} können. Das \emph{Denken} soll also gänzlich frei und
ungebunden sein, wohingegen das \emph{Erkennen} sich nach
objektiven Regeln der Wissenschaftlichkeit richtet und auf die Erfahrung als
Aktualisierung eines \emph{rezeptiven} Erkenntnisvermögens angewiesen bleibt.

Auf den zweiten Blick verliert sich der Reiz dieser Konzeption und sie erweist
sich als äußerst unbefriedigend. In Wahrheit expliziert sie nämlich gar keinen
Begriff autonomen und authentischen Wissens, sondern nennt zwei Bereiche, wobei
wir in dem einen Bereich Wissen erwerben und in dem anderen vielleicht
authentisch sind, aber weder in dem einen noch in dem anderen Bereich handelt es sich um
Autonomie. Denn nach diesem Vorschlag sollen in einem Bereich Regeln der
Vernunft walten (Wissenschaft), während wir in einem \emph{anderen} Bereich frei
sein dürfen (Religion und Metaphysik). Autonomie aber verlangt, dass Freiheit
durch Regeln der Vernunft konkretisiert wird und die Regeln der Vernunft Regeln der
Freiheit sind. Freiheit, Unterwerfung unter Regeln der Vernunft und damit die
Möglichkeit der kollektiven Verständigung müssen \emph{in demselben
Erkenntnisbereich} vorliegen, soll sinnvoll von Autonomie gesprochen werden
können.
 
Die genannte Deutung ist  nicht nur systematisch unbefriedigend, sondern auch
als \name[Immanuel]{Kant}interpretation fragwürdig. Denn \name[Immanuel]{Kant}
akzeptiert keine zügellose Freiheit außerhalb der Wissenschaft, sondern fordert
auch von Religion und Ethik -- zwei Bereichen, die er eindeutig \emph{nicht} den
Erfahrungswissenschaften zuweist -- Vernünftigkeit und Mitteilbarkeit. Wenn
verschiedene Moralphilosophien vertreten werden und Menschen verschiedenen
Religionen und Konfessionen angehören, wie es \emph{de facto} der Fall ist, dann
zeigt dies die Unzulänglichkeit der (meisten) eingenommenen Positionen und
religiösen Überzeugungen. Schließlich gebe es nur eine einzige \emph{korrekte}
Moralphilosophie und nur eine einzige \emph{wahre} Religion für alle
Menschen.\footnote{Zu letzterem siehe z.\,B. \cite[][A
45]{Kant:DerStreitderFakultaeten1977}, \cite[VII:
36.26--32]{Kant:GesammelteWerke1900ff.}.} Dass \name[Immanuel]{Kant} den Bereich
der religiösen Überzeugungen nicht der Beliebigkeit anheim stellt, kann zu dem Urteil
verleiten, er sei in seiner Religionskritik in aufklärerischer Absicht auf
halbem Wege stehen geblieben, insofern er versucht, religiöse Inhalte aus reiner
Vernunft zu
generieren.\footnote{\cite[Vgl.][135--137]{LaRocca:WasAufklaerungseinwird2004}.
\authorcite{Horkheimer:DialektikderAufklaerung1997} werten es als Inkonsequenz des
Aufklärungsdenkens, überhaupt Inhalte (außer der reinen Selbsterhaltung) als
vernünftig ausweisen zu wollen. Der einzig konsequente \singlequote{Aufklärer}
sei daher der \name[Donatien-Alphonse-Fran{\c{c}}ois]{Marquis de Sade} gewesen;
\cite[vgl.][101]{Horkheimer:DialektikderAufklaerung1997}: \enquote{Jedes
inhaltliche Ziel, auf das die Menschen sich berufen mögen, als sei es eine
Einsicht der Vernunft, ist nach dem strengen Sinn der Aufklärung Wahn, Lüge,
\singlequote{Rationalisierung}, mögen die einzelnen Philosophen sich auch die
größte Mühe geben, von dieser Konsequenz hinweg aufs menschenfreundliche Gefühl zu lenken.}}
Es ist aber keine Inkonsequenz, wenn \name[Immanuel]{Kant} es gerade nicht für Ziel und
Ergebnis einer aufgeklärten Religionskritik ansieht, \enquote{jede religiöse
Erfahrung als gleichrangige Möglichkeit zu
berücksichtigen}\footcite[][136]{LaRocca:WasAufklaerungseinwird2004}.
Dies ergibt sich vielmehr zwingend aus der Darstellung der Aufklärungsforderung
in der Form der drei Maximen des Denkens, die nicht nur von Wissenschaften,
sondern von \emph{jedem} vernünftigen Gedanken fordern, dass er der Vernunft und
der erweiterten Denkungsart gemäß ist.\footnote{Siehe oben, Kap.
\ref{section:KantalsliberalerAufklaerer}.}

Daraus ergibt sich auch die Bestimmtheit, mit der er auf Offenbarung gegründete
religiöse Überzeugungen kritisiert; ein nennenswertes Beispiel findet sich im
\titel{Streit der Fakultäten}, wo \name[Immanuel]{Kant} sich nicht scheut,
\singlename{Abraham}s Vertrauen in die Stimme Gottes -- welche ihm befiehlt,
seinen eigenen Sohn als Opfer darzubringen -- der Irrationalität zu
schelten.\footnote{\enquote{\singlename{Abraham} hätte auf diese vermeinte
göttliche Stimme antworten müssen: \enquote{daß ich meinen guten Sohn nicht töten solle, ist ganz
gewiß; daß aber du, der du mir erscheinst, Gott sei, davon bin ich nicht gewiß,
und kann es auch nicht werden, wenn sie auch vom (sichtbaren) Himmel
herabschallete}.} \mkbibparens{\cite[][A
102\,f.]{Kant:DerStreitderFakultaeten1977}; \cite[][VII:
63.34--38]{Kant:GesammelteWerke1900ff.}}. Siehe hierzu auch Kapitel
\ref{Beispiel:AbrahamOpfertSeinenSohn}, ab
S.~\pageref{Beispiel:AbrahamOpfertSeinenSohn}.} Jede religiöse Überzeugung --
auch und gerade Offenbarungserlebnisse -- sollen sich innerhalb der Grenzen der
bloßen Vernunft bewegen. Gerade weil er Aufklärung und Mündigkeit in
Religionsfragen fordert, kann er keine Beliebigkeit zulassen, sondern muss die Möglichkeit verbindlicher
Vernunftwahrheiten voraussetzen. Und deswegen gibt es laut \name[Immanuel]{Kant}
auch nur \emph{eine} Religion, die freilich in verschiedenen Ausprägungen
auftreten kann. Doch ist diese Varietät als ein Mangel anzusehen, den es zu
überwinden gilt.\footnote{\cite[Vgl.][B
167--183]{Kant:DieReligioninnerhalbderGrenzenderblossenVernunft1977}, \cite[VI:
115.1--124.5]{Kant:GesammelteWerke1900ff.}.}

\phantomsection\label{Absatz:UniversalitaetderAufklaerung} Ebenso wenig
akzeptiert er Unmündigkeit in Bereichen, deren Behandlung der
Erfahrungswissenschaft zusteht, denn \name[Immanuel]{Kant} betont stets,
Aufklärung sei die Maxime, \emph{jederzeit} selbst zu
denken,\footnote{\cite[Vgl.][A
229]{Kant:Washeisst:SichimDenkenorientieren?1977}, \cite[VIII:
146.30--31]{Kant:GesammelteWerke1900ff.}.} und die Maxime einer vorurteilsfreien
Denkungsart sei die einer \emph{niemals} passiven
Vernunft\footnote{\cite[Vgl.][\S~40]{Kant:KritikderUrteilskraft2009}, \cite[V:
294.20]{Kant:GesammelteWerke1900ff.}.}. Die Aufforderung, unseren eigenen
Verstand zu gebrauchen, betrifft jeden Erkenntnisbereich und alle Urteile, nicht
eine Auswahl aus diesen. Wir können keine Vereinbarkeit von individueller
Selbstbestimmung im Denken mit der Autorität wissenschaftlicher Erkenntnisse
herbeiführen, indem wir die Geltungsansprüche der letzteren beschneiden.

Gewiss bleibt an dieser Stelle noch die Deutungsmöglichkeit, Wissenschaft und
Aufklärung durch die \emph{methodische Orientierung} (nicht wie bei
\authorcite{Wolff:Psychologiaempirica1968} an der Mathematik, sondern) \emph{an den
Naturwissenschaften} zu vereinbaren. Selbstdenken sei dann also gerade in den
Wissenschaften und \emph{nicht} im Bereich der Religion zu verorten.
\name[Immanuel]{Kant} sagt, dass Aufklärung und Mündigkeit demjenigen leicht
fielen, der \enquote{das, was über seinen Verstand ist, nicht zu wissen
verlangt}\footnote{\cite[\S 40]{Kant:KritikderUrteilskraft2009}, \cite[V:
294.32--33]{Kant:GesammelteWerke1900ff.}.}. Der aufgeklärte Mensch wäre
derjenige, der die kritische, methodisch disziplinierte und auf stete Kontrolle
an der Erfahrung ausgerichtete Art naturwissenschaftlichen Denkens erlernt und
verinnerlicht hat und sich von \distanz{Spekulationen} über Unerkennbares fern
hält. Religion und Metaphysik wären dann gerade keine Bereiche, die freiem und
mündigem Denken offen stehen --  es sei denn es gelänge, sie auf eine feste
methodische Basis zu stellen. In Kapitel
\ref{section:KantalsliberalerAufklaerer} habe ich nur hervorgehoben, dass
\name[Immanuel]{Kant} \authorcite{Wolff:Psychologiaempirica1968} nicht darin
folgt, die Vereinbarkeit dadurch zu gewährleisten, dass er die Methode der
Mathematik überall zugrunde legt. Aber es könnte noch sein, dass er die Methodik
der Naturwissenschaften -- die stete Rückbindung an die Erfahrung oder auch die
experimentelle Methode -- für die geeignete Grundlage hält. Zumindest in
\name[Immanuel]{Kant}s \distanz{vorkritischer} Phase finden sich in seinen
Schriften Belege für die zweite Deutungsmöglichkeit.
\phantomsection\label{anm:jungerkantundaufklaerung}In den 1750er Jahren betrieb
\name[Immanuel]{Kant} Aufklärung aus der Perspektive einer
\name[Isaac]{Newton}'schen Naturphilosophie, indem er theologisch besetzte
Themen von moralphilosophischer Bedeutung -- z.\,B.\ das Seebeben von Lissabon
1755 in drei Abhandlungen aus dem darauf folgenden
Jahr\footnote{Siehe
\cite{Kant:VondenUrsachenderErderschuetterungenbeiGelegenheitdesUnglueckswelchesdiewestlicheLaendervonEuropagegendasEndedesvorigenJahresbetroffenhat1910},
\cite[][I: 417--427]{Kant:GesammelteWerke1900ff.},
\cite{Kant:GeschichteundNaturbeschreibungdermerkwuerdigstenVorfaelledesErdbebenswelchesandemEndedes1755stenJahreseinengrossenTheilderErdeerschuetterthat1910},
\cite[][I: 429--461]{Kant:GesammelteWerke1900ff.},
\cite{Kant:FortgesetzteBetrachtungderseiteinigerZeitwahrgenommenenErderschuetterungen1910}, \cite[vgl.][I:
463--472]{Kant:GesammelteWerke1900ff.}.} -- nach den zu seiner Zeit aktuellen
Standards der Wissenschaften untersuchte.  In dieser Hinsicht erweist sich
\name[Immanuel]{Kant} bis zu Beginn der 1760er Jahre tatsächlich als
Vertreter einer naturwissenschaftlich geprägten Aufklärung.\footnote{\name[Immanuel]{Kant}
sucht dabei nicht für die Theologie nach einer geeigneten Methodik, sondern für eine Grundlegung der Metaphysik.
Diese sollte auf Grundlage sicherer Erfahrungssätze möglich sein, wie er 1763
schreibt. \cite[Vgl.][A
69f.]{Kant:UntersuchungueberdieDeutlichkeitderGrundsaetzedernatuerlichenTheologieundderMoral1977},
\cite[][II: 275.8--24]{Kant:GesammelteWerke1900ff.}.}

Doch diese \distanz{szientistische} Vorstellung von Aufklärung unterscheidet
sich ganz offensichtlich von dem, was \name[Immanuel]{Kant} in der 1780er Jahren
in der Berlinischen Monatsschrift propagiert. Aushängeschild hierfür
ist seine Betonung der Religion als Themenbereich, der Mündigkeit und
Selbstdenken verlange und zulasse, was zusammen mit der Trennung von Wissen und
Glauben in der \titel{Kritik der reinen Vernunft} deutlich macht, dass
Selbstdenken nicht durch eine Orientierung an der
newtonschen Physik zu verstehen sei. \name[Immanuel]{Kant}s spätes
Aufklärungsverständnis ist das Ergebnis einer Entwicklung zwischen den 1760er
und den 1780er Jahren. In einer Anmerkung in seinem Durchschussexemplar der
\titel{Beobachtungen über das Schöne und Erhabene} relativiert \name[Immanuel]{Kant} in
Anlehnung an \name[Jean-Jacques]{Rousseau} den allgemeinen Wert von Wissenschaft und
Forschung, welche nur im Kontext der Herstellung der \enquote{[R]echte der
Menschheit} einen Wert
erhielten.\footnote{\phantomsection\label{Anmerkung:Rousseauhatmichzurechtgebracht}\cite[Vgl.][XX:
44.8-16]{Kant:GesammelteWerke1900ff.}: \enquote{Ich selbst bin aus Neigung ein
Forscher. Ich fühle den gantzen Durst nach Erkenntnis u.\ die begierige Unruhe
darin weiter zu kommen oder auch die Zufriedenheit bey jedem Erwerb. Es war eine
Zeit da ich glaubte dieses allein könnte die Ehre der Menschheit machen u.\ ich
verachtete den Pöbel der von nichts weis. \ori{\name[Jean-Jacques]{Rousseau}} hat mich zurecht
gebracht. Dieser verblendete Vorzug verschwindet, ich lerne die Menschen ehren
u.\ ich würde mich unnützer finden wie den gemeinen Arbeiter wenn ich nicht
glaubete daß die Betrachtung allen übrigen einen Werth ertheilen könne, die
[R]echte der Menschheit herzustellen}. Zur Problematik der Datierung siehe
\cite[][67--71]{Schwaiger:KategorischeundandereImperative1999}.} Dies markiert
die Abkehr von dem szientistischen
Aufklärungsverständnis. \Revision{\authorcite{Macor:DieBestimmungdesMenschen1748--18002013}
schreibt, \name[Immanuel]{Kant} nehme fortan \enquote{entschieden
  gegen intellektualistische Positionen
  Stellung}\footcite[][203]{Macor:DieBestimmungdesMenschen1748--18002013}
und sehe den Wert des Menschen in der Moral, die -- so wird sich noch
als bedeutsam herausstellen --, mitnichten von besonderen, nur wenigen
vorbehaltenen Kenntnissen und Kompetenzen abhängig sei.} Diese Abkehr
drückt sich eben auch in dem Aufsatz über den Aufklärungsbegriff aus, wenn
\name[Immanuel]{Kant} die Religion und nicht die empirischen Wissenschaften
thematisiert und dies auch explizit hervorhebt. Religion ist aber nach
\name[Immanuel]{Kant} ebenso wie die Ethik nicht mit den Mitteln empirischer
Forschung anzugehen -- ganz im Gegenteil: Eine vernünftige Religion erwächst aus
reiner Vernunft, wie \name[Immanuel]{Kant} beispielsweise in der \titel{Kritik
der Urteilskraft} verdeutlicht. Sie baut ihre Fundamente auf der
Moralphilosophie auf, nicht auf der (teleologischen) Betrachtung der Natur, und
ist infolge dessen nicht die Weiterführung empirischer Naturforschung
(Physikotheologie), sondern apriorischer Ethik
(Ethikotheologie).\footnote{\cite[Vgl.][\S~85\,f.,]{Kant:KritikderUrteilskraft2009}
\cite[V: 436.3--447.13]{Kant:GesammelteWerke1900ff.}.}

Wenn \name[Immanuel]{Kant} die Aufklärung vornehmlich in Fragen der
Religion setzt, dann kann die Vereinbarkeit von Selbständigkeit und Wissenschaft
nicht in der Übernahme einer Methodik der Naturwissenschaften bestehen. Zu den
Themen der Aufklärung gehören gerade auch solche, die den Bereich der Erfahrung
transzendieren. Auch zeitlich kongruiert die Akzentverschiebung
\name[Immanuel]{Kant}s mit der Entwicklung seines Aufklärungsprogramms.
So schreibt \authorfullcite{Kreimendahl:Kant1990}: \enquote{Man darf \punkt{}
festhalten, dass Kants Aufklärungsbegriff in den 1750er und 1760er Jahren noch
in keiner wie
auch immer gearteten Verbindung zum Programm des Selbstdenkens, zu Mündigkeit
oder Autonomie des Subjekts
steht.}\footnote{\cite[][134]{Kreimendahl:KantsvorkritischesProgrammderAufklaerung2009}}
Erst in dem \distanz{stillen Jahrzehnt} zwischen 1770 und 1781 taucht die
Verbindung der Begriffe \enquote{Aufklärung} und \enquote{Autonomie} oder
\enquote{Selbstdenken} erstmalig im handschriftlichen Nachlass
auf\footnote{\cite[Vgl.][\pno~134\,f.]{Kreimendahl:KantsvorkritischesProgrammderAufklaerung2009}}
Damit fällt die Entwicklung des Aufklärungsbegriffs zwar \emph{zeitlich}
mit der Entstehung der kritischen Philosophie zusammen, aber der
\emph{inhaltliche} Zusammenhang -- wenn ein solcher überhaupt besteht -- wird
zunehmend undurchsichtig.


\section{Die Bestimmung des Menschen und die \emph{conditio
humana}}\label{subsection:DieBestimmungdesMenschen} Die Forderung nach
Mündigkeit und epistemischer Selbstbestimmung wäre einfacher zu erfüllen, wenn
sie sich nur auf einen eingeschränkten Themenbereich bezöge; damit wäre eine
Vereinbarkeit mit der Ansicht gegeben, wir könnten uns zumindest in den meisten
Bereichen auf das Wissen von Experten verlassen, ohne dadurch unsere Mündigkeit
zu gefährden. Einer solchen Einschränkung widersprechen aber eindeutige
Äußerungen \name[Immanuel]{Kant}s. Die Verbindung von Aufklärung und
Selbstdenken verträgt sich nicht mit einer thematischen
\emph{Einschränkung} der Aufklärung; sie geht aber mit einer
\emph{Akzentuierung} -- und zwar in der Aufklärungsschrift: der Religion --
einher. Vielleicht mögen Akzentsetzungen helfen, die Forderung nach Mündigkeit
und Selbständigkeit erfüllbar zu halten: Niemand ist in der Lage, in
\emph{allen} Erkenntnisbereichen kompetent zu sein. Aber wenn jeder nur in
Bereichen kompetent sein soll, die von besonderer Relevanz sind, muss die
Forderung nach Mündigkeit und Selbständigkeit nicht überschwänglich
wirken.\footnote{\cite[Vgl.][\S~40]{Kant:KritikderUrteilskraft2009}, \cite[V:
294.29--37]{Kant:GesammelteWerke1900ff.}.}

\subsection{Die Ordnung der Erkenntnisse}
\name[Immanuel]{Kant} begründet seine Akzentuierung der Religion im Rahmen der
Aufklärungsprogrammatik mit dem Mangel an Aufmerksamkeit der Herrschenden auf
die Wissenschaften.\footnote{\cite[Vgl.][A
492]{Kant:BeantwortungderFrage:WasistAufklaerung?1977}, \cite[][VIII:
41.10--14]{Kant:GesammelteWerke1900ff.}.} In der Tat scheint es für eine
Regierung kaum von Interesse zu sein, in die naturwissenschaftliche Forschung
und Lehre einzugreifen, denn ein ungehindertes und erfolgreiches Forschen kann
ihr dort schon wegen der Entwicklung technischer Möglichkeiten nur von Vorteil
sein. Es ist -- um ein prominentes und sicherlich auch \name[Immanuel]{Kant}
bekanntes Beispiel zu verwenden -- kaum vorstellbar, dass Christian
\authorcite{Wolff:Psychologiaempirica1968} vom preußischen König für seine Lehr-
und Publikationstätigkeit sanktioniert worden wäre, wenn er sich auf Themen der
angewandten Mathematik beschränkt hätte.
Nur bei Lehre sowie Vortrags- und Publikationstätigkeit im Bereich der Religion
-- wie in \authorcite{Wolff:Psychologiaempirica1968}s Prorektoratsrede über die
praktische Philosophie der
Chinesen\footnote{\cite[Vgl.][]{Wolff:OratiodeSinarumphilosophiapractica1988}.
Zur Bedeutung der Prorektoratsrede für sein weiteres Schicksal siehe
\cite[][\pno~xlvi--liii]{Albrecht:Einleitung1988}.} -- wurde ein Eingreifen
denkbar, weil die Regierung hier ein eigenes Interesse haben konnte. Ein solches
Interesse der Regierungen an der Beeinflussung von Lehre und
Publikationstätigkeit nennt \name[Immanuel]{Kant} an anderer Stelle als
Kennzeichen der oberen Fakultäten gegenüber der freien unteren Fakultät. Es
handelt sich bei den oberen Fakultäten um diejenigen Disziplinen, mit denen sich
eine Regierung Einfluss auf die Verhaltensweise des Volkes verschaffen
kann.\footnote{\cite[Vgl.][A 6--8]{Kant:DerStreitderFakultaeten1977},
\cite[][VII: 18.30--19.20]{Kant:GesammelteWerke1900ff.}.} Dagegen ist die untere
(philosophische) Fakultät diejenige, die ihre Lehren nicht auf Befehl der
Regierung annimmt, sondern die Interessen einer aufgeklärten Mündigkeit
vertritt.\footnote{\cite[Vgl.][A~~25]{Kant:DerStreitderFakultaeten1977},
\cite[][VII: 27.32-35]{Kant:GesammelteWerke1900ff.}: \enquote{Also wird die
philosophische Fakultät, darum, weil sie für die \ori{Wahrheit} der Lehren, die
sie aufnehmen, oder auch einräumen soll, stehen muß, in so fern als frei und nur
unter der Gesetzgebung der Vernunft, nicht aber der Regierung stehend gedacht
werden müssen.}} Zwischen diesen Fakultäten sei kein dauerhafter Friedenszustand
möglich, sondern ein Aufbrechen des Konflikts jederzeit zu
befürchten,\footnote{\cite[Vgl.][A~35-42]{Kant:DerStreitderFakultaeten1977},
\cite[VII: 32.13-35.27]{Kant:GesammelteWerke1900ff.}.} was der Behauptung
korrespondiert, es gebe kein aufgeklärtes Zeitalter, sondern nur ein Zeitalter
der Aufklärung.\footnote{\cite[Vgl.][A
491]{Kant:BeantwortungderFrage:WasistAufklaerung?1977}, \cite[][VIII:
40.17--19]{Kant:GesammelteWerke1900ff.}: \enquote{Wenn den nun gefragt wird:
Leben wir jetzt in einem \ori{aufgeklärten} Zeitalter? so ist die Antwort: Nein,
aber wohl in einem Zeitalter der \ori{Aufklärung}.}} Die
oberen Fakultäten seien beliebte Ansprechpartner für all diejenigen, die sich in ihrer Unmündigkeit eingerichtet haben, weil sie kein Interesse an ihrer eigenen Freiheit nehmen.\footnote{\cite[][A~29-30]{Kant:DerStreitderFakultaeten1977}, \cite[VII:
29.34--30.9]{Kant:GesammelteWerke1900ff.}: \enquote{Nun wird der Streit der
Fakultäten um den Einfluß aufs Volk geführt, und diesen Einfluß können sie nur bekommen, so fern jede derselben das Volk glauben machen kann, daß sie das Heil desselben am besten zu befördern verstehe, dabei aber doch in der Art, wie sie dieses auszurichten gedenken,
  einander gerade entgegengesetzt sind.\\
  Das Volk aber setzt sein Heil zu oberst nicht in der Freiheit, sondern in
  seinen natürlichen Zwecken, also in diesen drei Stücken: nach dem Tode selig,
  im Leben unter andern Mitmenschen des Seinen, durch öffentliche Gesetze
  gesichert, endlich des physischen Genusses des Lebens an sich selbst (d.\,i.
  der Gesundheit und langen Lebens) gewärtig zu
  sein.}}

Im Falle der Religion stellt die untere Fakultät dem biblischen Theologen,
insofern dieser \enquote{von dem verschrienen Freiheitsgeist der Vernunft und
Philosophie noch nicht angesteckt
ist}\footnote{\cite[][A~18]{Kant:DerStreitderFakultaeten1977}, \cite[VII:
24.26--27]{Kant:GesammelteWerke1900ff.}.} und sich nicht auf Vernunft, sondern
auf Offenbarung beruft, eine Vernunftreligion
entgegen.\footnote{\cite[Vgl.][A~15-18]{Kant:DerStreitderFakultaeten1977},
\cite[VII: 23.9--24.30]{Kant:GesammelteWerke1900ff.}.} Nun betrachtet er
die Auseinandersetzung zwischen Vernunft und Offenbarung nur als einen
Teilbereich in einer Auseinandersetzung der Vernunft mit Vertretern der
oberen Fakultäten. Es gibt drei obere Fakultäten und somit kann nicht nur
die Theologie mit einer Vernunftreligion in Konflikt geraten, sondern auch
Jurisprudenz und Medizin geraten in einen Widerstreit mit der unteren Fakultät.
Also findet auch hier eine Auseinandersetzung um Aufklärung und Mündigkeit
statt. \name[Immanuel]{Kant} stellt in der Aufklärungsschrift aber nicht
Religion, Gesundheit und Recht in den Mittelpunkt, sondern ausschließlich die Religion,
wofür es also einen weiteren Grund geben muss. Dieser lautet, dass Religion eine
besondere Bedeutung für uns Menschen habe: Darin sei Unmündigkeit \enquote{so
wie die schädlichste, also auch die entehrendste unter
allen}\footnote{\cite[Vgl.][A~492]{Kant:BeantwortungderFrage:WasistAufklaerung?1977},
\cite[VIII: 41.14--15]{Kant:GesammelteWerke1900ff.}.}. Aber was ist es, was
Religion zu einem so zentralen Thema macht? Warum nennt er Unmündigkeit in
Fragen der Religion als besonders schädlich und nicht Unmündigkeit bezüglich
naturwissenschaftlicher, historischer, rechtlicher, politischer oder
medizinischer Ansichten? Der Verweis auf die Aufmerksamkeit der Herrschenden
stellt keine befriedigende Antwort dar, weil sie selbst der Begründung bedürfte:
Warum ist es für die Herrschenden so interessant, gerade die öffentliche Meinung
zu Religionsfragen zu regulieren?

Nach \name[Immanuel]{Kant} gibt es (auf Seiten der unteren Fakultät) eine
natürliche Hierarchie von Erkenntnissen, wobei an der Spitze zunächst nicht die
Religion, sondern die Philosophie (nach ihrem Weltbegriff) steht:
\begin{quote}
  Der Mathematiker, der Naturkündiger, der Logiker sind, so vortrefflich die
  ersteren auch überhaupt im Vernunfterkenntnisse, die zweiten besonders im
  philosophischen Erkenntnisse Fortgang haben mögen, doch nur Vernunftkünstler. Es
  gibt noch einen Lehrer im Ideal, der alle diese ansetzt, sie als Werkzeuge
  nutzt, um die wesentlichen Zwecke der menschlichen Vernunft zu befördern. Diesen
  allein müßten wir den Philosophen
  nennen\dots\footnote{\cite[][B~867]{Kant:KritikderreinenVernunft2003}, \cite[][III:
  542.33--543.2]{Kant:GesammelteWerke1900ff.}.}
\end{quote}
Der \enquote{Vernunftkünstler} betreibt Philosophie bloß nach ihrem
Schulbegriff, wonach \enquote{sie nur als eine von den Geschicklichkeiten zu
gewissen beliebigen Zwecken angesehen
wird.}\footnote{\cite[B~867]{Kant:KritikderreinenVernunft2003},
\cite[III: 543.33--34]{Kant:GesammelteWerke1900ff.}.} In diesem Sinne gibt es
eine mit wissenschaftlichem Anspruch auftretende Philosophie, die sich anhand ihres
\distanz{Handwerkszeugs} erkennen lässt. Die logische Strenge der
mathematischen Methode oder heutzutage die methodischen Maßstäbe, welche die Analytische
Philosophie sich selbst zugute hält, sind beispielsweise deutliche Erkennungsmarken einer
solchen Philosophie nach dem Schulbegriffe. Nach dem Schulbegriff sei
\enquote{Philosophie {\punkt} das System der philosophischen Erkenntnisse oder
der Vernunfterkenntnisse aus Begriffen.}\footnote{\cite[][A
23]{Kant:ImmanuelKantsLogik1977}, \cite[][IX:
23.30--31]{Kant:GesammelteWerke1900ff.}.} Dies beschreibt die Art von
Erkenntnissen, die \name[Immanuel]{Kant} ihrer Form nach -- als weder empirische
(sondern Vernunft-) Erkenntnisse, noch solche aus der Konstruktion von Begriffen
(wie in der Mathematik) --  als philosophisch ansieht. Ich werde auf diesen
Schulbegriff, insofern er die Philosophie als System der \emph{philosophischen}
oder \emph{Vernunft}erkenntnisse charakterisiert, in Kapitel
\ref{subsection:BewertungvonInformationennachihrerART} im Rahmen einer
Diskussion testimonialen Wissens zurückkommen. Zunächst ist der -- für
\name[Immanuel]{Kant} wichtigere -- \enquote{Weltbegriff} zu erläutern.
\name[Immanuel]{Kant} schreibt: \enquote{Nach dem \ori{Weltbegriffe} ist
sie [d.\,i. die Philosophie; A.\,G.] die Wissenschaft von den letzten Zwecken
der menschlichen Vernunft.}\footnote{\cite[][A~23]{Kant:ImmanuelKantsLogik1977}, \cite[][IX:
23.32--33]{Kant:GesammelteWerke1900ff.}. \cite[Siehe
auch][B~866]{Kant:KritikderreinenVernunft2003}, \cite[III:
542.3--24]{Kant:GesammelteWerke1900ff.}.} Ihrem Weltbegriffe nach ist sie keine
kontingent betriebene Wissenschaft, auf die wir auch verzichten könnten, sondern
notwendiges Thema der Vernunft. Das Wissen um die Bestimmung des Menschen und
die Beziehung aller Erkenntnisse auf die \enquote{notwendigen und wesentlichen
Zwecke der Menschheit} nennt \name[Immanuel]{Kant} dann auch
\enquote{Weisheit}.\footnote{\cite[][B~878]{Kant:KritikderreinenVernunft2003},
\cite[][III: 549.13--24]{Kant:GesammelteWerke1900ff.}. Zur Rolle des Begriffs
der \enquote{Weisheit} im Rahmen der \index{Kant, Immanuel}kantischen Aufklärungsphilosophie
siehe \cite{Trawny:DasIdealdesWeisen2008}.}

Was sind nun die \enquote{wesentlichen Zwecke}, um die es der Philosophie ihrem
Weltbegriffe nach geht? \name[Immanuel]{Kant} nennt den höchsten Zweck allen
Vernunftgebrauchs und damit auch dasjenige, was im Zentrum der Aufklärung steht, den Endzweck der Vernunft, der
nur ein einziger sein könne und auf den sich alle anderen Zwecke der Vernunft,
als solche, bezögen. Er ist also das Ziel letztlich \emph{aller} Wissenschaft. Und zwar
handle es sich um \enquote{die ganze Bestimmung des
Menschen}.\footnote{\cite[][B~868]{Kant:KritikderreinenVernunft2003}, \cite[][III:
543.11]{Kant:GesammelteWerke1900ff.}.
Dies spiegelt sich auch in \name[Immanuel]{Kant}s Zusammenfassung der
Philosophie zu der Frage \enquote{Was ist der Mensch?} wieder
\mkbibparens{\cite[vgl.][A 25]{Kant:ImmanuelKantsLogik1977}, \cite[][IX:
25.3--6]{Kant:GesammelteWerke1900ff.}}, die an der Parallelstelle in der
\titel{Kritik der reinen Vernunft}
\mkbibparens{\cite[vgl.][B 833]{Kant:KritikderreinenVernunft2003},
\cite[][III: 522.32--34]{Kant:GesammelteWerke1900ff.}} jedoch nicht vorkommt.
Norbert \name[Norbert]{Hinske} zufolge handelt es sich bei der Bestimmung des Menschen
um eine der \enquote{Basisideen} der Aufklärung
\parencite[vgl.][398]{Hinske:ArtikelAufklaerung1985}.}
\name[Immanuel]{Kant} schreibt in der \titel{Kritik der reinen Vernunft}:
\begin{quote}\label{Zitat:EndzweckalsganzeBestimmungdesMenschen}
  Wesentliche Zwecke sind darum noch nicht die höchsten, deren (bei
  vollkommener systematischer Einheit der Vernunft) nur ein einziger sein kann.
  Daher sind sie entweder der Endzweck, oder subalterne Zwecke, die zu jenem als
  Mittel notwendig gehören. Der erstere ist kein anderer, als die ganze
  Bestimmung des Menschen, und die Philosophie über dieselbe heißt
  Moral.\footnote{\phantomsection\label{Anmerkung:GanzeBestimmung}\cite[B~868]{Kant:KritikderreinenVernunft2003},
  \cite[III: 543.7--12]{Kant:GesammelteWerke1900ff.}.}
\end{quote}
Der Weltbegriff der Philosophie konstituiert eine Hierarchie unter den
philosophischen Wissenschaften, an deren Spitze die Moralphilosophie steht.
Auch bei unserem theoretischen Vernunftgebrauch, der selbst eine Form
zielgerichteten Handelns darstellt, gibt es vielfältige Verbindungen von Zwecken
und Mitteln. Menschen lösen Rechenaufgaben, um Prüfungen zu bestehen, die
Entfernung eines Gewitters oder die Tiefe eines Brunnens zu berechnen. Und oft
dienen unsere Erkenntnisse der Gewinnung weiterer Erkenntnisse, beispielsweise
wenn ein mathematischer Hilfssatz bewiesen wird, der später in den Beweis eines
wichtigen Theorems eingehen soll. \name[Immanuel]{Kant} spricht hier von
\singlequote{\emph{formaler} objektiver
Zweckmäßigkeit}.\footnote{\cite[Vgl.][\S~62]{Kant:KritikderUrteilskraft2009},
\cite[V: 362.6--364.2]{Kant:GesammelteWerke1900ff.}.} Je nachdem, welchen
Handlungen oder welchen weiteren Erkenntnisgewinnen eine Erkenntnis dienen kann,
nennen wir sie \enquote{nützlich} oder auch \enquote{unnütz}. Wir haben gesehen,
dass \name[Immanuel]{Kant} wissenschaftliche Erkenntnisse im Anschluss an
\name[Jean-Jacques]{Rousseau} nicht als Selbstzweck ansieht, sondern sie in den
Dienst der \enquote{Rechte der Menschheit} gestellt wissen
möchte.\footnote{Siehe oben, Anm.
\ref{Anmerkung:Rousseauhatmichzurechtgebracht} auf S.
\pageref{Anmerkung:Rousseauhatmichzurechtgebracht}.} Als \enquote{Endzweck}
bezeichnet \name[Immanuel]{Kant} einen \enquote{Zweck, der keines anderen als
Bedingung seiner Möglichkeit
bedarf}\footnote{\cite[\S~84]{Kant:KritikderUrteilskraft2009}, \cite[V:
434.7--8]{Kant:GesammelteWerke1900ff.}.}, und unterscheidet ihn von letzten Zwecken.

Man kann das Verhältnis der verschiedenen Zweckbegriffe folgendermaßen
illustrieren: Wenn mich jemand fragt, warum ich Mehl siebe, könnte ich
antworten, dass ich einen Teig herstelle. Das Sieben des Mehls ist dann das
Mittel zu einem Zweck, der Herstellung des Teigs. Wenn es nun möglich ist,
weiter zu fragen, warum ich einen Zweck verfolge, dann verfolge ich keinen
Endzweck. Es könnte beispielsweise jemand fragen, warum ich einen Teig herstelle
-- Teigherstellung ist kein Endzweck. Wenn ich auf eine solche Frage antworte,
gebe ich einen höheren Zweck an. Ich sage zum Beispiel: \enquote{Ich backe einen
Obstkuchen.} Und damit gebe ich einen Zweck an, von dem der niedere
Zweck -- die Teigherstellung -- abhängig ist; denn wollte ich keinen Obstkuchen
backen, bräuchte ich auch keinen Teig herstellen. Wenn ich nun einen Zweck
nennen kann, der selbst nicht mehr Mittel zu einem anderen Zweck ist,
so finde ich einen \emph{letzten Zweck}. Der Wunsch, Obstkuchen zu essen, mag
ein solcher letzter Zweck sein. Aber es drückt doch keinen Endzweck aus, denn er
ist davon abhängig, dass ich Obstkuchen essen möchte und mir diesen Zweck
\emph{setze}. Zu einer anderen Zeit habe ich diesen Zweck nicht und es ist auch
gerade jetzt zufällig, dass ich diesen Zweck verfolge. Im Falle eines Endzwecks
muss es sich von selbst verstehen, dass es sich um einen Zweck handelt, den
anzustreben richtig oder gut ist; ein Endzweck ist daher -- wie
\name[Immanuel]{Kant} sagt -- \enquote{in der Ordnung der Zwecke von keiner
anderweitigen Bedingung als bloß seiner Idee
abhängig}\footnote{\cite[\S~84]{Kant:KritikderUrteilskraft2009}, \cite[V:
435.13--14]{Kant:GesammelteWerke1900ff.}.}.


Wir kennen nur einen einzigen Endzweck, so behauptet
\name[Immanuel]{Kant} in der \titel{Kritik der Urteilskraft}: den Menschen als
\emph{Noumenon}. Denn nur von diesem -- dem Menschen als eines moralischen
Wesens -- könne nicht weiter gefragt werden, wozu er da
sei.\footnote{\cite[Vgl.][\S~84]{Kant:KritikderUrteilskraft2009}, \cite[V:
435.15--436.2]{Kant:GesammelteWerke1900ff.}.} Wenn ich auf die Frage nach dem
Grund meines Handelns sage: \enquote{Ich helfe jemandem, der in Not geraten
ist!}, so lässt sich nicht sinnvoll weiterfragen, warum ich das tue.
Jedenfalls könnte (dürfte) ich eine solche Frage als unsinnig zurückweisen und
sagen: Wir brauchen keinen weiteren Grund, um so zu handeln; es handelt sich um
einen Imperativ, der \emph{kategorisch} gebietet und nicht von weiteren Zwecken
abhängig ist. Was Pflicht ist, das ist nach \name[Immanuel]{Kant} ein
Endzweck.\footnote{Zumindest gilt dies dann, wenn derjenige, dem ich helfe,
nicht durch eigenes unmoralisches Verhalten in eine Notsituation geriet, wenn ich dadurch also nicht
sein unmoralisches Verhalten billige und fördere. Helfe ich hingegen einem
rechtmäßig verurteilten Mörder bei der Flucht, so ist die Frage nach einem Warum
weiterhin angebracht. Mein Zweck ist dann aber auch nicht ein Mensch als
\emph{moralisches} Wesen, denn seine Notsituation (die Haft) war gerade
moralisch geboten. Dies ist damit bezeichnet, dass der Mensch als
\emph{Noumenon} Endzweck der Vernunft sei; denn der Ausdruck \enquote{Mensch
als \emph{Noumenon}} bezeichnet ihn insbesondere als moralisches Wesen.
Und auch die \singlequote{Selbstzweck-Formel} des Kategorischen Imperativs betont dies, wenn
es heißt, man solle \singlequote{die Menschheit} in der Person eines jeden (und
nicht etwa einfach jeden Menschen) immer auch als Zweck betrachten
\mkbibparens{\cite[vgl.][BA
66\,f.,]{Kant:GrundlegungzurMetaphysikderSitten1965} \cite[][IV: 429.10--12]{Kant:GesammelteWerke1900ff.}. Trotz dieser ersten
Deutung dieser Formulierung ist diese Redeweise \name[Immanuel]{Kant}s
problematisch, insofern er keine Erläuterung des Ausdrucks \enquote{Menschheit
in der Person von\dots} angibt.}} So bestimmt
die Moralphilosophie den Endzweck unseres Handelns und wird von
\name[Immanuel]{Kant} als die Wissenschaft angeführt, die die Bestimmung des
Menschen untersucht.\footnote{Siehe dazu die Anm.
\ref{Anmerkung:GanzeBestimmung}, S.
\pageref{Anmerkung:GanzeBestimmung}.}

Wird Philosophie nicht nach dem Schul-, sondern nach ihrem Weltbegriff
verstanden, so bezieht sie unsere Erkenntnisse auf \distanz{wesentliche} Zwecke
unserer Vernunft -- das ist die Forderung des Weltbegriffs der Philosophie. Was
aber sind wesentliche Zwecke? Wesentliche Zwecke unserer Vernunft sind erstens
der Endzweck der Vernunft und zweitens diejenigen niederen Zwecke, die als
Mittel zur Umsetzung des Endzwecks unentbehrlich
sind.\footnote{\phantomsection\label{Anmerkung:wesentlicheZwecke}\cite[Vgl.][B
868]{Kant:KritikderreinenVernunft2003}, \cite[III:
543.7--10]{Kant:GesammelteWerke1900ff.}: \enquote{Wesentliche Zwecke sind
\punkt{} entweder der Endzweck, oder subalterne Zwecke, die zu jenem als Mittel
notwendig gehören.}} \name[Immanuel]{Kant} setzt voraus, dass die Vernunft zu vollkommener
systematischer Einheit fähig ist.\footnote{Siehe auch
\cite[A~xiii]{Kant:KritikderreinenVernunft2003}, \cite[IV:
10.11--16]{Kant:GesammelteWerke1900ff.}: \enquote{In der Tat ist auch reine
Vernunft eine so vollkommene Einheit: daß, wenn das Prinzip derselben auch nur
zu einer einzigen aller der Fragen, die ihr durch ihre eigene Natur aufgegeben
sind, unzureichend wäre, man dieses immerhin nur wegwerfen könnte, weil es
alsdenn auch keiner der übrigen mit völliger Zuverlässigkeit gewachsen sein
würde.}} Daher kann er fordern, dass ein \emph{einziger} höchster Zweck
angegeben werde. Dieser höchste Zweck sei die Bestimmung des Menschen und
die Wissenschaft, die ihn erforscht, sei die Moralphilosophie. Die Bestimmung
des Menschen muss einen Zweck anzeigen, der von keiner Bedingung abhängig ist
und sich von selbst versteht. Sie ist erstens nicht weiter begründungsbedürftig und stellt zweitens den Dreh- und
Angelpunkt aller unserer Erkenntnisse dar. Von ihrer Beziehung auf die
Bestimmung des Menschen erhalten diese erst ihren je eigenen inneren
Wert.\footnote{\cite[Vgl.][A 23]{Kant:ImmanuelKantsLogik1977}, \cite[IX:
23.33--24.2]{Kant:GesammelteWerke1900ff.}. Von dem Weltbegriffe aus habe die
Philosophie sogar einen absoluten Wert, d.\,i.\ Würde. Daneben können
Wissenschaften (nach ihrem Schulbegriff) natürlich auch einen \distanz{äußeren},
monetären \emph{Preis} haben. Nach dem Preis bewertet, liegt die
Moralphilosophie, der der höchste innere Wert zukommt, freilich weit abgeschlagen. 
Siehe auch die Parallele in \cite[][BA~77]{Kant:GrundlegungzurMetaphysikderSitten1965},
\cite[][IV: 434.31--435.4]{Kant:GesammelteWerke1900ff.}}
Wenn nun aber die Moral die Wissenschaft vom Endzweck der menschlichen
Vernunft ist, warum verweist \name[Immanuel]{Kant} dann auf die Bestimmung des Menschen
und nicht auf die Pflichten des Menschen?

\subsection{Aufklärung und
Anthropologie}\label{subsection:AufklaerungundAnthropologie} Es handelt sich bei
der Bestimmung des Menschen um ein Thema der Aufklärungstheologie des 18.~Jahrhunderts, was andeutet, dass dies die richtige Fährte ist, die Relevanz
der Religion für \name[Immanuel]{Kant}s Aufklärungsprogrammatik zu klären. Mehrfach wurde
behauptet, dass es sich um einen der zentralen Topoi der zweiten Hälfte der
Aufklärung
handelt.\footnote{\Revision{Zum Begriff der Bestimmung des Menschen siehe
  \cite{Macor:DieBestimmungdesMenschen1748--18002013}, und die dort
  ausführlich zitierte und besprochene
  Literatur.} Siehe außerdem
\cite[][476]{Zoeller:DieBestimmungderBestimmungdesMenschenbeiMendelssohnundKant2001}:
\enquote{Wohl kaum eine Wendung oder Formel dürfte so geeignet sein, das
philosophische Projekt der deutschen Spätaufklärung in der zweiten Hälfte des
18. Jahrhunderts zu bezeichnen wie die von der \enquote{Bestimmung des
Menschen}.} Auch Reinhard \name[Reinhard]{Brandt} sieht in
\authorcite{Spalding:BetrachtungueberdieBestimmungdesMenschen1749}s Buch
\enquote{die Programmschrift der zweiten Phase der deutschen Aufklärung}
\parencite[][61]{Brandt:DieBestimmungdesMenschenbeiKant2007}. Zur
Wirkungsgeschichte des Begriffs \enquote{Bestimmung des Menschen} siehe
\cite{DAlessandro:DieWiederkehreinesLeitworts1999}.
Vgl.\ außerdem \cite{Hinske:EineantikeKatechismusfrage1999},
\cite[][179--211]{Tippmann:DieBestimmungdesMenschenbeiJohannJoachimSpalding2011},
und allgemein zur Vorstellung einer Bestimmung des Menschen und ihrer
Erkennbarkeit mittels natürlicher Vernunft die Textsammlung in
\cite[][47--103]{Ciafardone:LIlluminismotedesco1983} (Kapitel über
\enquote{l'uomo e la sua destinazione}).
\Revision{\textcite{Macor:DieBestimmungdesMenschen1748--18002013}
  bespricht \name[Immanuel]{Kant}s Behandlung dieser Thematik und ihre
  Auswirkungen auf den Seiten 199--310.} Zur Wirkung
\authorcite{Spalding:BetrachtungueberdieBestimmungdesMenschen1749}s
auf \name[Immanuel]{Kant} siehe außerdem
 \cite[][189--198]{Tippmann:DieBestimmungdesMenschenbeiJohannJoachimSpalding2011},
 und die dort angeführte Literatur sowie
 \cite[][7--138]{Brandt:DieBestimmungdesMenschenbeiKant2007}.}
Mit der Bestimmung des Menschen als Thema der Moral und Endzweck der Vernunft
gibt \name[Immanuel]{Kant} aussagekräftige Auskunft über seine Selbstverortung innerhalb
der Aufklärung: Sie reflektiert die dominante Stellung der Moral
gegenüber der Religion und die \index{Kant, Immanuel}kantische Auffassung des
Verhältnisses von Vernunft und Glaube.


Der Ausdruck \enquote{Bestimmung des Menschen} war im 18.\ Jahrhundert bekannt
durch eine Schrift des Aufklärungstheologen und Predigers Johann Joachim
\authorcite{Spalding:BetrachtungueberdieBestimmungdesMenschen1749}, dessen auf
1748 datierte\footnote{Bei der auf dem Titelblatt angegebenen Jahreszahl handelt
es sich wohl um eine Vordatierung.
\cite[Vgl.][\pno~17,
Fn.~50]{Schwaiger:ZurFragenachdenQuellenvonSpaldingsemphBestimmungdesMenschen1999}.}
\titel{Betrachtung über die Bestimmung des Menschen} etliche Auflagen erlebte
und bis in den Deutschen Idealismus hinein breit rezipiert
wurde.\footnote{\Revision{Damit soll nicht gesagt sein, dass
    \authorcite{Spalding:BetrachtungueberdieBestimmungdesMenschen1749}
    den Begriff neu in die philosophische Sprache eingeführt habe, er
    hat aber sicherlich die bekannteste Schrift publiziert, welche die
    Bestimmung in ihrem Titel trägt und zum Aushängeschild dieser
    Thematik wurde. Siehe hierzu insbesondere die beeindruckende Studie
  von \authorfullcite{Macor:DieBestimmungdesMenschen1748--18002013}, die insgesamt 40 Auflagen von Spaldings Schrift
  zählt \parencite[vgl.][33]{Macor:DieBestimmungdesMenschen1748--18002013}
  und die Ursprünge des Begriffs der Bestimmung des Menschen bis in
  das 16. Jahrhundert
  zurückverfolgt \parencite[vgl.][36--109]{Macor:DieBestimmungdesMenschen1748--18002013}. \enquote{Weit
  davon entfernt, eine Neuschöpfung des 18. Jahrhunderts zu sein, läßt
sich das Wort \singlequote{Bestimmung} schon im 16. Jahrhundert und in
einigen Fällen bereits im letzten Jahrzehnt des 15. Jahrhunderts
verzeichnen} \parencite[vgl.][37]{Macor:DieBestimmungdesMenschen1748--18002013}.}
\cite[Für
  das Hineinreichen der Wirkungsgeschichte in die Philosophie des
  Deutschen
Idealismus steht][]{Fichte:DieBestimmungdesMenschen1800}. Siehe hierzu auch
\cite{Zoeller:BestimmungzurSelbstbestimmung1995} sowie
\cite[][477--482]{Zoeller:DieBestimmungderBestimmungdesMenschenbeiMendelssohnundKant2001}.}
\authorcite{Spalding:BetrachtungueberdieBestimmungdesMenschen1749} versucht in
dieser Schrift, eine Erkenntnis dessen, was vernünftige Ziele menschlichen
Lebens ist, ohne Berufung auf Offenbarungswissen oder andere übernatürliche
Erkenntnisquellen zu
etablieren.\footnote{\cite[Vgl.][3]{Spalding:BetrachtungueberdieBestimmungdesMenschen1749}.
Die Unabhängigkeit seiner Argumentation von Offenbarung zwingt ihn in der
dritten Auflage dazu, in einem Anhang den Vorwurf des Deismus abzuwehren, der
die natürliche Theologie gegen das offenbarte Christentum in Stellung bringe,
\cite[vgl.][26--32]{Spalding:BetrachtungueberdieBestimmungdesMenschen1749}.
Zu den Grundlagen und Ursprüngen von \authorcite{Spalding:BetrachtungueberdieBestimmungdesMenschen1749}s Abhandlung vgl.
\cite{Schwaiger:ZurFragenachdenQuellenvonSpaldingsemphBestimmungdesMenschen1999},
wo \authorcite{Spalding:BetrachtungueberdieBestimmungdesMenschen1749} als
methodischer Anhänger \authorcite{Wolff:Psychologiaempirica1968}s rekonstruiert wird, der den
Anstoß zu seinem \enquote{Appell an das Selbstdenken in Sachen Bestimmung des
Menschen}
\parencite[][13]{Schwaiger:ZurFragenachdenQuellenvonSpaldingsemphBestimmungdesMenschen1999}
und insbesondere das Bekenntnis zu der Lehre vom moralischen Gefühl
\parencite[vgl.][15]{Schwaiger:ZurFragenachdenQuellenvonSpaldingsemphBestimmungdesMenschen1999}
der Lektüre \name[Anthony Ashley-Cooper]{Shaftesbury}s entnimmt.} So schreibt er
zu Beginn des Buches:
\begin{quote}
  Ich sehe, daß ich die kurze Zeit, die ich auf der Welt zu leben habe, nach
  ganz verschiedenen Grundregeln zubringen kann, deren Wehrt und Folgen daher
  auch unmöglich einerley seyn können. Da ich nun unläugbar eine Fähigkeit zu
  wählen, und in meinen Entschliessungen eines dem andern vorzuziehen an mir
  finde, so muß ich auch hiebey nicht blindlings zufahren, sondern vorher nach
  meinem letzten Vermögen auszumachen suchen, welcher Weg der sicherste,
  anständigste und vortheilhafteste sey. \punkt{} Es ist doch einmal der Mühe
  wehrt, zu wissen, warum ich da bin, und was ich vernünftiger Weise seyn
  soll.\footcite[][4]{Spalding:BetrachtungueberdieBestimmungdesMenschen1749}
\end{quote}
Der Titel \enquote{Bestimmung des Menschen} verweist also zunächst auf die Sorge
um das gute, \distanz{glückende} Leben, wie sie genuiner Bestandteil der
Religion ist.\footnote{\Revision{Gewiss eröffnet
\authorcite{Spalding:BetrachtungueberdieBestimmungdesMenschen1749} kein
neues Thema, sondern greift eines auf, welches schon bei früheren
Autoren unter diesem und anderen Titeln
behandelt wird \parencite[siehe dazu
ausführlich][36--139]{Macor:DieBestimmungdesMenschen1748--18002013}.} \textcite[vgl.][47--103]{Ciafardone:LIlluminismotedesco1983}
lässt in seinem Kapitel über \enquote{l'uomo e la sua destinazione} nicht
\authorcite{Spalding:BetrachtungueberdieBestimmungdesMenschen1749} selbst zu
Wort kommen, sondern hebt die Präsenz des Themas bei \name[Gottfried
Wilhelm]{Leibniz}, \name[Christian]{Thomasius}, \authorcite{Wolff:Psychologiaempirica1968},
\authorcite{Crusius:Anweisungvernuenftigzuleben1744}, \name[Moses]{Mendelssohn}
\name[Immanuel]{Kant}, \authorcite{Lessing:EineDuplik1897} und
\name[Friedrich]{Schiller} hervor.} Seine Abhandlung ist dabei aber stets auch
ein \enquote{Appell an das Selbstdenken in Sachen Bestimmung des
Menschen}\footnote{\cite[][13]{Schwaiger:ZurFragenachdenQuellenvonSpaldingsemphBestimmungdesMenschen1999}.}
und damit ein Aufruf, der jedem Menschen eigenen Freiheit in der
Lebensgestaltung aufgrund jeweils eigener Einsicht gerecht zu werden. Und damit
nähern wir uns erkennbar einer Antwort auf die Frage nach dem Grund der Relevanz
der Religion als Thema der Aufklärung: Mündig zu sein bedeutet primär,
Verantwortung für die eigene Lebensgestaltung zu übernehmen.
 
 
\name[Immanuel]{Kant} schreibt keinen eigenständigen Text über die Bestimmung des
Menschen, verwendet diesen Ausdruck aber an etlichen Stellen. Häufig geschieht dies eher
beiläufig, wie an der zitierten Stelle in der \titel{Kritik der reinen
Vernunft}, wo sich weder eine Erläuterung des Begriffs findet noch die Verbindung zur
Moralphilosophie verdeutlicht
wird.\Revision{\footnote{\Revision{\textcite[vgl.][31]{Macor:DieBestimmungdesMenschen1748--18002013}
      behauptet (allerdings ohne eine Begründung zu liefern), dass
      alle drei Kritiken \name[Immanuel]{Kant}s die Bestimmung des 
      Menschen zu ihrem (im Titel allerdings nicht 
  genannten) Thema hätten.}}} Etwas größeren Raum nehmen Äußerungen über
die \enquote{Bestimmung des Menschen} oder die \enquote{Bestimmung des
Menschengeschlechts}\footnote{\cite[][B
330]{Kant:AnthropologieinpragmatischerHinsicht1977}, \cite[][VII:
331.28]{Kant:GesammelteWerke1900ff.}. Darin kommt zum Ausdruck, dass eine Bestimmung des Menschen eine Bestimmung ist, die ihm \emph{als Mensch} zukommt
(und nicht als dieses oder jenes Individuum). Dass \name[Immanuel]{Kant} in
gleicher Bedeutung von einer Bestimmung \emph{des Menschen} und einer Bestimmung
\emph{des Menschengeschlechts} sprechen kann, liegt aber sicherlich auch in
seiner Überzeugung begründet, dass der Mensch seiner Bestimmung nur als Gattung gerecht
werden kann; \cite[vgl.][A
388]{Kant:IdeezueinerallgemeinenGeschichteinweltbuergerlicherAbsicht1977},
\cite[][VIII: 18.29--32]{Kant:GesammelteWerke1900ff.}. In der \titel{Pädagogik}
wiederum wird die Menschheit selbst als Bestimmung des Menschen genannt;
\cite[vgl.][A 3]{Kant:UeberPaedagogik1977}, \cite[][IX:
442.3--4]{Kant:GesammelteWerke1900ff.}.} im zweiten Teil der
\titel{Anthropologie in pragmatischer Hinsicht} ein, wo es um den
\enquote{Charakter} des Menschen geht. Der Charakter eines Lebewesens sei
\enquote{das, woraus sich seine Bestimmung zum voraus erkennen
läßt.}\footnote{\cite[B~326]{Kant:AnthropologieinpragmatischerHinsicht1977},
\cite[VII: 329.14--15]{Kant:GesammelteWerke1900ff.}.} Somit geht es der
\enquote{Charakteristik} -- so heißt der zweite Teil der \titel{Anthropologie}
-- wenigstens indirekt um die Bestimmung des Menschen (und zwar als Person,
Geschlecht, Volk und
Gattung)\footnote{\cite[Vgl.][B~253]{Kant:AnthropologieinpragmatischerHinsicht1977},
\cite[VII: 285.1--3]{Kant:GesammelteWerke1900ff.}.}, was eine
philosophie-historisch interessante Entwicklung anspricht:
Im Laufe des 18.\ Jahrhunderts entwickelt sich aus dem Topos einer Bestimmung
des Menschen die philosophische Disziplin der
Anthropologie\footnote{\cite[Vgl.][]{DAlessandro:DieWiederkehreinesLeitworts1999},
sowie
\cite[][138--140]{Tippmann:DieBestimmungdesMenschenbeiJohannJoachimSpalding2011}.}.
Genauer müsste man sagen:
\authorcite{Spalding:BetrachtungueberdieBestimmungdesMenschen1749}s Abhandlung
ist eine von zwei wichtigen Quellen der neu entstehenden Anthropologie; die
andere Quelle ist die mit \authorcite{Wolff:Psychologiaempirica1968} anhebende
empirische Psychologie.\footnote{Siehe dazu
\cite{Bae:DieEntstehungderKantischenAnthropologieundihreBeziehungzurempirischenPsychologiederWolffschenSchule1994},
sowie
\cite{Hinske:WolffsempirischePsychologieundKantspragmatischeAnthropologie1999}.}

\name[Immanuel]{Kant} hielt beginnend mit dem Wintersemester 1772/1773 in jedem
Winterhalbjahr eine Vorlesung über Anthropologie, die sich an einen breiteren
Hörerkreis wendete. Wie im Falle der physischen Geographie handelt es sich um
Wissen, das jeden angehe, was darauf hindeutet, dass es aufklärungsrelevant ist.
Aus diesen Vorlesungen ging dann Ende der 1790er Jahre die Schrift
\titel{Anthropologie in pragmatischer Hinsicht} hervor.\footnote{Zur Entwicklung
der Anthropologie bei \name[Immanuel]{Kant} siehe
\cite{Stark:HistoricalNotesandInterpretiveQuestionsaboutKantsLecturesonAnthropology2003}.
\name[Immanuel]{Kant}s Anthropologie ist in den letzten Jahren immer st{"a}rker
zum Thema von Publikationen geworden, siehe etwa
\cite{Zoeller:DieBestimmungderBestimmungdesMenschenbeiMendelssohnundKant2001},
\cite{Zammito:KantHerderandtheBirthofAnthropology2002},
\cite{Brandt:TheGuidingIdeaofKantsAnthropologyandtheVocationoftheHumanBeing2003},
\cite{Wood:KantandtheProblemofHumanNature2003},
\cite{Wilson:KantsPragmaticAnthropology2006},
\cite{Cohen:KantandtheHumanSciences2009}, sowie
\cite{Sturm:KantunddieWissenschaftenvomMenschen2009}.}
Ursprünglich basierten diese Vorlesungen über Anthropologie auf den Abschnitten
über empirische Psychologie in
\authorcite{Baumgarten:Metaphysica---Metaphysik2011}s \titel{Metaphysica}.
Dennoch ist \name[Immanuel]{Kant}s Anthropologie nicht einfach eine Etappe in
der Entwicklung der empirischen Psychologie. Sie schließt zwar an diese an, aber
sie wandelt deren Erkenntnisinteresse merklich ab, gerade indem sie ihren
Ursprung in der empirischen Psychologie mit Fragestellungen bezüglich der
Bestimmung des Menschen
verbindet.\footcite[Vgl.][]{Brandt:TheGuidingIdeaofKantsAnthropologyandtheVocationoftheHumanBeing2003}
Und gerade dies macht die Aufklärungsrelevanz solchen Wissens aus. Eine solche
Verschmelzung ursprünglich getrennter Disziplinen lässt zunächst die Frage
aufkommen, was das Thema der Anthropologie ist, wenn sie überhaupt ein
einheitliches Thema hat und kein bloßes Konglomerat -- oder wie
\name[Immanuel]{Kant} sagen würde: Aggregat -- von Kenntnissen ist. Lautet die
leitende Frage \enquote{Was ist der Mensch?}, wie es in der Jäsche-Logik
heißt?\footnote{\cite[Vgl.][A~25]{Kant:ImmanuelKantsLogik1977},
\cite[IX: 25.1-10]{Kant:GesammelteWerke1900ff.}: \enquote{Das Feld der
Philosophie in dieser weltbürgerlichen Bedeutung läßt sich auf folgende Fragen bringen:
  \begin{nummerierung}
\item Was kann ich wissen?
\item Was soll ich thun?
\item Was darf ich hoffen?
\item Was ist der Mensch?
  \end{nummerierung}
  Die erste Frage beantwortet die Metaphysik, die zweite die Moral,
  die dritte die Religion und die vierte die Anthropologie. Im
  Grunde könnte man aber alles dieses zur Anthropologie rechnen, weil sich
  die drei ersten Fragen auf die letzte beziehen.} Dieser Deutung schließt sich
  z.\,B.\ \textcite{Alpheus:WasistderMensch1968} an.} Reinhardt \name[Reinhard]{Brandt}
  weist den Vorschlag, die Frage \enquote{Was ist der Mensch?} als Leitfrage der
  Anthropologie in pragmatischer Hinsicht anzusehen, explizit zurück und nennt
  stattdessen die \emph{Bestimmung} des Menschen als deren eigentliches
  Thema.\footcite[Vgl.][86-7]{Brandt:TheGuidingIdeaofKantsAnthropologyandtheVocationoftheHumanBeing2003}
  Dagegen nimmt \authorfullcite{Wood:KantandtheProblemofHumanNature2003} an,
  dass es der Anthropologie um die
\emph{Natur} des Menschen
gehe.\footcite[Vgl.][passim]{Wood:KantandtheProblemofHumanNature2003} Geht es
der Anthropologie also um den Menschen, um die Natur des Menschen oder um die
Bestimmung des Menschen? Oder meint dies vielleicht dasselbe? Was ist
\distanz{die Natur} oder \distanz{die Bestimmung} von etwas? Und um welche Art
von Fragen handelt es sich? Sind es empirische Fragen an eine Naturwissenschaft
vom Menschen? Oder handelt es sich um philosophische, vielleicht sogar
\singlequote{metaphysische} Fragen?

Solche Fragen sind auch deshalb von Bedeutung, weil es gute Gründe gibt, einer
philosophischen Disziplin, die sich mit der Natur oder der Bestimmung
des Menschen befasst, skeptisch gegenüber zu stehen. So wirft \authorcite{Wood:KantandtheProblemofHumanNature2003}
die Frage auf, ob wir heute nicht davon ausgehen sollten, dass es eine allgemeine und
bei allen gleiche Natur des Menschen gar nicht gebe (und dass sich dies auch und
gerade aus \name[Immanuel]{Kant}s aufklärerischer Perspektive einsehen
lasse).\footnote{\cite[Vgl.][38--39]{Wood:KantandtheProblemofHumanNature2003}:
\enquote{Kant was reluctant to address the most fundamental question. As we
shall see later, this reluctance anticipates some of the issues (about human
freedom and about the historical variability of human ways of life) that have
led others since Kant's time to declare that there is no such thing as
\enquote{human nature} uniformly and equally determining all human beings at all
times and places.}} Zudem scheint \name[Immanuel]{Kant}s \titel{Anthropologie
in pragmatischer Hinsicht} ein Sammelplatz von sexistischen, ethnischen und
weiteren Vorurteilen zu sein, den als zentrales Element von Aufklärung zu
bezeichnen so manchem Leser Magenschmerzen verursachen
dürfte.\footnote{So wirft er allen Angehörigen nichteuropäischer Nationen in
einer Nebenbemerkung Borniertheit -- \enquote{die Eingeschränktheit
aller übrigen [Völker; A.\,G.] an Geist}
\mkbibparens{\cite[][A 300]{Kant:AnthropologieinpragmatischerHinsicht1977},
\cite[][VII: 312.28--29]{Kant:GesammelteWerke1900ff.}} -- vor und stellt dann
allein Franzosen, Engländer und Deutsche als kosmopolitisch heraus:
\enquote{Die Eingeschränktheit des Geistes aller Völker, welche die
uninteressierte Neubegierde nicht anwandelt, die Außenwelt mit eigenen Augen
kennen zu lernen, noch weniger sich dahin (als Weltbürger) zu verpflanzen, ist
etwas Charakteristisches an denselben, wodurch sich Franzosen, Engländer und
Deutsche vor anderen vorteilhaft unterscheiden}
\mkbibparens{\cite[][A 306]{Kant:AnthropologieinpragmatischerHinsicht1977},
\cite[][VII: 316.33--37]{Kant:GesammelteWerke1900ff.}}. Zum Charakter des
Geschlechts bemerkt er u.\,a.: \enquote{Der Mann ist leicht zu erforschen, die
Frau verrät ihr Geheimnis nicht; obgleich anderer ihres (wegen ihrer
Redseligkeit) schlecht bei ihr verwahrt ist}
\mkbibparens{\cite[][A 285\,f.,]{Kant:AnthropologieinpragmatischerHinsicht1977}
\cite[][VII: 303.35--304.2]{Kant:GesammelteWerke1900ff.}}. Die Reihe an
Beispielen ließe sich fast beliebig fortsetzen.
Vgl.~\cite[271]{Boehme:AnthropologieinpragmatischerHinsicht1985}, sowie
\cite[79]{Louden:TheSecondPartofMorals2003}, und
\cite[173]{Schmidt:KantsTranscendentalEmpiricalPragmaticandMoralAnthropology2007}.}
\name[Immanuel]{Kant}s Behauptung, es läge den einzelnen Nationen eine je
gemeinsame Abstammung und ein erblicher Nationalcharakter
zugrunde,\footnote{\cite[Vgl.][A~297--301]{Kant:AnthropologieinpragmatischerHinsicht1977},
\cite[VII: 311.6--313.16]{Kant:GesammelteWerke1900ff.}.} ist ebenso Ausdruck
vorurteilsbehafteten Denkens wie seine Ausführungen über den jeweiligen
Charakter von Frauen und Männern. Und der Umstand, dass er die Erläuterung des Charakters der verschiedenen
\distanz{Menschenrassen} ausspart, lässt sich mit Gernot \name[Gernot]{Böhme}
wohl nur als großes Glück
bezeichnen.\footnote{\cite[Vgl.][271]{Boehme:AnthropologieinpragmatischerHinsicht1985}.
Auch \authorfullcite{Doerflinger:DieEinheitderMenschheitalsTiergattung2001},
der \name[Immanuel]{Kant}s Rassebegriff in der physischen Anthropologie sehr
aufgeschlossen gegenüber steht, konzidiert, dass \name[Immanuel]{Kant} auch in
großem Umfang Stereotypen und Vorurteile seine Zeit wiederhole
\parencite[vgl.][349]{Doerflinger:DieEinheitderMenschheitalsTiergattung2001}.}
\name[Immanuel]{Kant}s Anthropologie ist darüber hinaus in ihren mutigen
Verallgemeinerungen in weiten Teilen ein methodisch fragwürdiges Unternehmen, das in
der heutigen Wissenschaftslandschaft keinen Platz mehr hätte und von methodisch
disziplinierten empirischen Wissenschaften abgelöst
wurde.\footcite[Vgl.][167-173]{Schmidt:KantsTranscendentalEmpiricalPragmaticandMoralAnthropology2007}
Dazu trägt bei, dass viele von \name[Immanuel]{Kant}s Grundannahmen über
Verankerungen von Eigenschaften und Verhaltensweisen in der jeweiligen
(biologischen) Natur des Menschen inzwischen der \enquote{soziologischen
Aufklärung} erlegen sind.\footcite[Vgl.][272]{Boehme:AnthropologieinpragmatischerHinsicht1985}

Die Lage verschlechtert sich weiter, wenn und insofern es nicht nur um eine
\emph{Natur}, sondern um eine \emph{Bestimmung} des Menschen geht. Denn diese verweist nicht
bloß auf eine Definition oder das \distanz{Wesen} des Menschen, sondern bezieht
sich auf ein Wozu, einen Zweck oder ein Ziel menschlichen
Lebens.\footcite[Vgl.][86--87,
93]{Brandt:TheGuidingIdeaofKantsAnthropologyandtheVocationoftheHumanBeing2003}
Das Projekt einer pragmatischen Anthropologie stützt sich in Übereinstimmung mit einer solchen Interpretation durchgängig auf teleologische
Betrachtungen.\footnote{\cite[Vgl.][65--67]{Cohen:KantandtheHumanSciences2009}.}
Und teleologische Behauptungen, sofern sie nicht die Handlungen von Menschen,
sondern Produkte der Natur betreffen, sind heute durchgängig in Verruf geraten.
Dem aufgeklärten Zeitgeist gänzlich indiskutabel erscheint -- um ein eindeutiges
Beispiel aus der Tugendlehre anzuführen -- \name[Immanuel]{Kant}s Versuch, eine Pflicht
zur sexuellen Enthaltsamkeit über die Annahme eines Naturzwecks der
Fortpflanzung und die Rede von einer \enquote{Begierde wider den Zweck der
Natur} zu motivieren.\footnote{\cite[][\S~7]{Kant:DieMetaphysikderSitten1977Tugendlehre}, \cite[VI:
424.12--426.32]{Kant:GesammelteWerke1900ff.}.}

\phantomsection\label{Abschnitt:TwoConceptsofLiberty}
Es geht hier letztlich um die Frage, ob eine Philosophie, welche die
\distanz{Bestimmung des Menschen} zu ihrem Ausgangspunkt macht, mit einem
liberalen Freiheitsverständnis vereinbar ist,
welches die je individuelle Souveränität des Einzelnen über die Ausgestaltung
seiner Freiheit als Ausgangspunkt berücksichtigt, oder notwendig ein
\singlequote{positives} Freiheitsverständnis verlangt, bei dem von vornherein
festgelegt ist, was als vernünftige (und schützenswerte) Ausübung der Freiheit
zählt.\footnote{Die Unterscheidung negativer und positiver Freiheitsbegriff
stammt meines Wissens nach von
\authorfullcite{Berlin:TheProperStudyofMankind1997}. Negative Freiheit
orientiere sich an der Frage \enquote{What is the area within which the subject
-- a person or group of persons -- is or should be left to do or be what he is able to do or
be, without interference by other persons?}
\parencite[][194]{Berlin:TheProperStudyofMankind1997}. Positive Freiheit
hingegen orientiere sich an der Leitfrage \enquote{What, or who, is the source
of control or interference that can determine someone to do, or be, this
rather than that?} \parencite[][194]{Berlin:TheProperStudyofMankind1997} Frei im
negativen Sinne ist jeder, der in seinen Handlungen nicht von anderen Menschen
absichtlich behindert oder gelenkt wird, er kann seine eigenen Entscheidungen
treffen und ausführen, ganz gleich welchen Ursprungs diese Entscheidungen sind.
Der Vertreter positiver Freiheit  entwickelt eine anspruchsvolle Theorie davon,
was es heißt, sein eigener Herr zu sein. Wer frei ist und frei handelt folgt
nach dieser Auffassung einer bestimmten, als Autorität angesehenen Instanz.}
Verkehrt sich eine Berufung auf Freiheit und Mündigkeit nicht selbst in ihr
Gegenteil, wenn sie zugleich sagt, was Ausdruck der Freiheit ist und was nicht,
den Begriff der Freiheit also mit positiven Inhalten
füllt?\footnote{\cite[Vgl.][191-242]{Berlin:TheProperStudyofMankind1997}. Eine
Entgegnung findet sich bei Charles
\textcite[vgl.][]{Taylor:WhatsWrongWithNegativeLiberty2005}.} Ist die Teleologie
hinter der Bestimmung des Menschen mit der Aufklärung vereinbar?
\begin{comment}
Wir
sollten genau hinschauen, wenn \name[Immanuel]{Kant} glaubt, eine Natur
oder gar Bestimmung des Menschen aufweisen zu können. Möglicherweise handelt es
sich bloß um halbherzige Säkularisierungen religiöser Dogmen, die eine auf
Freiheit ausgerichtete Aufklärung besser ganz eliminieren sollte. Sobald man aber
versucht, von teleologischen Momenten innerhalb der Anthropologie abzusehen,
erhält man eine (veraltete) Etappe in der Entwicklung der empirischen
Psychologie, die von der Anknüpfung an die Bestimmung des Menschen entlastet
ist, damit aber auch ihre Relevanz für die Aufklärungsphilosophie
verliert.\end{comment}

Können wir trotz dieser Bedenken für einen liberalen Aufklärungsbegriff von dem
Grundgedanken einer pragmatischen Anthropologie lernen? Was erforscht eine
philosophische Anthropologie? Was ist ihr Zweck? (Dieser Zweck muss nach allem,
was wir bisher wissen, mit dem Endzweck der Vernunft -- der Moral oder dem
Menschen als \emph{Noumenon} -- in Zusammenhang stehen.) Als Anthropologie ist
ihr Thema zunächst der Mensch.
Aber in welcher Hinsicht und mit welchen Mitteln soll der Mensch untersucht
werden? Schließlich gibt es viele Möglichkeiten, den Menschen zum Thema zu
machen, etwa in der Biologie, der Psychologie oder der Moralphilosophie. Und
entsprechend viele \distanz{Anthropologien} sind denkbar.


\authorfullcite{Schmidt:KantsTranscendentalEmpiricalPragmaticandMoralAnthropology2007}
nennt vier verschiedene Hinsichten (transzendental, empirisch, moralisch und
pragmatisch), in denen der Mensch Thema anthropologischer Untersuchungen sein
könne und die sich in der \titel{Anthropologie in pragmatischer Hinsicht}
fänden.\footnote{\cite[Vgl.][]{Schmidt:KantsTranscendentalEmpiricalPragmaticandMoralAnthropology2007}.
Vor allem Reinhard \name[Reinhard]{Brandt} wendet sich stets gegen eine
Identifizierung dieser verschiedenen Hinsichten. \cite[Vgl.
z.\,B.][92]{Brandt:KritischerKommentarzuKantsenquoteAnthropologieinpragmatischerHinsicht1999}.}
\name[Immanuel]{Kant} nennt die relevante Hinsicht bereits im Titel des Buches \enquote{pragmatisch}. Er erläutert die
pragmatische Hinsicht 1775 in einer Vorlesungsankündigung, die nicht die Anthropologie, sondern die
physische Geographie betrifft, und die auch deswegen interessant ist, weil hier
der Begriff der Bestimmung vorkommt:
\begin{quote}
  Die physische Geographie, die ich hiedurch ankündige, gehört zu einer Idee,
  welche ich mir von einem nützlichen akademischen Unterricht mache, den ich:
  die Vorübung in der \ori{Kenntnis der Welt} nennen kann. Diese Weltkenntnis
  ist es, welche dazu dient, allen sonst erworbenen Wissenschaften und
  Geschicklichkeiten das \ori{Pragmatische} zu verschaffen, dadurch sie nicht
  bloß vor die \ori{Schule} sondern vor das \ori{Leben} brauchbar werden, und
  wodurch der fertig gewordene Lehrling auf den Schauplatz seiner Bestimmung
  nämlich in die \ori{Welt} eingeführet
  wird.\footnote{\cite[A~12]{Kant:VondenverschiedenenRassenderMenschen1977},
  \cite[II: 443.12--19]{Kant:GesammelteWerke1900ff.}.}
\end{quote}
Physische Geographie und pragmatische Anthropologie sind die beiden Felder der
Weltkenntnis.\footnote{\enquote{Weltkenntnis setzt sich zusammen aus den
Erfahrungen mit der Natur und dem Umgang mit Menschen. Die Erkenntnis der
Natur ist Aufgabe und Gegenstand der physischen Geographie, die Kenntnis des
Menschen lehrt die Anthropologie -- und das Leben selbst}
\parencite[][185]{Boehr:PhilosophiefuerdieWelt2003}.} Insofern eine
Anthropologie in pragmatischer Hinsicht abgefasst ist, hat sie also tatsächlich
die Bestimmung des Menschen zum Thema. Dies reflektiert zunächst die auf
Integration in die Praxis orientierte Wissenschaftsauffassung:
Wissenschaftliches Erkennen ist kein Selbstzweck, sondern hat seinen Zweck und
sein Ziel immer in den Bedürfnissen und Interessen der Menschen. Eine auf ihre
Tauglichkeit in Handlungen und Praxisformen hin orientierte Erkenntnis nennt
\name[Immanuel]{Kant} \enquote{Weltkenntnis}. Und diese Weltkenntnis
korrespondiert dem Welt\emph{begriff}, der ebenso auf die Nützlichkeit oder
Brauchbarkeit des Wissens
verweist.\footnote{\cite[Vgl.][A~23]{Kant:ImmanuelKantsLogik1977}, \cite[IX:
24.6--8]{Kant:GesammelteWerke1900ff.}: \enquote{In dieser scholastischen
Bedeutung des Worts geht Philosophie nur auf Geschicklichkeit; in Beziehung auf
den Weltbegriff dagegen auf die Nützlichkeit.}}

Es war von Anfang an ein Charakteristikum der deutschen
Aufklärung, die handlungsorientierende Funktion und den Praxisbezug des Wissens
herauszustellen.\footnote{\cite[Vgl.][xviii]{Berndt:PraxisundProgramm2012}:
\enquote{Im aktivistischen Programm der Aufklärer ist die Theorie vielmehr
zugleich ein Teil der Praxis, denn die Theorie zielt hier letztlich immer auf
die praktische Umgestaltung der Welt und dient als deren Instrument.
Aufklärerische Theorie wird nicht um ihrer selbst willen betrieben, sondern als
ein – dirigierender, reflektierender, kontrollierender – Teil der Praxis
begriffen, so dass man von einem Re-Entry sprechen kann.}}
Christian \name[Christian]{Thomasius} hat die Ausrichtung des Wissens auf seinen
außeruniversitären Nutzen und seine Anwendbarkeit am Hofe und in der Welt in
seiner \titel{Introductio ad philosophiam aulicam} schon im Titel zum Programm
werden lassen.\footnote{Siehe zu Grundausrichtung der Philosophie bei
\name[Christian]{Thomasius}
\cite[][]{Schneiders:300JahreAufklaerunginDeutschland1989},
sowie
\cite{Ciafardone:UeberdasPrimatderpraktischenVernunftvordertheoretischenbeiThomasiusundCrusiusmitBeziehungaufKant1982}.}
Er betrachtet in selbstbewusster Abgrenzung von der aristotelischen Tradition
die Logik aus der Perspektive eines ethischen Nutzens, womit er die Tradition
aufklärerischer Logiklehrbücher einleitet, die nicht nur formale
Schlussverfahren thematisieren, sondern als praktische Logiken dem Leser zu
Mündigkeit und Selbständigkeit in der Wissenschaft wie auch im Leben verhelfen
wollen.\footnote{\phantomsection\label{Fussnote:PraktischeLogiken}\cite[Vgl.][248]{Albrecht:ChristianThomasius1999},
und zur praktischen Ausrichtung der Logik in der deutschen Aufklärung
\cite{Schneiders:PraktischeLogik1980}.} Und auch Christian \authorcite{Wolff:Psychologiaempirica1968}, der
Antipode zu \name[Christian]{Thomasius}, betont in der \titel{Praefatio} zum
\titel{Discursus praeliminaris de philosophia in genere} zunächst zwei Aspekte philosophischen
Wissens: Gewissheit und
Nützlichkeit.\footnote{\cite[Vgl.][262]{Wolff:Discursuspraeliminarisdephilosophiaingenere1996}:
\enquote{Duo inprimis sunt, quae in omni philosophia hactenus desiderantur.
Deest illa evidentia, quae sola assensum gignit certum atque immotum, nec, quae
in ea traduntur, usui vitae respondent.}} Später fügt er hinzu, dass die
Sicherheit unserer Erkenntnis kein Selbstzweck ist, sondern gerade im Dienste
der Nützlichkeit
steht.\footnote{\cite[Vgl.][\S~139]{Wolff:Discursuspraeliminarisdephilosophiaingenere1996}:
\enquote{Nos certam consequi studemus cognitionem, non vanitati litantes, sed
progressui scientiarum {\&} utilitati in vita intenti.}}  Und
\authorcite{Baumgarten:Metaphysica---Metaphysik2011} schreibt: \enquote{Keine Wahrheit darff bey Vernünfftigen
gäntzlich Brache liegen. \punkt\ Was wir lernen, muß nützlich seyn. Was nützlich
seyn soll, muß gebraucht werden. Was gebraucht wird, hat in Thun und Lassen
seinen
Einfluß.}\footcite[][\S~9]{Baumgarten:GedanckenvomVernuenfftigenBeyfallaufAcademien2008}
Es geht daher nicht ausschließlich um den aufgeklärten Wissenserwerb, sondern
gerade um den eigenständigen, kompetenten Umgang mit unserem Wissen und dessen
Anwendung im Leben oder -- wie die deutsche Aufklärungsphilosophie sich
ausdrückt -- in der \emph{Welt}. Dies spiegelt sich auch in der Verwendung des
Begriffs der \singlequote{\emph{Weltweisheit}} als Übersetzung des lateinischen
\enquote{philosophia} wieder. So soll Weltweisheit als Medium der Aufklärung
bewirken, dass die Menschen auch
\enquote{der Freiheit zu handeln nach und nach fähiger}
werden.\footnote{\cite[][A
493\,f.,]{Kant:BeantwortungderFrage:WasistAufklaerung?1977}
\cite[][VIII: 41.36]{Kant:GesammelteWerke1900ff.}.} Die Vorsilbe \enquote{Welt-}
deutet oft auf eine Ausrichtung der Wissenschaft auf erfolgreiche
Praxis hin, ebenso wie das Adjektiv \enquote{pragmatisch}.\footnote{\name[Immanuel]{Kant}
artikuliert dies mit den Worten, es gehe der Weltkenntnis darum, Welt zu \emph{haben}. Siehe
\cite[BA~vii]{Kant:AnthropologieinpragmatischerHinsicht1977}, \cite[VII:
120.9--11]{Kant:GesammelteWerke1900ff.}: \enquote{Noch sind die Ausdrücke:
die Welt \ori{kennen} und Welt \ori{haben} in ihrer Bedeutung ziemlich weit
auseinander; indem der eine nur das Spiel \ori{versteht}, dem er zugesehen
hat, der andere aber \ori{mitgespielt} hat.}}


\phantomsection\label{Abschnitt:AufklaerungunddieNuetzlichkeit}
Während Philosophen der Aufklärung ganz selbstverständlich die Nützlichkeit
intellektueller Anstrengungen einforderten -- freilich mit der Warnung, nicht
vorschnell darüber zu urteilen, ob etwas Nutzen abwerfe\footnote{So schreibt
{z.\,B.} \name[Immanuel]{Kant}, dass \enquote{kein Vorwitz der Erweiterung unserer
Erkenntnis nachteiliger sei, als der, so den Nutzen jederzeit zum voraus wissen
will, ehe man sich den mindesten Begriff von diesem Nutzen machen könnte, wenn
derselbe auch vor Augen gestellt würde}
\mkbibparens{\cite[][B 296]{Kant:KritikderreinenVernunft2003}, \cite[][III:
203.25--29]{Kant:GesammelteWerke1900ff.}}.} --, war es
\authorcite{Hegel:GesammelteWerke}, der in der \titel{Phänomenologie des
Geistes} die Nützlichkeit als letzten gemeinsamen Nenner der Aufklärung
brandmarkte.\footnote{\cite[Vgl.][IX: 310.22--315.11]{Hegel:GesammelteWerke}.} Die
Vernachlässigung (oder gar das Bestreiten) verbindlicher \distanz{Werte} und
vernünftiger Handlungsziele sind gängige Themen einer Bildungskritik, die sich
auch bei \name[Immanuel]{Kant} findet, wenn dieser gegen eine lediglich auf
Zweckrationalität zielende Ausbildung gewendet schreibt:
\begin{quote}
  Weil man in der frühen Jugend nicht weiß, welche Zwecke uns im Leben aufstoßen
  dürften, so suchen Eltern vornehmlich ihre Kinder recht \ori{vielerlei} lernen
  zu lassen, und sorgen für die \ori{Geschicklichkeit} im Gebrauch der Mittel zu
  allerlei \ori{beliebigen} Zwecken, von deren keinem sie bestimmen können, ob
  er nicht etwa wirklich künftig eine Absicht ihres Zöglings werden könne, wovon
  es indessen doch \ori{möglich} ist, daß er sie einmal haben möchte, und diese
  Sorgfalt ist so groß, daß sie darüber gemeiniglich verabsäumen, ihnen das
  Urteil über den Wert der Dinge, die sie sich etwa zu Zwecken machen möchten,
  zu bilden und zu
  berichtigen.\footnote{\cite[BA~41--2]{Kant:GrundlegungzurMetaphysikderSitten1965},
  \cite[IV: 415.20--17]{Kant:GesammelteWerke1900ff.}.}
\end{quote}
Ein Wissen, das sich als Geschicklichkeit für beliebige Zwecke nutzen lässt,
nennt \name[Immanuel]{Kant} technisch. Und eine Wissensvermittlung, die sich in diesem
Sinne als technische Ausbildung versteht, vermittelt zwar die nötige
Mittelkompetenz, die jemand zur Umsetzung beliebiger Zwecke benötigt. Aber sie vernachlässigt
dabei doch die nicht minder nötige Zielkompetenz, also die Fähigkeit, sich
selbst vernünftige Zwecke und Ziele vorzunehmen.

Der Mensch hat die Fähigkeit, sich für unterschiedliche Handlungen, Grundsätze
und Lebensentwürfe zu entscheiden, und sucht nach Orientierung; und diese Suche
nach Orientierung ist nicht nur bei
\authorcite{Spalding:BetrachtungueberdieBestimmungdesMenschen1749} klar als
Thema seiner Abhandlung ausgesprochen. Auch \name[Immanuel]{Kant}s Anthropologie
ist als Orientierungswissen für das je eigene Leben
gedacht.\footcite[Vgl.][105--108]{Cohen:KantandtheHumanSciences2009} Zu den
notwendigen Zielen gehört, was \name[Immanuel]{Kant} die Bestimmung des Menschen
nennt und was die Vorgaben der Moral zu seinem Zentrum hat. Vernunft, als
anzustrebender Leitfaden des eigenen Lebens und Ziel von Bildung, ist bei
\name[Immanuel]{Kant} keine Zweckrationalität oder instrumentelle Vernunft,
schon weil sie Moral und Ethik enthält.\footnote{Ähnliches behauptet schon
Christian August \authorcite{Crusius:Anweisungvernuenftigzuleben1744}:
\enquote{Zu einem vernünftigen Leben gehört nicht nur, daß man klug, sondern
auch hauptsächlich, daß man tugendhaft lebe, worauf die gereinigte Vernunft am
allermeisten dringet} \parencite[][Vorrede,
unpaginiert]{Crusius:Anweisungvernuenftigzuleben1744}.} Wenn
\name[Immanuel]{Kant} nun sagt, dass die Moral die Wissenschaft von der
\emph{ganzen} Bestimmung sei\footnote{Siehe das Zitat zu Anm.
\ref{Anmerkung:GanzeBestimmung} auf S. \pageref{Anmerkung:GanzeBestimmung},
wo \name[Immanuel]{Kant} den Endzweck mit der ganzen Bestimmung des Menschen
identifiziert.}, dann sieht man, dass Moral nicht eine Zielvorgabe neben anderen
innerhalb der Weltkenntnis sein kann, sondern an zentraler Stelle auch die weiteren Zielvorgaben \emph{bestimmt} oder zumindest \emph{einschränkt}.

Es gibt nach \name[Immanuel]{Kant} zwei Themen der pragmatischen Weltkenntnis:
die Natur und den Menschen; und diesen entsprechen die physische Geographie und
die Anthropologie.\footnote{\authorfullcite{Cohen:KantandtheHumanSciences2009}
identifiziert die physische Geographie, insofern sie ebenfalls den Menschen
thematisiert, mit der
physiologischen Anthropologie. \enquote{The object of pragmatic anthropology is
the human being considered as a free rational being, whilst physical geography
studies him as one \enquote{thing} on earth, independent of his intentionality}
\parencite[63]{Cohen:KantandtheHumanSciences2009}. Danach sähe \name[Immanuel]{Kant} die
physische Geographie zumindest 1798 nicht (mehr) als pragmatische Wissenschaft
an. Dem widerspricht \name[Immanuel]{Kant} aber innerhalb der
\titel{Anthropologie in pragmatischer Hinsicht} in einer Anmerkung zur Vorrede
(\cite[vgl.][BA~xiii--xiv]{Kant:AnthropologieinpragmatischerHinsicht1977},
\cite[][VII: 122.8--15]{Kant:GesammelteWerke1900ff.}).} Unter diesen beiden
jedoch nehme die Anthropologie eine Sonderrolle ein, weil sie sich auf den
Menschen konzentriere. In der Vorrede zur \titel{Anthropologie in pragmatischer
Hinsicht} schreibt \name[Immanuel]{Kant}:
\begin{quote}
  Alle Fortschritte in der Kultur, wodurch der Mensch seine Schule macht, haben
  das Ziel, diese erworbenen Kenntnisse und Geschicklichkeiten zum Gebrauch für
  die Welt anzuwenden; aber der wichtigste Gegenstand in derselben, auf den er
  jene verwenden kann, ist \ori{der Mensch}: weil er sein eigener letzter Zweck
  ist.\footnote{\cite[][BA~iii]{Kant:AnthropologieinpragmatischerHinsicht1977},
  \cite[][VII: 119.2--6]{Kant:GesammelteWerke1900ff.}.}
\end{quote}
Der Mensch ist ein letzter Zweck, wenngleich noch nicht der Endzweck, welcher
er erst als \emph{Noumenon} sein kann. Weil Weltkenntnis immer auf den Menschen
als ihren Zweck ausgerichtet ist, verdiene besonders die Erkenntnis des Menschen den Namen
Weltkenntnis.\footnote{\cite[Vgl.][BA~iii--iv]{Kant:AnthropologieinpragmatischerHinsicht1977},
\cite[][VII: 119.6--8]{Kant:GesammelteWerke1900ff.}.} 
Eine pragmatische Anthropologie ist also zunächst eine auf Brauchbarkeit im Leben angelegte Lehre vom
Menschen, die sich nicht nur an Fachphilosophen, sondern an ein breites Publikum richtet.

In der Vorrede konkretisiert \name[Immanuel]{Kant} die pragmatische Hinsicht mit Bezug auf den Gegenstand
der Untersuchung:
\begin{quote}
  Eine Lehre von der Kenntnis des Menschen, systematisch abgefaßt
  (Anthropologie), kann es entweder in \ori{physiologischer} oder in
  \ori{pragmatischer} Hinsicht sein. -- Die physiologische Menschenkenntnis geht
  auf die Erforschung dessen, was die \ori{Natur} aus dem Menschen macht, die
  pragmatische auf das, was \ori{er}, als freihandelndes Wesen, aus sich selber
  macht, oder machen kann und
  soll.\footnote{\cite[][BA~iv]{Kant:AnthropologieinpragmatischerHinsicht1977},
  \cite[][VII: 119.9--14]{Kant:GesammelteWerke1900ff.}.}
\end{quote}
Mit der Frage danach, was der Mensch aus sich selbst machen \emph{kann} und
\emph{soll}, finden die Frage nach der Bestimmung des Menschen und die
Moralphilosophie Eingang in die Anthropologie. Denn die Anthropologie behandelt
auch \emph{Ziele} menschlichen Lebens, nicht bloß Mittel zur Erreichung
kontingenter Zwecke. Ebenso wichtig ist aber die Angabe, dass die pragmatische
Anthropologie untersucht, was der Mensch aus sich selbst \emph{macht}, und
nicht, was er von Natur aus \emph{ist}.


Der Mensch ist als biologisches Wesen Gegenstand empirischer Wissenschaften, die
ihn hinsichtlich der Eigenschaften untersuchen, die er von Natur aus hat. Aber
dies ist weder die einzige Art der Herangehensweise an den Menschen, noch aus
philosophischer Sicht besonders aufschlussreich, denn sie ist für
vernunftbegabte und darum freie Wesen nicht ausreichend. Dies heißt nicht, dass
sie zu vernachlässigen wäre; vielmehr ist sie zu ergänzen. (Möglicherweise prägt
dabei die pragmatische Sichtweise auf den Menschen die Art und Weise, wie er
Gegenstand der Naturwissenschaft werden kann und soll.) \name[Immanuel]{Kant}
ändert also die Herangehensweise an die
Untersuchung des Menschen in einer entscheidenden Hinsicht: Es geht nicht mehr
(primär) darum, was der Mensch \emph{ist}, sondern darum, was er aus sich und
der Welt um ihn herum
\emph{macht}.\footnote{\cite[Vgl.][61]{Cohen:KantandtheHumanSciences2009}:
\enquote{Kant's method entails a decisive re-evaluation of traditional enquiries
into human nature: it redirects the question, \enquote{what is the human being?}
from his passive essence to his active relationship with the world -- from what
he \ori{is} to what he \ori{does}.} Siehe
\cite[35--61]{Cohen:KantandtheHumanSciences2009}, sowie \cite{Mengusoglu:DerBegriffdesMenschenbeiKant1966}.
Siehe außerdem
\cite[][275]{Boehme:AnthropologieinpragmatischerHinsicht1985}:
\enquote{Der ganze erste Teil der \ori{Anthropologie} hat dann die pragmatische
Funktion, dem Leser an der Entwicklung der anthropologischen Strukturen
klarzumachen, daß diese Würde nicht gegeben ist, sondern durch Leistung errungen
und stabilisiert werden muß. Der Mensch ist eben erst Mensch, insofern er etwas
aus sich macht.}}


Nicht als Naturwesen ist der Mensch also Thema der pragmatischen Anthropologie,
sondern als eines, das frei handelt. Was aber soll die Anthropologie dann noch
als \distanz{Natur} des Menschen beschreiben? Kann es dann überhaupt noch eine
adäquate und zugleich allgemeine Beschreibung dessen geben, was
\name[Immanuel]{Kant} den Charakter des Menschen nennt? Sofern es um den
Charakter des Menschen als eines Vernunftwesens geht, kann nur auf dessen
Freiheit verwiesen werden, sich diesen Charakter selbst zu schaffen:
\begin{quote} 
  Es bleibt uns also, um dem Menschen im System der lebenden Natur seine Klasse
  anzuweisen und so ihn zu charakterisieren, nichts übrig, als: daß er einen
  Charakter hat, den er sich selbst schafft; indem er vermögend ist, sich nach
  seinen von ihm selbst genommenen Zwecken zu perfektionieren; wodurch er, als
  mit \ori{Vernunftfähigkeit} begabtes Tier (animal rationabile), aus sich
  selbst ein \ori{vernünftiges} Tier (animal rationale) machen
  kann[.]\footnote{\cite[A~315]{Kant:AnthropologieinpragmatischerHinsicht1977},
  \cite[VII: 321.29--35]{Kant:GesammelteWerke1900ff.}.}
\end{quote}
Der Mensch schafft sich seinen Charakter selbst. Nun gebe es im Kontext
einer pragmatischen Anthropologie zwei Bedeutungen von \enquote{Charakter}:
Dieses Wort meine zum einen den natürlichen (\enquote{physischen}) Charakter,
den jemand von Natur aus mitbringt. \name[Immanuel]{Kant} denkt, dass uns
Eigenschaften wie das Temperament angeboren sind. Darüber hinaus bezeichne das
Wort aber auch den \enquote{moralischen} Charakter, den \name[Immanuel]{Kant}
\enquote{Denkungsart} nennt. Der Ausdruck \enquote{Denkungsart} ist wiederum
zentral bei der Bestimmung des Begriffs der Aufklärung, der es ja gerade um eine
Änderung der Denkungsart zu tun sei.\footnote{Siehe oben Kapitel
\ref{subsection:SelbstdenkenbeiKant}.} Und dies ist der Charakter, den sich ein
Mensch selbst schafft und der genuiner Ausdruck von Freiheit
ist.\footnote{\cite[Vgl.][A~255-6]{Kant:AnthropologieinpragmatischerHinsicht1977},
\cite[VII: 285.6-21]{Kant:GesammelteWerke1900ff.}.} Wie schon bei
\authorcite{Spalding:BetrachtungueberdieBestimmungdesMenschen1749} geht es
darum, was ein frei handelnder Mensch aus sich selbst und seinem eigenen Leben
macht.

\phantomsection\label{Abschnitt:MaximenHandlungenFreiheit}
Einen (moralischen) Charakter hat überhaupt erst derjenige, der nach
\emph{Maximen}\footnote{Siehe zum Begriff der Maxime den Forschungsüberblick in
\cite{Gressis:RecentWorkonKantianMaxims2010,Gressis:RecentWorkonKantianMaximsI:EstablishedApproaches2010}.}
handelt, d.\,i. nach Grund\-sät\-zen, denen er Autorität über seine eigenen
Handlungen beimisst und die in seinem Handeln wirksam
werden.\footnote{\cite[Vgl.][A~256]{Kant:AnthropologieinpragmatischerHinsicht1977},
\cite[VII: 285.13--15]{Kant:GesammelteWerke1900ff.}: \enquote{Der Mann von
Grundsätzen, von dem man sicher weiß, wessen man sich, nicht etwa von seinem
Instinkt, sondern von seinem Willen zu versehen hat, hat einen Charakter.} Siehe
auch \cite[A~266]{Kant:AnthropologieinpragmatischerHinsicht1977}, \cite[VII:
292.6--9]{Kant:GesammelteWerke1900ff.}: \enquote{Einen Charakter aber
schlechthin zu haben, bedeutet diejenige Eigenschaft des Willens, nach welcher
das Subjekt sich selbst an bestimmte praktische Prinzipien bindet, die er sich
durch seine eigene Vernunft unabänderlich vorgeschrieben hat.}} Nun ist es
umstritten, ob \name[Immanuel]{Kant}s Handlungstheorie überhaupt ein
Handeln zulässt, welches \emph{nicht} durch Maximen geleitet ist.\footnote{Siehe
dazu
\cite[][89--92]{Schuessler:KantsethischesLuegenverbot--derSonderfallderLuegeausFurcht2013},
sowie die dort angeführte Literatur.} Seine Auskünfte zum moralischen Charakter
jedenfalls stützen die Ansicht, wonach dies möglich ist. Ich schließe
mich hier den von
\authorfullcite{Schuessler:KantsethischesLuegenverbot--derSonderfallderLuegeausFurcht2013}
angeführten Gründen für diese Auffassung an und gehe davon aus, dass
Menschen zwar aus Maximen handeln können, aber mitunter auch gegen ihre Maximen
und manchmal sogar ganz ohne Maximen handeln. Maximen sind -- grob gesprochen --
allgemeine Lebensregeln, die aus unterschiedlichsten Gründen aufgestellt werden
können und die wir als für uns jeweils selbst verbindlich
anerkennen.\footnote{Die Deutung von Maximen als \singlequote{Lebensregeln}
findet sich u.\,a. bei Rüdiger \textcite{Bittner:Maximen1974}. Siehe dazu
kritisch \cite[][62--67]{Schwartz:DerBegriffderMaximebeiKant2006}.}


Damit, dass jemand nach Maximen handelt, ist noch nicht gesagt, dass die
entsprechenden Maximen moralisch wertvoll sind. Auch derjenige, dessen Maximen den objektiven
Gesetzen der Moral widersprechen, hat Charakter und ist daher -- so behauptet
\name[Immanuel]{Kant} -- höher zu schätzen als derjenige, der sich ohne feste
Grundsätze nur von seinen Instinkten und kurzfristigen Wünschen treiben
lässt.\footnote{\cite[Vgl.][A~266]{Kant:AnthropologieinpragmatischerHinsicht1977},
\cite[VII: 292.10--14]{Kant:GesammelteWerke1900ff.}: \enquote{Ob nun diese
Grundsätze auch bisweilen falsch und fehlerhaft sein dürfte, so hat doch das
Formelle des Wollens überhaupt, nach festen Grundsätzen zu handeln (nicht wie in
einem Mückenschwarm bald hiehin bald dahin abzuspringen), etwas Schätzbares und
Bewundernswürdiges in sich; wie es denn auch etwas Seltenes ist.} Siehe ferner
\cite[][A~269]{Kant:AnthropologieinpragmatischerHinsicht1977}, \cite[VII:
293.14--23]{Kant:GesammelteWerke1900ff.}.} 
Nach festen Grundsätzen muss nicht nur derjenige handeln, der sich an der Moral
orientiert, sondern gerade auch derjenige, der sein eigenes Glück im Sinne eines
langfristigen und beständigen Zustands der Zufriedenheit erreichen
möchte. \name[Immanuel]{Kant} spricht hier von (Privat-) \emph{Klugheit}. Wenn jemand
seine Handlungen in Abhängigkeit von subjektiven Grundsätzen oder Maximen
bestimmt, so handelt er frei. Und daher ist sowohl moralisches Handeln, als auch
kluges Handeln Ausdruck von Freiheit, insofern es ein Handeln nach Grundsätzen
ist. Der Mensch, der nach Maximen handelt, wird ein Subjekt, das nicht mehr bloß
seiner Natur unterworfen ist, sondern sich seine eigene
Seinsweise in gewisser Weise selbst geschaffen hat.\footnote{\cite[Vgl.][A
267]{Kant:AnthropologieinpragmatischerHinsicht1977}, \cite[VII:
292.15--18]{Kant:GesammelteWerke1900ff.}: \enquote{Es kommt hiebei nicht auf das
an, was die Natur aus dem Menschen, sondern was dieser \ori{aus sich selbst
macht}; denn das erstere gehört zum Temperament (wobei das Subjekt großenteils
passiv ist) und nur das letztere gibt zu erkennen, daß er einen Charakter
habe.}} Die Entwicklung der Vernunft und die Ausgestaltung der je eigenen
Freiheit sind zunächst der gesuchte Endzweck. Die Bestimmung des Menschen ist
eine Bestimmung zur Freiheit.

Nun stellt sich die Frage, wozu \name[Immanuel]{Kant} bei der Untersuchung einer
Bestimmung des Menschen auf die empirische Anthropologie zurückgreift, wo es
doch um eine Bestimmung zur Freiheit geht, der Mensch also gerade nicht als
Naturwesen, sondern als \emph{Noumenon} in den Blick kommt. Die Antwort lautet:
Die Frage, welche Eigenschaften der Mensch \distanz{von Natur aus}, also ohne
sein Zutun hat (sein physischer Charakter), interessiert zunächst als sekundäre
Frage wegen der Bedeutung für die Handlungsmöglichkeiten, die der Menschen als
Rahmenbedingungen seines freien Handelns vorfindet. Wie wir im nächsten
Abschnitt sehen werden, versucht die Anthropologie damit die \emph{Endlichkeit} des
Menschen als Hintergrundfolie seines Handelns zu berücksichtigen. Die Welt als
das, was wir erfahren und empirisch erforschen können, gibt dem Menschen seine
Bestimmung nicht vor -- dies vermag nur die Vernunft --, aber sie ist doch der
\enquote{Schauplatz seiner Bestimmung}\footnote{\cite[A~12]{Kant:VondenverschiedenenRassenderMenschen1977},
\cite[II: 443.18]{Kant:GesammelteWerke1900ff.}.}, den es in seiner
Beschaffenheit zu berücksichtigen gilt. Eine Anthropologie hat also
zunächst nicht wegen der Handlungsziele, sondern wegen der Rahmenbedingungen
freien Handelns mit Erfahrung und nicht nur mit Vernunft und Freiheit zu tun.
Sonst benötigte sie keine Erfahrung und müsste nicht empirisch vorgehen, sondern
fiele mit der Moralphilosophie (und eventuell einer praktischen Logik) zusammen.
Zu diesem Schauplatz rechnet \name[Immanuel]{Kant} nun nicht nur das, was uns
als Natur und Welt umgibt, sondern auch das, was wir von Natur aus selbst sind.



Eine pragmatische Anthropologie befasst sich auch mit natürlichen Eigenschaften
des Menschen, immer aber aus der Perspektive desjenigen, der sich als
vernunftbegabtes Wesen frei zu entwickeln trachtet. Und aus dieser Perspektive
muss Anthropologie insbesondere auch die Hindernisse und
Widrigkeiten in den Blick nehmen, die unserer Vernunft und
Freiheit gerade im Wege stehen. Was der Mensch von Natur aus \emph{ist} -- im
Gegensatz zu dem, was er aus freien Stücken aus sich selbst \emph{macht} --
betrachtet \name[Immanuel]{Kant} primär als ein \emph{Hindernis} der Freiheit.
Freilich sind gerade diese
Hindernisse im Interesse einer realistischen Umsetzung unserer Freiheit damit
auch Thema der Philosophie, speziell der pragmatischen
Anthropologie.\footcite[Vgl.][68]{Cohen:KantandtheHumanSciences2009} Sie
betrachtet diese einerseits natürlich mit dem Ziel, die anderweitig vorgegebenen
Ziele effektiv \emph{umsetzen} zu können. Andererseits haben diese Betrachtungen aber
auch Auswirkungen auf die Ziele selbst: Was von uns realistischerweise nicht
umgesetzt werden kann, sollte vernünftigerweise auch nicht als Ziel verfolgt
werden. Es ist beispielsweise
unvernünftig, Ziele zu verfolgen, zu denen man kein Talent mitbringt. Wenn aber
die Entwicklung der eigenen Talente eine Forderung der Moral ist (wofür
\name[Immanuel]{Kant} sie hält), muss man natürlich wissen, welche Talente man
denn besitzt. (Freilich ist die Kenntnis der je eigenen Talente kein Thema einer
allgemeinen philosophischen Anthropologie. Aber sie erläutert doch recht
anschaulich, wie sich \name[Immanuel]{Kant} die Relevanz empirischen Wissens
auch für die moralische Zielsetzung denkt.) Kompetenz in der eigenen Zielsetzung
setzt voraus, über das Erreichbare und die Grenzen der eigenen
Handlungsmöglichkeiten Bescheid zu wissen.\footnote{Ähnlich beschreibt bereits
\authorfullcite{Doering:UeberKantsLehrevonBegriffundAufgabederPhilosophie1885}
die Aufgabe der Philosophie nach \name[Immanuel]{Kant}: \enquote{Die Bestimmung der Aufgabe der Philosophie [gemeint
ist \name[Immanuel]{Kant}s Bestimmung derselben; A.\,G.] ist in ihrer
Allgemeinheit richtig; nach ihr bildet die Philosophie das Mittel- und Bindeglied zwischen der reinen theoretischen
Wissenschaft als Erkenntniß des Thatsächlichen im weitesten Umfange und ohne
Nebengedanken und den praktischen Disciplinen als Anleitungen zur Verwirklichung
des dem Menschen Wichtigen und Werthvollen, indem sie lehrt, was das für den
Menschen Werthvolle ist und in welchem Umfange es nach Lage der gegebenen
Welteinrichtung verwirklicht werden kann}
\parencite[][479]{Doering:UeberKantsLehrevonBegriffundAufgabederPhilosophie1885}.}
Außerdem ergeben sich besondere Zwecke aus den besonderen Umständen, in denen
sich der Mensch als Mensch stets befindet.


Zu den auffälligsten Kennzeichen des
natürlichen Charakters des Menschen gehört nach \name[Immanuel]{Kant} die
\enquote{ungesellige Geselligkeit}:
\begin{quote}
  \punkt wobei aber das Charakteristische der Menschengattung, in Vergleichung
  mit der Idee möglicher vernünftiger Wesen auf Erden überhaupt, dieses ist: daß die
  Natur den Keim der \ori{Zwietracht} in sie gelegt und gewollt hat, daß ihre
  eigene Vernunft aus dieser diejenige \ori{Eintracht}, wenigstens die
  beständige Annäherung zu derselben,
  herausbringe\punkt\footnote{\cite[B~313--314]{Kant:AnthropologieinpragmatischerHinsicht1977},
  \cite[VII: 322.3--8]{Kant:GesammelteWerke1900ff.}. In der \titel{Idee zu
  einer allgemeinen Geschichte in weltbürgerlicher Absicht} schreibt er:
  \enquote{Ich verstehe hier unter dem Antagonism die \ori{ungesellige Geselligkeit}
  der Menschen; d.\,i. den Hang derselben, in Gesellschaft zu treten, der doch
  mit einem durchgängigen Widerstande, welcher diese Gesellschaft beständig zu
  trennen droht, verbunden ist}
  \mkbibparens{\cite[][A
  392]{Kant:IdeezueinerallgemeinenGeschichteinweltbuergerlicherAbsicht1977},
  \cite[][VIII: 20.30--33]{Kant:GesammelteWerke1900ff.}}.}
\end{quote}
Die Aufforderung an den Menschen, einen Zustand der Eintracht und des
friedlichen Miteinander herzustellen, -- die Bestimmung zur Errichtung einer
bürgerlichen Gesellschaft --  ist nur vor dem Hintergrund der Tatsache
verständlich, dass der Mensch von Natur aus auf Gemeinschaft angewiesen, hierzu
aber nur bedingt geeignet ist.\footnote{Von Vertretern einer
\singlequote{Kritischen Theorie} in der Traditionslinie von
\authorcite{Horkheimer:DialektikderAufklaerung1997} wird dieser Punkt mitunter
als bürgerliche Ideologie angesehen und behauptet, der Antagonismus liege nicht
in der Natur des Menschen, sondern resultiere aus gesellschaftlichen
Verhältnissen, die es zu überwinden gelte.
\cite[Vgl.][passim]{Staedtler:KantunddieAporetikmodernerSubjektivitaet2011}.}
Dass wir zu einem friedlichen und einträchtigen Leben verpflichtet sind, wissen
wir \emph{a priori}; aber dass wir als Menschen zur Zwietracht neigen,
können wir nur aus Erfahrung wissen. Wenn empirische Erkenntnisse also in die
Generierung notwendiger Handlungsziele eingehen, dann geschieht dies hier analog
zu dem Fall, in dem ich als moralisch notwendiges Ziel erkenne, einem Menschen
zu helfen, der in Not geraten ist. Dass ich zur Hilfe verpflichtet bin, weiß ich
zwar \emph{a priori}, aber dass jemand diese Hilfe benötigt, erkenne ich eben \emph{a posteriori}.

\name[Immanuel]{Kant}s pragmatische Anthropologie thematisiert die Grundlagen
und Bedingungen unseres Menschseins -- sie greift damit mit dem Ziel der
Förderung unserer Freiheit aus dem Humanismus die Thematik einer \emph{conditio
humana}
auf.\footcite[Vgl.][476]{Zoeller:DieBestimmungderBestimmungdesMenschenbeiMendelssohnundKant2001}
Dabei mögen \name[Immanuel]{Kant}s Vorgehen und seine Behauptungen im Einzelnen
fragwürdig erscheinen; zu würdigen ist zunächst das Projekt als solches. Es geht
darum, empirisches Wissen mit Blick auf seine Relevanz für die je eigene
selbstbestimmte Lebensführung und Persönlichkeitsentwicklung aufzuarbeiten.
Deswegen lässt sich pragmatische Anthropologie nicht durch Moral ersetzen. Sie
ist auch keine angewandte Ethik im Sinne der praktischen Anthropologie, von der
\name[Immanuel]{Kant} in der \titel{Grundlegung zur Metaphysik der Sitten}
spricht.\footnote{\cite[Vgl.][BA v]{Kant:GrundlegungzurMetaphysikderSitten1965},
\cite[][IV: 388.9--14]{Kant:GesammelteWerke1900ff.}.} Das Wissen um die
\emph{conditio humana} ist eine notwendige Bedingung zur Umsetzung der
Bestimmung des Menschen zur Freiheit und gehört damit zu den wesentlichen
Zwecken der Vernunft.

\begin{comment}
Ich hatte oben einen ersten Einwand formuliert und gefragt, wie sich die
Thematisierung einer Bestimmung des Menschen zu dem Begriff der Freiheit
verhält, ob sie sich mit einem negativen Freiheitsbegriff verträgt oder einen
positiven Freiheitsbegriff zugrunde legt, und wie sich dies auf das Projekt
einer liberalen Aufklärung auswirkt.\footnote{Siehe S.
\pageref{Abschnitt:TwoConceptsofLiberty}.} Eine naheliegende Antwort lautet,
dass verbindliche Vorgaben, wie die je individuelle Freiheit auszufüllen sei,
ausschließlich der Moral entstammen. Wenn es einen Sinn gibt, in dem
\name[Immanuel]{Kant} einen \singlequote{positiven} Freiheitsbegriff verficht,
dann ist es danach genau der Sinn, den
\authorcite{Berlin:TheProperStudyofMankind1997} bereits ohne Erwähnung der
Anthropologie oder einer Bestimmung des Menschen der Moralphilosophie
\name[Immanuel]{Kant}s attestiert. Eine Anthropologie in pragmatischer Hinsicht
versucht nicht, die Freiheit des einzelnen mit weiteren Inhalten zu füllen und
den Begriff der Freiheit damit -- aus liberaler Sicht -- \emph{ad absurdum} zu
führen. Dennoch gibt es Maßstäbe der Vernunft, anhand derer wir unser Denken und
Handeln bewerten können.

\phantomsection\label{Abschnitt:DialektikderAufkaerungEinwaende}
Ein zweiter, dem ersten diametral entgegengesetzter Einwand gegen die Betonung
\singlequote{pragmatischer} Wissenschaften war: Damit reduziere Aufklärung unser Denken auf die bloße
Nützlichkeit und lasse kein vernünftiges Ziel mehr bestehen. Diesen Vorwurf
machen \authorcite{Horkheimer:DialektikderAufklaerung1997} im Ausgang von
\authorcite{Hegel:GesammelteWerke} stark.\footnote{Siehe
S.~\pageref{Abschnitt:AufklaerungunddieNuetzlichkeit}. Kürzlich wurde dieser
Einwand ausführlich von
\textcite[vgl.][passim]{Staedtler:KantunddieAporetikmodernerSubjektivitaet2011}
aufgegriffen.} Er ist die unmittelbare Kehrseite der Antwort auf den letzten
Einwand, denn er argumentiert, dass ein rein negativer Freiheitsbegriff den
einzelnen Menschen ohne Orientierung lasse und letztlich jede Bestimmung
vernünftiger Ziele jenseits der bloßen Selbsterhaltung unterminiere. Im nächsten
Abschnitt zeige ich auf, welche Erkenntnisse wir nach \name[Immanuel]{Kant}
benötigen, um unser Handeln an der Vernunft auszurichten.
\end{comment}

\section{Aufklärung des endlichen Willens}
\subsection{Arten handlungsorientierenden
Wissens}\label{subsection:aufklaerungundpraxis} Wenn die Bestimmung eines
Menschen also das ist, wovon sein Charakter der Ausdruck ist, dieser Charakter aber frei und die Bestimmung eine Bestimmung zur Freiheit sein soll, dann bleibt als
Inhalt einer allgemeinen Bestimmung des Menschen nichts übrig als die
Entwicklung und Konkretisierung der Freiheit selbst vor dem Hintergrund der
Situation, mit der sich der Mensch \emph{qua} Mensch in dieser Welt konfrontiert
sieht (der \emph{conditio humana}). Die \emph{conditio humana} betrifft dabei insbesondere
auch die eigene Verfassung des Menschen als eines Naturwesens. Und genau dies
findet sich bei genauerem Hinsehen innerhalb der \titel{Anthropologie in
pragmatischer Hinsicht} als Angabe der Bestimmung des Menschen:
\begin{quote}
  \phantomsection\label{Zitat:KultivierungZivilisierungMoralisierungalsBestimmungdesMenschen}Die
  Summe der pragmatischen Anthropologie in Ansehung der Bestimmung des Menschen
  und die Charakteristik seiner Ausbildung ist folgende. Der Mensch ist durch
  seine Vernunft bestimmt, in einer Gesellschaft mit Menschen zu sein, und in
  ihr sich durch Kunst und Wissenschaft zu \ori{kultivieren}, zu
  \ori{zivilisieren} und zu \ori{moralisieren}; wie groß auch sein tierischer
  Hang sein mag, sich den Anreizen der Gemächlichkeit und des Wohllebens, die er
  Glückseligkeit nennt, \ori{passiv} zu überlassen, sondern vielmehr
  \ori{tätig}, im Kampf mit den Hindernissen, die ihm von der Rohigkeit seiner
  Natur anhängen, sich der Menschheit würdig zu
  machen.\footnote{\cite[A~321]{Kant:AnthropologieinpragmatischerHinsicht1977},
  \cite[VII: 324.33--325.4]{Kant:GesammelteWerke1900ff.}.}
\end{quote}
In der \titel{Pädagogik}, die ebenfalls auf die Bestimmung des Menschen
ausgerichtet ist,\footnote{\cite[Vgl.][A 17]{Kant:UeberPaedagogik1977},
\cite[IX:
447.30--33]{Kant:GesammelteWerke1900ff.}: \enquote{Kinder sollen nicht dem
gegenwärtigen, sondern dem zukünftig möglich bessern Zustande des menschlichen
Geschlechts, das ist: der Idee der Menschheit und deren ganzer Bestimmung
angemessen, erzogen werden.} Zu pädagogischen Ansprüchen des
Aufklärungsprogramms siehe \cite[][\pno~513\,f.]{Theis:KantetlAufklaerung2012}.}
nennt \name[Immanuel]{Kant} als erste Stufe die
\emph{Disziplinierung}, die in der \titel{Kritik der Urteilskraft} zur Kultur (der \enquote{Zucht}) und in der \titel{Anthropologie in pragmatischer Hinsicht}
zur Zivilisierung gezählt
wird,\footnote{\cite[Vgl.][B~392]{Kant:KritikderUrteilskraft2009}, \cite[V:
432.3--12]{Kant:GesammelteWerke1900ff.}; \cite[A
319]{Kant:AnthropologieinpragmatischerHinsicht1977}, \cite[VII:
323.26]{Kant:GesammelteWerke1900ff.}.} behält ansonsten aber das Schema aus
Kultivierung, Zivilisierung und Moralisierung
bei.\footnote{\cite[Vgl.][A~22f.]{Kant:UeberPaedagogik1977}, \cite[IX:
449.27--450.14]{Kant:GesammelteWerke1900ff.}.}

Wenn man mit \authorcite{Baumgarten:Metaphysica---Metaphysik2011}, von dem
ausgehend \name[Immanuel]{Kant} seine Position entwickelt\footnote{Vgl.
\cite[][5, 29--65]{Schwaiger:KategorischeundandereImperative1999}, sowie
\cite[][152]{Schwaiger:KlugheitbeiKant2002}.}, Zielkompetenz als Weisheit und
Mittelkompetenz als Klugheit
bezeichnet,\footnote{\cite[Vgl.][\S~882]{Baumgarten:Metaphysica---Metaphysik2011},
\cite[XVII: 172.22--29]{Kant:GesammelteWerke1900ff.}: \enquote{\ori{Sapientia}
nexus finalis \ori{generatim} est perspicientia, et quidem finium
\ori{sapientia speciatim}, remediorum \ori{prudentia}.} \authorcite{Baumgarten:Metaphysica---Metaphysik2011}
selbst schlägt vor, \enquote{sapientia} mit \enquote{Weisheit} und
\enquote{prudentia} mit \enquote{Klugheit} zu übersetzen. Siehe dazu
\cite[][152]{Schwaiger:KlugheitbeiKant2002}.} so liegt es nahe, für die Bildung
neben Klugheit auch Weisheit als Ziel zu fordern. Aber wenn man dann denkt, man müsste
nur die technische Ausbildung um die Vermittlung ethischer Erkenntnisse
erweitern, also die Klugheit mit der Kultur und die Weisheit mit der Moral
identifiziert, bleibt man hinter der Systematik
\name[Immanuel]{Kant}s zurück. Weisheit umfasst mehr als das Wissen um das moralisch
Richtige; und Klugheit geht nicht in technischem Wissen und Können auf. Um
besser sehen zu können, welche Kompetenzen und welches Wissen zu einem
selbstbestimmten Leben nötig sind, lohnt daher eine genauere Untersuchung von
Begriffen wie \enquote{pragmatisch}, \enquote{klug}, \enquote{technisch} und
\enquote{praktisch}. Die leitende Frage ist: Welche Funktionen können
Erkenntnisse bei unserer Handlungsorientierung übernehmen und welche Relevanz
hat dies für \name[Immanuel]{Kant}s Aufklärungsprogramm?

\authorfullcite{Meier:Vernunftlehre1752} unterscheidet zwischen praktischen,
spekulativen und theoretischen Erkenntnissen, um ihre Einbindung in unser
Handeln zu thematisieren. Zunächst heißt eine Erkenntnis nach \authorcite{Meier:Vernunftlehre1752} praktisch (im
Gegensatz zu spekulativ), wenn sie überhaupt geeignet ist, eine
handlungsorientierende Funktion zu übernehmen.
\begin{quote}
  \ori{Eine Erkenntniss ist praktisch} (cognitio practica), in so ferne sie uns
  auf eine merkliche Art bewegen kann, eine Handlung zu thun oder zu lassen.
  Alle vollkommenere Erkenntniss, die nicht praktisch ist, wird \ori{eine
  speculativische Erkenntniss} (cognitio speculativa, speculatio) genennet. Alle
  gelehrte Erkenntniss ist demnach entweder praktisch oder
  speculativisch.\footnote{\cite[][61]{Meier:AuszugausderVernunftlehre1752},
  \cite[][XVI: 516.20--22, 517.23--24]{Kant:GesammelteWerke1900ff.}.}
\end{quote}
Er verwendet den Ausdruck \enquote{praktisch} aber auch in einem anderen
Sinne, nämlich als Gegenbegriff zu \enquote{theoretisch}; und hier heißt
\enquote{praktisch} so viel wie \enquote{vorschreibend}:
\begin{quote}
  Eine Erkenntniss, in welcher wir uns vorstellen, dass etwas gethan oder
  gelassen werden solle, wird auch praktisch genannt, in so ferne sie \ori{der
  theoretischen Erkenntniss} (cognitio theoretica, theoria) entgegengesetzt
  wird, der Erkenntniss, die uns nicht vorstellt, dass etwas gethan oder
  gelassen werden solle. Alle gelehrte Erkenntniss ist entweder praktisch oder
  theoretisch, und beide Arten gehören entweder zu der praktischen oder
  spekulativischen
  Erkenntniss[.]\footnote{\cite[][\pno~61\,f.]{Meier:AuszugausderVernunftlehre1752},
  \cite[][XVI: 517.25--31]{Kant:GesammelteWerke1900ff.}.}
\end{quote}
Da diese doppelte Bedeutung von \enquote{praktisch} nicht ohne Härte ist (es
gibt dann praktische Erkenntnisse, die wiederum nicht praktisch sind), können
wir von \enquote{pragmatisch} als Gegenbegriff zu \enquote{spekulativ} sprechen.
Diesen Vorschlag macht \name[Immanuel]{Kant},\footnote{Siehe die Randbemerkung
zur Begriffseinführung in \authorcite{Meier:Vernunftlehre1752}s Lehrbuch in
\cite[][\nopp 2795]{Kant:Reflexionen1900ff.}, \cite[][XVI:
516.8-9]{Kant:GesammelteWerke1900ff.}.} aber dennoch dürfen wir uns nicht darauf
verlassen, dass es sich um eine terminologische Festlegung handelt, die in
seinen Schriften Bestand hat.\footnote{Zum Wandel der Begriffsverwendungen bei
\name[Immanuel]{Kant} siehe
\cite[][113--141]{Schwaiger:KategorischeundandereImperative1999}.} Zumindest
handelt es sich nicht um die einzige Bedeutung dieses Ausdrucks bei
\name[Immanuel]{Kant}; wir können sie die weite Bedeutung nennen und werden
gleich noch eine engere Bedeutung
finden.\footnote{\phantomsection\label{Fussnote:DoppelteBedeutungvonPragmatisch}Alix
\textcite[][69]{Cohen:KantandtheHumanSciences2009} macht eine solche doppelte
Bedeutung des Wortes \enquote{pragmatisch} aus, wohingegen Allen
\textcite[39--42]{Wood:KantandtheProblemofHumanNature2003} sogar vier
verschiedene Bedeutungen unterscheidet. Auf eine vergleichbare Doppeldeutigkeit
von \enquote{praktisch} verweist
\textcite[vgl.][77]{Louden:TheSecondPartofMorals2003}. Ich werde im folgenden
von \enquote{pragmatisch im weite(re)n Sinne} sprechen, wenn es allgemein um
handlungsorientierende Erkenntnisse geht, und von \enquote{pragmatisch im
enge(re)n Sinne} bei Erkenntnissen der Klugheit.}


Eine Erkenntnis heißt also pragmatisch im weiteren Sinne, wenn ich meine
Handlungen an ihr ausrichte bzw.
sie dazu geeignet ist, dass ich das tue. Da es unseren Erkenntnissen in gewisser
Hinsicht äußerlich ist, ob jemand sein Handeln an ihnen ausrichtet, sagt
\name[Immanuel]{Kant}, dass die Unterscheidung \enquote{pragmatisch}/\enquote{spekulativ}
nicht die Erkenntnisse, sondern ihren Gebrauch
beschreibe.\footnote{\cite[Vgl.][\nopp 2802]{Kant:Reflexionen1900ff.},
\cite[][XVI:
519.15--17]{Kant:GesammelteWerke1900ff.}: \enquote{Erkenntnis ist
(\textsuperscript{g} Satze sind) entweder practisch oder theoretisch. Der
Gebrauch der Erkenntnis entweder \sout{theoretisch} practisch oder speculativ.}} Als Handlungsorientierungen
kommen Aussagen unterschiedlichster Art in Betracht, beispielsweise
\enquote{Mein Wecker klingelt}, \enquote{Ich habe Hunger} oder \enquote{Der
Termin rückt näher}, aber auch \enquote{Beeile dich!}, \enquote{Du solltest ihn
nicht verletzen!} oder \enquote{Wenn du einen Pizzateig ansetzen möchtest,
solltest du erst das Mehl in eine Schüssel sieben}. Die Aussagen der
pragmatischen Anthropologie (und auch der physischen Geographie) sollen stets
zur Handlungsorientierung geeignet sein. Aber sie haben kein Monopol auf
Brauchbarkeit, sondern bilden nur den geeigneten Ausgangspunkt zur vernünftigen
und aufgeklärten Anwendung weiterer Erkenntnisse.\footnote{Philosophie ist
\enquote{die Wissenschaft von der Beziehung \myemph{aller} Erkenntnis auf die
wesentlichen Zwecke der menschlichen Vernunft}
\mkbibparens{\cite[][B 867]{Kant:KritikderreinenVernunft2003},
\cite[][III: 542.26--28]{Kant:GesammelteWerke1900ff.}}. Siehe auch
\cite[][A 23]{Kant:ImmanuelKantsLogik1977}, \cite[][IX:
23.30--24.2]{Kant:GesammelteWerke1900ff.}.} Auch und gerade technisches Wissen
ist handlungsorientierend.
Spekulative Erkenntnisse auf der anderen Seite sind müßig, ohne Relevanz für
unser Leben, in Wahrheit aber auch sehr selten. Denn letztlich kann fast jede
Erkenntnis eine mehr oder minder stark ausgeprägte
handlungsorientierende Funktion übernehmen, wenn wir auch manchmal
dazu neigen, diese zu übersehen.\footnote{\authorcite{Meier:Vernunftlehre1752} verwendet die etwas
hölzern klingende Form \enquote{speculativisch}, die ich hier dem
Sprachgebrauch \name[Immanuel]{Kant}s folgend in \enquote{spekulativ} umwandle.
\cite[Vgl.][\S~219]{Meier:AuszugausderVernunftlehre1752}, \cite[][XVI:
520.14--17]{Kant:GesammelteWerke1900ff.}: \enquote{Keine wahre gelehrte
Erkenntniss ist ihrer Natur nach speculativisch, sondern nur um des Mangels der
Einsicht eines Gelehrten willen, welcher ihren Zusammenhang mit dem Verhalten
des Menschen nicht einsehen kann, oder nicht einsehen will. In dem letzten Fall
beschimpft sich der Gelehrte selbst.}} Dabei können Erkenntnisse aus
unterschiedlichsten Gründen und in verschiedensten Hinsichten pragmatisch sein,
beispielsweise wenn sie zur Erlangung oder Erhaltung
unserer Zufriedenheit dienen oder Einfluss auf unser (gutes)
Verhalten
haben.\footnote{\cite[Vgl.][\S\S~222--225]{Meier:AuszugausderVernunftlehre1752},
\cite[XVI: 520--522]{Kant:GesammelteWerke1900ff.}.}

Nun sollten wir vielleicht eher von präskriptiven und deskriptiven Urteilen
und Äußerungen sprechen, um \authorcite{Meier:Vernunftlehre1752}s Einteilung in praktische und
theoretische Erkenntnisse zu artikulieren. Als präskriptiv zählen beispielsweise
Imperative der Ethik oder des Rechts oder auch göttliche Gebote und Befehle. Es versteht sich,
dass der Bereich des Präskriptiven viel kleiner ist als der des Pragmatischen
(im weiten Sinne) und dass es keinen Mangel einer Erkenntnis darstellt, nicht
präskriptiv zu sein. Die meisten unserer Urteile und Äußerungen sind es nicht
und es ist offensichtlich, dass die Klasse handlungsorientierender Erkenntnisse
nicht mit der Klasse präskriptiver Äußerungen zusammenfällt.

Eine begriffliche Neuerung \name[Immanuel]{Kant}s gegenüber
\authorcite{Meier:Vernunftlehre1752} wird für die hier interessierende
Fragestellung wichtig werden: Mit \name[Immanuel]{Kant} wird der Begriff des
Imperativs zum zentralen Mittel der Beschreibung von Handlungsweisen endlicher
Wesen (\authorcite{Meier:Vernunftlehre1752} spricht zwar von einem
\singlequote{Sollen}, aber noch nicht von
\singlequote{Imperativen}).\footnote{\cite[Vgl.][164]{Schwaiger:KategorischeundandereImperative1999}:
\enquote{Zu Kants erfolgreichsten Neuschöpfungen auf dem Gebiet der ethischen,
ja der philosophischen Terminologie überhaupt, bei der wiederum die
Auseinandersetzung mit Baumgartens Lehrbüchern eine entscheidende Rolle gespielt
hat, zählt der Begriff \singlequote{Imperativ}.}} \name[Immanuel]{Kant} greift
dabei \authorcite{Baumgarten:Metaphysica---Metaphysik2011}s Unterscheidung von
Notwendigkeit (\singlequote{\emph{necessitas}}) und Nötigung
(\singlequote{\emph{necessitatio}})\footnote{\cite[Vgl.][]{Baumgarten:Metaphysica---Metaphysik2011},
\S~102 (\enquote{\emph{necessitas}}),
\S~701 (\enquote{\emph{necessitatio}}).} auf, wenn er den Imperativ als etwas
bestimmt, dem nur endliche Wesen mit ihrer Diskrepanz zwischen Sollen und Wollen
unterliegen.\footnote{\cite[Vgl.][164--167]{Schwaiger:KategorischeundandereImperative1999}.
\enquote{Baumgarten seinerseits hat letzteren Begriff gegenüber Wolff neu
eingeführt und damit den zwingenden Charakter moralischer Vorschriften
wesentlich stärker zur Geltung gebracht}
\parencite[][167]{Schwaiger:KategorischeundandereImperative1999}.} Moralisches
Sollen gilt für alle Wesen mit Notwendigkeit, aber es nötigt nur endliche Wesen.
Es zeichnet uns als endliche Wesen aus, dass wir nicht automatisch gemäß
vernünftiger Einsicht handeln, weil die vernünftige Einsicht stets in Konkurrenz
zu unseren unmittelbaren Neigungen steht.\footnote{Siehe oben, Kap.
\ref{subsubsection:DieEndlichkeitdesWillens}.} Was
\authorcite{Meier:Vernunftlehre1752} als praktische Erkenntnisse den
theoretischen entgegenstellt, nennt \name[Immanuel]{Kant} Imperative, die uns
als nötigend gegenüberstehen.

Bei \name[Immanuel]{Kant} sind die \enquote{spekulativen} Erkenntnisse neben den
\enquote{Naturerkenntnissen} eine Untergruppe der theoretischen
Erkenntnisse,\footnote{\cite[Vgl.][B 662f.]{Kant:KritikderreinenVernunft2003}, \cite[][III:
422.16--20]{Kant:GesammelteWerke1900ff.}:
\enquote{Eine theoretische Erkenntnis ist \ori{spekulativ}, wenn sie auf einen
Gegenstand, oder solche Begriffe von einem Gegenstande, geht, wozu man in keiner
Erfahrung gelangen kann. Sie wird der \ori{Naturerkenntnis} entgegengesetzt,
welche auf keine anderen Gegenstände oder Prädikate derselben geht, als die in
einer möglichen Erfahrung gegeben werden können.}} während bei
\authorcite{Meier:Vernunftlehre1752} auch spekulative Erkenntnisse denkbar sind,
die nicht theoretisch
sind\footnote{\cite[Vgl.][62]{Meier:AuszugausderVernunftlehre1752}, \cite[][XVI:
517.29--31]{Kant:GesammelteWerke1900ff.}.}.
Die Einteilung in theoretische und praktische Erkenntnisse scheint er zu
übernehmen, insofern er zumindest an manchen Stellen ebenfalls das \emph{Sollen}
als Charakteristikum praktischer Erkenntnis herausstellt.\footnote{\cite[Siehe
z.\,B.][B 661]{Kant:KritikderreinenVernunft2003}, \cite[][III:
421.17--19]{Kant:GesammelteWerke1900ff.}.} \emph{Prima facie} ließe sich daraufhin
vermuten, dass \name[Immanuel]{Kant} die praktische Erkenntnis mit dem
Moralischen identifizierte, zumal das Praktische bei \name[Immanuel]{Kant} von
Imperativen und einem Sollen
handelt.\footnote{\enquote{Daß diese Vernunft
nun Kausalität habe, wenigstens wir uns eine dergleichen an ihr vorstellen, ist
aus den \ori{Imperativen} klar, welche wir in allem Praktischen den ausübenden
Kräften als Regeln aufgeben. Das \ori{Sollen} drückt eine Art von Notwendigkeit
und Verknüpfung mit Gründen aus, die in der ganzen Natur sonst nicht vorkommt}
\mkbibparens{\cite[][B 575]{Kant:KritikderreinenVernunft2003},
\cite[III: 371.15--17]{Kant:GesammelteWerke1900ff.}}.}

Doch auch wenn die praktische Vernunft uns endlichen Wesen generell in der Form
von Imperativen entgegentritt, darf \enquote{praktisch} nicht mit
\enquote{moralisch} identifiziert werden, sondern umfasst auch technische und
(in einem noch zu erläuternden engeren Sinne) pragmatische
Erkenntnisse.\footnote{\cite[Vgl.][B~828]{Kant:KritikderreinenVernunft2003}, \cite[III: 520.1-16]{Kant:GesammelteWerke1900ff.}. Mit den moralischen Gesetzen
koextensiv sind nicht die praktischen, sondern die \emph{reinen} praktischen
Gesetze.} In der \titel{Kritik der reinen Vernunft} schreibt
\name[Immanuel]{Kant}: \enquote{Praktisch ist alles, was durch Freiheit möglich
ist.}\footnote{\cite[B~828]{Kant:KritikderreinenVernunft2003}, \cite[III:
520.1]{Kant:GesammelteWerke1900ff.}. Siehe zur Unterscheidung von praktischer
und theoretischer Vernunft:
\cite{Engstrom:KantsDistinctionbetweenTheoreticalandPracticalKnowledge2002}.}
Und durch Freiheit ist auch aus \name[Immanuel]{Kant}s Sicht nicht nur
moralisches Handeln möglich.\footnote{Siehe oben, Kapitel \ref{Abschnitt:MaximenHandlungenFreiheit},
S.~\pageref{Abschnitt:MaximenHandlungenFreiheit}f.} Und in der \titel{Kritik
der Urteilskraft} heißt es:
\begin{quote}
[A]lles, was als durch einen Willen möglich (oder notwendig) vorgestellt wird,
heißt praktisch-möglich (oder -notwendig); zum Unterschiede von der physischen
Möglichkeit oder Notwendigkeit einer Wirkung, wozu die Ursache nicht durch
Begriffe (sondern, wie bei der leblosen Materie, durch Mechanismen und bei
Tieren durch Instinkt) zur Kausalität bestimmt
wird.\footnote{\cite[][B xii\,f.,]{Kant:KritikderUrteilskraft2009}
\cite[][V: 172.6--11]{Kant:GesammelteWerke1900ff.}.}
\end{quote}
Die Prinzipien, die die praktische Möglichkeit und Notwendigkeit bestimmen,
unterteilen sich daraufhin in \emph{technisch-praktische} und
\emph{moralisch-praktische}
Prinzipien -- je nachdem, ob ein Naturbegriff oder ein Freiheitsbegriff die
\singlequote{Kausalität} unseres Willens bestimmt. Die technisch-praktischen Prinzipien
beziehen sich naheliegenderweise auf hypothetische, die moralisch-praktischen
Prinzipien auf kategorische Imperative. Nur die moralisch-praktischen Prinzipien
gehören zur praktischen Philosophie oder \enquote{Sittenlehre}; die
technisch-praktischen Prinzipien hingegen gehören nach Auskunft der
\titel{Kritik der Urteilskraft} der theoretischen Philosophie
an.\footnote{\cite[Vgl.][B
xiii]{Kant:KritikderUrteilskraft2009}, \cite[][V:
172.14--22]{Kant:GesammelteWerke1900ff.}. Es ist daher zu ungenau, wenn
\authorfullcite{Fonnesu:KantspraktischePhilosophieunddieVerwirklichungderMoral2004}
behauptet, dass \enquote{das Praktische für \name[Immanuel]{Kant} mit der
Moral zusammen[fällt}
\parencite[][49]{Fonnesu:KantspraktischePhilosophieunddieVerwirklichungderMoral2004}.
Die praktische \emph{Philosophie} fällt mit der Moral zusammen, nicht aber der
Gesamtbereich der praktischen Erkenntnisse.}

\name[Immanuel]{Kant} nennt in der \titel{Grundlegung zur Metaphysik der Sitten}
nicht drei Arten von handlungsorientierenden Erkenntnissen, die er als
Imperative bezeichnet: problematische Imperative als \enquote{\ori{Regeln} der
Geschicklichkeit}, assertorische Imperative als \enquote{\ori{Ratschläge} der Klugheit} und
kategorische Imperative\footnote{\name[Immanuel]{Kant} spricht kategorische
Imperative an verschiedenen Stellen auch im Plural an, was zunächst verwirrend
erscheinend mag, aber recht plausibel ist, wenn man beachtet, dass \emph{der}
kategorische Imperativ (im Singular) Maximen als notwendig (geboten) ausweist,
die selbst wieder Imperative sind (uns nötigen) und kategorisch gelten.} als
\enquote{\ori{Gebote (Gesetze)} der Sittlichkeit.}\footnote{\cite[][BA
40--43]{Kant:GrundlegungzurMetaphysikderSitten1965}, \cite[IV:
414.32--416.20]{Kant:GesammelteWerke1900ff.}.}

Der kategorische Imperativ ist unabhängig von einem Zweck, da er \enquote{eine
Handlung als für sich selbst, ohne Beziehung auf einen andern Zweck, als
objektiv-notwendig
vorstellt[.]}\footnote{\cite[][BA~39]{Kant:GrundlegungzurMetaphysikderSitten1965},
\cite[][IV: 414.16--17]{Kant:GesammelteWerke1900ff.}.}. Das heißt nicht, dass
von Zwecken nicht die Rede sein könne, denn erstens fordert die
Selbstzweckformel des Kategorischen Imperativs, die \singlequote{Menschheit} in
der Person eines jeden Menschen als Zweck zu
betrachten,\footnote{\cite[Vgl.][BA~66f.]{Kant:GrundlegungzurMetaphysikderSitten1965},
\cite[][IV: 429.10--12]{Kant:GesammelteWerke1900ff.}.} und zweitens ist uns
durch die Moral das höchste Gut in der Welt als moralisch-notwendiger Zweck
vorgestellt\footnote{Siehe bspw. \cite[][B
842--847]{Kant:KritikderreinenVernunft2003}, \cite[][III:
528.13--531.23]{Kant:GesammelteWerke1900ff.};
\cite[][A 219--223]{Kant:KritikderpraktischenVernunft1974}, \cite[][V:
122.4--16]{Kant:GesammelteWerke1900ff.}.}.
Der kategorische Imperativ \emph{bestimmt} Zwecke, aber er \emph{setzt} keine Zwecke
\emph{voraus}.


Die problematischen und assertorischen Imperative wiederum fasst
\name[Immanuel]{Kant} zu der Gattung der hypothetischen Imperative zusammen. Sie
\enquote{stellen die praktische Notwendigkeit einer möglichen Handlung als
Mittel, zu etwas anderem, was man will (oder doch möglich ist, daß man es
wolle), zu gelangen, vor.}\footnote{\cite[][BA
39]{Kant:GrundlegungzurMetaphysikderSitten1965}, \cite[][IV:
414.13--15]{Kant:GesammelteWerke1900ff.}.} \name[Immanuel]{Kant}
implementiert zur Unterteilung wie so oft Vokabular, das eigentlich in der Logik
beheimatet ist: Sind die Zwecke bloß möglich, heißt der Imperativ
\enquote{problematisch} und bezeichnet eine Regel der
Geschicklichkeit, sind sie tatsächlich bei jedem Menschen vorhanden, dann heißt
er \enquote{assertorisch} und bezeichnet eine Regel der Klugheit. Der einzige
Zweck wiederum, den man als bei jedem Menschen vorhanden voraussetzen kann, ist
die \enquote{Glückseligkeit}.\footnote{\enquote{Es ist gleichwohl ein Zweck, den
man bei allen vernünftigen Wesen (so fern Imperative auf sie, nämlich als
abhängige Wesen, passen) als wirklich voraussetzen kann, und also eine Absicht,
die sie nicht etwa bloß haben \ori{können}, sondern von der man sicher
voraussetzen kann, daß sie solche insgesamt nach einer Naturnotwendigkeit
\ori{haben}, und das ist die Absicht auf \ori{Glückseligkeit}}
\mkbibparens{\cite[][BA 42]{Kant:GrundlegungzurMetaphysikderSitten1965};
\cite[][IV: 415.28--33]{Kant:GesammelteWerke1900ff.}}.}


Regeln der Geschicklichkeit heißen \emph{technisch} und gehören zur
\emph{Kunst}, Gebote der Sittlichkeit heißen \emph{moralisch} und die Ratschläge der Klugheit sind
diejenigen Imperative, die im engeren Sinne\footnote{Siehe Fußnote
\ref{Fussnote:DoppelteBedeutungvonPragmatisch} aus Seite
\pageref{Fussnote:DoppelteBedeutungvonPragmatisch}. Mit diesem Verweis auf die
Klugheit glaubt \name[Immanuel]{Kant} den Gebrauch des Wortes
\enquote{pragmatisch} charakterisieren zu können.
\cite[Vgl.][BA~4]{Kant:GrundlegungzurMetaphysikderSitten1965}, \cite[IV:
417.32--37]{Kant:GesammelteWerke1900ff.}: \enquote{Mich deucht, die eigentliche
Bedeutung des Worts \ori{pragmatisch} könne so am genauesten bestimmt werden.
{\punkt} Pragmatisch ist eine \ori{Geschichte} abgefaßt, wenn sie \ori{klug}
macht, d.\,i.\ die Welt belehrt, wie sie ihren Vorteil besser, oder wenigstens
eben so gut, als die Vorwelt, besorgen könne.}} \emph{pragmatisch} genannt
werden und zur \singlequote{\emph{Wohlfahrt}}
gehören.\footnote{\cite[Vgl.][BA~44]{Kant:GrundlegungzurMetaphysikderSitten1965},
\cite[IV: 416.28--417.2]{Kant:GesammelteWerke1900ff.}: \enquote{Man könnte die
ersteren Imperative auch \ori{technisch} (zur Kunst gehörig), die zweiten
\ori{pragmatisch} (zur Wohlfahrt), die dritten \ori{moralisch} (zum freien
Verhalten überhaupt, d.\,i.\ zu den Sitten gehörig) nennen.}} Die technischen
Regeln der Geschicklichkeit als Paradigmen der hypothetischen Imperative und die
moralischen Gebote der Sittlichkeit als Instanzen des kategorischen Imperativs
sind weithin bekannt. Anders verhält es sich mit den pragmatischen Ratschlägen
der Klugheit; dabei sind sie von erheblicher Bedeutung für
\name[Immanuel]{Kant}s
Entwicklung.\footnote{\cite[Vgl.][149]{Schwaiger:KlugheitbeiKant2002}.}
\phantomsection\label{Absatz:Weltklugheit}Was also ist Klugheit?
\name[Immanuel]{Kant} schreibt:
\begin{quote}
  Das Wort Klugheit wird in zwiefachem Sinn genommen, einmal kann es den Namen
  Weltklugheit, im zweiten den der Privatklugheit führen. Die erste ist die
  Geschicklichkeit eines Menschen, auf andere Einfluß zu haben, um sie zu seinen
  Absichten zu gebrauchen. Die zweite die Einsicht, alle diese Absichten zu
  seinem eigenen daurenden Vorteil zu vereinigen. Die letztere ist eigentlich
  diejenige, worauf selbst der Wert der erstern zurückgeführt wird, und wer in
  der erstern Art klug ist, nicht aber in der zweiten, von dem könnte man besser
  sagen: er ist gescheut und verschlagen, im ganzen aber doch
  unklug.\footnote{\cite[BA~42]{Kant:GrundlegungzurMetaphysikderSitten1965},
  \cite[IV: 416.30--37]{Kant:GesammelteWerke1900ff.}.}
\end{quote}
Der grundlegende Begriff ist also der der Privatklugheit, welche auf
Glückseligkeit geht. Ihr Ziel ist durch unser je eigenes Streben nach einer
dauerhaften und anhaltenden Zufriedenheit bestimmt. \name[Immanuel]{Kant}
spricht auch von \enquote{praktischen Klugheitsregeln nach dem Prinzip der
Selbstliebe}\footnote{\cite[][\S~91]{Kant:KritikderUrteilskraft2009}, \cite[][V:
470.9--10]{Kant:GesammelteWerke1900ff.}.} und der Glückselig als der
\enquote{Zufriedenheit mit seinem Zustande, sofern man der Fortdauer derselben
gewiß ist}\footnote{\cite[][A 16]{Kant:DieMetaphysikderSitten1977Tugendlehre},
\cite[][VI: 387.26--27]{Kant:GesammelteWerke1900ff.}. Siehe auch
\cite[][BA 1\,f.,]{Kant:GrundlegungzurMetaphysikderSitten1965}
\cite[][IV: 393.14--16]{Kant:GesammelteWerke1900ff.};
\cite[][A 168\,f.,]{Kant:DieMetaphysikderSitten1977Tugendlehre}
\cite[][VI: 480.23--25]{Kant:GesammelteWerke1900ff.};
\cite[][A 80]{Kant:DieReligioninnerhalbderGrenzenderblossenVernunft1977},
\cite[][VI: 67.20--23]{Kant:GesammelteWerke1900ff.}.}.
Es scheint mir hilfreich zu sein, den Begriff \enquote{Glückseligkeit} bei \name[Immanuel]{Kant} durch einen solchen Ausdruck wie \enquote{lang anhaltende (nicht nur kurzfristige) umfassende Zufriedenheit} zu erläutern. Es geht um einen Zustand, der erstens durch empirische Zufriedenheit bestimmt ist, aber zweitens nicht die
kurzfristige Befriedigung von Begierden meint, sondern auf die Ausgestaltung
längerer Lebensabschnitte (letztlich das gesamte Leben) zielt.

Damit ist der Begriff \enquote{Glückseligkeit} klarer, aber auch enger als der
Begriff der Eudaimonia bei \singlename{Aristoteles}\footnote{Hier bezogen auf
das erste Buch der \titel{Nikomachischen Ethik}, siehe
\cite[][1--12]{Aristoteles:NikomachischeEthik1972}.}, insofern
\name[Immanuel]{Kant} die Glückseligkeit eindeutig auf das somatische Wohlbefinden einschränkt. (Im Gegenzug stellt sie allerdings nicht mehr das höchste Gut dar, welches nun in der Verbindung von Glückseligkeit und Sittlichkeit
besteht.\footnote{\cite[Vgl.][B 841\,f.]{Kant:KritikderreinenVernunft2003};
\cite[][III: 527.33--528.15]{Kant:GesammelteWerke1900ff.}.}) Zu bestimmen und
erfolgreich anzuwenden sind nur noch die Mittel zum Erreichen dieses Ziels.
Zumindest der späte \name[Immanuel]{Kant} scheint den Ansatz zu verwerfen,
wonach die Klugheit sowohl die Inhalte des Glücks als auch die Mittel, diese zu
erreichen, bestimmt, weil durch das allen gemeinsame Ziel einer anhaltenden
\singlequote{Glückseligkeit} die Frage nach dem Inhalt bereits beantwortet
ist.\footnote{\cite[Vgl.][\pno~185f.]{Schwaiger:KategorischeundandereImperative1999}.
Nach \authorcite{Schwaiger:KategorischeundandereImperative1999} stellt sich
diese Doppelfrage beim späten \name[Immanuel]{Kant} nur noch hinsichtlich des
Glücks als \emph{summum bonum}.} Allerdings
sei es
\begin{quote}
  ein Unglück, daß der Begriff der Glückseligkeit ein so unbestimmter ist, daß,
  obgleich jeder Mensch zu dieser zu gelangen wünscht, er doch niemals bestimmt
  und mit sich selbst einstimmig sagen kann, was er eigentlich wünsche und
  wolle.\footnote{\cite[BA~46]{Kant:GrundlegungzurMetaphysikderSitten1965},
  \cite[IV: 418.1--4]{Kant:GesammelteWerke1900ff.}.}
\end{quote}
So könnte man auch sagen, dass durch die Unbestimmtheit des Begriffs der
Glückseligkeit die Frage nach dem Inhalt des Glücks weiter besteht. Dass
\name[Immanuel]{Kant} von der Glückseligkeit als einem
\singlequote{unbestimmten} Begriff spricht, lässt zunächst vermuten, dass er
doch den Gehalt des Begriffs und damit seine Inhalte für fraglich hält. Sachlich
ist der Punkt jedoch klar: Wenn wir auch die Mittel, dauerhafte Zufriedenheit zu
erlangen, als \emph{Inhalte} der Glückseligkeit bezeichnen können, steht doch
deren Ausrichtung auf die dauerhafte Zufriedenheit längst fest. Dieses
langfristige Wohlbefinden \emph{ist} die Glückseligkeit, wenngleich wir dazu
neigen, die je individuell als geeignet angesehenen Mittel als Inhalte der
Glückseligkeit anzusprechen.

Das Problem, mit dem wir stets konfrontiert sind, besteht darin, dass wir
verschiedene Zwecke zu verfolgen willens sind, die sich gegenseitig behindern. Wir wollen etwa
\emph{sowohl} beruflich erfolgreich sein, \emph{als auch} viel Freizeit haben
und stellen fest, dass beides zugleich nicht realisierbar ist. In diesem Fall --
so scheint es -- müssen wir nach dem Inhalt des \singlequote{Glücks} oder der
Glückseligkeit fragen. Allerdings ließe sich ebenso sagen, dass bloß die Mittel
fraglich sind, mit denen wir dauerhafte Zufriedenheit erlangen können. Wir
wissen eben nicht im Vorhinein, ob wir nach dem jahrelangen zielstrebigen
Verfolgen unserer Karriere zufrieden sind oder ausgebrannt. Noch weniger wissen
wir jemals, ob der andere Weg als der, den wir wählten, uns mehr an
Zufriedenheit eingebracht hätte. Diese dauerhafte Zufriedenheit ist letztlich
das einzige Ziel, zu dem die Klugheit die Mittel sucht.\footnote{\cite[Vgl.][185]{Schwaiger:KategorischeundandereImperative1999}: \enquote{[U]m
die Zeit der \ori{Grundlegung} ist der Gedanke, daß die Klugheitslehre neben den Mitteln auch
den Zweck der Glückseligkeit bestimmen müsse, völlig zurückgetreten. Die
Möglichkeit eines Irrtums bei der inhaltlichen Füllung des Glücksbegriffs gerät
aus dem Blick; ein bloßes Scheinglück wird mit keinem Wort mehr erwähnt. Die
Klugheit macht nur noch die Mittel ausfindig; allein der Sittlichkeit kommt es
zu, auch den Zweck zu bestimmen.}} Diese Mittel wiederum lassen sich als
(subalterne) Zwecke ansehen. Wir verfolgen sie nicht um ihrer selbst willen,
sondern weil wir sie für geeignete Mittel zur dauerhaften Zufriedenheit ansehen.
Dauerhaftes Glück jedoch wird um seiner selbst willen angestrebt -- es ist ein
letzter Zweck.



Glückseligkeit hat einen Sonderstatus unter unseren Zwecken, weil jeder Mensch
als endliches Vernunftwesen danach strebt. Denn jeder Mensch versucht, dauerhaft
glücklich zu sein, wenn auch die Vorstellungen auseinandergehen, wie dieser Zustand zu
erreichen ist. Dies liegt daran, dass der jeweils geeignete Weg zur
Glückseligkeit von biographischen Zufälligkeiten und der jeweils eigenen Wunsch-
und Bedürfnisstruktur abhängig ist. Was den einen Menschen glücklich macht, das
verursacht bei dem anderen negative Gefühle wie Stress oder Langeweile.
Der eine fühlt sich eben beim Angeln wohl, der andere auf einer anspruchsvollen
Hochtour. Selbstredend besitzen wir auch Wissen darüber, was Menschen ganz
allgemein glücklich oder unglücklich macht, und insofern ist es durchaus
möglich, allgemeines pragmatisches \emph{Wissen} zu haben, welches sich in
Ratschlägen artikuliert.\footnote{\cite[Vgl.][98]{Brandt:KlugheitbeiKant2005}:
\enquote{Obwohl das Glücksziel am Horizont des menschlichen Lebens, für dessen Erreichung die
Klugheit ihre Ratschläge in hypothetischen Imperativen liefert, zwar notwendig,
aber nur unbestimmt ist und bei jedem Menschen wechselt, gibt es ein weites
Klugheitsfeld im privaten und öffentlichen Handeln, das \name[Immanuel]{Kant} in
seiner \singlequote{pragmatischen Anthropologie} durchleuchtet; sie ist eine empirische
Klugheitslehre, die wegen ihrer Fundierung in der Natur des Menschen, also der
bloßen Erfahrung, nicht zur eigentlichen kritischen Philosophie als einer
Ver\-nunft\-er\-kennt\-nis aus Begriffen gehört; aber immerhin, Kant bezeichnet
sie als Wissenschaft.}} Der Verzicht auf harte Drogen ist beispielsweise für jeden
Menschen im Interesse seiner eigenen langfristigen Zufriedenheit empfehlenswert,
da wir begründeter Weise davon
ausgehen, dass der aus längerem Drogenkonsum resultierende Zustand für niemanden
wünschenswert ist.\footnote{Dieser Ratschlag ist nicht zu verwechseln mit einem
möglicherweise zu begründenden moralischen Verbot, sich selbst in einen
Rauschzustand zu versetzen. Ein solches Verbot ließe sich etwa dadurch
begründen, dass man darauf verweist, dass Drogenkonsum die Fähigkeit zu
moralischem Handeln reduziert, indem die Abhängigkeit zu
Beschaffungskriminalität verleitet oder der Rausch unmoralisches oder gar
kriminelles Handeln hervorruft. \name[Immanuel]{Kant} äußert sich dazu in der
\titel{Metaphysik der Sitten} \mkbibparens{\cite[vgl.][A
80--82]{Kant:DieMetaphysikderSitten1977Tugendlehre};
\cite[][VI: 427.1--428.26]{Kant:GesammelteWerke1900ff.}}.
Auffällig ist, dass zuverlässige Ratschläge oft negativ formuliert
sind, uns also sagen, was im Interesse unserer dauerhaften Zufriedenheit zu
unterlassen oder \singlequote{\emph{unklug}}
ist. \cite[Vgl.][190]{Schwaiger:KategorischeundandereImperative1999}:
\enquote{Im allgemeinen werden Ratschläge für das Glück eher negativ als positiv gehalten
sein müssen.} Ratschläge im positiven Sinne -- die nicht Unglück verhindern,
sondern Glück herbeiführen sollen -- sind weitaus schwerer zu finden. Dennoch
gibt es sie natürlich; man denke etwa an den Ernährungshinweis,
abwechslungsreiche Nahrung mit Obst und Gemüse zu sich zu nehmen. Medizinische
Ratschläge sind in der Regel Ratschläge der Klugheit, da uns unsere Gesundheit
stets im Interesse langfristiger Glückseligkeit am Herzen liegt.}


Weil die Überzeugungskraft von pragmatischen Imperativen der Klugheit von der
Übereinstimmung mit (mehr oder minder kontingenten) subjektiven Wünschen
abhängig ist, die jeder selbst beurteilen muss, handelt es sich bei positiven
wie negativen Ratschlägen der Klugheit nicht um Gebote, sondern eben nur um
Ratschläge.\footnote{\enquote{Die \ori{Ratgebung} enthält zwar Notwendigkeit,
die aber bloß unter subjektiver gefälliger Bedingung, ob dieser oder jener
Mensch dieses oder jenes zu seiner Glückseligkeit zähle, gelten kann} (\cite[BA
44]{Kant:GrundlegungzurMetaphysikderSitten1965}, \cite[IV:
416.23-26]{Kant:GesammelteWerke1900ff.}). \cite[Vgl.
auch][189]{Schwaiger:KategorischeundandereImperative1999}:
\enquote{Nach unserem alltäglichen Verständnis gehört aber offenbar wesentlich
zu einem Ratschlag, daß er selbst dann, wenn er gut und richtig ist, nicht
übernommen werden muß, sondern verworfen werden kann. Ratschläge lassen dem
anderen die Freiheit, über sein eigenes, individuelles Glück in letzter Instanz
selbst zu entscheiden.}} Es liegen zwischen den drei Arten von Imperativen
Unterschiede in der Art der Nötigung
vor,\footnote{\enquote{Das Wollen nach diesen dreierlei Prinzipien wird auch
durch die \ori{Ungleichheit} der Nötigung des Willens deutlich unterschieden}
\mkbibparens{\cite[][BA 43]{Kant:GrundlegungzurMetaphysikderSitten1965};
\cite[][IV: 416.15--16]{Kant:GesammelteWerke1900ff.}}.} wenngleich alle drei als
Imperative uns als nötigend begegnen. Der Rat des Arztes, weniger Alkohol zu trinken, hat nicht die Verbindlichkeit eines moralischen Gebotes. Wer es für wahrscheinlich hält, dass ein Leben in geselliger Bierlaune ihn glücklicher machen wird als lang anhaltende Gesundheit, der möge diesen Weg für sich einschlagen. Eine Erfolgsgarantie hat er dabei freilich so wenig wie derjenige,
der auf feucht-fröhliche Abende um der Gesundheit willen verzichtet.
Aber er hat doch das Recht, den von ihm für richtig befundenen Weg in seinem
eigenen Leben einzuschlagen und zu versuchen, nach seiner eigenen Fa\c{c}on
glücklich zu werden.


Gegenüber Kindern freilich, denen wir keine Mündigkeit zusprechen, artikulieren
wir Imperative der Klugheit als Anweisungen, denen sie Folge zu leisten haben.
Es steht nicht in ihrer eigenen Verantwortung,
sich zwischen schulischem Erfolg und Freizeit zu entscheiden. Aber wenn jemand
mündig zu sein beansprucht, muss er im Bereich der Klugheit Ratschläge auf
eigene Verantwortung annehmen oder ablehnen.
Dadurch unterscheiden sich aus der Perspektive mündiger Menschen Ratschläge der
Klugheit von moralischen Vorschriften. Unmündig ist mindestens derjenige, der
sich in der eigenen Lebensgestaltung ausschließlich am Beispiel anderer
orientiert, der sich von Freunden, Eltern oder ganz abstrakt \singlequote{der
Gesellschaft} zu einer bestimmten Lebensweise genötigt sieht, weil er sich
Selbständigkeit nicht zutraut.

Das heißt freilich nicht, dass der mündige Mensch sich von anderen Ratschläge
der Klugheit generell verbitten könnte. Das je eigene Leben klug einzurichten,
ist eine Sache der Vernunft, schon weil es um die langfristige Zufriedenheit
geht. Zwar müssen wir oft schlicht darauf achten, was uns selbst Freude bereitet
-- wir müssen auf unser eigenes Gefühl der Lust und Unlust hören. Dennoch gibt
es vernünftige und unvernünftige Wege, mit den eigenen Präferenzen und Neigungen
umzugehen, die als solche der intersubjektiven Überprüfbarkeit und Bewertbarkeit
zugänglich sein müssen. Denn damit, dass wir sie als vernünftig oder
unvernünftig bewerten, sind wir wieder auf die erweiterte Denkungsart verwiesen.
Wir sollen uns also auch in Fragen der Klugheit, des individuellen Strebens nach
Glückseligkeit nicht als logische Egoisten, sondern als Pluralisten verhalten.
Das kompetente Verfolgen je eigener Lebensentwürfe ist eine Sache der Vernunft
und daher auf den intersubjektiven Austausch mit anderen angewiesen. In diesem
Austausch erwerben und bewahren wir die Kompetenz der eigenen vernünftigen
Lebensführung.

Klugheit als die Kompetenz der je eigenen Lebensgestaltung rückt die
alltäglichste Form unserer individuellen Freiheit in den Mittelpunkt des theoretischen
Interesses.\footnote{\cite[Vgl.][158]{Schwaiger:KlugheitbeiKant2002}:
\enquote{Mit der Forderung, daß ein kluger Mensch divergierende bis
konfligierende Absichten zu seinem dauerhaften persönlichen Vorteil vereinigen
müsse, rückt nun die Freiheit des einzelnen hinsichtlich dessen, was er aus sich
selber machen möchte, in den Mittelpunkt.}} Als Provokation kann die Implikation
verstanden werden, dass neben der Moralität und der Glückseligkeit, die als
der empirische Zustand je eigener dauerhafter Zufriedenheit verstanden wird,
kein weiterer \singlequote{Wert} des eigenen Lebens zugelassen wird. Es gibt -- so
\name[Immanuel]{Kant} -- neben Recht und Moral keine verbindlichen Maßstäbe
unserer Lebensgestaltung. Die je eigene Glückseligkeit und Moralität -- und mit
der Moralität auch die \emph{fremde} Glückseligkeit -- sind die einzigen
Maßstäbe, anhand derer sich unsere je eigene Lebensgestaltung messen lässt.


Das Wissen und die Fähigkeit, sich selbst Ziele zu
stecken und erfolgreich zu verfolgen, die nicht kurzfristige Befriedigung,
sondern anhaltendes Glück gewähren, ist die Privatklugheit. Weltklugheit ist
dagegen \enquote{die Geschicklichkeit eines Menschen, auf andere Einfluß zu
haben, um sie zu seinen Absichten zu
gebrauchen.}\footnote{\cite[BA 42]{Kant:GrundlegungzurMetaphysikderSitten1965},
\cite[IV: 416.32--33]{Kant:GesammelteWerke1900ff.}. Entgegen einem möglichen
ersten Anschein konfligiert dies nicht mit der Forderungen der Moral, andere
\enquote{jederzeit zugleich als Zweck, niemals bloß als
Mittel} \mkbibparens{\cite[BA
66\,f.,]{Kant:GrundlegungzurMetaphysikderSitten1965} \cite[IV:
429.11--12]{Kant:GesammelteWerke1900ff.}, \ohio}, zu gebrauchen, da es nicht
verlangt, jemanden als \emph{bloßes} Mittel zu betrachten. Fast jede Kooperation
beinhaltet, andere Menschen \emph{auch} als Mittel zu betrachten, aber als freie
Kooperation beinhaltet sie eben auch, dass der andere zugleich Zweck ist --
insofern weder Zwang noch Unehrlichkeit die Grundlage bilden.}
Wer Weltklugheit besitzt, der -- so könnten wir heute sagen -- ist hinreichend
sozialkompetent, um sich sicher und zielgerichtet in der menschlichen
Gesellschaft bewegen zu können. Auch hier soll es sich um eine grundlegende
Kompetenz handeln, die nicht einfach die Anwendung einer anderen Fähigkeit
beschreibt. Insbesondere von technischem Können (\singlequote{Geschicklichkeit})
ist sie unterschieden, insofern Geschicklichkeit auf Sachen, Weltklugheit jedoch
auf Personen
geht.\footnote{\cite[Vgl.][128]{Schwaiger:KategorischeundandereImperative1999}.}
Gerade bei freien Kooperationen, in denen unsere Kooperationspartner zugleich
als Zwecke angesehen werden, die weder durch Zwang noch durch Täuschung zur
Mithilfe gebracht werden, sind beide grundlegend unterschieden.\footnote{Dies
macht die Aufgabe der \emph{Zivilisierung} aus: Wer zivilisiert ist, kann seine
eigenen Ziele in Zusammenarbeit mit seinen Mitmenschen verfolgen.
\cite[Vgl.][A~23]{Kant:UeberPaedagogik1977}, \cite[IX:
450.3-5]{Kant:GesammelteWerke1900ff.}: \enquote{Muß man darauf sehen, daß der
Mensch auch \ori{klug} werde, in die menschliche Gesellschaft passe, daß er
beliebt sei, und Einfluß habe. Hierzu gehört eine gewissen Art von Kultur, die
man \ori{Zivilisierung} nennet.} Siehe auch
\cite[A~319]{Kant:AnthropologieinpragmatischerHinsicht1977},
\cite[VII: 323.21-25]{Kant:GesammelteWerke1900ff.}: \enquote{\ori{Die
pragmatische Anlage} der Zivilisierung durch Kultur, vornehmlich der Umgangseigenschaften und der
natürliche Hang seiner Art, im gesellschaftlichen Verhältnisse aus der Rohigkeit
der bloßen Sachgewalt herauszugehen und ein gesittetes (wenn gleich noch nicht
sittliches), zur Eintracht bestimmtes, Wesen zu werden, ist nun eine höhere
Stufe.}}


Aber auch Weltklugheit generiert keine neuen Ziele, sondern setzt eine
Orientierung an Glückseligkeit (und Sittlichkeit) bereits voraus. Daher nützt
sie nur demjenigen, der zugleich die nötige Privatklugheit besitzt. Wer
sozialkompetent ist und andere zu seinen Zielen gebrauchen kann, aber nicht
selbständig sein Leben auf das Ziel der Glückseligkeit auszurichten vermag
(sondern hierin von anderen abhängig bleibt), wird durch Weltklugheit nicht
mündig. Daher ist Weltklugheit in ihrem Nutzen von der Privatklugheit
abhängig.\footnote{\cite[Vgl.][BA 42]{Kant:GrundlegungzurMetaphysikderSitten1965}, \cite[][IV:
416.34--37]{Kant:GesammelteWerke1900ff.}.}\phantomsection\label{Absatz:Weltklugheit-ENDE}


Wissen, welches uns zur Orientierung im Handeln dient, nennt
\name[Immanuel]{Kant} \emph{pragmatisch} (im weiten Sinne). Zu diesem zählen 
praktische Erkenntnisse, die uns endlichen Wesen als \emph{Imperative} begegnen.
Aber auch theoretisches Wissen zählt zu den pragmatischen Erkenntnissen,
insofern wir es für unser Handeln benötigen. Wie im letzten Kapitel gesehen,
zählt dazu insbesondere das Wissen der \titel{Anthropologie in pragmatischer
Hinsicht}, welches ich als Wissen um die \emph{conditio humana} umschrieb. Aus
dem Verhältnis zwischen dem, was wir als Menschen \emph{sollen} (praktische
Erkenntnis), und dem, was wir als Menschen \emph{sind} (\emph{conditio humana}),
ergibt sich die Bestimmung des Menschen. Diese Zusammenhänge verdeutlicht
Abbildung \ref{abbildung:ErkenntnisArtenHandlungsorientierungKant}.
\begin{figure}[htb]
\begin{minipage}[t]{\textwidth}
\centering
\begin{tikzpicture}[edge from parent fork down,
level 1/.style={sibling distance=8.5cm, level distance=1.5cm},
level 2/.style={sibling distance=4.2cm, level distance=2.5cm},
level 3/.style={sibling distance=3cm, level distance=2.5cm},
level 4/.style={sibling distance=3cm, level distance=4.5cm},
every node/.style={rectangle,draw=black,fill=gray!25, thin, inner sep=0.5em, minimum size=0.5em, align=center},
edge from parent/.style={thin,draw},
mylabel/.style={draw=none, fill=none, text width=5cm,text centered, inner sep=0.5em, anchor=base} ]
\node {pragmatische Erkenntnisse (i.\,w.\,S.)}
	child {node {praktische Erkenntnisse}
		child {node[text width=3cm] {kategorische(r) Imperativ(e)}
			child {node[text width=2cm] (gebote) {Gebote der Sittlichkeit}}}
		child {node[text width=3cm] {hypothetische Imperative}
			child {node[text width=2cm] (ratschlaege) {Ratschläge der Klugheit}}
			child {node[text width=2cm] (regeln) {Regeln der Geschicklichkeit}
				child {node (bestimmung) {Bestimmung des Menschen} edge from
				parent[draw=none]}}}}
	child {node {theoretische Erkenntnisse}
	    child {node[text width=3cm] {pragmatische Anthropologie}
	    	child {node[text width=2cm, fill=none] (ch) {\emph{conditio
	    	humana}}}} child {node[text width=3cm]
	    	{physische Geographie}}} ;
\draw [->,thick,decorate,decoration={snake,post
length=1mm,amplitude=.4mm,segment length=2mm}] (ch.south) to
node[below,sloped,draw=none,fill=none] {\tiny Antagonismus} (bestimmung.north);
\draw [->] (ratschlaege.south) to
node[above,sloped,draw=none,fill=none] {\tiny zivilisieren} (bestimmung.north);
\draw [->] (regeln.south) to
node[below,sloped,draw=none,fill=none] {\tiny kultivieren} (bestimmung.north);
\draw [->] (gebote.south) to
node[above,sloped,draw=none,fill=none] {\tiny moralisieren} (bestimmung.north);
\end{tikzpicture}
  \caption{Erkenntnisarten nach Handlungsorientierung bei
  \name[Immanuel]{Kant}}\label{abbildung:ErkenntnisArtenHandlungsorientierungKant}
\end{minipage}
\end{figure}

Da es drei Arten von Imperativen gibt und die Bestimmung des Menschen der
Gegenüberstellung von praktischen Erkenntnissen und \emph{conditio humana}
entspringt, entspricht es auch der sich einstellenden Erwartungshaltung, wenn
\name[Immanuel]{Kant} drei Aspekte der Bestimmung des Menschen als Konsequenzen
aus der Anthropologie nennt. Zu Beginn dieses Kapitels hatte ich Kultivierung,
Zivilisierung und Moralisierung als Ziele benannt, die sich aus der Bestimmung
des Menschen ergeben.\footnote{Siehe
S.~\pageref{Zitat:KultivierungZivilisierungMoralisierungalsBestimmungdesMenschen}.}
In der \titel{Pädagogik} verbindet \name[Immanuel]{Kant} nun die Kultur mit der
technischen Anlage des Menschen zur \emph{Geschicklichkeit}, die Zivilisierung
mit der pragmatischen Anlage zur \emph{Klugheit} und die Moralisierung mit der
moralischen Anlage nach dem
\emph{Freiheitsprinzip}\footnote{\cite[Vgl.][A~316-321]{Kant:AnthropologieinpragmatischerHinsicht1977},
\cite[VII: 322.13--325.10]{Kant:GesammelteWerke1900ff.};
\cite[A 22\,f.,]{Kant:UeberPaedagogik1977} \cite[IX:
449.27--450.14]{Kant:GesammelteWerke1900ff.}.}, also mit den drei Arten von
Imperativen.\footnote{Die Systematisierung mag sich an unterschiedlichen Stellen
geringfügig anders darstellen: In der \titel{Anthropologie} hingegen umfasst die Kultur neben der
Entwicklung der technischen auch die Entwicklung der pragmatischen Anlage
\mkbibparens{\cite[vgl.][A 319]{Kant:AnthropologieinpragmatischerHinsicht1977},
\cite[VII: 323.21--324.11]{Kant:GesammelteWerke1900ff.}}. Man könnte somit
Kultur und Moral als die beiden entscheidenden Bildungsziele bezeichnen. Und in der \titel{Kritik
der Urteilskraft} heißt es: \enquote{Die Hervorbringung der Tauglichkeit eines
vernünftigen Wesens zu beliebigen Zwecken überhaupt (folglich in seiner
Freiheit) ist die \ori{Kultur}}
\mkbibparens{\cite[\S~83]{Kant:KritikderUrteilskraft2009}, \cite[V:
431.28--30]{Kant:GesammelteWerke1900ff.}}. Doch letztlich bleibt der
durchgängige Bezug auf die drei Arten von Imperativen bestehen.} Moral, Klugheit und
Geschicklichkeit sind die wesentlichen Zwecke der Vernunft, die zu einer freien
und selbstbestimmten Gestaltung des eigenen Lebens notwendig sind. Somit sind es
die \emph{praktischen Erkenntnisse}\footnote{In der \titel{Kritik der
Urteilskraft} hebt \name[Immanuel]{Kant} hervor, dass technische Imperative
auf Naturbegriffen beruhen und daher nicht zur praktischen, sondern zur
theoretischen Philosophie zählen. Dennoch nennt er sie
\singlequote{technisch-praktische Erkenntnisse}. Die Einteilung der Philosophie in eine
theoretische und eine praktische Philosophie stimmt also nicht mit der
Einteilung der Erkenntnisse in theoretische und praktische überein. Die
praktische Philosophie beschränkt sich auf die Behandlung der
moralisch-praktischen Erkenntnisse unter Auslassung der Imperative der
Klugheit und der Geschicklichkeit 
\mkbibparens{vgl. \cite[][xi---xii]{Kant:KritikderUrteilskraft2009},
\cite[][V: 171.4--172.22]{Kant:GesammelteWerke1900ff.}}.}, über die wir verfügen
müssen, um Bescheid zu wissen über unsere Bestimmung als Menschen. Unsere
Bestimmung ist es, gerade in Fragen der Moral und der Klugheit kompetent
urteilen und so ein selbstbestimmtes, vernunftgeleitetes Leben führen zu können.

Welche technischen Regeln uns zu wissen obliegen, ist nun tatsächlich
individuell verschieden, denn es hängt von der Verfolgung von Zwecken ab, die in zweierlei
Hinsicht kontingent sind. Sie sind erstens kontingent in dem Sinne, dass wir bei
ihnen im Gegensatz zu dem Ziel der Glückseligkeit eine tatsächliche Absicht, sie
zu erreichen nicht \emph{voraussetzen} können. Sie sind außerdem kontingent, insofern
wir bei ihnen im Gegensatz zu den Zwecken, die zu verfolgen uns die Moral
vorschreibt, nicht \emph{fordern} können, sie zu verfolgen. Die Regeln, die wir
kennen und anwenden können müssen, um etwa ein Flugzeug zu konstruieren oder am
offenen Herzen zu operieren, müssen uns nur dann interessieren, wenn wir
Ingenieure oder Ärzte sind. Wir sind nicht unmündig, wenn wir sie nicht kennen.
Pragmatische Ratschläge und moralische Gebote hingegen gehen jeden von uns an,
denn sie hängen von keinen kontingenten Zwecken ab. Die Ratschläge der Klugheit
hängen von einem Zweck ab, der nicht kontingent ist, und die Gebote der
Sittlichkeit sind von gar keinen Zwecken abhängig, sondern schreiben Zwecke vor.
Über \emph{beide} müssen wir also im Interesse mündiger Lebensgestaltung je
selbst verfügen. Wer sich die Moral oder die Ratschläge der Klugheit von anderen
vorgeben lässt, der ist unmündig. Mündigkeit in der Lebensführung lässt sich
nicht auf die Fähigkeit zu moralischem Handeln allein oder klugem Handeln allein
reduzieren. \emph{Beides} geht \emph{alle} Menschen an; denn der Mensch kann
seine Bestimmung nur erfüllen, wenn er sich in beiden Bereichen hinreichend auskennt.


\subsection{Die Fehlbarkeit der praktischen
Vernunft}\label{section:AufklaerungdesendlichenWillens}
Ich möchte im folgenden zeigen, dass einer
selbständigen, vernünftigen und mündigen Lebensführung, die sich an der
Bestimmung des Menschen orientiert, keine epistemischen Schwierigkeiten
entgegenstehen. Stattdessen finden die Hindernisse einer aufgeklärten Lebensführung
ihren Niederschlag in \name[Immanuel]{Kant}s Konzeption der Endlichkeit unseres
Willens, wie sie in Kapitel \ref{subsubsection:DieEndlichkeitdesWillens}
beschrieben wurde. Unser Wille oder unsere praktische Vernunft ist endlich,
insofern praktische Erkenntnisse in der Form von Imperativen auftreten. Sie
\emph{nötigen} uns zu einem bestimmten Verhalten, welches wir als endliche
Vernunftwesen, deren Handeln wesentlich von Neigungen bestimmt wird, nicht
ohnehin ausführen. Gerade in den aufklärungsrelevanten Bereichen praktischer und
pragmatischer Erkenntnisse ist es nicht die Endlichkeit unseres Verstandes im
Erkennen, die Mündigkeit erschwert, sondern die Endlichkeit des Willens in der
Ausführung entsprechender Handlungsvorschriften.

\name[Immanuel]{Kant} versteht unter einem Imperativ die Formel für einen
Grundsatz, der für einen Willen nötigend
ist.\footnote{\cite[Vgl.][BA 37]{Kant:GrundlegungzurMetaphysikderSitten1965},
\cite[][IV: 413.9--11]{Kant:GesammelteWerke1900ff.}.} Nur dort, wo es zu
Diskrepanzen zwischen Prinzipien und unseren tatsächlichen Handlungen kommen
kann, die darin begründet sind, dass wir unseren Neigungen folgen, sind
Imperative denkbar. Deshalb sind Gebote der Moral paradigmatische, aber nicht
die einzigen Fälle von Imperativen. Denn nicht nur im Falle des kategorischen
Imperativs gibt es solche Diskrepanzen, sondern auch bei den hypothetischen Imperativen kann es sein, dass
Prinzipien bei uns nicht handlungswirksam werden, weil antagonistische Neigungen
dies verhindern.\footnote{\cite[Vgl.][114]{Paton:TheCategoricalImperative1948}:
\enquote{The principles of goodness thus appear in our finite human condition as
principles of obligation. This is true even where the principle in question is
one of skill or rational self-love and not of morality. Men are not wholly
rational in the pursuit of happiness or even in the adoption of means to ends.}}
Wer einen Halbmarathon unter 90 Minuten laufen möchte, muss dafür mehrfach in
jeder Woche trainieren. Ein hypothetischer Imperativ besagt daher:
\enquote{Trainiere mehrfach in jeder Woche, wenn du einen Halbmarathon unter 90
Minuten laufen möchtest!} Da es ein kontingentes Ziel ist, einen Halbmarathon
unter 90 Minuten zu laufen, handelt es sich um einen problematischen Imperativ
oder eine Regel der Geschicklichkeit. Sie gilt als objektives Prinzip für
diejenige, die dieses Ziel verfolgt. Und wie jeder weiß, können Neigungen die
Ausführung dieses objektiven Prinzips punktuell oder auch dauerhaft verhindern
--  sie verhindern damit, dass ein objektives Prinzip zugleich zu einem
subjektiven Prinzip wird. Uns tritt ein solches Prinzip
daher  als \emph{nötigend} entgegen. Selbst wenn wir oft
ohne zu zögern das Training beginnt, benötigen wir doch mitunter Disziplin oder
gar Überwindung. Und hier merken wir, dass Regeln der Geschicklichkeit uns
nötigen und daher Imperative sind. Ganz analog lautet ein Ratschlag der
Klugheit: \enquote{Rauche nicht!} Da Zigarettenrauch gesundheitsschädlich ist,
wirkt er langfristig der Glückseligkeit entgegen! Deswegen ist dieser Imperativ
einer der Klugheit. Nun weiß jeder, der selbst einmal rauchte oder noch immer
raucht, wie schwer es fällt, die zu diesem Imperativ
antagonistischen Begierden zu überwinden. Der Imperativ der Klugheit ist nötigend für den, der sich an ihm
orientieren möchte, so frei er ihn auch gewählt hat. Ein Imperativ verliert
seinen nötigenden Charakter nicht dadurch, dass andere ihn nur als Ratschlag
anbringen.
Die praktische Vernunft des Menschen ist also endlich, insofern er trotz
richtiger Einsicht dabei scheitert, vernünftig zu \emph{handeln}. Zwar
\emph{wissen} wir \name[Immanuel]{Kant} zufolge gut genug, was zu tun ist, um moralisch, klug und
geschickt zu handeln; aber dennoch scheitern wir in der Ausführung, weil wir
nicht die nötige Disziplin aufbringen.
\begin{comment}
Die Fehlbarkeit endlicher Wesen in der Ausübung der praktischen Vernunft besteht
oft nicht darin, dass die entsprechenden Erkenntnisse nicht verfügbar wären, sondern
darin, dass trotz korrekter Einsicht die Ausführung unterbleibt. Den
paradigmatischen Fall finden wir bei moralischen Verfehlungen, bei denen wir
davon ausgehen können, dass kein Mangel an moralischer Einsicht vorliegt,
sondern an korrekter Umsetzung vorhandener Einsichten. Ein nicht-endlicher
(ein \singlequote{heiliger}) Wille unterliegt denselben moralischen Gesetzen,
die auch uns bekannt sind (er ist uns kognitiv möglicherweise bloß gleichwertig);
aber wenngleich er dieselben moralischen Gesetze \emph{erkennt}, so \emph{handelt}
darüber hinaus auch stets nach ihnen:
\begin{quote}
  Ein vollkommen guter Wille würde also eben sowohl unter objektiven Gesetzen
  (des Guten) stehen, aber nicht dadurch als zu gesetzmäßigen Handlungen
  \ori{genötigt} vorgestellt werden können, weil er von selbst, nach seiner
  subjektiven Beschaffenheit, nur durch die Vorstellung des Guten bestimmt 
  werden kann. Daher gelten für den \ori{göttlichen} und überhaupt für einen
  \ori{heiligen} Willen keine Imperativen; das \ori{Sollen} ist hier am
  unrechten Orte, weil das \ori{Wollen} schon von selbst mit dem Gesetz
  notwendig einstimmig
  ist.\footnote{\cite[][BA 39]{Kant:GrundlegungzurMetaphysikderSitten1965},
  \cite[][IV: 414.1--8]{Kant:GesammelteWerke1900ff.}. Siehe auch
  \cite[][\S~76]{Kant:KritikderUrteilskraft2009},
  \cite[][V: 403.20--404.16]{Kant:GesammelteWerke1900ff.}.}
\end{quote}
Imperative gibt es nur für endliche Wesen, weil ein Wesen mit einem
nicht-endlichen Willen zur Befolgung vernünftiger Grundsätze nicht aufgefordert
werden müsste.\footnote{Auch in Vorlesungen hat \name[Immanuel]{Kant} diesen
Zusammenhang regelmäßig hervorgehoben. Siehe etwa
\cite{Kant:MoralphilosophieCollins1974}, \cite[][XXVII:
256.14--37]{Kant:GesammelteWerke1900ff.}: \enquote{Der göttliche Wille ist in
Ansehung der Moralitaet nothwendig, aber der menschliche Wille ist nicht
nothwendig sondern genöthigt. Also ist die practische Nothwendigkeit in Ansehung
des höchsten Wesens keine Obligation, das höchste Wesen handelt moralisch
nothwendig, aber hat keine Obligation. Warum sage ich nicht: Gott ist verbunden,
wahrhaftig heilig zu seyn? {\punkt} Also in Ansehung eines vollkommenen Willens,
bey dem die moralische Nothwendigkeit nicht allein objectiv sondern subjectiv
nothwendig ist, da findet keine Neceßitation und Obligation statt. Es müßen
demnach die sittlichen Handlungen nur zufällig seyn, wenn sie eine Nöthigung
haben sollen, und die einen moralisch unvollkommenen Willen haben, stehen unter
der Verbindlichkeit, und das sind Menschen.} Siehe außerdem
\cite{Kant:MetaphysikderSittenVigilantus1975}, \cite[][XXVII:
481.14--18]{Kant:GesammelteWerke1900ff.}: \enquote{Die Gesetze der Freiheit sind
nun {\punkt} 1. entweder blos nothwendige oder objective mere necessariae leges.
Diese finden allein bei Gott statt. 2. oder nöthigende, necessitantes. Diese finden bei Menschen Statt, und
sind objective necessaria, subjective aber zufällig.} Und schließlich schreibt
er in \cite{Kant:NaturrechtFeyerabend1979}, \cite[][XXVII:
1323.12--20]{Kant:GesammelteWerke1900ff.}: \enquote{Bei Gott ist sein guter
Wille nicht zufällig; daher findet auch bei ihm kein imperatives Gesetz statt,
um ihn zum guten Willen zu nöthigen. Denn das wäre überflüßig. Neceßitatio
einer an sich zufälligen Handlung durch objective Gründe ist praktische
Neceßitatio, das ist von praktischer Neceßitaet
unterschieden. Bei Gott sind auch Gesetze, aber die haben praktische
Nothwendigkeit. -- Praktische Neceßitation ist imperativ, ein Geboth. Ist der
Wille an sich selbst gut, so darf ihm gar nicht gebothen werden. Daher findet
bei Gott kein Geboth statt. Die objective praktische Nothwendigkeit ist bei Gott
auch subjective praktische Nothwendigkeit.}} Und wenngleich das Paradigma
endlicher Vernunft die \emph{moralische} Verfehlung ist, so macht doch die
Verbindung des Begriffs des Imperativs mit dem der Nötigung deutlich, dass
dieser Zusammenhang sich auf alle drei Arten von Imperativen
bezieht.\footnote{Siehe hierzu auch
\cite[][113]{Paton:TheCategoricalImperative1948}: \enquote{The principles
of goodness thus appear in our finite human condition as principles of
obligation. This is true even where the principle in question is one of skill or
rational self-love and not of morality. Men are not wholly rational in the
pursuit of happiness or even in the adoption of means to ends.}} Die
Endlichkeit der praktischen Vernunft bedingt, dass wir zwar wissen, wie zu
handeln klug wäre, aber dennoch
unklug handeln, weil wir auf kurzfristiges Glück setzen, obwohl wir wissen, dass
unser langfristiges Glück anderes erforderte. Und auch gegen die Regeln der
Geschicklichkeit handelt der endliche
Wille, der erkennt, was zu tun ist, und sich dennoch durch kurzfristige
Begierden von der Verfolgung eines Zieles abbringen lässt. Man denke an die
soeben beschriebene Langstreckenläuferin. Die Endlichkeit der menschlichen
Vernunft bezieht sich hier also nicht auf einen Mangel an Einsicht. Sie zeigt
sich gerade dort, wo uns die Einsicht leicht fällt.
\end{comment}


Zwischen Geschicklichkeit in verschiedenen Bereichen, Moralität und Klugheit
besteht hinsichtlich der jeweiligen epistemischen Zugänglichkeit ein
beachtlicher Unterschied.
Auf der einen Seite wissen wir nur sehr wenig darüber, wie Glückseligkeit -- die eigene
wie fremde -- zu erreichen ist. Denn Glückseligkeit meint nicht temporäre
Befriedigung, sondern langfristiges Glück. Und sie verlangt, die
unterschiedlichen und sich widerstreitenden (aktuellen wie zukünftigen)
Bedürfnisse in Einklang zueinander zu bringen oder gegeneinander abzuwägen. Wie
sich verschiedene Handlungsweisen und Entscheidungen langfristig bezahlt machen
werden, ist schwer abzuschätzen. \name[Immanuel]{Kant} neigt bezüglich unserer
Möglichkeiten, das je eigene langfristige Glück auf vernünftigen (und
moralkonformen) Wegen zu verfolgen, mitunter zum Pessimismus. Der
Mensch \emph{muss} sich zur Erreichung seiner Glückseligkeit seiner Vernunft
bedienen, hat damit aber ein denkbar schlechtes Instrument zur
Verfügung.\footnote{Man beachte hierzu die Ausführungen zur Misologie der
Vernunft in \cite[BA 4--8]{Kant:GrundlegungzurMetaphysikderSitten1965},
\cite[IV: 395.4--396.37]{Kant:GesammelteWerke1900ff.}.} Dennoch sind wir alle
nicht völlig ratlos. Wir können unser je eigenes Leben mit Blick auf anhaltendes
Glück kompetent führen, wenngleich eine gewisse Unsicherheit über den Ausgang
immer bleibt; und wir sind auch in begrenztem Umfang in der Lage, allgemeine
Ratschläge zu geben. Zumindest aber können wir sagen, dass niemand ohne unsere
Mithilfe sagen kann, wie wir selbst zur Glückseligkeit gelangen können. Wegen
ihrer Abhängigkeit von biographischen und anderen individuellen Zufälligkeiten
wie den eigenen Bedürfnissen und Neigungen, über die in aller Regel wir selbst
am zuverlässigsten Auskunft geben können, können Experten uns möglicherweise bei
eigenständigem Urteilen helfen, aber sie können uns dieses Urteil nicht
abnehmen.

Anders verhält es sich freilich mit den technischen Regeln der Geschicklichkeit.
Als Regeln, die sich auf zufällige Zwecke beziehen, sind sie prädestiniert für
die Ausbildung eines Expertenwesens. Wir werden jeweils über die meisten
technischen Regeln nicht selbst kompetent urteilen können und uns darauf
beschränken müssen, Kompetenzen bezüglich derjenigen Geschicklichkeiten zu
erwerben, deren Zwecke wir zu den eigenen machen wollen. Jede spezifische
Ausbildung liefert hierfür ein Beispiel.

\phantomsection\label{Abschnitt:moralischepistemischerOptimismus}
Schließlich befindet sich in Fragen der Moralität jeder in einer
epistemisch günstigen Position, weil \emph{jeder} über die entsprechende
Urteilskompetenz verfügt. Die These von der Aufklärungsrelevanz
ethischen Wissens ist bei \name[Immanuel]{Kant} mit der weiteren These korreliert, dass
jeder zu einem eigenständigen ethischen Urteil auch \emph{fähig} sei. In der
\titel{Grundlegung zur Metaphysik der Sitten} schreibt er:
\begin{quote}
  Was ich also zu tun habe, damit mein Wollen sittlich gut sei, darzu brauche
  ich gar keine weit ausholende Scharfsinnigkeit. Unerfahren in Ansehung des
  Weltlaufs, unfähig, auf alle sich eräugnende Vorfälle derselben gefaßt zu
  sein, frage ich mich nur: Kannst du auch wollen, daß deine Maxime ein
  allgemeines Gesetz
  werde?\footnote{\cite[BA 19--20]{Kant:GrundlegungzurMetaphysikderSitten1965},
  \cite[IV: 403.18--22]{Kant:GesammelteWerke1900ff.}.}
\end{quote}
In dieser These \name[Immanuel]{Kant}s liegt eine der Grundlagen des bekannten Primats des
Praktischen, der sich nicht erst aus dem Ergebnis der Vernunftkritik ergibt, sondern der Aufklärung weitgehend
programmatisch zugrunde
liegt.\footnote{\cite[Vgl.][127--130]{Ciafardone:UeberdasPrimatderpraktischenVernunftvordertheoretischenbeiThomasiusundCrusiusmitBeziehungaufKant1982}.
Wie \authorcite{Ciafardone:UeberdasPrimatderpraktischenVernunftvordertheoretischenbeiThomasiusundCrusiusmitBeziehungaufKant1982} darlegt, widerspricht dies dem fundamentalen
Missverständnis, welches in der \index{Kant, Immanuel}vorkantischen deutschsprachigen
Philosophie erstlinig den Hort einer \distanz{rationalistischen} Metaphysik
sieht. Vielmehr zeigt sich -- exemplarisch bei \name[Christian]{Thomasius} -- ein Vorbehalt
gegenüber metaphysischen Überlegungen, der sich einerseits aus
Relevanzüberlegungen und andererseits aus methodischen Vorbehalten speist.}
Es bestehe ein grundlegender Unterschied zwischen theoretischer und praktischer
Vernunft darin, dass wir in der theoretischen Vernunft anfällig für Irrtümer,
Aporien und Widersprüche in unseren Ansichten sind, sobald wir den Bereich der
Erfahrung verlassen und uns in die Metaphysik begeben. Die praktische Vernunft
hingegen habe es gerade dort besonders
leicht.\footnote{\cite[Vgl.][BA 21]{Kant:GrundlegungzurMetaphysikderSitten1965},
\cite[IV: 404.13--19]{Kant:GesammelteWerke1900ff.}: \enquote{In dem letzteren
[dem theoretischen Beurteilungsvermögen; A.\,G.], wenn die Vernunft es wagt, von
den Erfahrungsgesetzen und den Wahrnehmungen der Sinne abzugehen, gerät sie in
lauter Unbegreiflichkeiten und Widersprüche mit sich selbst, wenigstens in ein
Chaos von Ungewißheit, Dunkelheit und Unbestand. Im praktischen aber fängt die
Beurteilungskraft denn eben allererst an, sich recht vorteilhaft zu zeigen, wenn
der gemeine Verstand alle sinnliche Triebfedern von praktischen Gesetzen
ausschließt.}} Einfach ist freilich nicht das Aufsuchen des Prinzips der Moralität, also des
kategorischen Imperativs, sondern dessen Anwendung, welche unseren moralischen
Urteilen immer schon zugrunde liege. Dabei mögen sich auch die besten Köpfe
darin täuschen können, worin moralische Urteile fundiert sind --  so wie sich
nach \name[Immanuel]{Kant} etwa \name[David]{Hume} täuschte, als er unseren
Urteilen ein moralisches Gefühl zugrunde legte.
Die Anwendung hingegen erfordert kein Expertenwissen. Denn um zu wissen, welche Handlung
moralisch geboten oder verboten ist, muss ich nirgends nachschauen, ich muss
weder ein Buch (etwa einen Beichtspiegel) zur Hand nehmen noch einen Experten
befragen, sondern verlasse mich einfach auf den Maßstab, den ich in meiner
Vernunft finde. Der Kategorische Imperativ beschreibt die Grundlage unseres
moralischen Urteilens und Handelns, so wie sie uns auch vor jedem Philosophieren
längst vertraut sein
kann.\footnote{\cite[Vgl.][BA~20-1]{Kant:GrundlegungzurMetaphysikderSitten1965},
\cite[IV: 404.1-7]{Kant:GesammelteWerke1900ff.}: \enquote{Es wäre hier leicht zu
zeigen, wie sie [d.\,i.\ die gemeine Menschenvernunft; A.\,G.], mit diesem
Kompasse in der Hand, in allen vorkommenden Fällen sehr gut Bescheid wisse, zu
unterscheiden, was gut, was böse, pflichtmäßig, oder pflichtwidrig sei, wenn
man, ohne sie im mindesten etwas Neues zu lehren, sie nur, wie Sokrates tat, auf
ihr eigenes Prinzip aufmerksam macht, und daß es also keiner Wissenschaft und
Philosophie bedürfe, um zu wissen, was man zu tun habe, um ehrlich und gut, ja
sogar, um weise und tugendhaft zu sein.}} Nur handelt es sich freilich in der
Regel um ein implizites Wissen. Erst die Philosophie expliziere, was unsere
Praxis längst zum Ausdruck bringe, und zwar nicht, um neues Wissen zu
generieren, sondern um die Möglichkeit zu eliminieren, die jedem zugänglichen
ethischen Standards mittels ausgeklügelter Scheinargumente an unsere subjektiven
Bedürfnisse und Wünsche anzupassen -- die \emph{natürliche
Dialektik}.\footnote{\cite[Vgl.][BA
22\,f.]{Kant:GrundlegungzurMetaphysikderSitten1965},
\cite[][IV: 404.37--405.19]{Kant:GesammelteWerke1900ff.}.}


\subsection{Mündige Lebensführung und endliche
Vernunft}\label{section:MuendigeLebensfuehrung}
\begin{comment}
An verschiedenen stellen beschreibt \name[Immanuel]{Kant} unseren Willen als
endlich, insofern er nicht -- wie ein \singlequote{heiliger} Wille -- den
Vorschriften der praktischen Vernunft von sich aus genügt, sondern mit einem
Gegensatz von Wollen und Sollen konfrontiert ist. Weil wir über antagonistische
Neigungen verfügen, begegnen uns Imperative der Vernunft, die einen nötigenden
Charakter haben.\footnote{Siehe
\cite[][BA 40, 86\,f.,]{Kant:GrundlegungzurMetaphysikderSitten1965} \cite[][IV:
414.26--31, 439.28--440.13]{Kant:GesammelteWerke1900ff.};
\cite[][A 56--58, 145--150]{Kant:KritikderpraktischenVernunft1974}, \cite[][V:
32.1--33.5, 81.20--84.21]{Kant:GesammelteWerke1900ff.};
\cite[][\S~76]{Kant:KritikderUrteilskraft2009}, \cite[][V:
403.20--404.16]{Kant:GesammelteWerke1900ff.}.} Die Endlichkeit oder
Fehlbarkeit der praktischen Vernunft oder des Willens bezeichnet nach
\name[Immanuel]{Kant} den Sachverhalt, dass wir wider bessere Einsicht zu
handeln geneigt sind, weil unsere unmittelbaren Neigungen und Bedürfnisse der
Orientierung an vernünftigen Grundsätzen des Handelns -- praktischen
Erkenntnissen -- im Wege stehen. Wie ich im verbleibenden Teil dieses Kapitels
zeigen werde, ist es diese (praktische) Endlichkeit, die unserer Mündigkeit
zunächst im Weg steht, nicht eine vermeintliche (kognitive) Endlichkeit, der
zufolge wir die richtigen Einsichten nicht unabhängig vom Rekurs auf das Wissen
anderer zu generieren vermögen.
\end{comment}

\name[Immanuel]{Kant} weist die Behauptung zurück, Autonomie und Mündigkeit
seien mit Beliebigkeit verbunden, insofern sein Begriff der Mündigkeit und
Autonomie nicht besagt, dass wir Inhalte frei auswählen, sondern dass wir sie
eigenverantwortlich \emph{erkennen} sollen. Zu Beginn von Kapitel
\ref{subsection:DieBestimmungdesMenschen} artikulierte ich die Vermutung, dass
eine Akzentuierung bestimmter Erkenntnisse helfen könnte, Mündigkeit zu einem
realistischen Anspruch zu machen. Ginge es der Aufklärung nur um Mündigkeit in
bestimmten Fragen, wäre ihrem Anspruch in epistemischer Hinsicht leichter zu
genügen. Zumindest eine Akzentuierung ließ sich ausmachen: Aufklärung fokussiert
Wissen, welches ich vorhin als handlungsorientierend oder pragmatisch (im
weiteren Sinne) bezeichnete. In seinem Zentrum stehen Erkenntnisse, die wir
praktisch nennen, insofern sie nötigend sind und Imperative artikulieren. Eine
besondere Bedeutung kommt somit den drei Arten von Imperativen zu: Regeln der
Geschicklichkeit, Ratschlägen der Klugheit und Geboten der Sittlichkeit.

\name[Immanuel]{Kant}s moralisch-epistemischer
Optimismus\footnote{\phantomsection\label{Fussnote:moralischepistemischerOptimismus}Diesen
Ausdruck verwendete Martin \name[Martin]{Sticker} in einem Vortrag über
\titel{Educating the Common Agent -- Kant on Common Rational Capacities and the
Varieties of Moral Education} am 02.\,02.\,2013 an der Universität Wien. Dem
moralisch-epistemischen Optimismus \name[Immanuel]{Kant}s stehen zwei Formen des
Pessimismus in Fragen der Moral gegenüber: Erstens denkt \name[Immanuel]{Kant}
nicht, dass Menschen auch häufig moralisch (\enquote{aus Pflicht} und nicht nur
\enquote{pflichtmäßig}) handeln, und zweitens seien wir auch kaum fähig, die
eigene oder fremde Motivation zu erkennen, können also nie mit Gewissheit sagen,
\emph{ob} jemand (wir selbst oder andere) moralisch oder selbstsüchtig
handelten \mkbibparens{siehe dazu etwa
\cite[][A
222]{Kant:UeberdenGemeinspruch:dasmaginderTheorierichtigseintaugtabernichtfuerdiePraxis1977},
\cite[][VIII: 284.21--28]{Kant:GesammelteWerke1900ff.}}.} kommt der
Forderung nach Ausgang aus der selbst verschuldeten Unmündigkeit offensichtlich
entgegen, insofern ihm zufolge Mündigkeit bezüglich der Gebote der Sittlichkeit
zumindest insofern vorausgesetzt werden kann, dass wir die Gebote
\emph{erkennen} (wenngleich wir sie möglicherweise nicht befolgen).
Wir sind nicht nur \emph{berufen}, uns ein je eigenes Urteil in moralischen Fragen zu
bilden, sondern auch \emph{befähigt}. Der in der \titel{Kritik der reinen
Vernunft} gescholtene Naturalist der reinen Vernunft ist nicht derjenige, der
sich auf sein wenig geschultes (aber am Urteil anderer kontrolliertes!)
\emph{moralisches} Urteil verlässt. Denn wer sich in Fragen der Moral und Ethik
auf die gemeine (im Sinne von \emph{communis}, nicht von \emph{vulgaris}),
methodisch nicht speziell geschulte Vernunft verlässt, macht damit nichts falsch. Der Weg zur moralischen
Kompetenz führt über die Ausbildung der Urteilskraft im Austausch mit anderen,
nicht über den Erwerb besonderer Methoden. Lediglich bei den Fragen der
Metaphysik, die \name[Immanuel]{Kant} in der \titel{Kritik der reinen Vernunft}
diskutiert, kann nur kompetent urteilen, wer methodisch geschult ist. Und
deswegen bezieht \name[Immanuel]{Kant} die Ablehnung des Naturalismus der
Vernunft gerade auf \enquote{die erhabensten Fragen, die die Aufgabe der
Metaphysik ausmachen}\footnote{\cite[][B 883]{Kant:KritikderreinenVernunft2003};
\cite[][III: 551.36]{Kant:GesammelteWerke1900ff.}.}. In der Moral ist Mündigkeit
primär eine Frage des Handelns, nicht des Erkennens.

\begin{comment}
Mit Blick auf die Moral wird auch die Betonung der Religionsfragen in der
Aufklärungsschrift plausibel. Eine wichtige Eigenschaft von
\name[Immanuel]{Kant}s Ethik ist die Zurückweisung aller Formen von Heteronomie,
zu denen er insbesondere auch Versuche zählt, Moral in der Religion oder
Theologie zu fundieren. Die praktische Philosophie gibt die Bestimmung des
Menschen vor und prägt damit auch die Anthropologie, deren säkularer Charakter
gegenüber \authorcite{Spalding:BetrachtungueberdieBestimmungdesMenschen1749}s
Suche nach einer Bestimmung des Menschen trotz der Kontinuität gewiss offensichtlich
ist.\footnote{\cite[Vgl.][13]{Brandt:DieBestimmungdesMenschenbeiKant2007}.}
Dennoch ist es nun einfach, die Sonderstellung der Religion in
\name[Immanuel]{Kant}s Aufklärungsschrift zu erläutern. \name[Immanuel]{Kant}
stellt zunächst dem biblischen Theologen der oberen Fakultät einen reinen
Vernunftglauben gegenüber; und dieser sei ausreichend, der Bestimmung des
Menschen gemäß sich im eigenen Leben orientieren zu können:
\begin{quote}
  Ein reiner Vernunftglaube ist also der Wegweiser oder Kompaß, wodurch der
  spekulative Denker sich auf seinen Vernunftstreifereien im Felde
  übersinnlicher Gegenstände orientieren, der Mensch von gemeiner doch
  (moralisch) gesunder Vernunft aber seinen Weg, so wohl in theoretischer als
  praktischer Absicht, dem ganzen Zwecke seiner Bestimmung völlig angemessen
  vorzeichnen kann; und dieser Vernunftglaube ist es auch, der jedem anderen
  Glauben, ja jeder Offenbarung, zum Grunde gelegt werden
  muß.\footnote{\cite[A~320--1]{Kant:Washeisst:SichimDenkenorientieren?1977},
  \cite[VIII: 142.1--8]{Kant:GesammelteWerke1900ff.}.}
\end{quote}
Dies beruht auf wenigen, aber weitreichenden Voraussetzungen bezüglich des
Verhältnisses von Religion und Moral. Zu den einschlägigsten Voraussetzungen gehört
etwa, dass beide thematisch deckungs\-gleich
seien\footnote{\cite[A 44\,f.,]{Kant:DerStreitderFakultaeten1977} \cite[VII:
36.18--26]{Kant:GesammelteWerke1900ff.}: \enquote{Nicht der Inbegriff gewisser
Lehren als göttlicher Offenbarungen (denn der heißt Theologie), sondern der
aller unserer Pflichten überhaupt als göttlicher Gebote (und subjektiv der
Maxime, sie als solche zu befolgen) ist Religion.
  Religion unterscheidet sich nicht der Materie, d.i. dem Objekt nach in irgend
  einem Stücke von der Moral, denn sie geht auf Pflichten überhaupt, sondern ihr
  Unterschied von dieser ist bloß formal, d.\,i. eine Gesetzgebung der Vernunft,
  um der Moral durch die aus dieser selbst erzeugten Idee von Gott auf den
  menschlichen Willen zu Erfüllung aller seiner Pflichten Einfluß zu geben.}
  \cite[Des weiteren:][B 229--231]{Kant:DieReligioninnerhalbderGrenzenderblossenVernunft1977},
\cite[][VI: 153.28--154.5]{Kant:GesammelteWerke1900ff.}: \enquote{\ori{Religion}
ist (subjektiv betrachtet) das Erkenntnis aller unserer
 Pflichten als göttlicher Gebote. Diejenige, in welcher ich vorher wissen muß,
 daß etwas ein göttliches Gebot sei, um es als meine Pflicht anzuerkennen, ist
 die \ori{geoffenbarte} (oder einer Offenbarung benötigte) Religion: dagegen
 diejenige, in der ich zuvor wissen muß, daß etwas Pflicht sei, ehe ich es für
 ein göttliches Gebot anerkennen kann, ist die \ori{natürliche Religion}.}} und
 sich daher -- ähnlich wie in \authorcite{Lessing:EineDuplik1897}s Ringparabel -- religiöse
Überzeugungen ausschließlich anhand ihrer Zu- oder Abträglichkeit bezüglich der
Moralität der Gläubigen bewerten ließen.\footnote{\cite[Vgl.][A
103]{Kant:DerStreitderFakultaeten1977}, \cite[VII:
63.18--64.2]{Kant:GesammelteWerke1900ff.}: \enquote{Die Beglaubigung der Bibel
nun, als eines in Lehre und Beispiel zur Norm dienenden
evangelisch-messianischen Glaubens, kann nicht aus der Gottesgelahrtheit ihrer
Verfasser \punkt , sondern muß aus der Wirkung ihres Inhalts auf die Moralität
des Volks, von Lehrern aus diesem Volk selbst, als Idioten (im
Wissenschaftlichen), an sich, mithin als aus dem reinen Quell der allgemeinen,
jedem gemeinen Menschen beiwohnenden Vernunftreligion geschöpft, betrachtet
werden, die, eben durch diese Einfalt, auf die Herzen desselben den
ausgebreitetsten und kräftigsten Einfluß haben mußte.}} Dabei seien es die
\emph{a priori} erkennbaren Gesetze der Moral, die das uneingeschränkte Sagen
haben, während die Religion sich innerhalb des von der Moral gesteckten Rahmens
zu bewegen habe. Der biblische Theologe tritt nun aber als Experte in Fragen der Moral auf
und verweist auf schriftlich fixierte \enquote{Satzungen und Formeln} statt auf
die Vernunft.



Etwas anders als in Angelegenheiten der Moral verhält es sich mit Fragen der
Gerechtigkeit und der Bewertung von Politik. Geht es um das positive Recht oder um volkswirtschaftliche Fragen, so
fragen wir einen Juristen oder einen Ökonomen oder verweisen in Diskussionen auf
deren Aussagen; denn wir selbst haben normalerweise nicht die entsprechenden
Kenntnisse. Sich bei der Bewertung dessen, was positives Recht ist, auf einen
Maßstab der Gerechtigkeit hin, an einen Experten zu halten, wäre der Aufklärung
zuwider und eklatanter Ausdruck von Unmündigkeit.\footnote{Ich
sage \enquote{auf einen Maßstab der Gerechtigkeit hin}, weil es auch andere Maßstäbe gibt --
beispielsweise den der volkswirtschaftlichen Vernunft --, zu denen wir durchaus
Experten befragen sollten. Welche Folgen eine bestimmte Fiskalpolitik haben
wird, wie sich Monopolbildungen verhindern lassen usw., sind Fragen, die der
Ökonom (als Ökonom) beantworten kann, nicht der Staatsbürger als solcher.
In der Politik gibt es Bereiche, die ein technisches Wissen oder auch ein
besonderes Maß an Weltklugheit erfordern, das die Kompetenzen der meisten
Menschen übersteigt.} Wenn es in Diskussionen zu Meinungsverschiedenheit darüber
kommt, ob ein Zustand oder eine politische Entscheidung gerecht ist, reicht es
daher auch nicht aus, auf eine bestimmte Institution oder einen anerkannten
Experten zu verweisen. Wir müssen unsere eigenen Ansichten mit \emph{Argumenten}
untermauern, die unsere Gesprächspartner selbst als gültig einsehen können. Wir
sind also ebenso wie in Fragen der Moral berufen, unser eigenes kompetentes
Urteil nicht an \singlequote{Experten} zu delegieren -- zumindest wenn wir
beanspruchen, mündige Bürger eines demokratischen Gemeinwesens zu sein. Aber wir
sind in einer epistemisch schlechteren Position, denn Mündigkeit in Fragen
politischer Entscheidungen benötigt ein größeres Maß an Weltklugheit
als ethische Alltagsentscheidungen; man denke nur an die Feinheiten von
Diplomatie und Außenpolitik. Die allgemeine Zielsetzung von Politik gehört
jedoch nicht in die Hände von Experten, höchstens deren Umsetzung.\footnote{Dem widerspricht
\authorfullcite{Wilson:PoliticsandExpertise1971} unter Rückgriff auf eine These
\singlename{Platon}s: \enquote{I mean \punkt\ that there are people better
equipped than others to decide what is right, in the context of ends as well as means,
for a society or a state: the thesis maintained but inadequatly defended in
Plato's \ori{Republic}}
\parencite[][34]{Wilson:PoliticsandExpertise1971}. Allerdings scheint
\authorcite{Wilson:PoliticsandExpertise1971} dies letztlich auf die These einzuschränken, dass manche
Menschen charakterlich eher dazu geeignet sind, politische Mandate in einer repräsentativen Demokratie
einzunehmen, als andere. Schwieriger ist seine These, es gebe Güter, die für
alle rationalen Menschen erstrebenswert seien, zumal er explizit darauf
verzichtet, Argumente anzugeben. \authorcite{Wilson:PoliticsandExpertise1971} behauptet im Grunde:
\emph{Wenn} es eine allgemeingültige und einsehbare, rein rationale und apriorische Theorie
der Gerechtigkeit \emph{gäbe}, \emph{dann} spräche nichts gegen Experten zu
ihrer Umsetzung.} Wir können hier sagen, dass wir auf der einen Seite
\emph{berufen} sind, uns ein kompetentes Urteil in Fragen politischer
Entscheidungen zu bilden, dass wir uns auf der anderen Seite jedoch um unsere
\emph{Befähigung} stets zu bemühen haben.

Gewiss finden sich bei \name[Immanuel]{Kant} kaum Überlegungen dazu, was
den Menschen als mündigen Staatsbürger
ausmacht;\footnote{\name[Immanuel]{Kant}s Auskünfte beschränken
sich auf Angaben zur wirtschaftlichen Selbständigkeit, die den aktiven
Staatsbürger, dem allein das Wahlrecht zustehe, vom passiven unterscheide. Dabei
betont er, dass die Verweigerung der bürgerlichen Mündigkeit \enquote{der
Freiheit und Gleichheit derselben als Menschen}
\mkbibparens{\cite[][\S~46]{Kant:DieMetaphysikderSitten1977Rechtslehre},
\cite[][VI: 315.7--8]{Kant:GesammelteWerke1900ff.}} nicht entgegenstehe
\mkbibparens{\cite[vgl.][\S~46]{Kant:DieMetaphysikderSitten1977Rechtslehre},
\cite[][VI: 314.17--315.22]{Kant:GesammelteWerke1900ff.}}.} die politische
Emanzipation ist nicht sein Thema -- wie schon aus seiner ablehnenden Haltung
der Demokratie gegenüber hervorgeht\footnote{In \titel{Zum ewigen Frieden}
betont \name[Immanuel]{Kant}, ein Staatswesen müsse \emph{republikanisch}, aber
nicht unbedingt \emph{demokratisch} sein
\mkbibparens{vgl. \cite[][B 24--29]{Kant:ZumewigenFrieden1900ff.},
\cite[][VIII: 351.21--353.18]{Kant:GesammelteWerke1900ff.}}. Letztlich schließe
die Forderung nach einer republikanischen Verfassung die Demokratie sogar aus,
weil letztere eine wirksame Gewaltenteilung nicht zulassen. Dass Konzept einer
repräsentativen (parlamentarischen) Demokratie mit der Gewaltenverschränkung,
wie es uns heute als politische Wirklichkeit vertraut ist, stand
\name[Immanuel]{Kant} noch nicht zur Verfügung. Die demokratische Regierungsform
sei gerade nicht \singlequote{repräsentativ}, \enquote{weil alles da Herr sein
will} \mkbibparens{\cite[][B 27]{Kant:ZumewigenFrieden1900ff.},
\cite[][VIII: 353.1]{Kant:GesammelteWerke1900ff.}}.}.
so dass wir zunächst bemerken müssen, dass es sich bei der politischen
Mündigkeit um eine Forderung und eine Herausforderung handelt, die nicht die
Aufklärung, sondern höchstens die konsequente Weiterführung der Aufklärung
betrifft. Dass es ein zentrales Thema der Aufklärung sein sollte, ergibt sich
daraus, dass die Errichtung eines bürgerlichen Gemeinwesens zur Bestimmung des
Menschen gehört, die sich aus der anthropologischen Tatsache seiner
\singlequote{ungeselligen Geselligkeit} ergibt.
\end{comment}

Völlig anders verhält es sich mit den Regeln der Geschicklichkeit, die
paradigmatisch für Expertenwissen sind. Wir können nicht in allen Bereichen
selbst Kompetenzen erwerben, die für willkürlich zu wählende Zwecke nützlich
sein könnten. Es ist dies für den Ausgang aus selbst verschuldeter Unmündigkeit
offensichtlich auch nicht nötig. Niemand käme auf die Idee, einer Person die
Mündigkeit abzusprechen, weil sie sich in Fragen des Flugzeugbaus oder der
Elektrotechnik nicht auskennt. Ob wir solches Wissen benötigen, hängt von
unserem Lebensweg und den zufällig gewählten beruflichen Entscheidungen ab. Aber
selbst dann scheint es keine Frage der Mündigkeit zu betreffen. Ein Ingenieur,
der in Fragen der Mechanik nicht selbst kompetent urteilen kann, sondern immer
seine Kollegen fragen muss, ist kein unmündiger Mensch, sondern ein schlechter
Ingenieur. Regeln der Geschicklichkeit sollten also für Belange der Aufklärung
nicht ausschlaggebend sein.


Komplizierter ist die Situation bezüglich der Erkennbarkeit von Erkenntnissen
der Klugheit in ihrer Bedeutung für unsere je eigene Mündigkeit. Dass die
Ratschläge der Klugheit auch Thema des Aufklärungsaufsatzes sind, wird deutlich,
wenn man beachtet, dass medizinische Ratschläge solche der Klugheit sind. Auch
als Beispiel für Unmündigkeit wählt \name[Immanuel]{Kant} eines aus der Medizin:
\enquote{Habe ich {\punkt} einen Arzt, der für mich die Diät beurteilt, {\punkt}
so brauche ich mich ja nicht selbst zu bemühen.}\footnote{\cite[][A
482]{Kant:BeantwortungderFrage:WasistAufklaerung?1977}, \cite[][VIII:
35.13--16]{Kant:GesammelteWerke1900ff.}. Diese Forderung kritisiert
\textcite[vgl.][258]{ONeill:Therhetoricofdeliberation2002}.} Heißt dies nun
aber, dass wir Ratschläge von Experten generell zurückweisen sollten? Sollte dem
so sein, wäre durch die Akzentuierung wohl wenig gewonnen. Selbst den Anspruch
auf medizinische Kompetenz anzumelden, ist bestenfalls \singlequote{heroisch},
eher leichtfertig. \authorfullcite{ONeill:Therhetoricofdeliberation2002} nennt
das Beispiel eines besorgten Verbrauchers, der sich angesichts des BSE-Skandals
Gedanken über den Zusammenhang von BSE und Creutzfeld-Jakob-Krankheit macht.
Offensichtlich kann er ohne Hilfe von Experten hier nichts ausrichten. Er kann
-- wie \authorcite{ONeill:Therhetoricofdeliberation2002} verdeutlicht -- nicht
einmal auf der Grundlage eigenen Wissens beurteilen, welchem Experten er
vertrauen sollte.\footcite[Vgl.][257]{ONeill:Therhetoricofdeliberation2002} Wir
vertrauen in aller Regel lediglich darauf, dass staatliche
Zulassungsbestimmungen zuverlässig sind; und diese wiederum sehen Prüfungen der
Kompetenz durch andere Experten vor. Es ist in vielen Fällen so, dass die
Expertise, die jemand auf einem bestimmten Gebiet zu haben beansprucht, nur von
denjenigen bewertet werden kann, die selbst über sie verfügen, also von anderen
Experten. In medizinischen Fragen nicht auf Experten zu hören, weil man seine
Mündigkeit dadurch gefährdet sähe, ist freilich Unsinn.
Und wenn \name[Immanuel]{Kant} sich mit seiner Forderung genau hierauf festlegte, dann
wäre diese Forderung zu verwerfen.

Plausibler wird die Forderung nach Mündigkeit in medizinischen Fragen, wenn wir
nicht auf den \emph{Erwerb} medizinischen Wissens schauen, sondern auf dessen
\emph{Anwendung}. In vielen, wenn auch bei weitem nicht in allen Fällen ist uns
allen bekannt, was wir aus medizinischer Sicht tun sollten. (Wir sollten
beispielsweise nicht zu viel Zucker und zu viel Fett zu uns nehmen, nicht rauchen,
Alkohol nur in Maßen konsumieren und uns regelmäßig bewegen.) Hier liegt kein
epistemisches Problem vor, sondern eines der Anwendung:
Unsere kurzfristigen Begierden verhindern, dass entsprechende Ratschläge der
Klugheit auch als subjektive Prinzipien handlungswirksam werden.

Es geht \name[Immanuel]{Kant} nicht um die Zurückweisung
von Expertenwissen im Rahmen der Klugheit; unmündig ist nicht, wer in
medizinischen Fragen auf den Rat von Experten hört. Unmündig ist vielmehr, wer
wider besseres Wissen handelt und -- statt selbst die Verantwortung dafür zu
übernehmen -- hinterher auf das Wissen von Experten setzt.  Wir
suchen medizinischen Rat, der uns sagt, wie wir \emph{dennoch} -- trotz
unserer ungesunden Lebensweise -- dauerhaft gesund bleiben
können.\footnote{\cite[Vgl.][A 31]{Kant:DerStreitderFakultaeten1977},
\cite[][VII: 30.24--30]{Kant:GesammelteWerke1900ff.}: \enquote{Was ihr
\ori{Philosophen} da schwatzet, wußte ich längst von selbst; ich will aber von euch als Gelehrten
wissen: wie, wenn ich auch \ori{ruchlos} gelebt hätte, ich dennoch kurz vor
Torschlusse mir ein Einlaßbillett ins Himmelreich verschaffen, wie, wenn ich
auch \ori{Unrecht} habe, ich doch meinen Prozeß gewinnen, und wie, wenn ich
auch meine körperlichen Kräfte nach Herzenslust benutzt und \ori{mißbraucht}
hätte, ich doch gesund bleiben und lange leben könne.}} Statt auf \emph{solche}
Unterstützung von Experten zu bauen soll man laut \name[Immanuel]{Kant} zunächst
das längst vorhandene Wissen auch zur Anwendung bringen, wofür wir jedoch
Disziplin aufbringen müssen.\footnote{\name[Immanuel]{Kant} sagt, man solle
  \enquote{sich mäßig im Genusse und duldend in Krankheiten und dabei
  vornehmlich auf die Selbsthülfe der Natur rechnend {\punkt} verhalten; zu welchem allem es
  freilich nicht eben großer Gelehrsamkeit bedarf, wobei man dieser aber
  größtenteils entbehren kann, wenn man nur seine Neigungen bändigen und seiner
  Vernunft das Regiment anvertrauen wollte, was aber, als Selbstbemühung, dem
  Volk gar nicht gelegen ist} \mkbibparens{\cite[][A
  30\,f.,]{Kant:DerStreitderFakultaeten1977} \cite[][VII:
  30.14--19]{Kant:GesammelteWerke1900ff.}}.
Diesen Ratschlag sollten wir gewiss nicht \emph{en detail} übernehmen.
Möglicherweise ist es dem medizinischen Fortschritt zu verdanken, dass es uns töricht
erscheint, uns in Krankheiten \singlequote{duldend} zu verhalten; möglicherweise
war dies aber auch schon zu \name[Immanuel]{Kant}s Zeiten kein besonders kluger
Rat.} Das Moment der Unmündigkeit liegt dabei aber nicht in der Berufung auf
Experten, sondern darin, dass vorhandene eigene Wissen nicht zur Anwendung zu
bringen. Die Unmündigkeit ist also zunächst Folge der Endlichkeit unseres
Willens, nicht des Verstandes.\footnote{Auch
\authorfullcite{Mikalsen:TestimonyandKantsIdeaofPublicReason2010} behauptet,
dass sich die Bemerkung \name[Immanuel]{Kant}s in der Aufklärungsschrift primär
gegen unsere Bequemlichkeit richtet und deswegen keine \singlequote{heroischen}
Erkenntnisbemühungen fordert: \enquote{I do not think that Kant
is concerned with the relation between experts and non-experts in this passage. Nor do I think what he says has
any of the questionable implications discussed above. Rather than making a call
for foolhardy heroism, Kant points to a main obstacle to enlightenment: the
convenience of immaturity}
\parencite[][28]{Mikalsen:TestimonyandKantsIdeaofPublicReason2010}. Anders als
meine Interpretation sieht er die Bequemlichkeit wiederum im \emph{Erkennen}
und nicht in der \emph{Anwendung} der Erkenntnis. Ich sehe nicht, wie dies die Kritik
\authorcite{ONeill:Therhetoricofdeliberation2002}s entkräften soll.}

Möglicherweise überschätzt \name[Immanuel]{Kant} in seinen Texten (speziell im
\titel{Streit der Fakultäten}) unsere -- und auch seine eigenen -- medizinischen
Kompetenzen. Und er unterschätzt die Möglichkeiten der Schulmedizin, wie aus
heutiger Perspektive besonders deutlich wird. Dennoch ist es korrekt zu sagen,
dass es in der Medizin viele Beispiele für Ratschläge der Klugheit gibt, die
zeigen, dass wir eher in der Anwendung als in der Erkenntnis fehlgehen, und dass
diese Fehler in der Anwendung der Tatsache geschuldet sind, dass wir zu bequem
sind, unsere Vernunft gegenüber unseren Neigungen zu
behaupten.\footnote{Ganz analog
sieht \name[Immanuel]{Kant} zwei weitere Fälle: Wir alle wissen uns unseren
Mitmenschen gegenüber gerecht zu verhalten, ohne stets die Hilfe von Experten in
Anspruch nehmen zu müssen. Nur ist dies oft damit verbunden, persönliche
Begierden zurückzustellen. Wenn uns hingegen daran gelegen ist, ungerechte
Ansprüche vor Gericht durchzusetzen, seien wir auf Experten angewiesen. Und
während jeder Gläubige wissen könne, wie er ein gottgefälliges Leben führt
(indem er sich moralisch verhält), suche so mancher Rat bei seiner Kirche, um
sich unmoralisch verhalten und dennoch -- durch entsprechende Zeremonien -- die
Gnade Gottes erkaufen zu können. Der Ablasshandel ist ein anschauliches
Beispiel. In allen drei Fällen ist es kein Mangel an Wissen, der unserer
Mündigkeit entgegensteht, sondern die Verlockung durch kurzfristige Wünsche
und Begierden, \emph{gegen} dieses bessere Wissen zu handeln.}
Dies ist ein systematisch entscheidender Punkt bei der Frage, wie die Forderung
nach Mündigkeit und Selbstbestimmung mit unserer arbeitsteiligen
Wissensgesellschaft vereinbar sein kann. Aufklärung und Mündigkeit verlangen von
uns nicht, Experten zu werden. Sie verlangen lediglich ein gewisses Basiswissen,
etwa über eine gesunde Lebensführung, und vor allem die \emph{Anwendung} dieses
Basiswissens im Leben. Hier hat auch die pragmatische Anthropologie mit ihrem
Wissen um die \emph{conditio humana} ihren systematischen Ort: Sie versammelt
gerade zentrales Wissen über den Menschen, welches dieser benötigt, um sein
eigenes Leben kompetent führen zu können. Aber dieses Wissen ist kein
Expertenwissen, das einer breiteren Bevölkerung notwendig verschlossen bliebe.
Wer freilich auch über rudimentärste medizinische Kenntnisse oder basales Wissen
über die menschliche Natur nicht verfügt, den können wir in der Tat nicht als
mündig und aufgeklärt bezeichnen.

Auch dies ist eine Deutungsmöglichkeit der Aussage, diejenigen, die besonders
reich an Kenntnissen sind, seien mitunter \enquote{im Gebrauche derselben am
wenigsten aufgeklärt}\footnote{\cite[][A
329]{Kant:Washeisst:SichimDenkenorientieren?1977}, \cite[][VIII:
146.34]{Kant:GesammelteWerke1900ff.}.}, die dadurch gestützt wird, dass
\name[Immanuel]{Kant} anmerkt, die Unterscheidung der Erkenntnisse in
pragmatische und spekulative Erkenntnisse betreffe nicht die
Erkenntnisse selbst, sondern eben ihren
\emph{Gebrauch}.\footnote{\cite[Vgl.][\nopp 2802]{Kant:Reflexionen1900ff.},
\cite[][XVI: 519.15--17]{Kant:GesammelteWerke1900ff.}.} Aufklärung heißt nicht
nur, vernünftige Erkenntnisse zu gewinnen, sondern die eigenen Handlungen an
vernünftigen Erkenntnissen auszurichten. Es ist nicht nur unser eingeschränktes
Erkenntnisvermögen, welches Aufklärung zur Herausforderung für endliche Wesen
macht. Es reicht nicht zu wissen, wie eine gesunde Ernährung aussieht und welche
Vorteile sie uns bringt, wenn man nicht in der Lage ist, am Süßwarenregal vorbei zu
gehen. Als endliche Wesen müssen wir erst lernen, unsere längerfristigen
Intentionen gegen kurzfristige Neigungen zu behaupten, um vernünftig zu handeln.
Dies zu lernen, nennt \name[Immanuel]{Kant} \enquote{\emph{Disziplin}} oder auch \enquote{Kultur der
Zucht}.\footnote{\cite[Vgl.][\S~83]{Kant:KritikderUrteilskraft2009}, \cite[][V:
432.3--5]{Kant:GesammelteWerke1900ff.}.}

Auf den ersten Blick scheinen Disziplinierung und Zucht gerade der Freiheit
entgegengesetzt zu sein, insofern sie sich darauf zu richten scheinen, einen
bestimmten \singlequote{asketischen} Typus von Persönlichkeit, der frei ist von
Begierden und Neigungen, gewaltsam hervorzubringen. Sie richten sich aber nicht
gegen Begierden und Neigungen \emph{per se} (ohne Begierden und Neigungen gäbe
es auch keine langfristige Glückseligkeit), sondern gegen die Unfähigkeit,
einmal als richtig eingesehene Handlungsgrundsätze (\singlequote{Maximen}) auch
dann weiter zu verfolgen, wenn ihnen kurzfristige Begierden entgegenstehen.
Und sie dienen entsprechend nicht nur der Moral, sondern ermöglichen auch die
Befolgung technischer und pragmatischer Imperative der Geschicklichkeit und
Klugheit. Der konkrete Inhalt oder die Ausgestaltung der je eigenen Freiheit ist
noch immer mit den je konkreten Begierden und Neigungen verbunden, zu denen der
Handelnde sich nun jedoch bewusst und willentlich verhalten kann. Endliche Wesen
erlangen Mündigkeit daher nur auf dem Wege der Disziplinierung, die als
Grundlage von Kultivierung, Zivilisierung und Moralisierung fungiert, indem sie
uns befähigt, unserer kurzfristigen Neigungen Herr zu werden.



\begin{comment}
Ich hatte oben auf \name[Immanuel]{Kant}s Antinaturalismus im Gefolge
\authorcite{Wolff:Psychologiaempirica1968}s verwiesen\footnote{Siehe Kapitel
\ref{Terminus:methodischerNaturalismus}, insb.
S.~\pageref{Terminus:methodischerNaturalismus}.} und kann die gemachten Bemerkungen nun konkretisieren:
Genau wie \authorcite{Wolff:Psychologiaempirica1968} sieht \name[Immanuel]{Kant} die Unmündigkeit und
Autoritätsgläubigkeit nicht generell als das Resultat mangelnder
Entschlusskraft, sondern mitunter auch als Folge seiner Inkompetenz in vielen
Bereichen der Wissenschaft an. Unmündigkeit ist dort nicht selbst verschuldet,
wo die Urteilskompetenz fehlt, weil naturgemäß nur Experten über sie
verfügen. Und diese unverschuldete Unmündigkeit bleibt ohne negative
Folgen, wenn der der Wissenschaft im allgemeinen oder auch einer bestimmten
Wissenschaft Unkundige sich einfach auf das beschränkt, was zu wissen ihm
obliegt und auch möglich ist. Und dies sind Klugheit und Moral, bei denen er
gerade nicht unkundig, sondern zu einem eigenständigen Urteil fähig sein soll.
\end{comment}

Vorläufig zusammenfassend seien daher zwei Momente hervorgehoben, die wir zu
einer mündigen Lebensführung dringender benötigen als methodische Kenntnisse und
Fachwissen in verschiedenen Wissensbereichen: Nach den Überlegungen am Ende des
\ref{section:KantalsliberalerAufklaerer}. Kapitels benötigen wir eine geübte
Urteilskraft; nun zeigt sich, dass wir über Disziplin verfügen müssen.
Wer sich darauf beschränke, bezüglich dessen, was er tun soll und wie er sein eigenes Leben
einzurichten habe, eigenständig zu urteilen, dem falle dies auch nicht schwer.
Er muss lediglich seine Urteilskraft im Austausch mit anderen ausbilden und
Herr über seine eigenen kurzfristigen Begierden werden. Die Endlichkeit des
Menschen macht dies nötig, aber nicht unmöglich. Nur wer sich noch darüber
hinaus mit den heiklen Fragen der Metaphysik -- oder genauer: der Metaphysik der
Natur -- befasst, verfällt leicht der Versuchung, sich an Andere zu wenden, weil
ihm ein eigenständiges Urteil einfach nicht gelingen will. In der \titel{Kritik
der Urteilskraft} schreibt \name[Immanuel]{Kant}:
\begin{quote}
  Man sieht bald, daß Aufklärung zwar in Thesi leicht, in Hypothesi aber eine
  schwere und langsam auszuführende Sache sei: weil mit seiner Vernunft nicht
  passiv, sondern jederzeit sich selbst gesetzgebend zu sein, zwar etwas ganz
  Leichtes für den Menschen ist, der nur seinem wesentlichen Zwecke angemessen
  sein will und das, was über seinen Verstand ist, nicht zu wissen verlangt;
  aber da die Bestrebung zum Letzteren kaum zu verhüten ist, und es an anderen,
  welche diese Wißbegierde befriedigen zu können mit vieler Zuversicht
  versprechen, nie fehlen wird, so muß das bloß Negative (welches die
  eigentliche Aufklärung ausmacht) in der Denkungsart (zumal der öffentlichen)
  zu erhalten oder herzustellen sehr schwer
  sein.\footnote{\cite[][\S~40]{Kant:KritikderUrteilskraft2009}, \cite[][V:
  294.29--37]{Kant:GesammelteWerke1900ff.}.  Zur Verwendung des Begriffspaares
  \enquote{in thesi}/\enquote{in hypothesi} vgl.\
  \cite[][\nopp 5696]{Kant:Reflexionen1900ff.}, \cite[][XVIII:
  329.1--2]{Kant:GesammelteWerke1900ff.}, wo \name[Immanuel]{Kant} \enquote{in
  thesi} als der logischen (begrifflichen) Möglichkeit nach, \enquote{in hypothesi} als der realen
  Möglichkeit nach bedeutend erläutert. In \titel{Über den Gemeinspruch\ldots}
  wiederum sagt \name[Immanuel]{Kant}, mit der Wendung, ein Satz gelte zwar in
  thesi, nicht aber in hypothesi, meine man oft, dass es in der Theorie ganz
  gut und richtig, in der Praxis aber unbrauchbar sei
  \mkbibparens{\cite[vgl.][A
  204]{Kant:UeberdenGemeinspruch:dasmaginderTheorierichtigseintaugtabernichtfuerdiePraxis1977},
  \cite[][VIII: 276.9--18]{Kant:GesammelteWerke1900ff.}}. Eine ähnliche
  Verwendung findet sich in \titel{Zum ewigen Frieden}
  \mkbibparens{\cite[vgl.][BA 38]{Kant:ZumewigenFrieden1900ff.},
  \cite[][VIII: 357.12--13]{Kant:GesammelteWerke1900ff.}; siehe auch
  \cite[][B 22]{Kant:DieReligioninnerhalbderGrenzenderblossenVernunft1977},
  \cite[][VI: 29.24--30]{Kant:GesammelteWerke1900ff.}}.}
\end{quote}
Aufklärung ist da vonnöten, wo es um unsere je eigenen wesentlichen Zwecke geht.
Dass der Mensch \enquote{seinem wesentlichen Zwecke angemessen} sein solle,
bezieht sich auf die Bestimmung des Menschen, denn wesentliche Zwecke sind der
Endzweck, der in der Bestimmung des Menschen bestehe, und subalterne Zwecke,
die von diesem Endzweck abhängen.\footnote{\cite[Vgl.][B
868]{Kant:KritikderreinenVernunft2003}; \cite[][III:
543.7--12]{Kant:GesammelteWerke1900ff.}.}

\begin{comment}
Die Regeln der Geschicklichkeit verhalten sich neutral gegenüber den
wesentlichen Zwecken. Sie brauchen uns nur zu interessieren, wenn wir
entsprechende Zwecke verfolgen. Verfolgen wir die Zwecke nicht, dann sind wir
frei, uns des Urteils zu enthalten; wir urteilen dann weder unverantwortlich
noch fremdbestimmt, weil wir gar nicht urteilen.
Die Ratschläge der Klugheit und die Gebote der Sittlichkeit artikulieren
hingegen notwendige und vernünftige Zwecke des Handelns, die uns ebenso
angehen, wie die Mittel, mit deren Hilfe wir sie erreichen können.
Klugheit als freie Sorge um das je eigene Wohlergehen ist in der Tat ein
wichtiger Aspekt von Mündigkeit. Aber dieser Aspekt steht nicht alleine da,
sondern ist integriert in die Sorge um das moralisch Richtige. Und wenn
Vertreter der Aufklärung von Nützlichkeit sprechen, dann ist damit nicht bloß
gemeint, dass etwas für kontingente Zwecke oder Befriedigung je eigener
zufälliger Begierden brauchbar ist. Gerade die
notwendigen -- etwa moralisch zwingenden -- Ziele des Handelns \emph{respective}
ihre Erkenntnis gelten dem 18. Jahrhundert als \enquote{nützlich}. Da es im Falle der technischen
Regeln der Geschicklichkeit zufällig ist, ob wir ihrer bedürfen, und wir somit
auf diese verzichten können, lasse ich sie außen vor und konzentriere mich auf
Moral und Klugheit.

\end{comment}

Wenn die Hochaufklärung um \authorcite{Wolff:Psychologiaempirica1968} die
Problematik des Widerstreits von Endlichkeit und Selbständigkeitsforderung durch
einen allgemeinen Erkenntnisoptimismus
überging,\footcite[Vgl.][36]{Engfer:ChristianThomasius1989} dann ist eine
ähnliche Strategie mit Blick auf die Moralphilosophie auch \name[Immanuel]{Kant}
zuzuschreiben. Die Akzentuierung praktischer Erkenntnisse
hat eine gewisse Erleichterung gebracht, insofern die Forderung nach Mündigkeit
von uns dort etwas fordert, was uns nach \name[Immanuel]{Kant} zumindest keine
epistemischen Probleme bereitet. Mit der Aufklärung ist eine \emph{Akzentuierung} der Moral und mit ihr verwandter
Disziplinen verbunden, wenngleich sie keine \emph{Einschränkung} der
Forderung nach Selbstbestimmung auf bestimmte Themenbereiche erlaubt.
\phantomsection\label{Abschnitt:moralischepistemischerOptimismus-Ende}
Nun können wir uns in vielen Fragen unseres Urteils enthalten, wir dürfen
nur nicht unverantwortlich und heteronom urteilen. In Fragen, die unsere
Lebensausrichtungen und unser notwendiges Handeln und Entscheiden betreffen --
Fragen zu unserer \singlequote{Bestimmung} als Menschen --, ist eine solche
Urteilsenthaltung freilich nicht möglich, hier \emph{müssen} wir urteilen. Wie
die \titel{Kritik der reinen Vernunft} zeige, sei aber auch hier jeder
gleichermaßen \emph{fähig} zu urteilen. Denn ihr Ergebnis beinhalte
\begin{quote}
daß die Natur, in dem, was Menschen ohne Unterschied angelegen ist, keiner
parteiischen Austeilung ihrer Gaben zu beschuldigen sei, und die höchste
Philosophie in Ansehung der wesentlichen Zwecke der menschlichen Natur es nicht
weiter bringen könne, als die Leitung, welche sie auch dem gemeinsten Verstande
hat angedeihen lassen.\footnote{\cite[][B
859]{Kant:KritikderreinenVernunft2003}, \cite[][III:
538.11--16]{Kant:GesammelteWerke1900ff.}.}
\end{quote}
\name[Immanuel]{Kant} muss dabei nicht behaupten, dass Aufklärung leicht und
ohne Mühsal wäre.\footnote{So richtet sich der Aufklärungsaufsatz an den
\singlequote{Gelehrten} \mkbibparens{\cite[vgl.][A
485]{Kant:BeantwortungderFrage:WasistAufklaerung?1977}, \cite[][VIII:
37.11--13]{Kant:GesammelteWerke1900ff.}}.} Aufklärung sei \enquote{in Thesi}
leicht, \enquote{in Hypothesi} aber schwer auszuführen. Das heißt, dass sich,
erstens, leicht angeben lässt, was Aufklärung von uns fordert und was sie ihrem
Begriff nach (\enquote{in Thesi}) ist, dass aber, zweitens, ihre Ausführung und
Umsetzung (\enquote{in Hypothesi}) viel schwerer ist, gerade weil sie die
Ausbildung unserer Urteilskraft fordert. Aber die Vernunftkritik helfe doch,
Aufklärung als realistische Option auszuweisen, indem sie zeige, dass
tief schürfende Wissenschaft die wesentlichen Zwecke nicht besser verfolgen kann,
als eine im Austausch mit anderen kultivierte Urteilskraft.

\section{Zusammenfassung und Ausblick}
Wir haben nun gesehen, dass wir unser Augenmerk auf zwei Aspekte
richten müssen: Die \emph{Erkenntnis} dessen, was wir tun sollen, und seine
\emph{Durchführung}.
In Ansehung derjenigen Fragen, die uns als Menschen besonders betreffen,
gefährden keine epistemischen Defizite unsere mündige Lebensführung. Wir
sind in aller Abhängigkeit und Endlichkeit, bei aller Einschränkung unserer
kognitiven Vermögen doch in der Lage, hinreichend viel zu \emph{wissen} und
\emph{kompetent zu beurteilen}, was unsere je eigene Lebensführung anbelangt.
Doch die Endlichkeit unserer \emph{praktischen} Vernunft -- unser endlicher
Wille -- unterminiert mitunter die Durchführung, also die konsequente
Orientierung an der (zweifellos vorhandenen) vernünftigen Einsicht.

Die Fehlbarkeit endlicher Wesen in der Ausübung der praktischen Vernunft besteht
oft nicht darin, dass die entsprechenden Erkenntnisse nicht verfügbar wären, sondern
darin, dass trotz korrekter Einsicht die Ausführung unterbleibt. Den
paradigmatischen Fall finden wir bei moralischen Verfehlungen, bei denen wir
davon ausgehen können, dass kein Mangel an moralischer Einsicht vorliegt,
sondern an korrekter Umsetzung vorhandener Einsichten. Ein nicht-endlicher
(ein \singlequote{heiliger}) Wille unterliegt denselben moralischen Gesetzen,
die auch uns bekannt sind (er ist uns kognitiv möglicherweise bloß gleichwertig);
aber wenngleich er dieselben moralischen Gesetze \emph{erkennt}, so \emph{handelt}
darüber hinaus auch stets nach ihnen.

Doch damit wird das grundlegende Problem des Verweisens auf intellektuelle
Freiheit und Selbständigkeit zwar geschmälert, nicht jedoch gelöst. Wir
verfügen über genügend Wissen, um selbstbestimmt leben und Entscheidungen
treffen zu können. Aber dieses Wissen haben wir in der Regel nicht selbst
generiert, sondern von anderen übernommen. Man denke an medizinisches Wissen,
welches im Rahmen der Klugheit von Bedeutung ist. Dass Zigaretten und Alkohol
ungesund, Obst und Gemüse aber gesund sind, wissen wir, weil wir es -- in der
Regel von unseren Eltern -- gelernt haben. Aber wir können wir uns als frei und
selbständig verstehen, wenn wir lediglich das Wissen in unserem Handeln wirksam
werden lassen, welches wir von anderen übernommen haben? Dieser Frage nach der
Vereinbarkeit von intellektueller Freiheit und Selbständigkeit auf der einen und
dem Erwerb von Wissen durch das Lernen von anderen gehen die folgenden Kapitel
\ref{section:autonomieunddaszeugnisanderer}--\ref{Chapter:KantsSocialEpistemology}
nach. Kapitel \ref{section:autonomieunddaszeugnisanderer} wird dabei den Status
solchen Wissens in der Philosophie der Neuzeit mit Blick auf klassische Ansätze
bei \authorcite{Descartes:OeuvresdeDescartes1983} \name[David]{Hume}, \name[Thomas]{Reid} und
\authorcite{Crusius:WegzurGewissheitundZuverlaessigkeitdermenschlichenErkenntniss1965}
thematisieren, Kapitel \ref{chapter:MuendigerErwerbTestimonialenWissens} zwei
unterschiedliche Perspektiven auf dieses Problem differenzieren und Kapitel
\ref{Chapter:KantsSocialEpistemology} schließlich auf der Grundlage des zuvor
erarbeiteten die Position \name[Immanuel]{Kant}s eruieren.




\chapter{Abhängigkeit und der Erwerb von
Wissen}\label{section:autonomieunddaszeugnisanderer}
Die Forderung nach einem kritischen Umgang mit der je
eigenen geistigen Tradition und nach Unabhängigkeit von geistigen Autoritäten
bei gleichzeitigem Bewusstsein der eigenen Abhängigkeit von Anderen prägt weite
Teile der Aufklärung in Deutschland.
Zeugnis dessen ist der in Vorurteilstheorien der Aufklärung verbreitete
Dualismus von Vorurteilen der Autorität und Vorurteilen aus
\singlequote{Übereilung} oder
Selbstüberschätzung.\footnote{\cite[Vgl.][245--246]{Albrecht:ChristianThomasius1999}.}
Diese Bipolarität wird auch in der {\jaeschelogik}
gelehrt, wenn \name[Immanuel]{Kant} die \emph{Vorurteile} in solche
\emph{des Ansehens} -- einer Person, der Menge oder des Zeitalters (des
Altertums oder der \singlequote{Neuigkeit}) -- und solche \emph{aus Eigenliebe}
-- den \enquote{logische[n] Egoismus} --
unterteilt.\footnote{\cite[Vgl.][A~119--125]{Kant:ImmanuelKantsLogik1977},
\cite[][IX: 77.26--80.32]{Kant:GesammelteWerke1900ff.}. Die Grundlage findet
\name[Gottlob Benjamin]{Jäsche} in den Reflexionen 2563--2582 , \cite[][XVI:
417.12--427.4]{Kant:GesammelteWerke1900ff.}.} Und dieselbe Polarität besteht in
der \titel{Kritik der Urteilskraft} zwischen der Maxime des Selbstdenkens und
der Maxime der \enquote{erweiterten Denkungsart}, sein Urteil stets an dem
Urteil anderer zu messen.\footnote{Siehe hierzu Kapitel
\ref{section:sensuscommunis}.}


Dieses Spannungsverhältnis wirft unmittelbar die Frage auf, wie mit dem
Wissen und den Informationen anderer umzugehen ist, wenn man weder seine eigene
Urteilsfähigkeit überschätzen, noch sich leichtfertig der Autorität anderer
ergeben möchte. Es verwundert daher nicht, in der Literatur der Neuzeit seit
\authorcite{Descartes:OeuvresdeDescartes1983} mannigfache Thematisierungen des Problems testimonialen
Wissens zu finden. Oliver \authorcite{Scholz:DasZeugnisanderer2001} spricht zu Recht entgegen dem Mainstream nicht von einer neuen
Fragerichtung der Erkenntnistheorie\footnote{Dieser Mainstream findet sich noch
in:
\cite[][529--531]{Grundmann:AnalytischeEinfuehrungindieErkenntnistheorie2008},
sowie \cite[][46]{Wilholt:SozialeErkenntnistheorie2007}.}, sondern \enquote{von
einer Renaissance oder Wiedergeburt der erkenntnistheoretischen Diskussion der
Testimonialerkenntnis \punkt , da es im 17. und 18. Jahrhundert bereits eine
sehr extensive und intensive Diskussion gegeben
hat.}\footnote{\cite[][354]{Scholz:DasZeugnisanderer2001}.} Die Tatsache, dass
Überlegungen zu testimonialem Wissen aus der deutschen Aufklärungsphilosophie
innerhalb der aktuellen Debatten (die sich in historischen Rückverweisen fast
ausschließlich auf John \name[John]{Locke}, David \name[David]{Hume} und Thomas
\name[Thomas]{Reid} beziehen) unberücksichtigt bleiben, täuscht leicht darüber hinweg, wie ausführlich gerade
dort darüber diskutiert wurde, wann wer welche Erkenntnisse auf das Zeugnis
eines anderen hin übernehmen darf.


Tatsächlich sind gerade die Themen der Sozialen Erkenntnistheorie -- wie
\authorfullcite{Goldman:Experts:WhichOnesShouldYouTrust?2001} schreibt -- \enquote{practically quite
pressing.}\footnote{\cite[][85]{Goldman:Experts:WhichOnesShouldYouTrust?2001}. 
Besonders alltagsnah sei die Herausforderung, zu bestimmen \enquote{how
lay-persons should evaluate the testimony of experts and decide which of two or
more rival experts is most credible. It is of practical importance because in a
complex, highly specialized world people are constantly confronted with
situations in which, as comparative novices (or even ignoramuses), they must
turn to putative experts for intellectual guidance or assistance}
\parencite[][85]{Goldman:Experts:WhichOnesShouldYouTrust?2001}.} Denn bezüglich
der Fragen danach, welchen Informationen wir vertrauen sollen und wer zu Recht
als Experte gilt, stellt die moderne Informationsgesellschaft uns, die vom
Anspruch nach Mündigkeit geprägten Nachfahren der Aufklärung, tatsächlich jeden
Tag vor handfeste erkenntnistheoretische Probleme. Und so sehr sich die
Situation auch durch moderne Informationstechnologien wie das Internet
verschärft haben mag, ist sie doch strukturell mit der Situation im
18.~Jahrhundert vergleichbar. Solche \enquote{pragmatischen} Fragestellungen
passen daher wie wenige andere in die neuzeitlich-aufklärerische Tradition
\singlequote{praktischer Logiken}.\footnote{Siehe hierzu oben
Anm.~\ref{Fussnote:PraktischeLogiken}, S.~\pageref{Fussnote:PraktischeLogiken}.}
Bevor auf deren Inhalt eingegangen wird, ist noch einiges zum historischen
Hintergrund und zur allgemeinen Philosophiegeschichtsschreibung anzumerken. Das
vorherrschende Geschichtsbild bezüglich testimonialen Wissens fokussiert auf
David \name[David]{Hume} und Thomas \name[Thomas]{Reid} sowie, als
individualistischen Ausgangspunkt, der zumindest teilweise überwunden wird,
Ren{\'e} \authorcite{Descartes:OeuvresdeDescartes1983}. Bevor ich also auf Vertreter der deutschsprachigen
Aufklärung und speziell auf \name[Immanuel]{Kant} zu sprechen komme, soll dieser historische Hintergrund zumindest kurz skizziert
werden. Dies wird insbesondere den Problemhorizont erkennbar werden lassen, der
den systematischen Rahmen für die Entwicklung der Konzeption \name[Immanuel]{Kant}s zu
testimonialem Wissen darstellt. Den Anfang machen jedoch einige
systematisch-begriffliche Überlegungen zum Themenfeld, die bei der
Charakterisierung der zu behandelnden Autoren unverzichtbar sind.


\section[Vorbemerkungen zur Sozialen Erkenntnistheorie]{Systematische und
begriffliche Vorbemerkungen zur
Sozialen
Erkenntnistheorie}\label{section:SystematischeundbegrifflicheVorbemerkungen}


In Kapitel \ref{chapter:AufklaerungundWissenschaft} war zu sehen, dass die
Konfrontation der Aufklärungsforderung nach Mündigkeit und Selbstdenken mit der
epistemischen Situation des wissenschaftlichen Laien -- und ein solcher ist
jeder von uns zumindest bezüglich der meisten Disziplinen -- von
\name[Immanuel]{Kant} dadurch entschärft wird, dass er bestimmte Bereiche des Wissens und Denkens
akzentuiert. Moral und Klugheit sind die Kernkompetenzen des mündigen Menschen,
der zu seiner Mündigkeit noch Wissen um die \emph{conditio humana} benötigt,
nicht aber in Wissenschaften bewandert sein muss. In Fragen der Moral
und Klugheit benötigen wir darüber hinaus keine spezielle Methodenausbildung,
sondern müssen unser vorhandenes Wissen zur Anwendung bringen und unsere
Urteilskraft im intellektuellen Austausch mit anderen entwickeln. Selbstdenken ist -- so zeigte Kapitel
\ref{section:KantalsliberalerAufklaerer} -- eine genuin soziale Angelegenheit.
Es verweist nicht auf die autarke Wissensgewinnung des einsamen epistemischen
Heroen, der allein im Hausrock am Kamin das Wissen der Welt rekonstruiert,
sondern auf die aktive Teilnahme an einer als republikanisch verstandenen
Vernunft. Der Selbstdenker versteht sich und handelt als gleichberechtigter
Teilnehmer der \singlequote{\emph{r{e}publique des lettres}}.

Nun bringt jedoch die Tatsache, dass jeder von uns in den meisten Bereichen Laie
ist, mit sich, dass wir in der Regel gar nicht als gleichberechtigte Teilnehmer
eines intellektuellen Austausches auftreten können. Wir erlangen viel Wissen
dadurch, dass wir von anderen erfahren, was der Fall ist. Wir lesen über das
Geschehen in der Welt in einer Zeitung; wir informieren uns über
wissenschaftliche Erkenntnisse in Fachzeitschriften oder
populärwissenschaftlichen Magazinen; wir fragen Freunde und Bekannte, die sich
mit etwas besser auskennen als wir. Und gerade das Wissen, das unserer Klugheit
zugrunde liegt, entstammt zu einem erheblichen Teil nicht eigener Erfahrung,
sondern der Erziehung, die uns zuteil wurde, als wir selbst Kinder waren. All
dies sind Formen des Wissenserwerbs, in denen wir nicht mit gleichberechtigter
Stimme auftreten, in denen wir aber auch nicht sagen, dass sie unsere Mündigkeit
gefährden.

Zu Beginn möchte ich eine naheliegende, aber dennoch unbefriedigende Antwort auf
diese Frage zurückweisen. In Kapitel \ref{section:KantalsliberalerAufklaerer}
sagte ich, dass die Forderung der Aufklärung nach Selbstbestimmung ihr Augenmerk
vor allem auf handlungsorientierende Erkenntnisse in Ethik und Religion sowie
Fragen der Klugheit richtet. Wir sollen -- so ließe sich nun mutmaßen -- bei
Fragen, auf die naturwissenschaftliche Antworten schon vorhanden (oder doch
zumindest möglich) sind, auf Experten vertrauen, nicht aber in Fragen der Moral
und Religion sowie der Ausrichtung des eigenen Lebens auf Glückseligkeit hin.
Hierfür sind zwei Gründe denkbar:
\begin{nummerierung}
\item Der erste denkbare Grund wäre, dass eine wissenschaftliche Behandlung von
Moral und Religion und sicheres Wissen in diesen Bereichen \emph{niemandem}
möglich seien. Einen solchen Gedankengang erwähnte ich oben zu
\name[Christian]{Thomasius} bezüglich der Privatheit religiöser Überzeugung.\footnote{Siehe
S.~\pageref{ThomasiusZuPrivatheitreligioesenBekenntnisses}.} Wenn \name[Immanuel]{Kant}
nun behauptet, das Wissen begrenzt zu haben, um dem Glauben Platz zu verschaffen
und den Einfluss philosophischer \singlequote{Schulen} zu
reduzieren,\footnote{\cite[Vgl.][B
xxx--xxxii]{Kant:KritikderreinenVernunft2003}, \cite[][III:
19.5--37]{Kant:GesammelteWerke1900ff.}.} dann lässt sich dies \emph{prima facie}
ebenso interpretieren. Jedoch scheint mir eine solche Interpretation nicht
sonderlich fruchtbar zu sein. Denn zunächst ist offensichtlich, dass eine solche
Position gerade die Moral betreffend nicht \name[Immanuel]{Kant}s Ansicht ist.
Bei Fragen danach, wie wir handeln sollen, gelte vielmehr, dass \emph{jeder} sicheres Wissen besitzen
kann.\footnote{Siehe hierzu die Überlegungen zum
\enquote{moralisch-epistemischen Optimismus} auf den Seiten
\pageref{Abschnitt:moralischepistemischerOptimismus}--\pageref{Abschnitt:moralischepistemischerOptimismus-Ende}.}
Wenngleich dies bezüglich \emph{religiöser} Überzeugungen nicht ganz so leicht
einzusehen ist, postuliert \name[Immanuel]{Kant} doch zumindest, dass wir in
Fragen des \emph{moralisch} Richtigen \emph{alle} kompetent urteilen können (mit
Ausnahme vielleicht von kleinen Kindern und Menschen mit besonderen geistigen
Anomalien). Doch auch in Fragen der Religion distanziert sich
\name[Immanuel]{Kant} von der Vorstellung, ein aufgeklärter Glaube sei ein solcher, der dem jeweiligen
individuellen Belieben anheim gestellt und keinem intersubjektiv gültigen
Maßstab der Vernünftigkeit unterworfen sei. \name[Immanuel]{Kant} bewertet den
Religionsglauben (im Gegensatz zum Kirchenglauben) auf der Grundlage der
\emph{praktischen} statt der \emph{theoretischen} Vernunft, aber er bewertet ihn
damit tatsächlich an einem Maßstab der \emph{Vernunft}. Am ehesten ließe sich
eine Forderung nach \emph{eigenem} Urteilen mangels fundierten Wissens aus
\index{Kant, Immanuel}kantischer Sicht heraus noch im Bereich der \emph{Klugheit} --
speziell der \emph{Privat}klugheit -- vertreten.\footnote{Siehe hierzu Kapitel
\ref{subsection:aufklaerungundpraxis}.} Wenngleich es allgemein (für jeden
Menschen) gültige Regeln in Bezug auf das Erreichen eigenen Glücks geben mag,
sei dieses doch von so vielen Faktoren und insbesondere von biographischen
Zufälligkeiten abhängig, dass es sich doch immer nur um unsichere und
unverbindliche \enquote{Ratschläge} handle.

Hier spielt möglicherweise auch eine Rolle, dass Privatklugheit sich auf
Eigenschaften des Subjekts bezieht, zu denen dieses einen besonderen
\singlequote{privilegierten Zugang} zu haben scheint. Um zu wissen, was mich
glücklich macht, muss ich bei meinen Handlungen auf meine eigenes Gefühl der
Lust oder Unlust achten; und zu meinen Gefühlen der Lust und Unlust stehe ich
selbst in einer besonderen Relation, die mein Urteil diesbezüglich privilegiert.
Die Frage, ob ich lieber mit Freunden ausgehe, Sport treibe oder den Abend vor
dem Fernseher verbringe, kann ich nicht delegieren, weil außer mir niemand die
Antwort kennen kann. Denn dazu muss man wissen, welche Gefühle diese Tätigkeiten
in mir auslösen; und dazu habe wiederum nur ich Zugang. Es liegt -- sollte diese
Theorie eines privilegierten Zugangs wahr sein\footnote{Es gibt durchaus Gründe,
um mindestens die Allgemeinheit einer solchen Behauptung zu bestreiten. Man
beachte etwa das Phänomen der Selbsttäuschung: Wir können uns durchaus darin
täuschen, was uns selbst Freude bereitet; bspw.
wenn mit unseren tatsächlichen Neigungen ein Bild davon konfligiert, wie wir
gerne wären. Jemand mag sich selbst sagen, er möge eine aktive
Freizeitgestaltung, obwohl er sie stets als stressig erlebt und lieber zuhause
auf dem Sofa sitzt. Andere können eine solche Selbsttäuschung mitunter
durchschauen.} -- eine epistemische Asymmetrie in der Natur der Sache.
Allerdings wäre dieser Ansatz nur auf der Grundlage eines Intuitionismus oder
Emotivismus in der Metaethik auf Fragen der Moral
übertragbar und fällt für alle Bereiche außerhalb der Privatklugheit somit als
Erklärungsansatz innerhalb des von \name[Immanuel]{Kant} gesteckten Rahmens aus.
Da er nicht mehr als die Privatklugheit betreffen kann und bei der Explikation
des Begriffs der Mündigkeit keine weitere Funktion übernehmen wird, genügt hier
dieser allgemeine Verweis, ohne dass ich damit behaupten wollte, die Theorie
des privilegierten Zugangs überzeuge oder entspreche den Ansichten
\name[Immanuel]{Kant}s.

\item Der zweite denkbare Grund resultiert aus der gegenteiligen Annahme, dass
\emph{jeder} sicheres Wissen in diesen Bereichen erwerben könne. Eine solche
Annahme harmoniert besser mit \name[Immanuel]{Kant}s moralisch-epistemischen
Optimismus.\footnote{Siehe oben, S.
\pageref{Fussnote:moralischepistemischerOptimismus}.} Und auch diese Annahme
spricht für unsere Forderung nach Selbstdenken in den entsprechenden Bereichen.
Denn sowohl dann, wenn jeder das nötige Wissen hat, wie auch dann, wenn niemand
darüber verfügt, entfällt jede Grundlage für eine Berufung auf Experten, die nur
sinnvoll scheint, wo epistemische Asymmetrien vorliegen, also dort, wo
verschiedene Menschen unterschiedlich kompetent urteilen können. Das Dilemma von
Mündigkeit und Abhängigkeit erweist sich hier im Grunde als ein solches von
Mündigkeit als Forderung der Aufklärung und epistemischer Asymmetrie als dem
bestimmenden Merkmal der modernen arbeitsteiligen Wissenschaftsorganisation.
Epistemische Asymmetrie liegt dabei nicht \emph{per se} vor, sondern immer in
Bezug auf bestimmte Themengebiete und Kompetenzbereiche. Nur dort, wo eine
solche Asymmetrie vorliegt und wir uns auf der nicht-privilegierten Seite
befinden, scheint der Forderung der Aufklärung nach Selbstdenken nicht genügt zu
werden. Denn es ist nicht verständlich, warum wir uns auf unser eigenes Urteil
in Fällen verlassen sollen, in denen wir unsere epistemische Situation dadurch
signifikant verschlechtern.
\end{nummerierung}

Zur Steigerung der Plausibilität lassen sich Kombinationen und Variationen aus
beiden Gründen anführen. So ließe sich vorschlagen, den ersten Grund in Bezug
auf religiöse Überzeugungen und den zweiten Grund in Bezug auf Moral und Ethik
vorzubringen. Wir sollen dann in Fragen der Religion unseren eigenen
Überzeugungen folgen, weil objektives Wissen nicht möglich und uns infolge
dessen niemand epistemisch überlegen ist, und in moralischen Fragen, weil wir
selbst kompetent beurteilen können, was moralisch richtig ist.\footnote{Wir
können die Trennlinie auch anders verlaufend denken, bspw.~zwischen einem
vernünftigen Kern der Religion, über den jeder kompetent urteilen kann
(Vernunftreligion) und zusätzliche Komponenten, die sich jedem kompetenten
Urteil entziehen (Kirchenglauben). Da ich den Lösungsvorschlag in Gänze
zurückweise, sind solche Fragen hier nicht relevant.} Oder wir reklamieren den
ersten Grund für Fragen der Klugheit und den zweiten Grund für Fragen der Ethik,
Moral und Religion. Es ist diese Konstellation, die den Aussagen \name[Immanuel]{Kant}s am
nächsten zu kommen scheint. Denn während er sich sehr skeptisch bezüglich der
Möglichkeit einer allgemeinen Klugheitslehre äußert, spricht er jedem die
Fähigkeit zu, kompetent über Moral und Religion zu urteilen, wenn denn nur
\enquote{Satzungen und
Formeln}\footnote{\cite[][A~483]{Kant:BeantwortungderFrage:WasistAufklaerung?1977},
\cite[][VIII: 36.8--10]{Kant:GesammelteWerke1900ff.}: \enquote{Satzungen und
Formeln, diese mechanischen Werkzeuge eines vernünftigen Gebrauchs oder vielmehr Mißbrauchs
seiner Naturgaben, sind die Fußschellen einer immerwährenden Unmündigkeit.}},
also beispielsweise die Bindung durch Dogmen eines kirchlichen Lehramtes, überwunden sind.

Ein solcher Vorschlag versucht, das Dilemma von Mündigkeit und epistemischer
Abhängigkeit dadurch aufzulösen, dass Mündigkeit und Abhängigkeit jeweils eigene
epistemische Bereiche zugewiesen werden. Abhängig und auf Experten angewiesen
sind wir bei Angelegenheiten der Wissenschaft, weil dort epistemische
Asymmetrien vorliegen; mündig sein sollen wir bei Fragen der Moral/Ethik und
Religion sowie vielleicht in einigen (aber möglicherweise nicht in allen)
Fragen der (Privat\mbox{-)} Klugheit, wo keine solchen Asymmetrien vorliegen,
sondern die Ausgangsbedingungen gleich oder annähernd gleich sind. Welche Auswirkungen beispielsweise bestimmte Stoffe auf den
menschlichen Organismus haben, können andere oft besser beurteilen als ich
selbst, wenn sie die entsprechende Expertise haben. Ob meine Handlung mir
selbst Freude bereitet, das muss ich hingegen selbst beurteilen; denn wer sollte
hier mehr wissen als ich selbst?


Dieser Vorschlag wirkt verlockend, aber er erreicht -- wie noch zu sehen sein
wird -- sein Ziel nicht. Er liefert uns weder einen Begriff von Mündigkeit und
Selbstdenken, noch hilft er uns in unserem \name[Immanuel]{Kant}verständnis.
Zunächst lässt er sich nicht mit dem Wortlaut der \index{Kant,
Immanuel}kantischen Schriften in Einklang bringen, denn \name[Immanuel]{Kant}
sagt nicht, wir sollten uns bei bestimmten Fragen aus unserer Unmündigkeit
herausarbeiten und den Mut aufbringen, unseren eigenen Verstand zu gebrauchen.
Stattdessen betont er, wir sollen dies \emph{immer} tun. Aufklärung sei die
Maxime, \emph{jederzeit} selbst zu denken,\footnote{\cite[Vgl.][A
329]{Kant:Washeisst:SichimDenkenorientieren?1977},
\cite[][VIII: 146.30--31]{Kant:GesammelteWerke1900ff.}.} und die Maxime einer \emph{niemals}
passiven, also \emph{immer} aktiven Vernunft zeichne das vorurteilsfreie Denken
aus\footnote{Vgl. \cite[][\S~40]{Kant:KritikderUrteilskraft2009}, \cite[][V:
294.20]{Kant:GesammelteWerke1900ff.}.}. Aufklärung heiße dann auch, \enquote{mit
seiner Vernunft nicht passiv, sondern \myemph{jederzeit} sich selbst
gesetzgebend zu sein}\footnote{\cite[][\S~40]{Kant:KritikderUrteilskraft2009},
\cite[][V:
294.30--31]{Kant:GesammelteWerke1900ff.}.}. Unter den Vorurteilen ist wiederum
der Aberglaube das schlimmste, welcher darin bestehe, die Natur als den Gesetzen
unseres Verstandes nicht unterworfen zu
denken.\footnote{\cite[Vgl.][\S~40]{Kant:KritikderUrteilskraft2009}, \cite[][V:
294.22--24]{Kant:GesammelteWerke1900ff.}.} Aberglaube drückt sich zum Beispiel
im Glauben an die Wirkung von Gebeten, an die Existenz von Feen und
Hexen oder an Horoskope aus. Das sind alles aber keine moralischen Fehlurteile und auch
keine Urteile, die sich allein aufgrund der praktischen Vernunft korrigieren
lassen. Stattdessen handelt es sich wenigstens zu einem beachtlichen Teil um
Fehlurteile der theoretischen Vernunft, deren Korrektur wir den modernen
Wissenschaften zuschreiben, die uns vor das Dilemma von Mündigkeit und
epistemischer Asymmetrie stellen.\footnote{Auch \name[Immanuel]{Kant} sieht
hier Fehlurteile der theoretischen Vernunft vorliegen; aber er schreibt ihre
Korrektur nicht der empirischen Forschung, sondern der Metaphysik zu; siehe
dazu unten Kapitel \ref{subsection:MetaphysikundAutonomie}.}

Unsere Begriffe von Aufklärung und Mündigkeit haben sehr viel damit zu tun, wie
jemand mit seiner epistemisch nachteiligen Position gegenüber Experten in
unserer modernen Wissensgesellschaft umgeht. Deshalb befriedigt der Vorschlag,
Selbstdenken und Mündigkeit auf Bereiche zu beschränken, in denen es kein
Expertentum geben kann, auch systematisch nicht. Es fiele uns zu Recht schwer,
jemanden als aufgeklärt zu bezeichnen, der von Geozentrismus, Phlogistontheorie
oder einem Erdalter von etwa 7000 Jahren überzeugt ist, weil er selbsternannten
\singlequote{Experten} auf diesem Gebiet folgt oder solche Theorien selbst für
irgendwie \singlequote{plausibler} hält. Gerade Verfechter religiös
inspirierter Gedankengebäude wie des \emph{Intelligent Design} berufen sich
gerne auf das Selbstdenken und eine vermeintlich größere
\singlequote{Plausibilität} solch einfacher Erklärungen gegenüber den
komplexeren und dem Alltagsdenken entfernteren wissenschaftlichen Theorien. Sie
bringen damit -- gänzlich gegen die Intention der Aufklärung -- den
\singlequote{gemeinen} (hier im Sinne von \emph{vulgaris}, nicht von
\emph{communis}) Menschenverstand mit seinen Vorurteilen gegen die
wissenschaftliche Vernunft in Stellung.


Ob im Gegenzug mündig und aufgeklärt ist, wer zwar den \singlequote{richtigen}
Experten (Menschen mit tatsächlicher wissenschaftlicher Expertise auf dem
betreffenden Gebiet), diesen aber ebenso \emph{alles} glaubt, darf zumindest
bezweifelt werden. Zur Mündigkeit gehört wesentlich auch der \emph{kritische}
Umgang mit unserer Wissenschafts- und Informationsgesellschaft. Es sollte also
möglich sein, beide Aspekte so zu vereinen, dass Abhängigkeit und Mündigkeit bei
denselben Überzeugungen \emph{zugleich} vorliegen können. Sollte dem nicht so
sein, stellte dies die Aufklärung vor ein Problem, aus dem es kein Entrinnen
gäbe. Wir müssten ihre Grundintention dann entweder in der Gewinnung und in der
Verbreitung wahrer wissenschaftlicher Ergebnisse sehen, könnten aber ihre
Forderung nach Mündigkeit und Selbstdenken nicht mehr integrieren. Oder wir
sähen in der Forderung nach Mündigkeit und Selbstdenken eben die Grundintention;
dann aber ginge Aufklärung nicht mehr mit der neuzeitlichen
Wissen\-schafts\-ent\-wick\-lung einher, sonder stünde ihr diametral entgegen.
Als Zwischenfazit ergibt sich Folgendes: \emph{Das Grundproblem jeder
Aufklärungsphilosophie betrifft den \emph{gemeinsamen}, aber dennoch individuell
\emph{mündigen} Umgang mit Wissen und Erkenntnis gerade angesichts epistemischer
Asymmetrien.}



% \subsection{Rekonstruktion testimonialen Wissens: Reduktionismus, Skeptizismus
% und Credulismus}\label{subsection:ReduktionismusSkeptizismusCredulismus}
Einer verbreiteten philosophiehistorischen Auffassung zufolge befassen sich
Autoren der Aufklärung, die sich erkenntnistheoretischen Fragen widmen,
bevorzugt mit dem Problem, wie ein auf sich allein gestelltes Subjekt ohne
Interaktion mit anderen zu Wissen gelangt.\footnote{Siehe etwa
\cite[][passim]{Grundmann:DietraditionelleErkenntnistheorieundihreHerausforderer2001}.}
Stimmte dies, so versäumte es die Aufklärungsphilosophie, ihr eigenes
Grundproblem überhaupt zu fassen zu bekommen, geschweige denn zu lösen. Sie
steht vor der Aufgabe, sich mit dem
\emph{gemeinsamen} Umgang mit Wissen und Erkenntnis auseinanderzusetzen. Fragen
des Umgangs mit Wissen und Erkenntnissen, die wir durch Interaktion mit Anderen
haben, werden heute unter dem Schlagwort \enquote{\emph{Social Epistemology}} oder
\enquote{Soziale Erkenntnistheorie} diskutiert.\footnote{Einen Überblick über
diese Disziplin geben Thomas
\textcite[vgl.][529--541]{Grundmann:AnalytischeEinfuehrungindieErkenntnistheorie2008}
und Torsten \textcite[vgl.][]{Wilholt:SozialeErkenntnistheorie2007}. Klassische
Beiträge finden sich in den Sammelbänden von
\textcite[vgl.][]{Schmitt:SocializingEpistemology1994a} und von
\textcite[vgl.][]{Matilal:KnowingfromWords1994} sowie wiederum bei
\textcite[vgl.][]{Schmitt:SpecialIssue:SocialEpistemology1987}. Klassische
Arbeiten sind außerdem die Monographien von Alvin Ira
\textcite[siehe][]{Goldman:KnowledgeinaSocialWorld1999} und Tony
\textcite[][]{Coady:Testimony1992}.} Es fragt sich somit, ob es in der
Aufklärungsphilosophie bereits Vorläufer der heutigen \emph{Sozialen
Erkenntnistheorie} gibt. Wenn es entsprechende Überlegungen nicht geben sollte,
wäre dies für die Bewertung der Reflexion der Aufklärer über ihr eigenes Projekt
fatal. Innerhalb dieser Teildisziplin der Epistemologie werden in der Regel zwei
Arten des Eingebundenseins unseres Wissens und Erkennens in eine epistemische
Gemeinschaft diskutiert:
\phantomsection\label{Abschnitte:ZweiThemenSozialerErkenntnistheorie}
\begin{nummerierung}
\item Zum einen übernehmen wir oft Erkenntnisse von anderen, die diese bereits
besitzen. Wissen, welches wir von anderen haben, also etwas, was uns erzählt
wurde, was wir in einem Buch gelesen oder im Radio gehört haben, nennen wir
heute \emph{testimoniales Wissen}. Zu \name[Immanuel]{Kant}s Zeiten war hierfür
der Ausdruck \enquote{Glaube} (fides) oder -- eindeutiger --
\enquote{historischer Glaube} üblich, der aber einige terminologische
Schwierigkeiten mit sich bringt, weswegen ich hier der Eindeutigkeit halber auf
den moderneren \emph{terminus technicus} \enquote{testimoniales Wissen}
zurückgreife. Analog zu testimonialem Wissen werde ich dann auch von
testimonialen Überzeugungen, dem testimonialen Wissenserwerb und Ähnlichem
sprechen. Es ist wichtig zu beachten, dass die Mitteilung durch einen
anderen\footnote{Ich verwende die Ausdrücke \enquote{Mitteilung},
\enquote{mitteilen} und \enquote{mitteilbar} in der Regel anders als
\name[Immanuel]{Kant}, der dabei nur manchmal an (genuin) testimoniales Wissen denkt,
insgesamt aber einen weiteren Begriff verwendet. Es gibt auch Formen der
Mitteilung\textsubscript{Kant}, die nicht auf testimoniales Wissen
verweisen (siehe dazu Kapitel
\ref{subsubsection:EndlichesundUnendlichesErkennen}).} und das Vertrauen in
denjenigen hier \emph{konstitutiven} Charakter trägt. Wie Elizabeth
\authorcite{Anscombe:Intention2000} betont, ist das Phänomen, dass wir
\emph{jemandem glauben}, grundlegend unterschieden von Situationen, in denen wir
\emph{etwas glauben}, nachdem oder auch weil uns jemand dieses (oder etwas
anderes) erzählt hat, ohne dass das Vertrauen in die entsprechende Person
grundlegend wäre.\footcite[Vgl.][4]{Anscombe:WhatIsIttoBelieveSomeone2008} Wenn
uns jemand etwas mitteilt, was wir ohnehin wussten oder wofür wir unabhängige
Erkenntnisgründe haben, oder wenn uns jemand auf etwas aufmerksam macht, was wir
sonst nicht bemerkt hätten, liegt kein testimoniales Wissen vor, sondern nur
dann, wenn wir etwas \emph{auf die Autorität eines anderen hin} für wahr
halten.\footnote{Siehe zu diesem Punkt
\cite[][398--400]{Hawley:TestimonyandKnowingHow2010}.} Ich werde von
\enquote{genuin testimonialem Wissen} sprechen, wenn ich diesen Punkt betonen
möchte, setze es aber auch bei dem Begriff \enquote{testimoniales Wissen}
voraus. Testimoniales Wissen ist Wissen, das auf einem Wissens\emph{transfer}
beruht, dem eine epistemische Asymmetrie zugrunde liegt. Nur wenn jemand
anderes etwas weiß, was ich nur durch Berufung auf die Mitteilung als
gerechtfertigt ansehen kann, vertraue ich auf eine Mitteilung und gründe meine
eigene Überzeugung darauf. Diese Asymmetrie ist die notwendige -- wenngleich
noch nicht hinreichende -- \emph{Bedingung} testimonialen Wissens.

Es gibt nun zwei Personengruppen, von denen wir testimoniales Wissen bekommen:
Einerseits \emph{Experten}, also Menschen, die sich auf einem Themengebiet besser auskennen
als ihre Mitmenschen, etwa weil sie aufgrund ihre Ausbildung spezielle
epistemische Fähigkeiten und Kenntnisse besitzen, und \emph{Zeugen}, die zwar keine
besonderen Kenntnisse oder Fähigkeiten besitzen, aber etwas erfahren haben, was
uns noch unbekannt ist. Experten besitzen in der Regel erweiterte Kompetenzen,
Zeugen befanden sich in besonderen epistemischen Gelegenheiten; beides
privilegiert bestimmte Subjekte gegenüber anderen und zieht eine epistemische
Asymmetrie nach sich. Als Oberbegriff über beide Gruppen werde ich hier den
Ausdruck \enquote{\emph{Informant}} verwenden; Informanten sind also in diesem
Zusammenhang Menschen, die sich in epistemisch privilegierten Positionen
befinden, weil sie als Zeugen oder Experten etwas wissen, was wir (als nicht
privilegiert) nur unter Rückgriff auf ihr Wissen zu unserem eigenen Wissen
machen können. Im allgemeinen lasse ich offen, ob wir testimoniales Wissen
erwerben \emph{können} oder über eine bestimmte Person tatsächlich erwerben und
ob der Wissenstransfer legitim ist oder wäre oder die Kriterien für Mündigkeit
verletzt; auch können sowohl Einzelpersonen als auch Personengruppen als
Informanten bezeichnet werden. Und wenngleich der Ausdruck \enquote{testimonial}
sprachlich auf den Zeugen (und nicht auf den Experten) verweist, werde ich ihn
hier in Bezug auf Wissen auf der Grundlage beider Personengruppen verwenden.
Testimoniales Wissen ist dann also Wissen, welches jemand durch Rückgriff auf
einen Informanten erworben hat.

\item Zum anderen findet die Generierung neuen Wissens in aller Regel in der
Form kooperativer Forschung statt. Dies schließt an die grundlegenden
Überlegungen zu Selbstdenken und Vernunft in Kapitel
\ref{section:sensuscommunis} an.
Wissenschaft ist kein Unterfangen genialer Einzelkämpfer, die je individuell ihr
Fach bereichern und sich dann bloß hinterher ihre neuen Erkenntnisse
mitteilen.\footnote{Siehe hierzu
\cite[][49--52]{Wilholt:SozialeErkenntnistheorie2007}.} Erkennen -- zumal
wissenschaftliches -- beruht auf der Teilnahme an gemeinsamen
Erkenntnisprozessen, dem Erlernen von gemeinsam Methoden richtigen Erkennens,
der gemeinsamen Kontrolle der Ergebnisse und nicht zuletzt dem vorgängigen
Erwerb einer gemeinsamen Sprache. Subjekte sind möglicherweise auch dort
aufeinander angewiesen, wo keine epistemische Asymmetrie vorliegt. Ich spreche
von Intersubjektivität, um solche
Phänomene symmetrischer epistemischer Abhängigkeiten zu bezeichnen. Dabei
stellen asymmetrische epistemische Abhängigkeiten für das Projekt
intellektueller Selbständigkeit und Mündigkeit eine größere Gefahr dar
als die Abhängigkeit von anderen, denen gegenüber sich das Subjekt als
ebenbürtig verstehen kann. Wir wir gesehen haben, ist Intersubjektivität nicht
nur mit Selbstdenken und Mündigkeit vereinbar, sondern sogar eine Voraussetzung
kompetenten eigenen Urteilens, wenn es nur in der Form selbständiger Teilnahme,
also ohne Asymmetrien und Hierarchien geschieht.\footnote{Siehe Kapitel
\ref{section:sensuscommunis}.}
\end{nummerierung}

Wir gehen normalerweise davon aus, \emph{dass} es möglich ist, testimoniales
Wissen zu haben. Diese Annahme der Möglichkeit testimonialen
Wissens ist so fest in unserem Denken verankert, dass vorgeschlagen wurde, darin
eine Grundbedingung für jede angemessene philosophische Explikation des
Wissensbegriffs zu
sehen.\footnote{\phantomsection\label{Fussnote:WissensbegriffunddasZeugnisanderer}\cite[Siehe][57--58,
63]{Fricker:TheEpistemologyofTestimony1987}. \enquote{[W]e must
take the principle that \ori{knowledge can be spread through language-use} as a
constraint on our theorising about what knowledge is: it is a condition of
adequacy on an account of knowledge that it have this consequence.}
(\cite[][\pno~57\,f.]{Fricker:TheEpistemologyofTestimony1987}.) Siehe auch
\cite[][198]{McDowell:KnowledgebyHearsay1994}, wo die Möglichkeit aber nur
Bedingung eines adäquaten Begriffs testimonialen Wissens darstellt.
\enquote{[I]f a knowledgeable speaker gives intelligible expression to his
knowledge, it may become available at second hand to those who understand what
he says} (\cite[][417]{McDowell:KnowledgebyHearsay1994}).} Mir scheint eine
solche Vorgabe für den Wissensbegriff korrekt zu sein -- aus Gründen, die ich
gleich vorlegen werde. Zunächst sei bemerkt, dass mit der Einsicht, \emph{dass}
testimoniales Wissen möglich ist, noch nicht gesagt ist, \emph{wie} es möglich
ist. Handelt es sich bei testimonialem Wissen um eine eigenständige
Erkenntnisquelle neben Sinnlichkeit und Verstand? Gibt es ein
genuines Erkenntnisprinzip, das uns sagt, dass Mitteilungen anderer zumindest
prima facie zu trauen sei? Oder haben wir testimoniales Wissen, weil wir
Erfahrungswissen davon haben, dass andere bestimmte Aussagen tätigen und dass
von solchen Behauptungen in bestimmten Situationen auf das Vorliegen
entsprechender Sachverhalte geschlossen werden kann? Handelt es sich bei
testimonialem Wissen also um inferentielle Erkenntnisse, die auf einem logischen
Schluss beruhen?

Der Begriff einer \emph{Erkenntnisquelle} oder Quelle des Wissens (\emph{fountain of
knowledge}) findet sich bei \authorfullcite{Locke:TheWorksofJohnLocke1963}, der im \titel{Essay
Concerning Human Understanding} schreibt, diese Quellen lägen gänzlich in der
Erfahrung -- der äußeren Wahrnehmung (\emph{sensation}) einerseits und der
inneren Selbstwahrnehmung (\emph{reflexion}) andererseits.\footnote{\enquote{Our
observation employed either about external sensible objects, or about the
internal operations of our minds, perceived and reflected on by ourselves, is
that which supplies our understandings with all the material of thinking.
These two are the fountains of knowledge, from whence all the ideas we have,
or can naturally have, do spring} \mkbibparens{\cite[][Buch II, Kap.
I, \S~2]{Locke:AnEssayConcerningHumanUnderstanding1963}, in: \cite[][I:
82\,f.]{Locke:TheWorksofJohnLocke1963}}.} Den Erkenntnisquellen entspringen aber
keine Begründungen für Urteile, sondern das \emph{Material} des Denkens, also
die Inhalte, aus denen wir unsere Urteile bilden. Es sind die Vorstellungen
(\emph{ideas}), die den beiden Erkenntnisquellen entspringen, und unseren Geist,
der zunächst eine \emph{tabula rasa} (\enquote{white paper, void of all
characters, without any ideas}\footnote{\cite[][Buch II, Kap.
I, \S~2]{Locke:AnEssayConcerningHumanUnderstanding1963}, in: \cite[][I:
82\,f.]{Locke:TheWorksofJohnLocke1963}.}) sei, mit Inhalten des Denkens
versorgen, aus denen er dann Urteile bilden und Wissen generieren kann.

\name[Immanuel]{Kant} an äußert sich in der
ersten Auflage der \titel{Kritik der reinen Vernunft} sehr ausführlich zu
Erkenntnisquellen. Dabei benennt er drei konkrete Quellen -- Sinn,
Einbildungskraft und Apperzeption:
\begin{quote}
Es sind drei subjektive Erkenntnisquellen, worauf die Möglichkeit einer
Erfahrung überhaupt, und Erkenntnis der Gegenstände derselben beruht:
\ori{Sinn}, \ori{Einbildungskraft} und \ori{Apperzeption}; jede derselben kann
als empirisch, nämlich in der Anwendung auf gegebene Erscheinungen betrachtet
werden, alle aber sind auch Elemente oder Grundlagen a priori, welche selbst
diesen empirischen Gebrauch möglich machen. Der \ori{Sinn} stellt die
Erscheinungen empirisch in der \ori{Wahrnehmung} vor, die \ori{Einbildungskraft}
in der \ori{Assoziation} (und Reproduktion), die \ori{Apperzeption} in dem
\ori{empirischen Bewußtsein} der Identität dieser reproduktiven Vorstellungen
mit den Erscheinungen, dadurch sie gegeben waren, mithin in der
\ori{Rekognition}.\footnote{\cite[][A 115]{Kant:KritikderreinenVernunft2003},
\cite[][IV: 86.16--27]{Kant:GesammelteWerke1900ff.}.}
\end{quote}
Dass Sinnlichkeit und Verstand (Apperzeption\footnote{\enquote{\ori{Die
Einheit der Apperzeption in Beziehung auf die Synthesis der Einbildungskraft}
ist der \ori{Verstand}, und eben dieselbe Einheit, beziehungsweise auf die
\ori{transzendentale Synthesis} der Einbildungskraft, der \ori{reine Verstand}}
\mkbibparens{\cite[][A 119]{Kant:KritikderreinenVernunft2003},
\cite[][IV: 88.22--25]{Kant:GesammelteWerke1900ff.}}. \enquote{Und so ist die
synthetische Einheit der Apperzeption der höchste Punkt, an dem man allen Verstandesgebrauch, selbst die ganze Logik, und, nach ihr, die Transzendental-Philosophie heften muß, ja dieses Vermögen ist der Verstand selbst} \mkbibparens{\cite[][B 134]{Kant:KritikderreinenVernunft2003}, \cite[][III: 109.35--39]{Kant:GesammelteWerke1900ff.}}.}) als Erkenntnisquellen angesprochen werden, verwundert nicht weiter.
Aber auch die Einbildungskraft (als produktive wie reproduktive) ist nach dieser Auskunft eine Erkenntnisquelle. Diese
Darstellung von drei Erkenntnisquellen findet sich in der ersten Auflage im
Rahmen der Deduktion der reinen Verstandesbegriffe und folgt der dort zu
findenden Vorgehensweise. So folgen der \enquote{Synthesis der Apprehension in
der Anschauung}\footnote{\cite[][A 98]{Kant:KritikderreinenVernunft2003},
\cite[][IV: 77.2]{Kant:GesammelteWerke1900ff.}.} die \enquote{Synthesis der
Reproduktion in der Einbildung}\footnote{\cite[][A 100]{Kant:KritikderreinenVernunft2003},
\cite[][IV: 77.32]{Kant:GesammelteWerke1900ff.}.} und schließlich die
\enquote{Synthesis der Rekognition im Begriffe}\footnote{\cite[][A
103]{Kant:KritikderreinenVernunft2003},
\cite[][IV: 79.15]{Kant:GesammelteWerke1900ff.}.}.
In die zweite Auflage wurde diese Darstellung zwar nicht übernommen, es findet
sich dort nur der Verweis darauf, dass (innerer) Sinn, Einbildungskraft und
Apperzeption  Quellen von Vorstellungen \emph{a priori}
enthalten,\footnote{\cite[Vgl.][B 194]{Kant:KritikderreinenVernunft2003},
\cite[][III: 144.5--11]{Kant:GesammelteWerke1900ff.}.} und die Redeweise von
Sinnlichkeit und Verstand als den \enquote{zwei Grundquellen des
Gemüts}\footnote{\cite[][B 74]{Kant:KritikderreinenVernunft2003},
\cite[][III: 74.9]{Kant:GesammelteWerke1900ff.}.}, die in der Einleitung auch
als \enquote{Stämme der menschlichen
Erkenntnis}\footnote{\cite[][B 29]{Kant:KritikderreinenVernunft2003},
\cite[][III: 46.7]{Kant:GesammelteWerke1900ff.}.} bezeichnet
werden.\footnote{Siehe dazu auch oben, Kap. \ref{subsection:DiskursiverVerstandundsinnlicheAnschauung} dieser Arbeit.} Allerdings scheint
mir die Darstellung, wonach Sinnlichkeit, Einbildungskraft und Apperzeption Erkenntnisquellen sind,
für die hier interessierenden Fragen nicht hilfreich zu sein. Denn
\name[Immanuel]{Kant} bezieht sich hier in der \titel{Analytik der Begriffe}
nicht auf Erkenntnisse als \emph{Urteile}, sondern auf
Erkenntnisse im Sinne von bewussten objektiven Vorstellungen -- also auf
\emph{Begriffe} und \emph{Anschauungen} --, um zu zeigen, das reine
Verstandesbegriffe auch bei Anschauungen involviert sind.\footnote{Siehe zur
Ambiguität von \enquote{Erkenntnis} oben, Anm.
\ref{Anmerkung:ErkenntnisInZweierleiSinn} auf S.
\pageref{Anmerkung:ErkenntnisInZweierleiSinn}.} Sinn, Einbildungskraft und
Apperzeption können nur \emph{zusammen} Vorstellungen hervorbringen, denen
objektive Realität zukommt.


Auch an anderen Stellen spricht er von Erkenntnisquellen -- oder auch
von Quellen der Metaphysik\footnote{\cite[Vgl.][\S~2]{Kant:ProlegomenazueinerjedenkuenftigenMetaphysikdiealsWissenschaftwirdauftretenkoennen1977},
\cite[][IV: 265.6--266.8]{Kant:GesammelteWerke1900ff.}.} oder metaphysischer
Urteile\footnote{\cite[Vgl.][\S~3]{Kant:ProlegomenazueinerjedenkuenftigenMetaphysikdiealsWissenschaftwirdauftretenkoennen1977},
\cite[][IV: 270.9]{Kant:GesammelteWerke1900ff.}.}, der reinen Mathematik und
Naturwissenschaft\footnote{\cite[Vgl.][\S~5]{Kant:ProlegomenazueinerjedenkuenftigenMetaphysikdiealsWissenschaftwirdauftretenkoennen1977},
\cite[][IV: 280.10]{Kant:GesammelteWerke1900ff.}.}, der
Vernunft\footnote{\cite[Vgl.][A
3]{Kant:ProlegomenazueinerjedenkuenftigenMetaphysikdiealsWissenschaftwirdauftretenkoennen1977},
\cite[][IV: 255.8]{Kant:GesammelteWerke1900ff.}.}, von
Begriffen\footnote{\cite[Vgl.][A
62]{Kant:ProlegomenazueinerjedenkuenftigenMetaphysikdiealsWissenschaftwirdauftretenkoennen1977},
\cite[][IV: 288.22]{Kant:GesammelteWerke1900ff.}.} (des Raumes und der
Zeit\footnote{\cite[Vgl.][B 119\,f.,]{Kant:KritikderreinenVernunft2003}
\cite[][III: 101.13--15]{Kant:GesammelteWerke1900ff.}.}), der Urteile darüber,
was gerecht ist\footnote{\cite[Vgl.][BA 32]{Kant:DieMetaphysikderSitten1977Rechtslehre},
\cite[][VI: 229.18--230.6]{Kant:GesammelteWerke1900ff.}.} --, aber nirgends gibt
er so explizit die verschiedenen Quellen an. Einer ausführlicheren Diskussion der
Quellen einer Erkenntnis ist der erste Paragraph der \titel{Prolegomena}
gewidmet. \name[Immanuel]{Kant} unterscheidet dort zunächst Objekt, Quelle und
Art einer Erkenntnis und sagt, anhand dieser drei ließen sich Wissenschaften
einteilen. Der Art nach unterteilen sich Erkenntnisse in analytische und
synthetische,\footnote{\cite[Vgl.][\S~2]{Kant:ProlegomenazueinerjedenkuenftigenMetaphysikdiealsWissenschaftwirdauftretenkoennen1977},
\cite[][IV: ]{Kant:GesammelteWerke1900ff.}.} der Quelle nach -- dies sagt er in
den \titel{Prolegomena} jedoch nicht explizit -- offenbar in Erkenntnisse
\emph{a priori} und Erkenntnisse \emph{a posteriori}. In der \titel{Kritik der reinen
Vernunft} schreibt er: \enquote{Man nennt solche \ori{Erkenntnisse a priori},
und unterscheidet sie von den \ori{empirischen}, die ihre Quellen a
posteriori, nämlich in der Erfahrung haben.}\footnote{\cite[][B
2]{Kant:KritikderreinenVernunft2003}, \cite[][III:
28.4--6]{Kant:GesammelteWerke1900ff.}. Siehe auch
\cite[][B 35]{Kant:KritikderreinenVernunft2003},
\cite[][III: 50.35, 51.22]{Kant:GesammelteWerke1900ff.}.} Als empirische
Quellen spricht der die äußere und innere Erfahrung an, und offenbar gibt es auch Quellen in der reinen Vernunft und dem reinen Verstande.\footnote{\cite[Vgl.][\S~1]{Kant:KritikderreinenVernunft2003},
\cite[][IV: 265.6--266.8]{Kant:GesammelteWerke1900ff.}.} \name[Immanuel]{Kant}
spricht zwar weder die Erfahrung noch die Vernunft als Erkenntnisquellen an,
verortet diese Quellen aber \emph{in} der Erfahrung und \emph{in} der Vernunft,
so dass es von geringerer Bedeutung zu sein scheint, die Quellen selbst zu
individualisieren, als sie in diesem Sinne zuzuordnen. \name[Immanuel]{Kant}
nutzt das Wort \enquote{Quelle} häufig, im Zusammenhang von
\enquote{Erkenntnisquelle} oder \enquote{Quelle von Erkenntnissen} ist jedoch
zwischen 1781 und 1787 eine Verschiebung festzustellen. Während 1781 Sinn,
Einbildungskraft und Apperzeption als Erkenntnisquellen bezeichnet werden, weil
sie in Zusammenwirkung Vorstellungen mit objektiver Realität ermöglichen,
gebraucht er später diesen Begriff eher im Sinne des Ursprungs von
Rechtfertigungen von Urteilen, die \emph{a priori} oder \emph{a posteriori} sein
können. Dennoch bleibt die Redeweise von Quellen von Begriffen bestehen, wird
aber ebenso auf die Unterscheidung \emph{a priori}/\emph{a posteriori}
bezogen.\footnote{Siehe z.\,B. \cite[][B 57]{Kant:KritikderreinenVernunft2003},
\cite[][III: 63.37--64.2]{Kant:GesammelteWerke1900ff.}.}

Dass \name[Immanuel]{Kant} also einerseits die (äußere und innere) Erfahrung
und andererseits die Vernunft, nicht aber die Auskunft anderer als
Erkenntnisquellen anspricht, deutet \emph{prima facie} auf eine
individualistische Position hin. Doch es ist offensichtlich, dass wir auch aus
den Mitteilungen anderer Wissen erwerben. Wie lässt sich diese Möglichkeit
testimonialen Wissens rekonstruieren, wenn wir Erfahrung und Vernunft als
Erkenntnisquellen zulassen?

Hilfreich zur besseren systematischen Übersicht scheinen mir die begrifflichen
Grundlagen zu sein, die \authorfullcite{Fricker:AgainstGullibility1994}
entwickelt. Sie wirft die Frage auf, ob es sich bei der Mitteilung von
Erkenntnissen um einen \enquote{epistemic link} handelt, der den Empfänger mit allen nötigen Prämissen versorgt, um in
einer Überzeugung gerechtfertigt zu sein (\enquote{primary epistemic link}), oder
ob dem Empfänger dazu zusätzliche Prämissen bekannt sein müssen
(\enquote{secondary epistemic link}). Sehen, dass
$p$ der Fall ist, ist ein Beispiel für einen \emph{primary epistemic link} zu
der Überzeugung $p$. Wenn ich hingegen sehe, dass die Straße nass ist, dann ist
dies nur ein \emph{secondary epistemic link} zu der Überzeugung, dass es
geregnet hat, denn zur Begründung benötige ich die weitere Prämisse, dass Regen
die übliche Art und Weise ist, wie Straßen nass werden.

Die Frage nach dem Status testimonialen Wissens lässt sich nun folgendermaßen
formulieren: Wenn mir jemand mitteilt, dass $p$, habe ich dann bereits alle
nötigen Prämissen, um gerechtfertigt glauben zu können, dass $p$ (\emph{primary
epistemic link}), oder benötige ich weitere Prämissen wie die, dass die
angeführte Autorität vertrauenswürdig ist und immer die Wahrheit sagt (\emph{secondary epistemic
link})? \name[Elizabeth]{Fricker} behauptet, dass Mitteilungen nur einen \emph{secondary
epistemic link} zu dem jeweiligen testimonialen Wissen
darstellen.\footnote{\cite[Vgl.][]{Fricker:TheEpistemologyofTestimony1987}.} Wir
benötigen nach \name[Elizabeth]{Fricker}s Vorstellung also weitere Prämissen, um durch
Mitteilungen Wissen zu erlangen.
Erfährt man beispielsweise von einem Passanten, in welcher Richtung sich die
Kathedrale befindet, so reicht dies nach dieser Konzeption für sich genommen
nicht als Begründung um zu wissen, wo sich die Kathedrale befindet. Nur wenn man weiter (begründet?) überzeugt ist, dass
unser Informant selbst \emph{weiß}, wo die Kathedrale ist (weil er vielleicht
Einheimischer ist), und uns nicht belügt (weil es vielleicht eine
anthropologische Tatsache ist, dass Menschen meist ehrlich antworten), können
wir darauf \emph{schließen}, wo die Kathedrale ist.\footnote{Das Beispiel
stammt von \textcite[vgl.][\pno~197\,f.]{McDowell:KnowledgebyHearsay1994}.}



Es ergeben sich drei mögliche Positionen, je nachdem, wie ein Philosoph darüber
denkt, ob es sich bei testimonialen Erkenntnissen um einen \emph{primary
epistemic link} oder einen \emph{secondary epistemic link} handelt und ob eine
Rechtfertigung von testimonialem Wissen als eines \emph{secondary epistemic
link} funktionieren kann\footnote{Die intensive Diskussion der Frage
testimonialen Wissens hat freilich in den letzten Jahren weitere Misch- und
Zwischenpositionen hervorgebracht, z.\,B. der Versuch von
\authorfullcite{Lackey:ItTakesTwotoTango2006} eines dritten Weges jenseits
von Credulismus und Reduktionismus
\parencite[vgl.][]{Lackey:ItTakesTwotoTango2006}. Dennoch scheinen mir die hier
angeführten \singlequote{klassischen} Positionen das Feld am besten zu
strukturieren. Versuche wie der von \authorcite{Lackey:ItTakesTwotoTango2006}
sind letztlich nur vor diesem Hintergrund verständlich.}:
\begin{nummerierung}
\item Der \singlequote{Credulismus} oder \singlequote{defaultism} behauptet,
dass eine gültige Regel besagt, Mitteilungen seien \emph{prima facie}
hinreichende Begründungen für
Wissen.\footnote{\phantomsection\label{Fussnote:BegriffdesCredulismus}\authorfullcite{Wilholt:SozialeErkenntnistheorie2007}
verwendet die Ausdrücke \enquote{credulism} und \enquote{defaultism} für
Positionen, die behaupten, dass wir bei Mitteilungen zumindest \emph{prima
facie} berechtigt sind, ihnen zu vertrauen
\parencite[vgl.][48]{Wilholt:SozialeErkenntnistheorie2007}.} Credulisten sehen
den Erwerb testimonialen Wissens also als \emph{primary epistemic link} an.
\item Vertreter eines \singlequote{testimonialen Reduktionismus} hingegen
behaupten, dass es sich bei testimonialem Wissen um einen \emph{secondary epistemic link}
handelt: Wir erwerben nur dann Wissen durch die Mitteilung anderer, wenn wir
für die Überzeugung, dass uns wahre Erkenntnisse mitgeteilt werden, Gründe
haben, die von dieser Mitteilung selbst unabhängig
sind.\footnote{\cite[Vgl.][533]{Grundmann:AnalytischeEinfuehrungindieErkenntnistheorie2008}:
\enquote{Der Reduktionismus besagt, dass ein Hörer (oder Leser) nur dann eine
gerechtfertigte Überzeugung oder Wissen von dem, was der Sprecher (oder Autor)
behauptet, erwirbt, wenn er gerechtfertigt glaubt, dass der Sprecher (oder
Autor) zuverlässig und aufrichtig in dem ist, was er sagt (oder schreibt).} Nach
Torsten \textcite[][48]{Wilholt:SozialeErkenntnistheorie2007} vertritt der
testimoniale Reduktionismus die Auffassung, \enquote{in einer
mitteilungsbasierten Überzeugung könne nur gerechtfertigt sein, wer auch von der
Verlässlichkeit der fraglichen Mitteilung überzeugt sei und für diese
Überzeugung wiederum Rechtfertigungsgründe besitze, die sich letztlich nicht auf
Mitteilung anderer stützen.} Nach Oliver
\textcite[][358]{Scholz:DasZeugnisanderer2001} zeichnet sich der testimoniale
Reduktionismus durch folgende Annahme aus:
\enquote{Um das Zeugnis anderer als Quelle von gerechtfertigter Meinung und
Wissen zu vindizieren, ist eine Zurückführung auf andere epistemische Quellen erforderlich.}}
\item Der \singlequote{testimoniale
Skeptizismus} fordert uns schließlich auf, testimoniales Wissen gänzlich zu
verwerfen; danach irren wir uns, wenn wir glauben, wir seien berechtigt, unsere
testimonialen Überzeugungen als Wissen anzusehen. Er behauptet somit, dass
testimoniales Wissen nicht im Sinne eines \emph{primary epistemic link}
verstanden werden kann und dass eine Rechtfertigung als \emph{secondary
epistemic link} nicht funktioniert.
\end{nummerierung}

\begin{comment}
\subsection{Externalismus und Internalismus in der
\emph{Social Epistemology}}\label{subsubsection:DieSystematischePerspektive}
Die Möglichkeiten, auf die skeptische Herausforderung bezüglich der Möglichkeit
testimonialen Wissens -- den testimonialen Skeptizismus -- zu reagieren,
unterscheiden sich nicht grundlegend von unseren Möglichkeiten, auf einen
allgemeinen Wissensskeptizismus zu reagieren. Es scheint zunächst so zu sein,
dass wir nicht alle Möglichkeiten explizit ausschließen können, wonach unsere
Überzeugungen sich als falsch oder ihre epistemische Grundlage sich als
ungenügend herausstellen könnte. Wir können bei genuin testimonialen
Überzeugungen die Möglichkeit, belogen zu werden, vielleicht nicht restlos
eliminieren. Gerade dies macht den testimonialen Skeptizismus im Gegensatz zu
einem allgemeinen Skeptizismus attraktiv. Die Argumente, mit denen
sich ein universeller Skeptizismus mit Bezug auf Wissen motivieren lässt, das wir je
individuell generieren, muss auf ausgefallene Szenarien wie \authorcite{Descartes:OeuvresdeDescartes1983}'
bösen Dämon oder \authorcite{Putnam:TheMeaningofglqMeaninggrq1975}s Gehirn im
Tank rekurrieren, die uns nicht als \emph{reale} Möglichkeiten, sondern nur als
philosophische Gedankenspiele begegnen. Aber die Möglichkeiten, die dazu führen,
auch bei einfachen und alltäglichen Beispielen testimonialen Wissens in die Irre
geführt zu werden, sind reale Möglichkeiten.
Und so scheint es, dass sich die Möglichkeit testimonialen Wissens mit einem
einigermaßen starken Wissensbegriff nicht
verträgt.\footnote{\authorfullcite{Scholz:AutonomieangesichtsepistemischerAbhaengigkeiten2001}
unterstellt \name[Immanuel]{Kant} entsprechend die Ansicht, dass \enquote{die
illusorischen Ideale epistemischer Autonomie, die einen großen Teil der
neuzeitlichen Erkenntnistheorie geprägt haben, aufgegeben werden} müssten
\parencite[][839]{Scholz:AutonomieangesichtsepistemischerAbhaengigkeiten2001},
gerade weil er ihm attestiert, Mitteilungen als legitime Wissensquelle korrekt
einzuschätzen. Es gehe also darum, den Wissensbegriff, der traditionell viel zu
stark zu sein scheint, weil er eine \emph{Garantie} der Wahrheit unserer
Überzeugung verlangt, der epistemischen Realität anzupassen.
\authorfullcite{Kern:QuellendesWissens2006} spricht hier von
\enquote{Ermäßigung}, \cite[vgl.][112]{Kern:QuellendesWissens2006}:
\enquote{Positionen der Ermäßigung wollen sagen: Das Subjekt muß nicht alle
Umstände ausschließen, sondern nur die relevanten.} Ob die bloße Abschwächung
des Wissensbegriffs, die die Möglichkeit des Irrtums auch bei Wissen konzidiert
und die man gemeinhin \enquote{Fallibilismus} nennt, weiterhilft, mag zunächst dahingestellt sein.}
Aber können wir diese Möglichkeit in vielen oder zumindest manchen Fällen als
unbegründet verwerfen? Und falls dies so sein sollte: Lässt sich konkretisieren,
in \emph{welchen} Fällen wir Irrtumsszenarien als irrelevant ignorieren
dürfen?

Ich werde in diesem Kapitel erstens zu zeigen versuchen, dass die Forderung der
Aufklärung nach Mündigkeit internalistische epistemische Regeln verlangt, also
solche epistemischen Regeln, bei denen für das erkennende Subjekt selbst
ersichtlich ist, ob ihre Anwendungsbedingungen erfüllt sind. Externalistische
epistemische Regeln hingegen enthalten eine Bedingung, bei der für das Subjekt nicht
ersichtlich ist, ob sie tatsächlich erfüllt ist. Zweitens soll darauf
hingewiesen werden, dass aus diesem epistemischen Internalismus kein
epistemischer Individualismus resultiert.
Die Unvorsichtigkeit hinter der Behauptung, Aufklärung führe unmittelbar zu
einer individualistischen Erkenntnistheorie, liegt in dem Übergang von einem
Internalismus zu einem Individualismus. Doch bevor ich dies erläutere, werde ich
mit Hilfe von \authorfullcite{Lewis:ElusiveKnowledge1996}' Liste epistemischer
Regeln versuchen, den Unterschied zwischen internalistischen und
externalistischen epistemischen Regeln zu verdeutlichen.

Wenn wir etwas wissen, dann haben wir nicht nur eine wahre Überzeugung und auch
nicht nur eine Überzeugung, zu der wir eine Begründung haben, sondern eine
Überzeugung, die so gut begründet ist, dass nicht sinnvoll angenommen werden
kann, dass wir uns irren. Andernfalls haben wir bloß eine Vermutung. Auch
Vermutungen können gut begründet sein, aber ihre Begründung schließt die
Möglichkeit des Irrtums nicht aus; darin unterscheiden sie sich von Wissen. Wir
vermuten etwa, dass es bald regnet, weil wir sehen, dass Wolken aufziehen. Da es
aber auch möglich ist, dass die Wolken in eine andere Richtung ziehen oder ohne
Regen über uns hinweg ziehen, vermuten wir nur und wissen nicht.

Epistemische Regeln im Sinne von \authorcite{Lewis:ElusiveKnowledge1996} geben
an, welche möglichen Szenarien zu beachten sind und welche ignoriert werden können, wenn wir
Wissensansprüche bewerten. Er hat -- nicht exklusiv zum Thema testimonialen
Wissens, sondern allgemein zur Untersuchung des Wissensbegriffs -- eine Liste
von insgesamt sechs Regeln erstellt, die angeben, wann wir Möglichkeiten als
irrelevant betrachten dürfen.\footnote{\cite[Vgl. zum
Folgenden][554--560]{Lewis:ElusiveKnowledge1996}.} Ich liste sie hier zur
Übersicht auf, ohne damit behaupten zu wollen, dass alle sechs Regeln
überzeugen. Es handelt sich um Regeln, die systematisch interessieren und
aktuell diskutiert werden und die hier vor allem zur Illustration der
Überlegungen zu Internalismus und Externalismus dienen.
Wie ich zugleich deutlich machen möchte, lassen sie sich alle leicht auf den
Fall testimonialen Wissens anwenden. \authorcite{Lewis:ElusiveKnowledge1996}'
Regeln lauten (in geänderter Reihenfolge):
\begin{nummerierung}
\item \emph{Rule of Actuality}. Wir müssen stets diejenige Möglichkeit
berücksichtigen, die \emph{verwirklicht} ist. Wenn Jasmin glaubt, ein Haus zu
sehen, aber \emph{de facto} vor einer Hausattrappe steht, kann sie nicht wissen,
dass vor ihr ein Haus steht. Und ebenso wenig kann sie wissen, dass vor ihr ein
Haus steht, wenn dort tatsächlich eines steht, sie aber zeitgleich halluziniert;
denn die Halluzinationen sind ein Grund, die Möglichkeit des Irrtums in Betracht
zu ziehen, auch wenn sie selbst um diese Halluzinationen nicht weiß. Wenn Peter
gerade lügt oder sich selbst täuscht oder wenn Peter ein notorischer Lügner
oder im Erkennen bestimmter Dinge sehr unzuverlässig
ist\footnote{\phantomsection\label{Anmerkung:PetersOrientierungsschwierigkeiten}Man
denke an den Fall, dass Jasmin als Touristin in einer fremden Stadt Peter, den sie nicht kennt, aber für einen Einheimischen hält, nach dem Weg zur Kathedrale
fragt. Peter habe ein so schlechtes Orientierungsvermögen, dass niemand, der
ihn kennt, auf seine Ratschläge hören würde. In diesem Fall weise er Jasmin aber
in die richtige Richtung. \emph{Weiß} Jasmin, die Peters Orientierungsprobleme
nicht kennt, nun, wie sie zur Kathedrale gelangt? Ich denke, man wird diese
Frage verneinen müssen.}, dann handelt es sich um eine Irrtumsmöglichkeit, die
nicht ignoriert werden darf. Und Jasmin kann unabhängig davon, ob sie selbst
weiß, dass Peter lügt oder sich täuscht, kein testimoniales Wissen von Peter
erhalten.\footnote{Dies gilt natürlich auch in dem Fall, dass Peter tatsächlich
die Wahrheit sagt, aber im Erkenntnisprozess ein solcher Fehler enthalten ist,
wie wir ihn von den
\authorcite{Gettier:IsJuftifiedTrueBeliefKnowledge?1963}-Beispielen kennen.
Wenn Peter bspw. Jasmin in die Irre führen möchte, indem er ihr die falsche
Richtung weist, aufgrund seiner mangelnden Ortskenntnis Jasmin aber doch in die
richtige Richtung schickt, dann kann Jasmin nicht auf der Grundlage von Peters
Mitteilung \emph{wissen}, welche Richtung die richtige ist. Möglicherweise
könnte aus der in Anm.~\ref{Anmerkung:PetersOrientierungsschwierigkeiten}
geschilderte Fall hier als Beispiel dienen.}
\item \emph{Rule of Belief}. Jede Möglichkeit muss berücksichtigt werden, von
der das Subjekt selbst überzeugt ist oder von der es vernünftigerweise überzeugt
sein sollte, weil sie sich aus einer anderen Überzeugung ergibt.
Wenn Jasmin die Überzeugung hat, zu halluzinieren, (oder überzeugt ist, ein
starkes Halluzinogen eingenommen zu haben), dann muss sie die Möglichkeit in
Betracht ziehen, sich in ihrer Wahrnehmung zu täuschen. Und wenn Jasmin
\emph{glaubt}, dass Peter lügt, dann kann sie auf der Grundlage seiner
Auskunft kein testimoniales Wissen erlangen, auch dann nicht, wenn Peter nach
bestem Wissen und Gewissen die Wahrheit sagt. Das gleiche gilt, wenn Jasmin offensichtliche Gründe hat
anzunehmen, dass Peter ein unzuverlässiger Informant ist, ohne deshalb
misstrauisch zu werden. Jasmin könnte beispielsweise wissen, dass Peter eine
Amsel nicht von einer Krähe unterscheiden kann, ihm aber dennoch glauben, dass
eine Amsel im Garten sitzt, weil sie gerade nicht daran denkt, dass Peter
hierbei kein zuverlässiger Informant ist. Sie sollte dann vernünftigerweise von
der Möglichkeit überzeugt sein, dass Peter sich täuscht, weil sich diese
Überzeugung aus einer anderen Überzeugung ergibt, die sie hat.
\item \emph{Rule of Reliablity}. Wenn wir Informationen durch Prozesse erlangen,
die normalerweise zuverlässig ablaufen, dann dürfen wir die
Möglichkeit ihres Versagens außer Acht lassen. Zumindest gilt dies so lange, wie
keine Gegengründe (\enquote{\emph{defeaters}}) gegen ihre Zuverlässigkeit vorliegen.
Unsere visuelle Wahrnehmung gehört sicherlich zu diesen zuverlässigen Prozessen;
ihr dürfen wir vertrauen, solange nichts an der aktuellen Situation gegen ihre
Zuverlässigkeit spricht. Das Zeugnis anderer (\enquote{testimony}) wird von
\authorcite{Lewis:ElusiveKnowledge1996} selbst als ein solcher zuverlässiger
Prozess angeführt, der uns in der Regel verlässlich mit wahren Informationen
versorgt.\footcite[Vgl.][558]{Lewis:ElusiveKnowledge1996} Jasmin kann niemals
mit Gewissheit ausschließen, dass Peter lügt. (Andernfalls erwürbe sie kein
genuin testimoniales Wissen.) Aber sie darf diese Möglichkeit vielleicht
ignorieren, solange kein Grund zum Misstrauen vorliegt. Und zwar darf sie dies,
wenn Peters Mitteilungen (oder Mitteilungen im
allgemeinen\footnote{Offensichtlich besteht hier die Schwierigkeit zu
bestimmen, welche konkreten Vorgänge zu der relevanten Art von Prozessen
zusammenzufassen sind, deren Zuverlässigkeit hier zugrunde zu legen ist.
Gehört Peters Mitteilung, dass $p$, zu den legitimen Informationsvorgängen,
weil Mitteilungen generell zuverlässige Prozesse darstellen? Sind
speziell Peters Mitteilungen die bestimmende Referenzklasse, deren
Zuverlässigkeit zu bewerten ist? Oder sind vielleicht Mitteilung bezüglich
bestimmter Themen zuverlässig, bezüglich bestimmter anderer Themen jedoch
nicht? All dies scheint ein entsprechender Ansatz, der die \emph{Rule of
Reliability} in der Erläuterung testimonialen Wissens zugrunde legt, weiter
klären zu müssen.}) normalerweise zuverlässige Informationsquellen darstellen
und nur in Ausnahmefällen zu Fehlurteilen führen.
\end{nummerierung}

Bei dieser Auswahl handelt es sich vermutlich um die bekanntesten epistemischen
Regeln. Die anderen Regeln sind weniger einschlägig und vielleicht handelt es
sich bei näherer Analyse lediglich um Spezial- oder Anwendungsfälle der ersten
drei Regeln, wie \authorcite{Lewis:ElusiveKnowledge1996}
selbst einräumt.\footnote{\cite[Vgl.][559]{Lewis:ElusiveKnowledge1996}.
\authorfullcite{Kern:QuellendesWissens2006} reduziert ihre Diskussion der Regeln
von \name{Lewis} sogar auf eine einzige, die \emph{Rule of Reliablity}, da diese
die zentrale epistemische Regel des Externalismus artikuliere.
\cite[Siehe][120]{Kern:QuellendesWissens2006}.} Ich stelle sie hier der Vollständigkeit wegen mit vor:
\begin{enumerate}[resume]
\item \emph{Rule of Resemblance}. Wenn eine Möglichkeit sehr stark an eine
andere Möglichkeit erinnert, die wir nicht ignorieren dürfen, dann darf sie auch
nicht ignoriert werden. Wenn wir nach der Einnahme eines bestimmten Medikaments
unserer visuellen Wahrnehmung misstrauen und keinen relevanten Unterschied
zwischen unserer visuellen und unserer auditiven Wahrnehmung kennen, dann
müssen wir es für möglich erachten, dass wir uns in dem täuschen, was wir zu
hören glauben. Und wenn Jasmin die Möglichkeit, dass Peter lügt, für real hält,
weil sie Peter für einen notorischen Lügner hält, und Klaus und Peter sich in
ihrem Verhalten kaum voneinander unterscheiden, dann hilft es ihr möglicherweise
nichts, wenn Klaus ihr eine Auskunft gibt und nicht Peter.
\item \emph{Rule of Conservatism}. Wir dürfen all diejenigen Möglichkeiten
ignorieren, von denen bekannt ist, dass sie für gewöhnlich bewusst ignoriert
werden dürfen. Wenn jeder die Möglichkeit, dass Peter lügt, als lächerlich
abtut, dann dürfen wir diese Möglichkeit ebenfalls ignorieren. Und wenn Peter
weit und breit Anerkennung als aufmerksamer Beobachter erfährt,
brauchen wir seinen diesbezüglichen Kompetenzen gegenüber keinerlei Zweifel
haben.
\item \emph{Rule of Attention}. Wenn wir eine Irrtumsmöglichkeit selbst gerade
im Blick haben, dann gehört sie zu den relevanten Möglichkeiten. Aber sie gehört
nur zu den relevanten Möglichkeiten aus der Perspektive desjenigen, der auf sie
achtet. Wenn \emph{wir} also darüber nachdenken, dass Peter lügen könnte, dann
müssen \emph{wir} diese Möglichkeit berücksichtigen und dann können \emph{wir}
über Peters Mitteilungen kein Wissen erwerben (und möglicherweise können
\emph{wir} Jasmin \emph{aus unserer Sicht} kein Wissen zuschreiben).
Jasmin hingegen, die über diese Irrtumsmöglichkeiten gerade nicht nachdenkt, mag sie vernachlässigen
dürfen, insofern keine andere Regel vorschreibt, sie zu beachten. Diese Regel
führt letztlich dazu, dass wir, solange wir Erkenntnistheorie betreiben und
skeptische Szenarien entwickeln, jegliche Wissensansprüche zurückweisen müssen,
bis wir die skeptischen Szenarien widerlegt haben.
\end{enumerate}
Es könnte also sein, dass die eine oder andere dieser Regeln nur einen
Spezialfall einer anderen Regel artikuliert. Aber dieser Aspekt von
\authorcite{Lewis:ElusiveKnowledge1996}' Systematik braucht uns hier nicht
weiter zu beschäftigen. Wichtiger ist folgende Einteilung, die ich hier am
Beispiel der drei ersten Regeln expliziere:


Die \emph{Rule of Actuality} und die \emph{Rule of Reliability} artikulieren
\emph{externalistische}
Annahmen,\footnote{\authorcite{Lewis:ElusiveKnowledge1996} selbst nennt die
\emph{Rule of Actuality} externalistisch
\parencite[vgl.][554]{Lewis:ElusiveKnowledge1996}, aber es ist klar, dass gerade
die \emph{Rule of Reliabilism} eine Grundaussage des epistemischen Externalismus
artikuliert \parencite[vgl.][120]{Kern:QuellendesWissens2006}.} denn sie nehmen
auf etwas Bezug, das dem Subjekt nicht notwendigerweise zugänglich sein muss.
Die \emph{Rule of Actuality} gebietet uns die Berücksichtigung von
Möglichkeiten, die \emph{de facto} realisiert sind, unabhängig davon, ob wir um
ihr Vorliegen wissen. Ebenso erlaubt uns die \emph{Rule of Reliability}, die
Möglichkeit des Versagen \emph{de facto} zuverlässiger Prozesse so lange zu
ignorieren, bis uns Gründe bekannt werden, die explizit dafür sprechen, dass
eine solche Möglichkeit verwirklicht ist. Ihre Anwendung hängt explizit nicht
davon ab, dass wir um die Zuverlässigkeit der Prozesse wissen. Die
\emph{Rule of Belief} charakterisiert hingegen internalistische Positionen, denn
sie macht unsere Zustimmung zu Aussagen abhängig von Erkenntnissen, die dem
Subjekt selbst verfügbar sind. Sie fordert die Berücksichtigung von
Möglichkeiten, für deren Verwirklichung uns Gründe bekannt sind, unabhängig
davon, ob die Möglichkeiten \emph{de facto} realisiert sind.

Eine Eigenschaft externalistischer Erlaubnisregeln wird noch wichtig werden: Sie
gelten stets nur \emph{prima facie}; sie verlieren ihre Kraft, wenn es Anzeichen dafür gibt, dass
die entsprechende Grund\-an\-nahme in einem Fall falsch ist, wenn beispielsweise
etwas dafür spricht, dass ein in der Regel zuverlässiger Prozess nicht standardmäßig
abläuft. Man spricht in diesem Fall von einem \enquote{\emph{defeater}}. Ein
\emph{defeater} ist also das Wissen um einen Umstand, bei dessen Vorliegen die
Anwendung einer Erlaubnisregel unzulässig ist.

Betrachten wir zunächst die externalistische Verbotsregel, die \emph{Rule of
Actuality}. Sie sagt, dass tatsächlich vorliegende Störungen im
Informationsfluss nicht ignoriert werden dürfen, und schließt damit die
klassischen \authorcite{Gettier:IsJuftifiedTrueBeliefKnowledge?1963}-Beispiele, die sich selbstverständlich auch
im Bereich testimonialer Erkenntnisse konstruieren lassen, aus. Man kann so
reagieren, dass man die Bedingungen für das Vorliegen von testimonialem Wissen
von äußeren Faktoren abhängig macht und mit Thomas \name[Thomas]{Grundmann} etwa
Folgendes proklamiert:
\begin{quote}
  Das Zeugnis anderer ist dann eine zuverlässige
  Quelle, wenn der Informant aufrichtig ist und seine Äußerungen seinerseits auf
  zuverlässige Quellen stützt. Sobald diese Bedingungen objektiv erfüllt sind,
  kann das Zeugnis anderer Wissen oder gerechtfertigte Überzeugungen liefern,
  auch wenn der Rezipient davon kein explizites Wissen
  besitzt.\footnote{\cite[][537]{Grundmann:AnalytischeEinfuehrungindieErkenntnistheorie2008}.}
\end{quote}
Wir sind dann gerechtfertigt in unserer Überzeugung und besitzen testimoniales
Wissen, wenn unser Informant eine zuverlässige Wissensquelle ist, die in
unserem Fall tatsächlich über Wissen verfügt und uns dieses
mitteilt.\footnote{\cite[Vgl.][]{Grundmann:DietraditionelleErkenntnistheorieundihreHerausforderer2001}
sowie
\cite[][529--541]{Grundmann:AnalytischeEinfuehrungindieErkenntnistheorie2008}.}
Nehmen wir beispielsweise an, Peter weiß tatsächlich, dass es regnet, und Jasmin
hat keine Gründe, an Peters Glaubwürdigkeit zu zweifeln, und kommt infolge dessen zu
der Überzeugung, dass es regnet. Wenn Peter tatsächlich weiß, dass es regnet und
tatsächlich zuverlässig die Wahrheit sagt, dann weiß nun auch Jasmin, dass es
regnet. Wenn Peter aber \emph{de facto} gar nicht weiß, dass es regnet, oder
wenn Peter \emph{de facto} keine zuverlässige Informationsquelle ist, dann
erwirbt Jasmin kein testimoniales Wissen -- auch dann nicht, wenn es tatsächlich
regnet. Somit entzieht sich unserer eigenen Kenntnis, ob es sich bei unseren
eigenen Überzeugungen tatsächlich um testimoniales Wissen handelt. Ob eine
Rechtfertigung vorliegt \emph{respective} ob es sich um eine hinreichende
Rechtfertigung für Wissen handelt, hängt also von uns unbekannten
Kontextfaktoren
ab.\footnote{\cite[Vgl.][532]{Grundmann:AnalytischeEinfuehrungindieErkenntnistheorie2008}:
\enquote{Solange man {\punkt} annimmt, dass eine gerechtfertigte Überzeugung
(oder Wissen) nur zustande kommt, wenn man selbst die Gründe kennt, die die
Wahrheit der Überzeugung zumindest wahrscheinlich machen, dann fehlen genau
diese Gründe im Fall der Information durch Dritte. {\punkt} Hinter der Ablehnung
des Zeugnisses anderer als Wissens- und Rechtfertigungsquelle steht also der
erkenntnistheoretische Zugangsinternalismus.}}

Ein Vorteil in der Einbeziehung externalistischer Regeln liegt -- so die
Schlussfolgerung des Externalismus -- darin, dass Positionen, die wie diejenige
\name[Thomas]{Reid}s oder der lokale Reduktionismus unserer epistemischen
Situation Rechnung tragen, damit mühelos funktionieren.
Externalistische Regeln erlauben uns, mit der Tatsache umzugehen, dass wir
testimoniales Wissen erlangen und erlangen müssen, \emph{bevor} wir in der Lage
sind, die Zuverlässigkeit der Quellen zu beurteilen, und beschreiben diesen
Prozess des Wissenserwerbs doch zugleich als vernünftig und legitim. Wir können
testimoniales Wissen über Informanten erlangen, bevor wir ihre Zuverlässigkeit
oder die Zuverlässigkeit von Informanten im allgemeinen einschätzen können, weil
es sich \emph{de facto} um zuverlässige Informationsquellen handelt. Ihr größter
Vorteil liegt freilich darin, dass wir mit ihrer Hilfe erläutern können, wie wir
uns in Situationen befinden können, in denen kein Wissen vorliegt und die sich
dennoch aus unserer Perspektive in keiner Hinsicht von Situationen
unterscheiden, in denen Wissen vorliegt. Situationen, in denen wir alles richtig
gemacht zu haben scheinen und dennoch irren,\footnote{Es kann hier offen
bleiben, ob es denkbar ist, dass wir keinen Fehler begehen und dennoch falsche
Überzeugungen haben. Man könnte entgegnen, dass wir in jedem Fall eine Regel
falsch angewandt haben, wenn ein Irrtum resultiert, auch wenn uns dies nicht
bewusst ist und vielleicht jeder andere an unserer Stelle denselben Fehler
begangen hätte.} oder die
\authorcite{Gettier:IsJuftifiedTrueBeliefKnowledge?1963}-Beispiele illustrieren dies.

Die externalistischen Überlegungen haben sicherlich eine gewisse Plausibilität,
insofern es um den Begriff des Wissens geht. Aber taugen sie auch zur Explikation des Mündigkeitsbegriffs?
Aus der Perspektive einer am Ausgang des Menschen aus seiner selbst verschuldeten
Unmündigkeit orientierten Aufklärung bleibt die externalistische Lösung
unbefriedigend: Wenn uns unbekannt ist, ob die Anwendungsbedingung einer
epistemischen Regel erfüllt ist, dann ist schwer zu sehen, wie damit ein
kritischer Umgang mit Mitteilungen anderer von einem unkritischen abgegrenzt
werden soll. Denn es scheint weitgehend unserem \singlequote{epistemischen
Glück}\footnote{Dieser Ausdruck hat sich in Anlehnung an einen Ausdruck von
Bernard \textcite[vgl.][]{Williams:MoralLuck1981} verbreitet; siehe z.\,B.
\cite{Engel:IsEpistemicLuckCompatiblewithKnowledge1992}.} anheim gestellt, ob wir
in unserem Vertrauen in die Mitteilung richtig handeln oder nicht. Insofern
Aufklärung den Ausgang des Menschen aus seiner selbst verschuldeten Unmündigkeit
fordert, kann sie nicht auf externalistische Regeln verweisen, die
letztlich unabhängig von einen epistemisch verantwortungsvollen oder
verantwortungslosen Umgang des Einzelnen mit seinen Informationsquellen
lediglich festlegen, wann \emph{Wissen} vorliegt.


Die Frage, ob Wissen vorliegt, ist eine andere Frage als die, ob jemand
sich als mündig erweist. Und Mündigkeit ist wiederum nach \name[Immanuel]{Kant}s
Auffassung etwas, das man sich im Laufe der Jahre erst erarbeiten muss. Auch der
Unmündige verfügt bereits über Wissen. Umgekehrt mag es sogar möglich sein,
Mündigkeit und Irrtum zu verbinden. Wenn Aufklärung in dem Ausgang des Menschen
aus seiner selbst verschuldeten Unmündigkeit besteht und eine veränderte
\enquote{Denkungsart} erfordert, es ihr also nicht primär um wahre Inhalte,
sondern um die Freiheit des Denkens geht, dann könnte auch der aufgeklärteste
Mensch noch falsche Überzeugungen haben können.\footnote{Siehe hierzu auch die
Überlegungen in Kapitel \ref{Zitat:Lessing:EineDuplik}, die einer Äußerung
\authorcite{Lessing:EineDuplik1897}s kulminieren, die ich auf Seite
\pageref{Zitat:Lessing:EineDuplik} zitiere.} Wir irren auch dort, wo wir unseren
eigenen Verstand gebrauchen und unsere Vernunft zum obersten \enquote{Probirstein} der Wahrheit erheben.

Wenngleich es also möglich sein mag, Wissen zu erwerben, ohne mündig zu sein, so
besteht das Ziel der Aufklärung doch zweifelsfrei im mündigen Wissenserwerb. Und
dies wiederum muss heißen, dass der Wissenserwerb sich an internalistischen
epistemischen Regeln orientiert. Die Mutmaßung, wonach Aufklärungsdenken dem
Individualismus verpflichtet zu sein scheint, weil es sich notwendig gegen eine externalistische Position wenden
muss, habe ich weiter oben bereits angesprochen.\footnote{Siehe
S.~\pageref{Absatz:AufklaerungundZugangsInternalismus}--\pageref{Absatz:AufklaerungundZugangsInternalismus-ENDE}.}
Aufklärung und die Forderung nach epistemischer Autonomie forderten einen
epistemischen Internalismus und dieser ziehe wiederum einen epistemischen
Individualismus nach sich. Es ist klar, dass dieser Überlegung widersprochen
werden muss, wenn Aufklärung eine ernsthafte Option sein soll. Dabei scheint mir
aber nicht die Verbindung von Aufklärung und Internalismus zurückzuweisen zu
sein, sondern die Behauptung, der Internalismus ziehe den Individualismus
nach sich. Aufklärungsphilosophie ist \emph{per se} dem Internalismus
verpflichtet, denn sie erwartet von dem mündigen Subjekt die \emph{eigene}
Kontrolle der Gründe für die je eigenen Überzeugungen.

\end{comment}


\section{Individualistische Ansätze in der Philosophie der
Neuzeit}\label{subsection:TestimonialesWissenUndMuendigkeit}
Die Frage nach der Möglichkeit testimonialen Wissens wird oft vor dem
Hintergrund eines angenommenen erkenntnistheoretischen Individualismus innerhalb der Philosophie
der Neuzeit diskutiert,\footnote{Ich behaupte damit nicht, dass der
Individualismus ein prägendes Element oder gar Grundlage der neuzeitlichen Philosophie sei.
Vielmehr werde ich darauf verweisen, dass weite Teile der
Aufklärungsphilosophie den (erkenntnistheoretischen) Individualismus explizit
zurückweisen. Einen ersten Teil dieser Zurückweisung lieferte bereits Kapitel
\ref{section:sensuscommunis}.} als dessen Gewährsmann
\authorfullcite{Descartes:OeuvresdeDescartes1983}
gilt.\footnote{\cite[Vgl.][\pno~407f.]{Welbourne:IsHumeReallyaReductivist2002}:
\enquote{Perhabs the epistemic individualism inculcated by
\authorcite{Descartes:OeuvresdeDescartes1983} has caused philosophers either to
neglect or to despise testimony.}
\authorfullcite{Grundmann:AnalytischeEinfuehrungindieErkenntnistheorie2008}
nennt hingegen \singlename{Platon} und \name[John]{Locke} als Beispiele dieses
Individualismus (\cite[vgl.][\pno~530f.]{Grundmann:AnalytischeEinfuehrungindieErkenntnistheorie2008}).}
Der Individualismus muss bestreiten, dass testimoniales Wissen im Sinne eines
\emph{primary epistemic link} möglich ist. Das heißt, er muss den Credulismus
zurückweisen und sich zwischen den Positionen des testimonialen Reduktionismus
und des testimonialen Skeptizismus entscheiden. Wir werden mit Thomas
\name[Thomas]{Reid} sehen, wie unattraktiv beide Positionen sind (Abschnitt
\ref{subsubsection:ThomasReid}). Hier soll aber zunächst erläutert werden, wie
für sie argumentiert werden kann. Bevor ich auf den testimonialen Reduktionismus
eingehe, für den David \name[David]{Hume} Pate steht (Abschnitt
\ref{subsubsection:DavidHume}), sollen daher nun skeptische Argumente von
\authorcite{Descartes:OeuvresdeDescartes1983} Erwähnung finden.




\subsection{Ren{\'e} Descartes' testimonialer
Skeptizismus}\label{subsection:DescartesKritikantestimonialemWissen}
Während ich vorhin mit \name[Immanuel]{Kant} die Vorteile epistemischer Arbeitsteilung
anhand der Analogie zu Handel und Gewerbe aufzeigte\footnote{Siehe oben Kapitel
\ref{Abschnitt:EpistemischeArbeitsteilung}.},
\Revision[Pelletier]{beginnt} \authorcite{Descartes:OeuvresdeDescartes1983}
\Revision[Pelletier]{seine Überlegungen mit der Betonung problematischer
Aspekte des Vertrauens auf das mitgeteilte Wissen Anderer}.
Er behauptet im \titel{Discours de la M{\'e}thode}, \enquote{daß Werke, die aus
mehreren Teilen zusammengesetzt sind und von der Hand verschiedener Meister
gefertigt wurden, häufig nicht dieselbe Vollkommenheit aufweisen wie die, an
denen nur ein einzelner gearbeitet hat.}\footnote{\cite[][VI:
11]{Descartes:OeuvresdeDescartes1983}.} Und er vergleicht die Gewinnung
wissenschaftlicher Erkenntnis dann auch nicht mit der effizienteren
arbeitsteiligen Produktion von Gütern in Fabriken und Manufakturen, sondern mit
der Arbeit an Werken der Kunst und Architektur. Diese seien vollkommener, wenn
einer allein den Plan macht. Arbeiten hingegen viele Menschen nach ihren jeweils
eigenen Vorstellungen an Teilen des Werkes, ohne durch einen gemeinsamen, den
jeweiligen Vorstellungen der einzelnen Ausführenden vorausgehenden
ursprünglichen Plan zusammengehalten zu werden, dann gebe es kein
wohl proportioniertes Ganzes, sondern ein unproportioniertes Zufallsprodukt, das
an die verwinkelten Straßen und Wege einer alten und über Jahrhunderte
gewachsenen Stadt erinnere, wo statt vernünftiger Absichten bloß der Zufall
regiere.\footnote{\cite[Vgl.][VI: 11--13]{Descartes:OeuvresdeDescartes1983}.}

Diese Passagen artikulieren zunächst kein klar erkennbares Argument; und in
unserer heutigen Zeit neigt das ästhetische Empfinden vielleicht mehr als das von
\authorcite{Descartes:OeuvresdeDescartes1983} und seinen Zeitgenossen dazu, die natürliche und
ungleichförmige Gewachsenheit, wie wir sie in historischen Stadtkernen
exemplifiziert finden, mindestens ebenso, wenn nicht sogar höher zu schätzen als
das Regelmäßige, Systematische und Geplante, das uns in Neubaugebieten,
Satellitenstädten und Trabantensiedlungen entgegen schlägt. Freilich ist dieser
ästhetische Reiz verwinkelter und unübersichtlicher Gassen nicht das, was wir in
wissenschaftlichen Erkenntnissen suchen. Hier geht es uns um
Transparenz und Übersichtlichkeit. Aber dass kooperative Forschung hier
unterlegen sei, diesen Irrtum widerlegt die Erfahrung der
Wissenschaftsgeschichte der vergangenen Jahrhunderte. Selbst den Anschein eines
überzeugenden Arguments kann diese Analogie mit Stadtbildern daher nicht mehr
erwecken; sie macht lediglich \authorcite{Descartes:OeuvresdeDescartes1983}' Grundhaltung anschaulich, die
sich gegen das \singlequote{Studium der Büchergelehrsamkeit}
(\singlequote{l'{\'e}tude des lettres}) älterer Bildungsvorstellungen richtet,
diesem aber nicht das moderne Modell kooperativen Forschens, sondern die
Vorstellung eines zurückgezogenen Denkers
entgegenstellt.\footnote{\cite[Vgl.][VI: 9]{Descartes:OeuvresdeDescartes1983}.}

In den \titel{Regulae ad directionem
ingenii}, die ihm als Vorlage für den \titel{Discours de la M{\'e}thode}
dienten, entwickelt \authorcite{Descartes:OeuvresdeDescartes1983} eine sachhaltigere und systematisch
augefeiltere Argumentation zur Zurückweisung testimonialer Erkenntnisse, die in
drei Stufen vorgeht, die sich auf den Mangel an Deutlichkeit mitgeteilter
Erkenntnisse (\ref{Abschnitt:DescartesunddieUeberinterpretation}), tatsächlich
und möglicherweise widerstreitende Meinungen
(\ref{Abschnitt:DescartesundderWiderstreitderMeinungen}), und die Differenz von
Wissenschaft und bloß historischer Kenntnis
(\ref{Abschnitt:DescartesundhistorischeKenntnisse}) beziehen\footnote{Von
diesen wird jedoch nur das \ref{Abschnitt:DescartesundderWiderstreitderMeinungen}. Argument
in die weitere Argumentation des \titel{Discours} übernommen.}:
\begin{nummerierung}
 \item\phantomsection\label{Abschnitt:DescartesunddieUeberinterpretation}
 Zunächst sei mitgeteilten Informationen zu misstrauen, weil die Informanten oft
 selbst leichtgläubig und unvorsichtig und teils sogar missgünstig seien. Dabei sieht
 \authorcite{Descartes:OeuvresdeDescartes1983} durch Eitelkeit und Geltungsdrang begründete
 Anreizstrukturen, die der Genese und Weitergabe sicheren Wissens entgegenstehen
 und Übertreibungen provozieren.\footnote{\cite[Vgl.][X:
 367.5--23]{Descartes:OeuvresdeDescartes1983}.} Ein Autor sei durch eigenen
 Geltungsdrang stets versucht, seine Ergebnisse überzuinterpretieren und ihnen
 durch schwer verständliches Vokabular den Anschein größeren intellektuellen
 Gewichts oder besonderer Wissenschaftlichkeit zu geben, statt bloß das wirklich
 Erreichte in nüchterner Sprache ohne Übertreibung vorzutragen.
 Der Witz der Argumentation besteht darin, auf stets vorhandene Anreizstrukturen
 und Versuchungen aufmerksam zu machen, denen wohl jeder Autor zumindest
 gelegentlich erliegt und die jedem Adressaten einen ersten Grund liefern, an
 der Bonität seiner Informationsquelle zu zweifeln.
 Dieser Grund ist dabei gänzlich unabhängig von besonderen Umständen wie den
 Charaktereigenschaften der jeweiligen Person und liegt daher universell vor.
 Nicht zwingend handelt es sich um eine bewusste Täuschung, deren Möglichkeit
 unterstellt wird; es könnte auch eine unbewusste -- wenngleich möglicherweise
 schuldhafte -- Unvorsichtigkeit vorliegen, die mit einer \emph{Selbst}täuschung
 des Autors einhergeht. Das Argument zielt darauf, unser Vertrauen in die 
 \emph{Redlichkeit und Vorsichtigkeit} oder das
 \emph{\singlequote{intellektuelle Verantwortungsbewusstsein} der Autoren} zu
 unterminieren, ohne dabei ein besonders pessimistisches Menschenbild zugrunde
 legen zu müssen.
 
 Dass jemand ein zuverlässiger Informant ist, setzt zuallererst voraus, dass er
 selbst zuverlässig und vorurteilsfrei zu erkennen vermag, damit wir nicht seine Irrtümer und
 Vorurteile übernehmen. Als allgemeines Kriterium für Zuverlässigkeit und
 Vorurteilsfreiheit nennt \authorcite{Descartes:OeuvresdeDescartes1983} die Klarheit und Deutlichkeit einer
 Erkenntnis.\footnote{\enquote{[V]ideor pro regul{\^a} generali posse statuere,
 illud omne esse verum, quod valde clare {\&} distincte percipio}
 \mkbibparens{\cite[][VII:
 35.14--15]{Descartes:OeuvresdeDescartes1983}, siehe auch
 \cite[][VIII:
 16.18--17.10]{Descartes:OeuvresdeDescartes1983}}. Berühmt wurde die
 Unterscheidung durch die weitere Ausarbeitung bei
 \textcite[vgl.][]{Leibniz:Meditationesdecognitioneveritateetideis1999}.}
 \emph{Klar} nennt \authorcite{Descartes:OeuvresdeDescartes1983} eine Erkenntnis (hier:
 \emph{perceptio}), die uns gegenwärtig (\emph{praesens}) und zugänglich
 (\emph{aperta}) ist. Als Beispiel für eine klare Erkenntnis nennt
 \authorcite{Descartes:OeuvresdeDescartes1983} in den \titel{Principia
 philosophiae} einen starken
 Schmerz\footnote{\cite[Vgl.][VIII:
 22.10--11]{Descartes:OeuvresdeDescartes1983}.} und in den \titel{Meditationes
 de prima philosophia} die Erkenntnis, dass ich denke und dass ich 
 bin\footnote{\cite[Vgl.][VII:  35.3--15]{Descartes:OeuvresdeDescartes1983}.}.
 \emph{Deutlich} ist wiederum eine klare Erkenntnis, die darüber hinaus nichts
 enthält, was nicht klar ist (und zwar, \emph{weil} sie deutlich
 ist).\footnote{\enquote{Claram voco illam
 [perceptionem; A.\,G.], quae menti attendenti praesens {\&} aperta est:
 sicut ea clar{\`e} {\`a} nobis videri dicimus, qu{\ae}, oculo intuenti
 pr{\ae}sentia, satis fortiter {\&} apert{\`e} illum movent. Distinctam autem
 illam, quae, c{\`u}m clara sit, ab omnibus aliis ita sejuncta est {\&}
 praecisa, ut nihil plan{\`e} aliud, qu{\`a}m quod clarum est, in se contineat}
 \mkbibparens{\cite[][VIII: 22.3--9]{Descartes:OeuvresdeDescartes1983}}.} So ist die
 Wahrnehmung des Schmerzes nur dann klar, wenn sie nicht mit Annahmen über die Natur und Ursache des Schmerzes
 vermischt ist. Und das \enquote{cogito, sum} stellt nur dann eine deutliche
 Erkenntnis dar, wenn wir nicht glauben, es folge neben unserer Existenz als
 \emph{res cogitans} auch unsere körperliche Existenz als \emph{res extensa}.
 
 \authorcite{Descartes:OeuvresdeDescartes1983} behauptet, dass die meisten Menschen nie in ihrem Leben etwas
 klar und deutlich
 erkennen.\footnote{\cite[Vgl.][VIII: 22]{Descartes:OeuvresdeDescartes1983}.}
 Nun ist es alltäglich, dass sich unserer Aufmerksamkeit Dinge klar darstellen; jedes Kind erkennt nach
 \authorcite{Descartes:OeuvresdeDescartes1983} Dinge klar. Was fehlt, ist die Deutlichkeit unserer
 Erkenntnisse. Wer etwas klar, aber nicht deutlich erkennt, der erkennt eine
 Wahrheit mit Gewissheit, aber mischt Überzeugungen mit hinein, die in dem, was
 er erkennen konnte, nicht enthalten sind, und für die daher eine entsprechende
 Evidenz fehlt. Wir überschätzen in diesem Fall unsere Erkenntnis. Und wenn die
 meisten Menschen vielleicht vieles klar, aber kaum etwas deutlich erkennen,
 dann ist zu erwarten, dass ihre Mitteilungen Vorurteile enthalten, also
 Behauptungen, die durch das klar Erkannte nicht gedeckt sind. Eben dies liegt
 auch \authorcite{Descartes:OeuvresdeDescartes1983}' Argument zugrunde, Autoren neigten oft dazu, ihre
 eigenen Erkenntnisse zu überschätzen.\footnote{Siehe oben,
 S.~\pageref{Abschnitt:DescartesunddieUeberinterpretation}.}
 
 Es geht bei diesem Argument um die Verletzung epistemischer Pflichten bei dem
 ursprünglichen Erwerb einer Erkenntnis durch den Informanten selbst (durch
 Wahrnehmung, Vernunfttätigkeit oder auch wiederum durch die Mitteilung anderer)
 ebenso wie um die Verletzung seiner Aufrichtigkeitspflicht bei der Weitergabe
 des Wissens. \authorcite{Descartes:OeuvresdeDescartes1983} muss hierbei nicht behaupten, dass die
 Verletzung solcher Pflichten die Regel und nicht die Ausnahme sei. Ebenso wie sein
 Traumargument in den \titel{Meditationes de prima philosophia} nur auf der
 Prämisse aufbaut, dass wir \emph{manchmal} träumen, ohne ein Kriterium zur
 Abgrenzung von Traum und Wirklichkeit zu
 haben,\footnote{\phantomsection\label{Anmerkung:KriteriumzurUnterscheidungTraumWirklichkeitbeiDescartes}\cite[Vgl.][VII:
 19.8--23]{Descartes:OeuvresdeDescartes1983}. In der sechsten Meditation nennt
 er freilich als ein solches Kriterium die Kohärenz eines Erlebens mit unseren
 anderen (v.\,a~früheren) Erkenntnissen: \enquote{nunc enim
 adverto permagnum inter utrumque ese discrimen, in eo qu{\`o}d nunquam
 insomnia cum reliquis omnibus actionibus vitae a memori{\^a} conjugantur, ut
 ea quae vigilanti occurrunt} \mkbibparens{\cite[][VII: 89.21--25]{Descartes:OeuvresdeDescartes1983}}.} reicht hier die
 Prämisse, dass Autoren wenigstens \emph{manchmal} unredlich oder in ihrem eigenen Erkenntnisgewinn
 unvorsichtig sind, ohne dass wir aus unserer Perspektive darüber urteilen
 können, ob sie es in einem konkreten Fall sind. Wie das Argument des bösen
 Dämons in den \titel{Meditationes} zeigt, genügt sogar die bloße
 Denkmöglichkeit eines Irrtumsszenarios, um Wissensansprüche zu unterminieren.
 
 \item\phantomsection\label{Abschnitt:DescartesundderWiderstreitderMeinungen}
 Selbst wenn wir keinen Grund hätten, anderen Menschen Unredlichkeit oder
 Unvorsichtigkeit zu unterstellen, bliebe doch ein gewichtiger Grund zum Zweifel
 an testimonialem Wissen bestehen: Menschen vertreten \emph{widersprechende Meinungen}. Wie nach
 ihm David \name[David]{Hume} und Immanuel \name[Immanuel]{Kant}\footnote{Vgl.
 \cite[][3]{Hume:ATreatiseofHumenNature2007}; \cite[][A
 ix]{Kant:KritikderreinenVernunft2003}, \cite[][IV:
 8.4--9]{Kant:GesammelteWerke1900ff.}; \cite[][B
 vii--ix]{Kant:KritikderreinenVernunft2003}, \cite[][III:
 7.2--8.25]{Kant:GesammelteWerke1900ff.}.} sieht \authorcite{Descartes:OeuvresdeDescartes1983} durch die
 Verbreitung sich widersprechender Meinungen die Autorität von Wissenschaften
 untergraben, wobei \name[Immanuel]{Kant} und \name[David]{Hume} dadurch insbesondere das Ansehen
 der Metaphysik gefährdet sehen. Die Uneinheitlichkeit des für wahr
 Gehaltenen mag auf manchen Gebieten stärker und auf anderen weniger stark
 ausgeprägt sein, ja innerhalb der einen oder anderen Disziplin gibt es
 vielleicht gar keine offensichtlichen Differenzen oder es gibt nur eine sehr kleine
 Minderheit, die einer überwältigenden Mehrheit gegenübersteht.
 
 \authorcite{Descartes:OeuvresdeDescartes1983} sieht wiederum in der \emph{Möglichkeit} des
 Widerspruchs das Ansehen \emph{jedes} tradierten Wissens unterminiert, auch
 dort, wo faktisch keine große Gruppe an Abweichlern von einer verbreiteten
 Auffassung existiert. Das Ansehen einer Meinung kann schließlich nicht von der
 \emph{Anzahl} derer abhängen, die sie vertreten:
 Stimmenmehrheit ist kein Argument bei der Erforschung der Wahrheit. Man
 bedenke, dass die Vertreter aller wissenschaftlichen Revolutionen -- vom
 Heliozentrismus bis zur Relativitätstheorie -- zunächst eine kleine Minderheit
 bildeten.\footnote{Nach
 \authorfullcite{Kuhn:DieStrukturwissenschaftlicherRevolutionen1967}
 \enquote{werden die wissenschaftlichen Revolutionen durch ein wachsendes, doch
 {\punkt} oft auf eine kleine Untergruppe der wissenschaftlichen Gemeinschaft
 beschränktes Gefühl eingeleitet, daß ein existierendes Paradigma aufgehört hat,
 bei der Erforschung eines Aspekts der Natur {\punkt} in adäquater Weise zu
 funktionieren}
 \parencite[][104]{Kuhn:DieStrukturwissenschaftlicherRevolutionen1967}.}
 
 Hierin scheint \authorcite{Descartes:OeuvresdeDescartes1983} sein stärkstes Argument zu
 sehen, denn im \titel{Discours de la M{\'e}thode} wird diese Diagnose zum
 entscheidenden Argument gegen die leichtfertige Berufung auf testimoniales
 Wissen: Durch die Tatsache, dass viele einander widerstreitende Meinungen
 existieren, werde das Selbstdenken quasi zur unfreiwilligen Notwendigkeit. Denn
 selbst \emph{wenn} wir willens wären, unsere epistemische Selbständigkeit
 aufzugeben, müssten wir uns doch zumindest entscheiden, \emph{wessen} Meinung
 wir als die unsrige übernehmen. Aber ohne eigenes Wissen scheint es
 aussichtslos zu sein, eine begründete Entscheidung zugunsten bestimmter
 Informanten und gegen andere zu
 fällen.\footnote{\enquote{[I]e ne pouuois
 choisir personne dont les opinions me semblassent deuoir estre
 prefer{\'e}e a celles des autres, {\&} ie me trouuay comme
 contraint d'entreprendre moymesme de me conduire}
 \mkbibparens{\cite[][VI: 16.26--29]{Descartes:OeuvresdeDescartes1983}}.}
 Jedenfalls beweist der Dissens zwischen Informanten, dass falsche Auskünfte wirklich und wir daher im
 Falle testimonialer Überzeugungen tatsächlich mit der Möglichkeit der
 Fehlinformation konfrontiert sind. Ob unser Informant seinerseits
 epistemische Pflichten verletzt hat, kann dabei offen bleiben.
 
 \item\phantomsection\label{Abschnitt:DescartesundhistorischeKenntnisse} Obwohl
 \authorcite{Descartes:OeuvresdeDescartes1983} damit den Zweifel an der Möglichkeit testimonialen Wissens
 hinreichend begründet zu haben glaubt, liefert er ein drittes Argument.
 Selbst unter der kontrafaktischen Annahme der Zuverlässigkeit sämtlicher
 Mitteilungen, der Einhelligkeit unter Informanten und der absoluten Redlichkeit
 der Überbringer des Wissens, bliebe der Erwerb von Wissen aus Büchern
 unbefriedigend (und dasselbe gilt wohl auch für Wissen, das auf anderem Wege,
 aber immer noch aus zweiter Hand erworben wurde); denn wir erwürben dadurch
 keine Wissenschaft, sondern \singlequote{historische Kenntnisse} von
 Wissenschaften.\footnote{\enquote{[I]ta enim non scientias videremus didicisse,
 sed historias} \mkbibparens{\cite[][X:
 367.22--23]{Descartes:OeuvresdeDescartes1983}}.} Als Beispiel eignet sich die
 Mathematik als Disziplin, die all denen, die sich am ihrem
 Methodenideal orientieren, wegen der Gewissheit und Klarheit ihrer
 Erkenntnisse und der Einhelligkeit der Mathematiker untereinander als Paradigma
 von Wissenschaftlichkeit gilt.\footnote{Aber auch \name[Immanuel]{Kant}, der
 eine Orientierung an der Mathematik für die Philosophie zurückweist, sieht die
 Mathematik privilegiert; \cite[vgl.][B~x]{Kant:KritikderreinenVernunft2003},
 \cite[][III:  9.7--9]{Kant:GesammelteWerke1900ff.}. Siehe dazu auch
 unten, Kap.~\ref{subsubsection:EndlichesundUnendlichesErkennen}.} Hier können wir annehmen, dass die
 ersten beiden Kritikpunkte an testimonialem Wissen zu vernachlässigen sind. Und doch
 findet \authorcite{Descartes:OeuvresdeDescartes1983} einen gewichtigen Grund, die
 \singlequote{Büchergelehrsamkeit} auch -- vielleicht sogar gerade -- innerhalb der Mathematik
 zurückzuweisen: Wir mögen Lehrbücher der Mathematik zurate ziehen, deren Inhalt
 fehlerfrei ist und deren Autoren keinerlei Absicht haben, uns zu betrügen. So
 wären wir doch keine Mathematiker, wenn wir lediglich
 deren Inhalte korrekt reproduzieren könnten. Mathematiker wären wir erst, wenn
 wir dieselben \emph{Kompetenzen} erworben haben, die bei der Generierung der in den
 Lehrbüchern enthaltenen Erkenntnisse angewandt wurden. Und dazu muss man die
 Beweise \emph{selbst} führen und das Wissen \emph{selbst} erzeugen
 (oder dies zumindest \emph{können}). Testimoniales Wissen widerstreitet der
 Forderung nach Methodenkompetenz, die stures Auswendiglernen nicht als
 Verdienst anerkennt.
 
 Es ist diese dritte Überlegung, die in der Philosophie der
 Aufklärung\footnote{Siehe unten, Kapitel
 \ref{subsection:BewertungvonInformationennachihrerART}.} bis hin zu
 \name[Immanuel]{Kant}\footnote{Siehe Kap.
 \ref{section:MuendigkeitundPhilosophie}.} und sogar darüber hinaus\footnote{So
 schreibt \authorcite{Hegel:GesammelteWerke} in der \titel{Phänomenologie des
 Geistes}:
 \enquote{Was die \ori{mathematischen} Wahrheiten betrift, so würde noch weniger
 [als bei historischen Wahrheiten; A.\,G.] der für einen Geometer gehalten
 werden, der die Theoreme \singlename{Euklid}s \ori{auswendig} wüßte, ohne ihre
 Beweise, ohne sie, wie man im Gegensatze sich ausdrücken könne, \ori{inwendig}
 zu wissen.} \mkbibparens{\cite[][IX: 31.35--32.1]{Hegel:GesammelteWerke}}.} ihr Potential entfalten wird.
 Eine solche Aversion gegen reines Reproduzieren-können gelernter Inhalte und eine entsprechende Forderung
 nach der Ausbildung methodischer Kompetenzen bestimmt später die Wirkung von Christian
 \name[Christian]{Thomasius} als eines universitären
 Lehrers.\footnote{\cite[Vgl.][241--243]{Albrecht:ChristianThomasius1999}.}
 In der Aufklärungsphilosophie im deutschsprachigen Raum von \name[Christian]{Thomasius} über
 \authorcite{Wolff:Psychologiaempirica1968} bis hin zu \name[Immanuel]{Kant} wird gerade dieses Argument den zentralen
 Gesichtspunkt der Diskussion um testimoniales Wissen bestimmen. Deswegen werde
 ich in den Kapiteln \ref{subsection:BewertungvonInformationennachihrerART} und
 \ref{section:MuendigkeitundPhilosophie} noch ausführlicher darauf eingehen.
 \authorcite{Descartes:OeuvresdeDescartes1983} hingegen übernimmt diese Argumentation nicht in den \titel{Discours de la M{\'e}thode}, was dafür spricht, dass er ihr keine entscheidende Bedeutung oder wenig Überzeugungskraft
 beimisst. Und soweit ich sehe, spielt diese Überlegung auch in der aktuellen
 Diskussion in der Sozialen Erkenntnistheorie keine Rolle, ebenso wenig wie bei
 ihren Bezugsautoren \name[David]{Hume} und \name[Thomas]{Reid}. Deren Thema ist im Anschluss
 an \authorcite{Descartes:OeuvresdeDescartes1983}' Hauptargumentationsstrang die Frage nach der
 \emph{Gewissheit} testimonialer Erkenntnisse: Lassen sich alle Zweifel an der Wahrheit von
 überbrachten Informationen ausräumen? Und inwieweit ist dies nötig, um sie als
 \enquote{Wissen} bezeichnen zu können?
\end{nummerierung}


Es handelt sich bei \authorcite{Descartes:OeuvresdeDescartes1983}' Argumenten um drei Arten von
Vorbehalten gegenüber testimonialem Wissen, die voneinander weitgehend
unabhängig sind -- wobei die ersten beiden Argumente die Gemeinsamkeit haben,
dass sie die Möglichkeit der Fehlinformation zu erweisen suchen (wenngleich auf
unterschiedlichen Wegen), während das dritte Argument diese Möglichkeit gar
nicht voraussetzt. Alle drei Argumente intendieren zumindest \emph{prima facie}
die Zurückweisung \emph{jeglichen} testimonialen Wissens und nicht nur die
Zurückweisung eines eingrenzbaren Bereichs desselben. Sollten sie gültig sein,
so etablieren sie einen umfassenden testimonialen Skeptizismus.

Wie auch immer man diese Argumente nun bewerten mag, so ist
sich \authorcite{Descartes:OeuvresdeDescartes1983} doch der Folgen eines gänzlichen Verzichts auf
testimoniales Wissen bewusst, denn dieses kommt in jedem menschlichen Denken
notwendiger Weise in erheblichem Umfang vor: Schließlich haben wir alle als
Kinder unser Wissen von anderen übernehmen müssen, ohne auch nur die Chance zur
eigenen bewussten Kontrolle unseres Wissenserwerbs zu haben.
\footnote{\cite[Vgl.][VI: 13.1--12]{Descartes:OeuvresdeDescartes1983}:
\enquote{Et ainsi encore ie pensay que, pource que nous auons tous est{\'e} enfants auant que
  d'estre hommes, {\&} qu'il nous a fallu long temps estre gouuernez
  par nos appetis {\&} nos Precepteurs, qui estoient souuent
  contraires les vns aux autres, {\&} qui, ny les vns ny les autres, ne nous
  conseilloient peutestre pas tousiours le meilleur, il est presqu'impossible
  que nos jugements soient si purs, ny si solides qu'ils auroient est{\'e}, si
  nous auions eu l'vsage entier de nostre raison d{\`e}s le point de nostre
  naissance, {\&} que nous n'eussions iamais est{\'e} conduit que par elle.}}
Hier liegt das Hauptproblem des erkenntnistheoretischen Individualismus: Das
Vertrauen in das Wissen anderer steht zumindest hinsichtlich der zeitlichen
Entwicklung am \emph{Anfang} unseres Erkennens. Es ist keine Art des
Wissenserwerbs, die wir erst spät auf der Grundlage anderer Arten erwerben,
sondern gerade die erste Art, Wissen zu erwerben. \authorcite{Descartes:OeuvresdeDescartes1983} denkt wohl
noch nicht daran, dass bereits mit dem Erwerb einer Sprache Wissen über die Welt
erworben wird.\footnote{In diesem Sinne zeigt Peter
F.~\textcite{Strawson:KnowingfromWords1994}, dass unsere epistemische Praxis mit
Wissen aus zweiter Hand anfangen muss und ohne dieses gar nicht möglich wäre. Es
handelt sich dabei in der Regel nicht um die explizite Weitergabe von Wissen
durch Mitteilung, sondern um die implizite Weitergabe in der Vermittlung von
\enquote{cognitive-practical ways of proceeding into which we were initiated
when we learned our language} (\cite[][415]{McDowell:KnowledgebyHearsay1994}).}
Aber er erkennt doch, dass unser erstes Wissen oder zumindest der größte Teil
unseres anfänglich erworbenen Wissens nicht durch \emph{eigenes Nachdenken} und
eigene \emph{bewusste} Wahrnehmung (oder auch nur \emph{bewusstes} Erfassen
mitgeteilter Informationen) zustande kommt, weil unsere Fähigkeit zum bewussten
und kritischen Erkenntnisgewinn erst ausgebildet werden muss. Deshalb sei die Übernahme fremder Ansichten, wie sie notgedrungen am
Anfang eines jeden intellektuellen Lebensweges steht, zunächst ein unkritisch
hingenommener Prozess.\footnote{Dies entspricht freilich nur der Darstellung in
den \titel{Regulae ad directionem ingenii} und dem \titel{Discours de la
M{\'e}thode}; in den \titel{Principia philosophiae} argumentiert
\authorcite{Descartes:OeuvresdeDescartes1983} über das Zeugnis der
\emph{Sinne}, dem wir anfangs unkritisch folgten: \enquote{Quoniam infantes nati sumus, {\&}
varia de rebus sensibilibus judicia pri{\`u}s tulimus, qu{\`a}m integrum
nostr{\ae} rationis usum haberemus, multis praejudiciis {\`a} veri cognitione
avertimur} \mkbibparens{\cite[][VIII:
5.5--8]{Descartes:OeuvresdeDescartes1983}}. In
den \titel{Meditationes} lässt er zunächst offen, woher seine aus frühester
Jugend hingenommenen Vorurteile stammen \mkbibparens{vgl.
\cite[][VII: 17.2--10]{Descartes:OeuvresdeDescartes1983}}. Im weiteren Verlauf
argumentiert er dann aber lediglich gegen die Verlässlichkeit von Wahrnehmung
und Vernunftgebrauch, die Verlässlichkeit testimonialer Erkenntnis wird gar
nicht erst eigens thematisiert.} Wenn
\authorcite{Thomasius:ChristianThomasiuseroeffnetDerStudirendenJugendzuLeipzigineinemDiscoursWelcherGestaltmandenenFrantzoseningemeinemLebenundWandelnachahmensolle?1994}
seine Vorurteilstheorie mit der Feststellung beginnt, dass der Mensch als Kind
erst vieles lernen müsse, was er notgedrungen ungeprüft zu übernehmen und damit
als Vorurteil anzunehmen habe, dann schließt er damit genau an
\authorcite{Descartes:OeuvresdeDescartes1983}' Vorstellungen
an.\footnote{\cite[Vgl.][104]{Schneiders:AufklaerungundVorurteilskritik1983}.}


Der erkenntnistheoretische Individualismus taugt also nicht als Beschreibung
unserer normalen intellektuellen Entwicklung. Den Erwerb unserer ersten
Erkenntnisse können wir gerade nicht bewusst kontrollieren. Der Ausweg wiederum
kann nach \authorcite{Descartes:OeuvresdeDescartes1983} nur darin bestehen, später im Leben einmal alles
Gelernte als falsch anzusehen und sich zu fragen, welche Erkenntnisse sich dem
isolierten Denker als unbezweifelbar wahr
erweisen.\footnote{\enquote{\dots\unkern quibus [pr{\ae}judiciis; A.\,G.] non
aliter videmus posse liberari, qu{\`a}m si semel in vita de iis omnibus
studeamus dubitare, in quibus vel minimam incertitudinis suspicionem reperiemus}
\mkbibparens{\cite[][VIII: 5.8--11]{Descartes:OeuvresdeDescartes1983}}.
\enquote{Animadverti jam ante aliquot annos qu{\`a}m multa, ineunte aetate, falsa pro veris admiserim, {\&}
qu{\`a}m dubia sint quaecunque istis postea superextruxi, ac proinde funditus
omnia semel in vit{\^a} esse evertenda, atque a primis fundamentis denuo
inchoandum, si quid aliquando firmum {\&} mansurum cupiam in scientiis
stabilire} \mkbibparens{\cite[][VII: 17.2--8]{Descartes:OeuvresdeDescartes1983}}.} Da
unser Wissen mindestens im Normalfall auf testimonialem Wissen
aufbaut, kann ein Verzicht auf dieses Wissen nur unter der Bedingung realisiert
werden, den eigenen Wissensbestand grundlegend neu aufzubauen und dazu erst
einmal \emph{alles} vorhandene Wissen zu verwerfen. \authorcite{Descartes:OeuvresdeDescartes1983} betrachtet
die natürliche Reihenfolge, der zufolge testimoniales Wissen am Anfang steht und die Grundlage
unseres Wissensbestandes bildet, als kontingent: Es sei
möglich, das eigene Denken als erste Quelle des Wissens zu nutzen und darauf
erst die anderen Arten des Wissenserwerbs -- eigene sinnliche Erfahrung und Information
durch andere -- aufzubauen. Freilich gelingt dies nur demjenigen, der einige
Übung im Gebrauch der eigenen Vernunft hat und zudem die Motive des
Individualismus einsieht. Als Option für Kinder, ihre geistige Entwicklung
sogleich auf individualistischer Grundlage zu beginnen, taugt es nicht.
Die Möglichkeit, durch Mitteilung anderer informiert zu werden, wäre dann erst
vor dem Hintergrund dieses kartesischen Programms neu zu begründen.


\Revision[Pelletier]{Freilich sieht
\authorcite{Descartes:OeuvresdeDescartes1983}, welch gravierende Folgen eine solche Forderung zunächst hat, nur noch diejenigen
Erkenntnisse als hinreichend fundiert anzusehen, die ein einzelner von Grund auf
selbst kontrolliert hat, ohne sich dabei auf genuin testimoniales Wissen zu
stützen. Es erscheint ihm zunächst sogar eher als Anmaßung, ein solches Projekt
auch nur für sich selbst in Angriff zu nehmen.\footnote{\cite[Vgl.][VI:
13.13--16.29]{Descartes:OeuvresdeDescartes1983}.} Er hält es aber nicht für
prinzipiell unmöglich, sondern nur die Umsetzung in großem Rahmen -- als
gesamtgesellschaftliches Projekt oder als Neuausrichtung der Wissenschaft -- für
problematisch oder zumindest
unrealistisch.\footnote{\authorcite{Descartes:OeuvresdeDescartes1983} gibt sich
überzeugt, \enquote{qu'il n'y auroit veritablement point d'apparence qu'vn
particulier fist dessein de reformer un Estat, en y changeant tout d{\'e}s les
fondemens, {\&} en le renuersant pour le redresser; ny mesme aussy de refomer le
corps des sciences, ou l'ordre establi dans les escoles pour les enseigner}
(\cite[][VI: 13.21--26]{Descartes:OeuvresdeDescartes1983}).} Er selbst könne für
sich allein den Gesamtbestand seines Wissens jedoch auf explizit
individualistischer Grundlage rekonstruieren.\footnote{\cite[Vgl.][VI:
13.27--14.1]{Descartes:OeuvresdeDescartes1983}.} Es gelingt
dies, insofern er letztlich zu zeigen vermag, dass unsere herkömmlichen
Erkenntniswege -- die eigene Erfahrung und die Informationen, die wir von
anderen erlangen -- durchaus Wissen
konstituieren.\footnote{\Revision[Pelletier]{Allerdings benötigen wir dafür auch
das Wissen um die rationale Fundierung der Gewissheit solcher Erkenntnisweisen. Deshalb kann nach
\authorcite{Descartes:OeuvresdeDescartes1983} der Atheist kein Wissen erlangen;
schließlich steht ihm der Gottesbeweis nicht zur Verfügung, mittels dessen die
Zuverlässigkeit unserer Erkenntnisquellen erst demonstriert wird
\parencite[vgl.][VII: 141.3--13]{Descartes:OeuvresdeDescartes1983}.}} Inwieweit
ihm dies tatsächlich gelingt, kann hier außen vor bleiben. Der Weg, den er wählt
(der Rekurs auf einen Gottesbeweis), ist für Philosophen wie
\name[Immanuel]{Kant} jedenfalls nicht gangbar.}

Was wir durch den Gebrauch unserer Vernunft \enquote{klar und deutlich}
einsehen, an dessen Wahrheit besteht kein Zweifel -- so
jedenfalls meint \authorcite{Descartes:OeuvresdeDescartes1983} in der dritten
Meditation auf der Grundlage der zuvor bewiesenen Existenz Gottes nachweisen zu
können.\footnote{Nach \authorcite{Descartes:OeuvresdeDescartes1983} sind nur
Intuition und Deduktion zulässige Mittel wissenschaftlichen Erkenntnisgewinns.
Dabei versteht er unter Deduktion dasjenige, was aus dem intuitiv erkannten mit
Notwendigkeit erschlossen wird \mkbibparens{\cite[vgl.][X:
369.18--22]{Descartes:OeuvresdeDescartes1983}}.
Zur Wissenschaft zählt also nichts weiter als die Folgerungshülle intuitiver
Erkenntnis, weswegen der Begriff der Intuition das Fundament darstellt, worunter
er \emph{expressis verbis} nicht wie
\authorcite{Wolff:Discursuspraeliminarisdephilosophiaingenere1996} und andere
die Erfahrungserkenntnis (siehe dazu Kap.
\ref{subsection:VerstandundRezeptivitaet} dieser Arbeit), sondern ausschließlich
ein unmittelbares Begreifen des Geistes:
\enquote{Per \ori{intuitum} intelligo, non fluctuantem sensuum fidem, vel mal{\`e} componentis imaginationis judicium fallax; sed
  mentis pur{\ae} {\&} attent{\ae} tam facilem distinctumque conceptum, vt de
  eo, quod intelligimus, nulla prorsus dubitatio relinquatur; seu, quod idem est, mentis
  pur{\ae} {\&} attent{\ae} non dubium conceptum, qui {\`a} sol{\^a} rationis
  luce nascitur} \mkbibparens{\cite[][X:
  368.13--19]{Descartes:OeuvresdeDescartes1983}}.} \Revision[Pelletier]{Wäre
  dies die einzig legitime Basis unseres Wissens, so ergäbe sich ein
  Verständnis von Selbstdenken, das sich durch seine Radikalität auszeichnete,
  aber auch selbst \emph{ad absurdum} führte: Es handelte sich um eine radikale
  Form des Individualismus, die nichts gelten lassen will als das, was jeder
  einzelne aus seinem eigenen Denken heraus als wahr erkennen kann. Dies scheint
  zunächst jede Form rezeptiv erlangten Wissens und \emph{a fortiori} jede
  Form testimonialem Wissen \emph{per se} zurückzuweisen. Die Legitimität
testimonialen Wissens müsste erst über individualistische Wissensquellen
umständlich rekonstruiert
werden.\footnote{\cite[Vgl.][431]{Stevenson:WhyBelieveWhatPeopleSay?1993}:
\enquote{On \authorcite{Descartes:OeuvresdeDescartes1983}'s new individualistic
approach, testimony can have evidential force only in a very secondary way, if
at all.}}}

\subsection{David Humes testimonialer Reduktionismus}
\label{subsubsection:DavidHume}
Das Problem des radikalen Selbstdenkens liegt darin, dass sich aus reiner
Vernunft das Corpus unseres Wissens nicht rekonstruieren lässt. Niemand ist so
vermessen zu behaupten, wir könnten aus reiner Vernunft
erkennen, dass Rom im Jahre 753 vor unserer Zeitrechnung gegründet wurde, dass Bananen
beim Reifen gelb werden oder dass es draußen in diesem Moment regnet. Die
radikale Konzeption des Selbstdenkens hätte daher eine absurde Revision des
Wissensbegriffs zu Konsequenz. Um den Umfang dessen, was wir gemeinhin und
vernünftiger Weise als unser Wissen bezeichnen, einzufangen, müssen wir also
weitere Wissensquellen neben der Vernunft zulassen. Die nächste als legitim und
zuverlässig anerkannte Wissensquelle ist die je eigene (sinnliche) Erfahrung.
Daraufhin stellt sich dann die Frage, ob wir unseren Wissensbestand bereits mit
diesen zwei Quellen -- Vernunft und Erfahrung -- rekonstruieren können, oder ob auch testimonialem
Wissen eine eigene Wissensquelle zuzusprechen ist.

In der neueren Erkenntnistheorie nimmt die Diskussion testimonialen Wissens
entsprechend dieser Überlegung oft die Frage zu ihrem Ausgangspunkt, ob testimoniales
Wissen als \emph{secondary epistemic link} auf nicht-testimoniales Wissen (also
Wissen aus Vernunft und aus eigener Erfahrung) reduzierbar und die Annahme einer eigenständigen Wissensquelle somit
unnötig ist. Die Position des \enquote{testimonialen Reduktionismus} wird dabei
häufig David \name[David]{Hume} attestiert.\footnote{\cite[Siehe
z.\,B.][532]{Grundmann:AnalytischeEinfuehrungindieErkenntnistheorie2008}, sowie
die Ausführungen von Axel
\textcite[vgl.][]{Gelfert:HumeonTestimonyRevisited2010}, der sich selbst
kritisch zu diesem \enquote{received view} verhält: \enquote{In contemporary
discussions of the epistemic status of testimony-based beliefs, David Hume is
typically cast in the role of \enquote{global reductionist}, who demands that
each of us must have first-hand, non-testimonial evidence of the reliability of
(relevant reference classes of) testimony, before accepting any new instance of
it} \parencite[][60]{Gelfert:HumeonTestimonyRevisited2010}.} Dieser glaube, dass
testimoniales Wissen sich aus je eigenem Wahrnehmungswissen und weiteren Prämissen wie der Verlässlichkeit entsprechender
Zeugnisse, welche wir wiederum aus eigenen Erfahrungen erworben haben,
ergibt.\footnote{\cite[Vgl.][46,
48]{Wilholt:SozialeErkenntnistheorie2007}, \cite[][144]{Audi:Epistemology1998}.
Dass dies eine gute Charakterisierung \name[David]{Hume}s ist, bezweifelt
\textcite{Gelfert:HumeonTestimonyRevisited2010}.} Es ist aber bereits fraglich, ob \name[David]{Hume}
überhaupt eine Theorie testimonialen Wissens
vertritt.\footnote{\cite[Vgl.][410]{Welbourne:IsHumeReallyaReductivist2002}:
\enquote{\name[David]{Hume} has no \ori{theory} of testimony, properly so-called; hence
not even a bad theory.}
\cite[Ebenso][303]{Faulkner:DavidHumesReductionistEpistemologyofTestimony1998}:
\enquote{\name[David]{Hume} does \ori{not} give a theory of testimony.}} Wenn
überhaupt, dann muss der Leser sie erst aus seinen Ausführungen des zehnten
Abschnitts des \titel{Enquiry Concerning Human Understanding} heraus
rekonstruieren.
Nach Michael \authorfullcite{Root:HumeontheVirtuesofTestimony2001} vertritt \name[David]{Hume} nur im
\titel{Enquiry} einen testimonialen Reduktionismus, nicht aber im
\titel{Treatise}.\footnote{\cite[Vgl.][]{Root:HumeontheVirtuesofTestimony2001}.}
\authorfullcite{Gelfert:HumeonTestimonyRevisited2010} behauptet, dass \name[David]{Hume} eher einem
\emph{lokalen} Reduktionismus nahe komme.\footnote{\cite[Vgl.][73]{Gelfert:HumeonTestimonyRevisited2010}.}
Während der \singlequote{globale testimoniale Reduktionismus} behauptet, wir
könnten unser gesamtes Wissen auf der ausschließlichen Grundlage
nicht-testimonialen Wissens rekonstruieren, fordert der \singlequote{lokale
testimoniale Reduktionismus} für \emph{einzelne} Fälle von Mitteilungen, die
Glaubwürdigkeit der Information auch unter Rückgriff auf bereits vorhandenes
Wissen zu belegen, wobei dieses Wissen auch auf Mitteilungen beruhen
darf.\footnote{\cite[Vgl. hierzu][]{Fricker:AgainstGullibility1994}.} Ein
lokaler Reduktionismus akzeptiert, dass unser Wissenserwerb auf irreduzibel testimonialem Wissen
beruht, fordert aber, dass wir -- möglicherweise ab einem gewissen Alter,
welches wir dann als Eintritt in die Mündigkeit beschreiben könnten -- neues
testimoniales Wissen nur erwerben können, indem wir fundierte Erkenntnisse (etwa über die Glaubwürdigkeit von Mitteilungen über bestimmte Themen) als Prämissen zugrunde legen. Schließlich
sei es auch nach \name[David]{Hume} nicht notwendig, testimoniales Wissen auf Wissen
erster Hand zu reduzieren, um unser Vertrauen in Mitteilungen zu
regulieren.\footnote{\cite[Vgl.][73]{Gelfert:HumeonTestimonyRevisited2010}:
\enquote{Testimony, thus, needs to be regulated by the same kinds of maxims as
experience in general. But regulating our reliance on testimony in ways that
accord with reason and experience, does not require reducing testimonial
knowledge to first-hand experience.}} Auch \authorfullcite{Root:HumeontheVirtuesofTestimony2001}
bestreitet, dass \name[David]{Hume} einem globalen Reduktionismus verpflichtet sei. Denn sowohl bei der
Einschätzung der Glaubwürdigkeit eines Informanten, als auch bei der
Einschätzung der \mbox{(Un-)} Wahrscheinlichkeit des Behaupteten lasse sich auch
nach \name[David]{Hume} auf testimoniales Wissen
zurückgreifen.\footnote{\cite[Vgl.][30]{Root:HumeontheVirtuesofTestimony2001}:
\enquote{Hume’s account of testimony is individualistic in one sense. B must
weigh, for himself, A’s credibility against the implausibility that p before
believing that p based on A’s testimony. But Hume does not maintain that B must
rely entirely on his own evidence in estimating A’s credibility or p’s
implausibility. On Hume’s account, B’s backward-looking evidence that A is a
credible witness can include the testimony of other witnesses and his belief
about the prior probability that p can be based on the testimony of other
witnesses as well.}} \authorfullcite{Faulkner:DavidHumesReductionistEpistemologyofTestimony1998} behauptet, dass sich
\name[David]{Hume} bei eingehender Betrachtung als der \emph{einzige} wirkliche Vertreter eines
testimonialen Reduktionismus erweise. Aber auch er bestreitet die Korrektheit
der Standardinterpretation, schließlich sei die Erfahrungsbasis, die \name[David]{Hume}
für testimoniales Wissen in Anschlag bringt, nicht so beschaffen, dass wir aus
verschiedenen Erfahrungen auf die Glaubwürdigkeit von Informationen und
Informanten schließen müssten. Vielmehr nähmen wir die Glaubwürdigkeit direkt
wahr -- unter anderem in unserer
Selbstwahrnehmung.\footnote{\cite[Vgl.][305]{Faulkner:DavidHumesReductionistEpistemologyofTestimony1998}:
\enquote{Hume does \ori{not} state that we \ori{infer} the credibility of
testimony from our past observations of the conjunctions between testimony and
the testified facts, \ori{but} that we judge there will be such a conjunction
because \ori{we observe the veracity of testimony}. That is, we can make a
\ori{direct}, rather than inferential, judgement of the credibility of
testimony.}} Die Erfahrungsbasis, die \name[David]{Hume} beschreibt, besteht
nicht ausschließlich in Paaren von Berichtswahrnehmungen und zugehörigen
Tatsachenwahrnehmungen, sondern in anthropologischen Tatsachen, die die
Wahrhaftigkeit fördern.
Jeder weiß -- auch von sich selbst --, dass Menschen zur Wahrheit tendieren und
mit (einem Gefühl der) Scham auf aufgedeckte eigene Lügen
reagieren.\footnote{\enquote{Were not the memory tenacious to a certain degree;
had not men
commonly an inclination to truth and a principle of probity; were they not
sensible to shame, when detected in a falsehood: Were not these, I say,
discovered by \ori{experience} to be qualities, inherent in human nature, we
should never repose the least confidence in human testimony}
\parencite[][90]{Hume:AnEnquiryConcerningHumanUnderstanding1964}.} Es sind
solche Überlegungen, die die Vermutung nähren, die Rechtfertigungsgrundlage liege
nach \name[David]{Hume} nicht -- oder wenigstens nicht nur -- in der
bisherigen Erfahrung mit fremden
Informationen.\footnote{Eine weitere Deutungsmöglichkeit liefert
\authorfullcite{Root:HumeontheVirtuesofTestimony2001}, der die Rolle von
Konventionen und gesellschaftlich geforderten Tugenden verweist:
\enquote{First, B can rely on his past observations of witnesses (a
backward-looking reason). He can estimate A’s credibility based on his past
experience that witnesses like A speak the truth. Second, B can rely on
conventions (a forward-looking reason).
He can estimate A’s credibility based on the fact that A and B are members of
an epistemic community in which giving credible testimony or being a credible
witness is an artificial virtue, a conventional solution to a problem of human
coordination, viz., one member revising his beliefs to match the reasonable
beliefs of others} \parencite[][19]{Root:HumeontheVirtuesofTestimony2001}.}

Belege für die Interpretation \name[David]{Hume}s als eines testimonialen
Reduktionisten finden sich im zehnten Abschnitt des \titel{Enquiry Concerning Human Understanding}.
Dieser Abschnitt \titel{Of Miracles} ist nicht einfach als eigenständiger Essay
innerhalb des \titel{Enquiry} zu betrachten, der einen religionskritischen
Nebenschauplatz eröffnet. In erster Linie untersucht er eine zentrale
Fragestellung des \titel{Enquiry}, die sich direkt aus dem vierten Abschnitt
\titel{Sceptical Doubts concerning the Operations of the Understanding} ergibt.
Es sei von Interesse herauszufinden, was die Natur jener Evidenz ist, durch die
wir auch dort, wo das gegenwärtige Zeugnis unsere Sinne und unsere Erinnerung
nicht zulangt, erkennen, was es gibt und was der Fall ist.\footnote{\enquote{It
may, therefore, be a subject worthy of curiosity, to enquire what is the nature
of that evidence, which assures us of any real existence and matter of fact,
beyond the present testimony of our senses, or the records of our memory}
\parencite[][23]{Hume:AnEnquiryConcerningHumanUnderstanding1964}.}


Für weitgehend unproblematisch hält \name[David]{Hume} unseren Wissenserwerb im Bereich
der \emph{relations of ideas} und derjenigen Tatsachen (\emph{matters of
fact}), von denen wir durch eigene sinnliche Wahrnehmung oder durch die
Erinnerung an frühere eigene sinnliche Wahrnehmung wissen. Der
größte Teil unseres Wissens beruht jedoch nicht auf diesen wenigen Quellen:
Sonst wüssten wir tatsächlich neben mathematischen Wahrheiten bloß darüber
Bescheid, was uns direkt umgibt oder umgab. Ließe unser Wissensbegriff nur als Wissen
zu, was auf eigener Überlegung oder eigener Wahrnehmung beruht, so hätte dies
wiederum absurde Konsequenzen: Nur der Mörder selbst oder ein direkter
Augenzeuge kennte die Wahrheit über seine Tat; ein polizeilicher Ermittler oder
ein Richter könnte niemals wissen, was geschah, insofern er nicht zufällig
selbst Augenzeuge (oder der Mörder) ist.

\name[David]{Hume} legt sich bereits im vierten Abschnitt des \titel{Enquiry
Concerning Human Understanding} fest, dass es sich um ein Kausalitätsverhältnis
handeln muss, auf dessen Grundlage wir auf das Bestehen von Sachverhalten
schließen, die wir nicht selbst wahrnehmen oder wahrgenommen
haben,\footnote{\enquote{All reasonings concerning matter of fact seem to be
founded on the relation of \ori{Cause and Effect}. By means of that relation
alone we can go beyond the evidence of our memory and senses}
\parencite[][24]{Hume:AnEnquiryConcerningHumanUnderstanding1964}.} also auch im
Falle solcher Sachverhalte, die uns berichtet werden. Wenn wir keine direkte
Kenntnis einer Tatsache haben, sondern nur indirekte Kenntnis, so ist unser
Wissen Ergebnis einer Schlussfolgerung (\enquote{reasoning}). Als Beispiel einer
Tatsache, von der zu wissen das Ergebnis einer Schlussfolgerung ist, führt
\name[David]{Hume} den Fall an, dass jemand von dem Aufenthalt eines Freundes in
Frankreich weiß, weil er einen Brief von ihm erhalten hat, in dem er ihm dies
mitteilt. Dies könnte ein klarer Fall testimonialen Wissens sein (es wäre immer
noch ein Fall testimonialen Wissens -- wenngleich kein so
klarer Fall --, wenn als Grundlage des Wissens nicht eine Mitteilung im Wortlaut
des Briefes dient, sondern etwa der Poststempel); jedenfalls ist testimoniales Wissen ein Paradebeispiel für das
Wissen um Tatsachen, die unseren Sinnen weder zugänglich sind noch zugänglich
waren und auf die wir nach \name[David]{Hume}s Darstellung im vierten Abschnitt
des \titel{Enquiry} daher auf dem Wege einer Schlussfolgerung gelangen. Dazu
müssen wir -- so behauptet \name[David]{Hume} -- die Kette von
Wissensweitergaben rekonstruieren, die von dem berichteten Sachverhalt über den
oder die ersten Augenzeugen und möglicherweise einen, mehrere oder viele Mittler
(die deutschsprachige Aufklärungsphilosophie spricht von
\enquote{Ohrenzeugen}\footnote{Siehe
Anm.~\ref{Anmerkung:AugenzeugenundOhrenzeugen} auf
S.~\pageref{Anmerkung:AugenzeugenundOhrenzeugen}.}) bis zu uns führt.


\name[David]{Hume} scheint sich Geschichtsschreibung folgendermaßen
vorzustellen: Wir können Wissen erwerben, indem wir uns auf das Wissen eines
Informanten stützen, insofern wir Grund zu der Annahme haben, dass unser
Informant auch tatsächlich entsprechendes Wissen besitzt. Erwarb er dieses
Wissen seinerseits wieder von einem anderen Informanten, benötigen wir wiederum
die Annahme, dass dieser Informant wirkliches Wissen besitzt, entweder von einem
weiteren Informanten oder aus der eigenen Wahrnehmung. An irgendeiner Stelle
muss unsere Kette von Informanten bei jemandem ankommen, der das Wissen nicht
von jemand anderem hat, sondern aus eigener Anschauung.\footnote{\enquote{We learn the events of former ages from history; but then we must peruse the
  volumes, in which this instruction is contained, and thence carry up our
  inferences from one testimony to another, till we arrive at the eye-witnesses
  and spectators of these distant
  events} \parencite[][39]{Hume:AnEnquiryConcerningHumanUnderstanding1964}.} Nun sei für jeden
solchen Übergang eine eigene Schlussfolgerung nötig (was im Falle von Berichten
über Ereignisse, die lange vergangen sind, mitunter auch recht lange
Schlussfolgerungsketten ergäbe). Wenn $A$ beispielsweise als Anfang der
Informationskette wahrnimmt, dass $p$, dann ist $A$ hinreichend gerechtfertigt
in der Überzeugung, dass $p$, und \emph{weiß} daher, dass $p$. Wenn $A$ nun $B$
dieses Wissen mitteilt, dann kann auch $B$ in ihrer gewonnenen Überzeugung
hinreichend gerechtfertigt sein und somit \emph{wissen}, dass $p$. Aber dazu
müsse $B$ eben darauf \emph{schließen}, dass $p$. Als Prämissen für diesen
Schluss stehe ihr zum einen die eigene Wahrnehmung zur Verfügung, dass $A$ sagt,
dass $p$; jeder Schluss auf entfernte Tatsachen beginne bei einer direkten
Wahrnehmung oder
Erinnerung\footnote{\cite[Vgl.][39]{Hume:AnEnquiryConcerningHumanUnderstanding1964}:
\enquote{[O]ur conclusion from experience carry us beyond our memory and senses,
and assure us of matters of fact, which happened in the most distant places and
most remote ages; yet some fact must always be present to the senses or memory,
from which we may first proceed in drawing these conclusions.}}.
$B$ benötigt aber weiter einen Obersatz, demzufolge sie von $A$s Behaupten, dass
$p$, auf die Wahrheit von $p$ schließen darf. Und dieser Obersatz beschreibt
nach \name[David]{Hume} einen allgemeinen Ursache-Wirkungs-Zusammenhang, der
von einem berichteten Sachverhalt hin zur Aussage unseres Informanten reichen
muss. Und wenn es sich um eine längere Informationskette handelt, benötigen wir
eine entsprechend lange Kette von Obersätzen, die zusammen einen komplexen
Kausalzusammenhang von dem Ereignis über sämtliche Informanten bis zu uns
beschreiben.


Wie können Obersätze von Schlüssen, die den Übergang von der Erkenntnis, dass
jemand sagt, etwas sei der Fall, zu der Erkenntnis, eben dies sei der Fall,
begründet sein? Nach \name[David]{Hume} haben wir Zutrauen in menschliches
Zeugnis, weil wir aus Erfahrung von der Aufrichtigkeit menschlicher Berichte
wissen. Er behauptet, unsere Gewissheit entstamme lediglich unserer Beobachtung
der Wahrhaftigkeit menschlichen Zeugnisses und der üblichen Übereinstimmung der
Tatsachen mit den Berichten von Zeugen.\footnote{\enquote{It will be sufficient
to observe, that our assurance in any argument of this kind is derived from no
other principle than our observation of the veracity of human testimony, and of
the usual conformity of facts to the reports of witnesses}
\parencite[][90]{Hume:AnEnquiryConcerningHumanUnderstanding1964}.} Zwischen
einer Äußerung und einem Sachverhalt, den die Äußerung beschreibt, besteht ein
kausaler, regelmäßig wahrnehmbarer Zusammenhang. Es folgt also auf die
Beobachtung eines Sachverhalts regelmäßig die Beobachtung der Äußerung oder
umgekehrt auf die Beobachtung der Äußerung die Beobachtung des Sachverhalts. Wir
stellen auf diese Weise eine kausale Beziehung zwischen beiden fest, die so
geartet ist, dass die Äußerung kausale Folge (Wirkung) des Sachverhalts
(Ursache) ist. Die Äußerung ist dadurch ein Zeichen für einen Sachverhalt
ähnlich wie Krankheitssymptome Zeichen einer bestimmten Infektion sind, als
deren Wirkung sie auftreten.\footnote{Zu dieser Standardlesart und ihrer
Verbreitung in der jüngeren Literatur der Sozialen Erkenntnistheorie vergleiche
man die Übersicht von Axel
\textcite[vgl.][61--63]{Gelfert:HumeonTestimonyRevisited2010}.}


Dabei bleibt zunächst offen, ob unsere Erfahrungsbasis den jeweiligen
Informanten selbst betrifft oder eine Gruppe von
Informanten.\footnote{\cite[Vgl.][20]{Root:HumeontheVirtuesofTestimony2001}:
\enquote{$B$’s estimate of $A$’s credibility can be based on either (3) or (4):
(3) $B$ observes that $A$’s testimony has been conjoined with the truth and
infers that her testimony usually conforms to the truth; (4) $B$ observes that
the testimony of witnesses like $A$ has been conjoined with the truth and infers
that her testimony usually conformes to the truth.}} Während aber die erste
Option uns in vielen Fällen in die Situation bringt, gar keine
Bewertungsgrundlage zu haben, weil uns die jeweiligen Informanten zum ersten Mal
zur Verfügung stehen, ermöglicht uns die zweite Option, die
Zuverlässigkeit von Informanten zu bewerten, mit denen wir bisher noch keinerlei
Erfahrung gemacht
haben.\footnote{\cite[Vgl.][20]{Root:HumeontheVirtuesofTestimony2001}:
\enquote{Because witnesses can be alike, can form a natural kind, B may be able
to predict that A will give honest testimony, even if B has never observed A’s
testimony.}} Man denke hier etwa an die Einheimische, die wir als Touristen in
einer fremden Gegend nach dem Weg fragen. Wir haben in aller Regel noch keine
Erfahrungen mit ihr selbst als Informantin machen können, wohl aber mit der
Zuverlässigkeit von Einheimischen im allgemeinen. \name[David]{Hume} sagt allgemein
über Ursache-Wirkungs-Zusammenhänge, wir erwarteten, dass auf Ereignisse, die
einander \singlequote{ähnlich} sind, auch wiederum einander ähnliche Ereignisse
folgen.\footnote{\cite[Vgl.][63]{Hume:AnEnquiryConcerningHumanUnderstanding1964}:
\enquote{we may define a cause to be \ori{an object, followed by another, and
where all the objects, similar to the first, are followed by objects similar
to the second.}}} Worin jedoch die Ähnlichkeit besteht, bleibt gerade auch im
Falle von Informanten und Informationen offen. Es bleibt also die Schwierigkeit
bestehen, wie sich die relevanten Bezugsgruppen
zusammensetzen.\footnote{\cite[Vgl.
hierzu][83--85]{Coady:Testimony1992}. Nach Michael
\textcite[vgl.][20]{Root:HumeontheVirtuesofTestimony2001} muss \name[David]{Hume}
voraussetzen, dass es \singlequote{natürliche Arten} (\enquote{\emph{natural
kinds}}) von Informanten gibt. Es handelt sich freilich um einen Spezialfall des
allgemeinen Problems, wie die \singlequote{ähnlichen} Ereignisse in \name[David]{Hume}s
Definition des Kausalitätsbegriffs zu bestimmen sind.}

Der Schluss von einer Mitteilung auf eine bestehende Tatsache (\enquote{\emph{matter
of fact}}), die sich unserer unmittelbaren Wahrnehmung und unserer Erinnerung
verschließt, sei eine von mehreren Möglichkeiten des Schließens auf der
Grundlage eines
Ursache-Wirkungs-Zusammenhangs.\footnote{\cite[Vgl.][23--24]{Hume:AnEnquiryConcerningHumanUnderstanding1964}.}
Wir entdecken also einen konstanten Zusammenhang zwischen menschlichen
Behauptungen und dem Bestehen der behaupteten Sachverhalte, noch \emph{bevor}
wir damit beginnen, etwas auf die Behauptung anderer hin für wahr zu halten.
Erst \emph{nachdem} uns dieser Zusammenhang durch vielfältige Erfahrungen
bekannt werden konnte, beginnen wir, aus den Behauptungen anderer auf die
entsprechenden Sachverhalte zu schließen. Ob der Ausdruck \enquote{Kausalität}
den gesuchten Zusammenhang gut beschreibt, sei zwar fraglich (wobei \name[David]{Hume}
dies offensichtlich bejaht, aber bezweifelt, dass dies ohne weiteres
konsensfähig ist). Von größerer Bedeutung sei aber, dass die Evidenz einer
Überzeugung im Falle testimonialen Wissens auf einer Schlussfolgerung beruhe, die einer
Grundlage bedürfe. Und weil diese Grundlage nicht a priori sei, könne es sich
nur um die Erfahrung eines regelmäßigen Zusammenhangs
handeln.\footnote{\enquote{The reason, why we place any credit in witnesses and historians, is
not derived from any \ori{connexion}, which we perceive \ori{a priori},
between testimony and reality, but because we are accustomed to find a
conformity between them} \parencite[][\pno~91f.]{Hume:AnEnquiryConcerningHumanUnderstanding1964}.}

Wo keine solche Grundlage vorliegt und wir dennoch auf einen
Sachverhalt schließen, genau dort -- so scheint es zunächst -- liegt
Leichtgläubigkeit vor, denn ein weiser Mensch richtet sich in seinen
Überzeugungen nach den vorliegenden
Evidenzen.\footnote{\cite[Vgl.][89]{Hume:AnEnquiryConcerningHumanUnderstanding1964}:
\enquote{A wise man {\punkt} proportions his belief to the evidence.}} Doch
\name[David]{Hume}s Punkt ist ein anderer: Leichtgläubigkeit
(\enquote{\emph{credulity}}) sei die häufigste Schwäche der Menschen. Obwohl
Erfahrung mit Zeugnissen und der Wahrhaftigkeit die einzige Grundlage für
unser Vertrauen in die Mitteilungen von Geschehnissen sei, richteten wir uns
doch in der Bewertung von Mitteilungen nicht nach unseren Erfahrungen, sondern
neigten dazu, alles Mögliche zu glauben. Wir glaubten Berichte über
Geistererscheinungen, Zauberei und Wunder, so sehr ihnen auch die täglichen
Erfahrungen und Beobachtungen widersprechen.\footnote{\enquote{No weakness of
human nature is more universal and conspicuous than what we commonly call
credulity, or a too easy faith in the testimony of others; and this weakness is
also very naturally accounted for from the influence of resemblance. When we
receive any matter of fact upon human testimony, our faith arises from the very
same origin as our inferences from causes to effects, and from effects to
causes; nor is there any thing but our experience of the governing principles of
human nature, which can give us assurance of the veracity of men. But tho’
experience be the true standard of this, as well as of all other judgments, we
seldom regulate ourselves entirely by it; but have a remarkable propensity to
believe whatever is reported, even concerning apparitions, enchantments, and
prodigies, however contrary to daily experience and observation}
\parencite[][78]{Hume:ATreatiseofHumenNature2007}.}


Wichtig ist hier, dass \name[David]{Hume} sich gegen
Leichtgläubigkeit bei solchen Behauptungen wendet, die der Alltagserfahrung
widersprechen. Wenn \name[David]{Hume} die \singlequote{Leichtgläubigkeit}
(\enquote{credulity}) vieler Menschen geißelt, muss dies weder ein generelle
Ablehnung testimonialen Wissens nach sich ziehen, noch auch nur eine
Relativierung der Gewissheit testimonialen Wissens. Vielmehr attackiert er eine
übertriebene Bereitschaft, den Berichten anderer auch dort Glauben zu schenken,
wo jemand gute Gründe hat, dies nicht zu
tun.\footnote{\cite[Vgl.][420]{Welbourne:IsHumeReallyaReductivist2002}:
\enquote{\name[David]{Hume} (unlike \name[Thomas]{Reid}) always uses the word
\enquote{credulity} in a bad sense; for him, as for us, it is the name of an
intellectual vice, and does not refer, as it does in \name[Thomas]{Reid}, to the
general propensity to believe what one is told.}}


Gute Gründe, einer Mitteilung \emph{keinen} Glauben zu schenken, haben wir
\name[David]{Hume} zufolge insbesondere dort, wo jemand etwas sehr
Unwahrscheinliches, Wunderbares oder gar Absurdes
behauptet.\footnote{\cite[Vgl.][88--94]{Hume:AnEnquiryConcerningHumanUnderstanding1964}.}
Bei genauerem Hinsehen zeigt sich, dass es \name[David]{Hume} gerade bei seiner
allgemeinen Kritik an Wunderberichten nicht um die Evidenz \emph{für} eine
Überzeugung geht, die sich nach dem Informanten und dessen Eigenschaften
richtet, sondern um die entgegengesetzte Evidenz \emph{gegen} diese Überzeugung,
die mit dem Inhalt -- oder der \emph{Art} des Inhalts -- der Information zu tun
hat.\footnote{\cite[Vgl.][68]{Gelfert:HumeonTestimonyRevisited2010}:
\enquote{Hume rules out miraculous testimony on the basis of specific features
of the content of the testimony in question, rather than on the basis of
failures of the institution of testimony as such.}} Deswegen kann \name[David]{Hume}
seine Kritik an testimonialem Wissen auch auf eine bestimmte Klasse von Fällen
(die Wunderberichte) beschränken, während er gewöhnliche Fälle unangetastet
lässt.\footnote{\cite[Vgl.][74]{Gelfert:HumeonTestimonyRevisited2010}.} Erzählt
uns jemand, dass John F. Kennedy von den Toten auferstanden, Tante Ernas
unheilbare Krankheit geheilt oder Onkel Philipp über Wasser gelaufen sei,
spricht genug gegen die Glaubwürdigkeit des Berichts, um ihm unsere Zustimmung
zu verweigern. Dies gilt selbst dann, wenn die Glaubwürdigkeit der Autorität,
die uns berichtet, sonst über allen Zweifel erhaben
ist.\footnote{\cite[Vgl.][92]{Hume:AnEnquiryConcerningHumanUnderstanding1964}.}
Doch gerade in den kuriosesten Fällen zeige sich eine völlig irrationale
Bereitschaft, den Berichten zu vertrauen; und diese Bereitschaft resultiere aus
unseren Affekten der Überraschung (\emph{surprize}) und des Staunens
(\emph{wonder}).\footnote{\cite[Vgl.][95]{Hume:AnEnquiryConcerningHumanUnderstanding1964}:
\enquote{The passion of \ori{surprize} and \ori{wonder}, arising from miracles,
being an agreeable emotion, gives a sensible tendency towards the belief of
those events, from which it is derived.}} Sie sei im Bereich des Religiösen
besonders
auffällig.\footnote{\cite[Vgl.][95]{Hume:AnEnquiryConcerningHumanUnderstanding1964}:
\enquote{But if the spirit of religion join itself the love of wonder, there is
an end of common sense; and human testimony, in these circumstances, loses all
pretensions to authority.}} Religiöse Menschen glauben so abstruse Behauptungen
wie dass ein Toter wieder lebendig wurde oder dass ein Mensch über Wasser lief und Wasser
in Wein verwandelte. Leichtgläubig ist also im allgemeinen nicht derjenige, der
eine Mitteilung ohne Evidenz ihrer Glaubwürdigkeit zur Grundlage seiner
Überzeugungen macht -- denn für die Glaubwürdigkeit von Mitteilungen haben
wir genügend (nach \name[David]{Hume}: empirische) Evidenz --, sondern derjenige, der
die entgegenstehende Evidenz gegen die Wahrheit des Berichteten nicht würdigt.
Allerdings kann diese gegensätzliche Evidenz nur dadurch die Glaubwürdigkeit
einer Mitteilung unterminieren, weil sie von derselben Art ist. Deswegen ist die
Explikation der Grundlagen testimonialen Wissens so wichtig für
\authorcite{Hume:AnEnquiryConcerningHumanUnderstanding1964}. Es steht Erfahrung gegen
Erfahrung und das Ergebnis ist ein Zustand des Nichtwissens, weil sich die
Evidenzen auf beiden Seiten gegenseitig aufheben.


\name[David]{Hume}s Ziel im zehnten Abschnitt des \titel{Enquiry} besteht darin,
die Unmöglichkeit testimonialen Wissens von \emph{Wundern} nachzuweisen, wobei er
unter einem \enquote{Wunder} die Verletzung eines allgemeinen Naturgesetzes
(durch den Willen einer Gottheit oder ähnliches) versteht. Im Rahmen dieser
Kritik an Wunderberichten beansprucht er zu zeigen, dass unser Wissen von
Naturgesetzen, die zu verletzen das \emph{Definiens} des Wunderbegriffs
ausmacht\footnote{\cite[Vgl.][\pno~93,
Anm.:]{Hume:AnEnquiryConcerningHumanUnderstanding1964} \enquote{A miracle may be
accurately defined, \ori{a transgression of a law of nature by a particular
volition of the Deity, or by the interposition of some invisible agent.}} Im
Haupttext des zehnten Abschnitts hebt er ausschließlich auf die Verletzung eines
Naturgesetzes ab: \enquote{A miracle is a violation of the laws of nature}
(\cite[][93]{Hume:AnEnquiryConcerningHumanUnderstanding1964}).}, und
testimoniales Wissen dieselbe Grundlage und darum auch dieselben Grade an
Gewissheit haben, weswegen sie ihre jeweilige Gewissheit auch gegenseitig
aufheben, wenn sie sich widersprechen.


Widerspricht also ein menschliches Zeugnis einem allgemeinen Naturgesetz, dann
stehen sich zwei einander widersprechende Erkenntnisse gegenüber, die beide
die gleiche Grundlage haben: unsere Erfahrung. Da wir aber keinen Grund haben,
dem einen mehr als dem anderen zu trauen, und menschliches Zeugnis kein höheres
Ansehen genießt als das Zeugnis unserer Sinne, können wir nur schließen, dass
den Berichten über Wunder weniger Gewissheit zukommt als anderen Mitteilungen,
da unsere alltägliche Erfahrung gegen die Wahrheit des Berichts spricht. Im
Falle religiöser Schriften, wo uns nicht ein direkter Augenzeuge ein Wunder
berichtet, sondern eine lange Mitteilungskette vorliegt, dort komme es gar zu
gänzlichen Verlust an Gewissheit. Denn wenn $A$ sieht, dass $p$, und dies $B$
mitteilt, dann weiß möglicherweise auch $B$, dass $p$ (wenn sie hinreichende
Gründe hat, $A$s Auskunft zu vertrauen). Und wenn $B$ dieses Wissen wiederum an
$C$ weitergibt, erlangt auch $C$ dasselbe Wissen, dass auch $A$ und $B$ haben --
vorausgesetzt sie hat hinreichende Gründe anzunehmen, dass $B$ vertrauenswürdig
ist und selbst wiederum hinreichende Gründe hat zu glauben, dass $p$. Aber die
Zuverlässigkeit der Gründe oder die Evidenz (\emph{evidence}) von $p$ könne bei
jedem Übergang nur abnehmen, niemals größer werden. Dies werde ich das
\enquote{\label{DasHumeschePrinzip}\name[David]{Hume}sche Prinzip}
nennen.\footnote{\name[Immanuel]{Kant} spricht in diesem Fall von
\enquote{subordinirten Zeugen} und konstatiert ebenso eine abnehmende
Glaubwürdigkeit bei Zunahme der Länge der Mitteilungskette. Umgekehrt führe eine
größere Anzahl von Zeugen, die unabhängig voneinander, aber übereinstimmend
dasselbe berichten, zu größerer Glaubwürdigkeit: \enquote{In der reihe der
subordinirten Zeugen nimmt die historische Glaubwürdigkeit ab; in der Reihe der
coordinirten Zeugen nimmt sie zu. Die Reihe der coordinirten Zeugnisse heißt das
öffentliche Gerüchte, der subordinirten Zeugen hingegen eine mündliche
Ueberlieferung. Wenn in den coordinirten Zeugnisse der Augenzeuge unbekannt ist;
so heißts eine gemeine Sage}
\mkbibparens{\cite[][]{Kant:LogikPhilippi1966}, \cite[][XXIV:
450.23--28]{Kant:GesammelteWerke1900ff.}}} (Dieses Prinzip ist in seiner Allgemeinheit
falsch, wie später zu zeigen sein wird.\footnote{Siehe unten, Kapitel
\ref{AbschnittzuCrusiusundKritischemJournalismus} ab
S.~\pageref{AbschnittzuCrusiusundKritischemJournalismus}.}) $C$ hat also in
jedem Fall weniger gute Gründe zu glauben, dass $p$, als $B$. Hat $C$ nun
anderweitige Gründe für die Annahme, dass $\lnot p$, dann überragen diese Gründe
leicht die Gründe aus der Kette von Mitteilungen, die sie für die Überzeugung
hat, dass $p$. Handelt es sich bei $p$ um ein Wunder, dann haben wir aus unserer
Erfahrung direkte Evidenz dafür, dass $\lnot p$, die mindestens der Evidenz
entspricht, die wir für die Verlässlichkeit eines einzelnen Übergangs von einem
Mittelzeugen auf den nächsten haben.



In den meisten Fällen ist es eine völlig richtige Reaktion auf die Mitteilungen
anderer, dass wir unsere Überzeugungen nach ihren Äußerungen richten.
Schließlich gehört die Mitteilung anderer zu den wichtigsten, nützlichsten und
notwendigsten Arten, Wissen über Tatsachen zu
erlangen.\footnote{\cite[Vgl.][90]{Hume:AnEnquiryConcerningHumanUnderstanding1964}:
\enquote{[T]here is no species of reasoning more common, more useful, and even
more necessary to human life, than that which is derived from the testimony of
men, and the reports of eye-witnesses and spectators.}} \name[David]{Hume} vertritt
keine revisionistische Konzeption, die einen großen oder gar den größten Teil
dessen, was wir für testimoniales \emph{Wissen} halten, verwirft, sondern kritisiert
spezielle Fälle von Informationen. Je nachdem welche Art von Bericht vorliegt
und wer diesen Bericht abgibt, könne sogar von einem Beweis und nicht nur von
Wahrscheinlichkeit gesprochen werden.\footnote{\enquote{And as the evidence,
derived from witnesses and human testimony, is founded on past experience, so it
varies with the experience, and is regarded either as a \ori{proof} or a
\ori{probability}, according as the conjunction between any particular kind of
report and any kind of object has been found to be constant or variable}
\parencite[][91]{Hume:AnEnquiryConcerningHumanUnderstanding1964}.
\cite[Vgl.][]{Gelfert:HumeonTestimonyRevisited2010}.} Muss man
\name[David]{Hume} nun einen testimonialen Reduktionismus unterstellen, um sein
Argument gegen die Glaubwürdigkeit von Wunderberichten zu rekonstruieren?


Aber auch wenn nicht die Evidenz \emph{für} die Wahrheit einer Mitteilung im
Fokus der Kritik \name[David]{Hume}s steht, sondern die gegensätzliche Evidenz
\emph{gegen} ihre Glaubwürdigkeit, bleibt das Vertrauen auf die Mitteilung
anderer in seinem Bild doch stets rechtfertigungsbedürftig. Daher ist die
Bedeutung der Erfahrung auch für die Überzeugungskraft alltäglicher Mitteilungen
nicht von der Hand zu weisen. Denn solange wir keine Gründe haben, die
\emph{für} die Glaubwürdigkeit von bestimmten Mitteilungen oder Mitteilungen im
allgemeinen sprechen, haben wir auch keinerlei Gründe, etwas für wahr zu halten,
was uns erzählt wird. Und vernünftigerweise werden wir unsere Zustimmung
verweigern, wo wir keine Gründe haben. \name[David]{Hume} sagt, dass wir in
Alltagssituationen über solche Gründe tatsächlich in einem Maße verfügen, dass
wir in der Regel Wissen durch die Mitteilungen anderer akquirieren. Gewiss
neigen wir dazu, Informationen aus zweiter Hand viel zu leichtfertig zu
vertrauen, insofern wir die Gegengründe nicht hinreichend beachten. Aber
zunächst sei unser Vertrauen durch vorangegangene Erfahrungen zumindest
\emph{prima facie} berechtigt. Während \authorcite{Descartes:OeuvresdeDescartes1983} nur Gründe
aufzählt, die \emph{gegen} die Zuverlässigkeit von Informationen sprechen (und
auf das Thema testimoniales Wissen nicht wieder rekurriert), behauptet
\name[David]{Hume}, dass uns aus unserer Erfahrung viele Gründe \emph{für} die
Verlässlichkeit von Informationen vorlägen.

\section{Nicht-individualistische Ansätze}
\label{section:NichtindividualistischeAnsaetze}
\subsection{Thomas Reids Credulismus}
\label{subsubsection:ThomasReid}
Nach der Standardinterpretation \name[David]{Hume}s behauptet dieser, dass wir
erfahrungsabhängiges Wissen um die Zuverlässigkeit menschlichen Zeugnisses
benötigen, \emph{bevor} wir erstmalig durch Berichte anderer Wissen erwerben
können. Doch die Überlegung, dass wir erst Erfahrungen mit der Verlässlichkeit
menschlicher Zeugnisse machen müssen, um hinterher auf dieser Grundlage von
einer Aussage, die uns gegenüber gemacht wird, auf das Bestehen eines durch
diese Aussage beschriebenen Sachverhalts schließen zu können, erweist sich bei
näherer Betrachtung schlicht als absurd.\footnote{\cite[Vgl.][3]{Anscombe:WhatIsIttoBelieveSomeone2008}:
\enquote{The view needs only to be stated to be promptly rejected. It was always
absurd, and the mystery is how \name[David]{Hume} could ever have entertained it.}}
Thomas \name[Thomas]{Reid} weist darauf hin, dass wir nach diesem Modell als Kinder
zunächst misstrauisch sein müssten -- schließlich fehlt uns zunächst jede
Evidenz \emph{für} die Zuverlässigkeit testimonialen Wissens --,
um mit zunehmendem Alter immer unkritischer zu werden, nachdem wir immer mehr
positive Belege in unserer Erfahrung sammeln
konnten.\footnote{\cite[Vgl.][\pno~194\,f.:]{Reid:AnInquiryIntotheHumanMindonthePrinciplesofCommonSense1997}
\enquote{Children, on this supposition, would be absolutely incredulous; and
therefor absolutely incapable of instruction: those who had little knowledge of
human life, and of the manners and characters of men, would be in next degree
incredulous: and the most credulous men would be those of greatest experience,
and of the deepest penetration; because, in many cases, they would be able to
find good reasons for believing testimony, which the weak and the ignorant could
not discover.}} Im Gegenteil dazu sind offensichtlich gerade Kinder besonders
gutgläubig, während Menschen mit zunehmendem Alter eine kritischere Einstellung
zu testimonialem Wissen (zumindest in bestimmten Fällen) erwerben.



\name[Thomas]{Reid} artikuliert damit eine Einsicht, die später
\authorcite{Wittgenstein:UeberGewissheit1977} in \titel{Über Gewißheit} stark macht:
\enquote{Das Kind lernt, indem es dem Erwachsenen glaubt. Der Zweifel kommt \ori{nach} dem
Glauben.}\footnote{\cite[][\S~160]{Wittgenstein:UeberGewissheit1977}.} Erst spät
lernen wir dann zwischen glaubwürdigen und unglaubwürdigen Informanten zu
unterscheiden,\footnote{\cite[Vgl.][\S~143]{Wittgenstein:UeberGewissheit1977}:
\enquote{Ein Kind lernt viel später, daß es glaubwürdige und unglaubwürdige
Erzähler gibt, als es Fakten lernt, die ihm erzählt werden.}} auch wenn wir
dadurch zuvor vieles gelernt haben, was sich später möglicherweise als falsch
herausstellt\footnote{\cite[Vgl.][\S~161]{Wittgenstein:UeberGewissheit1977}:
\enquote{Ich habe eine Unmenge gelernt und es auf die Autorität von Menschen
angenommen, und dann manches durch eigene Erfahrung bestätigt oder entkräftet
gefunden.} Siehe auch \cite[][\S\S~159,
162, 170]{Wittgenstein:UeberGewissheit1977}. Die Ähnlichkeit der Darstellungen
\name[Thomas]{Reid}s und \name[Ludwig]{Wittgenstein}s erkennt
\textcite[][434]{Stevenson:WhyBelieveWhatPeopleSay?1993}.}.
Also sollte eher das \emph{Misstrauen} die Folge unserer Erfahrungen sein und
nicht das \emph{Vertrauen}, welches die natürliche Haltung des
Menschen am Beginn seines Lebens darstellt.

Thomas \name[Thomas]{Reid} erkennt also die Absurdität einer reduktionistischen
Position, wenn diese als Beschreibung unser intellektuellen Entwicklung
verstanden wird, wie sie tatsächlich vonstatten geht. Dabei
ist er vielleicht der erste Autor, der explizit den testimonialen Reduktionismus
verwirft und einen Credulismus, also eine nicht-individualistische Position
bezüglich testimonialem Wissen vertritt.\footnote{Nach
\authorfullcite{Grundmann:AnalytischeEinfuehrungindieErkenntnistheorie2008} ist \name[Thomas]{Reid} der erste Vertreter einer
\emph{externalistischen} Position und deshalb sei seine Theorie testimonialen Wissens derjenigen \name[David]{Hume}s
überlegen;
\cite[vgl.][537]{Grundmann:AnalytischeEinfuehrungindieErkenntnistheorie2008}.
Jedoch begründet
\authorcite{Grundmann:AnalytischeEinfuehrungindieErkenntnistheorie2008} diese
Einschätzung nicht weiter. Siehe zu dieser Einordnung auch
\cite{Woudenberg:ThomasReidbetweenExternalismandInternalism2013}.} Nach
\authorcite{Reid:EssaysontheIntellectualPowersofMan2002} gibt es zwei Arten
geistiger Tätigkeiten (\enquote{operations of mind}), von denen die einen von
einzelnen Subjekten ausgeführt und individualistisch verstanden werden können,
während die anderen eines Zusammenspiels mit anderen bedürfen und irreduzibel
sozial
sind.\footnote{\cite[Vgl.][68]{Reid:EssaysontheIntellectualPowersofMan2002}:
\enquote{Some operations of our minds, from their very nature, are \ori{social},
others are \ori{solitary}.} Weiter heißt es: \enquote{By the first, I understand
such operations as necessarily suppose an intercourse with some other
intelligent being.}} \authorcite{Reid:EssaysontheIntellectualPowersofMan2002}s
bemerkenswerte Behauptung ist, dass es aussichtslos sei, geistige Tätigkeiten,
die nur als soziale Tätigkeiten denkbar sind -- die \emph{social operations of
mind} wie Mitteilen, Versprechen, Fragen oder Befehlen --, in Elemente auflösen
zu wollen, die individualistisch verstanden werden können -- \emph{solitary
operations of mind}. Die
philosophische Tradition habe zwar stets versucht, sie in Termini
individualistisch verstandener Tätigkeiten zu analysieren; dabei zeige sich aber
nur, dass allgemein völlig Verständliches mysteriös
werde.\footnote{\cite[Vgl.][68]{Reid:EssaysontheIntellectualPowersofMan2002}:
\enquote{These acts of mind are perfectly understood by every man of common
understanding; but, when philosophers attempt to bring them within the pale of
ther divisions, by analysing them, they find inexplicable mysteries, and even
contradictions, in them.}}

\begin{comment}
Gerade in der Analytischen Philosophie ist bis heute die Tendenz verbreitet,
gemeinsames Handeln und kollektive Intentionalität als (komplexes) Gefüge jeweils
individueller Handlungen und Intentionen zu analysieren.
\footnote{\authorfullcite{Tuomela:We-intentions1988} fachten eine Diskussion um den
ontologischen Status kollektiver Intentionalität an, in der
\authorfullcite{Searle:CollectiveIntentionsandActions1992} die
anti-reduktionistische Gegenposition vertrat, wonach kollektive Intentionalität
ein nicht-reduzierbares und natürliches Phänomen
darstellt; \cite[vgl.][]{Tuomela:We-intentions1988} sowie die neueren
Ausführungen in \cite{Tuomela:We-IntentionsRevisited2005}, in denen Raimo
\authorcite{Tuomela:We-IntentionsRevisited2005} auf Einwände eingeht.
\cite[Vgl.][]{Searle:CollectiveIntentionsandActions1992}.
Siehe auch die von \textcite{Tuomela:We-IntentionsRevisited2005} angeführte und
diskutierte Literatur.}\end{comment}
Dabei legt \authorcite{Reid:EssaysontheIntellectualPowersofMan2002} sein
Augenmerk auf solche Verstandeshandlungen, deren Ausführung nur durch das
gemeinsame Handeln (\enquote{\emph{intercourse}}) mit anderen vernünftigen Wesen
möglich (und zwar \emph{logisch} möglich) sei.  Dabei liegt eine
\emph{gemeinsame} Tätigkeit vor und nicht das gleichzeitige Ausführen
individueller Tätigkeiten wie in dem Fall, in dem zwei Passanten zufällig
denselben Weg durch eine Stadt zurücklegen, ohne dieses Handeln als gemeinsames
verstehen zu
können.\footnote{\authorfullcite{Smith:DerWohlstandderNationen1993} scheint sich
das Gruppenverhalten bei Tieren so vorzustellen, dass es sich nur scheinbar um
gemeinsames Verhalten handelt, wenn bspw. ein Rudel Wölfe ein Reh jagt. Nach ihm
sollten wir diese Situation so auffassen, dass mehrere Wölfe unabhängig
voneinander jeweils ein Reh jagen und es sich dabei zufällig um ein und dasselbe
Reh handelt \parencite[vgl.][16]{Smith:DerWohlstandderNationen1993}. So abwegig
diese Beschreibungsform auch sein mag, sie verdeutlicht den hier anvisierten
Unterschied.} Nach \authorcite{Reid:EssaysontheIntellectualPowersofMan2002}
liegt ein solcher Unterschied zwischen individuellem und gemeinsamem Handeln
gerade auch bei Urteilen auf der einen und Mitteilungen auf der anderen Seite
vor. Mitteilen sei eine gänzlich andere Tätigkeit als Urteilen. Wenn ein Zeuge
einem Richter eine Auskunft gibt, dann drücke diese nicht sein Urteil
(\enquote{\emph{judgment}}), sondern eine Mitteilung
(\enquote{\emph{testimony}}) aus. Fragen wir hingegen jemanden nach seiner
Überzeugungen in einer Angelegenheit der Wissenschaft, dann drückt seine Antwort
ein Urteil (\enquote{\emph{judgment}}) aus, ist aber keine Mitteilung
(\enquote{\emph{testimony}}).\footnote{\enquote{Affirmation and denial is very often the expression of testimony, which is a
  different act of the mind, and ought to be distinguished from judgment.
  
  A judge asks of a witness what he knows of such a matter to which he was an eye
  or ear witness. He answers, by affirming or denying something. But his answer
  does not express his judgment; it is his testimony. Again, I ask a man his
  opinion in a matter of science or of criticism. His answer is not testimony; it
  is the expression of his
  judgment.}
  \parencite[][\pno~406\,f.]{Reid:EssaysontheIntellectualPowersofMan2002}.}

Urteil und Mitteilung unterscheiden sich nicht nur dadurch, dass eine Mitteilung
jederzeit in Anwesenheit anderer \emph{vernehmbar artikuliert} sein
muss\footnote{\cite[Vgl.][407]{Reid:EssaysontheIntellectualPowersofMan2002}:
\enquote{Testimony is a social act, and it is essential to it to be expressed by
words or signs. A tacit testimony is a contradiction: But there is no
contradiction in a tacit judgment; it is complete without being expressed.}},
sondern vor allem auch darin, dass es sich nur bei falschen Mitteilungen um
\emph{Lügen} handelt, nicht aber bei falschen
Urteilen\footnote{\cite[Vgl.][407]{Reid:EssaysontheIntellectualPowersofMan2002}:
\enquote{In testimony a man pledges his veracity for what he affirms; so that a
false testimony is a lie: But a wrong judgment is not a lie; it is only an
error.}}. Mitteilungen
ähneln Versprechen, die Verpflichtungen begründen, während Urteile in Analogie
zu bloßen Absichtsbekundungen zu verstehen sind, mit denen keine Verpflichtung
einhergeht und die daher auch keine Rechte auf der Seite des Hörers
begründen.\footnote{Einen ähnlichen Unterschied artikuliert
\authorfullcite{Austin:OtherMinds1979} bezüglich der Formulierungen \enquote{Ich
\emph{weiß}, dass $p$} und \enquote{Ich bin \emph{mir sicher}, dass $p$}. Im
ersten Fall gibt der Sprecher dem Hörer die Erlaubnis, etwas auf seine (des
Sprechers) Verantwortung hin unter seine Überzeugungen aufzunehmen und auch
weiterzugeben -- es handelt sich um eine Mitteilung im hier besprochenen Sinn.
Im zweiten Fall hingegen artikuliert der Sprecher nur ein eigenes Urteil, dass
nicht als Mitteilung aufzufassen ist und welches der Hörer auch nur auf seine
eigene (des Hörers) Verantwortung hin übernehmen und weitergeben darf;
\cite[vgl.][98--103]{Austin:OtherMinds1979}. Einen Zusammenhang zwischen
\name[Thomas]{Reid}s Theorie sozialer Verstandestätigkeiten und
\authorcite{Austin:OtherMinds1979}s Überlegungen zur Sprechakttheorie sehen
\textcite[vgl.][]{Schuhmann:ElementsofspeechacttheoryintheworkofThomasReid1990}.}
Versprechen wiederum sind die paradigmatischen Fälle genuin sozialer
intellektueller Tätigkeiten, die nur als ursprünglich soziale Phänomene
verständlich sind und die Vorlage zu Diskussion anderer Formen sozialer Verstandestätigkeiten
liefern.\footnote{\cite[Vgl.][]{Coady:ReidandtheSocialOperationsofMind2004}.}


Bei dem Versuch, soziale Phänomene wie Versprechen, Absprachen oder Verträge auf
individualistische Handlungen zurückzuführen, kommt uns der
Verpflichtungscharakter solcher Phänomene abhanden. Und dann wird mysteriös, wie
es sein kann, dass uns unser individuelles Handeln (das Äußern eines
Versprechens) mit einer Verbindlichkeit belegen kann. Etwas Ähnliches passiere
-- so \authorcite{Reid:EssaysontheIntellectualPowersofMan2002} -- auch bei
Mitteilungen: Der genuin soziale Charakter derselben lasse sich nicht erläutern,
indem man eine Mitteilung als ein (privates) Urteil \singlequote{plus $X$}
analysiert. Wir müssen die \emph{social operations of mind} \name[Thomas]{Reid}
zufolge als ursprünglich und natürlich
ansehen\footnote{\cite[Vgl.][69]{Reid:EssaysontheIntellectualPowersofMan2002}:
\enquote{The Author of our being intended us to be social beings, and has, for
that end, given us social intellectual powers, as well as social affections.
Both are original parts of our constitution, and the exertions of both no less
natural than the exertions of those powers that are solitary and selfish.}} und
ihnen als solchen mehr Aufmerksamkeit widmen. Es sei ein Fehler, dass
Philosophen sich bei der Untersuchung menschlichen Denkens stets auf die
Betrachtung des einzelnen Denkers in Situationen der Einsamkeit
fokussieren.\footnote{\cite[Vgl.][70]{Reid:EssaysontheIntellectualPowersofMan2002}:
\enquote{Why have speculative men laboured so anxiously to analyse our solitary
operations, and given so little attention to the social? I know no other reason
but this, that, in the divisions that have been made of the mind’s operations,
the social have been omitted, and thereby thrown behind the curtain.}} Diesen
Fehler macht offensichtlich
\authorcite{Hume:AnEnquiryConcerningHumanUnderstanding1964}, wenn er den Erwerb
von testimonialem Wissen so rekonstruiert, dass die Äußerungen unserer
Informanten analog zu Krankheitssymptomen oder Fossilien verstanden werden, von
denen wir auf eine Krankheitsursache oder den früheren Zustand der Fauna
schließen. Hier gerät der soziale Aspekt des Mitteilens völlig aus dem Blick.

Nun mag es sein, dass \name[Thomas]{Reid}s Thematisierung der sozialen Grundlagen
unseres Denkens der Komplexität dieses Zusammenhangs nicht abschließend gerecht
wird. Insbesondere mag man
bezweifeln, dass es möglich ist, so scharf zwischen individuellen und sozialen
Verstandestätigkeiten zu unterscheiden, wie \name[Thomas]{Reid} dies für richtig
hält.\footnote{\cite[Vgl.][197]{Coady:ReidandtheSocialOperationsofMind2004}:
\enquote{The basic problem is that Reid’s contrasting of the solitary and the social
operations draws him unwittingly into an equally sharp contrast between things
that in reality overlap.}} Vielleicht sollten wir selbst die Annahme in Zweifel
ziehen, dass es überhaupt Verstandestätigkeiten gibt, die als ursprünglich
individualistisch zu verstehen sind. \authorcite{Coady:ReidandtheSocialOperationsofMind2004} etwa
bezweifelt, dass \name[Thomas]{Reid} umfassenden Einblick hatte, wie weitreichend unsere
eigene Urteilsfähigkeit durch den vorherigen Einfluss anderer bedingt
ist.\footnote{\cite[Vgl.][197--201]{Coady:ReidandtheSocialOperationsofMind2004}.
Möglicherweise stellt \name[Immanuel]{Kant}s Position mit ihrer Betonung der
sozialen Grundlagen des je eigenen mündigen Urteilens (siehe Kap.
\ref{section:sensuscommunis}) hier einen entscheidenden Fortschritt dar.} Wie
auch immer dies zu bewerten ist, so sollte doch der Fortschritt gegenüber
\authorcite{Hume:AnEnquiryConcerningHumanUnderstanding1964} in Bezug auf
testimoniales Wissen beachtet werden: Während
\authorcite{Hume:AnEnquiryConcerningHumanUnderstanding1964} ohne weiteres davon
ausgeht, dass die eigene direkte Wahrnehmung und die eigene Erinnerung völlig
verschieden seien von dem Erkenntnisgewinn durch Mitteilungen -- ersteres ist
\singlequote{unmittelbar}, zweiteres beruht auf einer Schlussfolgerung --, sieht
\authorcite{Reid:EssaysontheIntellectualPowersofMan2002} keinen nennenswerten
Unterschied zwischen beiden. In seiner Schrift
\titel{An Inquiry Into the Human Mind on the Principles of Common Sense}
behauptet \name[Thomas]{Reid} eine enge Analogie zwischen dem Zeugnis anderer Menschen
und dem Zeugnis der Sinne, die es angeraten sein lasse, beides zugleich
abzuhandeln.\footnote{\cite[Vgl.][190]{Reid:AnInquiryIntotheHumanMindonthePrinciplesofCommonSense1997}:
\enquote{[S]o remarkable is the analogy between these two, and the analogy
between the principles of the mind which are subservient to the one and those
which are subservient to the other, that, without further apology, we shall
consider them together.}}


Nach \name[Thomas]{Reid} handelt es sich bei dem Erwerb von Informationen durch
andere um eine selbständige, aber mit der eigenen Erfahrung auf eine Stufe gestellte
Wissensquelle. Sie bedarf daher auch keiner Rechtfertigung auf der Grundlage von
Erfahrung, sondern ist von sich auch bereits legitim. Unser testimonialer
Wissenserwerb beruht auf zwei eigenständigen, ursprünglichen Prinzipien, die
keiner vorgängigen Rechtfertigung bedürfen:
\begin{nummerierung}
\item Das \emph{Prinzip der Aufrichtigkeit} (\enquote{principle of
veracity}\footnote{\cite[][194]{Reid:AnInquiryIntotheHumanMindonthePrinciplesofCommonSense1997}.})
besagt, dass wir dazu tendieren, die Wahrheit zu sagen und die Zeichen unserer
Sprache zur Bezeichnung unserer wirklichen Überzeugungen zu
gebrauchen.\footnote{\enquote{The first of these principles is, a propensity
to speak the truth, and to use the signs of language, so as to convey our real
sentiments}
\parencite[][193]{Reid:AnInquiryIntotheHumanMindonthePrinciplesofCommonSense1997}.
In den 21 Jahre später erschienen \titel{Essays on the Intellectual Powers of
Man} heißt es: \enquote{Another first principle I take to be, That certain
features of the countenance, sounds of the voice, and gestures of the body,
indicate certain thoughts and dispositions of mind}
\parencite[][484]{Reid:EssaysontheIntellectualPowersofMan2002}.}
\item Das \emph{Prinzip der \singlequote{Gutgläubigkeit}} (\enquote{principle of
credulity}\footcite[][194]{Reid:AnInquiryIntotheHumanMindonthePrinciplesofCommonSense1997})
besagt, dass wir Vertrauen in die Aufrichtigkeit der anderen haben und glauben,
was sie uns
erzählen.\footnote{\enquote{Another original principle implanted in us by the
Supreme Being, is a disposition to confide in the veracity of others, and to
believe what they tell us}
\parencite[][194]{Reid:AnInquiryIntotheHumanMindonthePrinciplesofCommonSense1997}.
Analog in den \titel{Essays on the Intellectual Powers of Man}:
\enquote{Another first principle appears to me to be, That there is a certain
regard due to human testimony in matters of fact, and even to human authority
in matters of opinion}
\parencite[][487]{Reid:EssaysontheIntellectualPowersofMan2002}.}
\end{nummerierung}
\name[Thomas]{Reid} attackiert nicht nur den testimonialen Reduktionismus in der Form,
wonach dieser behauptet, es gebe eine Erfahrungsgrundlage für unser Vertrauen in
die Glaubwürdigkeit anderer, sondern auch die Position, die auf individualistischer Grundlage Gründe für unsere
eigene Aufrichtigkeit entwickeln zu müssen meint. Wir müssen nicht erst Gründe
dafür kennenlernen, die Wahrheit zu sagen (etwa soziale Sanktionsmechanismen
gegenüber Lügnern), sondern tendieren von Natur aus dazu. Gewiss handelt es sich
bei den beiden Prinzipien um zwei Seiten derselben Medaille, wobei das Prinzip
der Gutgläubigkeit dennoch hier das einschlägigere ist.


Das Vertrauen in Behauptungen anderer ist gegenüber dem Misstrauen, das erst
über einen Lernprozess erworben wird, ursprünglich. Wir folgen dem \singlequote{Prinzip der
Gutgläubigkeit}, bevor wir auch nur in der Lage sind, Überlegungen zur
Verlässlichkeit menschlichen Zeugnisses
anzustellen.\footnote{\cite[Vgl.][487]{Reid:EssaysontheIntellectualPowersofMan2002}:
\enquote{Before we are capable of reasoning about testimony or authority, there
are many things which it concerns us to know, for which we can have no other
evidence.}} Erst in der Folge, wenn unsere Fähigkeiten entwickelt sind, können
wir Gründe für unsere Erkenntnisprinzipien bezüglich testimonialem Wissen
finden.\footnote{\cite[Vgl.][488]{Reid:EssaysontheIntellectualPowersofMan2002}:
\enquote{[W]hen our faculties ripen, we find reason to check that propensity to
yield to testimony and to authority, which was so necessary and so natural in
the first period of life.}} Erwachsenen Menschen mit gereiften intellektuellen
Fähigkeiten ist es überhaupt erst möglich, Gründe für die Verlässlichkeit
menschlichen Zeugnisses in einer Weise vorzubringen, wie sie \name[David]{Hume}
bespricht. Hinzufügen ließe sich noch, dass gerade Überlegungen zu Charakter und
Interesselosigkeit unserer Informanten, denen großes Gewicht bei der Abwägung
zukommt, ob eine Information glaubhaft ist, oft selbst auf testimoniales Wissen
bezüglich dieser Informanten angewiesen sind.\footnote{So
\textcite[vgl.][196]{Coady:ReidandtheSocialOperationsofMind2004} gegen
\name[Thomas]{Reid}.} Ohne ursprüngliches Vertrauen in die Aufrichtigkeit
anderer käme unser Denken gar nicht in
Gang.\footnote{\cite[Vgl.][193]{Coady:ReidandtheSocialOperationsofMind2004}:
\enquote{The quest for knowledge begins with dependence upon the word of others
rather than validating that dependence some way down the path as a secondary
supplement to individual knowledge. This \singlequote{beginning} is historical
ans also epistemically normative.}} Ebenso ist das wahrheitsgemäße Aussagen ursprünglich,
die Lüge hingegen etwas, was wir erst spät erlernen und wofür wir tatsächlich
erst Gründe kennenlernen müssen. Beide Prinzipien sind natürlich auch
voneinander unabhängig. Wir haben nicht deshalb Vertrauen in die Mitteilungen
anderer, weil wir von einem allgemeinen Prinzip der Aufrichtigkeit wissen,
sondern schon lange zuvor, wenn wir über solche Prinzipien noch gar nicht
nachgedacht haben.

\name[Thomas]{Reid} beschreibt hier nach eigenem Verständnis zunächst Prinzipien, nach
denen Menschen \emph{de facto} denken, also solche, die wir tatsächlich
anwenden, ohne sie zuvor begründet zu haben. Es handelt sich um Behauptungen in
Betreff der \emph{tatsächlichen} individuellen Entwicklung menschlichen Denkens, die
gleichwohl nicht zufällig so und nicht anders verlaufe. Heißt dies dann, dass
\name[David]{Hume} zwar in Bezug auf die \emph{tatsächliche} Entwicklung menschlichen
Denkens Unrecht hat, aber ein an \authorcite{Descartes:OeuvresdeDescartes1983} angelehntes Programm der
individualistischen \emph{Rekonstruktion} dennoch durchführbar sein könnte?
Können wir den Prinzipien, denen wir \emph{de facto} folgen, solche
entgegenstellen, denen wir \emph{idealiter} folgen \emph{sollten}?

\name[Thomas]{Reid} konzidiert zumindest, dass Menschen Überlegungen anstellen
können, wie \name[David]{Hume} sie generell fordert, wenn auch erst nachdem sie zunächst dem
Prinzip der Gutgläubigkeit entsprechend Informationen erworben haben.
Möglicherweise vertritt er zwar keinen globalen, wohl aber einen lokalen
Reduktionismus, der folgendes behauptet: Es sei zwar unmöglich, unseren gesamten
Wissensbestand auf individualistischer Grundlage zu rekonstruieren; wohl aber
könnten wir, die wir als erwachsene Menschen mit reifer Vernunft schon viel über
unsere Mitmenschen wissen (und sei es aus dem Zeugnis anderer), nun bei jeder
Mitteilung überlegen, welche allgemeinen Gründe für die Glaubwürdigkeit
sprechen. Dabei sei es uns erlaubt, auf psychologisches Wissen zurückzugreifen,
das wir nicht selbst erarbeitet haben, sondern welches wir von
anderen mitgeteilt bekamen. (Wir könnten es aus
\authorcite{Kant:GesammelteWerke1900ff.}s Anthropologievorlesungen erworben
haben.)

\authorfullcite{Fricker:AgainstGullibility1994}, die bekannteste Vertreterin
eines lokalen Reduktionismus, bezeichnet den Credulismus als Leichtgläubigkeit
(\enquote{gullibility})\footnote{\cite[Vgl.][145]{Fricker:AgainstGullibility1994}:
\enquote{The thesis I advocate in opposition to a PR thesis [der These von einem
präsumtiven epistemischen Recht; A.\,G.], is that a hearer should always engage
in some assessment of the speaker for trustworthiness. To believe what is
asserted without doing so is to believe blindly, uncritically.
This is gullibility.}}, der wegen der Realitätsferne eines \emph{globalen}
Reduktionismus eben durch die Alternative eines solchen \emph{lokalen}
Reduktionismus zu begegnen sei.\footnote{\cite[Vgl.][136]{Fricker:AgainstGullibility1994}: \enquote{My
issue is the local reductionist question: whether, within a subject's coherent
system of beliefs and inferential practices {\punkt}, beliefs from testimony can
be exhibited as justified in virtue of very general patterns of inference and
justification; or if a normative epistemic principle special to testimony must
be invoked to vindicate them and explain their status as knowledge.}}
\name[Elizabeth]{Fricker} behauptet, dass nur ein lokaler Reduktionismus, der
im Falle des Erwerbs von Informationen Evidenz für die Glaubwürdigkeit des
jeweils Mitteilenden einfordert -- sonst könnte aus Sicht des lokalen
Reduktionismus der Informationserwerb nicht als mündig beschrieben werden --,
Ausdruck einer kritischen Haltung sein
könne.\footnote{\cite[Vgl.][]{Fricker:AgainstGullibility1994}.} Wenn $A$, um
sich von der Wahrheit der Aussage $B$'s zu überzeugen, in \emph{diesem} Sinne
den eigenen Verstand gebrauchen soll, so muss er aus dem Hören, was $B$ sagt,
und seinen früheren Erfahrungen mit $B$'s Auskünften (oder mit Aussagen
vergleichbarer Zeugen) auf die Wahrheit der Aussage \emph{schließen}. Wir hätten
-- wäre ein solcher Schluss möglich -- einen Begriff von Mündigkeit, der auf der
einen Seite alle unsere Erkenntnisse der Forderung nach Selbständigkeit
unterwirft und auf der anderen Seite die Möglichkeit eröffnet, testimoniales
Wissen zu haben. Man kann nun aber bezweifeln, dass es möglich ist, mit einer
solchen Strategie testimoniales Wissen in einem nennenswerten Umfang zu
begründen.\footnote{Siehe hierzu die klassische Argumentation in
\cite{Anscombe:HumeandJuliusCaesar1973}.} Auch \name[Elizabeth]{Fricker} hält es
für vergeblich, unser Wissen auf einer vollständig individualistischen Grundlage
zu rekonstruieren. Aber gegeben unseren gesamten Wissensvorrat, der sich zu
großen Teilen aus genuin testimonialem Wissen speist, lasse sich doch für
einzelne Fälle von Mitteilungen eine lokale Reduktion leisten. Die Basis für
eine solche Reduktion bildet unser gesamtes Hintergrundwissen über das Verhalten
unserer Mitmenschen, all das, was wir (Volks-) Psychologie oder
\singlequote{Menschenkenntnis} nennen können. Auch \name[Immanuel]{Kant}
beispielsweise behandelt solches Wissen unter dem Titel der
\enquote{Weltklugheit} in der pragmatischen Anthropologie und schreibt ihm eine
wichtige Funktion im Rahmen der Aufklärung zu, wie wir oben
sahen.\footnote{Siehe Kapitel \ref{subsection:DieBestimmungdesMenschen},
v.\,a.~S.~\pageref{Absatz:Weltklugheit}--\pageref{Absatz:Weltklugheit-ENDE}.}

Elizabeth \name[Elizabeth]{Fricker} propagiert den lokalen Reduktionismus als einzig
verbleibende Möglichkeit einer sozialen Erkenntnistheorie, die die Möglichkeit
testimonialen Wissens mit dem Streben nach Mündigkeit vereint. Demnach wäre das
Prinzip der Gutgläubigkeit im Kindesalter unumgänglich, aber im weiteren Verlauf
immer mehr zurückzudrängen, bis wir irgendwann gänzlich auf seine Anwendung bei
dem Erwerb neuen Wissens verzichten könnten, wenngleich unser Wissen natürlich
weiterhin von früher erworbenem genuin testimonialen Wissen abhängig bliebe, für
das wir keine lokalen Reduktionen vorliegen haben. Sowohl aus der Sicht des
Credulismus, als auch aus der des lokalen Reduktionismus hängt es von unserer
Abwägung von Gründen für und wider die Glaubwürdigkeit des Informanten ab, ob
unser Umgang mit testimonialem Wissen kritisch oder unkritisch ist. Und der
Vorteil beider Positionen ist damit, dass sie uns ermöglichen, die Forderung
nach einem kritischen Umgang mit testimonialem Wissen durch eine Vielzahl
epistemischer Regeln zu konkretisieren.
\name[Thomas]{Reid} merkt an, dass ein kritisch denkender Mensch zunehmend
seinen Erwerb testimonialen Wissens durch Überlegungen wie die von
\name[David]{Hume} angeführten begleite. Schließlich liefert uns die Erfahrung
nicht nur Belege der Unzuverlässigkeit von Mitteilungen, sondern gerade auch die
von \name[David]{Hume} angeführten Belege ihrer
Verlässlichkeit\footnote{\cite[Vgl.][488]{Reid:EssaysontheIntellectualPowersofMan2002}:
\enquote{But when our faculties ripen, we find reason to check that propensity
to yield to testimony and to authority, which was so necessary and so natural in
the first period of life.}}; nur geschieht dies nicht, \emph{bevor} wir erstmals
auf Zeugnisse rekurrieren.
\authorcite{Reid:EssaysontheIntellectualPowersofMan2002} behauptet, dass wir mit
zunehmendem Alter immer mehr Wissen auf Weisen erwerben, die nicht auf dem
Vertrauen in andere beruhen. Aber dennoch könne es niemals zu einem
völligen Verzicht auf testimoniales Wissen kommen. Denn auch wenn die Vernunft zunehmend
weniger auf die Hilfe anderer rekurrieren müsse, sei doch jeder durch seine
gesamte Lebensspanne hindurch darauf angewiesen, durch Mitteilungen anderer
Erkenntnisse zu erwerben, die durch den \singlequote{eigenen Vernunftgebrauch}
nicht erworben werden können.\footnote{\enquote{Reason hath likewise her
infancy, when she must be carried in arms: then she leans entirely upon
authority, by natural instinct, as if she was conscious of her own weakness; and
without this support, she becomes vertiginous. When brought to maturity by
proper culture, she begins to feel her own strength, and leans less upon the
reason of others; she learns to suspect testimony in some cases, and to
disbelieve it in others; and sets bounds to that authority to which she was at
first entirely subject. But still, to the end of life, she finds a necessity of
borrowing light from testimony, where she has none within herself, and of
leaning in some degree upon the reason of others, where she is conscious of her
own imbecillity}
\parencite[][195.15--25]{Reid:AnInquiryIntotheHumanMindonthePrinciplesofCommonSense1997}.}
Einen Zustand völliger kognitiver Autarkie erreichen wir niemals, weder im Sinne
des globalen, noch im Sinne des lokalen Reduktionismus.

Das Problem des erkenntnistheoretischen Individualismus -- die dauerhafte
Angewiesenheit auf testimoniales Wissen -- ist unausweichlich. Wie
\name[Thomas]{Reid} erkennt, können wir nicht einfach beschließen, nur noch auf
den \emph{eigenen} Vernunftgebrauch zu setzen und auf jedes testimoniale Wissen zu
verzichten. Wir können ihm zufolge lediglich \emph{graduell} mündiger werden,
indem wir -- ausgehend von unserer ursprünglichen umfassenden Abhängigkeit von anderen --
zunehmend mehr Erkenntnisse \emph{selbst} kontrollieren. Das heißt einerseits,
dass wir Wissen gar nicht mehr in demselben Umfang als testimoniales
Wissen erwerben, sondern durch eigene Erfahrung und eigenen Vernunftgebrauch.
Dass dies aber nur einen Teil unseres Wissenserwerbs betreffen kann, wird klar,
wenn man versucht, sich etwa unsere Rechtspraxis ohne testimoniales Wissen, also
das Gerichtswesen ohne Zeugenaussagen vorzustellen. Testimoniales Wissen ist
kein geringer, sondern der quantitativ vielleicht größte Teil unseres
Wissens.\footnote{\cite[Vgl.][557]{Reid:EssaysontheIntellectualPowersofMan2002},
wo \name[Thomas]{Reid} in Bezug auf die verschiedenen Arten wahrscheinlichen (im
Gegensatz zu demonstrativem) Wissen schreibt: \enquote{The first kind is that of
human testimony, upon which the greatest part of human knowledge is built.}} Und
es ist illusorisch, diesen Teil vollständig zu ersetzen. Noch immer werden wir
darauf angewiesen sein, einen großen Teil unseres Wissens als testimoniales
Wissen zu erwerben. Und entsprechend verheerend wirkte sich ein Verzicht auf
testimoniales Wissen aus, welches wir \emph{ohne} nicht-testimoniale
Rechtfertigung, also allein auf Grundlage des Prinzips der Gutgläubigkeit
gewinnen.\footnote{\cite[Vgl.][194]{Reid:AnInquiryIntotheHumanMindonthePrinciplesofCommonSense1997}:
\enquote{It is evident, that, in the matter of testimony, the balance of human
judgment is by nature inclined to the side of belief; and turns to that side of
itself, when there is nothing put into the oposite scale. If it was not so, no
proposition that is uttered in discourse would be believed, until it was
examined and tried by reason; and most men would be unable to find reasons for
believing the thousandth part of what is told them. Such distrust and
incredulity would deprive us of the greatest benefits of society, and place us
in a worse condition than that of savages.}} Andererseits heißt dies, dass wir
im Falle testimonialen Wissens Gründe für und gegen die Glaubwürdigkeit einer
Mitteilung kennen, die sich auch, aber nicht nur auf die Glaubwürdigkeit eines
Informanten
stützen\footnote{\cite[Vgl.][558]{Reid:EssaysontheIntellectualPowersofMan2002}:
\enquote{The belief we give to testimony in many cases is not solely grounded
upon the veracity of the testifier. In a single testimony, we consider the
motives a man might have to falsity. If there be no appearances of any such
motive, much more if there be motives on the other side, his testimony has
weight independent of his moral character. If the testimony be circumstantial,
we consider how far the circumstances agree together, and with things that are
known. It is so very difficult to fabricate a story, which cannot be detected by
a judicious examination of the circumstances, that it acquires evidence, by
being able to bear such a trial. There is an art in detecting false evidence in
judicial proceedings, well known to able judges and barristers; so that I
believe few false witnesses leave the bar without suspicion of their guilt.}}.
Die Bewertung von Mitteilungen gehört zu den wichtigsten Kompetenzen, die jeder von uns erwirbt
und die zentrale Funktionen im gesellschaftlichen Zusammenleben
erfüllen.\footnote{\cite[Vgl.][\pno~557f.]{Reid:EssaysontheIntellectualPowersofMan2002}:
\enquote{The faith of history depends upon it, as well as the judgment of solemn
tribunals, with regard to mens acquired rights, and with regard to their guilt
or innocence when they are charged with crimes. A great part of the business of
the Judge, of Counsel at the bar, of the Historian, the Critic, and the
Antiquitarian, is to canvass and weigh this kind of evidence; and no man can act
with common prudence in the ordinary occurrences of life, who has not some
competent judgment of
it.}} Die Basis hierfür bilden freilich das von anderen erworbene Wissen und die
Kompetenzen, die wir durch Unterweisung erworben haben. Hier scheint
Mündigkeit eine Grenze zu haben; den schlechthin mündigen Menschen gibt es
danach nicht.

\subsection{Christian August Crusius' Präsumtionstheorie}
\label{subsubsection:ChristianAugustCrusius}
Die Position \authorfullcite{Reid:EssaysontheIntellectualPowersofMan2002}s ist
komfortabler als die Position des Individualismus. Allerdings hat sie einen
ernsthaften Nachteil: Wir können Wissen
aus den Mitteilungen anderer erhalten, weil diese Mitteilungen \emph{de facto}
eine zuverlässige Wissensquelle sind. Wir wissen nicht zuvor um die
Zuverlässigkeit dieser Wissensquelle -- denn woher sollten wir dies zuvor
erfahren --, sondern handeln unbewusst nach Prinzipien, die unser Erkennen
leiten. Diese Prinzipien sind von uns nicht erkannt, sondern
\singlequote{angeboren}; und so erwerben wir Wissen auf der Grundlage von
Grundsätzen, die uns selten bewusst sind und die selbst keine Rechtfertigung
besitzen. Aus einer Perspektive, wie sie uns als Ausgangspunkt der
Aufklärung interessiert, ist es hingegen nicht
sinnvoll, unsere epistemische Situation nach Kriterien zu bewerten, deren
Verletzt- oder Erfülltsein das Subjekt nicht kontrollieren kann. Angeborene
Grundsätze helfen uns generell nicht weiter, schon weil sie allzu sehr nach
Vorurteilen aussehen; wir wollen bewusst \emph{vernünftigen} Grundsätzen folgen.
Eben dies ist ja die Forderung \name[Immanuel]{Kant}s, wenn dieser das
Selbstdenken über die Kontrolle der Vernünftigkeit der Grundsätze des Denkens
und Erkennens bestimmt.\footnote{Siehe oben, Kap.
\ref{subsection:SelbstdenkenbeiKant}.} Sich mit dem vorliegen unvermeidbarer
angeborener epistemischer Prinzipien zu beruhigen, erweckt den Anschein eines
Aufrufs zu Bequemlichkeit und Unmündigkeit.

Der Credulismus muss jedoch nicht unkritisch sein, insofern wir im Laufe der
Zeit lernen, wann wir Gründe haben, einer Information zu misstrauen. Es gehört
zu den wichtigen Kompetenzen, die jeder von uns im Laufe seines Lebens mehr oder
minder stark ausprägt, die Bonität von Informationen und Informanten zu
bewerten. Nur handelt es sich dabei um eine Kompetenz, die sich erst im Laufe
der Jahre und auch auf der Grundlage testimonialen Wissens ausprägt und die
nicht schon dem ersten Erwerb testimonialen Wissens zugrunde liegt. Sie
\emph{konstituiert} nicht testimoniales Wissen, aber sie \emph{reguliert} es.
Wir könnten etwa mit \authorfullcite{Austin:OtherMinds1979} sagen, dass die
Übernahme von Überzeugungen die \emph{default}-Position ist, die wir erst in
Frage stellen, wenn \emph{berechtigte} Zweifel auftauchen:
\begin{quote}
Natürlicherweise sind wir vernünftig: Wir sagen nicht, wir wüssten etwas (aus
zweiter Hand), wenn es irgendeinen konkreten Grund gibt, das Zeugnis
anzuzweifeln. Aber es muss \ori{irgendeinen} Grund geben. Es ist fundamental für
Unterhaltungen (wie für andere Dinge), dass wir berechtigt sind, anderen zu
vertrauen, außer insofern es einen konkreten Grund für Misstrauen gibt.
Menschen zu glauben, ihr Zeugnis anzunehmen, das ist der, oder ein wichtiger,
Sinn von Unterhaltungen.\footnote{\enquote{Naturally, we are judicious: we
don't say we know (at second hand) if there is any special reason to doubt the testimony: but there has to be
\ori{some} reason. It is fundamental in talking (as in other matters) that we
are entitled to trust others, except in so far as there is some concrete reason
to distrust them. Believing persons, accepting testimony, is the, or one main,
point of talking} \parencite[][82]{Austin:OtherMinds1979}.}
\end{quote}
\authorcite{Austin:OtherMinds1979} behauptet, dass wir generell nicht alle Zweifel
ausschließen können, aber auch nicht ausschließen müssen, um Wissen zu erwerben, und dass
dies für alle Arten von Wissen gilt, testimoniales wie
Wahrnehmungswissen.\footnote{\cite[Vgl.][84]{Austin:OtherMinds1979}:
\enquote{Enough is enough: it does not mean everything. Enough means enough to
show that (within reason, and for present intents and purposes) it
\singlequote{can’t} be anything else, there is no room for an alternative,
competing, description of it. It does \ori{not} mean, for example, enough to
show it isn’t a \ori{stuffed} goldfinch.}} Nach \authorcite{Austin:OtherMinds1979} haben wir
ein präsumtives epistemisches Recht, das unser Vertrauen in die jeweilige Autorität
zunächst legitimiert, auch wenn uns keine Gründe für ein solches Vertrauen
vorliegen.\footnote{Der Ausdruck \enquote{präsumtives epistemisches Recht}
(\enquote{\emph{presumtive epistemic right}}) stammt von Elizabeth
\textcite[][140]{Fricker:AgainstGullibility1994}.} Diese Präsumtion der
Glaubwürdigkeit bildet somit das zentrale (und irreduzible) Prinzip, welches den
Erwerb testimonialen Wissens ermöglicht. Ich hatte für diese Position bereits
den Ausdruck \enquote{Credulismus} eingeführt.\footnote{Siehe oben,
S.~\pageref{Fussnote:BegriffdesCredulismus}, Fußnote
\ref{Fussnote:BegriffdesCredulismus}.}

Es kann gute Gründe dafür geben, misstrauisch zu sein. Aber Vertrauen oder die
Annahme der Zuverlässigkeit einer Information und eines Informanten bleibt die
\emph{default}-Position. Deswegen bietet sich hier der Begriff der Präsumtion
an: Ähnlich wie bei juristischen Präsumtionen -- etwa der Unschuldsvermutung --
handelt es sich nicht um einen inferentiellen Freifahrtschein (auf die
Glaubwürdigkeit), sondern um eine Bestimmung, wo die Beweislast zu Beginn liegt.
Der Vorteil einer Präsumtion gegenüber dem Verweis auf angeborene Grundsätze
liegt auf der Hand: Es handelt sich um einen Grundsatz, der uns sagt, was zu
glauben \emph{vernünftig} ist und dessen vernünftige Anwendung wir selbst
kontrollieren können. Eine Präsumtionstheorie passt also zu einer
internalistischen Position.


Explizit von einer \enquote{Präsumtion} zugunsten der Information oder des Informanten spricht bereits
\authorfullcite{Crusius:Anweisungvernuenftigzuleben1744} im \titel{Weg zur
Gewißheit und Zuverläßigkeit der menschlichen Erkenntniß}, einem Werk, welches
bereits einige Jahre vor \name[Thomas]{Reid}s einschlägigen Schriften publiziert
wurde.\footnote{\phantomsection\label{Anmerkung:KantundCrusiusPraesuppositionsTheorie}Vgl.
zu den folgenden Ausführungen die Überlegungen
\authorcite{Crusius:Anweisungvernuenftigzuleben1744}' zu testimonialem Wissen in
\cite[][\S\S~605--627]{Crusius:WegzurGewissheitundZuverlaessigkeitdermenschlichenErkenntniss1965}.
Ob \name[Immanuel]{Kant} diese Überlegungen bekannt waren, lässt sich nicht
zweifelsfrei ermitteln. Einerseits schien er selbst nicht über ein Exemplar
dieses Werkes von \authorcite{Crusius:Anweisungvernuenftigzuleben1744} zu
verfügen \mkbibparens{Es findet sich zumindest kein entsprechender Eintrag in
\cite{Warda:ImmanuelKantsBuecher1922}.}, andererseits findet sich in seinen
Anmerkungen in \authorfullcite{Meier:Vernunftlehre1752}s \titel{Auszug aus der
Vernunftlehre} eine Passage, die vermuten lässt, dass er
\authorcite{Crusius:Anweisungvernuenftigzuleben1744}' Überlegungen
kannte \mkbibparens{\cite[vgl.][\nopp 2589]{Kant:Reflexionen1900ff.};
\cite[][XVI: 430.4--431.8]{Kant:GesammelteWerke1900ff.}. \name[Immanuel]{Kant}
spricht dort im Kontext \enquote{historische Wahrscheinlichkeit} von der Präsumtion, dass
sich Unwahrheiten selbst verraten, und von der Bewertung der inneren Möglichkeit
und Wahrscheinlichkeit des Berichteten. Wir werden gerade hierin die zentralen
Überlegungen von \authorcite{Crusius:Anweisungvernuenftigzuleben1744}
ausmachen.}.} \Revision{Crusius ist gewiss nicht der erste Autor,
der von Präsumtionen spricht. Sein Lehrer
\authorfullcite{Hoffmann:Vernunft-Lehre1737} thematisiert die
\enquote{Präsumtions-Wahrscheinlichkeit} sehr ausführlich in seiner
\titel{Vernunft-Lehre} von 1737.\footnote{\Revision{\cite[Vgl.][1124--1144]{Hoffmann:Vernunft-Lehre1737}.
Dabei verweist \authorcite{Hoffmann:Vernunft-Lehre1737} darauf, dass bereits vor ihm
Autoren auf Präsumtionen verwiesen haben
\parencite[vgl.][Vorrede, unpaginiert]{Hoffmann:Vernunft-Lehre1737}.}}}
\authorcite{Crusius:Anweisungvernuenftigzuleben1744}' zentrale Aussage zu
testimonialem Wissen lautet: \enquote{Ein jedwedes Zeugniß hat eine historische
Beweiskraft, vermöge des Grundes, daß man die Wahrhaftigkeit bey einer Erzehlung
als den natürlichen Zustand präsumiren
müsse}\footnote{\Cite[][\S~617]{Crusius:WegzurGewissheitundZuverlaessigkeitdermenschlichenErkenntniss1965}.}.
Da \authorcite{Crusius:Anweisungvernuenftigzuleben1744} auch andere Präsumtionen
kennt, können wir der Eindeutigkeit halber hier mit
\authorcite{Crusius:Anweisungvernuenftigzuleben1744} selbst von einer
\enquote{historischen Präsumtion}\footnote{\Cite[][\S~617]{Crusius:WegzurGewissheitundZuverlaessigkeitdermenschlichenErkenntniss1965}.}
oder von der \emph{Bonitätspräsumtion} sprechen.
Diese besagt also, dass wir im Falle einer Information so lange davon auszugehen
haben, dass ihr tatsächlich Wissen und Aufrichtigkeit auf der Seite des
Informanten zugrunde liegt, wie uns keine Gründe bekannt sind, die die Präsumtion außer Kraft
setzen.\footnote{\cite[Vgl.][\S~606]{Crusius:WegzurGewissheitundZuverlaessigkeitdermenschlichenErkenntniss1965}:
\enquote{Ohne besondere Ursachen, welche die Präsumtion entkräften, oder das
Gegen\-theil erweisen, präsumiret man nicht, daß eine Erzehlung ohne Grund
sey.} Ähnlich schreibt \name[Immanuel]{Kant}: \enquote{Man hält eine Erzehlung
ohne Grund, wenn die Art, wie der Erzehlende es hat verstehn könen, schwer
einzusehen ist. Man hält sie aber darin nicht vor ungegründet, wenn man keine
Ursachen weiß} \mkbibparens{\cite[][\nopp 2589]{Kant:Reflexionen1900ff.};
\cite[][XVI: 430.7--10]{Kant:GesammelteWerke1900ff.}}.} Zwar sagt
\authorcite{Crusius:Anweisungvernuenftigzuleben1744}, dass bei testimonialem Wissen nur Wahrscheinlichkeit zu erreichen sei, die er \enquote{historische Wahrscheinlichkeit} nennt und die von
Demonstrationen zu unterscheiden sei; denn \emph{beweisen} lasse sich nicht, ob
es sich bei einer Information um eine in Wissen des Informanten fundierte
Nachricht handelt oder um eine vorsätzlich oder selbst fehlerhaft angenommene
Erdichtung.\footnote{\cite[Vgl.][\S~605]{Crusius:WegzurGewissheitundZuverlaessigkeitdermenschlichenErkenntniss1965}.}
Aber ein Blick auf alltägliche Erfahrungen\footnote{Bei diesen alltäglichen
Erfahrungen handelt es sich gewiss nicht um solche im Sinne \name[David]{Hume}s. Wir
machen keine alltäglichen Erfahrungen, auf deren Grundlage wir erst auf die
Bonität von Informationen und Informanten schließen. Vielmehr geht es um die
alltägliche Erfahrung, dass Informationen diese Bonität ohne weitere auf den
Einzelfall abgestimmte Legitimation bereits von uns zugesprochen bekommen.}
zeige bereits, dass es sich doch um \enquote{moralische Gewißheit} handeln
könne\footnote{\cite[Vgl.][\S~605]{Crusius:WegzurGewissheitundZuverlaessigkeitdermenschlichenErkenntniss1965}:
\enquote{Daß eine historische Wahrscheinlichkeit sey, und dieselbe sich auch in
unzehligen Fällen gar leicht in eine moralische Gewißheit verwandele, lehret so
gleich die innerliche Empfindung in den Exempeln, welche uns in dem menschlichen
Leben vorkommen.}}, was dem Status entspricht, der unserer empirischen
Erkenntnis generell zukommt.\footnote{Nach \authorcite{Meier:Vernunftlehre1752}
ist die \singlequote{moralische Gewissheit} charakterisiert durch einen
\enquote{Grad der Wahrscheinlichkeit, welcher in unserm regelmässigen
Verhalten so gut ist, als eine ausführliche Gewissheit}
\mkbibparens{\cite[][\pno~48\,f.,]{Meier:AuszugausderVernunftlehre1752}
\cite[][XVI: 432.23--24]{Kant:GesammelteWerke1900ff.}}.} Dass testimoniales
Wissen nicht wie mathematisches Wissen in Beweisen fundiert ist, tut seiner Validität keinen Abbruch.


Nun ergibt sich die Notwendigkeit, eine begründete Abgrenzung
zwischen Präsumtionen und Vorurteilen (sowie
\singlequote{Hypothesen}\footnote{Ausführlich setzt sich damit Christian
\authorcite{Wolff:Psychologiaempirica1968} auseinander; \cite[vgl.][\S\S~125--129]{Wolff:Discursuspraeliminarisdephilosophiaingenere1996}.}) zu
erarbeiten. Viele Philosophen in der Zeit der Aufklärung waren sich der Tatsache
bewusst, dass unser Denken auf grundlegenden Annahmen beruht. Sie erkannten aber auch, dass es kein
Ausweg ist, dem Begriff \enquote{Vorurteil} seine pejorative Konnotation zu
streiten und Vorurteile \emph{toto genere} zu akzeptieren. Vielmehr gilt es,
legitime Grundannahmen von (illegitimen) Vorurteilen (auch sprachlich) zu
unterscheiden. Denn während Vorurteile generell illegitime Urteile darstellen,
sollen Präsumtionen und vorläufige Urteile zumindest bei korrekter oder
\singlequote{kritischer} Anwendung legitim sein. Der Begriff der
Präsumtion ist also aus Sicht der Aufklärung schon wegen der Notwendigkeit
einschlägig, legitime von illegitimen vorläufig angenommenen
Grundsätzen (Präsumtionen von Vorurteilen) zu
unterscheiden.\footnote{\cite[Vgl.][44]{Scholz:VerstehenundRationalitaet1999}:
\enquote{Präsumtionen als vernünftige vorläufige Unterstellungen sind
strikt von den irrationalen Vorurteilen zu unterscheiden, die die Aufklärer
zurecht bekämpft haben}.}

Aus \authorcite{Crusius:Anweisungvernuenftigzuleben1744}' Sicht handelt es sich
im Falle der Bonitätspräsumtion nicht um ein Vorurteil, sondern um eine Präsumtion,
weil sie zwei Eigenschaften vereint:
\begin{nummerierung}
\item \emph{Die Bonitätspräsumtion ist als solche vernünftig.} Es wäre hingegen
gerade unvernünftig, einem Informanten zu misstrauen, wenn sich kein besonderer
Grund zu diesem Misstrauen findet. Als Gründe für die Vernünftigkeit der
Bonitätspräsumtion nennt \authorcite{Crusius:Anweisungvernuenftigzuleben1744}
die Tatsachen, dass Menschen generell ein natürliches Streben nach Wahrheit
haben und dass \enquote{zumahl in einer mercklichen Zusammensetzung, sich
Erdichtung, Unwahrheit und Affectation selbst verräth}\footnote{\cite[][\S~606]{Crusius:WegzurGewissheitundZuverlaessigkeitdermenschlichenErkenntniss1965}.}.
Bedeutsam sind sicherlich auch die Konsequenzen, die sich aus einem Verzicht auf
genuin testimoniales Wissen ergäben.

\item \emph{Die Anwendung der Bonitätspräsumtion ist im Einzelfall
kritisierbar.} Auch \authorcite{Crusius:Anweisungvernuenftigzuleben1744} kennt
bereits, was in der heutigen Diskussion \enquote{\emph{defeater}} genannt wird:
\enquote{daß ein Zeugniß schon seine Glaubwürdigkeit verliere, wenn sich der
Grund der allgemeinen historischen Präsumtion darauf nicht
schickt}\footnote{\Cite[][\S~617]{Crusius:WegzurGewissheitundZuverlaessigkeitdermenschlichenErkenntniss1965}.}.
Die Bonitätspräsumtion stellt ein vorläufiges Urteil dar, aber nicht in dem
Sinne, dass sich die Präsumtion selbst jederzeit als falsch herausstellen
könnte, sondern in dem Sinne, dass jederzeit zur Disposition steht, ob sie in
einem vorliegenden Einzelfall greift. Eine Zurückweisung der Anwendung auf einen
Einzelfall lässt die Geltung der Präsumtion selbst unangetastet. Die Anwendung
der Präsumtion zugunsten einer bestimmten Information könne daher
\enquote{entweder entkräftet, oder gar durch Gegengründe ausdrücklich widerleget
werden}\footnote{\Cite[][\S~606]{Crusius:WegzurGewissheitundZuverlaessigkeitdermenschlichenErkenntniss1965}.}.
Ein mündiger Umgang mit testimonialem Wissen zeigt sich somit in der korrekten
\emph{Anwendung} der Bonitätspräsumtion auf vorliegende Fälle.
\enquote{\ori{Keine Präsumtion gilt weiter, als wieferne sich auch ihr
Beweisgrund in dem vorhandenen Exempel appliciren lässet}}\footnote{\Cite[][\S~407]{Crusius:WegzurGewissheitundZuverlaessigkeitdermenschlichenErkenntniss1965}.}.
Das heißt hier, dass die Präsumtion der Glaubwürdigkeit des Informanten dann
nicht anzuwenden ist, wenn es Gründe gibt, den natürlichen Zustand der
Wahrhaftigkeit nicht vorauszusetzen. Fälle, in denen sie nicht zur Anwendung
kommen sollte, werden durch -- wie man heute sagt -- \emph{defeaters} bestimmt.
Solche \emph{defeaters} beweisen (oder begründen) nicht das Gegenteil, aber sie
unterminieren die Rechtfertigung einer Behauptung. Dadurch kann der an sich
bereits vernünftige Erwerb testimonialen Wissens zugleich ein kritischer und der
Empfänger einer Information ein mündiger Rezipient werden.
\end{nummerierung}


Um zu erkennen, welcher Art die \emph{defeater} sind, die zu berücksichtigen
sind, müssen wir uns zunächst fragen, welche Dinge überhaupt als
Informationsquelle genutzt werden können. Es erstaunt dabei die Ausführlichkeit,
mit der er über naheliegende Quellen wie mündliche und schriftliche Mitteilungen
hinausgeht:
\begin{quote}
  Die Data zu einer historischen Wahrscheinlichkeit können sehr vielerley seyn.
  Es gehören dazu nicht nur Geschichtbücher und Urkunden, ausführliche und
  beyläufige Zeugnisse und Erzehlungen, sondern auch Denckmale, Müntzen, Bilder,
  Aufschriften, Ueberbleibsale und hinterlassene Folgen, ja auch die innerliche
  Beschaffenheit der erzehlten Sache gehöret mit darzu, ob sie nemlich leicht
  möglich oder an sich vermuthlich ist, oder
  nicht.\footnote{\Cite[][\S~608]{Crusius:WegzurGewissheitundZuverlaessigkeitdermenschlichenErkenntniss1965}.}
\end{quote}
Dies sind die Ausgangspunkte, die zunächst als solche zu würdigen sind. Es ließe
sich hier einwenden, dass die Liste zu weit greift, indem sie Überbleibsel,
Folgen und die innere Beschaffenheit der Sache selbst mit einbezieht.
\authorcite{Crusius:Anweisungvernuenftigzuleben1744} scheint damit den Bereich testimonialen Wissens zu verlassen,
denn wer Fossilien findet und daraus Rückschlüsse auf vergangene Geschehnisse
zieht, der macht etwas ganz anderes als derjenige, der eine Mitteilung erhält
und testimoniales Wissen erwirbt. Dies war die Lehre, die im letzten Kapitel aus
\authorcite{Reid:EssaysontheIntellectualPowersofMan2002}s Unterscheidung von
\emph{solitary} und \emph{social operations of mind} gezogen werden
konnte. Wenn wir auch manchmal bei Fossilien und ähnlichen
Hinweisen auf frühere Geschehnisse von \singlequote{Zeugnissen} (früherer Zeiten) sprechen, so ist dies doch mit der Grundlage testimonialen Wissens nicht zu verwechseln.

Man beachte jedoch, wie \authorcite{Crusius:Anweisungvernuenftigzuleben1744}
diese \enquote{Data} in seiner Darstellung behandelt. Sie werden gerade nicht
einfach mit Informationen aus zweiter Hand und Erzählungen gleichgestellt, sondern in einer
Weise in die Überlegungen einbezogen, die an einen anderen Aspekt der
Darstellung \authorcite{Hume:ATreatiseofHumenNature2007}s erinnert:
\begin{quote}
  Was die Beweiskraft der \ori{Datorum} bey historischen Beweisen anlanget, so
  ist dieselbe theils in der Existenz und Beschaffenheit der Zeugnisse, d.\,i.
  der bezeugten Erfahrung anderer; theils in der Beschaffenheit der Sache selbst,
  die erzehlet wird,
  anzutreffen.\footnote{\Cite[][\S~609]{Crusius:WegzurGewissheitundZuverlaessigkeitdermenschlichenErkenntniss1965}.
  Siehe auch \name[Immanuel]{Kant}s Überlegungen in
  \cite[][\nopp 2589]{Kant:Reflexionen1900ff.},
  \cite[][XVI: 430.14--15]{Kant:GesammelteWerke1900ff.}: \enquote{Also aus der
  Beschaffenheit der Sache und der Beschaffenheit der Zeugen (vielheit, deren einer es nicht von den Andern hat).} }
\end{quote}
Ebenso wie \authorcite{Hume:ATreatiseofHumenNature2007} achtet
\authorcite{Crusius:Anweisungvernuenftigzuleben1744} nicht nur auf den
Informanten und dessen Kompetenz und Glaubwürdigkeit, sondern ebenso darauf, ob
es sich ihrem Inhalt nach um eine glaubwürdige Erzählung handelt. Hier trennt
\authorcite{Crusius:Anweisungvernuenftigzuleben1744} diejenigen
Ausgangspunkte, die genuin testimoniales Wissen begründen, von sonstigen
Ausgangspunkten. Welche Bedeutung letzteren zukommt, wird an folgender Stelle
deutlich:
\begin{quote}
  Man giebt dabey theils auf die Möglichkeit; theils auf die Wahrscheinlichkeit
  Achtung, welche der Sache auch schon zukommt, wiefern man auf die Zeugnisse,
  wodurch sie weiter bestätiget werden soll, noch nicht Acht
  hat.\footnote{\Cite[][\S~609]{Crusius:WegzurGewissheitundZuverlaessigkeitdermenschlichenErkenntniss1965}.}
\end{quote}
Es ist im Grunde nie der Fall, dass die Mitteilung durch andere die einzige
Informationsquelle ist, die uns zur Verfügung steht. Oft hat ein
Ereignis, das uns berichtet wird, weitere Spuren hinterlassen, wie etwa die
Hinterlassenschaften einer bedeutenden Schlacht. Fast immer können wir
einschätzen, wie wahrscheinlich es überhaupt ist, dass etwas so geschah, wie es
uns jemand berichtet. Und selbst wo wir über keine weiteren Informationen
verfügen, kann dies aussagekräftig sein; beispielsweise wenn uns ein
historisches Ereignis berichtet wird, welches -- seltsamer Weise -- keinerlei
Spuren hinterlassen hat. Solche Merkwürdigkeiten sprechen gegen die Annahme, es
habe tatsächlich stattgefunden. Vor allem aber können wir unabhängig von der
Mitteilung einschätzen, ob das, was uns berichtet wird, in sich schlüssig und
allgemein möglich oder sogar wahrscheinlich ist, oder ob die Erzählung etwas
sehr Außergewöhnliches\footnote{Hierzu auch \cite[][\nopp
2589]{Kant:Reflexionen1900ff.}, \cite[][XVI:
430.18]{Kant:GesammelteWerke1900ff.}:
\enquote{Seltene Dinge haben eine unwarscheinlichkeit. e. g. Große Armeen.}},
Unwahrscheinliches oder gar Unmögliches und in sich widersprüchliches behauptet.
Möglicherweise widerspricht sich unser Informant oder er behauptet etwas, was
bekannten Naturgesetzen widerspricht.

Dies war die Pointe der Argumentation \name[David]{Hume}s gegen die
Glaubwürdigkeit von Wunderberichten: Bei jeder Information gilt es nicht nur,
die Informanten zu bewerten, sondern auch den Informationsgehalt anhand
desjenigen Wissens, über welches wir bereits unabhängig von diesen bestimmten
Informanten verfügen.\footnote{Dabei verfügen wir über dieses Wissen nicht zwingend
unabhängig von Informanten überhaupt. Wenn wir eine Information bspw. mit
unserem allgemeinen physikalischen Wissen konfrontieren, dann handelt es sich
dabei in der Regel um Wissen, welches wir selbst aus zweiter Hand haben.} Zwar
mögen wir über Einzelfälle als solche kein vorgängiges Wissen haben, jedoch
liegt uns immer allgemeines Wissen über entsprechende Geschehnisse vor. Um
\name[David]{Hume}s Beispiel zu verwenden: Wenn uns berichtet wird, jemand sei
erst tot gewesen, später jedoch wieder erwacht, dann benötigen wir kein
testimoniales Wissen mit gegenteiligem Inhalt. Wir können unseren Informanten
bereits aufgrund unseres allgemeinen Wissens über die Irreduzibilität des Todes
der Lüge oder des Irrtums bezichtigen. Einen solchen
Umstand können wir insofern zu den \emph{defeaters} zählen, als durch ihn die
Bonitätspräsumtion hinfällig wird -- er hebt die Evidenz auf, die der
Mitteilung als solcher zukommt. Es reicht nicht, um das Gegenteil der bezeugten
Sache zu erweisen, aber es ist doch genug, um Zweifel an der Glaubwürdigkeit der
Information zu erregen. Und genau diese Funktion können wir denjenigen
\enquote{Data} zusprechen, die selbst nicht geeignet sind, testimoniales Wissen
zu begründen, weil sie nicht von der Art einer Mitteilung sind.

Dennoch stellt
\authorcite{Crusius:WegzurGewissheitundZuverlaessigkeitdermenschlichenErkenntniss1965}'
Theorie einen wichtigen Fortschritt gegenüber
\authorcite{Hume:AnEnquiryConcerningHumanUnderstanding1964}s Position dar: Der
Begriff der Präsumtion erlaubt es, mithilfe der \singlequote{defeater} kritische
Momente zu integrieren, ohne annehmen zu müssen, testimoniales Wissen habe die
gleiche Grundlage wie unsere sonstigen Erfahrungserkenntnisse. Ebenso wie
\authorcite{Reid:EssaysontheIntellectualPowersofMan2002} vertritt er eine
credulistische Position, der zufolge wir keine zusätzlichen
rechtfertigungsbedürftigen Prämissen benötigen, um von einer Mitteilung auf die
Wahrheit des Mitgeteilten zu schließen. Aber während
\authorcite{Reid:EssaysontheIntellectualPowersofMan2002} von einem angeborenen
Prinzip spricht, nach dem wir \emph{de facto} handeln, und damit die Frage nach
der Vernünftigkeit außer Acht lässt, geht es
\authorcite{Crusius:WegzurGewissheitundZuverlaessigkeitdermenschlichenErkenntniss1965}
gerade darum, unseren Umgang mit testimonialem Wissen als prinzipiengeleitet und
vernünftig auszuweisen. Freilich kann auch er nicht beweisen, dass allgemein
gilt: Wenn Personen mit der Eigenschaft $XYZ$ uns mitteilen, dass $p$, dann ist
$p$ wahr. Aber er kann erstens begründen, dass es vernünftig ist, zunächst von
der Wahrheit einer Information auszugehen, und zeigt zweitens, dass ein solches
Vorgehen nicht leichtgläubig sein muss.

\section{Zusammenfassung und Ausblick}
Ich habe in diesem Kapitel zwei Arten von Theorien testimonialen Wissens
unterschieden und dazu paradigmatische Darlegungen aus der Philosophie der
Neuzeit skizziert. Individualistische Ansätze behaupten, dass Mitteilungen von
anderen nicht als \emph{primary epistemic link} verstanden werden dürfen, der
uns ohne Umschweife mit Wissen versorgt. Wir können entweder die Möglichkeit
testimonialen Wissens \emph{toto genere} zurückweisen und einen testimonialen
Skeptizismus vertreten, oder aber im Sinne eines testimonialen Reduktionismus
behaupten, dass solche Mitteilungen einen \emph{secondary epistemic link}
darstellen, also nur unter Rekurs auf weitere, begründungsbedürftige Prämissen
Wisen generieren können. \authorfullcite{Descartes:OeuvresdeDescartes1983} und
David \name[David]{Hume} waren Vertreter eines solchen Individualismus.
Nichtindividualistische Ansätze wiederum sehen Mitteilungen als \emph{primary
epistemic link} an, der uns direkt mit Wissen versorgt, ohne dass es dafür einer
weiteren Begründung bedürfte. Der klassische Autor ist
\authorfullcite{Reid:EssaysontheIntellectualPowersofMan2002}; aber mit
\authorfullcite{Crusius:WegzurGewissheitundZuverlaessigkeitdermenschlichenErkenntniss1965}
und seiner Präsumtionstheorie vertritt  auch ein Vertreter der deutschsprachigen
Aufklärung einen solchen Ansatz. Demnach können wir Mitteilungen anderer ohne
Umschweife als Grundlage eigenen Wissenserwerbs ansehen, solange keine Gründe
vorliegen, diese in Zweifel zu ziehen. Im folgenden
\ref{chapter:MuendigerErwerbTestimonialenWissens}. Kapitel soll nun genauer
untersucht werden, was einen solchen Wissenserwerb aus der Sicht von Vertretern
der Aufklärung im Deutschland des 18. Jahrhunderts zu einem
\singlequote{kritischen} Umgang mit testimonialem Wissen macht.


\chapter{Zwei Arten der Bewertung testimonialen
Wissens}\label{chapter:MuendigerErwerbTestimonialenWissens}

Wir müssen die Möglichkeit testimonialen Wissens anerkennen und einräumen, dass
weder ein testimonialer Reduktionismus noch ein testimonialer Skeptizismus tragfähig sind.
Es liegt aber auch auf der Hand, dass Selbstdenken im Gegensatz zu
Leichtgläubigkeit steht und es möglich sein muss zu
erläutern, was einen leichtfertigen Umgang mit testimonialem Wissen von einem
kritischen Umgang unterscheidet.

So verwundert nicht, dass sich in der deutschen Aufklärungsphilosophie
Auseinandersetzungen mit der Frage finden, wie Informationen zu bewerten sind.
Zunächst fallen dabei diejenigen Ansätze auf, die die Informationsquelle
\emph{respective} den Prozess der Informationsübermittlung fokussieren. Es
bietet sich an, dabei zunächst an
\authorcite{Crusius:WegzurGewissheitundZuverlaessigkeitdermenschlichenErkenntniss1965}
und seine Theorie einer \singlequote{historischen Präsumtion} anzuschließen.
Immerhin spricht auch \name[Immanuel]{Kant} von einer Präsumtion, dass sich
Unwahrheiten selbst verraten.\footnote{\cite[Vgl.][\nopp
2589]{Kant:Reflexionen1900ff.}, \cite[][XVI: 430.6]{Kant:GesammelteWerke1900ff.}.} Neben
\authorcite{Crusius:WegzurGewissheitundZuverlaessigkeitdermenschlichenErkenntniss1965}
werde ich vor allem auf \authorfullcite{Meier:AuszugausderVernunftlehre1752} und
\authorfullcite{Reimarus:DieVernunftlehrealseineAnweisungzumrichtigenGebrauchderVernunftinderErkenntnisderWahrheit1756}
eingehen (Kapitel \ref{subsection:BewertungvonInformationenanhandihrerQuellen}), um schließlich zu
zeigen, dass \name[Immanuel]{Kant} entsprechende Ideen zwar thematisiert, sie
aber letztlich nicht hinreichen um zu beschreiben, wie ein mündiger Umgang mit
Informationen aus zweiter Hand beschaffen ist. Aussichtsreicher ist hingegen die
Bewertung der Informationen anhand der \emph{Art} der mitgeteilten Erkenntnis.
Entsprechende Überlegungen knüpfen direkt an
\authorcite{Descartes:OeuvresdeDescartes1983}' drittes Argument gegen die
Büchergelehrsamkeit an, indem sie explizieren, worin der Unterschied zwischen
(bloßer) historischer Kenntnis  und Wissenschaft besteht. Diese Herangehensweise untersuche
ich zunächst bei \authorfullcite{Wolff:Discursuspraeliminarisdephilosophiaingenere1996} (Kapitel
\ref{subsection:BewertungvonInformationennachihrerART}). Sie wird
sich später als Ausgangspunkt der Überlegungen \name[Immanuel]{Kant}s erweisen
(Kapitel \ref{Chapter:KantsSocialEpistemology} und darin insbesondere Kapitel
\ref{section:MuendigkeitundPhilosophie}).

\section[Bewertung anhand
der Quellen]{Bewertung von Informationen anhand
der Quellen}\label{subsection:BewertungvonInformationenanhandihrerQuellen}
\subsection{Kompetenz und
Aufrichtigkeit}\label{subsubsection:GeorgFriedrichMeier}
\authorcite{Crusius:Anweisungvernuenftigzuleben1744} gibt ausführliche Hinweise
dazu, wie testimoniale Erkenntnisse zu evaluieren sind.\footnote{In den {\S\S}
615--626 seiner Schrift \titel{Weg zur Gewißheit und Zuverläßigkeit der
menschlichen Erkenntniss} nennt er 22 konkrete Regeln im Umgang mit
Informationen aus zweiter Hand. \authorcite{Crusius:Anweisungvernuenftigzuleben1744} nummeriert die Regeln von 1 bis 21, jedoch ist die Nummer 17 doppelt vergeben;
\cite[vgl.][\S\S~623\,f.]{Crusius:WegzurGewissheitundZuverlaessigkeitdermenschlichenErkenntniss1965}.}
Sie beruhen auf allgemeineren Überlegungen zur Bewertung von Informationen, die
er diesen Regeln vorausschickt, und die für meine Belange interessanter sind, als
die konkreten Regelausformulierungen. Die Betonung der Seite des Gegenstandes
der Information, welche \authorcite{Crusius:Anweisungvernuenftigzuleben1744}
inhaltlich mit \name[David]{Hume} verbindet, stellt  eine dieser Überlegungen
dar. Relevant bei der Bewertung der
Bonität von Informanten und Informationen sind dann auf der Seite des
Informanten zunächst dessen Kompetenz und Aufrichtigkeit.
\authorcite{Crusius:Anweisungvernuenftigzuleben1744} schreibt:
\begin{quote}
  Bey den Zeugnissen können vier Umstände in Betrachtung kommen: ob
  einer ein wahres Zeugniß hat ablegen können; ob er es auch hat ablegen wollen;
  ob und wiefern er hat betrügen können; und wiefern sein Zeugniß mit andern
  Zeugnissen übereinstimmet.\footnote{\Cite[][\S~611]{Crusius:WegzurGewissheitundZuverlaessigkeitdermenschlichenErkenntniss1965}.}
\end{quote}
Der Bonitätspräsumtion folgend dürfen wir immer zunächst davon ausgehen, dass
unser Informant \enquote{ein wahres Zeugniß hat ablegen können} -- dass er etwa
\emph{kompetent} genug ist -- und dass \enquote{er es auch hat ablegen wollen} -- dass
er \emph{aufrichtig} ist.

Überlegungen zur Bonität von Informanten, wie wir sie nun als einen Aspekt der
Bewertung von Informationen bei
\authorcite{Crusius:Anweisungvernuenftigzuleben1744} fanden, lassen sich auch
bei \name[Immanuel]{Kant} belegen\footnote{Oliver
\textcite[][839\,f.]{Scholz:AutonomieangesichtsepistemischerAbhaengigkeiten2001}
sieht hierin den Schlüssel zu einem Verständnis, was Mündigkeit bei
testimonialem Wissen nach \name[Immanuel]{Kant} sein kann. Auch Thomas
\textcite[][143--146]{Kater:PolitikRechtGeschichte1999} betont diese
Überlegungen.}; Thomas \name[Thomas]{Kater} führt beispielsweise vier Stellen
aus Mitschriften zu \name[Immanuel]{Kant}s Logikvorlesungen
an.\footnote{\cite[Vgl.][143--146]{Kater:PolitikRechtGeschichte1999}. Die
einschlägigen Passagen finden sich in \cite[][]{Kant:LogikPoelitz1966}, \cite[][XXIV:
242.4--250.9]{Kant:GesammelteWerke1900ff.},
\cite[][]{Kant:LogikPhilippi1966}, \cite[][XXIV:
448.20--450.35]{Kant:GesammelteWerke1900ff.}, \cite{Kant:LogikPoelitz1966},
\cite[][XXIV: 561.34--563.25]{Kant:GesammelteWerke1900ff.}, und
\cite{Kant:WienerLogik1966}, \cite[][XXIV: 893.24--900.32]{Kant:GesammelteWerke1900ff.}.} In diesen
Vorlesungen geht auch \name[Immanuel]{Kant} von einer \emph{prima
facie} anzunehmenden Bonität von Informanten und Informationen aus und verweist
auf Kompetenz und Aufrichtigkeit als entscheidende Kriterien bei der Bewertung von Informanten.
Soweit stimmen seine Auskünfte mit denen von \authorcite{Crusius:Anweisungvernuenftigzuleben1744} überein. Doch der
Ausgangspunkt seiner Überlegungen ist ein anderer: Der Textaufbau
sowie die übernommene Paragrapheneinteilungen weisen die Überlegungen
\name[Immanuel]{Kant}s zunächst als Kommentierungen der \S\S~206--214 des
\titel{Auszugs aus der Vernunftlehre} von Georg
\authorcite{Meier:Vernunftlehre1752} aus, der darin selbst wiederum zumindest
der Tendenz nach der Position \authorcite{Wolff:Psychologiaempirica1968}s in den
\titel{Vernünftige[n] Gedanken von den Kräften des menschlichen Verstandes und
ihrem richtigen Gebrauche in Erkenntnis der Wahrheit} folgt\footnote{Zu
\authorcite{Wolff:Psychologiaempirica1968}s Position zum Umgang mit testimonialem Wissen siehe
\cite[][200--203]{Wolff:VernuenftigeGedankenvondenKraeftendesmenschlichenVerstandesundihremrichtigenGebraucheinErkenntnisderWahrheit1978},
sowie
\cite[][\S\S~149--161]{Wolff:Cogitationesrationalesdeviribusintellectushumani1983}.}.
\authorcite{Meier:Vernunftlehre1752}s Position zur Mündigkeit im Umgang mit
testimonialen Erkenntnissen (der \enquote{vernünftige Glaube}) liest sich wie
folgt:
\begin{quote}
  \ori{Ein Zeuge ist glaubwürdig} (testis fide dignus), wenn man auf eine
  gelehrte Art wenigstens wahrscheinlich erkennen kann, dass er genugsames
  Ansehen habe; das Zeugnis eines solchen Zeugen ist \ori{ein glaubwürdiges
  Zeugniss} (testimonium fide dignum). \ori{Der vernünftige} oder \ori{sehende
  Glaube} (fides oculata, rationalis) ist die Fertigkeit nur glaubwürdigen
  Zeugen zu
  glauben.\footnote{\cite[][\S~214]{Meier:AuszugausderVernunftlehre1752},
  \cite[][XVI: 509.20--24]{Kant:GesammelteWerke1900ff.}.}
\end{quote}
Einen unvernünftigen Glauben nennt \authorcite{Meier:Vernunftlehre1752} wie \authorcite{Wolff:Psychologiaempirica1968}
\enquote{Leichtgläubigkeit}.\footnote{\cite[Vgl.][\S~213]{Meier:AuszugausderVernunftlehre1752},
\cite[][XVI: 508.26--27]{Kant:GesammelteWerke1900ff.};
\textcite[][201]{Wolff:VernuenftigeGedankenvondenKraeftendesmenschlichenVerstandesundihremrichtigenGebraucheinErkenntnisderWahrheit1978}
schreibt ähnlich wie \authorcite{Meier:Vernunftlehre1752}: \enquote{Jedoch, damit wir nicht
leichtgläubig sind, und uns betrügen lassen; so müssen wir uns zweyer Dinge erst versichern: 1)
daß derjenige, welcher etwas zeuget, die Sache recht habe erkennen können,
damit er sich nicht selbst betrogen: 2) daß er die Sache so erzehlet, wie er
sie erkandt hat, damit er nicht den Vorsatz habe, andere zu betrügen. Mit
einem Worte, ich muß versichert seyn, daß der Zeuge klug und aufrichtig genug
sey.} Was bei \authorcite{Wolff:Psychologiaempirica1968} gegenüber \authorcite{Meier:Vernunftlehre1752} fehlt, ist lediglich der
Verweis auf die sprachliche Kompetenz, das korrekt Erkannte auch verständlich
auszudrücken.} Vernünftig ist es, die Zustimmung zu einer Aussage, die
Gegenstand einer Mitteilung ist, von dem \enquote{Ansehen} des Informanten
abhängig zu machen. Das Ansehen wiederum bestehe -- ähnlich wie bei
\authorcite{Crusius:Anweisungvernuenftigzuleben1744} -- in der Kompetenz oder
\enquote{Tüchtigkeit} (\authorcite{Wolff:Psychologiaempirica1968} spricht stattdessen von
\enquote{Klugheit}\footnote{\cite[Vgl.][201]{Wolff:VernuenftigeGedankenvondenKraeftendesmenschlichenVerstandesundihremrichtigenGebraucheinErkenntnisderWahrheit1978}.})
und der \enquote{Aufrichtigkeit} des Informanten, also darin, dass er zum
einen uns nicht absichtlich irre führt (Aufrichtigkeit), und zum anderen selbst
die nötigen intellektuellen und epistemischen Voraussetzungen mitbringt, um das,
worüber er Auskunft gibt, auch zuverlässig beurteilen zu
können.\footnote{\phantomsection\label{Fussnote:MeierzurBonitaetdesZeugen}\cite[Vgl.][\S~207]{Meier:AuszugausderVernunftlehre1752},
\cite[][XVI: 504.21--30]{Kant:GesammelteWerke1900ff.}} Die
\enquote{Tüchtigkeit} oder epistemischen Voraussetzungen eines Informanten
bestimmt \authorcite{Meier:Vernunftlehre1752} des weiteren wie folgt:
\begin{quote}
  1) er muss bei der Sache gegenwärtig sein, die er bezeugt; 2) er muss im Stande sein, eine richtige
Erfahrung zu bekommen; 3) er muss ein gutes und treues Gedächtnis haben, oder
seine Erfahrung alsobald aufschreiben; 4) er muss die Gabe besitzen, seine
eigene Erkenntniss auf eine richtige und hinlängliche Art zu
bezeichnen.\footnote{\cite[][\S~209]{Meier:AuszugausderVernunftlehre1752},
\cite[][XVI: 505.24--29]{Kant:GesammelteWerke1900ff.}.}
\end{quote}
Von der Kompetenz des Informanten ist also die notwendige Erfahrungssituation
nochmal zu unterscheiden, was besonders bei Zeugen im alltäglichen Sinn
dieses Wortes ersichtlich ist. Ein guter Zeuge ist, wer sich in einer
bestimmten Situation befand, um eine Erfahrung machen zu können, die andere
nicht gemacht haben (\enquote{er muß bei der Sache gegenwärtig sein}). Dies
betrifft in mehr oder weniger starker Ausprägung alle Informanten. Nur aus
bestimmten Positionen im Verhältnis zu einem Fußballfeld lassen sich
beispielsweise Abseitspositionen erkennen; wer falsch steht, wird in der Regel
zugeben, nicht erkannt zu haben, ob Abseits vorlag. Diese Wahrnehmungsposition
begründet eine epistemische Asymmetrie im Falle von Mitteilungen. Genauso
spielt die Kompetenz des Informanten eine Rolle. Wüsste der
Schiedsrichterassistent nicht, was eine Abseitsposition ist, dann nützte sein
Signal der Schiedsrichterin nicht bei ihrer Entscheidung. Die
Wahrnehmungsposition macht in der Regel die epistemische Asymmetrie im Falle von
Zeugen, die Kompetenz die Asymmetrie im Falle von Experten aus. Beide
Aspekte -- Wahrnehmungsposition und Kompetenz -- werden in der Regel
zusammengefasst zu der Kompetenz oder Fähigkeit, in einem konkreten Fall eine
bestimmte Aussage begründet tätigen zu können, was bei
\authorcite{Meier:Vernunftlehre1752} \enquote{Tüchtigkeit} heißt und der
Aufrichtigkeit gegenübergestellt wird.\footnote{Siehe z.\,B.
\cite[][108]{Goldman:KnowledgeinaSocialWorld1999}: \enquote{There are two types
of skepticism receivers might have about a speaker or her report. First,
receivers might worry about the source's competence to make an accurate
observation or interpretation of the alleged state of affairs, especially in a
technical subject matter. {\punkt} The second category of
possible skepticism concern's the reporter's honesty rather than competence.} An
einer anderen Stelle unterscheidet \authorcite{Goldman:KnowledgeinaSocialWorld1999} dann die genannten
\emph{drei} Momente: \enquote{I propose a {\punkt} classification with three elements: (A)
the reporter’s \ori{competence}, (B) the reporter’s \ori{opportunity}, and (C)
the reporter’s \ori{sincerety} or \ori{honesty}}
(\cite[][123]{Goldman:KnowledgeinaSocialWorld1999}).}
\authorcite{Scholz:DasZeugnisanderer2001} (der sich jedoch nicht auf
\authorcite{Meier:Vernunftlehre1752}, sondern auf \name[Immanuel]{Kant} beruft)
spricht von den Bedingungen der \enquote{\emph{Aufrichtigkeit}} und der
\enquote{\emph{Kompetenz}},\footnote{\cite[Vgl.][361]{Scholz:DasZeugnisanderer2001}:
\enquote{Für das Zeugnis anderer sind die Gültigkeitsbedingungen: (i) dass die bezeugende
Person bei ihrer Äußerung aufrichtig ist, d.\,h., dass sie glaubt, was sie
behauptet (Bedingung der \ori{Aufrichtigkeit}), und (ii) dass sie (bei der
Gelegenheit $O$) kompetent bezüglich des Sachverhalts $p$ ist, um den es in
ihrer Äußerung geht -- und zwar in dem starken Sinne, dass die durch die
Behauptung ausgedrückte Überzeugung wahr ist oder in einem geeigneten schwächeren Sinne (Bedingung der
\ori{Kompetenz}). An den beiden Bedingungen zusammengenommen bemißt [sic] sich
die Glaubwürdigkeit der bezeugenden Person.}} wobei vergessen zu werden droht,
dass zur \singlequote{Tüchtigkeit} neben der Kompetenz auch die richtige
Positionierung im Verhältnis zum Gegenstand des Erkennens gehört. Es ist aber
letztlich unbedeutend, ob wir die richtige Positionierung als Aspekt
\emph{neben} der Kompetenz anführen oder ob wir den Ausdruck \emph{Kompetenz} so
verstehen, dass er diese Positionierung mit enthält. Im folgenden wird dieses
Detail in der Regel keine nennenswerte Funktion übernehmen, sodass ich der
Einfachheit halber von \enquote{Kompetenz} spreche, ohne damit die
Positionierung auszuschließen.

\name[Immanuel]{Kant} folgt \authorcite{Meier:Vernunftlehre1752} in der obersten
Zweiteilung und fasst die Kriterien der Bewertung zu den beiden genannten zusammen: \enquote{Die
Tüchtigkeit des Zeugen besteht darin, daß er hat \ori{können} die Wahrheit
sagen. \punkt{} Die
Aufrichtigkeit des Zeugen, daß er hat \ori{wollen} die Wahrheit
sagen.}\footnote{\cite{Kant:WienerLogik1966}, \cite[][XXIV: 898.3--4,
20--21]{Kant:GesammelteWerke1900ff.}.} Es greift also zu kurz, wenn Thomas
\name[Thomas]{Kater}  nur auf die
Aufrichtigkeit oder \enquote{Wahrhaftigkeit}
abhebt\footnote{\cite[Vgl.][145]{Kater:PolitikRechtGeschichte1999}:
\enquote{Wahrhaftigkeit des Bezeugenden ist also das wesentliche Kriterium, das
die Beurteilung von Zeugnissen leitet.}}; der Informant muss nicht nur die
Wahrheit sagen \emph{wollen}, er muss dies auch \emph{können}. Doch wie zu
sehen sein wird reicht es nicht, die beiden Aspekte Aufrichtigkeit und
Tüchtigkeit anzuführen; mitunter sind ganz andere Überlegungen relevant, die
sich nicht bei \authorcite{Meier:Vernunftlehre1752}, sondern erst bei
\authorcite{Reimarus:DieVernunftlehrealseineAnweisungzumrichtigenGebrauchderVernunftinderErkenntnisderWahrheit1756}
finden.

Schaut man bei \name[Immanuel]{Kant} lediglich auf die Ausführungen, die aus
seinen Logikvorlesungen überliefert sind, so finden sich genau die Bemerkungen,
die bei allen genannten Autoren ebenfalls zu finden sind. Man solle -- so warnt
\name[Immanuel]{Kant} -- nur \enquote{aufrichtigen} und \enquote{tüchtigen}
Informanten, nicht aber dem \enquote{gemeinen Mann}
glauben\footnote{Vgl. \cite{Kant:LogikPoelitz1966},
\cite[][XXIV: 562.30--38]{Kant:GesammelteWerke1900ff.},
\cite{Kant:WienerLogik1966}, \cite[][XXIV: 898.1--22]{Kant:GesammelteWerke1900ff.}.}, eher das als wahr
annehmen, was von vielen unabhängigen Informanten mitgeteilt
wird\footnote{Vgl. \cite{Kant:LogikPhilippi1966}, \cite[][XXIV:
450.23--28]{Kant:GesammelteWerke1900ff.}, sowie \cite{Kant:LogikPoelitz1966},
\cite[][XXIV: 563.15--17]{Kant:GesammelteWerke1900ff.}.}, und
\enquote{Augenzeugen}, das sind Informanten, die etwas aus erster Hand erfahren
haben, von \enquote{Ohrenzeugen} unterscheiden, die selbst nur wissen, was
andere ihnen
erzählten\footnote{\phantomsection\label{Anmerkung:AugenzeugenundOhrenzeugen}Vgl.
\cite{Kant:LogikPhilippi1966}, \cite[][XXIV:
450.20--28]{Kant:GesammelteWerke1900ff.}. An dieser Stelle erläutert
\name[Immanuel]{Kant} den \S~208 des \titel{Auszugs aus der Vernunftlehre}, an
der \authorcite{Meier:Vernunftlehre1752} von
\enquote{Augenzeugen} und \enquote{Hörenzeugen} spricht
(\cite[vgl.][\S~208]{Meier:AuszugausderVernunftlehre1752}, \cite[][XVI:
505.20--23]{Kant:GesammelteWerke1900ff.}),
\authorcite{Reimarus:DieVernunftlehrealseineAnweisungzumrichtigenGebrauchderVernunftinderErkenntnisderWahrheit1756}
hingegen von \enquote{Hauptzeugen} oder \enquote{ersten Zeugen} und
\enquote{Mittelzeugen}
(\cite[vgl.][\S\S~249--251]{Reimarus:DieVernunftlehrealseineAnweisungzumrichtigenGebrauchderVernunftinderErkenntnisderWahrheit1756}).}.
Auch solle man darauf achten, dass ein Zeuge von Falschauskünften keinen Nutzen
habe.\footnote{Vgl. \cite{Kant:LogikPhilippi1966}, \cite[][XXIV:
450.3--4]{Kant:GesammelteWerke1900ff.}.} Aber welchen Stellenwert haben diese Überlegungen innerhalb der kantischen
Positionierung zur Verträglichkeit von Mündigkeit und testimonialem Wissen?

Es ist auf Grundlage des über
\authorcite{Crusius:Anweisungvernuenftigzuleben1744} und
\authorcite{Meier:Vernunftlehre1752} Gesagten offensichtlich, dass
\name[Immanuel]{Kant} damit lediglich Gemeingut seiner Zeit wiedergibt, ohne viel an eigenen Gedanken
hinzuzufügen.\phantomsection\label{AbschnittzuCrusiusundKritischemJournalismus}
Er bleibt in der Ausführlichkeit und systematischen Ausgestaltung sogar
hinter deren Darstellungen zurück.
Gerade bei der Behandlung der subordinierten Informanten zeigt sich, wie weit er
hinter dem etwa bei \authorcite{Crusius:Anweisungvernuenftigzuleben1744}
erreichten Reflexionsniveau zurückbleibt. Auch dieser sieht natürlich, dass
\singlequote{Augenzeugen} einen \emph{prima facie}-Vorrang vor
\singlequote{Ohrenzeugen}
haben.\footnote{\cite[Vgl.][\S~621]{Crusius:WegzurGewissheitundZuverlaessigkeitdermenschlichenErkenntniss1965}:
\enquote{Ein Zeuge, welcher etwas selbst gesehen  hat, hat bey sonst gleichen
Umständen einen Vorzug vor einem, welcher etwas aus den Nachrichten anderer
hat.} (\ohio)} Aber dieser Vorrang gilt nicht in allen Fällen.
\authorcite{Crusius:Anweisungvernuenftigzuleben1744} schreibt:
\begin{quote}
  Jedoch können auch Fälle vorkommen, da die Glaubwürdigkeit einer Begebenheit
  nicht so wohl von der Versicherung eines Augenzeugen abhanget, weil nemlich
  derselbe unbekannt ist, und keine Präsumtion vor sich hat; sondern da ihr mehr
  von dem Zeugnisse eines mittelbaren Zeugen zuwächset, dessen Verstand und
  Glaubwürdigkeit bekannt ist, und welcher die Präsumtion vor sich hat, daß er
  seine Nachricht auf das Zeugniß tüchtiger Zeugen gegründet, und dieselben wohl
  geprüfet
  habe.\footnote{\Cite[][\S~621]{Crusius:WegzurGewissheitundZuverlaessigkeitdermenschlichenErkenntniss1965},
  \ohio}
\end{quote}
Das hier entwickelte Szenario besagt: Wir haben von einem Ereignis zwei
Berichte. Der erste stammt von einem unmittelbaren Zeugen des Geschehens, von
dem wir allerdings nichts oder sehr wenig wissen. Wir können selbst seine
Bonität daher schwer einschätzen. Der zweite Bericht stammt von einem
Informanten, der selbst auf andere Informanten angewiesen ist, also nur über
testimoniales Wissen von dem Geschehen verfügt. Wir wissen aber auch, dass
dieser Informant über die nötige Kompetenz in der Bewertung von Informanten und
Informationen verfügt. Hier -- sagt
\authorcite{Crusius:Anweisungvernuenftigzuleben1744} -- ist die über mehrere
subordinierte Informanten vermittelte Information glaubwürdiger als diejenige
Information, die wir direkt von einem unmittelbaren Zeugen des Geschehens
erhalten haben.

\name[David]{Hume}s Behauptung, die Gewissheit testimonialer
Erkenntnis könne über die Zeit -- durch die Vermehrung der Zwischenstationen -- nur abnehmen, ist
somit schlicht falsch. Letztlich ändert sich die Gewissheit nicht im Laufe der
Zeit, sonst müsste eine historische Tatsache der Antike im Mittelalter mit viel
größerer Gewissheit für wahr gehalten worden sein als in unseren Tagen. Und dies ist
offensichtlicher Unsinn. Nur wenn sich neue Gegenbelege oder \emph{defeater}
finden, kann sich der Grad der Gewissheit
verringern.\footnote{\cite[Vgl.][\S~626]{Crusius:WegzurGewissheitundZuverlaessigkeitdermenschlichenErkenntniss1965}:
\enquote{Die historische Wahrscheinlichkeit nimmt durch die Länge der Zeit nicht
ab, es wäre denn, daß die Beweisgründe derselben in folgenden Zeiten durch neue
Gründe entkräftet oder ausdrücklich widerleget werden könten; oder daß die
Erkenntnißgründe selbst, und zwar dergestalt verloren giengen, daß die übrig
bleibenden zu einer Zuverläßigkeit oder moralischen Gewißheit nicht mehr
hinlänglich wären.} (\ohio)} Die Qualität einer Information
kann sich sogar erhöhen, wenn sie vermittelt über eine weitere Station
akquiriert wird, und zwar dann, wenn dieser mittlere Informant sich uns gegenüber durch eine höhere
Kompetenz auszeichnet, die Qualität der ursprünglichen Information zu bewerten. Ein
Beispiel liefert uns die Idee eines kritischen Journalismus'. Natürlich
ist es uns möglich, Pressemitteilungen von Parteien selbst zu lesen. Aber es stimmt nicht, dass wir
dann besser informiert wären, als wenn wir die Aufbereitung der Informationen in
einer Tageszeitung rezipieren: Ein guter Journalist weiß, welchen Informationen
er trauen kann, welche zu überprüfen sind und wie er im Zweifelsfall nachhaken
sollte. Ein allgemeiner Vorzug kurzer Informationsketten lässt sich also
mitnichten rechtfertigen. Die Forderung nach Mündigkeit sollte also nicht als
Forderung nach kurzen Informationsketten missverstanden werden.
\singlequote{Kritischer Journalismus} fördert gerade Mündigkeit -- und zwar
dadurch, dass er der Informationskette ein weiteres Glied hinzufügt.

\name[Immanuel]{Kant} verwendet wenig Energie auf die Ausarbeitung einer
systematischen und tragfähigen Darstellung der Bewertung von Informanten,
sondern versorgt lediglich die Zuhörer seiner Vorlesungen mit groben
Informationen über längst bekannte Einsichten und Überlegungen. Kompetenz oder
\singlequote{Tüchtigkeit} und Aufrichtigkeit sind gewiss wichtige Aspekte in der
Beurteilung testimonialer Erkenntnisse, aber sie sind nicht die Gesichtspunkte,
denen in \name[Immanuel]{Kant}s Philosophie systematische Relevanz zukommt. Dass
\name[Immanuel]{Kant} in seinen Vorlesungen gerade diese Aspekte anspricht, ist
möglicherweise nur der Tatsache geschuldet, dass Georg Friedrich
\authorcite{Meier:Vernunftlehre1752}, dessen Lehrbuch in diesen Punkten wenig ausführlich ist, gerade
die beiden behandelt. Für \name[Immanuel]{Kant}s Philosophie relevant
sind Überlegungen, die er selbst in seinen Logikvorlesungen nicht ausführlich
thematisiert, wie etwa die Hermeneutik und Quellenkritik.


\subsection{Hermeneutik und
Quellenkritik}\label{subsubsection:HermannSamuelReimarus} Hermann Samuel
\authorcite{Reimarus:DieVernunftlehrealseineAnweisungzumrichtigenGebrauchderVernunftinderErkenntnisderWahrheit1756}
hat eine systematisch äußerst ausgefeilte Darstellung davon erarbeitet, wann
Informationen oder Informanten glaubwürdig
sind.\footnote{\cite[Vgl.][\S\S~239--258]{Reimarus:DieVernunftlehrealseineAnweisungzumrichtigenGebrauchderVernunftinderErkenntnisderWahrheit1756}.}
Dabei ist zu beachten, dass
\authorcite{Reimarus:DieVernunftlehrealseineAnweisungzumrichtigenGebrauchderVernunftinderErkenntnisderWahrheit1756}
die Bonität nicht nur des \emph{Informanten}, sondern der
\emph{Information}\footnote{Der Ausdruck \enquote{Information} ist hier
systematisch mehrdeutig, insofern er sprachlich einerseits auf den gesamten
Prozess der Übermittlung eines Informationsgehalts durch einen Informanten und
ein Medium der Informationsübertragung verweisen kann, andererseits aber auch
auf den bloßen Informationsgehalt. Hier ist die erste Bedeutung angesprochen,
für die ich auch von Informationsprozess sprechen werde -- unter der Maßgabe,
dass der Informationsgehalt mit inbegriffen ist.} oder des
\singlequote{Zeugnisses} -- also des \emph{Gesamts} einer Nachricht, ihres
Überbringers und des Prozesses ihrer Überbringung --
thematisiert,\footnote{\cite[Vgl.][\S~239]{Reimarus:DieVernunftlehrealseineAnweisungzumrichtigenGebrauchderVernunftinderErkenntnisderWahrheit1756}:
\enquote{\ori{Glaubwürdig} heißt ein Zeugniß, wenn es wegen seiner Wahrheit
werth ist, in die Stelle unserer eigenen Erfahrung gesetzt zu werden.}} wovon
wiederum die Bonität des Informanten nur einen Teil beschreibt und nicht wie bei
\authorcite{Meier:Vernunftlehre1752} die der Information bereits
garantiert\footnote{\cite[Vgl.][\S~214]{Meier:AuszugausderVernunftlehre1752},
\cite[][XVI: 509.22--23]{Kant:GesammelteWerke1900ff.}.}.


Zu den beiden Aspekten Kompetenz und Aufrichtigkeit, welche diese Bonität des
Informanten beschreiben, kommen bei der Bewertung der Bonität des Zeugnisses
zwei weitere Aspekte hinzu. Mit ihrer Hilfe lassen sich weitere
Überlegungen auch bei \name[Immanuel]{Kant} identifizieren, die insbesondere seine
Religionsphilosophie prägen.\footnote{Das Interesse an Überlegungen zu
testimonialem Wissen innerhalb der Philosophie des 18.~Jahrhunderts speist sich
zu einem erheblichen Teil aus Fragen bezüglich der Zuverlässigkeit und Tragkraft
heiliger Schriften, speziell natürlich des Neuen Testaments. Bei David
\name[David]{Hume} finden sich einschlägige Überlegungen entsprechend in
Abschnitt 10 (\enquote{On Miracles}) des \titel{Enquiry Concerning Human
Understanding}. Und
\authorfullcite{Reimarus:DieVernunftlehrealseineAnweisungzumrichtigenGebrauchderVernunftinderErkenntnisderWahrheit1756},
dessen Bekanntheit sich primär den von \authorcite{Lessing:EineDuplik1897}
anonymisiert herausgegeben \singlequote{Fragmenten} verdankt und der uns
hauptsächlich wegen seiner Religionsphilosophie und Offenbarungskritik bekannt
ist, liefert mit die ausführlichsten Überlegungen zu testimonialem Wissen in der
deutschen Aufklärungsphilosophie.
\cite[Vgl.][\S\S~239--258]{Reimarus:DieVernunftlehrealseineAnweisungzumrichtigenGebrauchderVernunftinderErkenntnisderWahrheit1756}.
In allen mir bekannte Fällen geht es in irgendeiner Form um Offenbarungskritik.}
\authorcite{Reimarus:DieVernunftlehrealseineAnweisungzumrichtigenGebrauchderVernunftinderErkenntnisderWahrheit1756}
nennt als wichtige Aspekte im Umgang mit testimonialem Wissen:
\begin{quote}
  \begin{enumerate}
  \item Ob des Zeugen Erfahrung einerley sey mit dem, was wirklich geschehen
  ist? welches auf seine dargelegte Geschicklichkeit in der Erfahrung ankömmt.
  \item Ob seine Nachricht davon einerley sey mit dem, was er sich selbst
  vorstellet erfahren zu haben? welches von seiner willkürlichen Aufrichtigkeit
  im Berichte abhängt.
  \item Ob der Verstand, welchen wir aus seiner Nachricht gezogen, einerley sey
  mit dem, was er hat bezeugen wollen? das gehöret für die Erklärungskunst.
  \item Ob der vermeynte Zeuge eben die Person sey, welche die Nachricht
  gegeben? das gehöret für die
  Kritik.\footnote{\Cite[][\S~240]{Reimarus:DieVernunftlehrealseineAnweisungzumrichtigenGebrauchderVernunftinderErkenntnisderWahrheit1756}.}
  \end{enumerate}
\end{quote}
Die ersten beiden Teilaspekte der Bonitätsfrage entsprechen dem, was
wir mit \authorcite{Meier:Vernunftlehre1752} bereits als
\singlequote{Tüchtigkeit} oder Kompetenz (hier: \enquote{Geschicklichkeit}) und
Aufrichtigkeit beschrieben haben. Die beiden anderen Kriterien sind neu und ihre Beachtung
dadurch bedingt, dass nicht mehr der Informant, sondern der gesamte
Informationsprozess betrachtet wird. Während uns im Alltag zunächst Fragen nach
Kompetenz und Aufrichtigkeit des Informanten beschäftigen, sind Heuristik und
Kritik gerade bei der Beurteilung historischer Dokumente wie alter Urkunden oder
heiliger Schriften von Bedeutung.


Es ergeben sich vier Stichworte zum mündigen Umgang mit testimonialem Wissen:
(1) die \emph{Kompetenz} des Informanten, (2) die \emph{Aufrichtigkeit} des
Informanten, (3) \emph{Hermeneutik} oder die Frage nach dem korrekten
Verständnis des Zeugnisses auf der Seite des Informationsempfängers und (4)
\emph{Kritik} oder die Identifizierung des Informanten.
Freilich sind hermeneutische Überlegungen in diesem Zusammenhang nicht neu;
\authorcite{Meier:Vernunftlehre1752} thematisiert sie zwar nur am Rande
und sieht die Verständigung in der Verantwortung des Informanten, aber gerade
\authorcite{Crusius:Anweisungvernuenftigzuleben1744} setzt sich direkt im
Anschluss an seine Überlegungen zu testimonialem Wissen ausführlich mit solchen
Fragen auseinander.\footnote{\cite[Vgl.][\S\S~628--656]{Crusius:WegzurGewissheitundZuverlaessigkeitdermenschlichenErkenntniss1965}.
Siehe dazu auch \cite[][44--51]{Scholz:VerstehenundRationalitaet1999}.} Doch
\authorcite{Reimarus:DieVernunftlehrealseineAnweisungzumrichtigenGebrauchderVernunftinderErkenntnisderWahrheit1756}
thematisiert sie explizit als Aspekte der Bewertung von Informationen durch
einen mündigen Informationsempfänger.


Die Hermeneutik behandelt zunächst nicht Fähigkeiten und Neigungen der Urheber
testimonialer Erkenntnisse (der Informanten), sondern Kompetenzen und
Tätigkeiten der Empfänger oder Adressaten von Informationen. Während
\authorcite{Meier:Vernunftlehre1752} die Sorge um die rechte Verständlichkeit
von Mitteilungen Sache des Mitteilenden und seiner sprachlichen Kompetenzen sein
ließ, nimmt
\authorcite{Reimarus:DieVernunftlehrealseineAnweisungzumrichtigenGebrauchderVernunftinderErkenntnisderWahrheit1756}
den Empfänger der Mitteilungen in die Pflicht. Gerade Fragen der Hermeneutik fordern nach
\authorcite{Reimarus:DieVernunftlehrealseineAnweisungzumrichtigenGebrauchderVernunftinderErkenntnisderWahrheit1756}
das aktive und kompetente Mitdenken des Rezipienten, denn nur
derjenige kann verstehen, was ihm mitgeteilt wird, der die nötige Sprach-, aber
auch Fachkompetenz besitzt und diese im konkreten Einzelfall aktiv einsetzt.
Sprach- und Sachkompetenz sind dabei als zusammenhängend zu betrachten. Denn
erstens ist die Bedeutung sprachlicher Ausdrücke und ihr Verständnis nicht
unabhängig von \mbox{(Welt-)} Wissen über entsprechende
Gegenstände.\footnote{\cite[Vgl.][\S~256]{Reimarus:DieVernunftlehrealseineAnweisungzumrichtigenGebrauchderVernunftinderErkenntnisderWahrheit1756}:
\enquote{Die Sprachkunde ist mit der Sachenkunde genau verbunden, weil niemand
ein Wort verstehen kann, ohne von der Sache einen Begriff zu haben.}
\authorcite{Reimarus:DieVernunftlehrealseineAnweisungzumrichtigenGebrauchderVernunftinderErkenntnisderWahrheit1756} verwirft hier die oft als selbstverständlich unterstellte
Trennung zwischen begrifflichem Wissen und Weltwissen. Ähnliche Aussagen finden
sich im 20. Jahrhundert bei \textcite{Quine:TwoDogmasofEmpiricism1951} und
\cite{Davidson:OntheVeryIdeaofaConceptualScheme1974}.} Deswegen ist
\authorcite{Meier:Vernunftlehre1752}s Unterscheidung zwischen der Kompetenz,
eine richtige Erfahrung zu machen, und der Fähigkeit, diese korrekt mitzuteilen,
nicht unproblematisch; denn wer beispielsweise über den Begriff der Abseitsstellung
nicht verfügt, kann auch nicht die Erfahrung machen, dass Daniel im Abseits
steht, als Dominik ihm den Ball zuspielt. Zweitens sind sprachliche Ausdrücke in
ihrer Bedeutung
kontextsensitiv.\footnote{\cite[Vgl.][\S~257]{Reimarus:DieVernunftlehrealseineAnweisungzumrichtigenGebrauchderVernunftinderErkenntnisderWahrheit1756}:
\enquote{Bey vielerley bedeutenden Worten kann die eine wirkliche Bedeutung
jeder Stelle nicht anders, als durch den Zusammenhang der Wörter und Sachen,
bestimmt seyn.}} Man täuscht sich, wenn man glaubt von Informanten verlangen zu
können, sie sollten ihre Äußerungen so tätigen, dass der Äußerungskontext von
den Rezipienten nicht mitberücksichtigt werden müsste. Die unvermeidbare
Kontextsensitivität sprachlicher Äußerungen nimmt immer den Rezipienten in die
Pflicht. Und drittens sind auch die Gedanken der Informanten nicht von Anfang an
hinreichend
bestimmt.\footnote{\cite[Vgl.][\S~254]{Reimarus:DieVernunftlehrealseineAnweisungzumrichtigenGebrauchderVernunftinderErkenntnisderWahrheit1756}:
\enquote{Alle Menschen denken aber nicht gleich klar und deutlich: oder sie
bestimmen ihre Gedanken nicht so genau\dots}} Es gibt immer
Verständnismöglichkeiten, die wir im Zweifelsfall durch selbständiges Mitdenken
als unvernünftig ausschließen müssen, ohne dass dies durch die Mitteilung
explizit vorgegeben wäre oder dem Mitteilenden anderweitig
vorgeschwebt hätte.\footnote{Man denke an folgendes Beispiel
\name[Ludwig]{Wittgenstein}s: \enquote{Jemand sagt mir: \enquote{Zeige den
Kindern ein Spiel!} Ich lehre sie, um Geld würfeln, und der Andere sagt mir
\enquote{Ich habe nicht so ein Spiel gemeint}. Mußte ihm da, als er mir den
Befehl gab, der Ausschluß des Würfelspiels vorschweben?}
\parencite[][\S~70]{Wittgenstein:PhilosophischeUntersuchungen2003}. Siehe hierzu
auch \cite[][]{Kambartel:VersuchueberdasVerstehen1991}.}



\enquote{Kritik} steht hier für die Frage, ob eine Mitteilung auch tatsächlich
aus der Quelle stammt, die als solche ausgegeben wird; es handelt sich um
Untersuchungen, für die sich inzwischen der Ausdruck \enquote{Quellenkritik}
als \emph{terminus technicus} eingebürgert hat. Wenig verwunderlich ist, dass es sich
um einen Ausdruck handelt, der fast nur in speziellen Kontexten -- etwa in der
Geschichtswissenschaft oder in der Theologie --, nicht aber im Alltag beheimatet
ist. Wenn Jasmin von Peter erfährt, dass es regnet, dann stellt sich die Frage,
ob es tatsächlich Peter ist, der spricht, nicht. Im Alltag und gerade im mündlichen Gespräch können wir den
Absender einer Nachricht in der Regel mühelos identifizieren. Anders verhält es sich bei schriftlichen Mitteilungen und vor
allem historischen Dokumenten und Urkunden, die sich durchaus als Fälschungen
erweisen können.


Eine Form von Quellenkritik, die sich bei \name[Immanuel]{Kant} findet, ist das
Bestreiten der Möglichkeit, Gott als Quelle testimonialen Wissens zu
identifizieren.\footnote{Zu \name[Immanuel]{Kant}s Zurückweisung von
Offenbarung zugunsten einer Vernunftreligion siehe
\cite{Doerflinger:UeberdenaufgeklaertenUmgangmitGottesWort2009}.} Es gebe
keinerlei Möglichkeit zu erkennen, dass Gott der Urheber einer Mitteilung sei,
wohl aber Möglichkeiten, ihn als Urheber auszuschließen.
Das einzige zulässige, aber auch ein hinreichendes Kriterium sei die
Vereinbarkeit der Mitteilung oder Aufforderung mit dem moralischen
Gesetz.\footnote{\cite[Vgl.][A 102]{Kant:DerStreitderFakultaeten1977},
\cite[][VII: 63.9--17]{Kant:GesammelteWerke1900ff.}.} \name[Immanuel]{Kant}s
Beispiel im \titel{Streit der Fakultäten} macht seine Haltung mehr als deutlich:
\begin{quote}
  \phantomsection\label{Beispiel:AbrahamOpfertSeinenSohn}Zum Beispiel kann die
  Mythe von dem Opfer dienen, das Abraham auf göttlichen Befehl durch Abschlachtung und Verbrennung seines einzigen Sohnes -- (das
  arme Kind trug unwissend noch das Holz hinzu) -- bringen wollte. Abraham hätte
  auf diese vermeinte göttliche Stimme antworten müssen: \enquote{Daß ich
  meinen guten Sohn nicht töten solle, ist ganz gewiß; daß aber du, der du mir
  erscheinst, Gott sei, davon bin ich nicht gewiß und kann es auch nicht
  werden}, wenn sie auch vom (sichtbaren) Himmel
  herabschallte.\footnote{\cite[][A 102\,f.,]{Kant:DerStreitderFakultaeten1977}
  \cite[][VII: 63.32--38]{Kant:GesammelteWerke1900ff.}.}
\end{quote}
\name[Immanuel]{Kant}s Verhältnis zu heiligen Schriften ist nicht durch detaillierte
quellenkritische Analysen geprägt, sondern durch dieses eine globale Argument
gegen die Möglichkeit von als solchen identifizierbaren Offenbarungen Gottes:
Gott ist als Quelle einer Mitteilung nicht anders zu identifizieren als über die
moralische Beschaffenheit des Mitgeteilten. \emph{Wenn} es sich um eine
Mitteilung handelt, die moralisch Gebotenes als solches anweist oder doch
wenigstens mit unseren moralischen Einsichten kompatibel ist, \emph{dann} ist es
zumindest \emph{möglich}, dass Gott tatsächlich Urheber der Mitteilung ist.
Handelt es sich hingegen  um eine Information, die gegen unsere moralischen
Einsichten verstößt, dann können wir Gott als Urheber ausschließen.

Auf hermeneutische Überlegungen greift \name[Immanuel]{Kant} zurück, um zu
erläutern, wie aus der Perspektive der Aufklärung mit Schriften umzugehen sei,
deren Urheber nicht Gott ist, sondern ein Mensch, der selbst an eine Inspiration
seines Schreibens durch Gott oder eine direkte Offenbarung glauben mag. Weil der
Inhalt heiliger Schriften mit \name[Immanuel]{Kant}s religionsphilosophischer
Haltung nicht immer harmoniert, muss die Vereinbarkeit auf dem Wege
der Hermeneutik erfolgen. Und deswegen sei es Aufgabe des Lesers, den
\emph{symbolischen} Sinn der Aussagen herauszufinden, statt die Schriften
wörtlich auszulegen:
\begin{quote}
  Daß alle Völker der Erde mit dieser Vertauschung angefangen haben, und daß,
  wenn es darum zu tun ist, was ihre Lehrer selbst, bei Abfassung ihrer heiligen
  Schriften wirklich gedacht haben, man sie alsdann nicht symbolisch, sondern
  \ori{buchstäblich} auslegen müsse, ist nicht zu streiten; weil es unredlich
  gehandelt sein würde, ihre Worte zu verdrehen. Wenn es aber nicht bloß um die
  \ori{Wahrhaftigkeit} des Lehrers, sondern auch, und zwar wesentlich, um die
  \ori{Wahrheit} der Lehre zu tun ist, so kann und soll man diese, als bloße
  symbolische Vorstellungsart, durch eingeführte Förmlichkeit und Gebräuche jene
  praktischen Ideen zu begleiten, auslegen; weil sonst der intellektuelle Sinn,
  der den Endzweck ausmacht, verloren gehen
  würde.\footnote{\cite[][BA~108]{Kant:AnthropologieinpragmatischerHinsicht1977},
  \cite[][VII: 192.7--17]{Kant:GesammelteWerke1900ff.}.}
\end{quote}
Diese Vertauschung aufzuheben bezeichnet \name[Immanuel]{Kant} explizit als
\enquote{Aufklärung}.\footnote{\cite[Siehe][BA
107\,f.,]{Kant:AnthropologieinpragmatischerHinsicht1977} \cite[][VII:
191.36--192.5]{Kant:GesammelteWerke1900ff.}.} \singlequote{Heilige Schriften}
haben einen \singlequote{intellektuellen Sinn}, der nicht den Zweck der
Verfasser dieser Schriften ausmacht, sondern \emph{unseren}
\singlequote{Endzweck}. Damit spielt \name[Immanuel]{Kant} auf ein Thema des
Kapitels \ref{Zitat:EndzweckalsganzeBestimmungdesMenschen} an: Der
\singlequote{Endzweck} war nichts geringeres \enquote{als die ganze Bestimmung
des Menschen}\footnote{\cite[][B 868]{Kant:KritikderreinenVernunft2003},
\cite[][III: 543.11]{Kant:GesammelteWerke1900ff.}; siehe oben Seite
\pageref{Zitat:EndzweckalsganzeBestimmungdesMenschen} in Kapitel
\ref{Zitat:EndzweckalsganzeBestimmungdesMenschen}.}. Aufgeklärte Religion
interpretiert heilige Schriften so, dass genau dieser Inhalt dort hineingelesen
werden kann. Und sie ist darin so frei, gegen jede -- und sei sie noch so
offensichtliche -- Autorintention zu
verstoßen.\footnote{\cite[Vgl.][134]{Doerflinger:UeberdenaufgeklaertenUmgangmitGottesWort2009}.}

Ein solches Vorgehen hat freilich mit dem üblichen Umgang mit
testimonialen Erkenntnissen wenig gemein. Wir sollten sogar sagen, dass es sich
überhaupt nicht um einen Umgang mit testimonialen Erkenntnissen handelt, denn
die Mitteilung ist nicht die Grundlage unserer Erkenntnis. Wir sollen den
\singlequote{heiligen Schriften} ja nicht den von ihren Autoren intendierten
Literalsinn entnehmen und auf Grundlage des Ansehens ihrer Verfasser das
Gesagte für wahr halten. Stattdessen sollen wir von Anfang an davon ausgehen,
dass die Verfasser irren, und das in ihrem literal gemeinten Texten als
symbolisch ausgedrückt erkennen, was wir unabhängig von ihren Texten bereits
wissen. Es gleicht eher den Fällen, in denen uns jemand auf etwas aufmerksam
macht, was wir dann selbst sehen, oder in denen wir einen mathematischen Beweis vorgeführt
bekommen, den wir im Anschluss selbst kontrollieren können. Das Ansehen des
Informanten ist in beiden Fällen irrelevant, weswegen es sich nicht um
\emph{genuin} testimoniales Wissen handelt.\footnote{Siehe oben, Kapitel
\ref{Abschnitte:ZweiThemenSozialerErkenntnistheorie}, ab
S.~\pageref{Abschnitte:ZweiThemenSozialerErkenntnistheorie}.} Und so nützt es
auch nicht als allgemeiner Vorschlag zur Vereinbarung von epistemischer
Autonomie und testimonialem Wissen, weil es sich bei dem Ergebnis der
Bibelauslegung im Sinne \name[Immanuel]{Kant}s einfach nicht um testimoniales
Wissen handelt.

\section{Historische Kenntnis versus
Wissenschaft}\label{subsection:BewertungvonInformationennachihrerART}

Nach \name[Immanuel]{Kant} ist die Maxime einer niemals passiven Vernunft
durchaus mit dem Erwerb von Wissen durch das Zeugnis anderer verträglich, weil
man sich auch bei bei dem Wissenserwerb über andere mündig (aktiv) oder unmündig
(passiv) verhalten kann. Eine erste, sich ganz natürlich einstellende Idee ist
in diesem Zusammenhang, die Mündigkeit in die Bewertung der Bonität des
Informanten zu setzen. Vor dem Hintergrund einer solchen Überlegung wird
mitunter auch \name[Immanuel]{Kant} interpretiert, etwa von
\authorfullcite{Kater:PolitikRechtGeschichte1999} und
\authorfullcite{Scholz:DasZeugnisanderer2001}. Wie in Kapitel
\ref{subsection:BewertungvonInformationenanhandihrerQuellen} zu sehen war,
wird der von \authorcite{Wolff:Psychologiaempirica1968} und \authorcite{Meier:Vernunftlehre1752} aufgestellte kurze Kriterienkatalog, den
\name[Immanuel]{Kant} in seinen Vorlesungen erläutert, von ihm ebenso wenig
negiert wie die beiden zusätzlichen Kriterien, die
\authorcite{Reimarus:DieVernunftlehrealseineAnweisungzumrichtigenGebrauchderVernunftinderErkenntnisderWahrheit1756}
systematisiert und die in \name[Immanuel]{Kant}s Religionsphilosophie und
Offenbarungskritik so bedeutsam werden. Aber mit diesen vier Kriterien ist noch
keine befriedigende Darstellung dessen erreicht, wie \name[Immanuel]{Kant}
Aufklärung mit der Möglichkeit testimonialen Wissens vereinbart. Tatsächlich
sind der grundlegende Punkt noch gar nicht berührt.

\subsection{Vernunftwahrheiten und
Erfahrungstatsachen}\label{subsection:VernunftwahrheitenUndErfahrungstatsachen}
\name[Immanuel]{Kant} sagt es sei egal, ob wir eine Überzeugung auf der
Grundlage auf der Grundlage eigener oder fremder Erfahrung annehmen. Darin kommt
nicht nur zum Ausdruck, dass testimoniales Wissen kein von anderen Wissensarten
abgeleitetes oder gar defizientes Wissen ist. Die Informationen, die andere uns
geben, sind der eigenen Wahrnehmung gleichrangig. Es kommt aber implizit auch
zum Ausdruck, dass die Differenzierung von Erkenntnissen in
solche aus Erfahrung (empirisches Wissen) und solche aus Vernunft
(Vernunftwahrheiten) wichtig ist. Es gibt Erkenntnisse, bei denen wir Andere als
Autoritäten ansehen und ihr Zeugnis als hinreichend annehmen können, und es gibt
Erkenntnisse, bei denen das nicht so ist. In der {\jaeschelogik} lesen wir:
\begin{quote}\phantomsection\label{Zitat:Kant:TestimonialesWissenVernunftErfahrung}
Wenn wir in Dingen, die auf Erfahrungen und Zeugnissen beruhen, unsre
Erkenntnis auf das Ansehen andrer Personen bauen: so machen wir uns dadurch
keiner Vorurteile schuldig; denn in Sachen dieser Art muß, da wir nicht alles
selbst erfahren, und mit unserm eigenen Verstande umfassen können, das Ansehen
der Person die Grundlage unsrer Urteile sein. -- Wenn wir aber das Ansehen
anderer zum Grunde unseres Fürwahrhaltens in Absicht auf Vernunfterkenntnisse
machen: so nehmen wir diese Erkenntnisse auf bloßes Vorurteil
an.\footnote{\cite[][A~120]{Kant:ImmanuelKantsLogik1977}, \cite[][IX:
77.31--78.5]{Kant:GesammelteWerke1900ff.}. Für dieses Zitat gibt es -- soweit
ich sehen kann -- in den Notizen \name[Immanuel]{Kant}s keine direkte Vorlage.
Sein Inhalt wird sich im Verlauf der Überlegungen jedoch bestätigen.}
\end{quote}
Die Gewissheit ist bei eigenen Erfahrungen (Erfahrungswissen) und bei
Mitteilungen (testimoniales Wissen) dieselbe, beides ist fehlbar, aber dennoch
\emph{Wissen}.\footnote{\cite[Vgl.][A~111,
Anm.]{Kant:ImmanuelKantsLogik1977}\protect{,} \cite[][IX:
72.24]{Kant:GesammelteWerke1900ff.}.} Daher genügt es selbst den pedantischsten
Ansprüchen an die Sprache zu sagen, man \emph{wisse}, dass es eine Stadt
namens Rom gibt, dass \name[Immanuel]{Kant} in Königsberg lebte und dergleichen, ohne dies
jemals selbst überprüft zu
haben.\footnote{\cite[Vgl.][A~319]{Kant:Washeisst:SichimDenkenorientieren?1977},
\cite[][VIII:
141.13--17]{Kant:GesammelteWerke1900ff.}. Dagegen spricht jedoch der von
\name[Immanuel]{Kant} vorgeschlagene Sprachgebrauch in
\cite[][A~v]{Kant:MetaphysischeAnfangsgruendederNaturwissenschaften1977},
\cite[][IV:
468.17--19]{Kant:GesammelteWerke1900ff.}:
\enquote{\ori{Eigentliche} Wissenschaft kann nur diejenige genannt werden, deren
Gewißheit apodiktisch ist; Erkenntnis, die bloß empirische Gewißheit enthalten
kann, ist ein nur uneigentlich so genanntes \ori{Wissen}.}} \name[Immanuel]{Kant} unterscheidet
Erkenntnisse dieser Art, von denen testimoniales Wissen zu haben möglich und
legitim ist, von Vernunftwahrheiten, von denen testimoniales Wissen zu haben
unmöglich ist oder denen eine solche Form zumindest unangemessen ist.


\name[Immanuel]{Kant} hat die hier zugrunde liegende Unterscheidung nicht erst
eingeführt. Während der gesamten Neuzeit wurden Unterscheidungen zwischen zwei
Arten von Erkenntnissen artikuliert, die \name[Immanuel]{Kant}s Unterscheidung
von Erfahrungswissen und Vernunftwahrheiten ähneln.
\authorcite{Leibniz:Meditationesdecognitioneveritateetideis1999}' Unterscheidung
von \emph{v{\'e}rit{\'e}s de raison} und \emph{v{\'e}rit{\'e}s de
fait}\footnote{\cite[Vgl.][\S~33]{Leibniz:Lamonadologie2002}.}
sowie \authorcite{Hume:AnEnquiryConcerningHumanUnderstanding1964}s Unterscheidung von
\emph{relations of ideas} und \emph{matters of
fact}\footnote{\cite[Vgl.][20--23]{Hume:AnEnquiryConcerningHumanUnderstanding1964}.}
sind vielleicht die bekanntesten Darstellungen.
Angewandt auf die Frage nach dem Umgang mit testimonialem Wissen ergibt sich
jedoch eine nicht ganz so einfach sichtbare Verbindung zu
\authorcite{Descartes:OeuvresdeDescartes1983}, wenn dieser die
Büchergelehrsamkeit als Ausdruck bloß \enquote{historischer Kenntnisse}
kritisiert. \authorcite{Descartes:OeuvresdeDescartes1983} kritisiert, dass wir
keine Wissenschaft, sondern lediglich historische Kenntnisse erwürben, wenn wir
Wissen aus Büchern erwerben.\footnote{Siehe oben Kapitel
\ref{Abschnitt:DescartesundhistorischeKenntnisse},
S.~\pageref{Abschnitt:DescartesundhistorischeKenntnisse}.} Die sich
anschließende Frage ist, was den Unterschied zwischen Wissenschaft
und historischen Kenntnissen ausmacht. Diesen -- so werde ich argumentieren --
verstehen wir nur vor dem Hintergrund der neuzeitlichen Unterscheidung zweier
Erkenntnisarten und ihrer Relevanz für Fragen der Sozialen
Erkenntnistheorie.

Einen guten Ansatz zum Verständnis des Zusammenhangs der Unterscheidung beider
Erkenntnisarten mit der Frage nach der Legitimität des Fürwahrhaltens aufgrund
von Mitteilungen finden wir bei Thomas \name[Thomas]{Hobbes}, der schreibt:
\begin{quote}
\Revision[Selbst]{Es gibt zwei Arten von \emph{Wissen}
(\enquote{\emph{knowledge}}):
Das eine ist \emph{Tatsachenwissen} (\enquote{\emph{knowledge of fact}}), das andere ist das
\emph{Wissen vom Schließen von einer Behauptung auf eine andere}. Das erste ist
nichts anderes als sinnliche Wahrnehmung und Erinnerung und ist \emph{absolutes
Wissen} (\enquote{\emph{absolute knowledge}}); wie wenn wir sehen, dass etwas getan
wird, oder uns erinnern, dass es getan wurde. Und dieses Wissen verlangen wir von einem Zeugen.
Das andere Wissen heißt \emph{Wissenschaft} (\enquote{\emph{science}}) und ist
\emph{konditional} (\enquote{\emph{conditional}}); so wissen wir, dass,
\emph{wenn die gezeigte Figur ein Kreis ist, dass ihn dann jede gerade Linie
durch seinen Mittelpunkt in zwei gleich große Teile teilt}. Und dieses Wissen
erwarten wir von einem Gelehrten (\enquote{\emph{philosopher}}), also von
jemandem, der sich auf den Gebrauch der Vernunft
beruft.}\footnote{\Revision[Selbst]{\enquote{\ori{There} are of \ori{knowledge}
two kinds; whereof one is \ori{knowledge of fact:} the other \ori{knowledge of the consequence of one affirmation to another}. The former is nothing else, but sense and memory, and is \ori{absolute knowledge;} es when we see a fact doing, or remember it done:
an this is knowledge required in a witness. The latter is called \ori{science;}
and is \ori{conditional;} as when we know, that, \ori{if the figure shown be a
circle, then any straight line through the centre shall divide it into two equal
parts.} And this is the knowledge required in a philosopher; that is to say, of
him that pretends to reasoning} \parencite[][Chapter IX,
S.~71]{Hobbes:Leviathan1962}.}}
\end{quote}\enlargethispage{\baselineskip}
Was wir von einem Zeugen verlangen, ist \emph{absolutes} im Gegensatz zu
\emph{konditionalem} Wissen. Dabei liegt der entscheidende Unterschied zwischen
beiden Arten nicht etwa darin, dass sich der Zeuge bei absolutem Wissen
besonders sicher sein müsste -- \enquote{absolut} hat nicht das Geringste mit
\enquote{unfehlbar} zu tun, genauso wenig wir \enquote{konditional} mit
\enquote{fehlbar} oder \enquote{vorläufig}\footnote{Ich vermeide aus naheliegenden Gründen die
Übersetzungsmöglichkeiten \enquote{hypothetisch} oder \enquote{bedingt} für
\enquote{conditional} und bleibe bei der etwas sperrig klingenden Eindeutschung
\enquote{konditional}.} --, sondern in der inneren Struktur
der Erkenntnis selbst. Konditionales Wissen ist Wissen um den
\emph{Zusammenhang} von (Behauptungen über) Tatsachen, während absolutes Wissen
einfach einzelne Tatsachen beschreibt, die (kontingenter Weise) geschehen
sind.\footnote{Logisch scheint diese Unterscheidung schwierig zu artikulieren zu
sein, insofern sich jede \distanz{absolute} Aussage $p$ ganz einfach in die
\distanz{konditionale} Aussage $ \top \supset p$ umwandeln ließe, wenn $ \top $
hier für eine beliebige Tautologie steht. Umgekehrt kann jedes Konditional $p
\supset q$ auch nicht-konditional in der Form $\neg (p \land \neg q)$ dargestellt
werden.} So zählt als absolutes Wissen, dass 1755 vor Lissabon ein Seebeben
stattfand, dass Rom die Hauptstadt Italiens ist und 753 v.\,u.\,Z.\ gegründet wurde oder dass
\name[Immanuel]{Kant} in Königsberg lehrte. Überwiegend konditional ist hingegen
das Wissen der Mathematik oder der Newtonschen Physik. Es klärt uns nicht über
einzelne Tatsachen auf, sondern über mathematische oder physikalische
Zusammenhänge, die zwischen einzelnen Tatsachen bestehen.
Der Satz des \singlename{Pythagoras} sagt uns: \emph{Wenn} in einem
rechtwinkligen Dreieck die Katheten die Längen $a$ und $b$ (z.\,B.\ $\unit[3]{cm}$ und
$\unit[4]{cm}$) haben, \emph{dann} hat die Hypotenuse die Länge $c = \sqrt{a^2 +
b^2}$ ($\unit[5]{cm}$). Konditionales, nicht
absolutes Wissen zählt dabei im eigentlichen Sinne als
Wissenschaft\footnote{\enquote{\ori{Science} is the knowledge of consequences,
and dependence of one fact upon another}
\parencite[][35]{Hobbes:Leviathan1962}.} oder (dem Sprachgebrauch der deutschen
Aufklärung näher) Philosophie, welche dadurch der Geschichte
(\enquote{\emph{history}}) oder dem Tatsachenwissen entgegengesetzt
wird.\footnote{Tatsachenwissen sei --
wenn es aufgezeichnet wurde (\enquote{[t]he register of \ori{knowledge of
fact}}) -- \emph{Geschichte}, und zwar zum einen \emph{Naturgeschichte}, zum
anderen \emph{Kultur-} oder \emph{Menschheitsgeschichte} (\enquote{\ori{civil history;}
which is the history of the voluntary actions of men in commonwealths})
\parencite[vgl.][71]{Hobbes:Leviathan1962}.}

Interessant für meine Fragestellung ist \authorcite{Hobbes:Leviathan1962}, weil
er die beiden Erkenntnisarten unterscheidet als Wissen, welches wir von Zeugen
verlangen, auf der einen Seite und Wissen, welches ein Gelehrter haben soll, auf
der anderen Seite. Testimoniales Wissen erwerben wir in der Regeln bei
\singlequote{absolutem} Wissen oder Tatsachenwissen. Der Zeuge soll nicht allgemeine
Wahrheiten über Zusammenhänge explizieren, sonder verraten, was geschehen
ist. Es soll also für gewöhnlich kein konditionales Wissen oder
wissenschaftliche Erkenntnisse vortragen. Absolutes und
konditionales Wissen entsprechen weitgehend dem, was wir von einem \emph{Zeugen}
auf der einen und einem \emph{Sachverständigen} oder \singlequote{Experten} auf
der anderen Seite erwarten. Und dieser Unterschied ist relevant für die Frage
nach einem verantwortlichen Umgang mit testimonialem Wissen.

An \authorcite{Hobbes:Leviathan1962}' Darstellung sei ein weiterer Punkt
hervorgehoben, der auch bei
\authorcite{Wolff:Discursuspraeliminarisdephilosophiaingenere1996} und
\name[Immanuel]{Kant} wichtig sein wird: Das \enquote{Wissen vom Schließen von
einer Behauptung auf eine andere} (\enquote{knowledge of the consequence of one
affirmation to another}) bezeichnet ein \emph{know how}, kein \emph{know that},
weswegen die direkte Übertragung in die deutsche Sprache entsprechend sperrig
klingt. Es ist kein Wissen im eigentlichen Sinn dieses Wortes, sondern ein
\emph{Können} oder eine Kompetenz. Wer dieses \singlequote{Wissen} im Sinne von
\emph{know how} besitzt, kann nicht bloß diesen wahren Satz äußern, dass eine
gerade Linie durch den Mittelpunkt eines Kreises diesen in gleich große Teile
teilt. Er kann diesen Satz in verschiedenen Situationen \emph{anwenden}. Bei Tatsachenwissen mögen
ebenfalls Kompetenzen des Informanten vorausgesetzt sein (siehe Kapitel
\ref{subsubsection:GeorgFriedrichMeier}), aber das Tatsachenwissen selbst
\emph{besteht} doch nicht in einer solchen Kompetenz. Von einem Gelehrten
hingegen erwarten wir eine entsprechende Kompetenz; wir erwarten von einem
Mathematiker, dass er Beweise \emph{führen}, und nicht, dass er sie auswendig
aufsagen kann.

Ich hatte schon in Kapitel \ref{subsection:SelbstdenkenbeiKant} herausgestellt,
dass die Frage nach der Kompetenz für die Aufklärung bei
\authorcite{Wolff:Discursuspraeliminarisdephilosophiaingenere1996} und
\name[Immanuel]{Kant} entscheidend sein werde. Gerade deswegen schließen beide
an diesen Gedanken einer Unterscheidung von Wissensarten an, den wir in einer
sehr einfachen, aber klaren Ausgestaltung bei \authorcite{Hobbes:Leviathan1962}
finden. \name[Immanuel]{Kant}s direkte Vorlage für die Ausarbeitung seiner
eigenen Überlegungen ist
\authorcite{Wolff:Discursuspraeliminarisdephilosophiaingenere1996} Konzeption
dieser
Unterscheidung\footcite[vgl.][\pno~12\,f.]{Albrecht:KantsKritikderhistorischenErkenntnis--einBekenntniszuWolff?1982},
die nun dargestellt werden soll.

\subsection{Philosophische und historische Erkenntnis bei
Christian Wolff}\label{paragraph:wolffswarnung}
\authorfullcite{Wolff:Psychologiaempirica1968} unterscheidet in seinem
\titel{discursus praeliminaris de philosophia in genere} drei Arten der
Erkenntnis: (1) die historische Erkenntnis (\enquote{\emph{cognitio historica}},
\S~3) als bloße unzusammenhängende Kenntnis dessen, was ist oder geschieht
(\enquote{\emph{nuda facti notitia}}, \S~7), (2) die philosophische Erkenntnis
(\enquote{\emph{cognitio philosophica}}, \S~6) als eine \emph{rationale}, Gründe
und Ursachen erforschende (\enquote{\emph{in} [\emph{cognitione}; A.\,G.]
\emph{philosophica reddimus rationem}}, \S~17) Erkenntnis dessen, was ist oder
geschieht, und (3) die mathematische Erkenntnis (\enquote{\emph{cognitio
mathematica}}, \S~14) als Erkenntnis von
Quantitäten.\footnote{\cite[Vgl.][Cap.I.
De triplici cognitione
humana]{Wolff:Discursuspraeliminarisdephilosophiaingenere1996}. Siehe auch die
weniger ausführlichen Bemerkungen in
\cite[][Vorbericht]{Wolff:VernuenftigeGedankenvondenKraeftendesmenschlichenVerstandesundihremrichtigenGebraucheinErkenntnisderWahrheit1978},
sowie
\cite[][Praefatio]{Wolff:Cogitationesrationalesdeviribusintellectushumani1983}.}
Da er die historische Erkenntnis des weiteren in die gemeine (\emph{communis})
und die verborgene (\emph{arcana}) historische Erkenntnis unterteilt, ergibt
sich die in Abbildung \ref{abbildung:ZeichnungErkenntnisartennachWolff.pdf}
dargestellte Einteilung.

\begin{figure}[htb]
\begin{minipage}[t]{\textwidth}
\centering
\begin{tikzpicture}[edge from parent fork down,
level 1/.style = {level distance=1.5cm, sibling distance=3.5cm},
level 2/.style = {level distance=1.5cm, sibling distance=2.5cm},
every node/.style={rectangle,draw=black,fill=gray!25, thin, inner sep=0.5em, minimum size=0.5em, align=center},
edge from parent/.style={thin,draw},
mylabel/.style={draw=none, fill=none, text width=5cm,text centered, inner sep=0.5em, anchor=base} ]
\node {\emph{cognitio}}
	child {node {\emph{historica}}
		child {node {\emph{communis}}}
		child {node {\emph{arcana}}}}
	child {node {\emph{philosophica}}}
	child {node {\emph{mathematica}}}
;
\end{tikzpicture}
  \caption{Einteilung der Erkenntnisarten nach
  \authorfullcite{Wolff:Psychologiaempirica1968}}\label{abbildung:ZeichnungErkenntnisartennachWolff.pdf}
\end{minipage}
\end{figure}

\authorcite{Wolff:Psychologiaempirica1968}s Darstellung unterscheidet sich von
Hobbes' Systematik am augenscheinlichsten durch die weitere Kategorie der
mathematischen Erkenntnis, die \name[Immanuel]{Kant} im Anschluss an
\authorcite{Wolff:Discursuspraeliminarisdephilosophiaingenere1996} mit der
philosophischen zur rationalen Erkenntnis zusammenfassen
wird.\footnote{\cite[Vgl.][B 863--865]{Kant:KritikderreinenVernunft2003},
\cite[][III: 540.32--33, 541.18--20]{Kant:GesammelteWerke1900ff.}. Siehe dazu
weiter unten Kapitel \ref{section:MuendigkeitundPhilosophie}.} Ich stelle die
mathematische Erkenntnis der Einfachheit und Übersichtlichkeit halber zunächst
zurück und behandle sie als Sonderfall in der Auseinandersetzung mit
\name[Immanuel]{Kant}s Konzeption.\footnote{Siehe Kapitel
\ref{subsubsection:EndlichesundUnendlichesErkennen}.} Hier expliziere ich
zunächst den Unterschied zwischen historischer und philosophischer
Erkenntnis (Kapitel \ref{paragraph:wolffswarnung}), um dann zu
erörtern, wie sich mit \authorcite{Wolff:Discursuspraeliminarisdephilosophiaingenere1996} die
Behauptung \authorcite{Descartes:OeuvresdeDescartes1983}' konkretisieren lässt,
durch das Lesen von Büchern erwerbe man nur historische Kenntnisse statt echter
Wissenschaft (Kapitel \ref{subsubsection:HistorischeKenntnisDerPhilosophie}). Im
letzten Abschnitt gehe ich einem Einwand gegen meine
\authorcite{Wolff:Discursuspraeliminarisdephilosophiaingenere1996}interpretation
nach und zeige auf, dass es sich bei der Differenzierung von historischen und
philosophischen Erkenntnissen nicht, wie etwa
\authorfullcite{Kambartel:ErfahrungundStruktur1968} behauptet, um einen
Unterschied zwischen singulären und allgemeinen Urteilen handelt
(Kapitel \ref{subsubsection:HistorischeErkenntnisallgemeinerTatsachen}).

Die drei Erkenntnisarten werden im ersten Kapitel des \titel{Discursus
praeliminaris de philosophia in genere} jeweils so vorgestellt, dass zunächst
die Grundlage der jeweiligen Erkenntnisart beschrieben (\S 1:
\enquote{\emph{Fundamentum cognitionis historicae}}; \S 4:
\enquote{\emph{Fundamentum cognitionis philosophicae}}; \S 13: \enquote{\emph{Fundamentum
cognitionis mathematicae}}), danach die Definition derselben angeben (\S 3:
\enquote{\emph{cognitio historica}}; \S 6: \enquote{\emph{cognitio philosophica}}; \S 14:
\enquote{\emph{cognitio mathematica}}) und schließlich ihr Verhältnis zu den zuvor
besprochenen Erkenntnisarten diskutiert wird. Bezüglich der historischen und philosophischen
Erkenntnis wird darüber hinaus zwischen der Beschreibung ihrer Grundlage und der
Angabe ihrer Definition zusätzlich die Frage nach ihrer Reichweite angesprochen,
aber nicht abschließend beantwortet (\S 2: \enquote{\emph{Cur ejus [i.\,e.~cognitionis
historicae; A.\,G.] hic limites non constituantur}}; \S 5: \enquote{\emph{Cur cognitionis
philosophicae non constituantur limites}}). Interessant und aussagekräftig sind
jeweils Grundlage und Definition, die nur zusammen ein Verständnis der
jeweiligen Erkenntnisart ermöglichen. Im Falle der historischen Erkenntnissen
lesen sich diese wie folgt:
\begin{quote}
  \ori{Wir erkennen mit Hilfe der Sinne die Dinge, die in der materiellen
  Welt sind und geschehen, und der Geist ist sich der Veränderungen bewußt, die
  in ihm selbst
  geschehen}.\footnote{\Cite[][\S~1]{Wolff:Discursuspraeliminarisdephilosophiaingenere1996}:
  \enquote{\ori{Sensuum beneficio cognoscimus, quae in mundo materiali sunt atque
  fiunt, {\&} mens sibi conscia est mutationum, quae in ipsa accidunt}.}}\\
  Die \ori{Erkenntnis} der Dinge, die sind und geschehen, sei es dass sie in der
  materiellen Welt, sei es dass sie in den immateriellen Substanzen, wird von
  uns \ori{historisch}
  genannt.\footnote{\Cite[][\S~3]{Wolff:Discursuspraeliminarisdephilosophiaingenere1996}:
  \enquote{\ori{Cognitio} eorum, quae sunt atque fiunt, sive in mundo materiali,
  sive in substantiis immaterialibus accidant, \ori{historica} a nobis appellatur.}}
\end{quote}
Die historische Erkenntnis ist also die Erkenntnis dessen, was ist oder
geschieht, und ihre Grundlage ist die sinnliche Wahrnehmung. Als Beispiele für
historische Erkenntnisse nennt
\authorcite{Wolff:Discursuspraeliminarisdephilosophiaingenere1996}, dass jemand
aus Erfahrung wisse, dass die Sonne morgens auf- und abends untergehe und
dass Tiere sich durch Zeugung fortpflanzen.

Noch 1709 in den \titel{Vernünftige[n] Gedanken von den Kräften des menschlichen
Verstandes und ihrem richtigen Gebrauche in Erkenntnis der Wahrheit}
identifiziert \authorcite{Wolff:Psychologiaempirica1968} die historische
Erkenntnis mit der \emph{cognitio
communis}.\footnote{\cite[Vgl.][\S~6]{Wolff:VernuenftigeGedankenvondenKraeftendesmenschlichenVerstandesundihremrichtigenGebraucheinErkenntnisderWahrheit1978}:
\enquote{Hierdurch wird die gemeine Erkäntniß von der Erkäntniß eines
Welt-Weisen unterschieden.} Dass er die gemeine Erkenntnis mit der historischen
identifiziert wird einerseits dadurch deutlich, dass ihr Begriff als
Gegenbegriff zur Erkenntnis eines Weltweisen verwendet wird, andererseits
erhellt es aus der 1740 publizierten Übersetzung der \titel{Deutsche[n] Logik}
in die lateinische Sprache; \cite[vgl.][\S~6]{Wolff:Cogitationesrationalesdeviribusintellectushumani1983}:
\enquote{Atque hoc ipso \ori{cognitio philosophica a communi} (quam {\&} \ori{historicam} dicimus) differt.}}
Im \titel{Discursus praeliminaris} bildet die \emph{cognitio communis} eine Untergruppe
der historischen Erkenntnis. Diese -- die \emph{cognitio historica} --
zerfalle in die \emph{cognitio (historica) communis} und die
\emph{cognitio (historica) arcana}, die \singlequote{verborgene} Erkenntnis:
\begin{quote}
  Die \ori{gemeine historische Erkenntnis} [(\emph{cognitio historica
  communis})] ist die Erkenntnis der Tatsachen der Natur (auch der rationalen), die offenbar sind, die \ori{verborgene
  historische Erkenntnis} [(\emph{cognitio historica arcana})] aber ist die
  Erkenntnis der Tatsachen der Natur (auch der rationalen), die verborgen
  sind.\footnote{\Cite[][\S~21]{Wolff:Discursuspraeliminarisdephilosophiaingenere1996}:
  \enquote{\ori{Cognitio historica communis} est cognitio factorum naturae,
  etiam rationalis, quae patent: \ori{Cognitio} autem \ori{historica arcana
  est cognitio factorum naturae, etiam rationalis, quae latent.}}}
\end{quote}
Den Herausgebern \authorfullcite{Gawlick:Einleitung1996} folgend bezeichne ich
die \emph{cognitio communis} als \enquote{gemeine (historische) Erkenntnis} und die
\emph{cognitio vulgaris} als \enquote{gewöhnliche Erkenntnis}. Gemeine
historische Erkenntnis liegt dort vor, wo wir lediglich hinschauen (oder
hinhören, fühlen etc.) müssen, um zu erkennen, was ist oder geschieht. So fühlen
wir etwa, dass die Sonne uns wärmt, ohne erst einen Schluss auf diese Tatsache
vollziehen zu müssen. Bei der gemeinen historischen Erkenntnis handelt es sich
also um \emph{nicht-inferentielles Wissen von Tatsachen} auf der Grundlage von
Erfahrung. Dagegen liegt verborgene historische Erkenntnis dort vor, wo wir
ausgehend von einer Erkenntnis, die uns direkt zugänglich ist, auf eine Tatsache
erst schließen müssen. Eine verborgene historische Erkenntnis besteht
beispielsweise in dem Wissen, dass weißes Licht polychromatisch ist. Das
bekannte Experiment mit einem Prisma zeigt, dass weißes Licht in seine
Spektralfarben zerlegt werden kann und aus einer Mischung von monochromatischem
Licht verschiedener Farben besteht.\footnote{So lautet das Beispiel in
\cite[][\S~20]{Wolff:Discursuspraeliminarisdephilosophiaingenere1996}.} Auf
diese Tatsache müssen wir ausgehend von unserer Beobachtung bei dem Experiment
jedoch erst \emph{schließen}, wir sehen dies nicht einfach. Es handelt sich um
\emph{inferentielles Erfahrungswissen von Tatsachen}, also Wissen, das
inferentiell ist, aber letztlich auf (nicht-inferentiellen) Wahrnehmungsurteilen
aufbaut und von Tatsachen
handelt.\footnote{\phantomsection\label{Anmerkung:CognitioArcanaCognitioCommunis}
Mit Hilfe von Experimenten und dem Einsatz technischer Geräte könne jedoch die \emph{cognitio arcana} in eine \emph{cognitio communis} verwandelt werden \parencite[vgl.][\S~24]{Wolff:Discursuspraeliminarisdephilosophiaingenere1996}.
Im Alltag ist -- um
\authorcite{Wolff:Discursuspraeliminarisdephilosophiaingenere1996}s Beispiel
aufzugreifen -- die Elastizität der Luft nicht zu merken, das Wissen darum ist
verborgen und müsste aus anderen Erkenntnissen geschlossen werden.
Nimmt man jedoch eine Luftpumpe und verschließt die Ventile, dann merkt man beim Hineindrücken
des Kolbens sehr leicht, dass die enthaltene Luft komprimiert wird und sich
wieder ausdehnt, sobald wir die Kraft von außen auf den Kolben reduzieren. Ein
neueres Beispiel für die Transformation inferentiellen in nicht-inferentielles
Wissen ist die Nebelkammer, mit deren Hilfe sich Teilchen, die sonst nicht
sichtbar sind, anhand einer \singlequote{Nebelspur} sichtbar werden. Es ist
nicht leicht zu entscheiden, wie überzeugend diese Auffassung
\authorcite{Wolff:Discursuspraeliminarisdephilosophiaingenere1996}s ist. Ob es
sich dabei dann tatsächlich um nicht-inferentielles Wissen (\emph{cognitio
communis}), dass ein entsprechendes Teilchen die Nebelkammer passiert, oder ob
es sich noch immer um inferentielles Wissen (\emph{cognitio arcana}) handelt,
ist ein Streitpunkt in der Diskussion zwischen
\authorfullcite{McDowell:MindandWorld1994} und
\authorfullcite{Brandom:MakingItExplicit1994}, wobei
\authorcite{Brandom:MakingItExplicit1994} die Position
\authorcite{Wolff:Discursuspraeliminarisdephilosophiaingenere1996}s einnimmt
\parencite[vgl.][141]{McDowell:BrandomonObservation2010}.}


Der Verweis auf rationale Erkenntnis ist an dieser Stelle nicht
erkenntnistheoretisch zu verstehen, sondern verweist darauf, dass es sich auch
hier (wie allgemein bei historischen Erkenntnissen) sowohl um Erkenntnis dessen
handelt, was in der materiellen Welt geschieht, als auch um
Erkenntnis mentaler Vorgänge; letztere -- es ist nicht ganz klar, ob die mentalen Vorgänge oder die
Erkenntnis von ihnen -- nennt er
\singlequote{rational}.\footnote{\enquote{Addo vocem \ori{rationalis}, ut intelligitur, ad facta naturae hic
quoque referri ea, quae in substantiis immaterialibus finitis, veluti mentibus
nostris, accidunt, {\&} cognitionis historicae non minus objectum sunt, quam
quae in mundo materiali contingunt (\S~3)}
\parencite[][\S~21]{Wolff:Discursuspraeliminarisdephilosophiaingenere1996}.}
Auch diese Vorgänge können sowohl offensichtlich als auch verborgen sein.


Historische Erkenntnis ist (inferentielles oder nicht-inferentielles)
Erfahrungswissen davon, was ist oder geschieht. Philosophische Erkenntnis geht
über dieses Erfahrungswissen hinaus, insofern sie nach den \emph{Gründen} dessen
fragt, was ist oder geschieht. Während das einfache Volk sich mit dem
historischen Wissen zufrieden gebe, dass Wasser siedet, wenn es über eine offene
Flamme gesetzt wird, frage der Philosoph (oder derjenige, der nach
philosophischer Erkenntnis strebt), warum dies
geschieht.\footnote{\cite[Vgl.][\S~23]{Wolff:Discursuspraeliminarisdephilosophiaingenere1996}.}
Ich zitiere \authorcite{Wolff:Discursuspraeliminarisdephilosophiaingenere1996}
wiederum mit der Grundlage und der Definition dieser Erkenntnisart:
\begin{quote}
  \ori{Die Dinge, die sind oder geschehen, ermangeln nicht ihres Grundes, aus
  dem erkannt wird, warum sie sind oder
  geschehen.}\footnote{\Cite[][\S~4]{Wolff:Discursuspraeliminarisdephilosophiaingenere1996}:
  \enquote{\ori{Ea, quae sunt vel fiunt, sua non destituuntur ratione, unde
  intelligitur, cur sint, vel fiant.}}}\\
  Die \ori{Erkenntnis} des Grundes der Dinge, die sind oder geschehen, werden
  \ori{philosophisch}
  genannt.\footnote{\Cite[][\S~6]{Wolff:Discursuspraeliminarisdephilosophiaingenere1996}:
  \enquote{\ori{Cognitio} rationis eorum, quae sunt, vel fiunt,
  \ori{philosophica} dicitur.}}
\end{quote}
Philosophische Erkenntnis ist die Erkenntnis des Grundes, aus dem zu erkennen
ist, warum etwas ist oder geschieht, und ihre Grundlage wird durch den Satz des
zureichenden Grundes beschrieben. Weil alles, was ist, einen zureichenden Grund
hat, warum es ist oder geschieht, lässt sich auch jederzeit nach diesem Grund
fragen. Zu jeder historischen Erkenntnis lässt sich somit -- sollte der Satz vom
zureichenden Grunde allgemein gelten\footnote{Allerdings sei es in unserem
Zusammenhang nicht nötig, die streng allgemeine Geltung des Satzes vom
zureichenden Grund vorauszusetzen und zu beweisen
\parencite[vgl.][\S~5]{Wolff:Discursuspraeliminarisdephilosophiaingenere1996}.}
-- eine entsprechende philosophische Erkenntnis finden, die den Grund dessen
enthält, wovon die historische Erkenntnis handelt. Wer etwa weiß, dass ein Stück
Holz auf Wasser schwimmt, während ein Stück Eisen untergeht, hat historische Erkenntnis
dieser Tatsachen. Wer darüber hinaus weiß, dass Holz auf Wasser schwimmt, weil
seine Dichte geringer als die des Wassers ist, während Eisen untergeht, weil
seine Dichte wiederum größer als die des Wassers ist, hat philosophische
Erkenntnis. Zumindest gilt dies \emph{prima facie} unter einer gleich zu
besprechenden Einschränkung (siehe Abschnitt
\ref{subsubsection:HistorischeKenntnisDerPhilosophie} ab
S.~\pageref{subsubsection:HistorischeKenntnisDerPhilosophie}).

Wer über philosophisches Wissen verfügt, ist demjenigen, der lediglich
historische Kenntnisse besitzt, bei der Anwendung seines Wissens überlegen.
Nehmen wir an, Ingrid und Max wissen beide, dass Holz auf Wasser schwimmt und dass Steine
untergehen. Sie verfügen also beide über historisches Wissen bezüglich dieser
beiden Tatsachen. Im Gegensatz zu Max weiß Ingrid aber auch, \emph{warum} dies
so ist -- sie weiß, dass Holz schwimmt, weil seine Dichte geringer ist als die
Dichte des Wassers, und dass Steine untergehen, weil ihre Dichte größer ist als
die des Wassers. Nun sagt
\authorcite{Wolff:Discursuspraeliminarisdephilosophiaingenere1996}, dass die
\emph{Anwendung} der philosophischen Erkenntnis sicherer sei als die der
historischen Kenntnis. (Er sagt explizit nicht, dass das philosophische
\emph{Wissen} sicherer sei, sondern seine \emph{Anwendung}.) Natürlich können
beide -- Ingrid und Max -- ihr Wissen anwenden, indem sie ein Floß aus Holz
bauen und ihre Sachen am Strand mit Steinen beschweren, damit sie bei Flut und
Wellen nicht davongetrieben werden können. Aber Ingrid weiß auch, dass dies
unter der Bedingung gilt, dass Holz eine geringere und Steine eine größere
Dichte haben als Wasser. Sie wird daher ihr Floß nicht aus Azob{\'e} bauen und
ihre Sachen nicht mit Bimsstein
beschweren.\footcite[Vgl.][\S~41]{Wolff:Discursuspraeliminarisdephilosophiaingenere1996}
Außerdem sei die philosophische Erkenntnis in der Anwendung breiter als die
historische: Sollten keine Steine zur Hand sein, wird Ingrid wissen, dass auch
jeder andere hinreichend schwere Gegenstand die Aufgabe der Steine übernehmen
kann. Und ihr wird bewusst sein, dass nicht nur Holz zur Konstruktion eines
Floßes taugt, sondern auch leere Fässer oder Materialien mit geringer
Dichte.\footcite[Vgl.][\S~42]{Wolff:Discursuspraeliminarisdephilosophiaingenere1996}
In der Regel erklärt philosophische Erkenntnis die Tatsachen, von denen wir
historische Erkenntnis haben, indem sie sie unter einen allgemeineren Begriff
subsumiert und eine allgemeinere Tatsache als Erklärung anführt. Durch diese
größere Allgemeinheit benötigen wir weniger Erkenntnisse, um über mehr Fälle
Bescheid zu
wissen.\footcite[Vgl.][\S~43]{Wolff:Discursuspraeliminarisdephilosophiaingenere1996}
Es hat also ganz praktische Vorteile, von der historischen Erkenntnis zur
philosophischen fortzuschreiten.


Während das erste Kapitel des \titel{Discursus praeliminaris de philosophia in
genere} die drei Erkenntnisarten \emph{cognitio historica}, \emph{cognitio
philosophica} und \emph{cognitio mathematica} unterscheidet und in Relation
zueinander setzt, exponiert das zweite Kapitel den Begriff
\enquote{\emph{philosophia}}. Die Übersetzung mit dem deutschen
\enquote{Philosophie} wird in vielen Situationen unpassend klingen, denn
\authorcite{Wolff:Discursuspraeliminarisdephilosophiaingenere1996}s
Philosophiebegriff unterscheidet sich durch seine Weite grundlegend von unserer
heutigen Verwendung. Es wäre zu überlegen, andere Ausdrücke -- etwa
\enquote{Gelehrter} für \enquote{\emph{philosophus}} und \enquote{Weltweisheit}
für \enquote{philosophia} -- zu verwenden, was jedoch terminologische
Folgeprobleme mit sich brächte. Ich bleibe daher bei Ausdrücken
\enquote{Philosophie} und \enquote{Philosoph}, die ich mitunter in einfache
Anführungszeichen setze, wenn betont werden soll, dass es sich um \authorcite{Wolff:Discursuspraeliminarisdephilosophiaingenere1996}s
Terminologie handelt, und eine Verwechslung erhebliche Missverständnisse zur
Folge hätte.


Philosophie sollte dem Wort nach sicherlich diejenige Wissenschaft sein, die
philosophische Erkenntnis zu ihrem Inhalt hat. Und als Philosophen
bezeichnen wir denjenigen, der über philosophisches Wissen verfügt oder nach ihm
strebt. Doch \authorcite{Wolff:Discursuspraeliminarisdephilosophiaingenere1996}s
Definition von \enquote{\emph{philosophia}} ist etwas komplexer aufgebaut, als
zu erwarten wäre. Ich zitiere die Definition zusammen mit den beiden
anschließenden Paragraphen, die für ein Verständnis der Definition nötig sind:
\begin{quote}
  {[\S~29:]} \ori{Philosophie} ist die Wissenschaft der möglichen
  Dinge, wie und warum sie möglich
  sind.\footnote{\label{Anmerkung:UebersetzungDefinitioPhilosophieWolffDiscursus}\Cite[][\S~29]{Wolff:Discursuspraeliminarisdephilosophiaingenere1996}:
  \enquote{\ori{Philosophia} est scientia possibilium, quatenus esse
  possunt.} \authorcite{Gawlick:Einleitung1996} übersetzen stattdessen:
  \enquote{\ori{Philosophie} ist die Wissenschaft des Möglichen, insofern es
  sein kann.} Sie bleiben also näher am Wortlaut. Ich habe die Übersetzung
  gewählt, die bereits
  \authorcite{Stiebritz:ErlaeuterungenderVernuenftigenGedanckenvondenKraefftendesmenschlichenVerstandesWolffs1977}
  als solche angibt
  \parencite[vgl.][\S~29]{Stiebritz:ErlaeuterungenderVernuenftigenGedanckenvondenKraefftendesmenschlichenVerstandesWolffs1977},
  weil sie zum einen den Zusammenhang von philosophischer Erkenntnis und
  Philosophie transparenter macht und zum anderen auch
  \authorcite{Wolff:Discursuspraeliminarisdephilosophiaingenere1996} selbst
  vorzuschweben scheint \parencite[vgl.][Vorbericht, \S~1]{Wolff:VernuenftigeGedankenvondenKraeftendesmenschlichenVerstandesundihremrichtigenGebraucheinErkenntnisderWahrheit1978}:
  \enquote{Die Welt-Weisheit ist eine Wissenschaft aller möglichen Dinge, wie
  und warum sie möglich sind.} Der Text des \titel{Discursus praeliminaris} ist
  wörtlich übernommen in die Übersetzen dieses Zitats; \cite[vgl.][Prolegomena,
  \S~1]{Wolff:Cogitationesrationalesdeviribusintellectushumani1983}:
  \enquote{\ori{Philosophia} est scientia possibilium, quatenus esse
  possunt.} Die erste Erwähnung findet diese Definition der Philosophie bereits
  1709 in:
  \cite[][Praefatio]{Wolff:AerometriaeelementainquibusaliquotAerisviresacproprietatesjuxtamethodumGeometrarumdemonstrantur1981}:
  \enquote{Philosophiam ego definire soleo per rerum possibilium, qua talium,
  scientiam.} Es handelt sich nach allgemeiner Auffassung um
  \authorcite{Wolff:Discursuspraeliminarisdephilosophiaingenere1996}s eigene
  Neuschöpfung, die er 1705 fand und nach einer in Briefen geführten Diskussion
  mit Caspar \name[Caspar]{Neumann} 1709 in den \titel{A{\"e}rometriae elementa} erstmal
  öffentlich kommunizierte
  \mkbibparens{\cite[vgl.][\S~29]{Wolff:Discursuspraeliminarisdephilosophiaingenere1996},
  \cite[][\S~29]{Stiebritz:ErlaeuterungenderVernuenftigenGedanckenvondenKraefftendesmenschlichenVerstandesWolffs1977},
  sowie \cite[][xxvii]{Gawlick:Einleitung1996}}.}\\{}
  {[\S~30:]} Unter \ori{Wissenschaft} verstehe ich hier die Fertigkeit, seine
  Behauptungen zu beweisen, das heißt, sie aus gewissen und unerschütterlichen Grundsätzen
  durch gültigen Schluß
  herzuleiten.\footnote{\Cite[][\S~30]{Wolff:Discursuspraeliminarisdephilosophiaingenere1996}:
  \enquote{Per \ori{Scientiam} hic intelligo habitum asserta demonstrandi,
  hoc est, ex principiis certis {\&} immotis per
  legitimam consequentiam inferendi.} \cite[Vgl.][Vorbericht, \S~2]{Wolff:VernuenftigeGedankenvondenKraeftendesmenschlichenVerstandesundihremrichtigenGebraucheinErkenntnisderWahrheit1978}:
  \enquote{Durch die Wissenschaft verstehe ich eine Fertigkeit des
  Verstandes, alles, was man behauptet, aus unwidersprechlichen Gründen
  unumstößlich darzutun.} Der Text des \titel{Discursus praeliminaris} ist
  wiederum fast wörtlich in der Übersetzen dieses Zitats übernommen;
  \cite[vgl.][Prolegomena, \S 2]{Wolff:Cogitationesrationalesdeviribusintellectushumani1983}:
  \enquote{Per \ori{scientiam} intelligo habitum asserta demonstrandi, hoc
  est, ex principiis certis {\&} immotis per legitimam consequentiam
  inferendi.}}\\{}
  {[\S~31:]} \ori{In der Philosophie ist der Grund anzugeben, warum die
  möglichen Dinge Wirklichkeit erlangen können.}
  Philosophie nämlich ist die Wissenschaft der
  möglichen Dinge, wie und warum sie möglich sind (\S~29). {\punkt} Wer aber
  beweist, warum etwas geschehen kann, der gibt den Grund an, warum dies geschehen kann: Ein
  Grund ist nämlich dasjenige, woraus verstanden wird, warum etwas anderes
  ist.\footnote{\Cite[][\S~31]{Wolff:Discursuspraeliminarisdephilosophiaingenere1996}:
  \enquote{\ori{In philosophia reddenda est ratio, cur possibilia actum
  consequi possint.} Philosophia enim est scientia possibilium, quatenus esse
  possunt (\S~29). {\punkt} Enimvero qui demonstrat, cur aliquid fieri possit,
  is rationem reddit, cur id fieri queat: ratio enim id est, unde
  intelligitur, cur alterum sit.} \cite[Siehe
  auch][Praefatio]{Wolff:AerometriaeelementainquibusaliquotAerisviresacproprietatesjuxtamethodumGeometrarumdemonstrantur1981}:
  \enquote{Philosophi igitur est, non solum nosse, quae fieri possint, quae non;
  sed {\&} rationes perspicere, ob quas aliquid fieri potest, vel esse nequit.}}
\end{quote}
Das Definiens hat die nicht ganz einfache Form \enquote{ein $X$
(Wissenschaft) aller $Y$ (möglichen Dinge), insofern sie $Z$
(möglich/wirklich werdend) sind}. Klar ist, dass $X$ hier das \emph{genus}
angibt, welches in \S~30 weiter erläutert wird, und dass zu diesem \emph{genus}
eine \emph{differentia specifica} zu identifizieren ist.
Dabei ist es ratsam, mit
\authorfullcite{Stiebritz:ErlaeuterungenderVernuenftigenGedanckenvondenKraefftendesmenschlichenVerstandesWolffs1977}
in dieser Definition zwischen einer \emph{differentia specifica
materialis} (den Gegenstandsbereich, $Y$, \enquote{\emph{possibilium}}) und
einer \emph{differentia specifica formalis} ($Z$, \enquote{\emph{quatenus esse
possunt}}) zu unterscheiden und beides getrennt zu
erläutern.\footnote{\cite[Vgl.][\S~29]{Stiebritz:ErlaeuterungenderVernuenftigenGedanckenvondenKraefftendesmenschlichenVerstandesWolffs1977}.}
Es ist die \emph{differentia specifica formalis}, die in \S~31 nähere
Erläuterung erfährt und sich auf den Begriff der \emph{cognitio philosophica}
bezieht. Ich erläutere die drei Momente der Definition im einzelnen:

\begin{nummerierung}

\item Das \emph{genus} der Philosophie ist \enquote{Wissenschaft}
(\enquote{\emph{scientia}}), wobei Wissenschaft als Fähigkeit oder Kompetenz
(\enquote{habitus}) bestimmt wird. Nicht eine Menge wahrer Erkenntnisse zählt
als Wissenschaft -- sie ist keine (auch keine systematisch geordnete) Sammlung wahrer Sätze --, sondern die je individuelle
Kompetenz zu wissenschaftlichem Arbeiten. \Revision{Die wissenschaftliche
Tätigkeit wird hier als \emph{beweisen} (\enquote{\emph{demonstrare}})
näher beschrieben, was die Fundierung in \emph{letzten Gründen} -- und nicht
bloß die Prüfung (\enquote{\emph{probare}}) durch Kontrolle der nächsten vorausgehenden
Gründe bezeichnet. Der Beweis (\enquote{\emph{demonstratio}}) lässt im
Unterschied zur Prüfung (\enquote{\emph{probatio}}) ausschließlich
Definitionen, Axioma, unzweifelhafte Erfahrungssätze und bereits bewiesene
Aussagen gelten.}\footnote{\Revision{\enquote{\ori{Probatio igitur
probabilis a demonstratione non differt nisi principiis.} Nam demonstrationis
principia sunt definitiones, axiomata, experientiae indubitatae {\&}
propositiones jam demonstratae}
\parencite[][\S~588]{Wolff:PhilosophiarationalissiveLogica1740}. Siehe auch
\cite[][\S~562]{Wolff:PhilosophiarationalissiveLogica1740}.}} Diese Kompetenz
bestimmt \authorcite{Wolff:Discursuspraeliminarisdephilosophiaingenere1996} freilich über eine einheitliche Methodik: \Revision{die mathematische Methode}.\footnote{Siehe hierzu Kapitel \ref{Abschnitt:WolffunddieWissenschaftlichkeitderPhilosophiemoregeometrico}, ab Seite
\pageref{Abschnitt:WolffunddieWissenschaftlichkeitderPhilosophiemoregeometrico}.}
Wie wir in Kapitel \ref{subsection:SelbstdenkenbeiKant} sahen, ist damit die
Grundbedingung des Selbstdenkens nach
\authorcite{Wolff:Discursuspraeliminarisdephilosophiaingenere1996} angegeben:
Ein wirklicher Selbstdenker (\enquote{\emph{eclecticus}}) kann nur sein, wer
über eine methodisch geschulte Vernunft, also über Wissenschaft
(\emph{scientia}) im Sinne eigener Kompetenz (nicht im Sinne des Kennens vieler
wahrer Aussagen) verfügt. Da \emph{scientia} das \emph{genus} ist, ist
Philosophie eine, wenngleich nicht die einzige Wissenschaft.\footnote{Daraus
erhellt schon, dass der Begriff \enquote{\emph{philosophia}} nicht mit dem der Wissenschaft
(\enquote{scientia}) zusammenfällt; auch die Mathematik ist Wissenschaft, aber keine
Philosophie. \authorfullcite{Schneiders:Deusestphilosophusabsolutesummus1986}
behauptet dagegen, nach \authorcite{Wolff:Psychologiaempirica1968} zähle streng
genommen nur die Philosophie als Wissenschaft: \enquote{Die wahre Wissenschaft
beginnt erst mit der Ursachenerkenntnis, mit der cognitio philosophica; ja
Philosophie und Wissenschaft sind sogar mehr oder weniger identisch.} Dagegen
würden \enquote{die beiden anderen Erkenntnisarten zu vorwissenschaftlichem
Faktenkennen bzw. zu bloßer Methodologie {\punkt} degradier[t]}
\parencite[vgl.][15]{Schneiders:Deusestphilosophusabsolutesummus1986}.
Allerdings kann ich nicht erkennen, wie
\authorcite{Schneiders:Deusestphilosophusabsolutesummus1986} dieses Urteil
fundiert. Gerade die Mathematik scheint zwar ihre Relevanz nach
\authorcite{Wolff:Psychologiaempirica1968} von der Philosophie zu erhalten, aber
vor allem methodisch das Paradigma wissenschaftlichen Erkennens darzustellen.}

\item Dass Philosophie die Wissenschaft aller möglichen Dinge
(\enquote{\emph{scientia possibilium}}) ist, bestimmt ihren Gegenstandsbereich.
\authorcite{Wolff:Discursuspraeliminarisdephilosophiaingenere1996}
bestimmt das Mögliche als das, was keinen Widerspruch in sich oder zu etwas anderem,
von dem wir wissen, dass es ist oder geschieht, enthält.\footnote{\enquote{Möglich nenne ich alles, was seyn kan, es mag entweder würcklich da
seyn, oder nicht} \parencite[][Vorbericht,
\S~3]{Wolff:VernuenftigeGedankenvondenKraeftendesmenschlichenVerstandesundihremrichtigenGebraucheinErkenntnisderWahrheit1978}. \enquote{Weil
nichts zugleich seyn und nicht seyn kan (\S . 10.); so erkennet man, daß etwas
unmöglich sey, wenn es demjenigen widerspricht, davon wir bereits wissen, daß es
ist oder seyn kan\punkt{} Woraus man ferner ersiehet, daß \ori{möglich} sey, was
nichts widersprechendes in sich enthält, das ist, nicht
allein selbst neben andern Dingen, welche sind oder seyn können, bestehen kan,
sondern auch nur dergleichen in sich enthält, so neben einander bestehen kan}
\parencite[][\S~12]{Wolff:VernuenftigeGedanckenvonGottderWeltundderSeeledesMenschenauchallenDingenueberhauptDeutscheMetaphysik1983}.
\authorcite{Wolff:Psychologiaempirica1968}s Ausdifferenzierung des Begriffs des Möglichen ist von
\authorfullcite{Baumgarten:Metaphysica---Metaphysik2011} aufgegriffen worden,
der zwischen einem \enquote{in
sich Möglichen} (\emph{possibile in se}) und einem \enquote{hypothetisch
Möglichen} (\emph{possibile hypothetice}) unterscheidet
\mkbibparens{\cite[Vgl.][\S\S~15--18]{Baumgarten:Metaphysica---Metaphysik2011};
\cite[][XVII: 29.16--24, 30.8--23]{Kant:GesammelteWerke1900ff.}}.} Einen
Widerspruch in sich können wir inneren Widerspruch, einen Widerspruch zu etwas
anderem, von dem wir wissen, dass es ist oder geschieht, einen äußeren
Widerspruch nennen. Es ist beispielsweise unmöglich, dass Max sowohl Junggeselle
als auch verheiratet ist (innerer Widerspruch); und ebenso ist es unmöglich,
dass Max mit Ingrid in Urlaub ist, wenn wir bereits wissen, dass Ingrid im
Nachbarraum eine Lehrveranstaltung abhält (äußerer Widerspruch). Dass Max
Junggeselle ist, ist möglich (sofern uns nichts gegenteiliges bekannt ist). Und
dass Max mit Ingrid im Urlaub ist, ist ebenso möglich, insofern wir Ingrid schon
länger nicht gesehen haben (und sie uns vielleicht sogar von ihren Urlaubsplänen
mit Max erzählte). Da wir wissen, dass Pferde nicht fliegen können und Geräte
ohne Energiezufuhr nicht ewig in Bewegung bleiben können, stehen solche Dinge in
Widerspruch zu diesen Tatsachen, von denen wir Erkenntnis haben. Sie sind also
unmöglich. Entsprechend gehören fliegende Pferde und \emph{perpetua mobilia}
nicht zum Gegenstandsbereich der Philosophie.

Dabei stellt die Angabe, Philosophie beziehe sich auf sämtliche
\emph{possibilia}, eine Betonung der Weite des Gegenstandsbereichs in zwei
Hinsichten dar. \emph{Erstens} sagt dies, dass sie über alle wirklichen Dinge
hinausgeht und sich auch auf das bezieht, was nur möglich ist.
Denn -- wie \authorcite{Wolff:Discursuspraeliminarisdephilosophiaingenere1996}
betont -- betrachtet Philosophie nicht nur dasjenige, was tatsächlich ist oder
geschieht, sondern auch das, was sein oder geschehen \emph{kann}, wenngleich die
entsprechende Möglichkeit nicht realisiert ist.\footnote{\cite[Vgl.][sectio II,
caput I,
\S~3]{Wolff:RatiopraelectionumWolfianaruminMathesinetPhilosophiamuniversametOpusHugonisGrotiideJurebellietpacis1735}:
\enquote{Est nempe mihi philosophia scientia omnium possibilium qua talium, ita
ut ad obiectum philosophia referri debeant res omnes, qualescunque esse possunt,
\myemph{sive existant sive non}.}} Wenn sich die Physik etwa mit dem freien Fall
oder der Flugbahn einer Kanonenkugel beschäftigt, dann ist weder vorausgesetzt noch auch
nur von Interesse, ob tatsächlich gerade ein Gegenstand fällt oder eine
Kanonenkugel abgeschossen wurde. Die unmittelbare und sicherlich gewollte Folge
ist, dass sie die Dinge nicht im einzelnen, sondern im allgemeinen
betrachtet.\footnote{\enquote{Unde in philosophia res considerantur in
universali non in singulari} \parencite[][\S~3,
\pno~107\,f.]{Wolff:RatiopraelectionumWolfianaruminMathesinetPhilosophiamuniversametOpusHugonisGrotiideJurebellietpacis1735}.}
\emph{Dieser bestimmte} Flug \emph{dieser bestimmten} Kanonenkugel kommt -- wenn überhaupt --
nur als Instantiierung allgemeiner physikalischer Gesetze in Betracht.

\emph{Zweitens} ist sie nicht auf den Gegenstandsbereich einer einzelnen
Disziplin wie der Physik oder der Psychologie eingeschränkt, sondern umfasst
schließlich den Gesamtbereich unseres Wissens. Explizit zählt
\authorcite{Wolff:Discursuspraeliminarisdephilosophiaingenere1996}
Logik\footcite[Vgl.][\S~61]{Wolff:Discursuspraeliminarisdephilosophiaingenere1996},
Ethik und
Politik\footcite[Vgl.][\S\S~63--65]{Wolff:Discursuspraeliminarisdephilosophiaingenere1996},
Ökonomie\footcite[Vgl.][\S~66\,f.]{Wolff:Discursuspraeliminarisdephilosophiaingenere1996},
Naturrecht\footcite[Vgl.][\S~68]{Wolff:Discursuspraeliminarisdephilosophiaingenere1996},
Technologie\footcite[Vgl.][\S~71]{Wolff:Discursuspraeliminarisdephilosophiaingenere1996},
Ontologie\footcite[Vgl.][\S~73]{Wolff:Discursuspraeliminarisdephilosophiaingenere1996},
Physik\footcite[Vgl.][\S
76]{Wolff:Discursuspraeliminarisdephilosophiaingenere1996},
Kosmologie\footcite[Vgl.][\S
77]{Wolff:Discursuspraeliminarisdephilosophiaingenere1996}, Pneumatik und
Metaphysik\footcite[Vgl.][\S~79]{Wolff:Discursuspraeliminarisdephilosophiaingenere1996},
Meteorologie\footcite[Vgl.][\S~80]{Wolff:Discursuspraeliminarisdephilosophiaingenere1996}
und einige andere auf. \emph{Jede} Erkenntnis und
\emph{jede} handwerkliche Fähigkeit kann philosophisch betrachtet und dadurch
verbessert werden. Philosophie wird damit -- wie
\authorcite{Stiebritz:ErlaeuterungenderVernuenftigenGedanckenvondenKraefftendesmenschlichenVerstandesWolffs1977}
erläutert -- in einem weiteren Sinne betrachtet, insofern sie eine Eigenschaft
oder Fähigkeit des Subjekts bezeichnet (\enquote{\emph{philosophia subiective et
habitualiter considerata}}), nicht eine Menge von Erkenntnissen über einen
bestimmten Gegenstandsbereich (\enquote{\emph{philosophia obiective et
systematice
considerata}}\footnote{\label{Anmerkung:StiebritzZuSubiectiveundObiective}\cite[][\S~44]{Stiebritz:ErlaeuterungenderVernuenftigenGedanckenvondenKraefftendesmenschlichenVerstandesWolffs1977}.}).
Die \emph{differentia specifica materialis} bezieht sich damit auf die
Bestimmung des \emph{genus} als einer Fähigkeit. Wegen ihres Umfang sei es -- so
\authorcite{Wolff:Discursuspraeliminarisdephilosophiaingenere1996} -- auch
niemandem vergönnt, in allen Belangen Philosoph zu
sein;\footcite[Vgl.][\S~48]{Wolff:Discursuspraeliminarisdephilosophiaingenere1996}
es gibt den Universalgelehrten nicht, der zu allen Fragen der Philosophie
kompetent urteilen kann. Betrachtete
\authorcite{Wolff:Discursuspraeliminarisdephilosophiaingenere1996} hingegen die
Philosophie als durch ihren Gegenstandsbereich bestimmt, dann ergäbe sich ein
Begriff von Philosophie, der dem Weltbegriff bei \name[Immanuel]{Kant}s ähnelte,
insofern die so bestimmte Philosophie dasjenige Wissen akkumulierte, welches
jeder Mensch zur erfolgreichen Gestaltung seines eigenen Lebens (zur
Glückseligkeit)
benötigt.\footnote{\cite[Vgl.][\S~44]{Stiebritz:ErlaeuterungenderVernuenftigenGedanckenvondenKraefftendesmenschlichenVerstandesWolffs1977}:
\enquote{Erwäget man aber die Welt-Weisheit obiective, systematice, und also in
engern Verstande: so verstehet man darunter, nach dem angenommenen Gebrauch der
in Reden, nichts anders, als eine durch die blose Vernunft erlangte
Wissenschaft, worinnen einem Menschen deutlich vorgetragen wird, was ihm in
ieden Stande zu seiner Glückseligkeit zu wissen und zu thun nöthig ist.} Siehe
hierzu oben Kapitel \ref{subsection:DieBestimmungdesMenschen}.}
Was \authorcite{Wolff:Discursuspraeliminarisdephilosophiaingenere1996} in seiner
Definition im \titel{Discursus praeliminaris} vorschwebt, ist eine Art
Schulbegriff von Philosophie, wonach sie \enquote{nur als eine von den
Geschicklichkeiten zu gewissen beliebigen Zwecken angesehen
wird.}\footnote{\cite[][B 867]{Kant:KritikderreinenVernunft2003}; \cite[][III:
543.33--34]{Kant:GesammelteWerke1900ff.}.}

Die Definition von Möglichkeit als Widerspruchsfreiheit
suggeriert, dass der Gegenstandsbereich der Philosophie (die \emph{possibilia})
durch bloße logische Konsistenz definiert ist. Doch dann verstehen wir die
\emph{differentia specifica formalis} nicht, wonach Philosophie nach Gründen für
diese Möglichkeit sucht.\footnote{Siehe dazu auch
\cite[][23]{Schneiders:Deusestphilosophusabsolutesummus1986}: \enquote{Allem Anschein nach versucht er [d.\,i.
\authorcite{Wolff:Discursuspraeliminarisdephilosophiaingenere1996}; A.\,G.], mit
seiner Definition der Philosophie als Wissenschaft vom Möglichen zwei Dinge
miteinander zu vereinbaren. Einerseits möchte er die Philosophie (trotz des
betonten Ausgangs von der Erfahrung) von der Fesselung an die Wirklichkeit
befreien. Das Denken muß sich in den Raum des logisch Möglichen
als des widerspruchsfrei Denkbaren erheben können -- gegebenenfalls sogar unter
Vernachlässigung der Wirklichkeit. Andererseits soll die Philosophie gerade
erklären, wie das Wirkliche möglich (ermöglicht) ist, d.\,h. nach seinen Gründen
oder Bedingungen fragen -- wobei Wolff kaum deutliche Unterschiede zwischen
physischer und metaphysischer Ursache macht.}
\authorcite{Schneiders:Deusestphilosophusabsolutesummus1986} fragt allgemein,
ob der Möglichkeitsbegriff bei \authorcite{Wolff:Psychologiaempirica1968}
überhaupt als Grundlage eines Verständnisses seines Philosophiebegriffs dienen
kann, zumal es fraglich sei, \enquote{ob die Erörterung des
Möglichkeitsbegriffs zu klaren Resultaten geführt hat}
\parencite[][9]{Schneiders:Deusestphilosophusabsolutesummus1986}.} Dies
verstehen wir nur, wenn alles, was möglich ist, einen Grund seiner Möglichkeit
hat. Möglich ist dann aber nicht einfach dasjenige, was widerspruchsfrei denkbar
ist, sondern erst das \singlequote{real Mögliche}, also dasjenige, dessen
Verwirklichung in dieser Welt geschieht oder doch geschehen kann -- dasjenige,
von wir wissen können, unter welchen Bedingungen, die ihrerseits
\emph{realiter} möglich sind, es tatsächlich wirklich würde. Der Entwurf, den
die Ent\-wick\-lungs\-ab\-tei\-lung eines Automobilkonzerns vorlegt, handelt von
etwas, das zunächst nicht wirklich, aber (hoffentlich) in diesem Sinne
\emph{realiter} möglich ist. Die Entwickler wissen zugleich, wie es möglich ist -- welche Arbeiten
unternommen werden müssen, um es zur Wirklichkeit zu bringen. Es handelt sich um
eine reale Möglichkeit, insofern beschrieben werden kann, wie ein Prototyp zu
bauen ist. Mit diesem Prototypen tritt die reale Möglichkeit über zur
Wirklichkeit.\footnote{Auf diese Weise trägt
\authorfullcite{Seidl:ArtikelenquoteMoeglichkeit1984} die Unterscheidung von
logischer und realer Möglichkeit als Bestandteil von
\authorcite{Wolff:Psychologiaempirica1968}s wie auch von
\authorcite{Baumgarten:Metaphysica---Metaphysik2011}s Philosophie vor. Die
äußere -- bei \authorcite{Baumgarten:Metaphysica---Metaphysik2011}:
hypothetische
\mkbibparens{\cite[Vgl.][\S~16]{Baumgarten:Metaphysica---Metaphysik2011},
\cite[][XVII: 30.8--12]{Kant:GesammelteWerke1900ff.}} -- Möglichkeit sei eine
solche, die von Ursachen abhänge
\parencite[vgl.][86]{Seidl:ArtikelenquoteMoeglichkeit1984}.
\cite[Siehe auch][17]{Schneiders:Deusestphilosophusabsolutesummus1986}:
\enquote{Möglichsein wird also mit Bezug auf Wirklichsein verstanden, als das
Möglichsein eines zunächst nur möglichen, aber möglicherweise dann auch
wirklichen Dinges: die res possibilis ist als essentia zunächst potentia.}}
Scheinbar möchte
\authorcite{Wolff:Discursuspraeliminarisdephilosophiaingenere1996} behaupten,
dass diese reale Möglichkeit bereits damit garantiert ist, dass der Begriff
(oder eine Konzeption) von etwas keinen allgemeinen (Natur-) Gesetzen
widerspricht (äußere Möglichkeit).

\item Der dritte Aspekt nach \emph{genus} und \emph{differentia specifica
materialis} beantwortet die Frage, in welcher Hinsicht die möglichen Dinge
wissenschaftlich untersucht werden. Der \emph{differentia specifica formalis}
nach ist es Aufgabe der Philosophie zu erläutern, \emph{warum} etwas
\emph{möglich} ist oder sein kann (\enquote{quatenus [possibilia] esse
possunt}). Dabei liegt die Grundlage der philosophischen Erkenntnis in dem Satz
vom zureichenden Grunde, der sagt, dass alles, was \emph{ist}, seinen zureichenden
Grund hat, warum es ist. Aber gilt dies auch von allem, was möglich ist?
\authorcite{Baumgarten:Metaphysica---Metaphysik2011} behauptet, dass jedes
Mögliche als solches einen zureichenden Grund
habe.\footnote{\cite[Vgl.][\S~20]{Baumgarten:Metaphysica---Metaphysik2011},
\cite[][XVII: 31.9--12]{Kant:GesammelteWerke1900ff.}.} Und so erläutert auch
\authorcite{Stiebritz:ErlaeuterungenderVernuenftigenGedanckenvondenKraefftendesmenschlichenVerstandesWolffs1977}
\authorcite{Wolff:Discursuspraeliminarisdephilosophiaingenere1996}s
Definition.\footcite[Vgl.][\S~32]{Stiebritz:ErlaeuterungenderVernuenftigenGedanckenvondenKraefftendesmenschlichenVerstandesWolffs1977}
Nun scheint \authorcite{Wolff:Discursuspraeliminarisdephilosophiaingenere1996}
den Satz vom zureichenden Grunde jedoch stets nur auf \emph{wirkliche} Dinge zu
beziehen. Weder in der Fassung des
\titel{Discursus}\footnote{\enquote{\ori{Ea, quae sunt vel fiunt, sua non
destituuntur ratione, unde intelligitur, cur sint, vel fiant}}
\parencite[][\S~4]{Wolff:Discursuspraeliminarisdephilosophiaingenere1996}.
Der \titel{Discursus} spricht also von den Dingen, die sind oder geschehen
(\enquote{\emph{ea, quae sunt vel fiunt}}) und orientiert sich damit an die
Formulierung der Definition philosophischer Erkenntnis.} noch in der
\titel{Deutschen
Metaphysik}\footnote{\enquote{[S]o muß auch alles, was ist, seinen zureichenden
Grund haben, warum es ist}
\parencite[][\S~30,
\ohio]{Wolff:VernuenftigeGedanckenvonGottderWeltundderSeeledesMenschenauchallenDingenueberhauptDeutscheMetaphysik1983}.} noch
auch in der \titel{Philosophia prima sive ontologia}\footnote{Die lateinische
Metaphysik formuliert das Prinzip negativ: \enquote{nil sit sine ratione
sufficiente, cur potius sit, quam non sit}
\parencite[][\S~71]{Wolff:Philosophiaprimasiveontologia1977}.} ist von
\emph{möglichen} Dingen die Rede.
Aber dennoch scheint die Deutung, dass er auch auf mögliche Dinge anwendbar ist,
nicht ganz abwegig zu sein; zumindest ist es auch die Interpretation, die
\authorcite{Stiebritz:ErlaeuterungderWolffschenVernuenfftigenGedanckenvonGottderWeltundderSeeledesMenschenauchallenDingenueberhaupt1999}
zu \authorcite{Wolff:Discursuspraeliminarisdephilosophiaingenere1996}s
Metaphysik anführt: Der Satz vom zureichenden Grund spreche von allem, was
\emph{ist}, nicht von allem, was \emph{wirklich} ist. Und dieses \enquote{ist}
lasse sich auf mögliche so gut wie auf wirkliche Dinge beziehen.\footnote{\enquote{Der Satz heisset:
alles, was ist, hat seinen zureichenden Grund, warum es ist. Was da ist, ist entweder möglich
allein, oder auch wircklich. Darum gehet dieser Satz beydes auf mögliche, als
auch auf wirckliche Dinge. Und von beyden wird bekräftiget, daß iederzeit etwas
da seyn müste, woraus ich bey einem möglichen verstehen kan, woraus es vielmehr
möglich, als unmöglich; vielmehr auf die Art, als auf eine iede andre Art
möglich sey}
\parencite[][\S~75]{Stiebritz:ErlaeuterungderWolffschenVernuenfftigenGedanckenvonGottderWeltundderSeeledesMenschenauchallenDingenueberhaupt1999}.}
Und in der Tat spricht
\authorcite{Wolff:Discursuspraeliminarisdephilosophiaingenere1996} in seinen
Definitionen dieses Satzes zwar nicht von möglichen Dingen, grenzt dessen
Anwendungsbereich aber auch nicht explizit auf die wirklichen Dinge
ein.\footnote{Er erläutert den Satz vom zureichenden Grund zwar in der
\titel{Deutschen Metaphysik} als
\enquote{es muß allezeit etwas seyn, daraus man verstehen kan, warum es würcklich werden kan}, aber dabei ist offensichtlich nicht von \enquote{wirklich sein}, sondern von \enquote{wirklich werden können}
die Rede, was nicht voraussetzt, dass etwas bereits wirklich \emph{ist}.} Somit
bliebe die Option bestehen,
\authorcite{Wolff:Discursuspraeliminarisdephilosophiaingenere1996} so zu
interpretieren, dass das \enquote{alles, was
ist,}\footcite[][\S~30]{Wolff:VernuenftigeGedanckenvonGottderWeltundderSeeledesMenschenauchallenDingenueberhauptDeutscheMetaphysik1983},
von dem das \emph{principium rationis sufficientis} sagt, es habe einen
zureichenden Grund, sich auf alle wirklichen und möglichen Dinge
bezieht. Dies mag zunächst etwas gezwungen erscheinen, hat aber
den Vorteil, dass seine Definition von Philosophie als der Wissenschaft der möglichen Dingen, wie und
warum sie möglich sind, verständlich wird. \Revision{Und tatsächlich sagt
\authorcite{Wolff:Discursuspraeliminarisdephilosophiaingenere1996} in der
\titel{Philosophia prima sive Ontologia}, dass der Begriff des Seienden
(\enquote{\emph{ens}}) nicht nur das Wirkliche, sondern ebenso das Mögliche
umfasst.\footnote{\Revision{\enquote{[\ori{Q}]\ori{uod possibile est, ens
est}} \parencite[][\S~135]{Wolff:Philosophiaprimasiveontologia1977}. Siehe auch
\cite[][\S\S~102, 133\,f.]{Wolff:Philosophiaprimasiveontologia1977}.}} Und
insofern ist das Prinzip vom zureichenden Grund ebenso auf die
\emph{possibilia} anwendbar.}

\end{nummerierung}


\begin{comment}
Solche Fragen nach dem \enquote{Warum?} machen deutlich, dass
\authorcite{Wolff:Psychologiaempirica1968} in der Philosophie primär an
(letztlich empirisches) Wissen über Ursachen denkt. Tatsächlich ist die
Bandbreite der Disziplinen, die \authorcite{Wolff:Psychologiaempirica1968} zur
Philosophie zählt, jedoch noch größer:Somit transzendiert sein Philosophiebegriff sowohl unseren heutigen Begriff
genuin \emph{philosophischen} Wissens, als auch den der an der kausalen
Erklärung des Geschehenes in der Welt interessierten Naturwissenschaft.
Dies funktioniert in \authorcite{Wolff:Psychologiaempirica1968}s System, weil dieser
nicht -- wie nach ihm \name[Immanuel]{Kant} -- zwischen empirischen und
Vernunftwissenschaften unterscheidet, sondern jede Wissenschaft letztlich
empirisch fundiert sieht. Man beachte, dass nach
\authorcite{Wolff:Psychologiaempirica1968} (wie auch nach
\authorcite{Baumgarten:Metaphysica---Metaphysik2011}) selbst die Logik eine
empirische Grundlage (in der Psychologie) hat. Erst \name[Immanuel]{Kant} setzt
hier eine klare Trennung durch, die in der Folge auch verhindert,
den Begriff der Philosophie so breit zu
fassen. Denn es handelt sich um ganz unterschiedliche Fragen, die sich
empirische Psychologie und Logik bezüglich des menschlichen Denkens
stellen.\footnote{\cite[Vgl.][A 6]{Kant:ImmanuelKantsLogik1977}.} Während die
(empirische) Psychologie das \enquote{Warum?} als Frage nach den kausalen
Ursachen mentaler Prozesse deutet, versteht die Logik das \enquote{Warum?} als
Frage nach dem Grund der Berechtigung eines geistigen Aktes wie beispielsweise
eines Schlusses. Bei \authorcite{Wolff:Psychologiaempirica1968} ist
\enquote{philosophia} oder \enquote{Weltweisheit} allgemein der Oberbegriff zu
all denjenigen Wissenschaften, die sich um die Beantwortung von
\enquote{Warum}-Fragen bemühen, wobei dieses \enquote{warum?} zumindest aus
unserer nach-kantischen Perspektive in unterschiedlicher Bedeutung genommen zu
werden scheint, insofern wir einerseits von \enquote{Ursache}, anderseits von
\enquote{Grund} sprechen. Auch \authorcite{Wolff:Psychologiaempirica1968} ist
sich dessen bewusst, entscheidet sich aber ganz bewusst für die Verwendung von
\enquote{Grund} mit ausgeweiteter Bedeutung.\footnote{\cite[Vgl.][\S
13]{Wolff:VernuenftigeGedanckenvonGottderWeltundderSeeledesMenschenauchallenDingenueberhauptandererTheilbestehendinausfuehrlichenAnmerkungen1983}:
\enquote{Das Wort \ori{Raison} oder \ori{Ratio} ist allgemeiner, als das Wort
\ori{Causa} oder Ursache, und hat etwas mehrers, als dieses, zu sagen. Ich gebe
also dem Worte \ori{Ursache} die Bedeutung, welche die Weltweisen dem Worte
\ori{Causa} gegeben haben, und, weil ich demnach ein anderes dazu nöthig gehabt,
wenn ich in den Fällen Teutsch reden will, wo der Franzose \ori{Raison} fordert,
der Lateiner \ori{Rationem} wissen will, so nehme ich das Wort \ori{Grund} dazu,
weil dasselbe in dergleichen Fällen gebraucht wird.}}

Ein Missverständnis wäre es jedoch anzunehmen, die
philosophische Erkenntnis sei der historischen gegenüber überlegen, weil
letztere die Erfahrung als Grundlage habe und daher nicht so gewiss sei, wie
eine reine Vernunftwissenschaft. Besonders deutlich wird dies dort, wo
\authorcite{Wolff:Psychologiaempirica1968} die \enquote{\emph{psychologia
empirica}}, die nach ihm wie nach
\authorcite{Baumgarten:Metaphysica---Metaphysik2011} zur Metaphysik
gehört,\footnote{\cite[Vgl.][\S
99]{Wolff:Discursuspraeliminarisdephilosophiaingenere1996}, wo die Psychologie
schlechthin (und \emph{a fortiori} auch die empirische Psychologie) zur
Metaphysik gezählt wird;
\cite[][\S\S 501--503]{Baumgarten:Metaphysica---Metaphysik2011}.} sowie die
\enquote{\emph{psychologia rationalis}} einführt und ihr Verhältnis zueinander bestimmt. Denn von der \emph{rationalen} Psychologie sagt er:
\begin{quote}
  Weil dies ein neues und der vorgefaßten Meinung entgegengesetztes Unterfangen
  ist, Neues aber anfangs von den meisten ungern zugestanden wird, war das der
  Hauptgrund dafür, daß ich die rationale Psychologie von der empirischen
  trennte, damit nicht psychologische Erkenntnis ohne Unterschied zurückgewiesen
  würde. {\punkt} Die praktische Philosophie ist von größter Bedeutung: was
  daher von größter Bedeutung ist, wollten wir nicht auf solche Grundsätze
  aufbauen, die bestritten werden können. Aus diesem Grund bauen wir die
  Wahrheiten der praktischen Philosophie nur auf solche Grundsätze auf, die in
  der empirischen Psychologie durch Erfahrung evident festgestellt
  werden.\footnote{\cite[][\S
  112]{Wolff:Discursuspraeliminarisdephilosophiaingenere1996}:
  \enquote{Novus cum sit ausus {\&} praejudicatae opinioni adversus, nova vero
  ab initio a plerisque aegre admittantur; praegnans maxime ratio fuit, cur
  Psychologiam rationalem ab empiricam discernerem, ne psychologica promiscue
  rejicerentur. {\punkt} Philosophia practica est maximi momenti: quae igitur
  maximi sunt momenti, istiusmodi principiis superstruere noluimus, quae in
  disceptationem vocantur. Ea de causa veritates philosophiae practicae non
  superstruimus nisi principiis, quae per experientiam in Psychologia empirica
  evidenter stabiliuntur.}}
\end{quote}
Die historische Erkenntnis liefert gerade dadurch, dass sie der Erfahrung
entspringt, \enquote{eine feste und unerschütterliche
Grundlage.}\footnote{\cite[][\S
11]{Wolff:Discursuspraeliminarisdephilosophiaingenere1996}:
\enquote{\dots\unkern firmo ac inconcusso nititur fundamento.}} Gerade deswegen
sollen wir nach \authorcite{Wolff:Psychologiaempirica1968} die \singlequote{Ehe}
zwischen historischer und philosophischer Erkenntnis
heiligen.\footnote{\cite[Vgl.][\S
12]{Wolff:Discursuspraeliminarisdephilosophiaingenere1996}:
\enquote{\dots\unkern{}nobis per omnem philosophiam sanctum est utriusque
connubium.}} Denn beide stehen nicht unverbunden nebeneinander, sondern
philosophische Erkenntnis basiert auf historischer Erkenntnis. \enquote{\ori{Wenn durch Erfahrung dasjenige festgestellt wird, woraus sich anderes, was ist und geschieht oder geschehen
kann, begründen lässt, liefert die historische Erkenntnis die Grundlage der
philosophischen.}}\footnote{\Cite[][\S
10]{Wolff:Discursuspraeliminarisdephilosophiaingenere1996}:
\enquote{\ori{Si per experientiam stabiliuntur ea, ex quibus aliorum, quae sunt
atque fiunt, vel fieri possunt, ratio reddi potest, cognitio historica
philosophicae fundamentum praebet.}}} Wir wissen beispielsweise aus Erfahrung,
dass massive Gegenstände, die in Wasser gelegt werden, auf diesem schwimmen,
während andere massive Gegenstände untergehen. Unsere historische Erkenntnis
besagt also, dass massive Gegenstände schwimmen können. Aus der Erfahrung wissen
wir nun vielleicht, dass genau diejenigen massiven Gegenstände schwimmen, deren
Material eine Dichte aufweist, die kleiner als die des Wassers ist. Auf
Grundlage dieser historischen Erkenntnis erkennen wir, \emph{warum} es
möglich ist, dass massive Gegenstände auf Wasser schwimmen: \emph{weil} ihre
Dichte geringer ist als die des Wassers. Wir haben nun eine philosophische
Erkenntnis der Tatsache, dass massive Gegenstände auf Wasser schwimmen können.
\end{comment}

Während also in
\authorcite{Wolff:Discursuspraeliminarisdephilosophiaingenere1996}s Definition
der Philosophie das \emph{genus} die Philosophie als Wissenschaft auszeichnet,
legen die \emph{differentia specifica materialis} (\enquote{\emph{possibilium}})
ihre Gegenstandsbereich (die \emph{possibilia}) und die \emph{differentia
specifica formalis} die anzustrebende Erkenntnisart (\emph{cognitio
philosophica}) fest. Die \emph{possibilia} sind all diejenigen Dinge und
Geschehnisse, die \emph{realiter} möglich sind. Dass diese \emph{possibilia} in
der Philosophie dahingehend untersucht werden, \enquote{\emph{quatenus esse
possunt}}, heißt, dass die Philosophie philosophische Erkenntnis von ihnen
generiert. Sie ist die Wissenschaft, der es darum geht, von dem, was
\emph{realiter} möglich ist, \emph{Gründe} und \emph{Ursachen}\footnote{Dabei
ist sich \authorcite{Wolff:Discursuspraeliminarisdephilosophiaingenere1996}
durchaus der Tatsache bewusst, dass es einen Unterschied zwischen Gründen und
Ursachen gibt \parencite[vgl.][\S~71]{Wolff:Philosophiaprimasiveontologia1977}. Er nutzt
selbst das Wort \enquote{Grund} als Übersetzung des lateinischen
\enquote{\emph{ratio}} und \enquote{Ursache} für \enquote{\emph{causa}}
\parencite[vgl.][\S~13]{Wolff:VernuenftigeGedanckenvonGottderWeltundderSeeledesMenschenauchallenDingenueberhauptDeutscheMetaphysik1983}.
Siehe dazu auch
\cite[][\S~74]{Stiebritz:ErlaeuterungderWolffschenVernuenfftigenGedanckenvonGottderWeltundderSeeledesMenschenauchallenDingenueberhaupt1999}.}
seines möglichen Seins oder Geschehens anzugeben. Da Wissenschaft als Fähigkeit
zu wissenschaftlich strengem methodischem Vorgehen bestimmt wurde, ist sie damit
die Kompetenz, unter Wahrung der Methode der Vernunft -- \Revision{der
mathematischen Methode} -- philosophische Erkenntnis möglicher Dinge und Sachverhalte zu
erwerben und darzustellen.


\subsection{Historische Kenntnis der
Philosophie}\label{subsubsection:HistorischeKenntnisDerPhilosophie}
Der größte Teil unseres Wissen besteht nach
\authorcite{Wolff:Psychologiaempirica1968} jedoch nicht in philosophischer,
sondern in historischer Erkenntnis.
Das \singlequote{Volk} (\emph{vulgus}) besitze ausschließlich historische
Erkenntnis, welche darum auch die \singlequote{gewöhnliche Erkenntnis}
(\emph{cognitio vulgaris}) genannt werde. Die Philosophie ist nur wenigen
vorenthalten. Allerdings benötigten die meisten Menschen im Leben auch keine
andere Erkenntnis als die historische, denn bei allem Nutzen, den uns die
philosophische Erkenntnis bringt, sei im Alltag doch die historische Erkenntnis
vollends
ausreichend.\footnote{\cite[Vgl.][\S~23]{Wolff:Discursuspraeliminarisdephilosophiaingenere1996},
siehe zum Nutzen der historischen Erkenntnis auch
\cite[][\S~13]{Wolff:Discursuspraeliminarisdephilosophiaingenere1996}.} Sie wird
dadurch nicht herabgewürdigt und kann auch nicht durch eine andere
Erkenntnis ersetzt werden. Die historische Erkenntnis liefert gerade dadurch,
dass sie der Erfahrung entspringt, der Philosophie eine sichere
Basis.\footnote{\cite[Vgl.][\S~11]{Wolff:Discursuspraeliminarisdephilosophiaingenere1996}:
\enquote{\dots\unkern firmo ac inconcusso nititur fundamento.}} Gerade deswegen
sollen wir nach \authorcite{Wolff:Psychologiaempirica1968} die \singlequote{Ehe}
zwischen historischer und philosophischer Erkenntnis
heiligen.\footnote{\cite[Vgl.][\S~12]{Wolff:Discursuspraeliminarisdephilosophiaingenere1996}:
\enquote{\dots\unkern{}nobis per omnem philosophiam sanctum est utriusque
connubium.}} Denn beide stehen nicht unverbunden nebeneinander, sondern
philosophische Erkenntnis basiert auf historischer Erkenntnis und \emph{ergänzt}
dies bei dem gebildeten Teil der Menschheit.

Ich habe in Kapitel \ref{subsection:DescartesKritikantestimonialemWissen}
behauptet, dass \authorcite{Descartes:OeuvresdeDescartes1983}' Unterscheidung
von Wissenschaft (\emph{scientia}) und historischer Kenntnis (\emph{historia})
für die deutsche Aufklärungsphilosophie von Bedeutung sein werde und gerade für
die Frage nach einem mündigen Umgang mit Informationsquellen wichtig werde.
\authorcite{Wolff:Discursuspraeliminarisdephilosophiaingenere1996} liefert uns
eine erste Explikation des Begriffs der \singlequote{historischen Kenntnis}.
Doch was gewinnen wir für die Frage nach einem mündigen Umgang mit
testimonialem Wissen? Was wird aus
\authorcite{Descartes:OeuvresdeDescartes1983}' Kritik der
\singlequote{Büchergelehrsamkeit}? Ich vertrete im folgenden die Auffassung,
dass die Entgegensetzung von \emph{scientia} und \emph{historia} bei
\authorcite{Wolff:Discursuspraeliminarisdephilosophiaingenere1996} -- wie später
bei \name[Immanuel]{Kant} -- als Differenz von \emph{philosophia} und
\emph{cognitio(nes) historica(e) philosophiae} rekonstruiert wird.

\authorcite{Descartes:OeuvresdeDescartes1983} sagt, durch das Lesen von Büchern
erwürben wir keine Wissenschaft, sondern bloß historische Kenntnisse. Doch wie
ist dies zu verstehen? Jeder weiß, dass bei Gewitter der Donner mit einiger
Verzögerung auf den Blitz folgt. Es handelt sich um eine einfache historische
Erkenntnis. Die Wissenschaft erklärt dies nun dadurch, dass die
Schallgeschwindigkeit deutlich geringer ist als die Lichtgeschwindigkeit, und
bewahrt dieses philosophische Wissen in Büchern auf. Wenn ich nun ein solches
Buch aufschlage und lese, dass der Blitz dem Donner vorausgeht, weil die
Lichtgeschwindigkeit größer als die Schallgeschwindigkeit ist, dann -- so sollte
man doch annehmen, erwerbe ich eben jenes philosophische Wissen und damit einen
Teil jener Wissenschaft. Doch diese Annahme ist vorschnell. Wie
\authorcite{Wolff:Discursuspraeliminarisdephilosophiaingenere1996} darlegt,
erwerbe ich auf diese Weise möglicherweise gar keine philosophische Erkenntnis,
sondern historische Kenntnis der Philosophie.

Nach \authorcite{Wolff:Discursuspraeliminarisdephilosophiaingenere1996} können
wir sowohl von der philosophischen Erkenntnis anderer als auch von der
Philosophie historische Erkenntnis
haben.\footnote{\cite[Vgl.][\S\S~8,
50]{Wolff:Discursuspraeliminarisdephilosophiaingenere1996}.} Dass jemand eine
philosophische Erkenntnis hat, ist eine Tatsache.
Und wenn ich weiß, dass etwa \name[Isaac]{Newton} das Fallen schwerer Gegenstände
über die wechselseitige Anziehung von Körpern erklärte, dann habe ich historische Erkenntnis von dieser
philosophischen Erkenntnis (\enquote{\emph{cognitionis philosophicae cognitio
historica}}) \name[Isaac]{Newton}s. Und es ist ebenso eine Tatsache, dass das
Folgen von Donner auf Blitz in der Philosophie durch die unterschiedlichen
Geschwindigkeiten von Licht und Schall erklärt wird. Wer das
weiß, der hat also eine historische Erkenntnis der Philosophie
(\enquote{\emph{cognitio historica philosophiae}}). Jede Einführung in eine
Wissenschaft nimmt ihren Weg anfänglich über solche Erkenntnisse und auch der
fertig ausgebildete Wissenschaftler kommt ohne solche Erkenntnisse nicht aus,
insofern er die Forschungsergebnisse seiner Kollegen rezipiert oder sich in
einen neuen Themenbereich einarbeitet.

Zu den bekanntesten Aspekten von \authorcite{Wolff:Psychologiaempirica1968}s
Philosophie gehört seine Warnung vor einer bloßen \emph{historica
cognitio cognitionis
philosophicae} oder \emph{philosophiae}
\footnote{\cite[Vgl.][\S\S~9,
51]{Wolff:Discursuspraeliminarisdephilosophiaingenere1996}.}, die von
\name[Immanuel]{Kant} gerade in Anspielung an die Schulbildung im Gefolge
\authorcite{Wolff:Discursuspraeliminarisdephilosophiaingenere1996}s
aufgegriffen\footnote{\cite[Vgl.][61--3]{Hinske:ZwischenAufklaerungundVernunftkritik1993}.
\enquote{Es gibt kaum einen Gedanken \authorcite{Wolff:Discursuspraeliminarisdephilosophiaingenere1996}s, den Kant mit größerer Zustimmung
und Leidenschaft aufgenommen und weitergedacht hätte, als \authorcite{Wolff:Discursuspraeliminarisdephilosophiaingenere1996}s Warnung vor
einer bloßen cognitio philosophiae historica, durch die der ursprüngliche
Charakter der philosophischen Erkenntnis in sein blankes Gegenteil verkehrt
wird} \parencite[][62]{Hinske:ZwischenAufklaerungundVernunftkritik1993}.
Siehe \cite[][B~864]{Kant:KritikderreinenVernunft2003}, \cite[][III:
540.37--541.12]{Kant:GesammelteWerke1900ff.}: \enquote{Daher hat der, welcher
ein System der Philosophie, z.\,B.\ das \ori{wolffische}, eigentlich
\ori{gelernt} hat, \punkt\ doch keine andere als vollständige \ori{historische}
Erkenntnis der wolffischen Philosophie\punkt\ Er hat gut gefaßt und behalten,
d.\,i.\ gelernet, und ist ein Gipsabdruck von einem lebenden Menschen.}} und als
zentrales Element in seiner Aufklärungskonzeption integriert
wird. Doch wie sollen wir diese Warnung verstehen?
Denn angenommen, wir haben uns die Inhalte eines Lehrbuchs angeeignet. Besitzen
wir dann nicht philosophische Erkenntnisse? Wir wissen doch dann, wie man die
Phänomene des entsprechenden Teils der Philosophie erklärt. Wenn Ingrid Max
sagt, dass Holz schwimmt, weil seine Dichte geringer ist als die des Wassers,
dann teil sie ihm doch philosophische Erkenntnis mit. Max kann daraufhin auf die
Frage, warum Holz schwimmt, sagen: \enquote{Nun, es schwimmt, weil seine Dichte
geringer ist als die des Wassers. Wäre sie größer, ginge es unter.} Er scheint
nun auch über \wolffsprech{philosophisches} Wissen zu verfügen und in dieser (wenn auch
überschaubar großen) Hinsicht \wolffsprech{Philosoph} zu sein.

Es ließe sich darauf verweisen, dass die Definition der Philosophie sich auf
einen Begriff von Wissenschaft als \emph{genus} stützte, der diese nicht als
Inbegriff gewisser Lehren (\emph{obiective considerata}), sondern als
methodische Kompetenz des Subjekts (\emph{subiective considerata}) bestimmt. Wer
eine Wissenschaft nur durch das Lesen von Büchern erwirbt, der erwirbt damit
nicht die entsprechenden Fähigkeiten. Und diese (inzwischen auch von
\authorcite{Hobbes:Leviathan1962} bekannte)\footnote{Siehe Kapitel
\ref{subsection:VernunftwahrheitenUndErfahrungstatsachen}.} Strategie ist durchaus korrekt.
Doch um welche Kompetenzen geht es hier?


Es wäre zu kurz gegriffen, nur auf eine allgemeine Methodik zu verweisen und zu
sagen, man müsse die mathematische Methode
(\authorcite{Wolff:Discursuspraeliminarisdephilosophiaingenere1996})
beherrschen oder analog diejenige Methode, die führende
Wissenschaftstheoretiker als \singlequote{die} Methode der Wissenschaften
auszumachen glauben. Die Beherrschung allgemeiner Methoden wissenschaftlichen
Arbeitens ist \emph{ein Teil} der Kompetenzen, die wir erwerben müssen, um über
eine Wissenschaft zu verfügen, es ist nicht das Gesamt der Kompetenzen, zu dem
nur noch die Inhalte hinzugefügt werden müssen. Schließlich würde diese Deutung
zwar erklären, warum Max zwar nur eine historische Kenntnis der
\wolffsprech{Philosophie} erwirbt, wenn er sich Ingrids Erklärung merkt. Aber
sie erklärte nicht, warum er bloß historische Kenntnis der philosophischen
Erkenntnis Ingrids erwirbt und nicht einfach philosophische Erkenntnis; denn in
der Definition der \emph{cognitio philosophica} war von methodischer Strenge und
Wissenschaft keine Rede.

Wenn Max von Ingrid erfährt, dass Holz wegen seiner geringen Dichte schwimmt,
dann -- so behauptet
\authorcite{Wolff:Discursuspraeliminarisdephilosophiaingenere1996} -- erwirbt er
möglicherweise nur historische Kenntnis der Philosophie, aber kein
philosophisches Wissen, weil ihm eine bestimmte \emph{Kompetenz} fehlt. Und
\authorcite{Wolff:Discursuspraeliminarisdephilosophiaingenere1996} bestimmt
auch, welche Kompetenz ihm in diesem Fall fehlt: Max kann zwar den Grund dafür,
dass Holz schwimmt, \emph{benennen}. Um über philosophische Erkenntnis zu
verfügen, müsse man darüber hinaus \emph{beweisen} können, dass dies der Grund
ist: \enquote{{Wenn einer nicht beweisen kann, daß der von einem anderen
angeführte Grund einer Tatsache wirklich ihr Grund ist, hat er keine
philosophische Erkenntnis dieser
Tatsache.}}\footnote{\enquote{{Si quis demonstrare non noverit, rationem facti ab altero
allegatam esse ejus rationem, is cognitione philosophica ejusdem facti
destituitur}} \parencite[][\S~9,
\ohio]{Wolff:Discursuspraeliminarisdephilosophiaingenere1996}.}


Ingrid und Max wissen beide, dass Holz schwimmt, weil seine Dichte geringer ist
als die des Wassers, aber im Gegensatz zu Ingrid besitzt Max nur historische
Kenntnis dieses Sachverhalts. Dies liegt nicht daran, dass Max dieses
Wissen nur aus zweiter Hand hat und Ingrid dies selbst festgestellt hätte.
Ingrid mag dies etwa aus einem Lehrbuch der Physik und damit selbst nur aus
zweiter Hand wissen. Es muss möglich sein, \wolffsprech{philosophisches} Wissen
aus Lehrbüchern zu erwerben, wenn
\authorcite{Wolff:Discursuspraeliminarisdephilosophiaingenere1996}s Darstellung
kompatibel sein soll mit der epistemischen Situation, in der wir uns notwendiger
Weise befinden. Denn dazu gehört, dass wissenschaftliche Ausbildung über
Lehrbücher und Vorlesungen vermittelt wird und wir Vieles von Anderen lernen.
Sähen wir dies stets als bloß historische Kenntnis, könnten wir niemanden als
\wolffsprech{Philosophen} in irgendeiner Disziplin bezeichnen. Wir wären
epistemische Individualisten wie \authorcite{Descartes:OeuvresdeDescartes1983}
und diese Position ist -- so war in Kapitel \ref{section:autonomieunddaszeugnisanderer} zu sehen -- in
keiner Weise befriedigend. Wir müssen davon ausgehen, dass es prinzipiell
möglich ist, philosophisches Wissen durch das Lesen von Büchern oder Auskünfte
von anderen zu erwerben, wenn uns die Unterscheidung von philosophischer und
historischer Erkenntnis in unserem Bestreben, einen Begriff mündigen
Wissenserwerbs zu erarbeiten, irgendwie weiterhelfen soll.

Es ist zunächst instruktiv zu sehen, dass auch aus Erfahrung nichts als
historische Erkenntnis entspringen
kann.\footnote{\enquote{Quae per experientiam stabiliuntur, eorum nonnisi
historica est cognitio} \parencite[][\S~10]{Wolff:Discursuspraeliminarisdephilosophiaingenere1996}.}
Aus Erfahrung wissen wir also, \emph{was} ist und geschieht, jedoch nicht,
\emph{warum} es ist oder geschieht. Doch auch dies ist zunächst
schwer zu verstehen, denn Naturwissenschaften erkennen doch, warum sich die Dinge so
verhalten, wie sie sich nun einmal verhalten, indem sie mit Experimenten und
Beobachtungen -- also \emph{empirisch} -- herausfinden.


Eine wohlfeil zu habende Antwort läge darin zu vermuten, philosophische
Erkenntnis ginge über historische Erkenntnis nach
\authorcite{Wolff:Discursuspraeliminarisdephilosophiaingenere1996} hinaus, indem
sie das, was wir in der historischen Erkenntnis aus Erfahrung wissen, aus
\singlequote{reiner Vernunft} erweist. Wir interpretierten ihn dann als
denjenigen Rationalisten, der die Erfahrung nicht als gültige Grundlage echter
Wissenschaft zu akzeptieren bereit wäre. Jedoch ist
\authorcite{Wolff:Discursuspraeliminarisdephilosophiaingenere1996} kein
Rationalist in dem genannten Sinne.\footnote{Siehe dazu ausführlich
\cite{Kreimendahl:EmpiristischeElementeimDenkenChristianWolffs2007}.
\authorfullcite{Ecole:Enquelssenspeut-ondirequeWolffestrationaliste?1979}
liefert eine sehr differenzierte Darstellung, die rationalistische Elemente im
Denken \authorcite{Wolff:Discursuspraeliminarisdephilosophiaingenere1996}s
abseits des naiven Rationalismusverständnisses beschreibt
\parencite[vgl.][]{Ecole:Enquelssenspeut-ondirequeWolffestrationaliste?1979}.}
Besonders deutlich wird dies dort, wo er die \enquote{\emph{psychologia
empirica}}, die nach ihm wie nach
\authorcite{Baumgarten:Metaphysica---Metaphysik2011} zur Metaphysik
gehört,\footnote{\cite[Vgl.][\S~99]{Wolff:Discursuspraeliminarisdephilosophiaingenere1996},
wo die Psychologie schlechthin (und \emph{a fortiori} auch die empirische
Psychologie) zur Metaphysik gezählt wird;
\cite[][\S\S~501--503]{Baumgarten:Metaphysica---Metaphysik2011}.} sowie die
\enquote{\emph{psychologia rationalis}} einführt und ihr Verhältnis zueinander
bestimmt. Denn von der \emph{rationalen} Psychologie sagt er:
\begin{quote}
  Da dies ein neues und der vorgefaßten Meinung entgegengesetztes
  Unternehmen ist, neue Dinge aber anfangs von den meisten ungern zugestanden
  werden, war der wichtigste Grund, warum ich die rationale Psychologie von
  der empirischen trennte, dass sie nicht ohne Unterschied zurückgewiesen
  würden. {\punkt} Die praktische Philosophie ist von größter Wichtigkeit:
  welche Dinge aber von größter Bedeutung sind, wollten wir nicht auf solche
  Grundsätze aufbauen, die zu Gegenständen einer Streitfrage werden können. Aus
  diesem Grund bauen wir die Wahrheiten der praktischen Philosophie nur auf
  Grundsätzen auf, die durch die Erfahrung in der empirischen Psychologie evident gesichert
  werden.\footnote{\cite[][\S~112]{Wolff:Discursuspraeliminarisdephilosophiaingenere1996}:
  \enquote{Novus cum sit ausus {\&} praejudicatae opinioni adversus, nova vero
  ab initio a plerisque aegre admittantur; praegnans maxime ratio fuit, cur
  Psychologiam rationalem ab empiricam discernerem, ne psychologica promiscue
  rejicerentur. {\punkt} Philosophia practica est maximi momenti: quae igitur
  maximi sunt momenti, istiusmodi principiis superstruere noluimus, quae in
  disceptationem vocantur. Ea de causa veritates philosophiae practicae non
  superstruimus nisi principiis, quae per experientiam in Psychologia empirica
  evidenter stabiliuntur.}}
\end{quote}
Die historische Erkenntnis liefert dadurch, dass sie der Erfahrung
entspringt, \enquote{eine feste und unerschütterliche
Grundlage.}\footnote{\cite[][\S~11]{Wolff:Discursuspraeliminarisdephilosophiaingenere1996}:
\enquote{\dots\unkern firmo ac inconcusso nititur fundamento.}} Gerade deswegen
sollen wir nach \authorcite{Wolff:Psychologiaempirica1968} die \singlequote{Ehe}
zwischen historischer und philosophischer Erkenntnis
heiligen.\footnote{\cite[Vgl.][\S~12]{Wolff:Discursuspraeliminarisdephilosophiaingenere1996}:
\enquote{\dots\unkern{}nobis per omnem philosophiam sanctum est utriusque
connubium.}} Denn beide stehen nicht unverbunden nebeneinander, sondern
philosophische Erkenntnis basiert auf historischer Erkenntnis.
\enquote{\ori{Wenn durch Erfahrung dasjenige festgestellt wird, woraus sich
anderes, was ist und geschieht oder geschehen kann, begründen lässt, liefert die
historische Erkenntnis die Grundlage der
philosophischen.}}\footnote{\Cite[][\S~10]{Wolff:Discursuspraeliminarisdephilosophiaingenere1996}:
\enquote{\ori{Si per experientiam stabiliuntur ea, ex quibus aliorum, quae sunt
atque fiunt, vel fieri possunt, ratio reddi potest, cognitio historica
philosophicae fundamentum praebet.}}}  Generell könne uns zwar Erfahrung nur
historische, aber keine philosophische Erkenntnis liefern. Doch diese
historische Erkenntnis \emph{fundiere} die philosophische
Erkenntnis.\footnote{\enquote{Quae per experientiam stabiliuntur, eorum nonnisi historica est
cognitio (\S~3). Quodsi ergo ex iis rationem reddis aliorum, quae sunt atque
fiunt, cognitionem philosophicam iisdem superstruis (\S~6)}
\parencite[][\S~10]{Wolff:Discursuspraeliminarisdephilosophiaingenere1996}.}


Was zur historischen Erkenntnis hinzukommen muss, damit philosophische
Erkenntnis vorliegt, ist die Einsicht in den \emph{Zusammenhang} von
\emph{explanans} und \emph{explanandum}. Und dieser Zusammenhang lässt sich
durch die genannten Sätze nicht mitteilen; um um ihn zu wissen -- um über
philosophische Erkenntnis zu verfügen --, müssen wir die \emph{Kompetenz}
besitzen, diesen Zusammenhang -- wie
\authorcite{Wolff:Psychologiaempirica1968} sagt -- zu \singlequote{beweisen}
oder -- etwas abseits von \authorcite{Wolff:Psychologiaempirica1968}s Sichtweise
-- zu erläutern. Max und Ingrid können beide sagen, dass Holz wegen seiner
geringen Dichte schwimmt (und dass alle Körper mit einer entsprechend geringen Dichte
schwimmen). Aber wenn wir Max fragen, warum Körper mit einer so geringen Dichte
schwimmen, wird er dies vielleicht nicht beantworten, sondern nur auf Ingrid
verweisen können. Ingrid wiederum wird in der Lage sein zu erläutern, wie es
kommt, dass Körper mit einer geringen Dichte schwimmen. Denn wir nahmen an, dass Ingrid philosophisches
Wissen besitzt. Insofern Max nicht in der Lage ist,
den Zusammenhang zu erläutern, besitzt er keine eigene philosophische
Erkenntnis, sondern historische Erkenntnis der philosophischen Erkenntnis
(\emph{cognitio cognitionis philosophicae historica})
Ingrids\footcite[Vgl.][\S~8]{Wolff:Discursuspraeliminarisdephilosophiaingenere1996}
oder historische Erkenntnis der
Philosophie\footcite[Vgl.][\S~50]{Wolff:Discursuspraeliminarisdephilosophiaingenere1996}.
Beide -- Max und Ingrid -- haben ihr Wissen, warum Holz schwimmt, aus zweiter
Hand. Aber während Ingrid philosophisches Wissen erwarb, bleibt das Wissen bei
Max bloß historisch. Der Unterschied besteht schlicht darin, dass Ingrid den
Zusammenhang von \emph{explanans} und \emph{explanandum} verstanden hat und
erläutern kann, während Max nur die Angabe des Grundes zu wiederholen vermag.



\authorcite{Wolff:Discursuspraeliminarisdephilosophiaingenere1996} beabsichtigt
nicht die Abwertung historischer Erkenntnis der Philosophie, sondern die
begriffliche Differenzierung zwischen historischer und eigener philosophischer
Erkenntnis. Dabei fehlt die eigene philosophische Erkenntnis, solange jemand
nicht zeigen kann, dass der angeführte Grund wirklich die geforderte
Erklärungsleistung erbringen
kann.\footnote{\cite[Vgl.][\S~9]{Wolff:Discursuspraeliminarisdephilosophiaingenere1996}:
\enquote{Si quis demonstrare non noverit, rationem facti ab altero allegatam
esse ejus rationem, is cognitione philosophica ejusdem facti destituitur}
(\ohio ).} Wem diese Erkenntnis fehlt, der hat eine \emph{bloß} historische
Erkenntnis philosophischer Erkenntnis. Nach
\authorcite{Wolff:Psychologiaempirica1968} erkennt man den Mangeln philosophischer Erkenntnis an
der Unfähigkeit zu zeigen, dass der angeführte Grund zureichend
ist.\footnote{\cite[Vgl.][\S~9]{Wolff:Discursuspraeliminarisdephilosophiaingenere1996}:
\enquote{Si quis demonstrare non noverit, rationem facti ab altero allegatam
esse ejus rationem, is cognitione philosophica ejusdem facti destituitur.} Das
muss nicht heißen, dass er -- etwa bei physikalischem Wissen -- die nötigen
Ausgangsdaten selbst ermittelt, etwa selbst Experimente durchführt oder Studien
erstellt. Diese Ausgangsdaten sind historische Kenntnisse und können tradiert
werden. Ansonsten könnte niemand je ein Naturgesetz verstehen, ohne selbst
entsprechende Experimente durchzuführen, was mitunter zu absurden Konsequenzen
führte. (Man denke an Statistiken in den Sozialwissenschaften, Beobachtungen in
der physikalischen Kosmologie mit ihrem notorisch hohen technischen Aufwand oder
Fossilienfunde.) Dass die angeführten Gründe selbst wieder nur historisch
gewusst werde, ist kein Mangel, denn es handelt sich um Tatsachen. Aber ihr
Nexus mit dem, was sie erklären, ist Gegenstand philosophischen Wissens
\parencite[vgl.][\S~10]{Wolff:Discursuspraeliminarisdephilosophiaingenere1996}.}
Man muss den Zusammenhang erläutern können. Wer etwas versteht, der besitzt eine Reihe
von Fähigkeiten, zu denen auch, aber nicht nur die Artikulation des
Verstandenen gehört. Er hat vielmehr selbst Kompetenzen eines Naturforschers.
Und darum ist nicht derjenige aufgeklärt, der viel gelernt und behalten hat, was
er verbal reproduzieren kann, sondern derjenige, der viel verstanden und
entsprechende Kompetenzen erworben hat. Aufklärung in der
wolffschen Ausprägung geht daher darauf, die Welt um uns herum
nicht nur zu kennen, sondern zu verstehen, und stellt daher den Naturforscher
mit seinen Kompetenzen in den Mittelpunkt der Aufklärungsbemühungen.



\authorcite{Wolff:Discursuspraeliminarisdephilosophiaingenere1996} ist bewusst,
dass jeder Einzelne mit der Aufgabe, in allen Bereichen philosophisches Wissen zu haben, überfordert
wäre.\footnote{\cite[Vgl.][\S~48]{Wolff:Discursuspraeliminarisdephilosophiaingenere1996}:
\enquote{\ori{Nemo in omnibus philosophus}}.} Deswegen ist es nicht ehrenrührig,
in vielen Dingen bloß historische Erkenntnis der Weltweisheit zu haben. Wir sind
damit noch immer gebildeter als ohne jedes Wissen (wenngleich wir uns
eingestehen müssen, dass uns in vielen Fällen selbst das historische Wissen um
die Weltweisheit fehlt.)
Wer historische Kenntnis der Philosophie hat, hat in der Anwendung oft dieselben
Vorteile wie jemand, der tatsächlich philosophisches Wissen hat. Nachdem Ingrid
ihm sagte, warum Holz schwimmt und Steine untergehen, kann Max seine
historischen Kenntnisse genauso anwenden wie Ingrid ihr philosophisches Wissen:
Auch er wird sein Floß nicht aus Azob{\'e} bauen und seine Sachen am Strand
nicht mit Bimsstein beschweren. Und er weiß, welche Gegenstände denselben Dienst
verrichten können, wenn kein Holz oder keine Steine zur Hand sind.

Was Ingrid im Gegensatz zu Max vermag ist, philosophische Behauptungen zu
beurteilen.\footnote{\cite[Vgl.][\S~52]{Wolff:Discursuspraeliminarisdephilosophiaingenere1996}.}
Max kann Ingrids Behauptung über den Grund, warum Holz schwimmt, zwar benennen
und auch anderen mitteilen, aber er vermag auf Einwände gegen die Erklärung
nicht zu reagieren. Und wenn Ingrid ihm eine falsche Erklärung genannt hätte,
könnte er dies nicht bemerken. Ingrid wiederum kann beurteilen, welche
Eigenschaften des Holzes die Tatsache erklären, dass es schwimmt, und welche
dies nicht tun. Mündigkeit und Selbstdenken sind Forderungen an denjenigen, der
philosophisches Wissen zu haben beansprucht. Bei historischen Erkenntnissen
wären sie völlig fehl am Platze. Über philosophisches Wissen \emph{kann} aber
nur derjenige verfügen, der methodisch ausgebildet ist. Daraus ergibt sich
\authorcite{Wolff:Discursuspraeliminarisdephilosophiaingenere1996}s Behauptung,
dass Selbstdenken keine Frage des Entschlusses, sondern der Kompetenz ist, die
sich nur auf der Grundlage einer methodischen Schulung entwickeln lasse (siehe
Kapitel \ref{Abschnitt:WolffunddieWissenschaftlichkeitderPhilosophiemoregeometrico}).

Wer sein Wissen lediglich rezipierend aus Büchern aufnimmt, erlangt keine
Wissenschaft, sondern lediglich historische Kenntnis von Wissenschaften. Dies
liegt nicht daran, dass er die Wahrheit der Aussagen nicht selbst kontrolliert hat und diese
daher fraglich wäre -- es gibt kaum bessere Möglichkeiten, die Wahrheit von
Aussagen zu überprüfen, als in einschlägigen wissenschaftlichen Publikationen
nachzuschlagen --, sondern weil er damit noch nicht die entsprechenden
Kompetenzen erworben hat, den Zusammenhang der Wahrheiten zu beurteilen. Das
\singlequote{gemeine Volk} weiß um Ergebnisse der Wissenschaften und es hat --
auch in einem strengen Sinne -- \emph{Wissen}, aber es verfügt nicht über
Wissenschaft, weil es seine Erkenntnisse nicht entsprechend systematisiert. Und
so ist die Forderung nach einem \emph{System} der Wissenschaften zentral für das
entsprechende Aufklärungsverständnis. Denn das, was ist oder geschieht, in
seinem systematischen Zusammenhang darzustellen, heißt nichts anderes als zu
verstehen, warum es ist oder geschieht.

\subsection{Exkurs: Historische Erkenntnis allgemeiner
Tatsachen}\label{subsubsection:HistorischeErkenntnisallgemeinerTatsachen}
Bei den bisherigen Überlegungen bin ich davon ausgegangen, dass
historische Erkenntnis sich auch in allgemeinen Urteilen wie \enquote{\emph{Alle} Steine
sind schwer} äußern kann. Immerhin führt
\authorcite{Wolff:Psychologiaempirica1968} gerade solche Urteile als Beispiele
historischer Erkenntnisse an. Das Vermögen der philosophischen Erkenntnis, die
nicht einzelne Wahrheiten, sondern den Zusammenhang von Wahrheiten thematisiert, ist die Vernunft. Die
\emph{ratio} -- sagt \authorcite{Wolff:Psychologiaempirica1968} entsprechend -- ist das Vermögen, den
\emph{Zusammenhang} allgemeiner Wahrheiten (\emph{nexus veritatum universalium})
zu
erkennen.\footnote{\phantomsection\label{Anmerkung:Wolff:VernunftalsEinsichtinZusammenhangallgemeinerWahrheiten}\cite[Vgl.][\S~483]{Wolff:Psychologiaempirica1968}:
\enquote{\ori{Ratio} est facultas nexus veritatum universalium intuendi seu
perspiciendi.} Siehe auch
\cite[][\S~368]{Wolff:VernuenftigeGedanckenvonGottderWeltundderSeeledesMenschenauchallenDingenueberhauptDeutscheMetaphysik1983}:
\enquote{Die Einsicht, so wir in den Zusammenhang der Wahrheiten haben, oder
das Vermögen den Zusammenhang der Wahrheiten einzusehen, heisset
\ori{Vernunft}.}} Die allgemeinen
Wahrheiten selbst sind Gegenstand historischer Erkenntnis und können durch
eigene Beobachtung und Erfahrung oder durch Mitteilung von anderen erworben
werden.

Die Vernunft, der wir die philosophische Erkenntnis zuordnen können, ist auf der
anderen Seite bei \authorcite{Wolff:Psychologiaempirica1968} auch nicht einfach das Vermögen des Schließens oder des
Verbindens von Wahrheiten schlechthin, sondern des Verbindens
\emph{allgemeiner} Wahrheiten\footnote{Siehe oben, Anmerkung
\ref{Anmerkung:Wolff:VernunftalsEinsichtinZusammenhangallgemeinerWahrheiten},
Seite
\pageref{Anmerkung:Wolff:VernunftalsEinsichtinZusammenhangallgemeinerWahrheiten}.},
die als solche der Erfahrung entstammen. Nach
\authorcite{Meier:Vernunftlehre1752}, der hierin \authorcite{Wolff:Psychologiaempirica1968} folgt,
ist die Erkenntnis, dass \emph{alle} Menschen irren können, für sich noch keine
rationale, sondern eine historische Erkenntnis. Sie wird zur cognitio
rationalis, wenn sie \enquote{auf eine deutliche Art aus Gründen} erkannt
wird.\footnote{\Cite[][\S~17]{Meier:AuszugausderVernunftlehre1752},
\cite[][XVI: 93.17--18]{Kant:GesammelteWerke1900ff.}.} Die Verständlichkeit der
Welt ist nicht schon darin gegeben, dass alles, was geschieht, nach Regeln (allgemeinen Naturgesetzen) geschieht, sondern darin, dass diese allgemeinen
Gesetze selbst wiederum untereinander verbunden und auseinander erkennbar
sind.\footnote{\cite[Vgl.][\S~482]{Wolff:Psychologiaempirica1968}:
\enquote{\ori{Veritates universales inter se connectuntur.}} Und in der Anmerkung
dazu heißt es: \enquote{Veritatum universalium nexus fundatur in nexu rerum, quem in
Cosmologia stabilivimus \punkt\ quemadmodum veritas logica in veritate rerum
transcendentali}.} Hieran knüpft dann auch \name[Immanuel]{Kant} an, wenn er in der \titel{Kritik der
Urteilskraft} das transzendentale Prinzip der reflektierenden Urteilskraft als
eines beschreibt, das \enquote{die Einheit aller empirischen Prinzipien unter
gleichfalls empirischen, aber höheren Prinzipien und also die Möglichkeit der
systematischen Unterordnung derselben untereinander begründen soll.} Erst durch
eine solche Kombination mehrerer \emph{allgemeiner} Erkenntnisse entsteht
\enquote{ein System der Erfahrung nach besonderen
Naturgesetzen}\footnote{\cite[][B xxvii]{Kant:KritikderUrteilskraft2009},
\cite[][V: 180.8--11, 24--25]{Kant:GesammelteWerke1900ff.}.}. Und dies
entspricht dann wiederum dem regulativen Prinzip der systematischen Einheit, wie
es im Anhang zur transzendentalen Dialektik beschrieben und noch der Vernunft,
nicht der Urteilskraft zugeschrieben wird.\footnote{\cite[Vgl.][B
708]{Kant:KritikderreinenVernunft2003}, \cite[][III:
448.22--30]{Kant:GesammelteWerke1900ff.}: \enquote{Die reine Vernunft ist in der
Tat mit nichts als sich selbst beschäftigt, und kann auch kein anderes Geschäfte haben, weil ihr nicht
die Gegenstände zur Einheit des Erfahrungsbegriffs, sondern die
Verstandeserkenntnisse zur Einheit des Vernunftbegriffs, d.\,i.\ des
Zusammenhanges in einem Prinzip gegeben werden. Die Vernunfteinheit ist die
Einheit des Systems, und diese systematische Einheit dient der Vernunft nicht
objektiv zu einem Grundsatze, um sie über die Gegenstände, sondern subjektiv als
Maxime, um sie über alles mögliche empirische Erkenntnis der Gegenstände zu
verbreiten.}} Nicht allgemeine Gesetze schlechthin sind Ausdruck der Vernunft (im engeren
Sinne), sondern die Rückführung verschiedener Naturgesetze auf höhere und ihre
Verbindung in einem einheitlichen System. Es ist genau dies, was -- wie
\authorcite{Wolff:Psychologiaempirica1968} behauptet -- das \distanz{gemeine Volk} (\enquote{vulgus}) nicht
besitzt, welches zwar wisse, dass das Wasser auf dem Feuer siedet, aber nicht,
warum dies so
ist.\footnote{\cite[Vgl.][\S~23]{Wolff:Discursuspraeliminarisdephilosophiaingenere1996}.}
Behauptete er, dass gemeine Volk verfüge nicht über allgemeine Wahrheiten, weil
ihm nur historische Erkenntnisse zugänglich sind, könnten wir dies kaum ernst
nehmen.



Doch gegen diese Interpretation argumentieren unabhängig voneinander
\authorfullcite{Kambartel:ErfahrungundStruktur1968} und
\authorfullcite{Holzhey:KantsErfahrungsbegriff1970} jeweils unter Rückgriff auf
eine Stelle aus der Deutschen
Logik.\footnote{\cite[Vgl.][\pno~57\,f.,]{Kambartel:ErfahrungundStruktur1968}
und \cite[][91]{Holzhey:KantsErfahrungsbegriff1970}.} Diese handelt zwar \emph{prima
facie} nicht von historischer, sondern von Erfahrungserkenntnis; aber da
nach \authorcite{Wolff:Psychologiaempirica1968} der Begriff \enquote{historische
Erkenntnis} mit dem Begriff \enquote{Erkenntnis aus Erfahrung} zumindest
extensionsgleich ist, ist sie für den Begriff historischer Erkenntnis durchaus
einschlägig. Und sie ist hier von einigem Interesse, weil sich die entsprechende
Deutung auf das Verständnis der Konzeption \name[Immanuel]{Kant}s auswirkt
(auch \authorcite{Kambartel:ErfahrungundStruktur1968} und
\authorcite{Holzhey:KantsErfahrungsbegriff1970} geht es letztlich um eine
Interpretation \name[Immanuel]{Kant}s, nicht
\authorcite{Wolff:Psychologiaempirica1968}s).
\authorcite{Wolff:Psychologiaempirica1968} schreibt in der
\singlequote{Deutschen Logik} im Kapitel \titel{Von der Erfahrung, und wie
dadurch die Sätze gefunden werden}:
\begin{quote}\phantomsection\label{Zitat:Wolff:ErfahrungnurvoneinzelnenDingen}
 Da wir nun nichts als eintzele Dinge empfinden; so sind auch die Erfahrungen
nichts als Sätze von eintzelen Dingen. Derowegen wer sich auf die Erfahrung
geruffet, ist billig gehalten, einen besonderen Fall anzuführen, es sey denn,
daß die Erfahrung so beschaffen ist, daß sie entweder ein jeder gleich haben
kan, wenn er sie verlanget, oder wenigstens sich bald darauf zu besinnen weiß,
weil er sie vorhin selbst mehr als einmal gehabt.\footnote{\cite[][Capitel 5,
\S~2]{Wolff:VernuenftigeGedankenvondenKraeftendesmenschlichenVerstandesundihremrichtigenGebraucheinErkenntnisderWahrheit1978}.
Parallel heißt es in \authorcite{Wolff:Psychologiaempirica1968}s lateinischer
Übersetzung: \enquote{Quoniam nonnisi singularia percipimus; experientia
singularium est seu propositionibus singularibus constat, quarum subjectum est
individuum quoddam} \parencite[][\S~119]{Wolff:Cogitationesrationalesdeviribusintellectushumani1983}.
Siehe dazu auch
\cite[][\S\S~664--667]{Wolff:PhilosophiarationalissiveLogica1740}.
\enquote{Quoniam nonnisi singularia percipimus, \ori{experientia singularium
est}} \parencite[][\S~665]{Wolff:PhilosophiarationalissiveLogica1740}.}
\end{quote}
\begin{comment}
Empfindungen werden von \authorcite{Wolff:Psychologiaempirica1968} nicht mit Erfahrungen identifiziert,
weswegen der Zusammenhang nicht ganz so klar ist, wie es zunächst scheint.
\enquote{Wir erfahren alles dasjenige, was wir erkennen, wenn wir auf unsere
Empfindungen acht haben.}\footnote{\cite[][5. Capitel,
\S~1]{Wolff:VernuenftigeGedankenvondenKraeftendesmenschlichenVerstandesundihremrichtigenGebraucheinErkenntnisderWahrheit1978}.
Siehe auch die Parallelstelle in der Lateinischen Logik, wo der Begriff der
Erfahrung (\enquote{experientia}) folgendermaßen eingeführt wird:
\enquote{\ori{Experiri} dicimur, quicquid ad perceptiones nostras attenti
cognoscimus. Ipsa vero horum cognitio, qu\ae{} sola attentione ad perceptiones
nostras patent,
\ori{experientia} vocatur}
\parencite[][\S~664]{Wolff:PhilosophiarationalissiveLogica1740}.} Natürlich
empfinden wir nicht, dass die Flamme das Papier anzündet, aber wir erfahren es,
wenn wir auf die entsprechenden Empfindungen achten (und nicht etwa auf den
Begriff der Flamme oder des Papiers), d.\,h.\ in kantischer Terminologie wenn
wir unser rezeptives (sinnliches) Erkenntnisvermögen gebrauchen. Erfahren und
Empfindungen haben ist nicht dasselbe, schon weil wir Dinge
empfinden\footnote{\cite[Vgl.][1. Capitel,
\S~1]{Wolff:VernuenftigeGedankenvondenKraeftendesmenschlichenVerstandesundihremrichtigenGebraucheinErkenntnisderWahrheit1978}:
\enquote{Ein jeder nimmet bey sich selbst wahr, daß er viele Dinge empfindet.
Ich sage aber, daß wir etwas empfinden, wenn wir uns desselben als uns
gegenwärtig bewust sind. So empfinden wir den Schmertz, den Schall, das Licht
und unsere eigene Gedancken.} Durch den zweiten Satz wird die Angabe, dass wir
Dinge empfinden, nicht widerrufen, sondern um weitere Objekte unseres
Empfindens ergänzt, wie aus \S~3 hervorgeht.}, aber Tatsachen erfahren. Was wir
empfinden, sind einzelne Gegenstände wie Gläser, Wasser und Tische bzw.\ die
Wirkungen auf unsere Sinnesorgane wie Licht oder Nässe. Was wir erfahren oder
wahrnehmen ist, dass das Wasser den Tisch nass macht.
\end{comment}
Da historische Erkenntnisse genau diejenigen Erkenntnisse sind, die wir aus
Erfahrung haben, können wir schließen, dass historische Erkenntnisse
Erkenntnisse von einzelnen Dingen (\emph{singularia}) sind. Historische
Erkenntnisse gibt es \emph{a fortiori} -- so scheint es -- nur in der Form
singulärer Urteile wie \enquote{Peter ist
wütend} oder  \enquote{Dieser Stein ist schwer}, nicht aber als
allgemeine Urteile wie \enquote{Alle Menschen sind sterblich} oder
\enquote{(Alle) Steine sind schwer}. Auch schon
\authorcite{Hobbes:Leviathan1962}' Darstellung legt in diesem Sinne nahe,
dass absolutes Wissen, welches wir legitimer Weise als testimoniales Wissen
akzeptieren, ausschließlich singuläre oder besondere Urteile betrifft, also
solche Urteile, die von einem einzelnen Gegenständen aussagen, dass sie die und
die Eigenschaften haben oder ihnen dies und jenes zugestoßen ist. Allgemeine
Urteile, die eine Eigenschaft von \emph{allen} Gegenständen (einer bestimmten
Art) aussagen, scheinen eher auf die Seite konditionalen Wissens zu gehören. Sie sagen, immer
dann, \emph{wenn} etwas ein Gegenstand der Art $\Phi $ ist, \emph{dann} hat es
auch die Eigenschaft $\Psi $. \emph{Wenn} beispielsweise die Sonne auf einen
Stein scheint, \emph{dann} wird dieser erwärmt. Insbesondere allgemeine Naturgesetze gehören
zumindest nach \authorcite{Hobbes:Leviathan1962} zu dem konditionalen, nicht zu dem absoluten
Wissen.

Hinzu kommt eine terminologische Schwierigkeit: Der Ausdruck
\enquote{historisch} (\emph{historicus}) hat bei \authorcite{Wolff:Psychologiaempirica1968} je nach Kontext unterschiedliche
Gegenbegriffe. In der Unterteilung dreier Erkenntnisarten sind der historischen
Erkenntnis (\emph{cognitio historica}) die \enquote{mathematische}
(\emph{cognitio mathematica}) und die \enquote{philosophische} (\emph{cognitio
philosophica}) entgegengesetzt. Bei der Einteilung von Schriften ist
\enquote{historisch} (\emph{liber historicus}) der Gegenbegriff zu
\enquote{dogmatisch} (\emph{liber dogmaticus}).
\emph{Libri historici} berichten Tatsachen der Natur
oder der Menschen, während die \emph{libri dogmatici}
allgemeine Wahrheiten enthalten.\footnote{\enquote{\ori{In libris vel
recensentur facta sive naturae , sive hominum ; vel proponuntur dogmata , seu
veritates universales}}
\parencite[][\S~743]{Wolff:PhilosophiarationalissiveLogica1740}.
  \enquote{\ori{Libri} facta recensentes dicuntur \ori{historici}. Et
  propositiones singulares , si fuerint verae , \ori{veritatum historicarum}
  nomine veniunt}
  \parencite[][\S~744]{Wolff:PhilosophiarationalissiveLogica1740}.
  \enquote{\ori{Liber dogmaticus} a nobis dicitur , qui dogmata , seu veritates
  universales proponit}
  \parencite[][\S~750]{Wolff:PhilosophiarationalissiveLogica1740}.} Und dann
  kann \enquote{historisch} auch als Gegenbegriff zu \enquote{wissenschaftlich}
  oder \enquote{szientifisch} (\emph{scientificus}) verwendet werden: Die
  \emph{libri dogmatici} unterteilt
\authorcite{Wolff:Psychologiaempirica1968} wiederum in \emph{libri dogmatici
historici} und \emph{libri dogmatici scientifici}, je nachdem, ob sie selbst der
Forderung nach wissenschaftlicher Darstellung der Lehren genügen, oder bloß
von wissenschaftlichen Erkenntnissen berichten, ohne selbst wissenschaftliche Ansprüche zu
erfüllen.\footnote{\cite[Vgl.][\S~751]{Wolff:PhilosophiarationalissiveLogica1740}.}




Aus \authorcite{Wolff:Psychologiaempirica1968}s Sicht sind gerade allgemeine Naturgesetze
paradigmatische Fälle historischer Erkenntnisse. Als Beispiel einer historischen
Wahrheit nennt \authorcite{Wolff:Psychologiaempirica1968} den Fall, in dem jemand aus Erfahrung
weiß, dass ein Regenbogen entsteht, wenn Regen und Licht der Sonne
zusammentreffen.\footnote{\cite[Vgl.][\S~744]{Wolff:PhilosophiarationalissiveLogica1740}:
\enquote{Si quis observet, \ori{coelo pluvio ac lucente Sole apparuisse
iridem,} {\&} factum idem recenseat , is veritatem historicam proponit. Et
historicus erit liber , in quo plura istiusmodi facta recensentur.}}
\enquote{Historisch} nennt er jede Erkenntnis dessen, was ist oder geschieht
(\enquote{\ori{Cognitio} eorum, quae sunt atque
fiunt}\footnote{\cite[][\S~3]{Wolff:Discursuspraeliminarisdephilosophiaingenere1996}.}),
und er exemplifiziert dies durch die Erfahrungserkenntnisse, dass morgens die
Sonne aufgeht und sich Tiere durch Zeugung
fortpflanzen.\footnote{\enquote{E.\,gr.\ historica ejus est cognitio, qui
expertus novit, Solem mane oriri, vespere autem occidere; initio veris gemmas
effrondescere arborum; animalia propagari per generationem; nos nil appetere
nisi sub ratione boni}
\mkbibparens{\cite[][\S~3]{Wolff:Discursuspraeliminarisdephilosophiaingenere1996},
\cite[vgl.][\pno~165\,f.]{Seifert:Cognitiohistorica1976}}.} Aber wie lassen sich
\authorcite{Wolff:Psychologiaempirica1968}s Äußerungen zur Erfahrung als
Erkenntnis einzelner Dinge in der \titel{Deutschen Logik} dann mit seiner
Schilderung der historischen Erkenntnis im \titel{Discursus praeliminaris} in
Einklang bringen?

\authorcite{Wolff:Psychologiaempirica1968} vertritt im \titel{Discursus praeliminaris} die Ansicht,
dass wir allgemeine Erkenntnisse über kausale Zusammenhänge aus der Erfahrung haben. Wir erfahren, dass verschüttetes
Wasser den Tisch nass macht; nicht nur, dass Peter gestern ein Wasserglas
verschüttete und anschließend der Tisch nass war und dass Wolfgang am Tag zuvor
ein Glas Wasser verschüttete und ebenfalls der Tisch nass wurde und so weiter,
sondern dass verschüttetes Wasser \emph{immer} (oder simpliciter) den Tisch nass
macht. Aber wir nehmen dies nicht abstrakt wahr, sondern durch die Wahrnehmung
konkreter Beispiele. Es gibt -- so ein erster Deutungsversuch -- keine Erfahrung
allgemeiner Erkenntnisse ohne Wahrnehmung besonderer Fälle. Und von diesen konkreten Beispielen sagt
\authorcite{Wolff:Psychologiaempirica1968}, dass sie \emph{in bestimmten Fällen} zusätzlich zu der allgemeinen
Erfahrungserkenntnis anzuführen seien. Sie müssen angeführt werden, wenn nicht
jeder in der Lage ist, selbst die Erfahrung zu machen, oder nicht jeder die
Erfahrung selbst schon hätte gemacht haben können (wenn er nur auf seine
Wahrnehmungen geachtet hätte). Im Falle des Wassers, das den Tisch nass macht, ist kein
konkreter Fall anzuführen, da jeder in der Lage ist, eine solche Wahrnehmung
selbst zu haben. Hier liegt die Erfahrungserkenntnis einfach in dem allgemeinen
Urteil, dass Wasser nass macht. Nun gibt es Wahrnehmungen, die nicht jeder nach
Belieben erzeugen kann, aber jeder schon einmal gemacht hat. Dass Donner auf
Blitz folgt, ist eine solche Erfahrungserkenntnis, zu der nach \authorcite{Wolff:Psychologiaempirica1968}
ebenfalls keine Nennung konkreter Beispiele nötig ist. Aber es gibt auch Fälle,
bei denen nicht jeder die Erfahrung hat machen können und die auch nicht
reproduzierbar sind, ohne dass dies gegen ihre Zuverlässigkeit spräche. Man
denke an den Geologen, der sich auf seismologische Messungen verschiedener
Erdbeben beruft.
In solchen Fällen ist die konkrete Erfahrungssituation zu beschreiben, damit der
Zuhörer beurteilen kann, ob die Erfahrung richtig gedeutet wurde. Schließlich
täusche man sich leicht und oft darin, welche Erkenntnis in einer bestimmten
Situation wirklich erlangt werden kann.\footnote{\cite[Vgl.][Capitel 5,
\S~2]{Wolff:VernuenftigeGedankenvondenKraeftendesmenschlichenVerstandesundihremrichtigenGebraucheinErkenntnisderWahrheit1978}:
\enquote{Man erfordert aber dieses aus zweyerley Ursachen: einmahl, damit man
sehe, was einer vor Empfindungen gehabt, da er zu seiner Erfahrung gelanget:
darnach, damit man sehen kan, wie einer nach Anleitung seiner Empfindungen
seinen Satz formire. Und dieses ist höchstnöthig, denn wir finden, daß gar ofte
Leute einander widersprechen, und doch beyde sich auf die Erfahrung beruffen.}}

Entgegen der Deutung \authorcite{Kambartel:ErfahrungundStruktur1968}s geht es
hier also nicht um allgemeine im Unterschied zu singulären Urteilen, sondern um
inferentielles im Unterschied zu nicht-inferentiellem Wissen.\footnote{Siehe
dazu auch oben, Kap. \ref{subsubsection:BegriffderDiskursivitaet} ab S.
\pageref{Abschnitt:judiciumintuitivumdiscursivum}.} Im \titel{Discursus
praeliminaris} finden wir für diese Thematik den Ausdruck \emph{cognitio
(historica) arcana}, der in der \titel{Deutschen Logik} noch fehlt. Bei
\emph{beiden} Wissensarten handelt es sich um historische Erkenntnis und bei
\emph{beiden} kann es sich um singuläre wie auch um allgemeine Erkenntnisse
handeln. Dabei geht es nicht nur um eine gründliche Methodik empirischer
Forschung, sondern auch um die Möglichkeit kritischer Kontrolle vermeintlich
objektiver Erkenntnisse. Denn
\begin{quote}
 daß einer solche oder andere Empfindungen gehabt, kan er nicht erweisen, als
durch Zeugen, wenn einige mit dabey gewesen, und also fordert er mit Recht, man
solle ihm dies glauben. Allein er kan nicht fordern, daß ich glaube, er habe
richtig geschlossen: denn dieses kan nach den Regeln, vernünftig zu schlüssen,
beurtheilet werden, weil nichts gewöhnlicher, als daß man durch die Erfahrung
Sätze zu erschleichen suchet, wovon ich in der Methaphysick \punkt\ ein Exempel
gegeben.\footnote{\Cite[][Capitel 5, \S~4]{Wolff:VernuenftigeGedankenvondenKraeftendesmenschlichenVerstandesundihremrichtigenGebraucheinErkenntnisderWahrheit1978}.}
\end{quote}
Nicht alles, was wir aus der Erfahrung zu haben beanspruchen, entstammt
tatsächlich direkt unserer nicht-inferentiellen Wahrnehmung. Vieles muss erst
erschlossen werden. In der \titel{Deutschen Metaphysik} verweist
\authorcite{Wolff:Psychologiaempirica1968} auf die Behauptung, man könne aus der
Erfahrung auf die Einwirkung des Körpers auf die Seele schließen, weil man einen
steten Zusammenhang zwischen bestimmten körperlichen und seelischen Ereignissen
erfahre. Ein Schluss von der zeitlichen Koexistenz solcher Ereignisse auf eine
Einwirkung sei aber nicht statthaft, weil wir über keinen Begriff einer solchen
Einwirkung
verfügten.\footnote{\cite[Vgl.][529]{Wolff:VernuenftigeGedankenvondenKraeftendesmenschlichenVerstandesundihremrichtigenGebraucheinErkenntnisderWahrheit1978}.
In der Lateinischen Logik nennt
\authorcite{Wolff:PhilosophiarationalissiveLogica1740} das umgekehrte
Beispiel eines \enquote{\ori{[i]nfluxus physici anim\ae{} in corpus}} sowie die
magnetische Anziehung, von der wir nicht sagen könnten, wir nähmen sie wahr
\parencite[vgl.][\S~667]{Wolff:PhilosophiarationalissiveLogica1740}.}

Die kritische Funktion der Unterscheidung von inferentiellem und
nicht-inferentiellem Erfahrungswissen -- diskursive und intuitive Urteile,
verborgene und gemeine Erkenntnis -- liegt darin, dass nachprüfbar sein muss, ob
jemand tatsächlich erfahrungsbasiertes Wissen referiert. Die zu bannende Gefahr
liegt dabei nicht in einer möglichen Unaufrichtigkeit des Informanten, sondern
in der Fehldeutung und Überinterpretation des Berichteten. Es ist eine Maxime
des wissenschaftlichen Arbeitens nach
\authorcite{Wolff:Discursuspraeliminarisdephilosophiaingenere1996}, dass den
Rezipienten von Forschungsergebnissen transparent gemacht werden soll, auf
welcher Erfahrungsgrundlage die Ergebnisse erzielt wurden. Dabei handelt es sich
noch immer um historische Erkenntnisse, wenn mitgeteilt wird, dass etwas der
Fall ist (unabhängig davon, ob es sich um singuläre oder allgemeine Urteile
handelt). Es ist dies ein Grundsatz, der noch die Übernahme von historischen
Erkenntnisse aus zweiter Hand betrifft. Das Hauptaugenmerk legt
\authorcite{Wolff:Discursuspraeliminarisdephilosophiaingenere1996} aber
weiterhin auf die Differenz von historische und philosophischer Erkenntnis, die
durch die von \authorcite{Kambartel:ErfahrungundStruktur1968} und
\authorcite{Holzhey:KantsErfahrungsbegriff1970} angeführten Stellen gar nicht
berührt wird.

\section{Zusammenfassung und Ausblick}
In Kapitel \ref{section:autonomieunddaszeugnisanderer} hatte ich auf
\authorfullcite{Crusius:WegzurGewissheitundZuverlaessigkeitdermenschlichenErkenntniss1965}
als einen Vertreter der deutschen Aufklärung verwiesen, der eine
\singlequote{defaultistische} Position bezüglich der Möglichkeit testimonialen
Wissens vertritt: Wir präsupponieren die Zuverlässigkeit von Mitteilungen,
solange nicht gegen sie spricht. Auch \name[Immanuel]{Kant} scheint diese
Theorie zu übernehmen, wie aus einer Anmerkung im handschriftlichen Nachlass
hervorgeht.\footnote{Siehe Anm.
\ref{Anmerkung:KantundCrusiusPraesuppositionsTheorie} auf S.
\pageref{Anmerkung:KantundCrusiusPraesuppositionsTheorie} dieser Arbeit.} Die
Frage war nun, wie eine solche Position als aufgeklärt oder \singlequote{kritisch} verstanden
werden kann. Hierbei fanden sich zwei Herangehensweise:
\begin{nummerierung}
\item Die eine Möglichkeit, welche sich bei Autoren wie
\authorcite{Crusius:WegzurGewissheitundZuverlaessigkeitdermenschlichenErkenntniss1965},
\authorcite{Meier:AuszugausderVernunftlehre1752} und
\authorcite{Reimarus:DieVernunftlehrealseineAnweisungzumrichtigenGebrauchderVernunftinderErkenntnisderWahrheit1756}
ausgearbeitet findet, verweist auf die Betrachtung der Zuverlässigkeit der
Informationsquellen. Der Erwerb testimonialen Wissens ist also kritisch genau
dann, wenn wir keine Informationen übernehmen, deren Überbringer
Verdachtsmomente liefern oder deren Ursprung fraglich ist. Wir bewerten also die
Bonität eines Informanten oder einer Information.
\item Die zweite Möglichkeit besteht darin, bestimmte Arten von Erkenntnissen
von der Möglichkeit testimonialen Wissens auszunehmen beziehungsweise zu
konkretisieren, welche Gefahren bei der Übernahme solcher Erkenntnisse auf
Mitteilung hin zu erwarten sind.
\authorfullcite{Wolff:Discursuspraeliminarisdephilosophiaingenere1996}
Unterscheidung der drei Erkenntnisarten zu Beginn des \titel{Discursus
praeliminaris de philosophia in genere} führt dies exemplarisch vor und schließt
damit an \authorcite{Descartes:OeuvresdeDescartes1983}' Warnung vor einer bloß
historischen Kenntnis an, die aus der Büchergelehrsamkeit hervorgehe. Er macht
einen Anfang in dem Bemühen, den Begriff der \singlequote{historischen Kenntnis}
zu spezifizieren.
\end{nummerierung}
Die Warnung vor einer bloßen \emph{cognitionis philosophicae cognitio historica}
besagt freilich nicht, dass eine Übernahme philosophischen Wissens von anderen
nicht möglich oder nicht nicht wünschenswert sei. Sie konkretisiert, welche
Defizite möglich sind, wenn wir solche Erkenntnisse lediglich passiv rezipieren,
ohne zu beachten, dass der Erwerb wissenschaftlichen Wissens ganz wesentlich ein
Erwerb von wissenschaftlichen \emph{Kompetenzen} ist.
\authorcite{Descartes:OeuvresdeDescartes1983}' Warnung ist korrekt, wenn sie
verstanden wird als Warnung vor einem bloßen Nachsprechen wissenschaftlicher
Erkenntnisse ohne wissenschaftliche Ausbildung.

Bei \name[Immanuel]{Kant} finden sich -- wie bei vielen Autoren der damaligen
Zeit -- \emph{beide} Herangehensweisen; doch zeigte sich, dass seine Anknüpfung
an die Überlegungen zur Bonität des Informanten oder der Information nicht
ausreichen, einen Begriff aufgeklärten und mündigen Erwerbs testimonialen
Wissens zu erläutern. Das folgende \ref{Chapter:KantsSocialEpistemology}.
Kapitel soll zeigen, dass \name[Immanuel]{Kant} vielmehr an
\authorcite{Wolff:Discursuspraeliminarisdephilosophiaingenere1996} anschließt,
wenn es um den Umgang mit testimonialem Wissen geht. Seine Sorge gilt weniger
der Verlässlichkeit der Information als dem Verständnis derselben auf der Seite
des Rezipienten.


\chapter{Kants \emph{Soziale
Erkenntnistheorie}: Autonomie endlicher
Subjekte}\label{Chapter:KantsSocialEpistemology}
%
\Revision[Heidemann]{Das Ziel dieser Arbeit besteht darin, den Gehalt des
\enquote{sapere aude} so zu explizieren, dass zum einen die Endlichkeit des
Menschen kein unüberwindliches Hindernis seiner Aufklärung darstellt, die
Forderung der Aufklärung aber zum anderen aber auch nicht zu Leerformel wird. Es
geht daher um die Artikulation von Regeln des mündigen Denkens, die mit der Natur
des Menschen kompatibel sind. Nun liegt die größte Herausforderung für die
Explikation der geforderten Selbständigkeit darin, ihre Vereinbarkeit mit
unserer Angewiesenheit auf testimoniales Wissen zu zeigen. Die Frage lautet
also: Wie muss sich jemand gegenüber dem Wissen und den Mitteilungen Anderer
verhalten, um als mündig gelten zu können? Das Gesamt derjenigen Regeln, denen
jemand folgen muss, um als mündig gelten zu können, bezeichnete ich anfangs als
eine \emph{ethics of belief}. Somit lässt sich die Frage auch folgenderweise
artikulieren: Welche Aussagen beinhaltet eine kantische \emph{ethics of belief}
bezüglich testimonialem Wissen?}

\Revision[Heidemann]{Einige der Regeln, deren Beachtung \name[Immanuel]{Kant}
im Umgang mit testimonialem Wissen einfordert, sahen wir bereits im
\ref{subsection:BewertungvonInformationenanhandihrerQuellen}. Kapitel. Dabei
stellten wir aber zugleich fest, dass die Nennung dieser Regeln nicht
ausreicht, den Begriff der Mündigkeit hinsichtlich der Problematik
testimonialen Wissens zu erläutern. Zugleich deutete ich an, dass verstärkt auf
die Warnung vor bloß historischen Kenntnissen statt wissenschaftlichem Wissen
zu achten ist. Eben dies zu verstehen wird sich als Schlüssel zu
\name[Immanuel]{Kant}s \emph{ethics of belief} erweisen.}
%
%
\section{Der Status testimonialen Wissens bei Kant}
\label{Absatz:AufklaerungundZugangsInternalismus}
\authorfullcite{Schmitt:JustificationSocialityandAutonomy1987} führt \name[Immanuel]{Kant}s
Aufklärungsverständnis als ein mögliches Hindernis für eine philosophische
Konzeption an, die mit testimonialem Wissen umgehen kann, weil er ihm ein
entsprechendes Mündigkeitsideal
unterstellt.\footnote{\cite[Vgl.][46]{Schmitt:JustificationSocialityandAutonomy1987}:
\enquote{A different motivation for rejecting testimony, one which plausibly
applies to testimonial justification, might lie in the idea that testimonial
evidence, however conclusive, is not the sort on which an intellectually
autonomous subject would rely. \name[Immanuel]{Kant} might be seen as offering this
motivation\dots}} Ist der
erkenntnistheoretische Individualismus also ein notwendiger Bestandteil der
Aufklärungsphilosophie? Dies schiene nach dem bisher Gesagten für das Projekt
der Aufklärung fatal zu sein, denn ein solcher Individualismus gefährdete einen
großen Teil unseres Wissensbestandes, wenn er nicht sogar von Anfang an \emph{jeden} Wissensanspruch
zunichte machte. Wird die Möglichkeit von testimonialen Erkenntnissen also durch
Forderungen der Aufklärung in Frage gestellt?
\authorfullcite{Grundmann:DietraditionelleErkenntnistheorieundihreHerausforderer2001}
sieht einen solchen Zusammenhang zwischen der Forderung der Aufklärung nach
epistemischer Autonomie und einem erkenntnistheoretischen Individualismus, der
jegliche epistemische Arbeitsteilung ausschließe. Denn nur wer jederzeit
unabhängig von externen Autoritäten entscheiden könne, ob die ihm vorliegenden
Gründe seine Überzeugung hinreichend stützen, sei epistemisch autonom. Gerade
dies jedoch sei bei epistemischer Arbeitsteilung gerade nicht
möglich.\footnote{\cite[Vgl.][15--17]{Grundmann:DietraditionelleErkenntnistheorieundihreHerausforderer2001}.
\authorcite{Grundmann:AnalytischeEinfuehrungindieErkenntnistheorie2008} behauptet: \enquote{Wir können die epistemische
Autonomie folgendermaßen definieren: Ein Subjekt ist epistemisch autonom gdw.~es
(i) auf sich allein gestellt (unabhängig von externen Autoritäten), (ii)
unabhängig von empirischen Meinungen über die Welt, (iii) aufgrund reiner
Vernunfterwägungen beurteilen kann, welche seiner Meinungen gerechtfertigt sind.
Es ist offensichtlich, daß Descartes und Kant sich an diesem Ideal orientiert
haben}
(\cite[][17]{Grundmann:DietraditionelleErkenntnistheorieundihreHerausforderer2001}).
Es ist gerade dieses Bild epistemischer Autonomie, das es zurückzuweisen gilt!}
Dabei nimmt die Vorstellung, dass epistemische Autonomie einen
Zugangsinternalismus erfordere, wonach nur solche Gründe eine Überzeugung
rechtfertigen können, die dem Subjekt transparent
sind,\footnote{\cite[Vgl.][14]{Grundmann:DietraditionelleErkenntnistheorieundihreHerausforderer2001},
sowie \cite[][532]{Grundmann:AnalytischeEinfuehrungindieErkenntnistheorie2008}.}
eine Mittlerrolle in der Verbindung von Aufklärung und Individualismus ein: Wenn
ich eine Überzeugung unter Berufung auf eine Autorität (einen Zeugen oder
Experten) für wahr halte, dann sind die Gründe, die die Überzeugung
rechtfertigen, dieser Autorität transparent, nicht aber
mir.\footnote{\cite[Vgl.][532]{Grundmann:AnalytischeEinfuehrungindieErkenntnistheorie2008}.}


Nehmen wir an, Peter teilt uns mit, dass es regnet; dann weiß Peter, aus welchem
Grund er weiß, dass es regnet. Vielleicht schaute er aus dem Fenster oder er
hörte den Regen auf dem Wellblechdach der Garage. Aber \emph{wir} wissen nicht, welche
Gründe Peter für seine Überzeugung hat (vielleicht handelt es sich um
schlechte Gründe); im Grunde ist uns nicht einmal bekannt, ob er überhaupt
objektive Gründe für seine Überzeugung hat. (Die Überzeugung, dass es regnet,
könnte auch einfach Ausdruck von Peters Pessimismus sein.) Und selbst wenn Peter
uns die Gründe für seine Überzeugung nennt -- und beispielsweise sagt:
\enquote{Ich war gerade draußen und stellte fest, dass es regnet.} --, dann kann
doch nur Peter abschließend beurteilen, ob diese Angabe nicht vielleicht gelogen
ist. Es verwundert daher nicht, dass einige Interpreten schon in dem Ausspruch
\enquote{sapere aude!} und der Maxime einer niemals passiven Vernunft sowie in
Aufforderungen zum Selbstdenken Anlass genug sehen, die Möglichkeit
testimonialen Wissens im Rahmen des \index{Kant, Immanuel}kantischen Denkens
auszuschließen.\footnote{\cite[Vgl.~z.\,B.][46]{Schmitt:JustificationSocialityandAutonomy1987},
sowie
\cite{Grundmann:DietraditionelleErkenntnistheorieundihreHerausforderer2001} und
\cite[][529--541]{Grundmann:AnalytischeEinfuehrungindieErkenntnistheorie2008}.}


Nimmt man außerdem an, dass für \name[Immanuel]{Kant} der erkenntnistheoretische
Internalismus selbstverständlich ist\footnote{Es liegt durchaus nahe, die Formulierung zu
Beginn des \S~16 der \titel{Kritik der reinen Vernunft} als Bekenntnis zu einem
epistemologischen Internalismus zu interpretieren:
\enquote{Das: \ori{Ich denke}, muß alle meine Vorstellungen begleiten können}
\mkbibparens{\cite[][\S~16]{Kant:KritikderreinenVernunft2003},
\cite[][III: 108.19]{Kant:GesammelteWerke1900ff.}}. Vgl. z.\,B.
\cite[][73]{Schulting:KantNon-ConceptualContentandtheenquoteSecondStepoftheB-Deduction2012}.
Auch die Überlegungen im Architektonikkapitel der \titel{Kritik der reinen
Vernunft} sprechen für eine internalistische Lesart. Dagegen behauptet Andrew
\authorcite{Chignell:KantsConceptsofJustification2007}, dass \name[Immanuel]{Kant}s Begründungsbegriff von
seiner Grundausrichtung her externalistisch sei, wenngleich \name[Immanuel]{Kant}
stets auch zu internalistischen Intuitionen tendiere:
\enquote{[D]espite the at bottom externalist character of the account, there is
an emphatic nod to internalist intuitions \punkt . Knowledge cannot merely be
based on sufficient objective grounds; rather, the subject must also be in a
position, on reflection, to cite those grounds, although she need not be (and
usually is not) able to determine that they are  objectively sufficient}
\parencite[][\pno~49\,f.]{Chignell:KantsConceptsofJustification2007}.
\authorfullcite{Willaschek:DertranszendentaleIdealismusunddieIdealitaetvonRaumundZeit1997}
attestiert \name[Immanuel]{Kant} eine externalistische Position
bezüglich des Inhalts von Begriffen.
\enquote{\name[Immanuel]{Kant} nimmt hier bis in die Beispiele die
externalistischen Argumente \name[Hilary]{Putnam}s und
\authorcite{Kripke:NameundNotwendigkeit1981}s vorweg}
\parencite[][548]{Willaschek:DertranszendentaleIdealismusunddieIdealitaetvonRaumundZeit1997}.
Für eine Interpretation \name[Immanuel]{Kant}s als Inhaltsexternalist sprechen
sicherlich v.\,a. die Überlegungen in der Widerlegung des Idealismus
\mkbibparens{\cite[siehe][B 274--279]{Kant:KritikderreinenVernunft2003},
\cite[][III: 190.24--193.24]{Kant:GesammelteWerke1900ff.}}.} oder sich aus der
Forderung nach epistemischer Autonomie ergibt, dann hat man bereit einen ersten Grund, ihm eine individualistische Position zu unterstellen.
Außerdem schreibt \name[Immanuel]{Kant} selbst: \enquote{Es ist so bequem,
unmündig zu sein. Habe ich ein Buch, das für mich Verstand hat, \punkt\ so
brauche ich mich ja nicht selbst zu
bemühen.}\footnote{\cite[][A~482]{Kant:BeantwortungderFrage:WasistAufklaerung?1977},
\cite[][VIII: 35.13--16]{Kant:GesammelteWerke1900ff.}.} Allem Anschein nach wird
hier testimoniales Wissen (aus Büchern) mit Unmündigkeit identifiziert. Ist
\name[Immanuel]{Kant} also ein erkenntnistheoretischer Individualist, der die
Abhängigkeit von Autoritäten zurückweisen
muss?\phantomsection\label{Absatz:AufklaerungundZugangsInternalismus-ENDE}

Gegen eine individualistische Interpretation \name[Immanuel]{Kant}s sprechen
zunächst die Stellen, in denen er an einen \emph{sensus communis} und die Maxime
der \enquote{erweiterten Denkungsart} appelliert oder sich gegen den
\enquote{logischen Egoismus} wendet. Unter Rückgriff auf diese Momente des
\index{Kant, Immanuel}kantischen {\OE}vres ist die These von einem
erkenntnistheoretischen Individualismus \name[Immanuel]{Kant}s
leicht zurückzuweisen.\footnote{Siehe oben, Kapitel
\ref{section:sensuscommunis}.}
\authorcite{Schmitt:JustificationSocialityandAutonomy1987} behauptet aber nun,
die Forderung, aus einer allgemeinen Perspektive heraus zu denken, habe nichts
mit der Möglichkeit testimonialen Wissens zu tun, weswegen \name[Immanuel]{Kant}
genauso Individualist bleibe wie John
\name[John]{Locke}.\footnote{\enquote{Still, the way sociality enters here is
consistent with a \name[John]{Locke}an view, since there is no reliance on
testimony.
All the cognitive work is ones own.
\name[Immanuel]{Kant} remains an individualist in the same sense in which
\name[John]{Locke} is an individualist}
\parencite[][47]{Schmitt:JustificationSocialityandAutonomy1987}.} \emph{Sensus
communis} und erweiterte Denkungsart verweisen auf die Notwendigkeit von
Intersubjektivität und den sozialen Charakter der Vernunft, garantieren aber
nicht die Möglichkeit testimonialen Wissens. Die genannten
Stellen können den Individualismusvorwurf also noch nicht gänzlich, sondern nur
in einer bestimmten Hinsicht entkräften.

Oliver \authorcite{Scholz:DasZeugnisanderer2001} hat auf \authorcite{Schmitt:JustificationSocialityandAutonomy1987}s
Interpretation reagiert und die Lesart zurückgewiesen, wonach \name[Immanuel]{Kant}s Aufklärungskonzeption dem Gedanken
widerspricht, man könne sich als aufgeklärter Bürger durch das Lesen von Büchern
bilden.\footnote{\cite[Vgl.][]{Scholz:AutonomieangesichtsepistemischerAbhaengigkeiten2001}.}
\Revision[Theis, Pelletier]{Auch
\authorfullcite{Foucault:DieRegierungdesSelbstundderanderen2009} legt dar, dass \name[Immanuel]{Kant} nicht die Legitimität der Autorität von Büchern
anzweifelt, sondern bestimmte Arten des Verhaltens gegenüber solchen
Autoritäten. Es gehe um die \enquote{Art und Weise, wie das Individuum seinen
eigenen Verstand durch das Buch
ersetzt}\footnote{\Revision[Theis,
Pelletier]{\cite[][49]{Foucault:DieRegierungdesSelbstundderanderen2009}.}}.
In Übereinstimmung mit den Interpretationen
\authorcite{Scholz:DasZeugnisanderer2001}' und
\authorcite{Foucault:DieRegierungdesSelbstundderanderen2009}s} ist sich
\name[Immanuel]{Kant} vollkommen darüber im klaren, wie gering die Menge unseres
Wissens wäre, versuchten wir, auf testimoniales Wissen zu verzichten. Von
früheren Zeiten und anderen Ländern beispielsweise hätten wir gar kein
Wissen.\footnote{Vgl.
\cite{Kant:LogikBlomberg1966}, \cite[][XXIV:
245.10--13]{Kant:GesammelteWerke1900ff.}: \enquote{Ein großer Teil unserer
Erkenntniße entspringt durch Glauben, und ohne den Glauben würden wir sehr
vieler historischen Erkenntniße Entbehren müßen.
wir würden keine größere Erkenntniße haben, als höchstens des Orts, wo wir
leben, und der Zeit, in der wir leben.}} Aus einer Logikvorlesung \name[Immanuel]{Kant}s
wird das Beispiel berichtet, dass wir nur auf der Grundlage von
Berichten wissen können, dass Madrid die Hauptstadt Spaniens ist. Dies gilt ganz
offensichtlich für alle, die selbst noch nie in Madrid waren und ihr Wissen
diesbezüglich nur aus mündlicher oder schriftlicher Mitteilung (beispielsweise
aus Atlanten) haben. Aber selbst wenn wir nach Madrid gingen, um uns mit eigenen
Augen zu überzeugen, könnten wir nur von Einwohnern (oder durch von Einwohnern
aufgestellte Schilder und ähnliches Informationsmaterial) erfahren, dass dies
tatsächlich Madrid und überdies die Hauptstadt Spaniens
ist.\footnote{Vgl. \cite{Kant:WienerLogik1966}, \cite[][XXIV,2:
896.1--9]{Kant:GesammelteWerke1900ff.}} Die vermeintlich unmittelbar eigene
Erfahrung, die wir machen könnten, erweist sich somit als fundiert in Wissen,
das selbst wiederum testimonial
ist.\footnote{G.\,E.\,M.~\authorcite{Anscombe:WhatIsIttoBelieveSomeone2008} hat ein ganz
ähnliches Beispiel entwickelt, um die Unhintergehbarkeit testimonialen Wissens
aufzuzeigen. Sie legt dar, dass auch Einheimische in solchen Fällen nicht ohne
Rückgriff auf genuin testimoniales Wissen auskommen können. In ihrem Beispiel
muss selbst der gebürtige New Yorker Informationen von anderen bekommen haben,
um auch nur die einfache Frage beantworten zu können, ob New York in Nordamerika
liegt. \cite[Vgl.][4]{Anscombe:WhatIsIttoBelieveSomeone2008}:
\enquote{Even if you inhabit New York and you have simply learned its name as
the name of the place you inhabit, there is the question: How extensive a region is this place you are
calling \enquote{New York}? And what has New York got to do with this bit of a
map? Here is a complicated network of received Information.}}


Auch in vielen anderen Fällen lässt sich zeigen, dass von vorausgehenden
Informationen abhängig ist, was wir wahrnehmen können. So kann zum Beispiel nur derjenige wahrnehmen,
dass eine Banane reif ist, der weiß, dass Bananen beim Reifen ihre Farbe von
Grün hin zu Gelb ändern, aber noch nicht schwarz werden. Und solche
Informationen hat jeder von uns in der Regel zuvor von anderen gelernt, weswegen
auch die je eigene unmittelbare Wahrnehmung ihre Grundlage (zumindest häufig) in
testimonialem Wissen hat. Wir könnten die Dinge nicht \emph{als} das wahrnehmen,
als was wir sie wahrnehmen (zum Beispiel ein Stück Obst als reife Banane), wenn
uns nicht zuvor von anderen gelehrt worden wäre, dies zu
tun.\footnote{\cite[Vgl.][26]{Strawson:KnowingfromWords1994}:
\enquote{[\ori{W}]\ori{hat} we in fact perceive, the very nature or character of
our perceptual experience itself, is determined by the instruction, the information,
we have received from the words of others.}} Und dabei erwerben wir nicht
einfach eine Fähigkeit, sondern lernen auch allgemeine Wahrheiten über die
Dinge, die wir in der Folge zu beurteilen beherrschen. Dies ist die Lehre, die
wir auch schon aus \name[Immanuel]{Kant}s Madrid-Beispiel ziehen können.


Gerade weil es nicht möglich ist, unser testimoniales Wissen auf dem Wege
eigener Erfahrung einzuholen, diese vielmehr in vielen Fällen selbst auf testimonialem
Wissen beruht, ist es keine Option, den Ausdruck \enquote{Wissen} für diejenigen
Überzeugungen zu reservieren, deren Rechtfertigung ohne Berufung auf Autoritäten
auskommt. Eine solche Haltung nennt \name[Immanuel]{Kant} einen
\enquote{historischen Unglauben}; und diesen könne \enquote{man sich gar nicht
als vorsätzlich, mithin auch nicht als zurechnungsfähig
denken (weil jeder einem Faktum, welches nur hinreichend
bewährt ist, eben so gut als einer mathematischen
Demonstration glauben muß, er mag wollen oder nicht)}\footnote{\Cite[][A
328]{Kant:Washeisst:SichimDenkenorientieren?1977}, \cite[][VIII:
146.8--11]{Kant:GesammelteWerke1900ff.}.}.


Für \name[Immanuel]{Kant} ist es zunächst gleichgültig, ob wir Erkenntnisse durch je
eigene Wahrnehmung belegen können oder lediglich durch den Bericht Anderer
wissen. \enquote{Das Fürwahrhalten auf ein Zeugnis ist weder dem Grade noch der
Art nach vom Fürwahrhalten durch eigene Erfahrung
unterschieden.}\footnote{\label{Fussnote:ZeugnisundeigeneErfahrunggemaessJaescheLogik}\cite[][A~103]{Kant:ImmanuelKantsLogik1977},
\cite[][IX: 69.2--4]{Kant:GesammelteWerke1900ff.}.
\name[Gottlob Benjamin]{Jäsche}s Vorlage zu diesem Satz war eine Randnotiz
\name[Immanuel]{Kant}s zu \cite[][\S~206]{Meier:AuszugausderVernunftlehre1752}
\parencite[][XVI: 496.28--30, 497.18--22]{Kant:GesammelteWerke1900ff.}:
\enquote{Das Glauben auf ein Zeugnis ist von dem Vorwahrhalten durch eigene
Erfahrung gar nicht dem Grade nach unterschieden, auch nicht der Art nach. Aber
wohl das Glauben auf (logisch) unzureichende Gründe der Vernunft} (\cite[][\nopp
2765]{Kant:Reflexionen1900ff.}, \cite[][XVI:
501.9--12]{Kant:GesammelteWerke1900ff.}). Ein Beleg in einer von
\name[Immanuel]{Kant} selbst publizierten Schrift findet sich in:
\cite[][A~319]{Kant:Washeisst:SichimDenkenorientieren?1977}, \cite[][VIII:
141.10--17]{Kant:GesammelteWerke1900ff.}.} Dass es der Art nach nicht von Wissen
durch direkte Wahrnehmung unterschieden ist, heißt auch, dass es nicht auf einem
Schluss beruht, der die Aussage eines Anderen, die guten Erfahrungen mit dessen
früheren Aussagen und Ähnliches zu
Prämissen hat.\footnote{In demselben Sinne schreibt später \authorcite{Austin:OtherMinds1979},
dass es sich bei testimonialem Wissen um eine (legitime) eigenständige
Wissensquelle handle, die keiner externen Rechtfertigung
bedarf. \cite[Vgl.][82]{Austin:OtherMinds1979}: \enquote{The statement
of an authority \punkt\ enables me to know something, which I shouldn't
otherwise have known. It is a source of knowledge.}} Testimoniales Wissen ist
ebenso wie Wahrnehmungswissen nicht-inferentiell -- was nicht heißt, dass man
nicht über die Fähigkeit zu formal und material
gültigen Inferenzen verfügen müsste, um solches Wissen erwerben zu
können.\footnote{Dies ist eine der Kernaussagen
\authorfullcite{Sellars:EmpiricismandthePhilosophyofMind1997}', der darauf
verweist, dass auch solch \singlequote{unmittelbar} gewonnenes Wissen nur möglich ist,
weil wir über eine Reihe allgemeiner Erkenntnisse verfügen und zu entsprechenden
Inferenzen fähig sind. \enquote{I presume that no philosopher who has attacked
the philosophical idea of givenness or, to use the
\authorcite{Hegel:GesammelteWerke}ian term, immediacy, has intended to deny that
there is a difference between \ori{inferring} that something is the case and,
for example, \ori{seeing} it to be the case. If the term \enquote{given}
referred merely to what is observed as being observed, or, perhaps, to a proper
subset of the things we are said to determine by observation, the existence of
\enquote{data} would be as noncontroversial as the existence of philosophical
perplexities} \parencite[][13]{Sellars:EmpiricismandthePhilosophyofMind1997}.}


Dass testimoniales und Wahrnehmungswissen dem Grade nach nicht unterschieden
sind, heißt vermutlich, dass wir zwar Irrtümern und Täuschungen
anheimfallen können, wenn wir etwas für wahr halten (was uns jemand sagt
oder was wir glauben wahrzunehmen), diese Möglichkeit aber den Status der
Erkenntnis als Wissen in den alltäglichen Fällen nicht
unterminiert.\footnote{\authorfullcite{Pasternack:KantonOpinion2014} schreibt:
\enquote{It is not that \name[Immanuel]{Kant} is overly optimistic about the
reliability of testimony. Rather, he is equally pessimistic about it and our own judgment and
so accepts testimony as evidence because it is no worse than relying upon our
own experience} \parencite[][52]{Pasternack:KantonOpinion2014}.} Es gibt auf dem
Weg zu testimonialem Wissen und Wahrnehmungswissen Irrtumsmöglichkeiten, die es bei mathematischem Wissen nicht
gibt; deswegen ist der Grad der Gewissheit geringer. Er ist aber doch höher als
bei Meinungen, bloßen Vermutungen und
Hypothesen. Von besonderer Bedeutung ist aber, dass es sich um dieselbe Art
(nicht-inferentiellen) Wissens handelt; die Mitteilungen anderer sind also keine
abgeleitete, sondern eine ursprüngliche Wissensquelle. Wir können hier von einer
Wissensquelle \emph{sui generis} sprechen oder sagen, dass Mitteilungen und
eigene Erfahrungen \emph{derselben} Wissensquelle angehören, die sich in zwei
Arten aufteilen lässt. Ihren Niederschlag findet diese
Darstellungsweise in dem normalsprachlichen Gebrauch von \enquote{Erfahrung}:
Wir \singlequote{erfahren} nicht nur etwas \singlequote{am eigenen Leib}, wir
erfahren insbesondere auch von anderen, das etwas der Fall ist. Wir erfahren
beispielsweise aus der Zeitung die letzten Fußballergebnisse. Von diesem
Sprachgebrauch scheint sich \name[Immanuel]{Kant} jedoch zu distanzieren, wie
\authorfullcite{Holzhey:KantsErfahrungsbegriff1970}
behauptet.\footnote{\enquote{Das Bekanntwerden mit dem, was andere erfahren
haben, versteht Kant -- im Unterschied zum alltäglichen Sprachgebrauch, in dem das Wort \enquote{Erfahren}
gerade auch für ein Kennenlernen durch Hörensagen verwendet wird -- nicht mehr
als Erfahrung} \parencite[][124]{Holzhey:KantsErfahrungsbegriff1970}. Er beruft
sich dabei auf eine Bemerkung \name[Immanuel]{Kant}s in seinem Handexemplar des
\authorcite{Meier:Vernunftlehre1752}schen Logikbuchs
\mkbibparens{\cite[siehe][2844]{Kant:Reflexionen1900ff.}; \cite[][XVI:
542.7]{Kant:GesammelteWerke1900ff.}: \enquote{Wir erfahren keine Begebenheit,
die uns erzehlt wird.}}.
Dabei bleibt jedoch fraglich, ob es sich um eine eigene terminologische
Festlegung oder eine Erläuterung des Sprachgebrauch
\authorcite{Meier:Vernunftlehre1752}s handelt.} Wir wir dies auch terminologisch
fassen, es bleibt von Wichtigkeit, dass es nach \name[Immanuel]{Kant}s
Überzeugung zwischen eigener und fremder Erfahrung keinen philosophisch
relevanten Unterschied gibt.

Der These von einem Gegensatz zwischen Aufklärung und epistemischer
Arbeitsteilung hat \name[Immanuel]{Kant} also selbst widersprochen. Wir sind
darauf angewiesen, unser Wissen auch durch das Wissen anderer -- auf dem Wege
mündlicher Mitteilungen sowie durch Bücher -- zu erweitern.
Wer auf das Lesen von Büchern verzichtete, wäre hinterher nicht aufgeklärt,
sondern ungebildet und am Ende gerade unfähig, sich etwa in Fragen der
Gerechtigkeit und Richtigkeit politischer Entscheidungen ein angemessenes Urteil
zu bilden.\footnote{In Kapitel \ref{section:MuendigeLebensfuehrung}
argumentierte ich für die Notwendigkeit zumindest basalen Wissens für die
eigene Lebensführung. So gering dieses dem Umfang nach im Vergleich zum
Gesamtumfang menschlichen Wissens auch sein mag, wir erwerben es doch zumindest
partiell aus Büchern. Wenn \name[Immanuel]{Kant} diese Ansicht nicht teilte,
wäre es unverständlich, warum er eine \titel{Anthropologie in pragmatischer
Hinsicht} schreibt.} Mag auch -- wie Michael \name[Michael]{Albrecht}
behauptet\footcite[Vgl.][18--23]{Albrecht:KantsKritikderhistorischenErkenntnis--einBekenntniszuWolff?1982}
-- eine zur Schau gestellte Abneigung gegen das Lesen von Büchern die
Philosophie der Neuzeit und dabei insbesondere den eklektizistischen Teil der
deutschen Aufklärung prägen, so liegt doch auf der Hand, dass Bildung durch
das Lesen von Büchern letztlich grundlegend für Mündigkeit ist.

Um das Mündigkeitsideal als vernünftig auszuweisen, sind wir gezwungen, eine
Differenz zwischen einem mündigen und einem unmündigen Umgang mit Büchern
anzunehmen. Bücher sollen unser Denken bereichern, nicht uns das Denken abnehmen
-- denn von genau dieser Gefahr, sich von Büchern das Denken \emph{abnehmen} zu
lassen, handelt der Satz aus der Schrift \titel{Was ist Aufklärung?}. Worin
liegt aber der Unterschied zwischen einem mündigen und einem unmündigen Umgang
mit Büchern oder allgemein mit Informationen? Wann lässt man sich das Denken von
Büchern \singlequote{abnehmen} und wann ist es für das Denken bereichernd?

In den Reflexionen zu \authorcite{Meier:Vernunftlehre1752}s \titel{Auszug aus der
Vernunftlehre}, die der Vorbereitung seiner Logikvorlesungen dienen, schreibt
\name[Immanuel]{Kant}:
\enquote{Der historische Glaube ist vernünftig, wenn er critisch
ist.}\footnote{\cite[][\nopp 2763]{Kant:Reflexionen1900ff.}, \cite[][XVI:
501.4]{Kant:GesammelteWerke1900ff.}.} Was aber heißt es, dass ein Glaube
kritisch ist? Unkritisch sind wir im Falle testimonialer Erkenntnis sicherlich,
wenn wir leichtfertig jedem Glauben schenken und alles für bare Münze nehmen,
was uns erzählt wird. Dies kann aber nicht heißen, dass wir jede Form genuin
testimonialen Wissens als Ausdruck von Unmündigkeit zurückweisen. Es scheint
erforderlich zu sein, einen Kompromiss zwischen beiden Extrempositionen zu finden\footnote{Im
\enquote{Vorbericht, der sehr wenig für die Ausführung verspricht} zu den
\titel{Träumen eines Geistersehers} schreibt \name[Immanuel]{Kant}, es sei \enquote{eben
so wohl ein dummes Vorurtheil, von vielem, das mit einigem Schein der Wahrheit
erzählt wird, ohne Grund Nichts zu glauben, als von dem, was das allgemeine
Gerücht sagt, ohne Prüfung Alles zu glauben}
(\cite[][A~5]{Kant:TraeumeeinesGeisterseherserlaeutertdurchTraeumederMetaphysik1968},
\cite[][II: 318.9--12]{Kant:GesammelteWerke1900ff.}).} und einen testimonialen
Skeptizismus zu vermeiden, ohne in Leichtgläubigkeit zu verfallen. Hierzu
wiederum müssen konkrete Regeln im Umgang mit testimonialen Erkenntnissen
angegeben werden, die einen mündigen Umgang mit Informationen von einem unmündigen
abheben.\footnote{Dem \singlequote{positiven Begriff} des Selbstdenkens zufolge
sollen wir uns fragen, ob wir im Denken Grundsätzen folgen, die wir als
allgemeine Regeln unseres Vernunftgebrauchs \enquote{tunlich} finden. Siehe zum
positiven Begriff des Selbstdenkens oben Kapitel
\ref{positiverBegriffdesSelbstdenkens},
S.~\pageref{positiverBegriffdesSelbstdenkens}, und die Angaben in Anmerkung
\ref{Fussnote:positiverBegriffdesSelbstdenkens},
S.~\pageref{Fussnote:positiverBegriffdesSelbstdenkens}. Den Grundsatz,
epistemisch gänzlich autark zu leben, können wir nicht akzeptieren; und daher
sollten wir uns nach einer Alternative umschauen.
Eine solche Alternative muss beinhalten, dass wir auf Äußerungen anderer hin
vieles als Wissen akzeptieren und in unser Überzeugungssystem aufnehmen, einiges
aber auch ausschließen.} Zur Bewertung epistemischer Regeln gibt uns
\name[Immanuel]{Kant} eine Formel an die Hand, die dem kategorischen Imperativ
ähnelt. Auch wenn ich sie in der vorliegenden Arbeit nun bereits oft zitierte,
sei sie an dieser Stelle wiederholt:
\begin{quote}
  Sich seiner \ori{eigenen} Vernunft bedienen will nichts weiter sagen, als bei
  allem dem, was man annehmen soll, sich selbst fragen: ob man es wohl tunlich
  finde, den Grund, warum man etwas annimmt, oder auch die Regel, die aus dem,
  was man annimmt, folgt, zum allgemeinen Grundsatze seines Vernunftgebrauches
  zu
  machen?\footnote{\cite[][A~229]{Kant:Washeisst:SichimDenkenorientieren?1977},
  \cite[][VIII: 146--7]{Kant:GesammelteWerke1900ff.}.}
\end{quote}
Diesen allgemeinen Maßstab gilt es natürlich auch auf den Erwerb testimonialen
Wissens anzuwenden.

Zunächst einmal lässt sich die Maxime eines generellen Misstrauens verwerfen:
Informationen, die uns andere mitteilen, keinen Glauben zu schenken, führte
dazu, dass unsere epistemische Praxis zusammenbräche. Wir verfügten nicht einmal
mehr über einen Grundstock an Wissen, und deshalb kann dies keine vernünftige
epistemische Regeln sein.\footnote{Auch \authorfullcite{Cohen:KantontheEthicsofBelief2014} zeigt,
dass eine Zurückweisung der Möglichkeit testimonialen Wissens mit der
Universalisierbarkeitsforderung an epistemische Grundsätze unverträglich ist.
Sie argumentiert dabei für drei Thesen
\parencite[vgl.][326--328]{Cohen:KantontheEthicsofBelief2014}:
\begin{nummerierung}
\item Ohne testimoniales Wissen wäre die \singlequote{erweiterte Denkungsart}
nicht möglich.
\item Die Universalisierung der Maxime, Mitteilungen nicht zu vertrauen, führte
zur Unmöglichkeit von Mitteilungen selbst und damit in einen Widerspruch.
\item Wir unterliegen einem je eigenen Hang, unsere Erkenntnis mitzuteilen. Die
Universalisierung der Maxime des Misstrauens führte jedoch dazu, dass wir diesem
nicht mehr nachgehen könnten, was zu wollen uns unmöglich ist.
\end{nummerierung}} Ebenso wenig wollen wir freilich unkritische Leichtgläubigkeit zu unserer
Maxime machen. Also benötigen wir konkretere Regeln zum Umgang mit testimonialem
Wissen. Nehmen wir -- um ein einfaches Beispiel zu konstruieren -- einmal an,
Peter erzählt Jasmin, dass es regnet. Nehmen wir weiter an, dass Jasmin
tatsächlich explizit der Frage nachgeht, was sie beachten muss, um auf Peters
Auskunft kritisch zu reagieren. Vermutlich wird sie zunächst überlegen, ob Peter
überhaupt vertrauenswürdig ist. Wenn Peter ständig
die Unwahrheit sagt, wird sie vorsichtiger werden. \enquote{Wer einmal lügt, dem
glaubt man nicht.} So lautet eine vertraute Maxime des kritischen Umgangs mit
testimonialem Wissen. Auch wird sie sich fragen, ob Peter vielleicht ein
spezielles Interesse daran haben könnte, dass sie glaubt, dass es regnet.
Möglicherweise möchte er sich aus der Verabredung zu einem Ausflug herauswinden.
Eine weitere Maxime besagt daher, dass wir mit Auskünften vorsichtiger umgehen,
wenn der Zeuge ein eigenes Interesse hat, uns von etwas zu überzeugen. Gerade
wenn erste Verdachtsmomente aufgekommen sind, fallen uns auch noch weitere
Fragen ein: Woher kann Peter überhaupt wissen, dass es regnet? War er selbst
gerade draußen? Konnte er aus dem Fenster schauen? Dies ergibt eine dritte
mögliche Maxime: Wir sollten uns nur von solchen Leuten überzeugen
lassen, die in der Lage sind, entsprechende Informationen zu erlangen. Und das
kann einerseits heißen, dass potentielle Quellen testimonialen Wissens sowohl
die persönlichen Kompetenzen haben (wer eine Amsel kaum von einer Krähe
zu unterscheiden weiß, kann keine vertrauenswürdige Wissensquelle in
ornithologischen Fragen sein), als auch sich in der richtigen Situation befinden müssen (nur der ist ein
guter Zeuge eines Verbrechens, der sich in Tatortnähe aufgehalten hat und das
Geschehen auch beobachten konnte). Es gibt also viele Möglichkeiten, den Begriff
in der Form von Regeln zu konkretisieren, ohne die
Möglichkeit testimonialen Wissens zurückzuweisen. Dabei liegt es zunächst nahe davon auszugehen, dass diese
Regeln sich darauf beziehen, \emph{wer} uns eine Information überbringt -- es
handelt sich um Regeln der Bewertung der Bonität von Informanten. Welche
konkreten Regeln aber sind aus der Sicht der Aufklärungsphilosophie speziell bei
\name[Immanuel]{Kant} vernünftig?

\name[Immanuel]{Kant} spricht  bei \enquote{historischen
Wahrscheinlichkeiten} von der Präsumtion, dass sich Unwahrheiten
verraten.\footnote{\cite[Vgl.][\nopp 2589]{Kant:Reflexionen1900ff.},
\cite[][XVI: 430.4--6]{Kant:GesammelteWerke1900ff.}.} Er benennt also zumindest
eine Präsumtionstheorie, die wie
\authorcite{Crusius:WegzurGewissheitundZuverlaessigkeitdermenschlichenErkenntniss1965}
behauptet, dass wir die Glaubwürdigkeit einer Informationsquelle präsumieren
können, solange sich kein Grund findet, daran zu zweifeln. Überlegungen, wann die Glaubwürdigkeit einer Information in
Zweifel zu ziehen ist, finden sich vor allem auch in seinen Vorlesungen; aber --
so hatten wir gesehen -- diese sind nicht geeignet, \name[Immanuel]{Kant}s
Position zur Vereinbarkeit von Mündigkeit und testimonialem Wissen zu
erläutern (siehe Kapitel
\ref{subsection:BewertungvonInformationenanhandihrerQuellen}). Es wird sich nun
zeigen, dass die Überlegungen zum Unterschied von Philosophie und historischer
Kenntnis bei \authorcite{Wolff:Discursuspraeliminarisdephilosophiaingenere1996}
(Kapitel \ref{subsection:BewertungvonInformationennachihrerART}) einen
geeigneteren Ausgangspunkt bietet.



\section{Kant und die historische
Kenntnis der Philosophie}\label{section:MuendigkeitundPhilosophie}

Bei \authorcite{Wolff:Discursuspraeliminarisdephilosophiaingenere1996} bleibt
Mündigkeit ein Status, den nur Gelehrte und auch diese nur in jeweils eng
umgrenzten Wissenschaftsbereichen erlangen können. Mündigkeit wird zur
kompetenten Teilnahme am Projekt einer methodisch disziplinierten und letztlich
empirisch fundierten Wissenschaft, die Einsicht in die Gründe und Ursachen allen
Geschehens zu erwerben versucht. Für \name[Immanuel]{Kant}s Forderung, jederzeit
seinen eigenen Verstand zu gebrauchen, ist dieser Begriff von Mündigkeit viel zu
anspruchsvoll. Es könnte niemand mündig in diesem Sinne werden, denn Mündigkeit
wird so zu einer Art von Spezialistentum.

Damit \name[Immanuel]{Kant} also von der Anknüpfung an
\authorcite{Wolff:Discursuspraeliminarisdephilosophiaingenere1996}s Systematik
dennoch profitieren kann, muss er Änderungen vornehmen. Im folgenden werde ich
zeigen, wie \name[Immanuel]{Kant} die Begriffe der historischen und
philosophischen Erkenntnis anders expliziert, als
\authorcite{Wolff:Discursuspraeliminarisdephilosophiaingenere1996}. Das Ergebnis
wird sein, dass \name[Immanuel]{Kant} den Bereich der Erkenntnisse, über die als
bloß historische Erkenntnis zu verfügen ein defizienter Zustand ist, auf die
metaphysischen Erkenntnisse, also den Kern der Philosophie beschränkt. Im
Anschluss expliziere ich, was \name[Immanuel]{Kant} unter \enquote{Metaphysik}
versteht (Kapitel \ref{section:MetaphysikausderPerspektivedesMenschen}) und
\emph{warum} letztlich nur die Philosophie (in einem Sinne, der moderner ist,
als \authorcite{Wolff:Discursuspraeliminarisdephilosophiaingenere1996}s Begriff der
\emph{philosophia} oder \singlequote{Weltweisheit}) für Fragen der Mündigkeit
von Bedeutung ist (Kapitel \ref{section:AutonomieundtestimonialesWissen}).

Im Kapitel über die \titel{Architektonik der reinen Vernunft} in der
\titel{Kritik der reinen Vernunft} unterteilt \name[Immanuel]{Kant} die Gattung
\enquote{Erkenntnis} in die Arten historische/empirische, mathematische und philosophische
Erkenntnis. Die historische Erkenntnis ist Erkenntnis \emph{ex datis}, die
rationale oder Vernunfterkenntnis ist Erkenntnis \emph{ex principiis}. Diese
unterteilt sich wiederum in philosophische Erkenntnis als rationaler Erkenntnis
\emph{aus Begriffen} und mathematische Erkenntnis als rationaler Erkenntnis aus
der \emph{Konstruktion} von Begriffen (in der reinen
Anschauung).\footnote{\cite[Vgl.][B 863--865]{Kant:KritikderreinenVernunft2003},
\cite[][III: 540.23--542.2]{Kant:GesammelteWerke1900ff.}. Dies ist zu trennen
von der \enquote{Stufenleiter} der \emph{Vorstellungsarten}, die
\enquote{Vorstellung} als Gattungsbegriff betrachtet und in der auch von
\enquote{Erkenntnis} als einer Art von Vorstellungen gehandelt wird. Die
Erkenntnisse als objektive bewusste Vorstellungen unterteilen sich hier in
Anschauungen und Begriffe \mkbibparens{\cite[vgl.][B
376\,f.,]{Kant:KritikderreinenVernunft2003} \cite[][III:
249.37--250.14]{Kant:GesammelteWerke1900ff.}}. Mit dieser Stufenleiter verwandt
sind die Grade der Erkenntnis hinsichtlich ihres objektiven Gehalts, die die
\titel{Jäsche-Logik} aufzählt: vorstellen, wahrnehmen, kennen, erkennen,
verstehen, einsehen, begreifen
\mkbibparens{\cite[vgl.][A 96\,f.,]{Kant:ImmanuelKantsLogik1977}
\cite[][IX: 64.29--65.24]{Kant:GesammelteWerke1900ff.}}. Der
\titel{Jäsche-Logik} zufolge handelt es sich bei diesen Stufenleitern oder
Graden um Unterschiede der Erkenntnisse bezüglich ihrer Qualität oder
\emph{Klarheit} nach, während es uns hier um Unterschiede bezüglich der
logischen Vollkommenheit der Modalität oder \emph{Gewissheit} nach geht
\mkbibparens{\cite[vgl.][A 107]{Kant:ImmanuelKantsLogik1977},
\cite[][IX: 70.27--35]{Kant:GesammelteWerke1900ff.}, sowie \cite[][A
96\,f.,]{Kant:ImmanuelKantsLogik1977} \cite[][IX:
64.29--65.24]{Kant:GesammelteWerke1900ff.}}.} Seine Konzeption baut dabei auf
der uns nun von \authorcite{Hobbes:Leviathan1962} und insbesondere
\authorcite{Wolff:Psychologiaempirica1968} vertrauten Unterscheidung von
historischen und philosophischen Erkenntnissen -- bei
\authorcite{Hobbes:Leviathan1962} zwischen \emph{absolute knowledge} und
\emph{conditional knowledge}\footnote{Siehe oben, Kapitel
\ref{subsection:VernunftwahrheitenUndErfahrungstatsachen}.} -- auf.
Die Grundlage, an der er sich orientiert, ist vermutlich
\authorfullcite{Wolff:Psychologiaempirica1968}s Systematik im \titel{Discursus
praeliminaris de philosophia in genere}\footnote{Dass \name[Immanuel]{Kant} die
Unterscheidung der Erkenntnisarten von
\authorcite{Wolff:Psychologiaempirica1968} -- wenngleich möglicherweise
vermittelt über einige andere Autoren -- übernommen habe, behauptet
\authorfullcite{Albrecht:KantsKritikderhistorischenErkenntnis--einBekenntniszuWolff?1982}
\parencite[vgl.][\pno~12\,f.]{Albrecht:KantsKritikderhistorischenErkenntnis--einBekenntniszuWolff?1982}.},
aber er übernimmt sie nicht einfach, sondern nutzt sie für seine Zwecke, indem
er sie seinen Interessen gemäß variiert und dadurch in seine Programmatik, die
stärker als die \authorcite{Wolff:Psychologiaempirica1968}s durch eklektische Einflüsse geprägt
ist,\footnote{\phantomsection\label{Anmerkung:BegriffderEklektik}Zur
eklektischen Aufklärung bei \name[Christian]{Thomasius} siehe
\cite[][398--416]{Albrecht:Eklektik1994}, sowie
\cite{Gerlach:EklektizismusoderFundamentalphilosophie?2001}.
Es ist unklar, ob Philosophen der Aufklärung wie Christian
\name[Christian]{Thomasius} korrekt als Eklektiker zu bezeichnen sind, ist doch
bei diesem schon ein Übergang von einer Philosophie des Eklektizismus zu einer
Philosophie des Selbstdenkens zu erkennen.
\cite[Vgl.][403]{Albrecht:Eklektik1994}: \enquote{In der Sache begründete
\name[Christian]{Thomasius} den Übergang von der Eklektik zum
\enquote{Selbstdenken} -- das Wort fand er noch nicht}. Siehe hierzu außerdem
\cite{Albrecht:Thomasius--keinEklektiker?1989,Holzhey:PhilosophiealsEklektik1983,Schneiders:VernuenftigerZweifelundwahreEklektik1985}.
Zur Differenz einer eklektischen Philosophie in der Tradition
\name[Christian]{Thomasius}' zu \authorcite{Wolff:Psychologiaempirica1968} und
dem \authorcite{Wolff:Psychologiaempirica1968}ianismus siehe
\textcite{Gerlach:EklektizismusoderFundamentalphilosophie?2001}, zur Eklektik
insgesamt die sehr ausführliche begriffsgeschichtliche Darstellung in
\textcite[][hier v.\,a.\ \pno~398-603]{Albrecht:Eklektik1994}.
\name[Immanuel]{Kant} beschreibt die antiken Eklektiker der {\jaeschelogik}
zufolge als \enquote{Selbstdenker, die sich zu keiner Schule bekannten, sondern
die Wahrheit suchten und annahmen, wo sie sie fanden} \mkbibparens{\cite[][A
37]{Kant:ImmanuelKantsLogik1977}; \cite[][IX:
31.29--30]{Kant:GesammelteWerke1900ff.}}, stellt also einen expliziten
Zusammenhang zwischen Aufklärung und antiker Eklektik heraus.
Andererseits schreibt er selbst abwertend, dass die Neuplatoniker \enquote{sich
Eklektiker nannten, weil sie ihre eigenen Grillen allenthalben in älteren
Autoren zu finden wußten, wenn sie solche vorher hineingetragen hatten}
\mkbibparens{\cite[][A 324]{Kant:Washeisst:SichimDenkenorientieren?1977};
\cite[][VIII: 144.34--36]{Kant:GesammelteWerke1900ff.}}.
\authorfullcite{Tonelli:ConditionsinKoenigsbergandtheMakingofKantsPhilosophy1975}
sieht \name[Immanuel]{Kant} als \emph{Vertreter} der Eklektik:
\enquote{I also came to the conclusion that he was an
\ori{anti-\authorcite{Wolff:Psychologiaempirica1968}ian} eclectic}
\parencite[][139]{Tonelli:ConditionsinKoenigsbergandtheMakingofKantsPhilosophy1975}.
\authorcite{Tonelli:ConditionsinKoenigsbergandtheMakingofKantsPhilosophy1975}
schreibt dabei gegen die Einordnung \name[Immanuel]{Kant}s in die
\name[Gottfried Wilhelm]{Leibniz}-\authorcite{Wolff:Psychologiaempirica1968}sche
Schulmetaphysik an: \enquote{\name[Immanuel]{Kant} non \`{e} \punkt\ mai stato
un wolffiano, e il suo sviluppo si spiega appunto con una costante
intenzione polemica nei confronti del wolffismo, che lo spinge ad
accettare viepi\'{u} radicalmente le dottrine avverse a tale filosofia}
(\cite[][vii]{Tonelli:ElementimetodologiciemetafisiciinKantdal1745al17681959},
o.\,H.\ im Orig.).
Auch dank der Arbeiten von
\authorcite{Tonelli:TheProblemoftheClassificationoftheSciencesinKantsTime1975}
lässt sich heute nicht mehr pauschal von einer wolffianischen Anfangsphase
\name[Immanuel]{Kant}s sprechen, mag auch
\authorcite{Tonelli:TheProblemoftheClassificationoftheSciencesinKantsTime1975}s
Darstellung \enquote{ihrerseits polemisch überspitzt sein}
\mkbibparens{\cite[][31]{Schwaiger:KategorischeundandereImperative1999}}.} integriert.
\name[Immanuel]{Kant} versucht, die begriffliche Klarheit
\authorcite{Wolff:Psychologiaempirica1968}s mit der Forderung nach geistiger
Selbständigkeit, die den Gehalt des \enquote{sapere aude!} ausmacht und in
seiner Forderung nach Selbstdenken zum Ausdruck kommt, zu
verbinden.\footnote{\cite[Vgl.][]{Albrecht:KantsKritikderhistorischenErkenntnis--einBekenntniszuWolff?1982}.
Dem widerspricht auch nicht, dass das bekannte \singlename{Horaz}-Zitat auf
einer Medaille der Alethophilen in der Form \enquote{sapere audent} (womit
\authorcite{Leibniz:Meditationesdecognitioneveritateetideis1999} und
\authorcite{Wolff:Psychologiaempirica1968} gemeint waren) den Weg in den
Diskurs der Aufklärung fand, wie
\textcite[vgl.][256]{Bronisch:WasistAufklaerung?2011} behauptet, und dass damit
zum Ausdruck gebracht wurde, dass gerade diese beiden Autoren dem Wahlspruch
beispielgebend gefolgt seien
\parencite[vgl.][263]{Doering:enquoteSapereaudeoderdieNotwendigkeitderautonomenVernunft2011}.
Zumindest \name[Immanuel]{Kant}s \emph{Deutung} des \enquote{sapere aude!} verlässt den
Boden der Philosophie \authorcite{Wolff:Psychologiaempirica1968}s.} Eine solche
Verbindung lässt die von \authorcite{Wolff:Psychologiaempirica1968} herrührende
Begrifflichkeit von historischer, philosophischer und mathematischer Erkenntnis
freilich nicht unberührt.
\begin{figure}[htb]
\begin{minipage}[t]{\textwidth}
\centering
\begin{tikzpicture}[grow=down,%
%edge from parent fork down,
level 1/.style={sibling distance=7cm,level distance=0cm},
level 2/.style={sibling distance=7cm,level distance=2cm},
level 3/.style={sibling distance=3cm,level distance=2cm},
level 4/.style={sibling distance=3.5cm,level distance=3cm},
every node/.style={rectangle,draw=black,fill=gray!25, thin, inner sep=0.5em, minimum size=0.5em, align=center},
edge from parent/.style={draw},
mylabel/.style={draw=none, fill=none, text width=5cm,text centered, inner sep=0.5em, anchor=base} ]
\node[draw=none,fill=none] {}
child {node[rounded corners] {Immanuel Kant:} edge from
parent[draw=none] child {node {Erkenntnis} edge from parent[draw=none]
 child {node {empirisch/\\ historisch}
 		edge from parent node[draw=none,fill=none,above,sloped] {\tiny \emph{ex
 		datis}}} child {node {\footnotesize rational}
	child {node {philosophisch}
 		edge from parent node[draw=none,fill=none,above,sloped] {\tiny diskursiv}}
  	child {node (mathematisch) {mathematisch}
 		edge from parent node[draw=none,fill=none,above,sloped,text width=1.8cm]
 		{\tiny{intuitiv}}} edge from parent
 		node[draw=none,fill=none,above,sloped] {\tiny \emph{ex principiis}}}}}
child {node[rounded corners] {Christian Wolff:} edge from
parent[draw=none] child {node {\emph{cognitio}} edge from parent[draw=none]
 child {node {\emph{historica}}
 		edge from parent node[draw=none,fill=none,above,sloped] {\tiny \emph{facta}}}
 child {node {\emph{mathematica}}
 		edge from parent node[draw=none,fill=none,above,sloped] {\tiny \emph{quantitates}}}
 child {node {\emph{philosophica}}
 		edge from parent node[draw=none,fill=none,above,sloped] {\tiny
 		\emph{rationes}}}}} ;
\end{tikzpicture}
%\includegraphics{ErkenntnisartennachWolffundKant.pdf}
  \caption{Einteilung der Erkenntnisarten nach
  \authorcite{Wolff:Psychologiaempirica1968} und \name[Immanuel]{Kant} im
  Vergleich}\label{abbildung:ErkenntnisartennachWolffundKant.pdf}
\end{minipage}
\end{figure}

Im folgenden soll \name[Immanuel]{Kant}s Fassung
dieser Systematik erläutert werden in Kontrast zu und in Anlehnung an
\authorcite{Wolff:Psychologiaempirica1968}s Darstellung, der sie zumindest in
ihren groben Konturen noch immer entspricht (siehe Abbildung
\ref{abbildung:ErkenntnisartennachWolffundKant.pdf}). Ich beginne hierzu im
nächsten Abschnitt mit dem Begriff der rationalen Erkenntnis, der bei
\name[Immanuel]{Kant} gegenüber \authorcite{Wolff:Psychologiaempirica1968} neu
eingeführt wird (Kapitel
\ref{subsection:Vernunfterkenntnis:MathematikPhilosophie}). Dazu ist auch der
zugehörige Gegenbegriff der historischen oder empirischen Erkenntnis zu klären
(Kapitel \ref{subsection:HistorischeundempirischeErkenntnis}). Eigentliches Ziel
ist die Herausstellung des Begriffs und der Bedeutung philosophischer
Erkenntnis, die -- wie sich herausstellen wird -- auch aus methodischen Gründen
im Zentrum der Forderung nach epistemischer Autonomie steht.\footnote{Siehe
dazu Kap. \ref{section:MetaphysikausderPerspektivedesMenschen}, insb.
\ref{paragraph:facettendesmetaphysikbegriffs}, sowie
\ref{section:AutonomieundtestimonialesWissen}.} Den Unterschied zwischen
philosophischer und mathematischer Erkenntnis als Unterarten der rationalen
Erkenntnisse und die besondere Stellung der mathematischen Erkenntnis werde ich
in Kapitel \ref{subsubsection:EndlichesundUnendlichesErkennen} eingehender
beleuchten, um die besonderen Schwierigkeiten der philosophischen Erkenntnis zu
besprechen, die sich aus der Endlichkeit unseres Denkens ergeben.

\subsection{Vernunfterkenntnisse}\label{subsection:Vernunfterkenntnis:MathematikPhilosophie}
Eine Abweichung von \authorcite{Wolff:Psychologiaempirica1968} ist bereits bei
oberflächlicher Betrachtung sichtbar:
\name[Immanuel]{Kant} fasst die mathematische und philosophische Erkenntnis
zunächst zur rationalen Erkenntnis zusammen, um sie anschließend
wieder zu differenzieren. Dahinter jedoch steht eine grundlegende Neufassung der
Begriffe philosophischer und mathematischer Erkenntnis sowie ein gegenüber
\authorcite{Wolff:Discursuspraeliminarisdephilosophiaingenere1996}
abgewandelter Begriff der Vernunfterkenntnis. Gerade \name[Immanuel]{Kant}s
Begriff rationaler Erkenntnis wird im weiteren Verlauf wichtig für ein
Verständnis dessen, was es nach ihm heißt, mündig Informationen zu rezipieren.


\authorfullcite{Wolff:Psychologiaempirica1968} nennt eine Erkenntnis
mathematisch, wenn sie von Quantitäten handelt.\footnote{\cite[Vg.][\S~14]{Wolff:Discursuspraeliminarisdephilosophiaingenere1996}:
\enquote{Cognitio quantitatis rerum est ea, quam \ori{mathematicam}
appellamus.}} So verfügt derjenige über eine \emph{cognitio mathematica}, der
weiß, dass die Fallbeschleunigung auf der Erdoberfläsche im Schnitt etwa $9,81
\frac{m}{s^2}$ beträgt oder dass im Kühlschrank noch zwei Flaschen Schwarzbier
liegen. Diese Erkenntnisse sehen auf den ersten Blick wie \emph{cognitiones
historicae} aus, aber wegen ihres thematischen Bezugs auf Quantitäten werden sie
\enquote{\emph{mathematicae}} genannt. Ebenso hat derjenige eine \emph{cognitio
mathematica} und nicht \emph{philosophica}, der bei der Angabe der Gründe einer
Tatsache auch die involvierten Quantitäten ausgehend von den Angaben der bei den
Gründen vorhandenen Quantitäten mit berechnen kann. Wer also
nicht nur die Erdanziehung auf der Grundlage der Tatsache der Gravitation,
\emph{dass} sich schwere Körper gegenseitig anziehen, zu erklären weiß (und so
über eine \emph{cognitio philosophica} verfügt), sondern darüber hinaus auch
auch mittels einer Berechnung ausgehend von Radius und Masse der Erde und dem
\name[Isaac]{Newton}schen Gravitationsgesetz $ F_G = G \cdot \frac{m_1 \cdot
m_2}{r^2} $ erklären kann, warum die Fallbeschleunigung gerade besagte $9,81
\frac{m}{s^2}$ beträgt, verfügt über eine \emph{cognitio mathematica}. Diese
setzt freilich die \emph{cognitio philosophica} voraus und bekräftigt
diese.\footnote{Der größte Nutzen der \emph{cognitio mathematica} ist nach
\authorcite{Wolff:Psychologiaempirica1968} gerade darin zu sehen, dass sie die
enthaltene \emph{cognitio philosophica} mit größerer Gewissheit ausstattet, und
nicht etwa in dem Nutzen, den uns mathematische Zusammenhänge auf technischem
Gebiet bringen \parencite[vgl.][\S~27]{Wolff:Discursuspraeliminarisdephilosophiaingenere1996}.}


Grundlage der \emph{cognitio mathematica} kann somit sowohl die \emph{cognitio
historica} als auch die \emph{cognitio philosophica} sein; es hat mitunter gar
den Anschein, als handle es sich bei der \emph{cognitio mathematica} wahlweise
um eine \emph{cognitio historica} oder \emph{cognitio philosophica} von
Quantitäten, also gar nicht um eine Art \emph{neben} diesen, sondern um eine
jeweilige \emph{Unterart} derselben. Eine solche Begriffsbestimmung ist aber
unbefriedigend, wie auch \name[Immanuel]{Kant} bemängelt, der darauf hinweist,
dass die Philosophie auch von Quantitäten und die Mathematik auch von Qualitäten
handle.\footnote{\enquote{Übrigens handelt die
Philosophie eben sowohl von Größen, als die Mathematik, z.\,B. von der Totalität, der Unendlichkeit usw.
Die Mathematik beschäftiget sich auch mit dem Unterschiede der Linien und
Flächen, als Räumen, von verschiedener Qualität, mit der Kontinuität der
Ausdehnung, als einer Qualität derselben} \mkbibparens{\cite[][B
743]{Kant:KritikderreinenVernunft2003}, \cite[][III:
470.15--20]{Kant:GesammelteWerke1900ff.}}.} Deswegen sei es ein Fehler, den
Unterschied der Erkenntnisarten auf dieser Grundlage fassen zu wollen, statt die
Art des Erkennens als Ausgangspunkt zu wählen.



Die Thematik einer jeweiligen Erkenntnisart, das, \emph{wovon} Erkenntnisse
einer bestimmten Art handeln, falls dies einheitlich und klar bestimmt sein
sollte, ergebe sich erst hinterher auf der Grundlage von Definitionen, die sich
auf die \singlequote{Form} der jeweiligen Erkenntnis beziehen:
\begin{quote}
In dieser Form besteht also der wesentliche Unterschied dieser beiden Arten der
Vernunfterkenntnis, und beruhet nicht auf dem Unterschiede ihrer Materie, oder
Gegenstände. Diejenigen, welche Philosophie von Mathematik dadurch zu
unterscheiden vermeineten, daß sie von jener sagten, sie habe bloß die
\ori{Qualität}, diese aber nur die \ori{Quantität} zum Objekt, haben die Wirkung
für die Ursache genommen. Die Form der mathematischen Erkenntnis ist die
Ursache, daß diese lediglich auf Quanta gehen
kann.\footnote{\cite[][B 742]{Kant:KritikderreinenVernunft2003},
\cite[][III: 469.34--470.4]{Kant:GesammelteWerke1900ff.}.}
\end{quote}
Unter einer Erkenntnis versteht \name[Immanuel]{Kant} hier\footnote{Siehe
Anmerkung \ref{Anmerkung:ErkenntnisInZweierleiSinn} auf S.
\pageref{Anmerkung:ErkenntnisInZweierleiSinn}. In der \singlequote{Stufenleiter}
bestimmt \name[Immanuel]{Kant} die Erkenntnis abweichend als eine Vorstellung,
die mit Bewusstsein auf Objekte bezogen wird.
Dass es sich dabei aber um einen anderen Begriff handelt, der nur mit demselben
Wort belegt ist, wird daraus deutlich, dass sich Erkenntnisse in diesem Sinne in
Anschauungen und Begriffe unterteilen. Sie \emph{bestimmen} ein Objekt nicht,
sondern dienen dazu, dieses \emph{in Urteilen} zu bestimmen, also in
Erkenntnissen der anderen Art. Sie sind aber selbst keine Urteile, sondern nur
mögliche Bestandteile von Urteilen.} die \enquote{bestimmte[.] Beziehung
gegebener Vorstellungen auf ein Objekt.}\footnote{\cite[][B
137]{Kant:KritikderreinenVernunft2003}, \cite[][III:
111.17--18]{Kant:GesammelteWerke1900ff.}.} Wir erkennen einen Gegenstand -- etwa
\singlequote{dieses Buch} --, wenn wir Vorstellungen -- die Anschauung des
Buches und den Begriff der Farbe Grün -- auf diesen Gegenstand beziehen und ihn
dadurch näher Bestimmen: \enquote{Dieses Buch ist grün.} Dadurch bringen wir die
beiden Vorstellungen -- hier die Anschauung und den Begriff -- zur objektiven
Einheit der Apperzeption und fällen ein Urteil. Denn ein Urteil ist
\enquote{nichts anderes {\punkt}, als die Art, gegebene Erkenntnisse
[Anschauungen und Begriffe; A.\,G.] zur objektiven Einheit der Apperzeption zu
bringen.}\footnote{\cite[][B 141]{Kant:KritikderreinenVernunft2003},
\cite[][III: 114.7--8]{Kant:GesammelteWerke1900ff.}.} Die \singlequote{Form}
einer Erkenntnis ist dann anhand dieser Beziehung zu charakterisieren, also
anhand der Frage, \emph{wie} oder auf welcher Grundlage wir die Vorstellung zur
objektiven Einheit der Apperzeption bringen. Die grundlegende Unterscheidung, die \name[Immanuel]{Kant}
hier als relevant erachtet, ist: Wir können die Vorstellungen auf der Grundlage
von Erfahrung (\emph{a posteriori}) oder ohne diese Grundlage (\emph{a priori}) zur
Einheit bringen.\footnote{\cite[Vgl.][B
12\,f.,]{Kant:KritikderreinenVernunft2003}
\cite[][III: 35.1--36.5]{Kant:GesammelteWerke1900ff.}.} Und dies spiegelt sich
wieder in der Art und Weise, wie \name[Immanuel]{Kant} nicht erst mathematische
und philosophische, sondern bereits historische und rationale Erkenntnisse
differenziert. \enquote{Wenn ich von allem Inhalte
der Erkenntnis, objektiv betrachtet, abstrahiere, so ist alles Erkenntnis,
subjektiv, entweder historisch oder rational. Die historische Erkenntnis ist
cognitio ex datis, die rationale aber cognitio ex
principiis}\footnote{\cite[][B~863\,f.,]{Kant:KritikderreinenVernunft2003}
\cite[][III: 540.30--33]{Kant:GesammelteWerke1900ff.}.} Der Unterscheidungsgrund
liegt -- wie hier schon zu vermuten ist -- darin, ob eine Erkenntnis
ausschließlich der Spontaneität entstammt oder nur durch Rekurs auf Rezeptivität
möglich ist. Die rationale Erkenntnis ist -- so werde ich später zeigen -- eine
Erkenntnis des autonomen oberen Erkenntnisvermögens, die historische Erkenntnis
ist eine Erkenntnis des unteren Erkenntnisvermögens.\footnote{Siehe
dazu Kap. \ref{subsection:MetaphysikundAutonomie}.}




Statt ihres Gegenstandsbereichs ist bei \name[Immanuel]{Kant} also der Ursprung
oder die Art ihres \emph{Erwerbs} Ausgangspunkt der Differenzierung von
Erkenntnisarten. Dabei unterscheidet er das Empirische vom
Rationalen und historische von rationalen Erkenntnissen.\footnote{\cite[Vgl.][B
863\,f.,]{Kant:KritikderreinenVernunft2003} \cite[][III:
540.27--33]{Kant:GesammelteWerke1900ff.}.} Auf das nicht ganz einfache
Verhältnis des Empirischen zu den historischen Erkenntnissen werde ich gleich
eingehen,\footnote{Siehe Kapitel
\ref{subsection:HistorischeundempirischeErkenntnis}.} bis dahin werde ich das
Empirische unberücksichtigt lassen und mich ausschließlich dem Unterschied
zwischen historischen und rationalen Erkenntnissen widmen.


\name[Immanuel]{Kant} bestimmt den Unterschied historischer und
rationaler Erkenntnis folgendermaßen: Historische Erkenntnis ist \emph{cognitio
ex datis}, rationale Erkenntnis \emph{cognitio ex principiis}. Rational ist eine
Erkenntnis, wenn sie \enquote{aus allgemeinen Quellen der Vernunft, {\punkt}
d.\,i. aus Prinzipien}\footnote{\cite[][B
864\,f.,]{Kant:KritikderreinenVernunft2003} \cite[][III:
541.15--17]{Kant:GesammelteWerke1900ff.}.} erkannt wird. Aus Prinzipien
oder den allgemeinen Quellen der Vernunft -- und nicht \emph{ex datis} -- stammen wiederum Philosophie und Mathematik. Mathematische und philosophische Erkenntnis haben somit eine wichtige Gemeinsamkeit: Sie sind beide \emph{Vernunfterkenntnisse}, also Erkenntnisse,
für deren Erwerb wir keine Erfahrung benötigen, die uns nicht gegeben (\emph{ex
datis}) werden müssen, sondern die wir aus dem je eigenen Gebrauch der Vernunft
schöpfen können. Dabei gibt es zwei Arten, eine Erkenntnis aus Prinzipien zu
generieren, und nach diesen zwei Arten differenziert \name[Immanuel]{Kant} die beiden Arten
rationaler Erkenntnisse, Mathematik und Philosophie: Eine rationale Erkenntnis
ist eine \singlequote{\emph{diskursive}} Vernunfterkenntnis \emph{aus Begriffen}
oder eine \singlequote{\emph{intuitive}} Vernunfterkenntnis \emph{aus der
Konstruktion von Begriffen}. Vernunfterkenntnisse aus Begriffen sind
philosophische, Vernunfterkenntnisse aus der Konstruktion von Begriffen
mathematische
Erkenntnisse.\footnote{\enquote{Alle Vernunfterkenntnis ist nun entweder die
aus Begriffen, oder aus der Konstruktion von Begriffen; die erstere heißt
philosophisch, die zweite mathematisch} \mkbibparens{\cite[][B
865]{Kant:KritikderreinenVernunft2003},
\cite[][III: 541.18--20]{Kant:GesammelteWerke1900ff.}}.}



Damit ergibt sich die Einteilung, die im Ergebnis (ihrer resultierenden
Dreiteilung) derjenigen \authorcite{Wolff:Psychologiaempirica1968}s sehr ähnlich
sieht, insofern sie die Erkenntnisse in empirisch/historische (bei
\authorcite{Wolff:Psychologiaempirica1968}: \emph{cognitio historica}),
mathematische (bei \authorcite{Wolff:Psychologiaempirica1968}: \emph{cognitio
mathematica}) und philosophische Erkenntnisse (bei
\authorcite{Wolff:Psychologiaempirica1968}: \emph{cognitio philosophica})
einteilt (siehe wiederum
Abbildung \ref{abbildung:ErkenntnisartennachWolffundKant.pdf} auf Seite
\pageref{abbildung:ErkenntnisartennachWolffundKant.pdf}). In ihrem Konstruktionsprinzip hingegen unterscheidet sich diese
Einteilungen gewaltig voneinander, insofern
\authorcite{Wolff:Psychologiaempirica1968} die Erkenntnisse hinsichtlich ihres
Gegenstandes -- \emph{facta},
\emph{rationes}, \emph{quantitates} -- unterscheidet, \name[Immanuel]{Kant}
hingegen hinsichtlich der Art ihres Erwerbs -- \emph{ex datis}, \emph{ex
principiis}, aus Begriffen, aus der Konstruktion von Begriffen.

Der wichtigste Unterschied zwischen \name[Immanuel]{Kant}s und
\authorcite{Wolff:Psychologiaempirica1968}s Einteilung ist folgender:
\authorcite{Wolff:Psychologiaempirica1968} erklärt uns, dass \emph{jede} Erkenntnis ihre Grundlage
in der Erfahrung habe, denn die rationale Erkenntnis zeichnet sich nicht durch
die Quellen ihres ursprünglichen Erwerbs aus, sondern durch ihre rationale
\emph{Verbindung}.\footnote{\cite[Vgl.][\S~483]{Wolff:Psychologiaempirica1968}:
\enquote{\ori{Ratio} est facultas nexus veritatum universalium intuendi seu
perspiciendi.} \cite[Außerdem][\S\S~10\,f., 26,
34]{Wolff:Discursuspraeliminarisdephilosophiaingenere1996}.
\enquote{Praemittenda adeo est cognitioni philosophicae historica atquecum ista
constanter conjugenda, ne firmum desit fundamentum}
\parencite[][\S~11]{Wolff:Discursuspraeliminarisdephilosophiaingenere1996}.
Dabei zeigt \S~10, dass wir um die Gründe für eine Tatsache wiederum aus
Erfahrung wissen können und also historische Erkenntnis die feste Grundlage
(\enquote{firm[um] ac inconcuss[um] \punkt\ fundament[um]}, \S~11) der
philosophischen liefern kann. Zusätzlich ist die philosophische Erkenntnis der
Bestätigung durch historische Erkenntnis im Experiment bedürftig (\S~26).
Insofern die Philosophie Wissenschaft sein soll, muss sie nach \authorcite{Wolff:Psychologiaempirica1968} in
Experiment und Beobachtung fundiert sein: \enquote{In philosophia itaque
principia ab experientia derivanda, quae demonstrantur experimentis ac
observationibus confirmanda}
\parencite[][\S~34]{Wolff:Discursuspraeliminarisdephilosophiaingenere1996}.
\cite[Siehe
auch][\pno~44\,f.]{Schneiders:VernunftundVerstand--KriseneinesBegriffspaares1995}.}
Und diese Erfahrungsgebundenheit gelte nicht nur innerhalb der Naturforschung,
sondern auch für die \emph{philosophia prima}, Moral- und politische Philosophie
und sogar für die Mathematik, die ihre Begriffe, teilweise sogar ihre Grundsätze aus der Erfahrung
hätten.\footnote{\cite[Vgl.][\S~12]{Wolff:Discursuspraeliminarisdephilosophiaingenere1996}.}
Das Neue an \name[Immanuel]{Kant}s Systematik ist die Annahme eines Bereichs von Erkenntnissen, die ihre Quelle in der
Vernunft -- der \enquote{Spontaneität} als
Selbsttätigkeit\footnote{\enquote{Selbsttätigkeit} ist bei \name[Immanuel]{Kant}
wie schon bei
\authorcite{Baumgarten:Metaphysica---Metaphysik2011} die Übersetzung des lateinischen \enquote{spontaneitas}
\mkbibparens{\cite[vgl.][B~68]{Kant:KritikderreinenVernunft2003}, \cite[][III:
70.22, 70.27, 107.25]{Kant:GesammelteWerke1900ff.};
\cite[][\S~704]{Baumgarten:Metaphysica---Metaphysik2011}, auch in \cite[][XVII:
131.26, 131.33]{Kant:GesammelteWerke1900ff.}}. Noch
\authorcite{Wolff:Psychologiaempirica1968} übersetzt \enquote{spontaneitas} als
\enquote{Willkühr}, wie er im ersten Register der \enquote{Deutschen Metaphysik}
vermerkt
\parencite[vgl.][677]{Wolff:VernuenftigeGedankenvondenKraeftendesmenschlichenVerstandesundihremrichtigenGebraucheinErkenntnisderWahrheit1978}.}
des erkennenden Subjekts -- haben.



\subsection{\emph{Cognitio ex datis}: Historische und empirische
Erkenntnis}\label{subsection:HistorischeundempirischeErkenntnis}
Einen weiteren bereits oberflächlich erkennbaren Unterschied zwischen
\authorcite{Wolff:Psychologiaempirica1968}s und \name[Immanuel]{Kant}s
Systematik stellt die Verwendung des Wortes \enquote{empirisch} dar. Während
\authorcite{Wolff:Psychologiaempirica1968} zumindest im \titel{Discursus
praeliminaris} unisono von \emph{cognitiones historicae} spricht, verwendet
\name[Immanuel]{Kant} manchmal den Ausdruck \enquote{historisch} und mitunter
den Ausdruck \enquote{empirisch}, um eine Erkenntnis zu bezeichnen, die keine
Vernunfterkenntnis ist. Zunächst ließe sich vermuten, dass \enquote{empirisch}
einfach ein Synonym zu \enquote{historisch} ist; schon bei
\authorcite{Wolff:Discursuspraeliminarisdephilosophiaingenere1996} waren die
historischen Erkenntnisse mit den Erfahrungserkenntnissen zumindest
koextensional. Bei \name[Immanuel]{Kant} scheint sich bereits aus der
Begriffsbestimmung zu ergeben, dass genau die empirischen Erkenntnisse
historisch sind. Eine weitere sprachlich naheliegende Möglichkeit besagt, dass
wir unter empirischen Erkenntnissen die Erkenntnisse aus eigener Erfahrung,
unter historischen Erkenntnissen aber die uns mitgeteilten Erkenntnisse aus
zweiter Hand verstehen.\footnote{Dies schlägt
\authorfullcite{Kater:PolitikRechtGeschichte1999} vor
\parencite[vgl.][140]{Kater:PolitikRechtGeschichte1999}.} Beide Vorschläge
taugen nicht als Interpretation der Unterscheidung, wie ich in diesem Abschnitt
darlegen werde.


\authorfullcite{Kambartel:ErfahrungundStruktur1968}
behauptet, es sei zwischen der \emph{cognitio ex datis} als historischer
Erkenntnis auf der einen und empirischer Erkenntnis auf der anderen Seite scharf
zu unterscheiden, da \name[Immanuel]{Kant} mit der \emph{cognitio ex datis} im
Anschluss an \authorcite{Wolff:Psychologiaempirica1968} ausschließlich singuläre
Aussagen über einzelne Gegenstände und einzelne Geschehnisse
bezeichne\footnote{\authorcite{Kambartel:ErfahrungundStruktur1968}
sieht Allgemeines bei
\authorcite{Wolff:Discursuspraeliminarisdephilosophiaingenere1996} in
historischen Erkenntnissen nur in Form von \emph{Begriffen} enthalten. Wegen
deren Allgemeinheit brauche es auch für
\authorcite{Wolff:Psychologiaempirica1968} und alle anderen, die hierin in der
Tradition des \singlename{Aristoteles} stehen, mehrere Wahrnehmungen, um eine
Erfahrung bilden zu können
\parencite[Vgl.][55--57]{Kambartel:ErfahrungundStruktur1968}. Mit
\name[Immanuel]{Kant} gesprochen ließe sich sagen: Es müssen \emph{viele} Fälle
gegeben sein, damit wir mittels \textit{reflektierender} Urteilskraft Begriffe
\emph{bilden} können, die dann die \textit{bestimmende} Urteilskraft auf einen
einzelnen Fall \emph{anwenden} kann. Aber die \emph{Urteile} bleiben doch
singuläre.}, während empirische Erkenntnisse wesentlich allgemein seien oder
wesentlich allgemeine Behauptungen enthielten.\footnote{\cite[Vgl.][54--58, 85,
99]{Kambartel:ErfahrungundStruktur1968}.} \enquote{Von der \enquote{cognitio ex
datis} ist bei \name[Immanuel]{Kant} wie bei \name[Gottfried Wilhelm]{Leibniz}
die \enquote{empirische \ori{Erkenntnis}} wohl zu unterscheiden.
Diese, d.\,h.\ z.\,B.\ die Physik, ist Philosophie, nämlich \enquote{empirische
Philosophie}.}\footnote{\Cite[][85]{Kambartel:ErfahrungundStruktur1968}.
Interessant ist, dass \authorcite{Kambartel:ErfahrungundStruktur1968} den
Ausdruck \enquote{Erkenntnis} in Entgegensetzung gegen die \emph{cognitio ex
datis} betont, als ob es sich bei dieser nicht um eine Erkenntnis
handelte.} Während historische Erkenntnisse nur in singulären, nicht aber in
generellen Urteilen bestünden, verwende \name[Immanuel]{Kant}
\enquote{empirische Erkenntnis} gleichbedeutend mit der \enquote{rationale[n]
Verarbeitungsstufe der Wahrnehmung, die exemplarisch die physikalische
Erkenntnis
darstellt}\footnote{\Cite[][99]{Kambartel:ErfahrungundStruktur1968}.}.
In dieser Unterscheidung sieht \authorcite{Kambartel:ErfahrungundStruktur1968} die wichtige Neuerung, die
\name[Immanuel]{Kant}s Philosophie gegenüber ihren Vorläufern erbringe -- ein
besseres Verständnis des Empirischen, welches er dem bloß Historischen der
Neuzeit gegenüberstelle. Die Grundlage hierfür bestehe in den von
\name[Immanuel]{Kant} angeführten \emph{empirischen}
Prinzipien\footnote{\cite[Vgl.][85]{Kambartel:ErfahrungundStruktur1968}.}, so
dass die empirische Erkenntnis wegen ihrer in Prinzipien gründenden
Allgemeinheit auf die Seite der \emph{cognitio ex principiis} zu gehören scheint.

Dass zumindest ein Teil dieser Behauptungen nicht haltbar ist, zeigte bereits
das letzte Kapitel. Die Tradition, in der \name[Immanuel]{Kant} steht, versteht
unter der \emph{cognitio historica} keine singulären Urteile in dem Sinn, den
\authorcite{Kambartel:ErfahrungundStruktur1968} hier voraussetzt. Und auch
\name[Immanuel]{Kant} gibt -- soweit ich sehe -- keinerlei Anlässe, den Begriff
historischer Erkenntnis in diesem Sinne zu
fassen.\footnote{\name[Immanuel]{Kant} bringt den Begriff historischer
Erkenntnis nur im Kontrast zu \singlequote{dogmatischer} Erkenntnis mit dem
Mangel an Allgemeinheit in Verbindung, welcher ihr aber auch nur manchmal
anhafte. Eine solche Darstellung findet sich zumindest in der \titel{Logik
Blomberg}:
\leftquote Eine Erkenntniß ist \ori{Historisch}, wenn sie mit der Form der
Vernunft nicht übereinstimmet, \ori{rational} wenn sie mit derselben
übereinstimmet, und zwar ohne Ansehung des Objects. Hier aber müßen wir auch
auf das Object reflectiren, wenn wir den Unterschied zwischen der
Dogmatischen, und Historischen Erkenntniß feste setzen wollen.
\begin{enumerate}
  \item[1.\textsuperscript{mo}] Die \ori{Dogmatische} ist eine allgemeine
  Erkenntniß, die a priori aus der Vernunft entspringet
  \item[2.\textsuperscript{do}] Die \ori{Historische} aber ist \myemph{nicht
  immer} allgemein, und beruhet auf den aussagen geschehener Dinge, auf der
  aussage der anderen, sie entstehet also \ori{a posteriori}.\rightquote{}
  \mkbibparens{\cite{Kant:LogikBlomberg1966}, \cite[][XXIV:
99.14--24]{Kant:GesammelteWerke1900ff.}}\end{enumerate} Daraus scheint
  aber doch zu folgen, dass das Historische
  \emph{manchmal} allgemein ist.}
Fraglos handelt es sich bei Vernunfterkenntnissen ausnahmslos
um allgemeine Erkenntnisse und nicht um singuläre Aussagen. Wenn eine Aussage also ein
singuläres Urteil artikuliert, dann liegt notwendig eine historische Erkenntnis
vor. Aber daraus folgt nicht, dass alle historischen Erkenntnisse in singulären
Urteilen bestehen. Und noch viel weniger lässt sich daraus etwas über den
Unterschied zwischen historischen und empirischen Erkenntnissen lernen.
Empirischen Erkenntnissen kommt \enquote{nur angenommene und komparative
\ori{Allgemeinheit}}\footnote{\cite[][B 3]{Kant:KritikderreinenVernunft2003};
\cite[][III: 29.3--4]{Kant:GesammelteWerke1900ff.}.} im Unterschied zur strengen Allgemeinheit von
Vernunfterkenntnissen zu -- \name[Immanuel]{Kant} nennt sie auch
\enquote{empirische Allgemeinheit}\footnote{\cite[][B
4]{Kant:KritikderreinenVernunft2003}, \cite[][III:
29.9]{Kant:GesammelteWerke1900ff.}.}. Aber für die Annahme, historischer
Erkenntnis fehle nach \name[Immanuel]{Kant} auch diese Stufe an Allgemeinheit,
fehlt schlicht jegliche textbasierte Evidenz. Und auch für die Annahme,
empirische Erkenntnisse seien -- wie
\authorcite{Kambartel:ErfahrungundStruktur1968} schreibt -- als
\enquote{rationale
Verarbeitungsstufe}\footcite[][99]{Kambartel:ErfahrungundStruktur1968} zu
verstehen, sehe ich keine Grundlage in \name[Immanuel]{Kant}s Schriften.

Dagegen ist es \emph{prima facie} nicht abwegig, von empirischen Prinzipien zu
sprechen und \emph{a fortiori} physikalische Erkenntnisse, wie sie
\authorcite{Wolff:Psychologiaempirica1968} als philosophische Erkenntnisse
vorschwebten, als \emph{cognitiones ex principiis} anzusehen.
\name[Immanuel]{Kant} spricht oft von \enquote{empirischen Prinzipien} und sogar von \enquote{Vernunfterkenntnis aus empirischen
Prinzipien}, die er \enquote{empirische Philosophie}\footnote{\cite[][B
868]{Kant:KritikderreinenVernunft2003}, \cite[][III:
543.25--26]{Kant:GesammelteWerke1900ff.}.} nennt.
Und damit scheint die hier vorgelegte Deutung der Unterscheidung von
historischen und Vernunfterkenntnissen durchaus gefährdet zu sein, insofern ich
vorausgesetzt habe, dass es sich bei den Vernunfterkenntnissen oder
\emph{cognitiones ex principiis} um Erkenntnisse \emph{a priori} handelt. Was
also sind Prinzipien und was sind Erkenntnisse aus Prinzipien?

Erkenntnisse aus Prinzipien sind zunächst ganz allgemein solche, die wir mittels
eines Vernunftschlusses erhalten, der als Obersatz eine allgemeine Aussage
enthält, die \emph{in dieser Verwendung als Obersatz eines Vernunftschlusses}
\enquote{Prinzip} genannt wird. \name[Immanuel]{Kant} schreibt:
\begin{quote}
Ich würde daher
Erkenntnis aus Prinzipien diejenige nennen, da ich das Besondre im Allgemeinen durch Begriffe erkenne. So ist denn
ein jeder Vernunftschluß eine Form der Ableitung einer Erkenntnis aus einem
Prinzip. Denn der Obersatz gibt jederzeit einen Begriff, der da macht, daß
alles, was unter der Bedingung desselben subsumiert wird, aus ihm nach einem
Prinzip erkannt wird. Da nun jede allgemeine Erkenntnis zum Obersatze in einem
Vernunftschlusse dienen kann, \punkt\ so können diese denn auch, \myemph{in
Ansehung ihres möglichen Gebrauchs}, Prinzipien genannt werden.\footnote{\cite[][B
357]{Kant:KritikderreinenVernunft2003}, \cite[][III:
238.24--32]{Kant:GesammelteWerke1900ff.}, \myherv . Einen \emph{Vernunft}schluss
nennt \name[Immanuel]{Kant} einen Schluss mit mehr als einer Prämisse; der
Gegenbegriff ist der des \emph{Verstandes}schlusses \mkbibparens{\cite[vgl.][B
360]{Kant:KritikderreinenVernunft2003},
\cite[][III: 240.14--20]{Kant:GesammelteWerke1900ff.}}.}
\end{quote}
Es kann also eine jede allgemeine Aussage -- wenngleich dies eine uneigentliche
Redeweise ist -- auch Prinzip genannt werden. Ob eine allgemeine Aussage ein Prinzip ist, das ist
keine Frage ihrer Form, insofern diese isoliert von anderen Urteilen betrachtet
wird. Eine allgemeine Aussage ist dann ein Prinzip, wenn sie in einem
entsprechenden logischen Kontext steht. Ein Vorurteil ist entsprechend der
\titel{Logik} zufolge ein vorläufiges Urteil, welches als Prinzip oder Grundsatz
(beide Ausdrücke sind synonym) akzeptiert wird und dadurch unbegründete oder
Fehlurteile generiert.\footnote{\enquote{Vorurteile sind vorläufige Urteile,
\ori{in so ferne sie als Grundsätze angenommen werden}. -- Ein jedes Vorurteil
ist als ein Prinzip irriger Urteile anzusehen und aus Vorurteilen entspringen
nicht vorurteile, sondern irrige Urteile} \mkbibparens{\cite[][A
116]{Kant:ImmanuelKantsLogik1977}, \cite[][IX:
75.24--27]{Kant:GesammelteWerke1900ff.}}.}

Die entsprechende allgemeine Aussage können wir nach \name[Immanuel]{Kant} nun
aber aus der Vernunft oder aus der Erfahrung haben; und somit gibt es 
Vernunftprinzipien und eben auch empirische
Prinzipien.\footnote{\cite[Vgl.][B~868]{Kant:KritikderreinenVernunft2003},
\cite[][III: 543.24--26]{Kant:GesammelteWerke1900ff.}: \enquote{Alle Philosophie
aber ist entweder Erkenntnis aus reiner Vernunft, oder Vernunfterkenntnis aus
empirischen Prinzipien. Die erstere heißt reine, die zweite empirische
Philosophie.}} Im ersten Fall läge ein \enquote{realer Vernunftgebrauch}
zugrunde, in dem die Vernunft selbst unabhängig von jeder Erfahrung (Begriffe
und) Grundsätze aufstellt, im zweiten Fall handelte es sich nur um einen
logischen Vernunftgebrauch, in dem die Vernunft ihr gegebene allgemeine
Erkenntnisse miteinander verknüpft, ohne selbst auf den konkreten Inhalt bezogen
zu sein.\footnote{\cite[Vgl.][B 355]{Kant:KritikderreinenVernunft2003},
\cite[][III: 237.26--30]{Kant:GesammelteWerke1900ff.}.} Entsprechend bezeichnet
\name[Immanuel]{Kant} den Begriff \enquote{Prinzip} als zweideutig:
Eine allgemeine Aussage könne relativ zu ihrem Gebrauch ein Prinzip sein,
insofern sie den Obersatz eines Vernunftschlusses bildet, und sie könne
\enquote{an sich selbst} und ihrem \enquote{eigenen Ursprung nach} ein Prinzip
sein.\footnote{\Cite[Vgl.][B 356]{Kant:KritikderreinenVernunft2003},
\cite[][III: 238.12--15]{Kant:GesammelteWerke1900ff.}.} Nur letztere nennt er
\enquote{schlechthin Prinzipien}, während \enquote{alle allgemeine Sätze
überhaupt komparative Prinzipien heißen
können.}\footnote{\Cite[][B~358]{Kant:KritikderreinenVernunft2003}, \cite[][III:
359.7--8]{Kant:GesammelteWerke1900ff.}.} Im Architektonikkapitel der
\titel{Kritik der reinen Vernunft} scheint mir dann wiederum gar kein Zweifel
daran zu bestehen, dass \name[Immanuel]{Kant} Prinzipien im Sinne hat, die
gänzlich \emph{a priori} sind.\footnote{Ein weiteres Indiz findet sich zu Beginn
der \titel{Kritik der Urteilskraft} innerhalb der Einteilung des
\singlequote{Gebiets der Philosophie}:
\enquote{So weit Begriffe a priori ihre Anwendung haben, so weit reicht der
Gebrauch unseres Erkenntnisvermögens nach Prinzipien und mit ihm die
Philosophie} \mkbibparens{\cite[][xvi]{Kant:KritikderUrteilskraft2009},
\cite[][V: 174.3--5]{Kant:GesammelteWerke1900ff.}}.} Gerade um dies
herauszustellen bemüht er sich um mehrfache Abgrenzung der \enquote{reinen Philosophie} auch gegen die
\enquote{empirische Philosophie} und ihre \enquote{empirischen
Prinzipien}.\footnote{\enquote{Alle Philosophie aber ist entweder Erkenntnis aus
reiner Vernunft, oder Vernunfterkenntnis aus empirischen Prinzipien. Die erstere
heißt reine, die zweite empirische Philosophie} \mkbibparens{\cite[][B
868]{Kant:KritikderreinenVernunft2003}, \cite[][III:
543.24--26]{Kant:GesammelteWerke1900ff.}}.} Prinzipien im eigentlichen Sinn
dieses Wortes sind \emph{a priori}\footnote{In der \titel{Jäsche-Logik} heißt
es: \enquote{Wir haben die Vernunfterkenntnisse für Erkenntnisse aus Prinzipien
erklärt; und hieraus folgt: daß sie a priori sein müssen}
\mkbibparens{\cite[][A~21]{Kant:ImmanuelKantsLogik1977}, \cite[][IX:
22.33--34]{Kant:GesammelteWerke1900ff.}}.} und darin unterscheidet sich
\name[Immanuel]{Kant}s Begriff der \emph{cognitio ex principiis} von
\authorcite{Wolff:Psychologiaempirica1968}s Begriff der \emph{cognitio
philosophica}. Die empirischen Prinzipien, von denen \name[Immanuel]{Kant}
spricht sind in ihrer Verwendung Prinzipien, aber doch keine Erkenntnisse aus
Prinzipien, sondern gerade aus Erfahrung. Nun ist es möglich, etwas auf der
Grundlage solcher empirischer Prinzipien zu erkennen, was dann in gewisser
Hinsicht eine Erkenntnis \singlequote{\emph{aus} (empirischen) Prinzipien} ist.
So kann derjenige, der das Fundament seines Hauses untergräbt, aus empirischen
Prinzipien und damit in einer gewissen Hinsicht \emph{a priori} wissen, dass es
einstürzt.\footnote{\cite[Siehe hierzu][B 2]{Kant:KritikderreinenVernunft2003},
\cite[][III: 28.7--18]{Kant:GesammelteWerke1900ff.}.} Doch wie er es doch nicht
gänzlich \emph{a priori} wissen kann, so ist seine Erkenntnis auch nicht
gänzlich \emph{ex principiis}. Denn die empirischen Erkenntnisse, die er als
Prinzipien verwenden kann, müssen doch zunächst in der Erfahrung gegeben
(\emph{ex datis}) sein.

Eine alternative Deutung der Unterscheidung historischer und empirischer
Erkenntnis entwickelt \authorfullcite{Kater:PolitikRechtGeschichte1999}, der überdies
\name[Immanuel]{Kant}s Unterscheidung von objektiv und subjektiv betrachteten
Erkenntnissen berücksichtigt. Nach \authorcite{Kater:PolitikRechtGeschichte1999}
zeichnet sich die empirische Erkenntnis -- ebenso wie die rationale Erkenntnis
-- durch Objektivität aus, weil sie aus eigener Erfahrung -- die
rationale Erkenntnis aus eigener Vernunft -- entspringe. Die historische
Erkenntnis hingegen sei bloß mitgeteilt, wie \name[Immanuel]{Kant} der
\titel{Jäsche-Logik} zufolge die historische Gewissheit als abgeleitete
Gewissheit aus \emph{fremder} Erfahrung der Gewissheit aus \emph{eigener}
Erfahrung gegenüberstellt\footnote{\cite[Vgl.][A
107\,f.,]{Kant:ImmanuelKantsLogik1977}
\cite[][IX: 71.3--7]{Kant:GesammelteWerke1900ff.}. Man beachte jedoch, dass an
dieser Stelle zum einen von historischer \emph{Gewissheit}, nicht aber von
historischer \emph{Erkenntnis} gesprochen wird und dass zum anderen die
historische Gewissheit als abgeleitete \emph{empirische} (\enquote{derivative
empirica}) Gewissheit der empirischen Gewissheit \emph{per se} gar nicht
entgegengestellt, sondern der \emph{ursprünglichen} empirischen
(\enquote{originarie empirica}) Gewissheit gegenübergestellt und damit doch noch
immer als eine von zwei Formen empirischer Gewissheit betrachtet wird.}; und so
wird auch plausibel, warum es sowohl von empirischer als auch von rationaler
Erkenntnis wiederum historische Erkenntnis geben kann. Nach diesem Modell
unterscheidet \name[Immanuel]{Kant} also zunächst zwischen objektiver und (bloß)
subjektiver Erkenntnis und unterteilt die objektive Erkenntnis wiederum in
rationale Erkenntnis (eigener Vernunftgebrauch) und empirische Erkenntnis
(eigene Erfahrung).\footnote{\cite[Vgl.][141]{Kater:PolitikRechtGeschichte1999}:
\enquote{Ist der Ursprung der Erkenntnis objektiv, beruht sie also auf
unmittelbarer Sinneswahrnehmung oder eigener Vernunftanstrengung. Die Erkenntnis
subjektiven Ursprungs ist die je mitgeteilte Erkenntnis, und zwar unabhängig
davon, ob sie auf fremder Erfahrung beruht oder auf fremder Vernunft. Damit ist
sichergestellt, daß eine mitgeteilte Vernunfterkenntnis aus Sicht dessen, dem
sie mitgeteilt worden ist, keine Erkenntnis \ori{der} Vernunft darstellt.}}


Diese grundlegende Unterscheidung ließe sich auch auf \name[Immanuel]{Kant}s
Unterscheidung subjektiv und objektiv zureichenden Fürwahrhaltens und die
Bezeichnung des lediglich subjektiv, nicht aber objektiv zureichenden
Fürwahrhaltens als \emph{Glauben} sowie des sowohl subjektiv als auch objektiv
zureichenden Fürwahrhaltens als \emph{Wissen} beziehen.\footnote{\cite[Vgl.][B
850]{Kant:KritikderreinenVernunft2003}, \cite[][III:
532.36--533.5]{Kant:GesammelteWerke1900ff.}, sowie \cite[][A 98--107]{Kant:ImmanuelKantsLogik1977},
\cite[][IX: 65.33--70.31]{Kant:GesammelteWerke1900ff.}.} Da die Bezeichnung
testimonialer Erkenntnis als \enquote{historischer Glaube} im 18. Jahrhundert durchaus
verbreitet war, liegt es nahe, diese Unterscheidung auch in die Deutung des
Begriffs historischer Erkenntnis hinein zu
lesen.\footnote{\authorfullcite{Kater:PolitikRechtGeschichte1999} stellt diese
Verbindungen nicht her. Er bezieht den subjektiven Ursprung nicht auf ein bloß
subjektiv zureichendes Fürwahrhalten und erkennt auch, dass
\name[Immanuel]{Kant} keinen Unterschied bzgl. der Evidenz von Wahrnehmungs-
und testimonialem Wissen behauptet. Dennoch deutet
\authorcite{Kater:PolitikRechtGeschichte1999} den Begriff \enquote{historisch}
im Sinne von \enquote{testimonial} und behauptet, solcher Erkenntnis komme nach
\name[Immanuel]{Kant} nur ein subjektiver, aber kein objektiver Ursprung zu,
obwohl er selbst zuvor sieht, dass \name[Immanuel]{Kant} die mittelbare
Erfahrung wie jede andere Erfahrung auch ansieht und dem historischen allein das
\emph{a priori} erkennbare
entgegensetzt \parencite[siehe][140]{Kater:PolitikRechtGeschichte1999}.} Der
Begriff historischer Erkenntnis wird hier also im Sinne testimonialer Erkenntnis
gedeutet und als (bloß) subjektiv abgetan, hinterher aber durch Überlegungen zur
Bonität des Zeugen partiell
rehabilitiert.\footnote{\cite[Vgl.][143--146]{Kater:PolitikRechtGeschichte1999}.}


Dem widerspricht jedoch der Status, der testimonialen Erkenntnissen bei
\name[Immanuel]{Kant} zugeschrieben wird (siehe Kapitel
\ref{Absatz:AufklaerungundZugangsInternalismus}). \name[Immanuel]{Kant}
negiert jeden graduellen wie auch jeden qualitativen Unterschied zwischen dem
Fürwahrhalten auf Grund eigener Erfahrung und dem auf Grund fremder
Erfahrung.\footnote{\cite[Vgl.][A 103]{Kant:ImmanuelKantsLogik1977}, \cite[][IX:
69.2--4]{Kant:GesammelteWerke1900ff.}.} In der \titel{Jäsche-Logik}
wird die historische Gewissheit explizit unter der Rubrik \enquote{Wissen}
thematisiert.\footnote{\cite[Vgl.][A 108]{Kant:ImmanuelKantsLogik1977},
\cite[][IX: 71.6--7]{Kant:GesammelteWerke1900ff.}.} Auch in \titel{Was heißt:
sich im Denken orientieren?} wird gesagt, dass testimoniale Erkenntnisse völlig
zu Recht als Wissen bezeichnet werden.\footnote{\cite[Vgl.][A
319]{Kant:Washeisst:SichimDenkenorientieren?1977},
\cite[][VIII: 141.10--17]{Kant:GesammelteWerke1900ff.}.} Wissen ist aber ein
Fürwahrhalten, das subjektiv \emph{und objektiv} zureichend
ist.\footnote{\cite[Vgl.][B 850]{Kant:KritikderreinenVernunft2003},
\cite[][III: 533.4--5]{Kant:GesammelteWerke1900ff.}, sowie
\cite[][A 107]{Kant:ImmanuelKantsLogik1977},
\cite[][IX: 70.27--31]{Kant:GesammelteWerke1900ff.}.} Wenn historische
Erkenntnis also Wissen ist oder sein kann, dann kann sie nicht dadurch
charakterisiert werden, dass man sagt, sie sei eine nur subjektiv zureichende
Erkenntnis. Und auch die Deutung, sie sei nur subjektiven, nicht aber objektiven
Ursprungs, konfligiert mit der kategorischen Gleichstellung von Erkenntnissen
aus eigener und solchen aus fremder Erfahrung.


In der Tat finden wir bei \name[Immanuel]{Kant} zunächst eine grundlegende
Unterscheidung, die sich des Begriffspaares
\enquote{subjektiv}/\enquote{objektiv} bedient und die begriffliche
Unterscheidung zwischen historischen und empirischen Erkenntnissen erst
ermöglicht. Erst im Anschluss unterscheidet er  rationale von historischen sowie
rationale von empirischen Erkenntnissen -- und setzt in der \titel{Kritik der
reinen Vernunft} das Rationale dem Empirischen entgegen. Daraus folgt gerade
deswegen nicht die Identität des Empirischen und des Historischen, weil beide
Unterscheidungen auf verschiedenen \singlequote{Ebenen} liegen. Und eben diese
unterschiedlichen \singlequote{Ebenen} werden mit Hilfe der Ausdrücke
\enquote{subjektiv} und \enquote{objektiv} beschrieben. Den deutlichsten
Ausdruck findet dies in der {\jaeschelogik}:
\begin{quote}
 Man kann nämlich Erkenntnisse unterscheiden\\
 1) nach ihrem \ori{objektiven} Ursprunge, d.\,i.\ nach den Quellen, woraus eine
Erkenntnis allein möglich ist. In dieser Rücksicht sind alle Erkenntnisse
entweder \ori{rational} oder \ori{empirisch};\\
 2) nach ihrem \ori{subjektiven} Ursprunge, d.\,i.\ nach der Art, wie eine
Erkenntnis von den Menschen kann erworben werden. Aus diesem letztern
Gesichtspunkte betrachtet sind die Erkenntnisse entweder \ori{rational} oder
\ori{historisch}, sie mögen an sich entstanden sein, wie sie wollen. Es kann
also \ori{objektiv} etwas ein Vernunfterkenntnis sein, was \ori{subjektiv} doch
nur historisch ist.\footnote{\cite[][A~20f.,]{Kant:ImmanuelKantsLogik1977}
\cite[][IX: 22.11--20]{Kant:GesammelteWerke1900ff.}. Dies fügt sich auch sehr
gut in die -- weitaus weniger ausführlichen und daher deutungsoffeneren -- Ausführungen der ersten Kritik ein:
\enquote{Ich verstehe hier aber unter Vernunft das ganze obere
Erkenntnisvermögen, und setze also [objektiv betrachtet, A.\,G.] das Rationale
dem Empirischen entgegen.\\
 Wenn ich [hingegen; A.\,G.] von allem Inhalte der Erkenntnis, objektiv
betrachtet, abstrahiere, so ist alles Erkenntnis, subjektiv, entweder historisch
oder rational} (\cite[][B~863f.,]{Kant:KritikderreinenVernunft2003}
\cite[][III: 540.27--31]{Kant:GesammelteWerke1900ff.}).}
\end{quote}
Wir dürfen dies nicht so verstehen, dass einige Erkenntnisse einen objektiven,
andere lediglich einen subjektiven Ursprung hätten. \emph{Jede} Erkenntnis, die
als konkretes Wissen eines Subjekts von einem Sachverhalt vorliegt, kann sowohl
hinsichtlich ihres subjektiven als auch hinsichtlich ihres objektiven Ursprungs
betrachtet werden. Jede Erkenntnis hat folglich sowohl einen objektiven als auch
einen subjektiven Ursprung. Gerade \emph{nicht} unterschieden werden hier
subjektive von objektiven Erkenntnissen oder ein nur subjektiv zureichendes
Fürwahrhalten (Glauben) von einem solchen, das auch objektiv zureichend ist
(Wissen).

Eine Erkenntnis subjektiv betrachten heißt, auf das jeweilige
Verhältnis des Denkenden zu dieser Erkenntnis achten. Hierbei interessiert nicht
der Inhalt der Erkenntnis, sondern aus welchen Gründen jemand diese für wahr
hält. Sie ist dabei historisch, wenn sie in einem konkreten Fall für wahr
gehalten wird, weil die in Rede stehende Person selbst wahrgenommen oder von
anderen erfahren hat, dass etwas der Fall ist. Zwischen eigener und fremder Erfahrung existiert aus
\name[Immanuel]{Kant}s Perspektive kein relevanter Unterschied\footnote{Dies gilt für
seine publizierten Werke. In den Logikvorlesungen unterschied \name[Immanuel]{Kant} sehr
wohl zwischen eigener und fremder Erfahrung und auch zwischen Erfahrungen
erster, zweiter und dritter Hand und diskutiert auch die Frage, wie die
Glaubwürdigkeit sukzessiv abnimmt (bei \enquote{subordinirten Zeugen},
\enquote{mündliche Überlieferung}) oder durch das übereinstimmende Zeugnis
Vieler \enquote{coordinirte[r] Zeugen} (\enquote{das öffentliche Gerüchte})
zunimmt \mkbibparens{vgl. \cite{Kant:LogikPhilippi1966}, \cite[][XXIV:
450.20-28]{Kant:GesammelteWerke1900ff.}}.}, beides ist subjektiv betrachtet
historische Erkenntnis.

Eine Erkenntnis objektiv betrachten heißt hingegen, von dem
Verhältnis eines besonderen Subjekts zu ihr gerade absehen und
fragen: Was für eine Art von Erkenntnis liegt vor? Ebenso wie wir unter
\enquote{Wissenschaft} einerseits die Institution und das Gesamt an vorliegendem
Wissen und andererseits die je eigene Fähigkeit und Expertise verstehen
können,\footnote{Diese Unterscheidung spielt in der deutschen
Aufklärungsphilosophie durchaus eine Rolle. Als Belege können die bereits oben
angeführten Erläuterungen von
\authorcite{Stiebritz:ErlaeuterungderWolffschenVernuenfftigenGedanckenvonGottderWeltundderSeeledesMenschenauchallenDingenueberhaupt1999}
zu \authorcite{Wolff:Discursuspraeliminarisdephilosophiaingenere1996}s Logik
dienen
\mkbibparens{\cite[vgl.][\S~44]{Stiebritz:ErlaeuterungenderVernuenftigenGedanckenvondenKraefftendesmenschlichenVerstandesWolffs1977},
sowie oben Kapitel \ref{paragraph:wolffswarnung},
S.~\pageref{Anmerkung:StiebritzZuSubiectiveundObiective}}. Siehe als weiteren
Beleg \cite[][4]{Mendelssohn:UeberdieFrage:washeisstaufklaeren?2008}.} so können
wir \enquote{Erkenntnis} einerseits als objektiv vorliegenden Teil der Wissenschaft
und andererseits als mein je eigenes Urteil verstehen. Im ersten Fall
betrachten wir eine Erkenntnis objektiv, im zweiten Falle subjektiv. Und auch im
Falle objektiver Erkenntnisse lässt sich fragen: Handelt es sich um eine
Erkenntnis \emph{a priori}, eine Vernunfterkenntnis? Oder handelt es sich um
eine Behauptung, deren Wahrheit ausschließlich durch Rekurs auf (eigene oder
fremde) Sinneswahrnehmung auszumachen ist, also eine Erkenntnis \emph{a
posteriori}? Wir fragen dann nicht mehr, ob dieser oder jener etwas ohne Rekurs
auf Erfahrungen oder Mitteilungen erkannte, sondern ob die Erkenntnis selbst
derart ist, dass es \emph{möglich} und \emph{angemessen} ist, sie ohne Hilfe der
Wahrnehmung zu erkennen. Der Unterschied zwischen rationalen und empirischen
Erkenntnissen fällt also mit der Unterscheidung zwischen Erkenntnissen oder
Urteilen \emph{a priori} und \emph{a posteriori} zusammen, die ebenso nur das
Urteil an sich, nicht aber die besondere Situation und die zufälligen
Kompetenzen des gerade Urteilenden berücksichtigt.


Eine Erkenntnis, die subjektiv betrachtet rationale Erkenntnis sein kann, ist
immer auch objektiv betrachtet rationale Erkenntnis. Aber während es
nicht möglich ist, von objektiv empirischer Erkenntnis subjektiv rationale
Erkenntnis zu erlangen (man müsste aus reiner Vernunft erkennen, was nur
empirisch erkennbar ist), lässt sich doch subjektiv historische Erkenntnis von
objektiv rationalen Erkenntnissen erwerben, insbesondere indem man sich sagen
lässt, was andere rational erkannt haben.\footnote{\authorcite{Wolff:Psychologiaempirica1968}
thematisiert außerdem den Fall, dass wir mathematische Sätze anhand der
Erfahrung belegen
\parencite[vgl.][\S~19]{Wolff:Discursuspraeliminarisdephilosophiaingenere1996}.
Wir könnten beispielsweise den Satz des \singlename{Pythagoras} dadurch stützen
wollen, dass wir verschiedene rechtwinklige Dreiecke ausmessen und induktiv darauf schließen,
dass $a^2 + b^2 = c^2$ allgemein gilt. Ähnlich verhält es sich mit den
Beispielen, die \authorcite{Kripke:NameundNotwendigkeit1981} anführt.
Dieser vermengt in seiner Kritik an der Auffassung, die Klassen apriorischer und
empirischer Erkenntnisse seien disjunkt, beide Ebenen, insofern er durch das
Anführung von Erkenntnissen, die objektiv rational, aber subjektiv historisch
sind, die Möglichkeit gegeben sieht, empirische Erkenntnis mathematischer
Wahrheiten zu haben \parencite[vgl.][\pno
45\,ff.]{Kripke:NameundNotwendigkeit1981}.} \begin{figure}[htb]
\begin{minipage}[t]{\textwidth}
\centering
\begin{tikzpicture}[edge from parent fork down,
level 1/.style={level distance=1.5cm},
level 2/.style={level distance=2.5cm},
sibling distance=2.5cm,
every node/.style={rectangle,draw=black,fill=gray!25, thin, inner sep=0.5em, minimum size=0.5em, align=center},
edge from parent/.style={->,thick,draw},
mylabel/.style={draw=none, fill=none, text width=5cm,text centered, inner sep=0.5em, anchor=base} ]
\node {Erkenntnis}
	child {node (or) {rational}
		child {node (sr) {rational}} edge from parent[-,thin]}
	child {node[fill=none,draw=none] {(objektiv)} edge from parent[draw=none]
		child {node[fill=none,draw=none] {(subjektiv)} edge from
		parent[draw=none]}} child {node (oe) {empirisch}
		child {node (sh) {historisch}} edge from parent[-,thin]}
;
\draw [->,decorate,decoration={snake,post
length=1mm,amplitude=.4mm,segment length=2mm},thick] (or.south) to
(sh.north);
\end{tikzpicture}
  \caption{Objektive und subjektive Differenzierung von Erkenntnissen nach
  Immanuel
  \name[Immanuel]{Kant}}\label{abbildung:ZeichnungErkenntnisartennachKant.pdf}
\end{minipage}
\end{figure}
Eine graphische Übersicht dazu gibt Abbildung
\ref{abbildung:ZeichnungErkenntnisartennachKant.pdf}, in der der problematische
Fall einer historischen Erkenntnis von Vernunftwahrheiten hervorgehoben ist.
Wir verstehen damit auch die Aussage der {\jaeschelogik} besser, die ich
bereits weiter oben\footnote{Siehe Kapitel
\ref{Zitat:Kant:TestimonialesWissenVernunftErfahrung}, Seite
\pageref{Zitat:Kant:TestimonialesWissenVernunftErfahrung}.} anführte:
\begin{quote}
 Wenn wir in Dingen, die auf Erfahrungen und Zeugnissen beruhen, unsre
Erkenntnis auf das Ansehen andrer Personen bauen: so machen wir uns dadurch
keiner Vorurteile schuldig; denn in Sachen dieser Art muß, da wir nicht alles
selbst erfahren, und mit unserm eigenen Verstande umfassen können, das Ansehen
der Person die Grundlage unsrer Urteile sein. -- Wenn wir aber das Ansehen
anderer zum Grunde unsers Fürwahrhaltens in Absicht auf Vernunfterkenntnisse
machen: so nehmen wir diese Erkenntnisse auf bloßes Vorurteil
an.\footnote{\cite[][A~120]{Kant:ImmanuelKantsLogik1977}, \cite[][IX:
77.31--78.5]{Kant:GesammelteWerke1900ff.}.}
\end{quote}
Testimoniale Erkenntnis ist eine Unterart historischer Erkenntnisse und wer
testimoniales Wissen um die Vernunftwahrheiten hat, die ein anderer erkannte,
der hat historische Erkenntnis objektiv rationaler Erkenntnisse. Die
Grundforderung der Aufklärung wertet testimoniales Wissen bezüglich
rationaler Erkenntnisse ab, erlaubt sie aber bei empirischen Erkenntnissen. Ob
wir etwas aus eigener oder fremder \emph{Erfahrung} haben, das markiert aus
Sicht des \enquote{sapere aude!} keinen relevanten Unterschied, wohl aber, ob
wir etwas aus eigener oder fremder \emph{Vernunft} haben.

Mit der Kritik an der historischen Erkenntnis von
Vernunftwahrheiten schließt sich \name[Immanuel]{Kant}
\authorcite{Descartes:OeuvresdeDescartes1983}' Kritik an der
Büchergelehrsamkeit an, aus der keine echte Wissenschaft resultiere, weil sie
bloß den Erwerb historischer Kenntnisse beinhalte.\footnote{Siehe dazu Kapitel
\ref{subsection:DescartesKritikantestimonialemWissen}, insb. ab
S.~\pageref{Abschnitt:DescartesundhistorischeKenntnisse}.} Wie auch schon
\authorcite{Wolff:Discursuspraeliminarisdephilosophiaingenere1996} 
unterscheidet sich \name[Immanuel]{Kant} darin von
\authorcite{Descartes:OeuvresdeDescartes1983}, dass er den Begriff
der \singlequote{bloß historischen Kenntnis} ausführlich ausarbeitet und dabei
nicht historische Erkenntnisse \emph{per se} ablehnt, sondern nur solche von
(objektiv) philosophischen (oder allgemeiner: rationalen)
Erkenntnissen. Im Unterschied zu
\authorcite{Wolff:Discursuspraeliminarisdephilosophiaingenere1996} grenzt er
den Bereich der philosophischen Erkenntnisse auf die Erkenntnisse \emph{a
priori} ein, zu deren Rechtfertigung wir nicht auf Erfahrung verweisen müssen.
Damit ist es ein weitaus engerer Bereich, der für Fragen der Mündigkeit
relevant ist, als noch bei \authorcite{Wolff:Discursuspraeliminarisdephilosophiaingenere1996}.
Übernähme er dessen Konzeption ohne Änderung, so wäre nur derjenige schlechthin
aufgeklärt, der in allen Bereichen der Wissenschaft eigene Kompetenz besitzt. In der Tat aber ist der
Bereich dessen, was für Mündigkeit relevant ist, viel geringer. Es ist
plausibel zu sagen, dass auch derjenige mündig sein kann, der Aussagen der
Quantenmechanik oder der Allgemeinen Relativitätstheorie weder zu erläutern noch
zu begründen in der Lage ist, während es eine zwingende Voraussetzung von
Mündigkeit ist, in Fragen des moralisch Richtigen selbst kompetent urteilen zu
können. Fragen der Moral sind nicht nur der Form nach philosophische (also
diskursive Vernunft-) Erkenntnisse -- während quantenmechanisches Wissen
(zumindest überwiegend, vielleicht aber auch zur Gänze) empirisches Wissen
darstellt --; die Moral ist darüber hinaus auch von grundlegender Bedeutung für
das, was \name[Immanuel]{Kant} die Bestimmung des Menschen nennt (siehe Kapitel
\ref{chapter:AufklaerungundWissenschaft}), während wir dieser Bestimmung auch
dann selbstbestimmt folgen können, wenn wir über kein tiefgreifendes
physikalisches Wissen verfügen.

Nach \name[Immanuel]{Kant}s Aufklärungsprogrammatik ist es problematisch, von
philosophischen Erkenntnissen \emph{bloß} testimoniales Wissen zu haben, weil dadurch
das Selbstdenken durch bloßes \singlequote{Nachbilden} ersetzt
wird.\footnote{\cite[Vgl.][B 864]{Kant:KritikderreinenVernunft2003},
\cite[][III: 541.8]{Kant:GesammelteWerke1900ff.}. \name[Immanuel]{Kant} spricht
dort von dem \enquote{nachbildende[n] Vermögen}, welches \enquote{nicht das
erzeugende} ist.} Dies ist die primäre Konkretisierung des \enquote{Sapere
aude!}: Man soll nicht dasjenige, was aus dem je eigenen Gebrauch der Vernunft
erkannt werden kann, stattdessen als genuin testimoniales Wissen übernehmen,
also als Wissen, dessen Korrektheit man unabhängig von der Autorität eines
Anderen nicht beurteilen kann. Und darauf bezieht sich die an
\authorcite{Wolff:Psychologiaempirica1968} angelehnte Abneigung gegen eine bloß historische Erkenntnis philosophischen Wissens hinter \name[Immanuel]{Kant}s
Diktum, man könne und solle nicht Philosophie, sondern Philosophieren
lernen.\footnote{\cite[Vgl.][A~5]{Kant:NachrichtvonderEinrichtungseinerVorlesungenindemWinterhalbenjahrevon1765-17661900ff.},
\cite[][II: 306.30--32]{Kant:GesammelteWerke1900ff.}, sowie
\cite[][B~865]{Kant:KritikderreinenVernunft2003}, \cite[][III:
541.34--542.2]{Kant:GesammelteWerke1900ff.}. Siehe zur Vorgeschichte dieser
Redeweise die Überlegungen von Norbert
\textcite[][]{Hinske:UrspruenglicheEinsichtundVersteinerung1995}, der
ebenfalls
\authorfullcite{Wolff:Discursuspraeliminarisdephilosophiaingenere1996}s drei
Erkenntnisarten als wichtige Grundlage ausmacht
\parencite[vgl.][19--23]{Hinske:UrspruenglicheEinsichtundVersteinerung1995}.
Dass wir häufig und mit einer gewissen Notwendigkeit den Unterricht mit einer
solchen historischen Darstellung wissenschaftlicher Lehren beginnen, worauf erst
die Einübung in wissenschaftliche Techniken, das eigene Verstehen
wissenschaftlicher Theorien und die selbständige Teilnahme an
Forschungstätigkeiten folgt, ist möglicherweise Hintergrund der zunächst schwer
verständlichen Aussage: \enquote{In einer Wissenschaft \ori{wissen} wir oft nur
die \ori{Erkenntnisse}, aber nicht die dadurch
\ori{vorgestellten Sachen}; also kann es eine Wissenschaft von demjenigen geben,
wovon unsre Erkenntnis kein Wissen ist}
(\cite[][A~110]{Kant:ImmanuelKantsLogik1977}, \cite[][IX:
72.8--10]{Kant:GesammelteWerke1900ff.}).}

\name[Immanuel]{Kant} nennt ein System diskursiver Vernunfterkenntnisse auch
\emph{Metaphysik}\footnote{\enquote{[D]as System der reinen Vernunft
(Wissenschaft), die ganze (wahre sowohl als scheinbare) philosophische
Erkenntnis aus reiner Vernunft im systematischen Zusammenhange {\punkt} heißt
\ori{Metaphysik}} \mkbibparens{\cite[][869]{Kant:KritikderreinenVernunft2003},
\cite[][III: 543.29--544.2]{Kant:GesammelteWerke1900ff.}}.} und stellt damit --
ausgerechnet -- eine Disziplin in den Mittelpunkt seines Aufklärungsprogramms,
die eben durch die Aufklärung in Verruf geraten zu sein
scheint.\footnote{Dass Metaphysik nicht im Gegensatz zur Aufklärung stehen
muss, sondern gerade deren Anliegen sein kann, ist freilich bekannt.
\authorfullcite{Beiser:Kantsintellectualdevelopment1992} etwa schreibt:
\enquote{Only metaphysics, the young Kant believed, could rescue the
\ori{Aufklärung}'s faith in reason from the attacks of the Pietists. Only it could
provide a rational justification for our moral and religious beliefs, and thus a
middle path between skepticism and fideism}
\parencite[][30]{Beiser:Kantsintellectualdevelopment1992}. Gerade das Bemühen
um eine Religion nach Maßgabe der Vernunft motiviert metaphysische Bemühungen
\parencite[vgl.][]{Schmidt-Biggemann:MetaphysikalsProvokation2010}.
Gegen die Metaphysik wendet sich in der deutschen Aufklärungsphilosophie
insbesondere der Versuch von \name[Christian]{Thomasius}, den Einfluss der
Theologie auf die Jurisprudenz zurückzuweisen
\parencite[vgl.][602]{Hunter:ChristianThomasiusandtheDesacralizationofPhilosophy2000}.
Zu einer \singlequote{bürgerlichen} Aufklärung in dieser Traditionslinie siehe
\cite[][passim.]{Hunter:RivalEnlightenments2001}.} Dies überrascht, insofern
Metaphysik landläufig gerade nicht mit der Aufklärung in Verbindung gebracht
wird und \name[Immanuel]{Kant} sie mit der Vernunftkritik gerade zu überwinden
scheint. Doch zunächst handelt es sich nicht um die (positive) Aufforderung, Metaphysik
selbst zu betreiben, sondern um die (negative) Warnung, keine metaphysischen
Behauptungen von anderen zu übernehmen. Es ist ja denkbar, dass wir auf
Metaphysik auch weitestgehend verzichten können. Wir müssten sie dafür zunächst
als solche identifizieren und jede Übernahme metaphysischer Überzeugungen auf
der Grundlage von Mitteilungen verweigern. Ich werde im folgenden zunächst
\name[Immanuel]{Kant}s Begriff der Metaphysik skizzieren, damit überhaupt
feststeht, um welchen Themenbereich es der Aufklärungsforderung geht. Ich werde
dabei in mehreren Schritten vorgehen: Zum ersten ist der Begriff der Metaphysik zu konkretisieren und sein enges Verhältnis zum Begriff
der Philosophie darzulegen. Es wird sich zeigen, dass \name[Immanuel]{Kant}s
gegenüber seinen Vorgängern deutlich veränderter Metaphysikbegriff die
methodischen Überlegungen aus diesem Kapitel mit den inhaltlichen Überlegungen
in Kapitel \ref{chapter:AufklaerungundWissenschaft} verbindet.

\section{Selbstdenken als
Metaphysik}\label{section:MetaphysikausderPerspektivedesMenschen}

Ob \name[Immanuel]{Kant} als Metaphysiker zu lesen ist, gehört zu den
Fragestellungen, welche die \name[Immanuel]{Kant}forschung der ersten Hälfte des
20. Jahrhunderts prägten.\footnote{Siehe dazu den Überblick in
\cite{Funke:DieDiskussionumdiemetaphysischeKantinterpretation1976}.} Man kann
aber kaum behaupten, dass die Diskussion zu einem abschließenden Ergebnis
gekommen wäre, denn noch immer ist unklar, wie sich \name[Immanuel]{Kant}s
Philosophie zur Metaphysik verhält.\footnote{Siehe dazu das schon etwas ältere,
aber noch immer aktuelle Urteil von
\textcite[vgl.][372]{Walsh:KantandMetaphysics1976}.} Einige Interpreten sehen
das Problem darin, dass der Ausdruck \enquote{Metaphysik} bei
\name[Immanuel]{Kant} keine einheitliche Bedeutung
habe.\footnote{\cite[Vgl.
bspw.][\pno~26\,f.]{Beiser:Kantsintellectualdevelopment1992}.
\authorfullcite{Brandt:KantalsMetaphysiker1990} schreibt: \enquote{Ein wichtiges Ziel der folgenden Ausführungen ist es zu zeigen, daß eine Pauschalvorstellung von \ori{der} Metaphysik in \ori{der} Kantischen Philosophie nur die Kreation des
Interpreten sein kann -- in den Werken, die die Quelle unserer Erkenntnisse
bilden, läßt sich eine einheitliche Vorstellung von Metaphysik und von \ori{dem}
Problem \ori{der} Metaphysik nicht finden}
\parencite[][77]{Brandt:KantalsMetaphysiker1990}.}

Eine wichtige Rolle spielen möglicherweise diachrone Unterschiede in der
Auffassung \name[Immanuel]{Kant}s zu Begriff, Wesen und Möglichkeit von
Metaphysik. So nennt \authorfullcite{Beiser:Kantsintellectualdevelopment1992}
vier verschiedene Phasen der zeitlichen Entwicklung \name[Immanuel]{Kant}s, die
mit unterschiedlichen Positionen zur Frage nach Metaphysik verbunden
sind.\footnote{Er spricht von der \enquote{\emph{period of infatuation}} (1746--1759),
der \enquote{\emph{period of disillusionment}} (1760--1766), der \enquote{\emph{period of partial
reconciliation}} (1766--1772) und der \enquote{\emph{period of divorce}} (1772--1780)
\parencite[vgl.][26]{Beiser:Kantsintellectualdevelopment1992}.} Und
\authorfullcite{Foerster:KantsMetaphysikbegriff1987} spricht von drei
Metaphysikbegriffen bei \name[Immanuel]{Kant} -- \singlequote{vor-kritisch},
\singlequote{kritisch} und
\singlequote{nach-kritisch}.\footnote{\cite[Vgl.][]{Foerster:KantsMetaphysikbegriff1987}.}
Nun interessiert hier der \singlequote{kritische} Begriff von
Metaphysik, insofern dieser auf der vernunftkritischen Analyse
unserer Endlichkeit basiert (der \singlequote{vor-kritische}
Begriff also nicht thematisch ist und das \titel{Opus postumum}
in dieser Arbeit nicht besprochen werden soll).

Was ist das neue an einer \singlequote{kritischen Metaphysik}?
\name[Immanuel]{Kant}s Explikation des Begriffs klingt einfach: Eine Metaphysik
ist kritisch genau dann, wenn ihr eine Untersuchung über die Möglichkeit von
Metaphysik vorangeht; andernfalls heißt sie dogmatisch.\footnote{Dogmatismus ist
die \enquote{Anmaßung, mit einer reinen Erkenntnis aus Begriffen (der
philosophischen), nach Prinzipien, so wie sie die Vernunft längst im Gebrauche
hat, ohne Erkundigung der Art und des Rechts, womit sie dazu gelanget ist,
allein fortzukommen. Dogmatism ist also das dogmatische Verfahren der reinen
Vernunft, \ori{ohne vorangehende Kritik ihres eigenen Vermögens}}
\mkbibparens{\cite[][B xxxv]{Kant:KritikderreinenVernunft2003}, \cite[][III:
21.27--30 ]{Kant:GesammelteWerke1900ff.}}. Siehe auch
\cite[][B 7]{Kant:KritikderreinenVernunft2003},
\cite[][III: 31.7--12]{Kant:GesammelteWerke1900ff.}.} Die Vernunftkritik
wiederum lässt sich unschwer als eine Analyse unserer
Endlichkeit\footnote{Darauf insistiert besonders 
\authorfullcite{Heidegger:KantunddasProblemderMetaphysik1965}
\parencite[vgl.][\S~5 und
passim]{Heidegger:KantunddasProblemderMetaphysik1965}.} und verstehen. Wir sind
endlich, insofern unsere Spontaneität nur durch Rückgriff auf unsere
Sinnlichkeit die eigene Tätigkeit mit Gehalt und Weltbezug ausstatten kann. Kein
Gedanke kann einen Inhalt haben, ohne dass er sich direkt oder indirekt auf
unsere Erfahrungen bezieht; ein Denken ohne Rezeptivität wäre entsprechend ein
\enquote{frictionless spinning in the
void}\footnote{\cite[][11]{McDowell:MindandWorld1994}.} (siehe hierzu Kapitel
\ref{chapter:endlichkeitmenschlichendenkens}).
Deswegen müssen sich metaphysische Begriffe und Erkenntnisse mindestens dadurch auf Sinnlichkeit
beziehen, dass sie für \emph{mögliche} Erfahrungen gelten.
Dies wiederum hat Auswirkungen auf den Begriff von Metaphysik.


\authorcite{Beckmann:ZurTransformationvonMetaphysikdurchKritik1985}
unterscheidet vier verschiedene mögliche Auswirkungen der Kritik auf
Metaphysik:
\begin{quote}
Kritik kann versuchen, Metaphysik \punkt
 entweder zu ersetzen \punkt
 oder zu zersetzen \punkt
 oder allererst zu ermöglichen \punkt
 oder prinzipiell zu verändern.
Im ersten Fall steht Kritik der Metaphysik gegenüber in der
Beziehung der \ori{Substitution}, im zweiten Fall in der Beziehung der \ori{Destruktion}, im dritten Fall
in der Beziehung der \ori{Konstitution} und im vierten Fall schließlich in der
Beziehung der \ori{Transformation}[.]\footnote{\cite[][292]{Beckmann:ZurTransformationvonMetaphysikdurchKritik1985}.}
\end{quote}
Dabei versuche \name[Immanuel]{Kant}s Vernunftkritik, Metaphysik
allererst zu ermöglichen oder zu \emph{konstituieren}. Damit scheint sich an den Inhalten der
Metaphysik nicht viel zu ändern, außer dass möglicherweise \emph{weniger}
metaphysische Wahrheiten erkannt werden können, als man zuvor fälschlicher Weise
annahm. Dagegen sehen auch einige Interpreten eine fundamentale Änderung.
\authorfullcite{Hoeffe:EthikohneundmitMetaphysik2007} etwa behauptet, gemessen
an dem überlieferten Begriff von Metaphysik sei \name[Immanuel]{Kant}s
Transzendentalphilosophie gar nicht
metaphysisch.\footcite[Vgl.][\pno~416\,f.]{Hoeffe:EthikohneundmitMetaphysik2007}
\begin{comment}
Und als paradigmatisch für eine verbreitete Auffassung kann eine Behauptung
\authorfullcite{Schnaedelbach:WirKantianer2005}s gelten:
\begin{quote}
Hinter dem \singlequote{alles zermalmenden Kant}, der die herkömmliche Metaphysik in Grund
und Boden kritisiert, wird allzu leicht der Begründer einer kritischen
Metaphysik der Natur und der Sitten übersehen. Das Neue daran ist, dass diese
neue Metaphysik ohne Gottesbezug auszukommen versucht -- ein für die
metaphysische Tradition unfassbarer Gedanke. {\punkt} der rationalen Theologie
ging es ja nicht um Religion, sondern um die Existenz einer Instanz, die mit der
Einheit der Welt auch ihre Erkennbarkeit sichert; so schien Metaphysik nur
möglich zu sein in der Perspektive des Gottesstandpunktes. Nach Kant gibt es
Metaphysik nur nach menschlichem Maß, das heißt nach Maßgabe der menschlichen
Vernunft, und daran vermochte die philosophische Episode des \singlequote{deutschen
Idealismus}, der Kant nicht angehört, nichts zu
ändern.\footcite[][837]{Schnaedelbach:WirKantianer2005}
\end{quote}
\name[Immanuel]{Kant}s \singlequote{kritische} Metaphysik unterscheide sich also
darin von anderen, dass sie nicht auf eine Erkenntnis Gottes als Basis unserer
Geltungsansprüche zurückgreife. Während beispielsweise
\authorcite{Descartes:OeuvresdeDescartes1983} glaubt, zunächst die
Existenz Gottes beweisen zu müssen, auf dieser Grundlage dann aber zeigen zu
können, dass alles, was wir klar und deutlich einsehen, unzweifelhaft wahr sein
müsse, verwerfe \name[Immanuel]{Kant} gerade den Gedanken einer solchen
\singlequote{metaphysischen} Fundierung. Damit entfalle aber auch die
gedankliche Möglichkeit eines Wissens \emph{sub specie aeternitatis}, eines
absoluten Wissens. Die Endlichkeit unseres Erkennens lässt sich demnach nicht
mehr durch metaphysische Überlegungen überwinden.
\end{comment}

Dies zeigt zur Genüge, dass der Metaphysikbegriff klärungsbedürftig ist, soll
die These von ihrem zentralen Stellenwert für die Forderung der Aufklärung nach
Mündigkeit gehaltvoll sein. Ich werde im folgenden zunächst
\name[Immanuel]{Kant}s Metaphysikbegriff rekonstruieren und sein Verhältnis zum Begriff
der Philosophie beleuchten (Kapitel \ref{paragraph:facettendesmetaphysikbegriffs}),
um anschließend zu zeigen, warum \name[Immanuel]{Kant} denkt, dass wir auf
Metaphysik nicht verzichten können (Kapitel
\ref{paragraph:DieUnverzichtbarkeitderMetaphysik}). Am Ende werde ich dann
zeigen, dass autonomes Denken tatsächlich darin besteht, \Revision{über
Kompetenz im Bereich der Metaphysik zu verfügen} (Kapitel
\ref{subsection:MetaphysikundAutonomie}). Freilich heißt dies nicht,
\Revision{über eine \singlequote{eigene} Metaphysik zu verfügen, die
  sich möglicherweise durch Originalität auszeichnen und sich von
derjenigen anderer unterscheiden soll.} Wir sollen im Bereich der einen
Metaphysik je selbst kompetent urteilen können.

\subsection{Kants
Metaphysikbegriff}\label{paragraph:facettendesmetaphysikbegriffs} Die
Philosophie der Neuzeit kennt keinen einheitlichen Metaphysikbegriff, sondern
vielfältige Verwendungsweisen von \enquote{Metaphysik} und
\enquote{metaphysisch}, die sich in synchroner Betrachtung zwischen den
Regionen, Strömungen und auch Autoren beträchtlich unterscheiden und in
diachroner Betrachtung den Wandlungen einer im Entstehen begriffenen
philosophischen Disziplin
unterliegen.\footnote{\cite[Vgl.][]{Borsche:ArtikelMetaphysikVI.Neuzeit1980}.}
Man beachte etwa, dass das bekannte Schema einer in Ontologie, rationale
Kosmologie, rationale Psychologie und rationale oder natürliche Theologie
einzuteilenden Metaphysik erst innerhalb der an
\authorcite{Wolff:Psychologiaempirica1968} anschließenden Schulmetaphysik entstand und die (terminologisch
fixierte) Einteilung in eine \textit{metaphysica generalis} und eine
\textit{metaphysica specialis} sogar noch jüngeren Datums
ist.\footnote{\authorfullcite{Vollrath:DieGliederungderMetaphysikineineMetaphysicaGeneralisundeineMetaphysicaSpecialis1962}
bemerkt, dass sich die Einteilung in eine allgemeine und eine spezielle
Metaphysik in der Philosophie vor \name[Immanuel]{Kant} gar nicht explizit,
sondern höchstens implizit finde
\parencite[vgl.][258--263]{Vollrath:DieGliederungderMetaphysikineineMetaphysicaGeneralisundeineMetaphysicaSpecialis1962}.
\authorfullcite{Albrecht:ChristianWolffundderWolffianismus2014} schreibt,
\authorcite{Wolff:Discursuspraeliminarisdephilosophiaingenere1996} habe diese
Unterscheidung von seinen Vorgängern nicht übernommen
\parencite[vgl.][105]{Albrecht:ChristianWolffundderWolffianismus2014}, wobei er
die Antwort auf die Frage schuldig bleibt, bei welchen Vorgängern die
Unterscheidung explizit getroffen wurde. Am nächsten kommt der Bezeichnung
\enquote{\emph{metaphysica generalis}} noch Alexander
\authorcite{Baumgarten:Metaphysica---Metaphysik2011}, der in seiner Metaphysik
-- und dort auch nur beiläufig im Sinne einer auch möglichen Bezeichnungsweise
-- von der Ontologie als einer \enquote{metaphysica universalis} spricht:
\enquote{\ori{Ontologia} (ontosophia, metaphysica, \punkt\ \myemph{metaphysica
universalis}, architectonica, philosophia prima,) est scientia praedicatorum
entis generaliorum}
\mkbibparens{\cite[][\S~4]{Baumgarten:Metaphysica---Metaphysik2011},
\cite[][XVII: 24.5--6]{Kant:GesammelteWerke1900ff.}}. Üblicher war es, die
Ontologie gar nicht als Metaphysik, sondern als \textit{prima philosophia} zu
bezeichnen \parencite[z.\,B.][]{Wolff:Philosophiaprimasiveontologia1977}.}
Der Begriff der Metaphysik schillert nicht minder als der der Aufklärung, weswegen die These, es
stünde gerade die Metaphysik im Zentrum der Aufklärung, erläuterungsbedürftig
ist.

\name[Immanuel]{Kant} gibt in der \titel{Kritik der reinen Vernunft}
verschiedene Bestimmungen des Metaphysikbegriffs an und unterscheidet mehrfach zwischen weiteren und engeren
Wortverwendungen.\footnote{\cite[Vgl.][B
869\,f.,]{Kant:KritikderreinenVernunft2003} \cite[][III:
543.27--544.24]{Kant:GesammelteWerke1900ff.}.} Die folgende Systematisierung,
bei der die Metaphysik bezüglich ihrer \textit{Quellen}
(\ref{MethodikundQuellenderMetaphysik}), ihrer \textit{Teildisziplinen}
(\ref{TeilDisziplinenDerMetaphysik}) und ihres \textit{Interesses}
(\ref{InteresseDerMetaphysik}) betrachtet wird, mag dabei zu einem besseren
Überblick beitragen und die Fragen nach Einheitlichkeit und Wahl der relevanten
Begriffsbestimmung zu beantworten.\footnote{In den \titel{Prolegomena}
betrachtet \name[Immanuel]{Kant} selbst die Metaphysik nach ihrem Objekt, ihren
Quellen und der Erkenntnisart. Die Erkenntnis\emph{art} betrifft in diesem
Zusammenhang die Unterscheidung synthetischer und analytischer Urteile.
\name[Immanuel]{Kant} betont diese Unterscheidung, weil eine Verwechslung zu
einem Irrtum über die eigentlichen \emph{Quellen} der Metaphysik führe. Denn zu
glauben, die Sätze der Metaphysik folgten aus dem Satz vom ausgeschlossenen
Widerspruch (seien also analytisch), heiße fälschlich die Quellen der
Metaphysik in ihr selbst, nicht in der reinen Vernunft suchen
\mkbibparens{\cite[vgl.][\S~3]{Kant:ProlegomenazueinerjedenkuenftigenMetaphysikdiealsWissenschaftwirdauftretenkoennen1977},
\cite[][IV: 270.5--15]{Kant:GesammelteWerke1900ff.}}. Die Betrachtung des
Objekts der Metaphysik entspricht wiederum der Betrachtung der Teildisziplinen.}
\begin{nummerierung}
\item\label{MethodikundQuellenderMetaphysik} In Bezug
auf ihre \emph{Erkenntnisquellen} wird Metaphysik als die Erkenntnis aus reiner
Vernunft
bestimmt.\footnote{\cite[Vgl.][\S~1]{Kant:ProlegomenazueinerjedenkuenftigenMetaphysikdiealsWissenschaftwirdauftretenkoennen1977},
\cite[][IV: 265.5--266.8]{Kant:GesammelteWerke1900ff.};
\cite[][B 869]{Kant:KritikderreinenVernunft2003},
\cite[][III: 543.27--544.2]{Kant:GesammelteWerke1900ff.}.}
\enquote{Metaphysische Erkenntnis muß lauter Urteile a priori enthalten, das erfordert das Eigentümliche ihrer Quellen.}\footnote{\cite[][\S~2]{Kant:ProlegomenazueinerjedenkuenftigenMetaphysikdiealsWissenschaftwirdauftretenkoennen1977}, \cite[][IV: 266.15--16]{Kant:GesammelteWerke1900ff.}.} Das ist die Bestimmung, der zufolge die Metaphysik den Überlegungen im vorangehenden Kapitel \ref{section:MuendigkeitundPhilosophie} zufolge im Fokus der Aufklärung steht.
Allerdings gibt es zwei Disziplinen, die gänzlich a priori erkennen: Philosophie und Mathematik.
Nun bestimmt \name[Immanuel]{Kant} eine Erkenntnis als mathematisch oder als
\emph{intuitive} rationale Erkenntnis, wenn sie aus der \emph{Konstruktion} von
Begriffen in der reinen Anschauung des Raumes (Geometrie) oder der Zeit
(Arithmetik) erfolgt. Diejenige apriorische Erkenntnis hingegen, die nicht durch
die Konstruktion von Begriffen erfolgt, nennt \name[Immanuel]{Kant}
philosophisch oder auch \emph{diskursive} rationale Erkenntnis. Und das System
der philosophischen Erkenntnisse oder der diskursiven Vernunfterkenntnisse heißt
Metaphysik.\footnote{\cite[Vgl.][B 869]{Kant:KritikderreinenVernunft2003}, \cite[][III: 543.27--544.2]{Kant:GesammelteWerke1900ff.}.} Dies ist die weiteste
und systematisch ursprüngliche unter den Bestimmungen, die \name[Immanuel]{Kant}
angibt. Daraus ergibt sich weiter, dass Metaphysik
schon begrifflich mit der (reinen) Philosophie
zusammenfällt,\footnote{\cite[Vgl.][B~869,
878]{Kant:KritikderreinenVernunft2003}, \cite[][III: 544.2--8,
549.13--16]{Kant:GesammelteWerke1900ff.}.} insofern sie die
Ver\-nunft\-er\-kennt\-nis aus Begriffen (nicht: aus der Konstruktion der
Begriffe) ist\footnote{\cite[Vgl.][B 865]{Kant:KritikderreinenVernunft2003}, \cite[][III:
541.18--20, 542.3]{Kant:GesammelteWerke1900ff.}.}.

Dabei ist sich \name[Immanuel]{Kant} der Tatsache bewusst, einen gegenüber den ihm stets präsenten
Vorgängern \authorcite{Wolff:Psychologiaempirica1968} und
\authorcite{Baumgarten:Metaphysica---Metaphysik2011} \textit{neuen}
Ausgangspunkt der Bestimmung des Metaphysikbegriffs zu wählen, indem er
festlegt, Metaphysik sei das, was wir aus reiner Vernunft
wissen. \authorcite{Wolff:Discursuspraeliminarisdephilosophiaingenere1996}
bestimmt die Metaphysik über Ihre Teilbereiche Ontologie, Kosmologie und
Pneumatik (Psychologie und natürliche
Theologie).\footcite[Vgl.][\S~79]{Wolff:Discursuspraeliminarisdephilosophiaingenere1996}
\authorcite{Baumgarten:Metaphysica---Metaphysik2011} definiert die Metaphysik
über ihre Stellung im System der Wissenschaften, ohne auf den Ursprung
metaphysischer Erkenntnis in der Vernunft oder der Erfahrung zu rekurrieren.
Eine Disziplin zählt nach ihm zur Metaphysik, wenn sie die Grundlage einer
Reihe anderer Disziplinen
bildet.\footnote{\label{Anmerkung:SchulmetaphysikMitEmpirischerPsychologie}\enquote{\ori{Metaphysica}
est scientia primorum in humana cognitione principiorum}
\mkbibparens{\cite[][\S~1]{Baumgarten:Metaphysica---Metaphysik2011},
\cite[][XVII: 23.16]{Kant:GesammelteWerke1900ff.}}. Dass dies nicht im Sinne
der obersten (apriorischen) Grundsätze \name[Immanuel]{Kant}s zu verstehen ist,
erhellt aus der Integration der empirischen Psychologie in die Metaphysik
\mkbibparens{\cite[vgl.][\S\S~502\,f.,]{Baumgarten:Metaphysica---Metaphysik2011}
\cite[][XVII: 130.17--22]{Kant:GesammelteWerke1900ff.}}.} Beide integrieren
daher auch bedenkenlos empirische Wissenschaften in die
Metaphysik; \authorcite{Wolff:Discursuspraeliminarisdephilosophiaingenere1996}
sieht sogar die empirische Psychologie gerade wegen ihres empirischen Charakters als
besonders geeignet an, eine tragende Rolle für andere Disziplinen zu
spielen.\footnote{\cite[Vgl.][\S~112]{Wolff:Discursuspraeliminarisdephilosophiaingenere1996}.}
Und auch \authorcite{Baumgarten:Metaphysica---Metaphysik2011} folgt
\authorcite{Wolff:Discursuspraeliminarisdephilosophiaingenere1996} in der
Integration der empirischen Psychologie in die Metaphysik,\footnote{Die
Abschnitte der \titel{Metaphysica}, die die empirische Psychologie behandeln,
wurden von \name[Immanuel]{Kant} der pragmatischen Anthropologie, wie er sie in
seinen Vorlesungen behandelte, zugrunde gelegt
\parencite[vgl.][\pno~54\,f.]{Falduto:TheFacultiesoftheHumanMindandtheCaseofMoralFeelinginKantsPhilosophy2014}.
Sie sind abgedruckt in Band XV der Akademieausgabe der Werke
\name[Immanuel]{Kant}s \parencite[siehe][XV:
5--54]{Kant:GesammelteWerke1900ff.}.} betrachtet Metaphysik also nicht als reine
Vernunftwissenschaft in einem Sinn, der \name[Immanuel]{Kant}s Metaphysikbegriff
bestimmt.

Hatte sich \name[Immanuel]{Kant} in früheren Schriften noch
\authorcite{Baumgarten:Metaphysica---Metaphysik2011}s Bestimmung des
Metaphysikbegriffs angeschlossen\footnote{\enquote{Die Metaphysik ist nichts
anders als eine Philosophie über die ersten Gründe unseres Erkenntnisses}
\mkbibparens{\cite[][A
79]{Kant:UntersuchungueberdieDeutlichkeitderGrundsaetzedernatuerlichenTheologieundderMoral1977},
\cite[][II: 283.13--14]{Kant:GesammelteWerke1900ff.}}. Offensichtlich handelt es
sich um eine Übersetzung der Definition
\authorcite{Baumgarten:Metaphysica---Metaphysik2011}s (siehe Anm. \ref{Anmerkung:SchulmetaphysikMitEmpirischerPsychologie}).}
und ebenfalls die empirische Psychologie zur Metaphysik gezählt\footnote{In den
1760er Jahren nennt er sie \enquote{die metaphysische Erfahrungswissenschaft vom
\ori{Menschen}} \mkbibparens{\cite[][A
9]{Kant:NachrichtvonderEinrichtungseinerVorlesungenindemWinterhalbenjahrevon1765-17661900ff.},
\cite[][II: 309.2--3]{Kant:GesammelteWerke1900ff.}}.}, so behauptet er in der
\titel{Kritik der reinen Vernunft}, nur durch die Neubestimmung die Metaphysik
als \emph{einheitliche} philosophische Disziplin begründen zu können:
\enquote{Alle reine Erkenntnis a priori macht also, vermöge des besonderen
Erkenntnisvermögens, darin es allein seinen Sitz haben kann, eine besondere
Einheit aus, und Metaphysik ist diejenige Philosophie, welche jene Erkenntnis in
dieser systematischen Einheit darstellen
soll}\footnote{\cite[][B~873]{Kant:KritikderreinenVernunft2003}, \cite[][III:
546.8--11]{Kant:GesammelteWerke1900ff.}.
\authorfullcite{Heidegger:KantunddasProblemderMetaphysik1965} behauptet, damit
entferne sich \name[Immanuel]{Kant} gar nicht von
\authorcite{Baumgarten:Metaphysica---Metaphysik2011}s Begriff der Metaphysik
\parencite[vgl.][\S~1]{Heidegger:KantunddasProblemderMetaphysik1965}.}.
\name[Immanuel]{Kant} behauptet, als erster eine klare Bestimmung der alten Idee einer Wissenschaft unter dem Namen
\enquote{Metaphysik} angegeben zu haben, indem er die Wissenschaften \emph{ihren
Quellen} und nicht ihren Objekten nach einteilt.\footnote{\cite[Vgl.][B
870--873]{Kant:KritikderreinenVernunft2003}, \cite[][III:
544.25--546.15]{Kant:GesammelteWerke1900ff.}.}
%
%
\item\label{TeilDisziplinenDerMetaphysik} \textit{Inhaltlich} oder in Bezug auf
ihre \emph{Teildisziplinen} weist \name[Immanuel]{Kant} der Metaphysik im
engeren Sinne folgende Themenfelder zu:
Ontologie, rationale Physiologie (unterteilt in \emph{physica rationalis} und
\emph{psychologia rationalis}), rationale Kosmologie und rationale Theologie. In einem
weiteren Sinne enthält sie außerdem die Kritik der reinen Vernunft als
Propädeutik und die Metaphysik der Sitten. Diese Inhalte ergeben sich aber nicht
rhapsodisch, sondern werden aus der methodischen Bestimmung als
Vernunfterkenntnis aus Begriffen entwickelt, wie Abbildung
\ref{abbildung:metaphysikeinteilung}
(auf Seite \pageref{abbildung:metaphysikeinteilung}) illustriert.
\begin{figure}[htb]
\begin{minipage}[t]{\textwidth}
\centering
% \includegraphics{Metaphysikeinteilung.pdf}
\input{SystematikdesMetaphysikBegriffsnachKant}
  \caption{Systematik des Metaphysikbegriffs nach
  \cite[][B~869--875]{Kant:KritikderreinenVernunft2003}, \cite[][III:
  543.27--547.15]{Kant:GesammelteWerke1900ff.}.}\label{abbildung:metaphysikeinteilung}
\end{minipage}
\end{figure}
Dabei unterscheidet er sogleich die \enquote{eigentliche Metaphysik} von der Vernunftkritik als deren
Propädeutik, wenngleich man letztere auch ohne weiteres zur Metaphysik
hinzurechnen könne. In einem engeren Sinn hingegen verstehe man unter Metaphysik
das System der reinen Erkenntnisse der Vernunft im \distanz{spekulativen}
Gebrauch, also unter Ausschluss der Metaphysik der Sitten (Tugendlehre und
Rechtsphilosophie). Die Metaphysik als Vernunfterkenntnis aus Begriffen -- wie
sie \name[Immanuel]{Kant} unter Bezug auf ihre Quellen bestimmt -- ist ohne
solche Einschränkungen offenkundig nicht deckungsgleich mit dem, was man für
gewöhnlich (und auch zu \name[Immanuel]{Kant}s Zeit) unter Metaphysik versteht.
Schließlich enthält sie insbesondere die gesamte Moral- und Rechtsphilosophie,
die \name[Immanuel]{Kant} unter dem Namen einer Metaphysik der Sitten
publiziert.
Diese wiederum ergibt damit einen Metaphysikbegriff, der nach den genannten
Einschränkungen, also der Eliminierung von Propädeutik und praktischer
Philosophie, koextensiv zu sein
scheint mit damals geläufigen Bestimmungen, wenn man ihn etwa mit der Einteilung
bei Christian \authorcite{Wolff:Psychologiaempirica1968}
vergleicht.\footnote{\label{anmerkung:wolffseinteilungdermetaphysik} Wobei
\authorcite{Wolff:Psychologiaempirica1968} die Psychologie und Theologie zunächst zu einer Pneumatologie
zusammenfasst.
\cite[Vgl.][\S~79]{Wolff:Discursuspraeliminarisdephilosophiaingenere1996}:
\enquote{Psychologia \&\ Theologia naturalis nonnunquam Pneumatic\ae\ nomine communi insigniuntur, \&\ Pneumatica
per spirituum scientiam definiri solet. Ontologia vero, Cosmolgia generalis \&\
Pneumatica communi Metaphysic\ae\ nomine compellantur. Est igitur Metaphysica
scientia entis, mundi in genere atque spirituum.} Dies entspricht der heute
geläufigen Aufteilung in eine \emph{metaphysica generalis} (Ontologie) und die drei
Teildisziplinen der \emph{metaphysica specialis} (Psychologie, Kosmologie, Theologie),
wobei sich eine solche Einteilung in allgemeine und spezielle Metaphysik weder
bei \authorcite{Wolff:Psychologiaempirica1968}, noch bei \name[Immanuel]{Kant} findet. Im Vergleich zu
\name[Immanuel]{Kant} fehlt bei \authorcite{Wolff:Psychologiaempirica1968} die rationale Physik, die
\name[Immanuel]{Kant} als Gegenstück zur rationalen Psychologie unter der Rubrik
\enquote{rationale Physiologie} neu einführt und dann in den
\titel{Metaphysischen Anfangsgründe der Naturwissenschaften} ausführlicher
behandelt. \mkbibparens{Dass die rationale Physiologie das Thema der
\titel{Metaphysischen Anfangsgründe} beschreibt, ergibt sich aus der Rolle, die
dem Begriff der Materie in der \titel{Kritik der reinen Vernunft} zugewiesen wird;
\cite[vgl.][B 875\,f.,]{Kant:KritikderreinenVernunft2003}
\cite[][III: 547.23--548.4]{Kant:GesammelteWerke1900ff.}}. Dafür schließt er
die empirische Psychologie aus der Metaphysik aus und räumt ihr nur noch unter
pragmatischen Gesichtspunkten einen Platz in der Philosophie ein
\mkbibparens{\cite[vgl.][B~876\,f.,]{Kant:KritikderreinenVernunft2003}
\cite[][III: 548.9--28]{Kant:GesammelteWerke1900ff.}}. Gerade in dem letzten
Punkt kommt zum Ausdruck, dass \name[Immanuel]{Kant} die Systematik nicht mehr wie
\authorcite{Wolff:Psychologiaempirica1968} inhaltlich -- anhand des Objekts --,
sondern methodisch -- nach Maßgabe der Erkenntnisquellen -- fundiert.}

Die Einteilung folgt dabei Überlegungen zur Art des Erwerbs einer Erkenntnis,
die -- wie in Kapitel \ref{subsection:Vernunfterkenntnis:MathematikPhilosophie}
gesehen -- deren Form ausmacht. Beispielsweise unterteilt er die immanente
Physiologie in Physik und Psychologie in Abhängigkeit davon, ob uns die
Gegenstände durch den äußeren oder den inneren Sinn gegeben
werden.\footnote{\cite[Vgl.][B 874]{Kant:KritikderreinenVernunft2003},
\cite[][III: 546.36--547.10]{Kant:GesammelteWerke1900ff.}.} Insofern ist
die inhaltliche Bestimmung des Metaphysikbegriffs sekundär und von der
methodischen Bestimmung, die sich an den Erkenntnisquellen orientiert, abhängig.
In der \titel{Kritik der Urteilskraft} heißt es entsprechend, dass \enquote{es
in der Einteilung einer Vernunftwissenschaft gänzlich auf diejenige
Verschiedenheit der Gegenstände ankommt, deren Erkenntnis verschiedener
Prinzipien bedarf}\footnote{\cite[][B xiii]{Kant:KritikderUrteilskraft2009},
\cite[][V: 172.17--19]{Kant:GesammelteWerke1900ff.}.}.
%
%
\item\label{InteresseDerMetaphysik} Bezüglich ihres \emph{Interesses} ist die
Metaphysik auf die Ideen Gott, Freiheit und Unsterblichkeit ausgerichtet, welche ihren eigentlichen und
einzigen Zweck ausmachen.\footnote{\enquote{Diese unvermeidlichen Aufgaben der
reinen Vernunft selbst, sind \ori{Gott, Freiheit} und \ori{Unsterblichkeit}. Die
Wissenschaft aber, deren Endabsicht mit allen ihren Zurüstungen eigentlich nur
auf die Auflösung derselben gerichtet ist, heißt \ori{Metaphysik}}
\mkbibparens{\cite[][B~7]{Kant:KritikderreinenVernunft2003}, \cite[][III:
31.6--9]{Kant:GesammelteWerke1900ff.}}. \enquote{Die Metaphysik hat zum
eigentlichen Zwecke ihrer Nachforschung nur drei Ideen: \ori{Gott, Freiheit} und
\ori{Unsterblichkeit} \punkt. Alles, womit sich diese Wissenschaft sonst
beschäftigt, dient ihr bloß zum Mittel, um zu diesen Ideen und ihrer Realität zu
gelangen} \mkbibparens{\cite[][B~395]{Kant:KritikderreinenVernunft2003},
\cite[][III: 260.20--24]{Kant:GesammelteWerke1900ff.}}.} Dabei nutzt
\name[Immanuel]{Kant} aus, dass die wichtigsten Fragen unserer Vernunft, welche auch wegen ihrer
Relevanz im Fokus der Aufklärung stehen (vornehmlich also die Religion), gerade
diejenigen Fragen sind, die (ausschließlich) aus reiner Vernunft heraus zu
behandeln sind.\footnote{\enquote{Und gerade in diesen letzteren Erkenntnissen,
welche über die Sinnenwelt hinausgehen, wo Erfahrung gar keinen Leitfaden, noch
Berichtigung geben kann, liegen die Nachforschungen unserer Vernunft, die wir,
der Wichtigkeit nach, für weit vorzüglicher, und ihre Endabsicht für viel
erhabener halten, als alles, was der Verstand im Felde der Erscheinungen lernen
kann, wobei wir, sogar auf die Gefahr zu irren, eher alles wagen, als daß wir
so angelegene Untersuchungen aus irgend einem Grunde der Bedenklichkeit, oder
aus Geringschätzung und Gleichgültigkeit aufgeben sollten}
\mkbibparens{\cite[][B~6\,f.,]{Kant:KritikderreinenVernunft2003} \cite[][III:
30.24--31.12]{Kant:GesammelteWerke1900ff.}}. Dies spricht auch gegen eine immer
wieder zu vernehmende Unterstellung, das traditionelle Interesse an
Erkenntnissen \emph{a priori} gründe in deren Immunität gegenüber Irrtümern
\parencite[siehe
z.\,B.][\pno~2\,f.]{Christensen:TestimonyMemoryandtheLimitsoftheaPriori1997}.}
Sie betreffen das \singlequote{Übersinnliche}, also diejenigen Gegenstände einer
Erkenntnis, die nicht auf der Grundlage von Erfahrung zu beantworten sind,
sondern verlangen, über den Bereich der Erfahrung hinaus zu gehen.
In den \titel{Fortschritten der Metaphysik} nennt \name[Immanuel]{Kant} die
Themenfelder, die das Interesse an der Metaphysik begründet, sogar als mögliches
\emph{Definiens} des Metaphysikbegriffs.\footnote{Siehe die retrospektive
Darstellung in \cite[][A
9\,f.,]{Kant:WelchessinddiewirklichenFortschrittediedieMetaphysikseitLeibnitzensundWolfsZeiteninDeutschlandgemachthat?1900ff.} \cite[][XX: 260.3--6]{Kant:GesammelteWerke1900ff.}: \enquote{Dieser Endzweck,
auf den die ganze Metaphysik angelegt ist, ist leicht zu entdecken, und kann in
dieser Rücksicht eine Definition derselben begründen: \enquote{sie ist die
Wissenschaft, von der Erkenntnis des Sinnlichen zu der des Übersinnlichen durch
die Vernunft fortzuschreiten.}}} Es sind dies die Themen, die -- so glaubt
\name[Immanuel]{Kant} -- in der Entwicklung des Metaphysikbegriffs stets prägend
blieben und welche die eigentliche Bestimmung der Philosophie -- ihren
Weltbegriff -- ausmachen.
\end{nummerierung}



Die quellenorientierte Definition dient dem Ziel, dem \emph{Schulbegriff} der
Philosophie gerecht zu werden, indem sie eine einheitliche und trennscharfe,
also den Richtlinien der Definitionslehre gemäße Bestimmung des Begriffs der
Metaphysik festlegt. Da Metaphysik den Kern der Philosophie, nämlich die
eigentliche Philosophie bildet, ist damit der Begriff der Philosophie ebenso
festgelegt. Die Berechtigung dieser Festlegung ergibt sich nun aus der Kongruenz
mit dem Weltbegriff der Philosophie, der sich an \enquote{der Beziehung aller
Erkenntnis auf die wesentlichen Zwecke der menschlichen
Vernunft ({teleologia rationis humanae})}\footnote{\cite[][B
867]{Kant:KritikderreinenVernunft2003}, \cite[][III: 542.27--28]{Kant:GesammelteWerke1900ff.}.} orientiert.
Wir könnten sagen, dass die grundlegende Definition derselben nach
ihren Quellen den \emph{Schulbegriff} der Metaphysik angibt, ihre Beziehung zu den
Themen Gott, Freiheit und Unsterblichkeit aber auf den \emph{Weltbegriff} derselben
verweist, \enquote{der dasjenige [enthält], was jedermann notwendig
interessiert}\footnote{\cite[][B~867]{Kant:KritikderreinenVernunft2003}, \cite[][III:
543.31--32]{Kant:GesammelteWerke1900ff.}.}, und was daher \enquote{doch als
Naturanlage (metaphysica naturalis)
wirklich}\footnote{\cite[][B~21]{Kant:KritikderreinenVernunft2003}, \cite[][III:
41.3]{Kant:GesammelteWerke1900ff.}.} ist.

Inhaltlich versucht \name[Immanuel]{Kant} Kontinuität zu wahren. Insofern die
Metaphysik im engere Sinne nur die Metaphysik der Natur umfasse, ergibt sich am
Ende der Einteilung die \singlequote{klassische} Aufteilung in Ontologie,
Physiologie (unter Einschluss der Psychologie), Kosmologie und Theologie.
Differenzen sind vor allem im Ausschluss aller empirischen Erkenntnisse
auszumachen und in den übergeordneten Einteilungen.
\authorcite{Wolff:Discursuspraeliminarisdephilosophiaingenere1996} fasst
beispielsweise Theologie und Psychologie zur Pneumatologie zusammen, weil sie
beide von \singlequote{Geistern} (\emph{spiritus})
handeln.\footcite[Vgl.][\S~79]{Wolff:Discursuspraeliminarisdephilosophiaingenere1996}
Da \name[Immanuel]{Kant} auf Einteilungen nach solchen rein inhaltlichen
Gesichtspunkten verzichtet, ergeben sich die Ähnlichkeiten erst auf der
untersten Ebene der Systematisierung.


Es ist von Bedeutung festzuhalten, dass der zentrale
Ausgangspunkt der Bestimmung des Metaphysikbegriffs der methodische ist, also
der Verweis auf die Quellen der Metaphysik aus reiner Vernunft. Damit ist
grundlegend, was vorhin im Zentrum der Überlegungen zu \name[Immanuel]{Kant}s
sozialer Erkenntnistheorie stand: Was wir aus reiner Vernunft erkennen können
-- also die Metaphysik --, das darf nicht im Modus bloß historischer Kenntnis
für wahr gehalten werden. Bis hierher könnte man versucht sein zu vermuten, wir
könnten nun einfach auf Metaphysik verzichten und damit die Forderung des
Selbstdenkens umgehen. Dass dieser Weg aber nicht gangbar ist, ergibt sich -- so
werden wir gleich sehen -- daraus, dass Metaphysik den Kern der Philosophie
ausmacht: \enquote{Und doch ist Metaphysik die eigentliche, wahre
Philosophie!}\footnote{\cite[][A 39]{Kant:ImmanuelKantsLogik1977}, \cite[][IX:
32.36--37]{Kant:GesammelteWerke1900ff.}.}  \name[Immanuel]{Kant} verwendet die
Termini Philosophie und Metaphysik (fast) gleichbedeutend -- genau genommen
identifiziert er die Metaphysik mit der \emph{reinen} Philosophie, der
Erkenntnis aus reiner Vernunft, die sich nicht auf empirische Prinzipien
stützt.\footnote{\cite[Vgl.][B~869]{Kant:KritikderreinenVernunft2003},
\cite[][III: 543.27--544.8]{Kant:GesammelteWerke1900ff.}. Siehe auch
\cite[][B~878]{Kant:KritikderreinenVernunft2003}, \cite[][III:
549.13--16]{Kant:GesammelteWerke1900ff.}: \enquote{Metaphysik also, sowohl der
Natur, als der Sitten, vornehmlich die Kritik der sich auf eigenen Flügeln
wagenden Vernunft, welche \ori{vorübend} (propädeutisch) vorhergeht, machen
eigentlich allein dasjenige aus, was wir im echten Verstande Philosophie nennen
können.}} Es ist diese reine Philosophie, deren Inhalte wir niemals auf die
Autorität anderer hin für wahr halten dürfen, wenn wir mündig und selbstbestimmt
denken wollen. Und dies erläutert dann auch, warum wir nach
\name[Immanuel]{Kant} eigentlich nicht Philosophie als System wahrer
Erkenntnisse, sondern ausschließlich die Tätigkeit des Philosophierens zu lernen
haben.

Ich rekurrierte oben auf den \emph{Weltbegriff der Philosophie}, um die
Akzentuierung bestimmter Erkenntnisbereiche innerhalb der
Aufklärungsprogrammatik zu beschreiben. Danach war sie die Wissenschaft von den
letzten Zwecken der Vernunft und als solche mit der Bestimmung des Menschen
befasst. Im Kapitel \ref{subsection:DieBestimmungdesMenschen} über die
Bestimmung des Menschen sahen wir, dass die Philosophie ihrem Weltbegriffe nach
im Fokus der Aufklärung steht, weil sie Inhalte thematisiert, die jeden Menschen
notwendig angehen. Auf der Grundlage der Kapitel
\ref{section:MuendigkeitundPhilosophie} lässt sich nun zeigen, dass
Philosophie ebenso ihrem Schulbegriff nach im Zentrum des \enquote{\emph{sapere
aude!}} steht, weil sie diesem Schulbegriffe nach der Inbegriff rationaler
Erkenntnisse ist. \enquote{Philosophie ist {\punkt} das System der
philosophischen Erkenntnisse oder der Vernunfterkenntnisse aus Begriffen. Das
ist der Schulbegriff von dieser Wissenschaft.}\footnote{\cite[][A
23]{Kant:ImmanuelKantsLogik1977}; \cite[][IX:
23.30--32]{Kant:GesammelteWerke1900ff.}.} Philosophische Erkenntnis ist nun aber
eine rationale Erkenntnis, die im Unterschied zu mathematischer Erkenntnis
diskursiv und nicht intuitiv ist; der Gesamtbereich rationaler Erkenntnis
zerfällt also in Mathematik und (\singlequote{reine}) Philosophie. Rationale
Erkenntnis ist diejenige Erkenntnis, der eine bloß historische Kenntnis nicht
angemessen ist (wobei Einschränkungen für die Mathematik
gelten\footnote{Siehe dazu Kapitel
\ref{subsubsection:EndlichesundUnendlichesErkennen}.}).
Also ist es gerade die Metaphysik und mit ihr die (\singlequote{reine})
Philosophie, bei der wir aufgefordert sind, uns in unseren Überzeugungen nicht
nach der Auskunft anderer zu richten, sondern unsere \emph{eigene} vernünftige
Einsicht zu suchen. Wie der Weltbegriff die
Philosophie aus inhaltlichen Gesichtspunkten heraus in das Zentrum der
Aufklärung stellte, so ist sie ihrem Schulbegriff nach formal im Fokus des \enquote{sapere aude}.

Zwischen Schul- und Weltbegriff der Philosophie besteht kein Widerstreit, so als
beschrieben sie verschiedene und miteinander unvereinbare, in Konkurrenz
zueinander oder auch nur unverbunden nebeneinander stehende Projekte.
Es ließe sich zwar \emph{prima facie} vermuten, dass der Weltbegriff beschreibt,
wie eine aufgeklärte Philosophie sein \emph{soll}, während der Schulbegriff den \emph{Status quo}
der damaligen akademisch betriebenen Philosophie beschreibt, die in ihrem Tun
die Absichten der Aufklärung verkennt, weil \enquote{sie nur als eine von den
Geschicklichkeiten zu gewissen beliebigen Zwecken angesehen
wird.}\footnote{\cite[][B 867]{Kant:KritikderreinenVernunft2003}; \cite[][III:
543.33--34]{Kant:GesammelteWerke1900ff.}. Siehe zu dieser Deutung z.\,B.
\cite[][481]{Doering:UeberKantsLehrevonBegriffundAufgabederPhilosophie1885}:
\enquote{Das ist also das Gesammtresultat [sic] dieser Untersuchung, daß uns
Kant vor die Alternative stellt, die Philosophie als Vernunftkünstler nach dem
Schulbegriffe als universelles System des Wissens, oder als weisheitsstrebende
nach dem Weltbegriffe als Inbegriff der allen Menschen angelegenen Erkenntnisse
zu betreiben.}} In der Tat aber gehen beide Hand in Hand, insofern der
Schulbegriff das zur Praxis notwendige Modell philosophischer Theorie beschreibt
(auf die Anbindung an die Praxis verweist der Begriff der
\singlequote{Welt}).\footnote{\cite[Vgl.][89]{Kleinhans:DerenquotePhilosophinderneuerenGeschichtederPhilosophie1999}.
\authorfullcite{Trawny:DasIdealdesWeisen2008} sieht im
Schulbegriff der Philosophie die (notwendigen, aber nicht hinreichenden) formalen Voraussetzungen
artikuliert, die eine Erkenntnis erfüllen muss, um als philosophisch zu
gelten \parencite[vgl.][467]{Trawny:DasIdealdesWeisen2008}. Siehe zum
Zusammenhang beider Begriffe auch die Arbeit von
\authorfullcite{Boehr:PhilosophiefuerdieWelt2003}, der den Weltbegriff mit der
Forderung nach Popularität, den Schulbegriff hingegen mit der Forderung nach
Gründlichkeit verbunden sieht, dabei aber einen engen Zusammenhang beider
Begriffe hervorhebt \parencite[vgl.][183--190]{Boehr:PhilosophiefuerdieWelt2003}.}
Wie in Kapitel \ref{chapter:AufklaerungundWissenschaft} dargestellt, verlangt
der Weltbegriff der Philosophie, demgemäß sie dasjenige Wissen vereint, dass wir
benötigen, um unserer Bestimmung als Menschen gerecht zu werden, in geringem
Umfang auch empirisches Wissen. Wir benötigen die \singlequote{Weltkenntnis} der
physischen Geographie und vor allem der pragmatischen Anthropologie. Aber dieses
Wissen ist doch ausgerichtet auf einen Kern, den uns nur die Metaphysik, und innerhalb der
Metaphysik speziell die Metaphysik der Sitten gewährt; schließlich ist die Moral
die Wissenschaft von der ganzen Bestimmung des Menschen.\footnote{Siehe Seite
\pageref{Anmerkung:GanzeBestimmung}.}


Der Schulbegriff bestimmt die Philosophie als System aller philosophischen
Erkenntnisse, also aller diskursiven\footnote{Zu dem Begriff der Diskursivität
siehe Kapitel \ref{subsubsection:BegriffderDiskursivitaet}.} Vernunfterkenntnisse, d.\,i.
der rationalen, aber nicht mathematischen, oder der \emph{metaphysischen} Erkenntnisse. Ihrem
Schulbegriffe nach ist die Philosophie daher im Kern Metaphysik: \enquote{das
System der reinen Vernunft (Wissenschaft), die ganze (wahre sowohl als scheinbare)
philosophische Erkenntnis aus reiner Vernunft im systematischen Zusammenhange
{\punkt} heißt
\ori{Metaphysik}}\footnote{\cite[][B 869]{Kant:KritikderreinenVernunft2003},
\cite[][III: 543.29--544.2]{Kant:GesammelteWerke1900ff.}. Siehe auch
\cite[][B 873]{Kant:KritikderreinenVernunft2003},
\cite[][III: 546.8--11]{Kant:GesammelteWerke1900ff.}:
\enquote{Alle reine Erkenntnis a priori macht also, vermöge des besondern
Erkenntnisvermögens, darin es allein seinen Sitz haben kann, eine besondere
Einheit aus, und Metaphysik ist diejenige Philosophie, welche jene Erkenntnis in
dieser systematischen Einheit darstellen soll.}}. \enquote{Metaphysik also
{\punkt} mach[t] eigentlich allein dasjenige aus, was wir im echten Verstande
Philosophie nennen können.}\footnote{\cite[][B
878]{Kant:KritikderreinenVernunft2003}, \cite[][III:
549.13--16]{Kant:GesammelteWerke1900ff.}.} Und es sind  die Themen und Urteile
der Metaphysik, die \name[Immanuel]{Kant} als philosophische Erkenntnisse
bezeichnet und von der die Aufklärungsforderung sagt, wir sollen sie nicht auf
die Autorität anderer hin übernehmen.

\subsection{Die Unverzichtbarkeit der
Metaphysik}\label{paragraph:DieUnverzichtbarkeitderMetaphysik}
Die Aufforderung, metaphysische Urteile nicht auf die Autorität anderer hin zu
übernehmen, impliziert für sich genommen noch nicht die Aufforderung, selbst
Metaphysik zu betreiben. Könnten wir annehmen, dass \name[Immanuel]{Kant} die
Ansicht vertritt, dass sich metaphysische Fragen gar nicht objektiv beantworten
lassen -- wie er es bezüglich der \emph{Noumena} zu behaupten scheint --, dann bliebe uns
auch die gänzliche Urteilsenthaltung. Dass wir anderen nicht glauben sollen, was
sie uns über Metaphysik sagen, heißt nicht, dass wir dies selbst erkennen
müssten. Wir glauben niemandem, der behauptet, uns die Lottozahlen vom nächsten
Samstag sagen zu können; aber deswegen denken wir nicht, wir müssten diese
selbst vorhersagen.

Die Frage, wie \name[Immanuel]{Kant} zur Möglichkeit von Metaphysik steht, ist
nicht leicht zu beantworten, wie aus den gegensätzlichen Angaben
\name[Immanuel]{Kant}s und insbesondere auch den unterschiedlichen
Interpretationen hervorgeht.\footnote{So spricht \authorcite{Walsh:KantandMetaphysics1976} recht
passend von einem \enquote{scandal to philosophy generally and to Kantian scholarship in
particular that commentators are unable to agree about \name[Immanuel]{Kant}'s
attitude to metaphysics} (\cite[][372]{Walsh:KantandMetaphysics1976}). Zur
Auseinandersetzung um die Frage, ob \name[Immanuel]{Kant} selbst Metaphysiker
war, siehe die Übersicht in
\cite{Funke:DieDiskussionumdiemetaphysischeKantinterpretation1976}.} So sei es
einerseits \enquote{demütigend für die menschliche Vernunft, daß sie in ihrem
reinen Gebrauche nichts ausrichtet, und sogar noch einer Disziplin bedarf, um
ihre Ausschweifungen zu bändigen, und die Blendwerke, die ihr daher kommen, zu
verhüten.}\footnote{\cite[][B 823]{Kant:KritikderreinenVernunft2003},
\cite[][III: 517.4--7]{Kant:GesammelteWerke1900ff.}.} Solche Äußerungen nähren
die Ansicht, \name[Immanuel]{Kant} sei der \distanz{Alleszermalmer der
Metaphysik}, wie \authorfullcite{Mendelssohn:MorgenstundenoderVorlesungenueberdasDaseynGottes1785} ihn in den
\titel{Morgenstunden}
nannte.\footnote{\cite[Vgl.][Vorbericht]{Mendelssohn:MorgenstundenoderVorlesungenueberdasDaseynGottes1785}.} Auf der anderen Seite beteuert \name[Immanuel]{Kant}, das \distanz{kritische
Geschäft} sei nur der Auftakt zum
doktrinalen\footnote{\cite[Vgl.][B x]{Kant:KritikderUrteilskraft2009},
\cite[][V: 170.20--22]{Kant:GesammelteWerke1900ff.}.}, und stellt eine
Metaphysik der Natur in Aussicht, die zumindest teilweise 1786 in Form der \titel{Metaphysischen
Anfangsgründen der Naturwissenschaft} das Licht der literarischen Welt erblickt.

Während \name[Immanuel]{Kant}s
Standpunkt zur Frage nach der \emph{Möglichkeit} eigentümlich dunkel bleibt --
man bedenke, dass dies gerade die Hauptfrage seines Hauptwerkes ist --, äußert
er sich sehr klar und unmissverständlich zur \emph{Notwendigkeit} von
Metaphysik. Das Pendant zum Weltbegriff und seiner Beziehung auf den
Endzweck der Vernunft -- die ganze Bestimmung des Menschen -- im Falle der
Philosophie ist dann im Falle der Metaphysik das \singlequote{Bedürfnis} der
Vernunft. Fraglich bleibt nur, ob Metaphysik damit eine Aufgabe unseres
Nachdenkens bleibt, die nicht endgültig aufzulösen ist oder von deren Auflösung
wir noch weit entfernt sind -- wie dies der Verweis auf das Ideal des
Philosophen und den Lehrer im Ideal der \titel{Kritik der reinen Vernunft}
suggeriert\footnote{\cite[Vgl.][B 866\,f.,]{Kant:KritikderreinenVernunft2003}
\cite[][III: 542.19-542.37]{Kant:GesammelteWerke1900ff.}.} --, oder ob
wenigstens ein Grundstein bereits gelegt ist. Auf diesen Punkt werde ich gleich
eingehen; doch bevor die Frage nach der Möglichkeit von Metaphysik diskutiert
wird, möchte ich zunächst fragen, inwiefern wir sie überhaupt \emph{benötigen}.


Bekanntlich bezeichnet \name[Immanuel]{Kant} die Metaphysik als
unvermeidlich, weil der menschlichen Vernunft ein \singlequote{Bedürfnis} nach
Metaphysik innewohne. Deswegen könne auch niemand auf die Beschäftigung mit
Metaphysik und das Fällen metaphysischer Urteile verzichten:
\begin{quote}
Es ist nämlich umsonst, \ori{Gleichgültigkeit} in Ansehung solcher
Nachforschungen erkünsteln zu wollen, deren Gegenstand der menschlichen Natur
\ori{nicht gleichgültig} sein kann. Auch fallen jene vorgeblichen
\ori{Indifferentisten},
so sehr sie sich auch durch die Veränderung der Schulsprache in einem populären Tone unkenntlich zu machen gedenken, wofern sie
nur überall etwas denken, in metaphysische Behauptungen unvermeidlich zurück,
gegen die sie doch so viel Verachtung
vorgaben.\footnote{\cite[][A~x]{Kant:KritikderreinenVernunft2003};
\cite[][IV: 8.28--34]{Kant:GesammelteWerke1900ff.}. Eine Parallelstelle findet
sich in der \titel{Logik}: \enquote{Was aber Metaphysik betrifft: so scheint
es, als wären wir bei Untersuchung metaphysischer Wahrheiten stutzig geworden.
Es zeigt sich jetzt eine Art von \ori{Indifferentism} gegen diese
Wissenschaft, da man es sich zur Ehre zu machen scheint, von metaphysischen
Nachforschungen, als von bloßen \ori{Grübeleien}, verächtlich zu reden.}
\mkbibparens{\cite[][A 39]{Kant:ImmanuelKantsLogik1977},
\cite[][IX: 32.31--36]{Kant:GesammelteWerke1900ff.}}}
\end{quote}
Der Indifferentist glaubt, es sei gar nicht nötig, in metaphysischen Fragen ein
objektiv gültiges Urteil abzugeben.\footnote{Hier ist nicht der Ort, sich
ausführlich mit dem Begriff des Indifferentismus auseinander zu setzen, den
\name[Immanuel]{Kant} den theologischen Auseinandersetzungen des 17. und 18.
Jahrhunderts entnimmt. Siehe dazu die kurze Einordnung sowie die weiteren
Verweise in \cite[][\pno~511\,f.,]{Gierl:PietismusundAufklaerung1997} sowie
\cite[][\pno~513\,f.]{Albrecht:Eklektik1994}. \name[Immanuel]{Kant} übersetzt
den Ausdruck \enquote{Indifferentism} durchgängig mit
\enquote{Gleichgültigkeit} \mkbibparens{siehe
\cite[][A 96]{Kant:BeobachtungenueberdasGefuehldesSchoenenundErhabenen1977},
\cite[][II: 250.11]{Kant:GesammelteWerke1900ff.}} und bezeichnet die
Indifferentisten in Fragen der Moral auch als \emph{Latitudinarier
der Neutralität} \mkbibparens{\cite[][B
9]{Kant:DieReligioninnerhalbderGrenzenderblossenVernunft1977}, \cite[][VI:
22.19--28]{Kant:GesammelteWerke1900ff.}}.} Diese Gleichgültigkeit gegenüber
metaphysischen Fragen sei -- so \name[Immanuel]{Kant} -- eine \enquote{Wirkung
{\punkt} der gereiften \ori{Urteilskraft} des Zeitalters,}\footnote{\cite[][A
xi]{Kant:KritikderreinenVernunft2003}, \cite[][IV:
9.2--3]{Kant:GesammelteWerke1900ff.}.} welches auch \enquote{das eigentliche
Zeitalter der \ori{Kritik}}\footnote{\cite[][A
xi]{Kant:KritikderreinenVernunft2003}, \cite[][IV:
9.33]{Kant:GesammelteWerke1900ff.}.} sei und \emph{als solches}
von \name[Immanuel]{Kant} wohl auch noch immer als das Zeitalter der Aufklärung
angesehen wird. Doch eine solche Gleichgültigkeit lasse sich nur scheinbar
aufrecht erhalten, denn Metaphysik -- oder Philosophie im eigentlichen Sinne --
sei nicht verzichtbar, sie lasse sich höchstens mehr oder minder gut unkenntlich
machen. \name[Immanuel]{Kant} nennt -- so ergibt sich bei genauerem Lesen --
zwei Gründe für die Unmöglichkeit eines Verzichts auf Metaphysik, die ich im
folgenden explizieren werde:
Zum einen gebe es innerhalb der Metaphysik Aufgaben, die wir nicht vermeiden können
(Abschnitt
\ref{subsubsection:UnvermeidlicheAufgabenderMetaphysik}), zum
zweiten liege die Metaphysik jedem Denken als solchem zugrunde
(Abschnitt \ref{subsubsection:MetaphysikalsGrundlageunseresDenkens}).\footnote{Diese Punkte
entsprechen der Einteilung der Metaphysik in eine \emph{metaphysica generalis}
und eine \emph{metaphysica specialis}. Ich gehe an dieser Stelle nicht weiter
auf diese Einteilung  und ihre historischen Hintergründe ein. Siehe dazu ausführlich
\cite{Vollrath:DieGliederungderMetaphysikineineMetaphysicaGeneralisundeineMetaphysicaSpecialis1962}.}

\begin{nummerierung}
\item\label{subsubsection:UnvermeidlicheAufgabenderMetaphysik}
Als ersten Grund für die Unmöglichkeit, auf Metaphysik einfach zu verzichten,
führt \name[Immanuel]{Kant} an, es gebe Themen und Fragestellungen, mit denen
wir uns früher oder später beschäftigen müssen, weil sie der menschlichen
Natur nicht gleichgültig sein \emph{können}:
\begin{quote}
Diese unvermeidlichen Aufgaben der reinen Vernunft selbst, sind \ori{Gott},
\ori{Freiheit} und \ori{Unsterblichkeit}. Die Wissenschaft aber, deren
Endabsicht mit allen ihren Zurüstungen eigentlich nur auf die Auflösung
derselben gerichtet ist, heißt \ori{Metaphysik}, deren Verfahren im Anfange
\ori{dogmatisch} ist, d.\,i. ohne vorhergehende Prüfung des Vermögens oder
Unvermögens der Vernunft zu einer so großen Unternehmung zuversichtlich die
Ausführung übernimmt.\footnote{\cite[][B 7]{Kant:KritikderreinenVernunft2003},
\cite[][III: 31.6--12]{Kant:GesammelteWerke1900ff.}.}
\end{quote}
Warum sind diese Aufgaben unvermeidlich? Können wir nicht sehr gut auskommen,
ohne die Frage nach der Existenz Gottes zu stellen? Müssen wir diese Frage
beantworten oder können wir uns nicht einfach auf den Standpunkt eines
generellen \emph{Ignorabimus} zurückziehen? Gerade wenn die Vernunftkritik
zeigt, dass genannte Fragen die Grenzen unseres Wissens übersteigen, wenn sie
\enquote{das Wissen aufheb[t], um zum Glauben Platz zu
bekommen,}\footnote{\cite[][B xxx]{Kant:KritikderreinenVernunft2003},
\cite[][III: 19.6]{Kant:GesammelteWerke1900ff.}.} scheint doch Metaphysik als
\emph{Erkenntnis} des Übersinnlichen obsolet zu werden.
Die Frage beispielsweise, ob das Universum einen Anfang in der Zeit hat, oder ob
es räumlich in Grenzen eingeschlossen oder unendlich (oder unbegrenzt, aber
endlich) ist, können wir unbeantwortet lassen. Wir können genauso eine Antwort
verweigern wie in vertrauten Fällen, wenn wir nach einer Auskunft gefragt werden
und eine Antwort unter Verweis auf unser eigenes Nichtwissen verweigern. In
vielen Fällen ist möglich, was die antiken Skeptiker als
{\epoche} bezeichnen.

Es ist verlockend, \name[Immanuel]{Kant}s Festhalten an der Notwendigkeit dieser
Fragen einfach der Tatsache zuzuschreiben, dass er im 18. Jahrhundert lebte und
die Selbstverständlichkeiten dieser Zeit artikulierte. Aber diese
Selbstverständlichkeiten müssen nicht die unseren sein. In Kapitel
\ref{chapter:AufklaerungundWissenschaft} war die Situation ähnlich:
\name[Immanuel]{Kant} stellt in der Aufklärungsschrift ganz selbstverständlich
die Religion in das Zentrum seiner Ausführungen, weil Unmündigkeit in Fragen der
Religion die schädlichste und entehrendste sei. Warum dies so ist, erörtert er
nicht; und der Leser mag den Verdacht hegen, dass dieses Urteil den
zeitlichen Umständen des 18. Jahrhunderts geschuldet ist. Es fand sich jedoch eine
Rechtfertigung dieses Urteils in den Überlegungen zur Bestimmung des Menschen,
die sich an der Ethik orientiert und Fragen unserer Handlungsausrichtung
betrifft.\footnote{Siehe oben, Kap. \ref{subsection:DieBestimmungdesMenschen}.}
Und dies scheint mir auch im Falle der Metaphysik der Hintergrund von
\name[Immanuel]{Kant}s Urteil zu sein, es gebe ein Bedürfnis der Vernunft. Unter
einem \singlequote{Bedürfnis} darf freilich kein empirisches Bedürfnis
verstanden werden, welches wir \singlequote{empfinden} und das uns dazu drängt,
diese Fragen zu stellen. Wenn das Bedürfnis ein Bedürfnis \emph{der Vernunft}
ist, dann kann es nicht bloß subjektiv unbefriedigend, sondern dann muss es
objektiv unvernünftig sein, diese Fragen nicht zu stellen.

Urteilsenthaltung ist überall dort keine vernünftige
Option, wo eine Frage gestellt wird, auf die eine Antwort zu haben für unser
Handeln nötig ist. \name[Immanuel]{Kant} denkt, dass die Fragen nach Gott und
Unsterblichkeit der Seele eben solche Fragen sind, die wir nicht übergehen
können, weil wir sie bei der vernünftigen Orientierung unseres Handelns
beantworten \emph{müssen}. Er wird diese Fragen in den Bereich des
\enquote{moralischen Glaubens} verweisen und behaupten, wir könnten sie wegen
eines Bedürfnisses der Vernunft, die Bedingungen unseres moralischen
Handlungserfolgs als gegeben anzunehmen, nicht abweisen. Entsprechend schreibt
\name[Immanuel]{Kant}:
\begin{quote}
Man kann aber das Bedürfnis der Vernunft als zwiefach ansehen: \ori{erstlich} in
ihrem \ori{theoretischen}, \ori{zweitens} in ihrem \ori{praktischen} Gebrauch.
Das erste habe ich eben angeführt; aber man sieht wohl, daß es nur bedingt sei,
d.\,i. wir müssen die Existenz Gottes annehmen, wenn wir über die ersten
Ursachen alles Zufälligen, vornehmlich in der Ordnung der wirklich in der Welt
gelegten Zwecke, \ori{urteilen wollen}.  Weit wichtiger ist das Bedürfnis der
Vernunft in ihrem praktischen Gebrauche, weil es unbedingt ist, und wir die
Existenz Gottes voraus zu setzen nicht bloß alsdann genötigt werden, wenn wir
urteilen \ori{wollen}, sondern weil wir \ori{urteilen
müssen}.\footnote{\cite[][A 315]{Kant:Washeisst:SichimDenkenorientieren?1977},
\cite[][VIII: 139.6--15]{Kant:GesammelteWerke1900ff.}.}
\end{quote}
Wir können Fragen der theoretischen Vernunft unbeantwortet lassen und unser
Urteil zurückhalten. Aber Fragen der praktischen Vernunft lassen keine
{\epoche} zu, weil wir \emph{handeln müssen} und dazu \emph{urteilen
müssen}.

Nun ist es zunächst die Metaphysik \emph{der Sitten}, also Metaphysik im
praktischen Vernunftgebrauch, die wir aus diesem Grund nicht vermeiden können.
Wir können nicht handeln, ohne uns irgendwie zu Fragen der Moral zu verhalten.
Und moralische Aussagen, die nicht davon handeln, wie wir \emph{de facto}
handeln, sondern davon, wie wir handeln \emph{sollen}, können nicht durch die
Erfahrung beantwortet werden, sind also allesamt metaphysisch. Dass
\name[Immanuel]{Kant} sie im genannten Zitat nicht erwähnt, ist sicherlich der
Tatsache geschuldet, dass er sich in der \titel{Kritik der reinen Vernunft} eben
ausschließlich mit dem reinen spekulativen Vernunftgebrauch, also der Metaphysik
der Natur befasst. Im praktischen Vernunftgebrauch wiederum scheint
\name[Immanuel]{Kant} die Möglichkeit objektiver Urteile für nicht weiter
problematisch zu halten.

Nach \name[Immanuel]{Kant} gibt es darüber hinaus im spekulativen
Vernunftgebrauch metaphysische Urteile, die durch keine objektiv gültigen Gründe
entschieden werden können, die aber mit unseren Handlungen so eng verbunden
sind, dass uns eine Antwort abverlangt wird: Dazu gehören die Existenz Gottes
und die Unsterblichkeit unserer Seele, sowie unsere Freiheit im Sinne einer
Unabhängigkeit von uns determinierenden Ursachen.\footnote{\enquote{Diese
Postulate sind die der \ori{Unsterblichkeit}, der \ori{Freiheit}, positiv
betrachtet (als der Kausalität eines Wesens, so fern es zur intelligibelen Welt
gehört), und das \ori{Dasein Gottes}. Das \ori{erste} fließt aus der praktisch
notwendigen Bedingung der Angemessenheit der Dauer zur Vollständigkeit der
Erfüllung des moralischen Gesetzes; das \ori{zweite} aus der notwendigen
Voraussetzung der Unabhängigkeit von der Sinnenwelt und des Vermögens der
Bestimmung seines Willens, nach dem Gesetze einer intelligibelen Welt, d.\,i.
der Freiheit; das \ori{dritte} aus der Notwendigkeit der Bedingung zu einer
solchen intelligibelen Welt, um das höchste Gut zu sein, durch die Voraussetzung
des höchsten selbständigen Guts, d.\,i. des Daseins Gottes}
\mkbibparens{\cite[][A 238\,f.,]{Kant:KritikderpraktischenVernunft1974}
\cite[][V: 132.19--29]{Kant:GesammelteWerke1900ff.}}.}
\name[Immanuel]{Kant} sagt, dass die Unsterblichkeit der Seele und die Existenz
Gottes jenseits des Bereichs unseres Wissens liegen, wir aber begründeter und
vernünftiger Weise an sie \singlequote{glauben} können.\footnote{Bezüglich
unserer Freiheit sagt er freilich, dass wir wenigstens ihr \emph{Möglichkeit}
wissen können: \enquote{Freiheit ist aber auch die einzige unter allen Ideen
der spek. Vernunft, wovon wir die Möglichkeit a priori \emph{wissen}, ohne sie
doch einzusehen, weil sie die Bedingung des moralischen Gesetzes ist, welches
wir wissen} \mkbibparens{\cite[][A 5]{Kant:KritikderpraktischenVernunft1974},
\cite[][V: 4.7--10]{Kant:GesammelteWerke1900ff.}}.} Diese Fragen, bei denen
unsere Erkenntnis an ihre Grenzen stößt, betreffen ausgerechnet das
\emph{Interesse} der Metaphysik und korrespondieren dem Weltbegriff der
Philosophie.

\item\label{subsubsection:MetaphysikalsGrundlageunseresDenkens}
Der andere Grund lautet: Jeder stellt metaphysische Behauptungen auf, sobald
\enquote{er nur überall etwas
denk[t]}\footnote{\cite[][A x]{Kant:KritikderreinenVernunft2003}, \cite[][IV:
8.33]{Kant:GesammelteWerke1900ff.}.}. Die transzendentale Analytik zeigt, dass
jeder objektive Gedanke ein metaphysisches Fundament hat: Es gibt keinen
Gedanken, der frei von Metaphysik ist, weil jedes Denken von Kategorien und
Grundsätzen des reinen Verstandes Gebrauch macht. Der \textit{Verstand} schreibt
der Natur \emph{a priori} theoretische Gesetze vor, die Kategorien und Grundsätze der
transzendentalen Analytik. Hier fügt sich die \singlequote{kopernikanische
Wende}\footnote{Der Ausdruck \enquote{kopernikanische Wende} oder
\enquote{kopernikanische Revolution} lässt sich bei \name[Immanuel]{Kant}
selbst nicht belegen und auch der genaue Sinn der Analogie ist umstritten
\parencite[siehe
dazu][]{Schoenecker:Kantskopernikanisch-newtonischeAnalogie2011}.} oder -- im
originalen Wortlaut \name[Immanuel]{Kant}s -- die \enquote{Revolution der
Denkart}\footnote{\cite[][B xi]{Kant:KritikderreinenVernunft2003}, \cite[][III:
9.19]{Kant:GesammelteWerke1900ff.}.}, die in der Vorrede zur zweiten Auflage der
Vernunftkritik beschrieben wird, in die Aufklärungskonzeption ein:
Es ist denkbar, dass der Verstand autonom ist, wir uns also in der Erkenntnis
der natürlichen Welt als frei handelnd und nicht als passiv erleidend verstehen
können, weil angenommen werden kann, dass sich die Dinge in der
Welt als Gegenstände unserer Erkenntnis nach unserem Verstand und seinen
Gesetzen richten.\footnote{\cite[Vgl.][B
xvi--xviii]{Kant:KritikderreinenVernunft2003}, \cite[][III:
12.3--33]{Kant:GesammelteWerke1900ff.}.} Die Notwendigkeit metaphysischer
Grundlagen des Denkens führt uns zum Begriff der Autonomie des Denkens und
Erkennens. Dieser Begriff wurde bereits im
\ref{section:KantalsliberalerAufklaerer}. Kapitel verwendet und vorausgesetzt,
aber noch nicht geklärt. Hier ergibt sich gleich die Gelegenheit, den Begriff autonomen
Erkennens mit Inhalt zu füllen.
\end{nummerierung}

\Revision{Man mag hier einwenden: Im Anhang zur transzendentalen Dialektik
findet sich ein weiterer Ansatz einer \distanz{kritischen Metaphysik}, die sich von einer dogmatischen
Metaphysik primär darin unterscheidet, dass ihre Erkenntnisse als
\emph{regulativ} verstanden werden. \name[Immanuel]{Kant} rekurriert dort
zunächst auf die Ansicht, dass es ein natürlicher Hang \emph{der Vernunft} (und
nicht etwa individueller Irrtum) sei, dass wir uns über die Grenzen der Vernunft
hinaus bewegen.\footnote{\cite[Vgl.][B 670]{Kant:KritikderreinenVernunft2003},
\cite{Kant:GesammelteWerke1900ff.}.} Es ist kein Irrtum, sondern Ausdruck von
Vernunft, sich um immer größere Systematik und Einheitlichkeit unseres Wissens
zu bemühen. Und die größte Einheitlichkeit sei erst durch Abschluss der
Wissenschaft im Bereich der Metaphysik möglich. Dies sei auch nicht schädlich,
wenn man nur im Auge behalte, dass damit nichts objektiv über die Gegenstände
der Metaphysik erwiesen
ist.\footnote{\cite[Vgl.][B 673--675]{Kant:KritikderreinenVernunft2003},
\cite[][III: 428.19--430.2]{Kant:GesammelteWerke1900ff.}.}}

\Revision{Diese Überlegungen überführt \name[Immanuel]{Kant} in der
\titel{Kritik der Urteilskraft} in die Idee einer formalen Zweckmäßigkeit der Natur für unsere
Urteilskraft. Ich werde dies unter dem Titel der \enquote{Heautonomie}
diskutieren\footnote{Siehe unten, ab S. \pageref{AutonomiederUrteilskraft}.}.
Nach Auskunft der \titel{Prolegomena} gehören diese Überlegungen als
\singlequote{Scholien} gar nicht in das System der Metaphysik und gingen auch
nur diejenigen etwas an, die professionell Metaphysik betreiben und um
Vollständigkeit bemügt
sind.\footnote{\cite[Vgl.][\S~60]{Kant:ProlegomenazueinerjedenkuenftigenMetaphysikdiealsWissenschaftwirdauftretenkoennen1977},
\cite[][IV: 362.5--365.4]{Kant:GesammelteWerke1900ff.}.} In der \titel{Kritik
der Urteilskraft} wird die Frage nach der größtmöglichen Systematik und
Einheitlichkeit schließlich nicht mehr als Aufgabe der Vernunft beschrieben,
sondern der reflektierenden Urteilskraft zugeschrieben und unter dem Stichwort
der \enquote{Heautonomie} oder der \enquote{formalen Zweckmäßigkeit der Natur}
behandelt. Ich werde dieser (späteren) Einordnung folgen und die
\singlequote{regulative Metaphysik} der \titel{Kritik der reinen Vernunft} hier
nicht weiter behandeln.}

Bevor ich auf die Autonomie des Denkens eingehe, fasse ich kurz zusammen:
Es gibt zwei Gründe, aus denen wir nicht auf Metaphysik verzichten können; und
diesen Gründen korrespondieren jeweils eigene Bereiche der Metaphysik: Wir
müssen metaphysische Behauptungen machen, sobald wir \emph{handeln} wollen. Und wir müssen andere
metaphysische Behauptungen machen, wenn wir \emph{denken} wollen. Zu den
Voraussetzungen des Handelns gehören die Metaphysik der Sitten, die rationale
Psychologie und die rationale Theologie. Zur Voraussetzung des Denkens gehört
die Ontologie.

\subsection{Autonomie und Spontaneität}\label{subsection:MetaphysikundAutonomie}
Nach \name[Immanuel]{Kant}s Bestimmung dessen, was es heißt, selbst zu denken,
sollen wir unsere je eigene Vernunft zum obersten Probierstein der Wahrheit
machen. Diese Forderung erwies sich jedoch \emph{prima facie} als zu
anspruchsvoll, wenn wir daraus die Aufforderung ableiten, die Wahrheit einer
jeden Überzeugung selbst -- ohne Vertrauen auf die Mitteilungen anderer oder gar
ohne Rekurs auf die eigene Wahrnehmung -- zu kontrollieren. In Kapitel
\ref{section:MuendigkeitundPhilosophie} zeigte ich, dass dieser Anspruch nur auf
rationale Erkenntnisse, also Mathematik und Philosophie bezogen werden kann. Im
folgenden werde ich zeigen, dass die Beziehung dieser Forderung auf die
Metaphysik als den Kern der Philosophie geeignet ist, die Aussage zu
interpretieren, wir sollten unsere eigene Vernunft zum obersten Probierstein der
Wahrheit machen.

Es sind die oberen Erkenntnisvermögen, die auf
Prinzipien a priori beruhen\footnote{\cite[Vgl.][B 243]{Kant:KritikderUrteilskraft2009}, \cite[][V:
345.5--6]{Kant:GesammelteWerke1900ff.}.} und daher a priori gesetzgebend und
somit \emph{autonom} sind. Obere Erkenntnisvermögen sind der Verstand, die
Vernunft und die Urteilskraft, wobei das Gesamt des oberen Erkenntnisvermögens
wahlweise als Vernunft oder als Verstand bezeichnet wird. Denn neben der
\singlequote{engeren} Verwendung als Ausdruck für eines der drei oberen
Erkenntnisvermögen neben \enquote{Urteilskraft} und \enquote{Vernunft} könne der
Ausdruck \enquote{Verstand} auch für das gesamte obere Erkenntnisvermögen
einschließlich Vernunft und Urteilskraft stehen.\footnote{\cite[Vgl.][BA
115\,f.,]{Kant:AnthropologieinpragmatischerHinsicht1977} \cite[][VII:
196.17--197.3]{Kant:GesammelteWerke1900ff.}.} Und manchmal verwendet
\name[Immanuel]{Kant} eben auch den Terminus \enquote{Vernunft} in einem weiten
Sinne als Ausdruck für das gesamt obere
Erkenntnisvermögen.\footnote{\cite[Vgl.][B 863]{Kant:KritikderreinenVernunft2003}, \cite[][III:
540.28]{Kant:GesammelteWerke1900ff.}.}

\phantomsection\label{Abschnitt:IstSinnlichkeiteinErkenntnisvermögen} Das untere
Erkenntnisvermögen hingegen ist nach \name[Immanuel]{Kant} das Vermögen sinnlicher Erkenntnisse.
Hier ließe sich aber fragen: Handelt es sich bei der Sinnlichkeit überhaupt um
ein Vermögen? In der Anthropologie findet sich ein Satz, der dies verneint:
\enquote{In Ansehung des Zustandes der Vorstellungen ist mein Gemüt entweder
\ori{handelnd} und zeigt \ori{Vermögen} (facultas), oder es ist \ori{leidend} und besteht in \ori{Empfänglichkeit}
(receptivitas).}\footnote{\cite[][BA
25]{Kant:AnthropologieinpragmatischerHinsicht1977},
\cite[][VII: 140.16--18]{Kant:GesammelteWerke1900ff.}.} Demnach ist der Verstand
als das gesamte obere Erkenntnisvermögen ein Vermögen, aber die Sinnlichkeit als bloße
Rezeptivität ist kein Vermögen; denn der Begriff des Vermögens setzt die
Fähigkeit zu einer Handlung -- zu einem aktiven Tun -- voraus, während die
Sinnlichkeit in einem bloß passiven Erleiden besteht. Später bezeichnet Kant die
Sinnlichkeit jedoch explizit als unteres Erkenntnisvermögen:
\begin{quote}
\ori{Verstand}, als das Vermögen zu \ori{denken} (durch \ori{Begriffe} sich
etwas vorzustellen), wird auch das \ori{obere} Erkenntnisvermögen (zum
Unterschiede von der Sinnlichkeit, als des \ori{unteren}) genannt, darum, weil
das Vermögen der Anschauungen (reiner oder empirischer) nur das Einzelne in
Gegenständen, dagegen das der Begriffe das Allgemeine der Vorstellungen
derselben, die \ori{Regel}, enthält, der das Mannigfaltige der sinnlichen
Anschauungen untergeordnet werden muß, um Einheit zur Erkenntnis des Objekts
hervorzubringen.\footnote{\cite[][BA
115]{Kant:AnthropologieinpragmatischerHinsicht1977},
\cite[][VII: 196.17--24]{Kant:GesammelteWerke1900ff.}.}
\end{quote}
In der \titel{Anthropologie} stellt Kant zunächst fest, dass Erkenntnis nur aus
dem \emph{Zusammenspiel} von Rezeptivität und Spontaneität resultieren kann. Die
Möglichkeit, Erkenntnis zu haben, die aus diesem Zusammenspiel beider Stämme
oder Grundquellen resultiert -- und nicht die Stämme selbst --, heiße nun
Erkenntnisvermögen.\footnote{\enquote{Ein \ori{Erkenntnis} enthält beides verbunden in sich und die Möglichkeit, eine solche zu haben, führt den Namen
des \ori{Erkenntnisvermögens} von dem vornehmsten Teil derselben, nämlich der
Tätigkeit des Gemüts, Vorstellungen zu verbinden, oder voneinander zu sondern}
\mkbibparens{\cite[][BA 25]{Kant:AnthropologieinpragmatischerHinsicht1977},
\cite[][VII: 140.18--22]{Kant:GesammelteWerke1900ff.}}.} Danach ist weder die
Sinnlichkeit noch der Verstand als Erkenntnisvermögen zu bezeichnen, sondern die Vermögen, die wir
vermittels des Zusammenspiels von Sinnlichkeit und Verstand besitzen. 

Nach \name[Immanuel]{Kant} gehören alle diejenigen Vorstellungen zum unteren
oder sinnlichen Erkenntnisvermögen, bei denen sich das Gemüt (auch!) leidend
verhält. Zum oberen oder intellektuellen Erkenntnisvermögen gehören hingegen
diejenigen Vorstellungen, in Ansehung derer sich das Gemüt \emph{ausschließlich} tätig verhält, die also nur ein Tun
(das Denken) enthalten:
\begin{quote}
Vorstellungen, in Ansehung deren sich das Gemüt leidend verhält, durch welche
also das Subjekt \ori{affiziert} wird (dieses mag sich nun selbst affizieren
oder von einem Objekt affiziert werden), gehören zum \ori{sinnlichen}:
diejenigen aber, welche ein bloßes \ori{Tun} (das  Denken) enthalten, zum
\ori{intellektuellen} Erkenntnisvermögen. Jenes wird auch das \ori{untere},
dieses aber das \ori{obere} Erkenntnisvermögen
genannt.\footnote{\cite[][BA 25]{Kant:AnthropologieinpragmatischerHinsicht1977},
\cite[][VII: 140.23--28]{Kant:GesammelteWerke1900ff.}. Nach
\authorfullcite{Brandt:KritischerKommentarzuKantsenquoteAnthropologieinpragmatischerHinsicht1999}
handelt es sich dabei um eine distanzierte Darstellung fremder Redegebräuche,
denen sich \name[Immanuel]{Kant} nicht selbst anschließe, sie sogar zurückweise
\parencite[vgl.][174]{Brandt:KritischerKommentarzuKantsenquoteAnthropologieinpragmatischerHinsicht1999}.
Leider nennt
\authorcite{Brandt:KritischerKommentarzuKantsenquoteAnthropologieinpragmatischerHinsicht1999}
keine Gründe für diese These, die mit dem Wortlaut des Textes so gar nicht
harmonieren möchte.}
\end{quote}
Es gibt kein Erkenntnisvermögen, das ausschließlich sinnlich oder rezeptiv wäre;
in diesem Falle wäre es eben nicht einmal ein Vermögen. Nach dieser Einteilung
gehören lediglich die \emph{reinen} Verstandesbegriffe (und Grundsätze) zum
oberen Erkenntnisvermögen, während alle empirischen Er\-kennt\-nis\-se -- also
alle empirischen Begriffe und Urteile -- zum unteren Erkenntnisvermögen zu
zählen sind, wenngleich es natürlich dennoch Erkenntnisse sind, die der
\emph{Verstand} hervorbringt. Sinnlichkeit (Rezeptivität) ist selbst kein
Erkenntnisvermögen, weil sie selbst nicht urteilt und daher keine Erkenntnisse
hervorbringt. Nur der Verstand urteilt, aber er urteilt zum einen über das, was
er sinnlich wahrnimmt, und zum anderen über das, wofür er keiner Wahrnehmung
bedarf.

Es scheint mir notwendig zu sein zu sagen, dass wir in unterschiedlichen
Bedeutungen von Sinnlichkeit und Verstand sprechen, je nachdem, ob wir sie als
Erkenntnisstämme (oder -quellen) oder als Erkenntnisvermögen betrachten. Die
beiden Stämme der Erkenntnis sind jeweils für sich genommen nicht in der Lage,
Erkenntnisse zu produzieren. Als Erkenntnisvermögen genommen handelt es sich
nicht um die jeweiligen Erkenntnisstämme in Reinform, sondern um das Vermögen
(also die Fähigkeit), Erkenntnisse zu generieren, die als rationale Erkenntnisse
(\emph{a priori} und \emph{ex principiis}) dem oberen Erkenntnisvermögen
(\singlequote{Verstand}) oder als empirische Erkenntnisse (\emph{a posteriori}
und \emph{ex datis}) dem unteren Erkenntnisvermögen (\singlequote{Sinnlichkeit}) angehören.

Soll der Zusammenhang mit dem Begriff der Autonomie betont werden, dann bietet
sich der Ausdruck \enquote{Vernunft} an: \enquote{Nun nennt man das Vermögen,
nach der Autonomie, d.\,i. frei (Prinzipien des Denkens überhaupt gemäß) zu
urteilen, die Vernunft.}\footnote{\cite[][A 25]{Kant:DerStreitderFakultaeten1977},
\cite[][VII: 27.30--32]{Kant:GesammelteWerke1900ff.}.} Hier ist die Vernunft als
das obere Erkenntnisvermögen angesprochen und als solches mit dem Begriff der
Autonomie verbunden. Es ist nicht nur die Vernunft im engeren Sinne, sondern die
oberen Erkenntnisvermögen als solche, denen Autonomie zukommt. Autonomie
ist die Eigenschaft des Subjekts, nach \enquote{Prinzipien des Denkens
überhaupt} zu urteilen, und ein Vermögen, welches über eigene Prinzipien
\emph{a priori} verfügt, ist ein oberes
Erkenntnisvermögen.\footnote{\enquote{Daß es drei Arten der Antinomie gibt, hat
seinen Grund darin, daß es drei Erkenntnisvermögen: Verstand, Urteilskraft und
Vernunft gibt, deren jedes (als oberes Erkenntnisvermögen) seine Prinzipien a
priori haben muß} \mkbibparens{\cite[][B 243]{Kant:KritikderUrteilskraft2009},
\cite[][V: 345.3--6]{Kant:GesammelteWerke1900ff.}}.} Die oberen
Erkenntnisvermögen sind daher der Ort, an dem sich \name[Immanuel]{Kant}s
Aufklärungsprogramm entfaltet.

Prinzipien \emph{a priori} und die aus ihnen gewonnenen Erkenntnisse nennt
\name[Immanuel]{Kant} philosophisch und das System
philosophischer Erkenntnisse Metaphysik. Autonom sein heißt entsprechend, über
ein metaphysisches Fundament zu verfügen, wie es bei den oberen
Erkenntnisvermögen der Fall ist. Da es drei obere Erkenntnisvermögen gibt --
Verstand, Vernunft und Urteilskraft --, finden wir auch drei Fälle von Autonomie
vor\footnote{\cite[Vgl.][32]{Kant:ErsteEinleitungindieenquoteKritikderUrteilskraft2009},
\cite[][XX: 225.21--32]{Kant:GesammelteWerke1900ff.}.}: Die \textit{Vernunft}
gibt der \emph{Freiheit} das Gesetz, den kategorischen Imperativ.
Der \emph{Verstand} gibt der \emph{Natur} das Gesetz, die Kategorien und
Grundsätze des reinen Verstandes.\footnote{\enquote{Der Verstand ist a priori
gesetzgebend für die Natur als Objekt der Sinne, zu einem theoretischen
Erkenntnis derselben in einer möglichen Erfahrung. Die Vernunft ist a priori
gesetzgebend für die Freiheit und ihre eigene Kausalität, als das Übersinnliche
in dem Subjekte, zu einem unbedingt-praktischen Erkenntnis}
\mkbibparens{\cite[][B liii]{Kant:KritikderUrteilskraft2009},
\cite[][V: 195.4--8]{Kant:GesammelteWerke1900ff.}}.
Siehe auch \cite[][B xi--xiii]{Kant:KritikderUrteilskraft2009}, \cite[][V:
171.4--172.22]{Kant:GesammelteWerke1900ff.}.} Und die \emph{Urteilskraft} gibt
\emph{sich selbst} ein Gesetz (das transzendentale Prinzip der formalen
Zweckmäßigkeit der Natur), weswegen sie nicht einfach als \emph{autonom},
sondern genauer als \emph{heautonom} bezeichnet werden
könnte.\footnote{\enquote{Die Gesetzgebung
durch Naturbegriffe geschieht durch den Verstand und ist theoretisch. Die Gesetzgebung
durch den Freiheitsbegriff geschieht von der Vernunft und ist bloß praktisch}
\mkbibparens{\cite[][B xvii]{Kant:KritikderUrteilskraft2009},
\cite[][V: 174.32--34]{Kant:GesammelteWerke1900ff.}}. \enquote{Die Urteilskraft hat also auch ein Prinzip a priori
für die Möglichkeit der Natur, aber nur in subjektiver Rücksicht, in sich, wodurch die nicht der Natur (als Autonomie),
sondern ihr selbst (als Heautonomie) für die Reflexion über jene ein Gesetz
vorschreibt, welches man das Gesetz der Spezifikation der Natur in Ansehung
ihrer empirischen Gesetze nennen könnte} \mkbibparens{\cite[][B
xxxvii]{Kant:KritikderUrteilskraft2009}, \cite[][V:
185.35--186.3]{Kant:GesammelteWerke1900ff.}}.
\enquote{Diese Gesetzgebung müßte man eigentlich Heautonomie nennen, da die
Urtheilskraft nicht der Natur, noch der Freyheit, sondern lediglich ihr selbst
das Gesetz giebt und kein Vermögen ist, Begriffe von Objekten hervorzubringen,
sondern nur mit denen, die ihr anderweitig gegeben sind, vorkommende Fälle zu
vergleichen und die subjective Bedingungen der Möglichkeit dieser Verbindung a
priori anzugeben}
\mkbibparens{\cite[][32]{Kant:ErsteEinleitungindieenquoteKritikderUrteilskraft2009},
\cite[][XX: 225.27--32]{Kant:GesammelteWerke1900ff.}}.}


Dass ein Erkenntnisvermögen autonom ist, heißt gerade, dass es auf Prinzipien
\emph{a priori} beruht. Und in Anlehnung an \name[Immanuel]{Kant}s Begriff der
Metaphysik ergibt sich: Ein
Erkenntnisvermögen ist autonom genau dann, wenn es eine Grundlage in der
Metaphysik hat. Das ist eine begriffliche Wahrheit über den Begriff der
Autonomie. Diese Autonomie ist eine \emph{Folge} davon, dass es sich um
obere Erkenntnisvermögen handelt; um den Ursprung der Autonomie zu erkennen,
gilt es somit, den Begriff des oberen Erkenntnisvermögens zu explizieren. In der
\titel{Kritik der reinen Vernunft} klärt \name[Immanuel]{Kant} den Begriff eines
oberen Erkenntnisvermögens nicht explizit, gibt aber folgende Auskunft, die
zunächst dafür spricht, den Begriff des oberen Erkenntnisvermögens unter
Rückgriff auf den Begriff der Autonomie zu erläutern:
\enquote{Ich verstehe hier aber unter Vernunft das ganze obere
Erkenntnisvermögen, und setze also das Rationale dem Empirischen
entgegen.}\footnote{\cite[][B 863]{Kant:KritikderreinenVernunft2003},
\cite[][III: 540.27--29]{Kant:GesammelteWerke1900ff.}.}
Das Wort \enquote{also} zeigt an, dass die Identifizierung der Vernunft mit dem
Rationalen eine Folge dessen ist, dass sie als das ganze obere
Erkenntnisvermögen betrachtet wird. Das obere Erkenntnisvermögen ist also ein
Vermögen rationaler Erkenntnis und wird als solches dem Vermögen, empirische
Erkenntnis zu generieren, gegenübergestellt. Danach ist ein oberes
Erkenntnisvermögen ein solches, mittels dessen wir zu rationalen Erkenntnissen,
also Erkenntnissen \emph{ex principiis} fähig sind. Das untere
Erkenntnisvermögen, zu dem Sinnlichkeit und Einbildungskraft gehören, liefert
empirische Erkenntnisse oder Erkenntnisse \emph{ex datis}.\footnote{Siehe hierzu
oben Kapitel \ref{section:MuendigkeitundPhilosophie}.}

In der \titel{Anthropologie in pragmatischer Hinsicht} findet sich eine
Bestimmung des Begriffs des oberen Erkenntnisvermögen, die auf ein anderes
Merkmal als den Erwerb rationaler Erkenntnisse verweist. Hier steht nicht der
Begriff der Vernunft, sondern der Begriff des \emph{Verstandes} im Vordergrund:
\begin{quote}
In Ansehung des Zustandes der Vorstellungen ist mein Gemüt entweder
\ori{handelnd} und zeigt \ori{Vermögen} (facultas), oder es ist \ori{leidend}
und besteht in \ori{Empfänglichkeit} (receptivitas). {\punkt} Vorstellungen, in
Ansehung deren sich das Gemüt leidend verhält {\punkt} gehören zum
\ori{sinnlichen}: diejenigen aber, welche ein bloßes \ori{Tun} (das Denken)
enthalten, zum \ori{intellektuellen} Erkenntnisvermögen. Jenes wird auch das
\ori{untere}, dieses aber das \ori{obere} Erkenntnisvermögen
genannt.\footnote{\cite[][BA 25]{Kant:AnthropologieinpragmatischerHinsicht1977},
\cite[][VII: 140.16--28]{Kant:GesammelteWerke1900ff.}.}
\end{quote}
Der Verstand ist das Vermögen der Spontaneität im Unterschied zur Sinnlichkeit
als der Rezeptivität: \enquote{Wollen wir die \ori{Rezeptivität} unseres Gemüts,
Vorstellungen zu empfangen, so fern es auf irgend eine Weise affiziert wird,
\ori{Sinnlichkeit} nennen; so ist dagegen das Vermögen, Vorstellungen selbst
hervorzubringen, oder die \ori{Spontaneität} des Erkenntnisses, der
\ori{Verstand}.}\footnote{\cite[][B 75]{Kant:KritikderreinenVernunft2003},
\cite[][III: 75.5--8]{Kant:GesammelteWerke1900ff.}. Siehe auch Kapitel
\ref{subsection:DiskursiverVerstandundsinnlicheAnschauung} dieser Arbeit.}. Im
Unterschied zu den bisher betrachteten Explikationen geht es \name[Immanuel]{Kant} hier
\emph{expressis verbis} um eine Bestimmung des Begriffs des oberen
Erkenntnisvermögens; das angeführte Merkmal ist also grundlegend und nicht --
wie das der Autonomie -- abgeleitet. Dieses grundlegende Merkmal des oberen
Erkenntnisvermögens ist die Selbsttätigkeit oder \emph{Spontaneität}, die den
Verstand als den einen Stamm der menschlichen Erkenntnis neben der Sinnlichkeit
ausmacht. Zu sagen, dass ein solches Vermögen \emph{als} oberes
Erkenntnisvermögen ein Vermögen ist, das über Prinzipien \emph{a priori}
verfügt, heißt, den Begriff der Selbsttätigkeit oder Spontaneität weiter zu
explizieren.

In Kapitel \ref{subsection:DerBegriffdesSelbstdenkens} hatte ich von einem
positiven und einem negativen Begriff des Selbstdenkens in Analogie zum
negativen und positiven Begriff der Freiheit in der \titel{Grundlegung zur
Metaphysik der Sitten} gesprochen. Hier begegnet ein solcher Kontrast wiederum.
Als Verstand ist das obere Erkenntnisvermögen nicht leidend äußeren Einflüssen
unterworfen, sondern selbsttätig -- spontan. Dies entspricht dem negativen
Begriff des Selbstdenkens. Als Vernunft wiederum verfügt das obere
Erkenntnisvermögen über eigene Grundsätze \emph{a priori}; dies entspricht dem
positiven Begriff des Selbstdenkens. Diese Grundsätze, die autonomen Gesetze des
Verstandes und der Urteilskraft sowie der Vernunft sind der letzte Probierstein
der Wahrheit, den \name[Immanuel]{Kant} in seiner Konkretisierung des Begriffs
des Selbstdenkens anführt. Dieser Maßstab reicht freilich nicht aus, um an ihm
allein die Wahrheit unserer Urteile zu überprüfen.
Urteile können diesem Maßstab entsprechen und dennoch falsch sein. Aber ein
Urteil, welches diesen Maßstäben schon nicht entspricht, dessen Wahrheit ist
unmöglich und darf von uns auch nicht angenommen werden. Ich werde im folgenden
die Autonomie der Vernunft (\ref{AutonomiederVernunft}), des Verstandes
(\ref{AutonomiedesVerstandes}) und der Urteilskraft
(\ref{AutonomiederUrteilskraft}) beschreiben, um diesen Gedanken zu
konkretisieren.
\begin{nummerierung}
\item\label{AutonomiederVernunft} Die Vernunft gibt der
Freiheit das Gesetz. Wie \name[Immanuel]{Kant} insbesondere in der
\titel{Grundlegung zur Metaphysik der Sitten} betont, ist nur die Vorstellung von Autonomie in der Lage, den
allgemein bindenden Charakter moralischer Normen verständlich zu
machen.\footnote{\enquote{Es ist kein Wunder, wenn wir auf alle bisherige
Bemühungen, die jemals unternommen worden, um das Prinzip der Sittlichkeit
ausfündig zu machen, zurücksehen, warum sie insgesamt haben fehlschlagen müssen.
Man sahe den Menschen durch seine Pflicht an Gesetze gebunden, man ließe es sich
aber nicht einfallen, daß er nur seiner eigenen und dennoch allgemeinen
Gesetzgebung unterworfen sei, und daß er nur verbunden sei, seinem eigenen, dem
Naturzwecke nach aber allgemein gesetzgebenden, Willen gemäß zu handeln}
\mkbibparens{\cite[][BA 73]{Kant:GrundlegungzurMetaphysikderSitten1965},
\cite{Kant:GesammelteWerke1900ff.}}.}  Alle anderen, auf Heteronomie gegründeten
ethischen Theorien konnten den Begriff der Pflicht, der wir als endliche Wesen
unterworfen sind, nicht entwickeln. Und nur der Begriff der Autonomie mache es
möglich, dass ein Wesen als endliches Wesen Gesetzen unterworfen ist, die es
\emph{nötigen} und ihm \emph{befehlen}, und dennoch seine Würde (als
selbst gesetzgebend) bewahrt.\footnote{\Revision[Selbst]{\cite[Vgl.][BA 86\,f.]{Kant:GrundlegungzurMetaphysikderSitten1965},
\cite[][IV: 440.7--13]{Kant:GesammelteWerke1900ff.}.}} Dies mag zunächst paradox anmuten,
denn Grundsätze, die ich mir lediglich selbst gebe, scheinen doch solche zu
sein, von denen ich mich auch jederzeit selbst entbinden kann.\footnote{Vgl.
\cite[][226]{Pinkard:GermanPhilosophy1760-18602002}.} Deshalb dürfen wir
Autonomie nicht in einer Weise vorstellen, nach der wir uns die
bestimmenden Grundsätze selbst in einem willkürlichen Akt selbst geben. Es sind
Grundsätze, die uns als Vernunftwesen
ausmachen.\footnote{\cite[Vgl.][90-130]{Korsgaard:TheSourcesofNormativity1996},
hier speziell S.~90\,f. } Nach \authorcite{Roedl:Selbstgesetzgebung2011} ist das Gesetz identisch mit dem
Willen, weil es lediglich ausdrückt, dass der Wille praktische Vernunft
ist.\footnote{\cite[Vgl.][]{Roedl:Selbstgesetzgebung2011}. Sie dazu auch
\cite[][]{Menke:AutonomieundBefreiung2010}.} Sollte eine solche Interpretation
korrekt sein, dann wäre klar, wie dieser Begriff der Autonomie die Vereinbarkeit
von Freiheit und Vernunft gewährleistet: Frei sein hieße, in seinem Denken und
Handeln die Grundsätze zur Geltung zu bringen, die uns als Vernunftwesen
ausmachen.\footnote{Der Begriff der Autonomie als Vermögen der Vernunft, der Freiheit Gesetze
vorzuschreiben, wird in unzähligen Beiträgen zur \name[Immanuel]{Kant}forschung
besprochen. Diese Beiträge können an dieser Stelle weder referiert werden, noch
soll ihnen ein weiterer Beitrag an die Seite gestellt werden. Neben den bereits
genannten Beiträgen sind zu nennen:
\cite{Sturma:KantsEthikderAutonomie2004,Gunkel:SpontaneitaetundmoralischeAutonomie1989,
Shell:KantandtheLimitsofAutonomy2009,Shell:KantandtheenquoteParadoxofAutonomy2012}..}

Da theoretische und praktische Vernunft keine getrennten Vermögen sind, sondern
unterschiedliche Gebrauchsarten desselben Vermögens, liegt es nahe, dass die
Vernunft im engeren Sinne nicht nur in der Ethik und allgemein im Handeln als
praktische, sondern gerade im Erkennen als theoretische Vernunft wirksam
ist.\footnote{Dies behauptet auch
\authorfullcite{Larmore:TheAutonomyofMorality2008}, der sich dazu auf die
Beschreibung der kopernikanischen Wende in der Vorrede zur zweiten Auflage der
\titel{Kritik der reinen Vernunft} beruft
\parencite[vgl.][41]{Larmore:TheAutonomyofMorality2008}.} Von der Vernunft ist
nun aber sowohl ein realer als auch ein bloß formaler Gebrauch denkbar. Der
Gebrauch eines Erkenntnisvermögens heißt dabei \emph{real}, wenn es selbst
Begriffe und Grundsätze hervorbringt. Er heißt logisch oder \emph{formal}, wenn
das Erkenntnisvermögen lediglich auf bereits gegebene Begriffe und Erkenntnisse
angewandt wird.\footnote{\enquote{Es gibt von ihr [der Vernunft; A.\,G.], wie
von dem Verstande, einen bloß formalen, d.\,i. logischen Gebrauch, da die
Vernunft von allem Inhalte der Erkenntnis abstrahiert, aber auch einen realen,
da sie selbst den Ursprung gewisser Begriffe und Grundsätze enthält, die sie
weder von den Sinnen, noch vom Verstande entlehnt} \mkbibparens{\cite[][B
355]{Kant:KritikderreinenVernunft2003}, \cite[][III:
237.26--30]{Kant:GesammelteWerke1900ff.}}. Siehe auch
\cite[][\S~5]{Kant:Demundisensibilisatqueintelligibilisformaetprincipiis1968},
\cite[][II: 393.16--22]{Kant:GesammelteWerke1900ff.}.} In Ansehung der
Naturerkenntnis können wir von unserer Vernunft (als theoretische Vernunft)
jedoch nur einen {formalen}, keinen {realen} Gebrauch machen:
\begin{quote}
Die Gesetzgebung durch Naturbegriffe geschieht durch den Verstand und ist
theoretisch. Die Gesetzgebung durch den Freiheitsbegriff geschieht von der
Vernunft und ist bloß praktisch. Nur allein im Praktischen kann die Vernunft
gesetzgebend sein; in Ansehung des theoretischen Erkenntnisses (der Natur) kann
sie nur (als gesetzkundig vermittelst des Verstandes) aus gegebenen Gesetzen
durch Schlüsse Folgerungen ziehen, die doch immer nur bei der Natur
stehenbleiben.\footnote{\cite[][B xvii]{Kant:KritikderUrteilskraft2009},
\cite[][V: 174.32--175.2]{Kant:GesammelteWerke1900ff.}.}
\end{quote}
Dies ist damit gemeint, dass die Vernunft der Freiheit, nicht der Natur das
Gesetz gebe. Der formale oder logische Gebrauch der Vernunft liegt unserem
Denken in allen Bereichen zugrunde; aber nur in Bezug auf unser Handeln gibt sie
uns genuine Vernunftbegriffe (den \emph{Freiheitsbegriff}) und
Vernunftprinzipien (moralisch-praktische Gesetze, den \emph{kategorischen
Imperativ}) an die Hand. Deswegen ist die Vernunft nur in der
Moralphilosophie autonom, während in der Naturphilosophie der Verstand Autonomie
ausübt.

\begin{comment}
Nicht nur in Bezug auf unser äußeres Handeln, sondern gerade auch
in Bezug auf unser Denken ist von einem Sollen zu sprechen. Im Gegensatz zu
\authorcite{Wolff:Discursuspraeliminarisdephilosophiaingenere1996}, der die
empirische Psychologie als eine der Grundlagendisziplinen der Logik ansieht,
führt \name[Immanuel]{Kant} hier eine klare Trennung ein:
\begin{quote}
Einige Logiker setzen zwar in der Logik \ori{psychologische} Prinzipien voraus.
Dergleichen Prinzipien aber in die Logik zu bringen, ist ebenso ungereimt, als
Moral vom Leben herzunehmen. Nähmen wir die Prinzipien aus der Psychologie,
d.\,h. aus den Beobachtungen über unsern Verstand, so würden wir bloß sehen,
\ori{wie} das Denken vor sich geht und \ori{wie es ist} unter den mancherlei
subjektiven Hindernissen und Bedingungen; dieses würde also zur Erkenntnis bloß
\ori{zufälliger} Gesetze führen. In der Logik ist aber die Frage nicht nach
\ori{zufälligen}, sondern nach \ori{notwendigen} Regeln; -- nicht, wie wir
denken, sondern, wie wir denken
sollen.\footnote{\cite[][A 6]{Kant:ImmanuelKantsLogik1977},
\cite[][IX: 14.3--12]{Kant:GesammelteWerke1900ff.}.}
\end{quote}



\name[Immanuel]{Kant} glaubt, dass der Begriff der Autonomie das Rätsel lösen
könne, wie Gesetze präskriptive Kraft haben können, ohne sich auf einen
vorausgesetzten Zweck zu beziehen. Nun wird häufig genug bestritten, dass dieser
Argumentationsweg überzeugt. 

\authorfullcite{Larmore:TheAutonomyofMorality2008} schließt daraus gar, dass es
sich bei der Vernunft um ein \emph{rezeptives} Vermögen -- also nach
\name[Immanuel]{Kant} um \emph{Sinnlichkeit} -- handelt.\footnote{\enquote{[W]e
must conclude, in a very un-\name[Immanuel]{Kant}ian spirit, that reason at
bottom is a receptive faculty} \parencite[][44]{Larmore:TheAutonomyofMorality2008}.} Aber das ist
zunächst einfach aporetisch, denn es lässt die Frage offen, woher die Maßstäbe
der Vernunft überhaupt kommen sollen, wenn nicht aus der Vernunft selbst. Die
Vernunft ist ihr einziger Maßstab, weil es nichts anderes gibt noch geben kann, was ihr rechtmäßiger
Weise etwas vorschreiben kann. Die Vermutung, wir müssten darauf schauen, wie
Menschen \emph{de facto} denken, scheint ebenso absurd wie der Verweis auf
Introspektion. Wir wissen einfach nicht, woher wir Regeln der Vernunft
\emph{rezeptiv} nehmen sollten, was es heißt, den Gebrauch unserer Vernunft an
etwas anderem als ihr selbst zu messen.\footnote{Allerdings gibt es im Anschluss
an Überlegungen von \authorfullcite{Frege:LogischeUntersuchungen1993} eine Diskussion, ob Gesetze der Logik nicht einfach allgemeinere Gesetze sind.
\authorfullcite{Glueer:BedeutungzwischenNormundNaturgesetz2000} greift die
These, unser Denken sei wesentlich in normativen Vorgaben der Vernunft fundiert,
unter explizitem Verweis auf die in der \jaeschelogik{} vertretene Position an.
Wir sollten die Gesetze der Logik nicht als kategorische Vorschriften an die
Vernunft, die diese sich selbst gibt, ansehen, sondern als \singlequote{Gesetze
des Wahrseins} im Sinne \authorcite{Frege:LogischeUntersuchungen1993}s analog zu
Naturgesetzen: Logische Gesetze handeln von den allgemeinsten Zusammenhängen der
Welt. Und unter der Bedingung, dass wir Wahres denken wollen, sollten wir im Einklang
mit diesen Gesetzen denken
\parencite[vgl.][]{Glueer:BedeutungzwischenNormundNaturgesetz2000}. Dies
schließt offenkundig an Überlegungen
\authorfullcite{Quine:TwoDogmasofEmpiricism1951}s an, wonach es keine strikte
Trennung von apriorischen und aposteriorischen Erkenntnissen gibt, sondern nur
mehr oder minder zentrale und periphere Erkenntnisse
\parencite[vgl.][]{Quine:TwoDogmasofEmpiricism1951}. Siehe dagegen:
\cite{Roedl:NormundNatur2003}.}

Die Kritikpunkte setzen jedoch ein Autonomieverständnis voraus, das nicht
dasjenige \name[Immanuel]{Kant}s ist, insofern Autonomie und Vernunft als
getrennt betrachtet
werden.\footnote{\cite[Vgl.][75--77]{ONeill:ConstructionsofReason1989}.} Eine
eindeutige Folge ist, dass wir uns von einem legislatorischen
Autonomieverständnis distanzieren
müssen.\footnote{\authorfullcite{Menke:AutonomieundBefreiung2010} sieht den
Bruch mit diesem Autonomieverständnis in \name[Immanuel]{Kant}s Redeweise von
der \singlequote{\emph{eigenen} Gesetzgebung} (statt der
\singlequote{\emph{Selbst}gesetzgebung}) darüber hinaus auch sprachlich
vollzogen \parencite[vgl.][677]{Menke:AutonomieundBefreiung2010}.} Wenn
Autonomie heißt, dass wir nur an solche Gesetze gebunden sind, die wir uns
selbst \emph{gegeben} haben, dann treffen die genannten Einwände zu. Stattdessen
müssen wir Autonomie in der Richtung interpretieren, die
\authorfullcite{Korsgaard:TheSourcesofNormativity1996} vorgegeben hat: Es geht
nicht um Gesetze, die wir uns nach Belieben \emph{geben}, sondern um
die Gesetze, in denen wir unsere Identität
ausdrücken\footnote{\enquote{[T]he principle or law by which you determine your
actions is one that you regard as being expressive of \ori{yourself}. To
identify with such a principle or way of choosing is to be, in St Paul's famous
phrase, a law to yourself}
\parencite[][100]{Korsgaard:TheSourcesofNormativity1996}.} -- oder, um mit
\authorfullcite{Menke:AutonomieundBefreiung2010} zu sprechen: um ein Gesetz,
welches ein Subjekt \emph{ist}.\footnote{\enquote{Das Gesetz, in dessen Gehorsam
ich frei bin, ist nicht das Gesetz, das ich mir selbst gegeben habe, sondern das
ich mir selbst \ori{bin}} \parencite[][678]{Menke:AutonomieundBefreiung2010}.}
Der Autonomiebegriff \name[Immanuel]{Kant}s ist nicht \emph{legislatorisch}.\footnote{Siehe
\cite[][677--679]{Menke:AutonomieundBefreiung2010}, sowie die dort zitierte
Literatur.} Ob er \emph{expressivistisch} ist, mag hier dahingestellt sein. Aus
\name[Immanuel]{Kant}s Auskünften lässt sich eines leicht entnehmen: Unsere
Vernunft ist nicht autonom, weil wir die obersten Prinzipien selbst geben,
sondern weil wir sie \emph{a priori} erkennen. Die Prinzipien sind metaphysische
Erkenntnisse, die objektiv rational sind, von denen wir subjektiv aber
rationale oder historische Erkenntnis haben können.
\end{comment}
%
%
%
\item\label{AutonomiedesVerstandes} Der Verstand schreibt der Natur sein Gesetz
vor. Durch diesen Gedanken einer Autonomie des theoretischen Erkennens erläutert
\name[Immanuel]{Kant} die Möglichkeit einer Metaphysik der Natur: Der Ausgangspunkt der Verwirrung, mit
der die \titel{Kritik der reinen Vernunft} sich auseinandersetzt, lautet: Wie kann es
sein, dass wir durch bloßes Denken, ohne auf Erfahrung zurückzugreifen, etwas
darüber erkennen können, wie die Dinge, die unabhängig von uns existieren, sind?
In anderen Worten lässt sich fragen:
Wie ist Metaphysik (der Natur) möglich? Nicht verwunderlich ist es nach
\name[Immanuel]{Kant}, dass wir mit bloßer Vernunft etwas über unsere Vernunft
selbst ausmachen können. Deswegen kommt die Verwunderung, die ihn bei der Kritik
der reinen \emph{spekulativen} Vernunft antreibt, in \emph{dieser} Form nicht im
Kontext einer Kritik der reinen \emph{praktischen} Vernunft vor. Es wäre
naheliegend davon auszugehen, dass Erkenntnisse darüber, wie die Dinge in der Welt
unabhängig von uns sind, nur durch die Aktualisierung \emph{rezeptiver}
Fähigkeiten zu erlangen sind. In der zeitlichen Reihenfolge ist nach
\name[Immanuel]{Kant} auch klar, dass wir unser rezeptives Erkenntnisvermögen
aktualisieren müssen, ehe wir Wissen über die Welt erlangen können. Von diesem
Gedanken geht er in der \titel{Kritik der reinen Vernunft} aus, ohne ihn eigens
zu begründen. Dennoch dürfe man davon, dass der menschliche Verstand gleichsam
als \emph{tabula rasa} beginne, nicht darauf schließen, dass unsere
Erfahrungserkenntnis rein rezeptiv gewonnen werden könne. Vielmehr beinhalte sie
rationale Erkenntnis:
\begin{quote}
 Wenn aber gleich alle unsere Erkenntnis mit der Erfahrung anhebt, so entspringt
sie darum doch nicht eben alle aus der Erfahrung. Denn es könnte wohl sein, daß
selbst unsere Erfahrungserkenntnis ein Zusammengesetztes aus dem sei, was wir
durch Eindrücke empfangen, und dem, was unser eigenes Erkenntnisvermögen (durch
sinnliche Eindrücke bloß veranlaßt,) aus sich selbst hergibt, welchen Zusatz wir
von jenem Grundstoffe nicht eher unterscheiden, als bis lange Übung uns darauf
aufmerksam und zur Absonderung desselben geschickt gemacht
hat.\footnote{\cite[][B 1\,f.,]{Kant:KritikderreinenVernunft2003} \cite[][III:
27.14--21]{Kant:GesammelteWerke1900ff.}.}
\end{quote}
Sollte Metaphysik (der Natur) jedoch tatsächlich
möglich sein, dann muss einsichtig gemacht werden, wie ein Vermögen der
\emph{Spontaneität} die Welt so soll abbilden können, wie sie \emph{de facto
ist} (und nicht wie sie sein \emph{sollte}).

Die scheinbare Aporie, die mit der Frage nach der Möglichkeit metaphysischer
Naturerkenntnis einhergeht, hat die Form eines Dilemmas:
Unser Denken soll wahre Sätze generieren, die sich auf Gegenstände beziehen,
diese bestimmen und damit Sinn und Bedeutung
haben.\footnote{\cite[Vgl.][B 299]{Kant:KritikderreinenVernunft2003},
\cite[][III: 204.37--205.3]{Kant:GesammelteWerke1900ff.}. Bedeutung ist
\enquote{Beziehung aufs Objekt} \mkbibparens{\cite[][B 300]{Kant:KritikderreinenVernunft2003},
\cite[][III: 205.20--21]{Kant:GesammelteWerke1900ff.}}.} Nur wenn es eine
Beziehung zwischen unseren Vorstellungen und Dingen in der Welt gibt, verfügen
wir über Erkenntnisse.\footnote{Dies geht aus dem Begriff der Erkenntnis hervor.
Erkenntnis in einem \singlequote{eigentlichen} Sinn von objektiv gültigen
Urteilen (siehe dazu Anmerkung \ref{Anmerkung:ErkenntnisInZweierleiSinn} auf
Seite \pageref{Anmerkung:ErkenntnisInZweierleiSinn}) besteht in der
\enquote{bestimmten Beziehung gegebener Vorstellungen auf ein Objekt}
\mkbibparens{\cite[][B 137]{Kant:KritikderreinenVernunft2003},
\cite[][III: 111.17--18]{Kant:GesammelteWerke1900ff.}}. Erkenntnis im
\singlequote{uneigentlichen} Sinn von Anschauungen und Begriffen sind
\emph{objektive} Perzeptionen und damit ebenso auf Gegenstände bezogen
\mkbibparens{\cite[siehe][B 376]{Kant:KritikderreinenVernunft2003},
\cite[][III: 250.3]{Kant:GesammelteWerke1900ff.}}.} und dies scheint nur auf
zwei Arten möglich zu sein: Entweder macht der Gegenstand die Vorstellung
möglich oder die Vorstellung den Gegenstand. Der erste Fall liegt etwa dort vor,
wo wir sinnlich wahrnehmen, dass etwas der Fall ist. Sehe ich, dass es regnet,
dann gibt es eine klare Verbindung zwischen dem Gegenstand und meiner
Vorstellung, die erklärt, warum meine Vorstellung gerade eine Vorstellung dieses
Regens ist. Aber diese Verbindung kommt über die Sinnlichkeit zustande und ist
daher nicht metaphysisch, sondern empirisch, kann uns also die Möglichkeit des
Gegenstandsbezugs reiner Verstandeserkenntnisse nicht erhellen.
Der andere Fall findet sich im Handeln: Wenn ich einen Gegenstand nach einem
vorgefertigten Plan anfertige, dann besteht eine einsichtige Beziehung zwischen
diesem Plan und dem Gegenstand und es ist verständlich, warum dieser Plan eben
ein Plan dieses Gegenstandes ist. Nun entstammt der Plan tatsächlich meiner
Spontaneität und nicht der Sinnlichkeit, aber er beschreibt nicht den
Gegenstand, sondern schreibt vor, wie dieser sein soll. Daher eignet sich diese
Möglichkeit ebenfalls nicht dazu einsichtig zu machen, wie metaphysische
Erkenntnisse aus reiner (theoretischer) Vernunft entspringen und dennoch
Gegenstandsbezug haben können.

Der Grund hierfür liegt in der Endlichkeit des Verstandes, wie sie in Kapitel
\ref{chapter:endlichkeitmenschlichendenkens} beschrieben wurde: Unser Verstand
ist diskursiv, insofern er sich nur mittels weiterer Vorstellungen auf
Gegenstände zu beziehen vermag. Ein direkter Bezug auf Gegenstände, der nicht
über Merkmale vermittelt ist, nennt \name[Immanuel]{Kant} Anschauung. Eine
solche Anschauung ist nun entweder sinnlich oder intellektuell.
Und eine intellektuelle Anschauung können wir uns kaum anders als produktiv
vorstellen: sie bringt ihre Gegenstände selbst hervor.\footnote{Siehe dazu Kap.
\ref{subsubsection:UnterscheidungvonDenkenundErkennen}.} Produktiv kann der
Bezug unseres Verstandes auf seinen Gegenstand aber nur als praktische Vernunft
sein. Deswegen scheitern also beide Ansätze, wenn es darum geht, die Möglichkeit
metaphysischer Naturerkenntnisse zu
erklären.\footnote{\cite[Vgl.][\S~14]{Kant:KritikderreinenVernunft2003},
\cite[][III: 104.6--17]{Kant:GesammelteWerke1900ff.}. Siehe zu diesem Dilemma
auch den Brief an \name[Marcus]{Herz} vom 21. Februar 1772 \parencite[][X:
124\,f.]{Kant:GesammelteWerke1900ff.}.}


Nun lasse sich die Aporie auflösen, wenn gezeigt werden kann, dass die reinen
Verstandesbegriffe und Grundsätze des reinen Verstandes die Existenz der
Gegenstände zwar nicht bewirken, dass sie allein es aber möglich machen, etwas
\emph{als einen Gegenstand} zu erkennen. An dieser Stelle greift die Überlegung
zur Änderung der Denkart in der Vorrede zur zweiten Auflage der \titel{Kritik
der reinen Vernunft}.\footnote{Die Literatur zu dieser Überlegung und der
Analogie, mit der \name[Immanuel]{Kant} sie beschreibt, ist schier
unüberschaubar. Einen Überblick geben
\textcite[vgl.][]{Schoenecker:Kantskopernikanisch-newtonischeAnalogie2011}.}
\name[Immanuel]{Kant} arbeitet sie in folgender Form heraus:
\begin{quote}
Die Vernunft muß mit ihren Prinzipien, nach denen allein übereinkommende
Erscheinungen für Gesetze gelten können, in einer Hand, und mit dem Experiment,
das sie nach jenen ausdachte, in der anderen, an die Natur gehen, zwar um von
ihr belehrt zu werden, aber nicht in der Qualität eines Schülers, der sich alles
vorsagen läßt, was der Lehrer will, sondern eines bestallten Richters, der die
Zeugen nötigt auf die Fragen zu antworten, die der er ihnen
vorlegt.\footnote{\cite[][B xiii]{Kant:KritikderreinenVernunft2003},
\cite[][III: 10.21--28]{Kant:GesammelteWerke1900ff.}.}
\end{quote}
\name[Immanuel]{Kant} beschreibt den Erwerb theoretischen Wissens explizit in
Analogie zum mündigen Erwerb testimonialen Wissens und bringt damit zum
Ausdruck, dass die Art der Mündigkeit in beiden Fällen dieselbe ist. Die
Mündigkeit liegt in der Selbsttätigkeit (und damit in der \emph{Spontaneität})
des erkennenden Subjekts begründet und diese liegt im Rahmen der wissenschaftlichen
Naturerkenntnis, die \name[Immanuel]{Kant} hier beschreibt, in zweierlei Form
vor:
\begin{enumerate}
\item[a)] Die Vernunft hat eigene \enquote{Prinzipien, nach denen allein
übereinkommende Erscheinungen für Gesetze gelten können}. Diese Prinzipien sagen
etwa, dass jedes Ereignis, das geschieht, nach einer Regel auf ein
vorhergehendes Ereignis folgt, von dem es verursacht wurde (zweite Analogie der
Erfahrung)\footnote{\enquote{Alle Veränderungen geschehen nach dem Gesetze der
Verknüpfung der Ursache und Wirkung}
\mkbibparens{\cite[][B 232]{Kant:KritikderreinenVernunft2003},
\cite[][III: 166.32--33]{Kant:GesammelteWerke1900ff.}, \ohio}.}, oder das die
Substanz in einer Veränderung bestehen bleibt (erste Analogie der
Erfahrung)\footnote{\enquote{Bei allem Wechsel der Erscheinungen beharret die
Substanz, und das Quantum derselben wird in der Natur weder vermehrt noch
vermindert} \mkbibparens{\cite[][B 224]{Kant:KritikderreinenVernunft2003},
\cite[][III: 162.4--6]{Kant:GesammelteWerke1900ff.}, \ohio}.}. Und diese
Prinzipien sind selbst der Erfahrung nicht entlehnt, sondern stellen
metaphysische Erkenntnisse dar, die in jeder Erfahrung vorausgesetzt werden.
Denn nur wenn gilt, dass alles Geschehen nach Kausalgesetzen geschieht, lassen
sich überhaupt objektive Ereignisse denken.

In dieser Überlegung besteht die Auflösung des genannten Dilemmas in der
Analytik des Verstandes der \titel{Kritik der reinen Vernunft}. Vorstellungen
seien \enquote{in Ansehung des Gegenstandes alsdenn a priori bestimmend, wenn
durch sie allein es möglich ist, etwas \ori{als einen Gegenstand zu
erkennen}.}\footnote{\cite[][\S~14]{Kant:KritikderreinenVernunft2003},
\cite[][III: 104.15--17]{Kant:GesammelteWerke1900ff.}.} Die reinen
Verstandesbegriff sind \enquote{Begriffe von einem Gegenstande überhaupt,
dadurch dessen Anschauung in Ansehung einer der \ori{logischen Funktionen} zu
Urteilen als bestimmt angesehen
wird.}\footnote{\cite[][\S~14]{Kant:KritikderreinenVernunft2003},
\cite[][III: 106.17--19]{Kant:GesammelteWerke1900ff.}.}
\item[b)] Der zweite Punkt fokussiert die Vorgehensweise neuzeitlicher
Naturwissenschaft, die sich nicht nur auf schlichte Beobachtungen stützt,
sondern gezielt \emph{Experimente} einsetzt. Dass die Fallbeschleunigung für alle Gegenstände,
unabhängig von ihrer Masse, dieselbe ist, das lässt sich nicht einfach
beobachten. In unserer alltäglich Wahrnehmung fallen Steine und Federn durchaus
unterschiedlich schnell. Erst durch gezielte Versuche lässt sich erweisen, dass
die Unterschiede durch Faktoren wie den Luftwiderstand verursacht werden.

Dieser Punkt betrifft nicht speziell die Metaphysik, sondern die empirischen
Wissenschaften, die neben der metaphysischen Grundlage weitere Aktivitäten
unseres Denkvermögens erfordern, die sich nicht in synthetischen Urteilen
\emph{a priori} äußern. \name[Immanuel]{Kant} zieht diese Überlegungen auch
nicht heran, um die Möglichkeit der Metaphysik zu begründen, sondern um sich dem
anderen Punkt unter Verweis auf Analogien zu etablierten Wissenschaften
anzunähern.
\end{enumerate}

John \authorcite{McDowell:DeReSenses1984} betont, dass Vernunfterkenntnisse
auch in allen Erkenntnissen des Verstandes involviert sind und wir deswegen auch
im Falle der Aktualisierung rezeptiver Erkenntnisvermögen Freiheit
ausüben.\footnote{\enquote{The idea of a faculty of spontaneity is the idea of
something that empowers us to take charge of our lives. Kant points the way to a
position in which we can satisfyingly apply that idea to empirical thinking: we
can hold that empirical inquiry is a region of our lives in which we exercise a
responsible freedom, and not let that thought threaten to dislodge our grip on
the requirement that empirical thinking be under constraint from the world
itself} \parencite[][43]{McDowell:MindandWorld1994}.}
Gerade weil unsere empirische Erkenntnis nicht Ergebnis bloßer Rezeptivität,
sondern des Zusammenwirkens der beiden Stämme unseres Erkenntnisvermögens --
Sinnlichkeit (Rezeptivität) und Verstand (Spon\-ta\-ne\-i\-tät) -- ist, kann
\name[Immanuel]{Kant} sagen, wir müssten \emph{immer} aktiv und autonom sein.
Ansonsten müsste er sagen, wir sollten beim Nachdenken über das Wahrgenommene
oder Mitgeteilte mündig und niemals passiv sein.\footnote{Auf dieses Moment
weist auch \authorcite{Hegel:GesammelteWerke} hin, wenn er gegen \name[Friedrich
Heinrich]{Jacobi}s \name[Immanuel]{Kant}-Auslegung schreibt:
\enquote{Wichtiger aber ist es, bey dieser Behandlung der \name[Immanuel]{Kant}ischen Kritik der Vernunft nicht zu
übersehen, daß das unendliche Verdienst derselben nicht bemerklich gemacht ist,
die \ori{Freyheit des Geistes} auch in der \ori{theoretischen} Seite als Princip
erkannt zu haben. Dies Princip, freylich in einer abstracten Form, liegt in der
Idee einer ursprünglich-synthetischen Apperception des Selbstbewußtseyns,
welches auch im Erkennen wesentlich
\ori{selbstbestimmend} seyn will}
\mkbibparens{\cite[][XV: 16.14--19]{Hegel:GesammelteWerke}}.}

\name[Immanuel]{Kant}s Konzeption von Autonomie und Mündigkeit auf der Grundlage einer
Unterscheidung von historischer und rationaler Erkenntnis setzt gerade nicht
voraus, historische Erkenntnisse beruhten ausschließlich auf unserer
Sinnlichkeit, ohne eines Beitrags des Verstandes zu bedürfen. Es gibt keine
Erfahrungs- oder auch nur Wahrnehmungserkenntnis, die nicht durch metaphysische
Urteile und Begriffe fundiert wäre. Es gibt keine Tatsachenurteile, die ohne Vernunfterkenntnisse und
ohne über reine Verstandesbegriffe zu verfügen möglich wären. Die Unterscheidung
rationaler und empirischer Erkenntnisse kann nicht dadurch getroffen werden,
dass man sagt, das, was ausschließlich Wirkung des Verstandes ist, heiße
rationale Erkenntnis, dasjenige hingegen, was ausschließlich Wirkung der
Sinnlichkeit ist, heiße historische Erkenntnis. Es gäbe dann keine historische
Erkenntnis und vermutlich auch keine rationale Erkenntnis; denn auch die reinen
Verstandeserkenntnisse -- Begriffe wie Grundsätze -- und sogar die Mathematik
als Paradigma einer erfolgreichen Gewinnung synthetischer Erkenntnisse \emph{a
priori} setzen in der Darstellung der \titel{Kritik der reinen Vernunft} voraus,
dass wir über Sinnlichkeit verfügen, die ihnen erst Gehalt und Gegenstandsbezug
geben.\footnote{Zur Abhängigkeit des Status von Begriffen -- des reinen
Verstandes wie auch der Mathematik -- als Erkenntnissen von der Erfahrung siehe
\cite[][B 147]{Kant:KritikderreinenVernunft2003},
\cite[][III: 117.17--26]{Kant:GesammelteWerke1900ff.}. Siehe außerdem \cite[][B
196]{Kant:KritikderreinenVernunft2003}, \cite[][III:
145.11--20]{Kant:GesammelteWerke1900ff.}.} Und ohne durch unsere Sinne angeregt
zu werden, vermag unser Verstand auch seine reinen Erkenntnisse nicht zu
generieren, weswegen in zeitlicher Hinsicht unsere Erkenntnis zweifellos mit der
Erfahrung anfange.\footnote{\cite[Vgl.][B 1]{Kant:KritikderreinenVernunft2003},
\cite[][III: 27.5--13]{Kant:GesammelteWerke1900ff.}.}


\item\phantomsection\label{AutonomiederUrteilskraft} Die \textit{Urteilskraft}
gibt nicht der Natur oder der Freiheit, sondern sich selbst ein Gesetz, weswegen ihr genau
genommen nicht Autonomie, sondern \emph{Heautonomie}
zukommt.\footnote{\cite[Vgl.][B~xxxvii]{Kant:KritikderUrteilskraft2009},
\cite[][V: 185.35-186.1]{Kant:GesammelteWerke1900ff.}. Genau genommen ist sie
als \emph{reflektierende} Urteilskraft autonom, nicht aber als bestimmende
Urteilskraft, als welche sie sich ausschließlich nach den Gesetzen des
Verstandes richtet
\mkbibparens{\cite[Vgl.][\S~71]{Kant:KritikderUrteilskraft2009}, \cite[][V:
389.20-17]{Kant:GesammelteWerke1900ff.}}.}
Es ist zunächst nicht offensichtlich, dass die Urteilskraft überhaupt \emph{a
priori} gesetzgebend ist -- wobei die Tatsache, dass sie zu den oberen
Erkenntnisvermögen zählt, dies vermuten lässt. Urteilskraft ist ihrem Begriff
nach \enquote{das Vermögen, das Besondere als enthalten unter dem Allgemeinen zu
denken.}\footnote{\cite[][B xxv]{Kant:KritikderUrteilskraft2009},
\cite[][V: 179.19--20]{Kant:GesammelteWerke1900ff.}.} Wer eine Anschauung unter
einen Begriff bringt, etwa indem er ein Rotkehlchen als solches erkennt, oder
einen Anwendungsfall einer Regel als solche identifiziert, gebraucht seine
Urteilskraft. Über eine gute (reflektierende) Urteilskraft zu verfügen, nennt
\name[Immanuel]{Kant} \enquote{Witz}\footnote{In der \titel{Anthropologie in
pragmatischer Hinsicht} identifiziert \name[Immanuel]{Kant} den Witz mit der
\emph{reflektierenden} Urteilskraft, die Urteilskraft selbst mit dem, was er in
der \titel{Kritik der Urteilskraft} als \emph{bestimmende} Urteilskraft
bezeichnet: \enquote{So wie das Vermögen, zum Allgemeinen (der Regel) das
Besondere auszufinden, \ori{Urteilskraft}, so ist dasjenige: zum Besonderen das
Allgemeine auszudenken, der \ori{Witz} (ingenium)} \mkbibparens{\cite[vgl.][BA
123]{Kant:AnthropologieinpragmatischerHinsicht1977}, \cite[][VII:
201.12--14]{Kant:GesammelteWerke1900ff.}}.} und die ursprüngliche
Anlage oder das Talent zu einer guten Urteilskraft
\enquote{Mutterwitz}.\footnote{Siehe z.\,B. \cite[][B
172]{Kant:KritikderreinenVernunft2003}, \cite[][III:
131.29]{Kant:GesammelteWerke1900ff.}.} Urteilskraft muss im Zusammenspiel mit
anderen -- in der gegenseitigen Kontrolle der korrekten Anwendung von Begriffen
und Regeln -- geschult werden, sie ist das notwendige Gegenstück zur
theoretischen Ausbildung durch Vermittlung von Kenntnissen.\footnote{\cite[Vgl.][B
172]{Kant:KritikderreinenVernunft2003}, \cite[][III:
131.21--132.2]{Kant:GesammelteWerke1900ff.}.} Deswegen findet sich hier auch der
Kontrast zu einem Verständnis von Aufklärung, welches diese in
\singlequote{Kenntnisse} setzt, wie in Kapitel
\ref{section:sensuscommunis} hervorgehoben wurde. Wer über viele
Kenntnisse (viel Theorie) verfügt, kann sich dennoch als unfähig zur korrekten
Anwendung dieser Theorie erweisen, insofern er nicht beurteilen kann, wann
welche Regel zu wählen und wie sie in konkreten Situationen umzusetzen ist.
Der Arzt, der aus seinem Studium die richtige Therapie zu jeder hinreichend
erforschten Krankheit zu nennen weiß, aber in der praktischen Arbeit die
Krankheiten am Patienten nicht zu bestimmen und die Therapie nicht einzustellen
vermag, wäre ein Beispiel eines Gelehrten ohne
Urteilskraft.\footnote{\cite[Vgl.][B
172\,f.,]{Kant:KritikderreinenVernunft2003}
\cite[][III: 132.2--10]{Kant:GesammelteWerke1900ff.}.} Urteilskraft zu besitzen
heißt, über Fähigkeiten oder ein Können, nicht über Wissen zu verfügen; die
englische Sprache stellt hierfür das Begriffspaar \emph{know how} und
\emph{know that} bereit; und die Einsicht, die sich bei \name[Immanuel]{Kant}
bereits findet, ist die, dass jedes Wissen in einem Können (einem \emph{know
how}) fundiert ist, das sich in jeder korrekten Verwendung von Begriffen
zeigt.\footnote{Der \emph{locus classicus} zu solchen Überlegungen ist heute
zweifelsohne \cite[][Kap.~2]{Ryle:TheConceptofMind2002}.}

In der Regel dient Urteilskraft der Anwendung einer vorhandenen Regel oder
eines vorhandenen Begriffs, die oder der ihr anderweitig gegeben ist.
\name[Immanuel]{Kant} spricht von der \emph{bestimmenden} Urteilskraft.\footnote{\enquote{Ist das Allgemeine (die
Regel, das Prinzip, das Gesetz) gegeben, so ist die Urteilskraft, welche das
Besondere darunter subsumiert, (auch, wenn sie als transzendentale
Urteilskraft a priori die Bedingungen angibt, welchen gemäß allein unter
jenem Allgemeinen subsumiert werden kann) \ori{bestimmend}}
\mkbibparens{\cite[][B xxvi]{Kant:KritikderUrteilskraft2009},
\cite[][V: 179.20--24]{Kant:GesammelteWerke1900ff.}}.} Sie ist daher nicht
selbstbestimmt, sondern heteronom -- auch dort, wo sie im
Schematismuskapitel der \titel{Kritik der reinen Vernunft} als
\emph{transzendentale} Urteilskraft die Bedingungen der Anwendung reiner
Verstandesbegriffe auf die reine Anschauung (die Grundsätze des reinen
Verstandes) angibt. Die Regeln, an denen sie sich dabei orientiert, gibt sie
sich nicht selbst, sondern diese werden ihr von dem Verstand vorgegeben. Noch in
der \titel{Kritik der reinen Vernunft} ist daher von einer Gesetzgebung der
Urteilskraft keine Rede.


Die \emph{bestimmende} Urteilskraft kann kein eigenes Prinzip und keine eigene
Gesetzgebung haben, weil sie \emph{per definitionem} nur der Anwendung der
Prinzipien und Begriff anderer Erkenntnisvermögen dient. Aber spätestens in der
\titel{Kritik der Urteilskraft} kennt \name[Immanuel]{Kant} auch eine der
bestimmenden Funktion gegenüber gestellte \emph{reflektierende} Funktion der
Urteilskraft. Die Urteilskraft sei
\enquote{bloß \ori{reflektierend}}, wenn \enquote{nur das Besondere gegeben
[ist], wozu sie das Allgemeine finden
soll}\footnote{\cite[][B xxvi]{Kant:KritikderUrteilskraft2009},
\cite[][V: 179.24--25]{Kant:GesammelteWerke1900ff.}. Umstritten ist, inwieweit
\name[Immanuel]{Kant} die Urteilskraft bereits in der \titel{Kritik der reinen
Vernunft} sowohl in ihrer bestimmenden als auch in ihrer reflektierenden
Funktion in den Blick nimmt. \emph{Prima facie} ließe sich vermuten, dass die
erste Kritik die Urteilskraft ausschließlich als bestimmend auffasst. Dagegen
argumentiert jedoch \authorfullcite{Longuenesse:KantandtheCapacitytoJudge1998},
dass \name[Immanuel]{Kant} bereits in der Vernunftkritik beide Funktionen kennt
und dass das eigentlich Neue innerhalb der Darstellung der \titel{Kritik der
Urteilskraft} nicht die Beschreibung einer reflektierenden Urteilskraft
überhaupt sei, sondern die Berücksichtigung von Fällen, in denen sie
\emph{ausschließlich} reflektierend ist. Siehe dazu
\cite[][163--166]{Longuenesse:KantandtheCapacitytoJudge1998}, sowie
\cite[][16]{Allison:KantsTheoryofTaste2001}.}. Ein solcher Fall liegt also
beispielsweise dann vor, wenn wir eine Reihe von Gegenständen in unserer
Anschauung haben, die wir noch nicht begrifflich klassifiziert haben und die es
nun zu benennen gilt. Ein anderes Beispiel ist für die weiteren Überlegungen
wichtig: Nehmen wir an, wir verfügen über eine Menge von empirischen Gesetzen.
Aufgabe der reflektierenden Urteilskraft ist es nun, ein System von
Naturgesetzen zu finden, welches diese Menge an empirischen Gesetzen
vereinheitlicht. Als \name[Isaac]{Newton} die Gesetze des Himmels und der Erde
in der Mechanik vereinheitlichte und zu Anwendungsfällen desselben
Gravitationsgesetzes macht, war dies eine Leistung seiner reflektierenden
Urteilskraft. Es gab zuvor keine Regeln und Begriffe, unter die er die
Erscheinungen bloß zu subsumieren gehabt hätte. Das Gravitationsgesetz und die
anderen der von ihm entdeckten Regeln waren nicht bereits vorhanden, sondern
mussten in gewisser Hinsicht tatsächlich \singlequote{erfunden} werden.
Natürlich sind die Phänomene, die eine systematische Darstellung der Natur
beschreibt, unabhängig von dieser vorhanden und insofern die Bezeichnung
\enquote{erfinden} für das Finden von allgemeinen Naturgesetzen unpassend. Aber die Darstellung eines
auf Einfachheit hin angelegten Systems von Naturgesetzen ist doch das Produkt
unserer geistigen Tätigkeit.

Während die bestimmende Urteilskraft sich auf ihr anderweitig gegebene Begriffe,
Regeln und Grundsätze stützt und insofern keiner eigenen Prinzipien bedarf, lässt die
Beschreibung der reflektierenden Urteilskraft Raum für eine eigene Gesetzgebung.
In der dritten Kritik behauptet \name[Immanuel]{Kant} entsprechend, die
Urteilskraft verfüge als reflektierende Urteilskraft über ein eigenes
\emph{transzendentales Prinzip}, \enquote{durch welches die allgemeine
Bedingung a priori vorgestellt wird, unter der allein Dinge Objekte unserer
Erkenntnis überhaupt werden können.}\footnote{\cite[][B
xxix]{Kant:KritikderUrteilskraft2009}, \cite[][V:
181.15--17]{Kant:GesammelteWerke1900ff.}.} Kontrastiert werden die
transzendentalen den \emph{metaphysischen} Prinzipien, die ebenfalls
Erkenntnisse \emph{a priori} darstellen, dabei aber ihre Objekte, deren Begriffe
wir aus empirischen Quellen haben, lediglich \emph{a priori} weiter
bestimmen. In der \titel{Kritik der reinen Vernunft} heißen solche Erkenntnisse
\emph{rein}, die nicht nur \emph{a priori} sind, sondern die auch keine
empirisch gewonnenen Begriffe voraussetzen. Zu sagen, jede Veränderung habe eine
Ursache, artikuliert demnach ein metaphysisches Prinzip, weil der Begriff der
Veränderung nach \name[Immanuel]{Kant} nur auf der Grundlage von Erfahrung
gebildet werden kann.\footnote{\enquote{Von den Erkenntnissen a priori heißen
aber diejenigen rein, denen gar nichts Empirisches beigemischt ist. So ist z.\,B.
der Satz: eine jede Veränderung hat ihre Ursache, ein Satz a priori, allein
nicht rein, weil Veränderung ein Begriff ist, der nur aus der Erfahrung gezogen
werden kann} \mkbibparens{\cite[][B 3]{Kant:KritikderreinenVernunft2003},
\cite[][III: 28.23--27]{Kant:GesammelteWerke1900ff.}}.} Dagegen sei die
Behauptung, alle Körper seien als Substanzen zu erkennen, transzendental, insofern die Körper darin
lediglich \enquote{durch ontologische Prädikate (reine Verstandesbegriffe)} gedacht
werden.\footnote{\cite[Vgl.][B xxix]{Kant:KritikderUrteilskraft2009},
\cite[][V: 181.20--27]{Kant:GesammelteWerke1900ff.}. \name[Immanuel]{Kant}
scheint in der \titel{Kritik der Urteilskraft} von der Darstellung der
\titel{Kritik der reinen Vernunft} abzuweichen, insofern nach letzterer der
Begriff der Veränderung ein empirischer Begriff sein soll, während die erstere
den Begriff der veränderlichen Substanz als Begriff \emph{a priori} behandelt.}

Um transzendental zu sein, muss ein Prinzip also drei Bedingungen erfüllen: Es
muss, erstens, \emph{a priori} erkannt werden; es darf, zweitens, keine
empirischen Begriffe enthalten (es muss also \singlequote{rein} sein); und es
muss, drittens, die notwendige Bedingung dafür explizieren, dass Dinge
überhaupt Gegenstände unserer Erkenntnis werden können. Die Grundsätze des
reinen Verstandes sind solche transzendentalen Prinzipien, denn sie werden
\emph{a priori} erkannt, enthalten lediglich reine Verstandesbegriffe und
explizieren Bedingungen für objektive Erkenntnisse. Aber auch die Urteilskraft
soll ein solches Prinzip enthalten, welches vorausgesetzt werden müsse, um
überhaupt Erkenntnisse von Dingen haben zu können. Dieses transzendentale
Prinzip der reflektierenden Urteilskraft besagt, dass
\begin{quote}
 die besonderen empirischen Gesetze in Ansehung dessen, was in ihnen durch jene
[allgemeinen Gesetze des Verstandes; A.\,G.] unbestimmt gelassen ist, nach einer
solchen Einheit betrachtet werden müssen, als ob gleichfalls ein Verstand (wenn
gleich nicht der unsrige) sie zum Behuf unserer Erkenntnisvermögen, um ein
System der Erfahrung nach besonderen Naturgesetzen möglich zu machen, gegeben
hätte.\footnote{\cite[][B~xxvii]{Kant:KritikderUrteilskraft2009}, \cite[][V:
180.21-26]{Kant:GesammelteWerke1900ff.}.}
\end{quote}
Inhaltlich konkretisiert wird dieses Prinzip durch drei Maximen, die ähnlich
bereits im Anhang zur Transzendentalen Dialektik der \titel{Kritik der reinen
Vernunft} vorkommen, dort aber nicht der Urteilskraft, sondern der Vernunft zugeschrieben
werden. Dort ist von Prinzipien der Kontinuität, der Spezifikation und der
Homogenität die Rede.\footnote{\cite[Siehe][B
686]{Kant:KritikderreinenVernunft2003}, \cite[][III:
435.33--35]{Kant:GesammelteWerke1900ff.}.} In der Einleitung in die
\titel{Kritik der Urteilskraft} werden diese nun zu Maximen der
(reflektierenden) Urteilskraft.

Maximen der Urteilskraft sind subjektive Grundsätze
derselben.\footnote{\cite[Vgl.][B xxxiv]{Kant:KritikderUrteilskraft2009},
\cite[][V: 184.10--16]{Kant:GesammelteWerke1900ff.}.} Deshalb ist auch die
Autonomie der Urteilskraft eine subjektive oder \emph{He}autonomie. Die
Autonomie von Vernunft und Verstand ist hingegen \emph{objektiv}, insofern sie
ihre Gesetze a priori \enquote{durch Begriffe von Dingen oder möglichen
Handlungen} geben. Wenn wir einen Gegenstand unter einen Begriff gebracht haben,
etwa eine Handlung unter den Begriff der Lüge, dann können wir auf dieser
Grundlage weitere Aussagen über ihn machen, etwa von der Handlung, die wir als
Lüge klassifizierten, sagen, dass sie der Moral zuwider ist. Aber Urteile der
reflektierenden Urteilskraft hängen nicht davon ab, unter welchen Begriff wir
einen Gegenstand bringen. Wir können beispielsweise nicht sagen, dass ein
Gegenstand schön sei, weil es sich um ein Gemälde oder eine Skulptur (oder ein
Gemälde von diesem oder jenem) handelt. Auch ist ein Gegenstand (oder die
Abbildung eines Gegenstandes) nicht darum schön, weil er in seiner Art -- seinem
Begriff nach -- vollkommen ist. Es heißt aber weiter, dass Gesetzgeber und
Gesetzesadressat unterschieden werden. Vernunft und Verstand geben sich nicht
selbst das Gesetz, sondern der Verstand der Natur und die Vernunft der Freiheit.

Aber auch wenn diese Maximen
\emph{subjektive} Prinzipien sind, sind sie doch mehr als nur ein Ausdruck
unserer Gewohnheiten. Sie beschreiben nicht, wie wir \emph{de facto} denken,
sondern wie wir denken \emph{sollen}. Gerade daran erkenne man, dass es sich bei
der formalen Zweckmäßigkeit der Natur (welche ein Prinzip beschreibt, in welchem
die genannten drei Maximen zusammengefasst sind) um ein \emph{transzendentales}
Prinzip der Urteilskraft handelt.\footnote{\cite[Vgl.][B
xxxi]{Kant:KritikderUrteilskraft2009}, \cite[][V:
182.26--36]{Kant:GesammelteWerke1900ff.}.} Sie sind weder durch äußere Erfahrung
noch durch (empirisch-) psychologische Begebenheiten (aus innerer Erfahrung) zu
fundieren, sondern durch eine rationale (philosophische) Begründung. Die Maximen
können wir begründet anwenden, wenn wir vorauszusetzen Grund haben, dass die
Natur hinsichtlich ihrer Erkennbarkeit durch unser Erkenntnisvermögen zweckmäßig
eingerichtet ist. Dieses Prinzip drückt die formale Zweckmäßigkeit der Natur aus, welche nicht der
Natur selbst oktroyiert werden kann (denn diese ist auch ohne eine solche
Zweckmäßigkeit möglich), aber unserer Urteilskraft, welche eines solchen
Prinzips bedarf, um allgemeine Begriffe und Naturgesetze zur Beschreibung der
mannigfachen Phänomene zu finden. Die Rechtmäßigkeit einer solche
Unterstellung der Zweckmäßigkeit der Natur nachzuweisen ist Aufgabe einer
\enquote{Deduktion}, die \name[Immanuel]{Kant} im fünften Abschnitt der
Einleitung in die dritte Kritik durchzuführen beansprucht und die in der ersten
Einleitung in die \titel{Kritik der Urteilskraft} noch nicht
vorkommt.\footnote{Ich werde auf die Schwierigkeiten dieser Deduktion hier nicht
ausführlich eingehen. Siehe dazu
\cite[][35--42]{Allison:KantsTheoryofTaste2001},
\cite{Allison:ReflectiveJudgmentandtheApplicationofLogictoNature2012}, sowie
\cite{Guyer:ReasonandReflectiveJudgment1990}.}
\end{nummerierung}

\section{Autonomie und testimoniales
Wissen}\label{section:AutonomieundtestimonialesWissen}
\Revision[Heidemann]{Aus dem Gesagten lassen sich nun Regeln spezifizieren, wie
ein mündiger Mensch mit testimonialem Wissen umgeht. Die erste dieser Regeln
findet sich dabei in der Darstellung des Aberglaubens, die auf der Einsicht
beruht, dass jedes unserer Urteile einen \singlequote{metaphysischen Kern}
besitzt: Es setzt die Wahrheit oder Falschheit bestimmter metaphysischer Urteile
voraus, die in dem Urteil nicht explizit erwähnt werden. Manche Urteile, die dem äußeren Anschein
nach keine metaphysischen Urteile sind, verletzen damit aber gerade
metaphysische Erkenntnisse. Und die oberste Regel mündigen Urteilens besagt,
man solle die in allen Urteilen vorliegenden metaphysischen Erkenntnisse als
solche ernst nehmen und nicht durch vermeintliche empirische Erkenntnisse oder
bloß subjektiv historische Erkenntnisse ersetzen.}

Urteile sind Akte oberer Erkenntnisvermögen, es gibt entsprechend Urteile der
Vernunft, des Verstandes und der Urteilskraft. Nicht immer sind solche Urteile
frei von empirischen Bestandteilen. Die Urteile, dass dieser Winter sehr mild
war, dass Peter durch sein ständiges Lügen Unrecht tut oder dass sich bestimmte
beobachtbare Phänomene unter einheitliche Gesetze bringen lassen, kommen nur auf
Erfahrungsgrundlage zustande und sind insofern \singlequote{\emph{empirisch}}.
Sie müssen uns -- \singlequote{historisch} -- \emph{gegeben} werden. Aber sie
enthalten doch einen \singlequote{autonomen} Bestandteil, der uns gerade
\emph{nicht} gegeben werden kann. Offensichtlich ist dies bei dem Urteil über
Peter. Denn dieses Urteil setzt voraus und beinhaltet, dass Lügen Unrecht ist.
Aber auch das Urteil, dass der Winter mild war, setzt Kategorien und Grundsätze
voraus und das Urteil über die Möglichkeit der Vereinheitlichung der Beschreibung der
Phänomene beruht auf dem transzendentalen Prinzip der reflektierenden
Urteilskraft. Sie alle gelten \emph{a priori} und stellen daher rationale oder
Vernunfterkenntnisse dar.

Durch die Notwendigkeit von Vernunfterkenntnissen in \emph{allen} Urteilen
ergibt sich auch bei historischen Erkenntnissen, die wir sowohl aus eigener
Erfahrung als auch von Informanten haben können, eine Verwendung für den je
eigenen Verstand: zu überprüfen, ob das, was uns erzählt wird oder was wir
wahrnehmen, aus (allgemein- und transzendental-) logischen Gesichtspunkten
heraus war sein \emph{kann}.\footnote{\cite[Vgl.][B
84]{Kant:KritikderreinenVernunft2003}, \cite[][III:
80.11--16]{Kant:GesammelteWerke1900ff.}: \enquote{Also ist das bloß logische
Kriterium der Wahrheit, nämlich die Übereinstimmung einer Erkenntnis mit den
allgemeinen und formalen Gesetzen des Verstandes und der Vernunft zwar die
conditio sine qua non, mithin die negative Bedingung aller Wahrheit: weiter aber
kann die Logik nicht gehen, und den Irrtum, der nicht die Form, sondern den
Inhalt trifft, kann die Logik durch keinen Probierstein entdecken.}} Und dies
ist auch der Umgang mit Zeugen, wie wir ihn aus der Rechtspraxis kennen: Wir
haben selbst keine Möglichkeit zu überprüfen, ob ein Bericht der Wahrheit
entspricht. Aber wir können doch überprüfen, ob er konsistent ist und
\emph{möglicherweise wahre} Aussagen macht.
So dient die Vernunft bei historischer Erkenntnis zwar nicht dem Erwerb neuer,
aber der Berichtigung bereits vorhandener Erkenntnisse, deren natürlicher
Maßstab die eigene und fremde Erfahrung ist. \name[Immanuel]{Kant} drückt dies
auch so aus, dass er sagt, die Logik sei ein \emph{Kanon} (wenn auch kein
\emph{Organon}) der Erkenntnis.\footnote{\cite[Vgl.][B
86]{Kant:KritikderreinenVernunft2003}, \cite[][III:
81.11--13]{Kant:GesammelteWerke1900ff.}: \enquote{Nun kann man es als eine
sichere und brauchbare Warnung anmerken: daß die allgemeine Logik, \ori{als
Organon betrachtet}, jederzeit eine Logik des Scheins, d.\,i. dialektisch sei.}
Siehe auch \cite[][B 87\,f.,]{Kant:KritikderreinenVernunft2003} \cite[][III:
81.30--82.32]{Kant:GesammelteWerke1900ff.}.}

\phantomsection\label{BestimmungdesBegriffsAberglaube}
Die Vermutung liegt nahe, dass eine Kritik an Wundern und Aberglauben primär aus
der Richtung einer naturwissenschaftlichen Aufklärung zu erwarten
ist.\footnote{So schreibt bspw.
\authorfullcite{Schroeder:UrspruengedesAtheismus1998}: \enquote{Das Wunder ist
in den Metaphysiken und Rationaltheologien, in deren Rahmen außergewöhnliche
Eingriffe Gottes in die Naturordnung vorgesehen sind, der systematische Ort, an
dem, wie es scheint, der Gegensatz zwischen Theismus und Naturwissenschaft am
schärfsten hervorbrechen muß}
\parencite[][268]{Schroeder:UrspruengedesAtheismus1998}.} Es lässt sich jedoch
leicht zeigen, dass \name[Immanuel]{Kant} die Möglichkeit von Wundern -- wie
zuvor bereits
\authorcite{Spinoza:SpinozaOpera1972} im \titel{Tractatus
theologico-politicus}\footnote{\cite[Vgl.][III:
82.26--84.19]{Spinoza:SpinozaOpera1972}.} -- durch \emph{metaphysische}
Erkenntnisse widerlegt sieht, insbesondere durch den \enquote{Grundsatz der
Zeitfolge nach dem Gesetze der Kausalität}, den die zweite Erfahrungsanalogie
begründet.\footnote{\enquote{Alle Veränderungen geschehen nach dem Gesetze der
Verknüpfung der Ursache und Wirkung} \mkbibparens{\cite[][B
232]{Kant:KritikderreinenVernunft2003}, \cite[][III:
166.32--33]{Kant:GesammelteWerke1900ff.}, \ohio}.}

Der Wunderglaube ist eine Form des Aberglaubens. Aberglaube wiederum ist nach
der \titel{Kritik der Urteilskraft} das Vorurteil, \enquote{sich die Natur
Regeln, welche der Verstand ihr durch sein eigenes wesentliches Gesetz zum
Grunde legt, als nicht unterworfen
vorzustellen}\footnote{\cite[][\S~40]{Kant:KritikderUrteilskraft2009},
\cite[][V: 294.22--24]{Kant:GesammelteWerke1900ff.}}.
\name[Immanuel]{Kant}s Begriff des Aberglaubens lässt sich als Konkretisierung
dieses Gedankens verstehen, die davon ausgeht, dass nur das als möglich betrachtet werden darf,
was den Gesetzen des Verstandes entspricht. Und diese sind wiederum Gegenstand
der Metaphysik, welche Disziplin daher jeder berücksichtigen muss, der
selbstbestimmt urteilen möchte. Denn selbstbestimmt urteilen heißt hier nichts
anderes, als den Gesetzen der eigenen Vernunft (im weiteren Sinne) und damit des
eigenen Verstandes gemäß urteilen. Wer Wundergeschichten glaubt, der macht
den Fehler, die Autorität anderer höher zu schätzen als die Autorität des
eigenen Verstandes; und damit den eigenen Vernunftgebrauch nicht mehr als
oberstes Kriterium der Wahrheit\footnote{\cite[Vgl.][A
329]{Kant:Washeisst:SichimDenkenorientieren?1977}, \cite[][VIII:
146.29--35]{Kant:GesammelteWerke1900ff.}. Siehe dazu oben Kapitel
\ref{def:selbstdenken}, S.~\pageref{def:selbstdenken}.} gelten zu
lassen.\footnote{\enquote{\ori{Aberglaube} ist der Hang, in das, was als nicht
natürlicher Weise zugehend vermeint wird, ein größeres Vertrauen zu setzen, als was sich nach
Naturgesetzen erklären läßt -- es sei im Physischen oder
Moralischen} \mkbibparens{\cite[][A 106]{Kant:DerStreitderFakultaeten1977},
\cite[][VII: 65.26--28]{Kant:GesammelteWerke1900ff.}.}.
Eine Parallele zum Begriff des Aberglaubens als Missachtung rationaler
Erkenntnisse in der Erkenntnis der Natur findet sich auch in der praktischen
Philosophie: \enquote{Der durch Furcht abgenötigte Gehorsam in Ansehung eines
solchen Kirchenglaubens, als zur Seligkeit erforderlich, ist also
Aberglaube} \mkbibparens{\cite[][A 108]{Kant:DerStreitderFakultaeten1977},
\cite[][VII: 66.36--38]{Kant:GesammelteWerke1900ff.}}. Aberglaube ist hier das
Vertrauen in Schriftquellen, die uns Rituale und Überzeugungen vorschreiben, die
sich nicht aus reiner Vernunft nachvollziehen lassen.}

Auf diese Weise lässt sich zusammenfassen, dass Aberglaube überall dort
vorliegt, wo metaphysische Erkenntnisse nicht als solche behandelt, sondern dem
unterworfen werden, was uns von irgend woher \emph{gegeben} wird. In der Schrift
\titel{Was heißt: sich im Denken orientieren?} findet sich etwa folgende
Bestimmung des Begriffs \enquote{Aberglaube}:
\begin{quote}
[S]o müssen zuletzt aus innere Eingebungen durch Zeugnisse äußere
bewährte Facta, aus Traditionen, die anfänglich selbst gewählt waren, mit der
Zeit \ori{aufgedrungene} Urkunden, mit einem Worte die gänzliche Unterwerfung
der Vernunft unter Facta, d.\,i. der \ori{Aberglaube} entspringen, weil dieser
sich doch wenigstens in eine \ori{gesetzliche Form} und dadurch in einen
Ruhestand bringen läßt.\footnote{\cite[][A
327]{Kant:Washeisst:SichimDenkenorientieren?1977}, \cite[][VIII:
145.30--35]{Kant:GesammelteWerke1900ff.}.}
\end{quote}
Abergläubisch ist nicht immer derjenige, der explizit von objektiv
philosophischen Erkenntnissen subjektiv historische Erkenntnis hat, sondern oft
derjenige, der die philosophischen Erkenntnisse, die ihm selbst zur Verfügung
stehen, in der Anwendung seiner Erkenntnisvermögen einfach ignoriert. In jeder
Erkenntnis und in jedem Urteil sind wir nicht bloß passiv rezipierend, sondern
immer auch selbst tätig: Unsere Spontaneität ist integraler Bestandteil selbst
historischer Erkenntnisse. Spontaneität äußert sich in metaphysischen
Erkenntnissen, also objektiv rationalen, dabei aber diskursiven Urteilen. Und
diese unvermeidliche Metaphysik konstituiert die oberen Erkenntnisvermögen und
durchdringt zugleich sämtliche Erkenntnisse auch des unteren.
Aufklärung fordert von uns, diese metaphysische Basis unseres Denkens ernst zu
nehmen und nichts für wahr zu halten, was diesem obersten Probierstein unseres
Denkens widerspricht.

\Revision[Heidemann]{Mündig urteilen heißt also, die Ansprüche der Vernunft zu
achten, dasjenige, was sie ohne Rekurs auf Erfahrung zu erkennen vermag, auch
entsprechend selbst zu verantworten und nicht dem Diktat einer (vermeintlichen)
Erfahrung zu unterwerfen. Was dem Bereich objektiv rationaler Erkenntnisse
zugehört, soll nicht subjektiv historisch erkannt werden. Genau dies passiert
jedoch im Bereich des Aberglaubens, der gerade in dort vorliegt, wo der
metaphysische Charakter des Wissens nicht erkannt oder nicht beachtet wird.
Dieser Begriff des Aberglaubens beruht also offensichtlich auf den Überlegungen
zur Metaphysik und zum Begriff historischer Erkenntnis. Wie ich im folgenden Kapitel
zeigen möchte, integriert ihn \name[Immanuel]{Kant} in eine \emph{Ethics of
Belief}, die auf genau dieser Konzeption aufbauend (mündige oder aufgeklärte) Überzeugung von
(unmündiger oder unaufgeklärter) Überredung unterscheidet.}




\chapter{Kants \emph{Ethics of Belief}}\label{section:KantsEthicsofBelief}
Wenn jemand absichtlich die Unwahrheit sagt, dann lügt er und verdient Tadel.
Ebenso handelt unmoralisch, wer ein falsches Versprechen abgibt oder wer andere
Menschen durch rassistische Äußerungen zum Hass aufwiegelt. Aber wie verhält es
sich mit den Ansichten, die jemand hegt, ohne sie anderen mitzuteilen? Falsche
Versprechen und Lügen sind Verstöße gegen ethische Normen, die Menschen durch
sprachliche Handlungen gegenüber anderen begehen. Wenn wir nun nicht nur die äußeren
Handlungen eines Menschen ethisch bewerten, sondern bereits seine
Überzeugungen selbst sowie die Art und Weise, wie er Überzeugungen bildet; wenn
wir Menschen tadeln, weil sie an Hexen und Magie, an Horoskope und Homöopathie
glauben, weil sie rassistische, sexistische oder andere Vorurteile hegen oder
weil sie leichtgläubig sind und anderen Menschen alles Mögliche glauben, dann
vertreten wir eine \emph{ethics of belief}. Wir kritisieren Menschen für
ihre Überzeugungen oder die Weisen ihrer Überzeugungsbildung unabhängig davon,
was sie mit diesen Überzeugungen anstellen.

\authorfullcite{Clifford:TheEthicsofBelief1877} argumentiert, dass all unsere
Überzeugungen Gegenstand ethischer Bewertung sein sollten, weil jede
Überzeugung, die wir haben, sich auf irgendeinem Weg in Handlungen äußern wird,
die andere Menschen
affektieren.\footnote{\cite[Vgl.][289--295]{Clifford:TheEthicsofBelief1877}.}
Auch \name[Immanuel]{Kant} hält manche Überzeugungen oder vielmehr
die Art und Weise, wie sie gebildet werden, für tadelns- oder
lobenswert.\footnote{\enquote{Daher können wir freylich jemand Tadeln, welcher
einer falschen Erkenntniß beyfall gegeben hat; wenn nemlich die Schuld
wircklich an ihm selbst lieget, daß er nemlich die jenige gründe, welche ihm
von dem Gegenstand der Erkenntniß, die er hat, hätten überzeugen, und also von
seinem Irrthum befreyen können, abweist}
\mkbibparens{\cite{Kant:LogikBlomberg1966}, \cite[][XXIV:
160.12--17]{Kant:GesammelteWerke1900ff.}}.
Diese Stelle zitiert \textcite[vgl.][317]{Cohen:KantontheEthicsofBelief2014}.} Sie sind als
Ausdruck von Mündigkeit oder eben Unmündigkeit bewertbar, auch ohne auf die
Konsequenzen der Überzeugungen zu achten. Systematischen Niederschlag findet
dies dort, wo er auch die Wahrhaftigkeit \emph{sich selbst gegenüber} zu den
moralischen Pflichten
zählt.\footnote{\cite[Vgl.][\S~9]{Kant:DieMetaphysikderSitten1977Tugendlehre},
\cite[][VI: 430.9--26]{Kant:GesammelteWerke1900ff.}.} Aber es scheint sich auch
direkt aus dem Programm der Aufklärung und der Aufforderung zum Ausgang aus
selbstverschuldeter Mündigkeit zu ergeben, denn schon die Anwendung des
Prädikats \enquote{selbstverschuldet} setzt die Möglichkeit der ethischen Bewertung voraus.

Seit geraumer Zeit wird über eine \emph{ethics of belief} bei
\name[Immanuel]{Kant} gesprochen, die sich im vorletzten Abschnitt der
\titel{Kritik der reinen Vernunft} ausmachen lasse. Allerdings sind seine
Bemerkungen im \titel{Kanon der reinen Vernunft} durchgehend kurz und
skizzenhaft,\footcite[Vgl.][323]{Chignell:BeliefinKant2007} was ihre
Interpretation merklich erschwert. Seit einigen Jahren steigt das Interesse an
diesem Thema,\footnote{Siehe
\cite{Stevenson:OpinionBelieforFaithandKnowledge2003}, \cite{Chignell:BeliefinKant2007},
\cite{Chignell:KantsConceptsofJustification2007},
\cite{Cohen:KantontheEthicsofBelief2014}.} was nicht zuletzt an dem Begriff des
Glaubens liegt, mit dem
\name[Immanuel]{Kant} in den Augen mancher Interpreten einen interessanten
Ansatz zu einem \enquote{nicht-epistemischen Rechtfertigungsbegriff}
entwickle\footnote{\cite[Vgl.][\pno~33\,f.]{Chignell:KantsConceptsofJustification2007}.}
und durch den seine \emph{ethics of belief} in Richtung des späteren
Pragmatismus tendiere\footnote{\cite[Vgl.][335]{Chignell:BeliefinKant2007}.}.

Nun liegt es nahe, gerade hier eine systematische Verbindung von
Aufklärungsprogramm und Vernunftkritik und damit der Endlichkeit des Menschen,
auf deren Konzeption die Vernunftkritik basiert, zu vermuten.
Diese Vermutung muss sich freilich mit zwei Vorbehalten auseinandersetzen:
\name[Immanuel]{Kant} spricht dort erstens nicht explizit von Aufklärung,
Mündigkeit und Selbstdenken, aber es lässt sich leicht zeigen, dass genau diese
Programmatik den Ausführungen zugrunde liegt. Und zweitens ließe sich fragen, in
welchem Zusammenhang diese kurzen und skizzenhaften Abschnitte, die in Arbeiten
zur Vernunftkritik kaum Erwähnung
finden,\footcite[Vgl.][\pno~323\,f.]{Chignell:BeliefinKant2007} mit den
Kernthemen der Vernunftkritik -- Deduktion der Kategorien, Grundsätze des reinen
Verstandes, transzendentale Dialektik -- stehen \emph{respective} ob es
überhaupt einen näheren Zusammenhang gibt.

Ich werde argumentieren, dass die Überlegungen im Abschnitt \titel{Vom Meinen,
Wissen und Glauben} eine \emph{ethics of belief} auf Grundlage des
\enquote{sapere aude!} enthalten und dass es sich in der Tat um einen
interessanten und brauchbaren Ansatz handelt. Vielleicht sollte ich genauer sagen, dass sich
auf dieser Grundlage eine solche entwickeln lässt, denn die Skizzenhaftigkeit
seiner Ausführungen überlässt doch noch sehr viel dem Leser. Es sind dabei vor
allem die Begriffe der \emph{Überzeugung} und der \emph{Überredung}, die seinen
Ansatz auszeichnen (Abschnitt \ref{subsubsection:UEberredungundÜberzeugung}). Die Frage, warum es im Umgang
mit rationalen Erkenntnissen zu beachtende Unterschiede zwischen Metaphysik und
Mathematik gebe, führt auf einige abschließende Beobachtungen dazu, wie
philosophische Inhalte mündig erkannt werden (Abschnitt
\ref{subsubsection:EndlichesundUnendlichesErkennen}).

\Revision[Heidemann]{In diesen beiden Abschnitten werden die relevanten Punkte
einer auf Mündigkeit hin ausgerichteten \emph{ethics of belief} bereits enthalten sein. Nun ist es jedoch
so, dass \name[Immanuel]{Kant}s Überlegungen sich besonders durch seinen
Begriff des \emph{Vernunftglaubens} auszeichnen, der in der Darstellung
sicherlich nicht stillschweigend übergangen werden darf.}
Die Besonderheit der Position \name[Immanuel]{Kant}s liegt in der
Behauptung, es gebe auch Gründe für die Vernünftigkeit einer Überzeugung, die sich nicht auf
die Wahrheit der Überzeugung, sondern ein Bedürfnis des Subjekts richten. Die
Motivation dieser Position liegt darin, dass \name[Immanuel]{Kant}s
Aufklärungsprogramm einerseits religiöse Überzeugungen fokussiert, andererseits
aber nur vernünftig begründete Überzeugungen als zulässig ansieht. Da sich aus
unserer Endlichkeit ergibt, dass wir bezüglich zentraler religiöser Aussagen
keine Gründe finden, welche die Wahrheit der Überzeugungen belegen können, sieht
sich \name[Immanuel]{Kant} gezwungen, einen Begriff subjektiv zureichenden
Fürwahrhaltens zu entwickeln (Abschnitt
\ref{section:HandelnAufEpistemischDuennerGrundlage}).

\section{Überreden und
Überzeugen}\label{subsubsection:UEberredungundÜberzeugung}
\Revision[Pelletier]{Dem Titel nach befassen sich
\name[Immanuel]{Kant}s Überlegungen im Abschnitt \KapitelTitel{Vom
Meinen, Wissen und Glauben} im \KapitelTitel{Kanon der reinen Vernunft} mit
einer Begriffstrias, die sich beispielsweise im siebenten Kapitel der
\titel{Deutschen Logik} von
\cite{Wolff:VernuenftigeGedankenvondenKraeftendesmenschlichenVerstandesundihremrichtigenGebraucheinErkenntnisderWahrheit1978}
findet.\footcite[Vgl.][200--205]{Wolff:VernuenftigeGedankenvondenKraeftendesmenschlichenVerstandesundihremrichtigenGebraucheinErkenntnisderWahrheit1978}
Er beginnt seine Darstellung jedoch mit einem anderen Begriffspaar, nämlich} mit
der Unterscheidung von Überredung und Überzeugung. \Revision[Pelletier]{Auch
dieses Begriffspaar findet sich freilich bereits bei früheren Autoren.
\authorcite{Baumgarten:Metaphysica---Metaphysik2011} etwa bezeichnet Überredung
als sinnliche, Überzeugung hingegen als intellektuelle
Gewissheit.\footnote{\Revision[Pelletier]{\cite[Vgl.][531]{Baumgarten:Metaphysica---Metaphysik2011}.}}
Und \authorcite{Meier:AuszugausderVernunftlehre1752} sieht in der Überredung
einen deffizienten Zustand, in dem wir einen Irrtum fälschlich für gegründetes
Wissen halten.\footnote{\cite[][\S~184]{Meier:AuszugausderVernunftlehre1752}.}
\name[Immanuel]{Kant}s Bestimmung weicht von beiden ab}:
\begin{quote}
Das Fürwahrhalten ist eine Begebenheit in unserem Verstande, die auf objektiven
Gründen beruhen mag, aber auch subjektive Ursachen im Gemüte dessen, der da
urteilt, erfordert. Wenn es für jedermann gültig ist, so fern er nur Vernunft
hat, so ist der Grund desselben objektiv hinreichend, und das Fürwahrhalten
heißt alsdenn \ori{Überzeugung}. Hat es nur in der besonderen Beschaffenheit des
Subjekts seinen Grund, so wird es \ori{Überredung} genannt.\\
Überredung ist ein bloßer Schein, weil der Grund des Urteils, welcher lediglich
im Subjekte liegt, für objektiv gehalten
wird.\footnote{\label{Anmerkung:KU90UeberredenUeberzeugen}\cite[][B
848]{Kant:KritikderreinenVernunft2003}, \cite[][III: 531.27--532.4]{Kant:GesammelteWerke1900ff.}. Eine Parallelstelle
findet sich in der \titel{Kritik der Urteilskraft}: \enquote{Zuerst wir zu
jedem Beweise, er mag (wie bei dem Beweise durch Beobachtung des Gegenstandes
oder Experiment) durch unmittelbare empirische Darstellung dessen, was bewiesen
werden soll, oder durch Vernunft a priori aus Prinzipien geführt werden,
erfordert: daß er nicht \ori{Überrede}, sondern \ori{Überzeuge}, oder wenigstens
auf Überzeugung wirke, d.\,i. daß der Beweisgrund oder der Schluß nicht ein bloß
subjektiver (ästhetischer) Bestimmungsgrund des Beifalls (bloßer Schein),
sondern objektiv-gültig und ein logischer Grund der Erkenntnis sei; denn sonst
wird der Verstand berückt, aber nicht überführt}
\mkbibparens{\cite[][\S~90]{Kant:KritikderUrteilskraft2009}, \cite[][V:
461.14--22]{Kant:GesammelteWerke1900ff.}}.}
\end{quote}
Jedes Urteil hat subjektive Ursachen, aber manche Urteile haben darüber hinaus
objektive Gründe. Im Idealfall ist es wohl so, dass beides zusammenfällt und
der objektive Grund eines Urteils zugleich seine subjektive Ursache ist.
Wenn Ingrid sieht, dass Max den Kaffee verschüttet, dann hat ihre Überzeugung,
dass Max seinen Kaffee verschüttet hat, eine subjektive Ursache, die zugleich ein
objektiver Grund ist: ihre Wahrnehmung von Max' Ungeschicklichkeit. Es kommen
dabei nur solche Gründe als objektive Gründe in Betracht, die dem Urteilenden
\emph{de facto} zugänglich sind. Wenn Ingrid unaufmerksam war und nicht sah, wie
Max den Kaffee verschüttet, dann hat sie keinen objektiven Grund für diese
Überzeugung, auch wenn sie einen solchen haben \emph{könnte}, wäre sie nicht in
Gedanken gewesen. Und ebenso hatte Pierre de \name[Pierre de]{Fermat} keine
objektiven Gründe für die Behauptung, dass die Gleichung $a^n + b^n = c^n$ für
$a,b,c,n \in \mathbb{N}^{+}$ und $n > 2$ keine Lösungen hat. Diese Behauptung
war daher zunächst nicht als \emph{Theorem} aufzufassen, sondern als
\name[Pierre de]{Fermat}s \emph{Vermutung}, auch wenn ein Beweis dieser Aussage
natürlich schon damals \singlequote{existierte}\footnote{Zumindest gilt dies in einer naheliegenden
Bedeutung des Ausdrucks \singlequote{exisitieren}, nach der mathematische
Beweise auch dann existieren, wenn sie noch nicht entdeckt wurden.}, wie sich
später zeigen ließ.
Als objektive Gründe zählen also nur solche Gründe, die ein Urteilender als
Gründe eines Urteils nennen kann und von denen er weiß, dass sie objektive
Gründe sind.

\subsection{Vernünftige Gründe}\label{subsection:VernuenftigeGruende}
Paradigmatische Fälle von Überredung finden wir bei Wunschdenken und
Selbsttäuschung.\footnote{\cite[Vgl.][331]{Chignell:BeliefinKant2007}, sowie
\cite[][323]{Cohen:KantontheEthicsofBelief2014}.} In der \titel{Logik Blomberg}
findet sich als Beispiel der Gedanke an ein Leben nach dem Tod, den wir
annehmen, weil uns dieser Gedanke gefällt \emph{respective} weil der Gedanke,
dass es kein solches Leben gibt, Unbehagen bereitet.\footnote{Vgl.
\cite{Kant:LogikBlomberg1966}, \cite[][XXIV:
198.25--30]{Kant:GesammelteWerke1900ff.}. Weiter heißt es dort: \enquote{Das
menschliche Gemüth ist auf diese Art wircklich sehr vielen Illusionen, und
Blendwercken unterworfen, blos darum öfters, weil uns etwas gefällt, so halten
wir es vor gewiß, und blos darum, weil uns etwas mißfällt, oder verdreußt,
halten wir es vor ungewiß. Diese Gewisheit [sic], oder Ungewißheit aber ist
nicht objectiv, sondern vielmehr Subjectiv [sic]}
\mkbibparens{\cite{Kant:LogikBlomberg1966}, \cite[][XXIV:
198.31--36]{Kant:GesammelteWerke1900ff.}}.} Wäre der doxastische Voluntarismus
wahr, dann \enquote{würden wir uns beständig Chimären von einem glücklichen
Zustande machen und sie sodann auch immer für wahr halten.}\footnote{\cite[][A
113]{Kant:ImmanuelKantsLogik1977}, \cite[][IX:
74.4--5]{Kant:GesammelteWerke1900ff.}.} Aber auch wenn wir nicht
einfach für wahr halten können, was wir möchten, tendieren wir doch gerade dort,
wo wir keine objektiven Gründe kennen, dazu, zu glauben, was uns gefällt. Dass
der doxastische Voluntarismus falsch ist,\footnote{Siehe
dazu auch Anm. \ref{Anmerkung:KantundderDoxastischeVoluntarimus} auf S.
\pageref{Anmerkung:KantundderDoxastischeVoluntarimus} und die dort zitierte
Literatur.} heißt zumindest nach \name[Immanuel]{Kant} nicht, dass der Wille und
dass unsere Wünsche und Neigungen \emph{keinen} Einfluss auf unser Fürwahrhalten
haben. Es heißt lediglich, dass sie gegen überzeugende Beweise von Wahrheiten
nichts ausrichten können und dass es uns nicht gelingt, bewusst unsere
Überzeugungen zu wählen. Es heißt nicht, dass sie nicht unbewusst unsere Urteile
beeinflussen.

Ein Beispiel für Selbsttäuschung lässt sich im Anschluss an
\name[Immanuel]{Kant}s Ausführungen zur \emph{natürlichen Dialektik} der
\titel{Grundlegung} finden.
Wir erwerben moralische \emph{Überzeugungen}, wenn wir erkennen, dass etwas eine
moralische Forderung darstellt (oder dass es erlaubt ist), weil wir die
objektiven Gründe dieses Urteils einsehen. Aber allzu oft reden wir uns und anderen ein, etwas sei
(generell oder in einem speziellen Fall) erlaubt, weil wir ein subjektives
Bedürfnis haben. Wir bestreiten dann die Allgemeingültigkeit eines moralischen
Gebotes oder behaupten, dass in einem gegebenen Fall eine Ausnahme legitim sei.
Die Gründe, die wir dafür vorbringen, sind allesamt unzureichend, es ist
lediglich eine subjektive Neigung, die uns dazu antreibt, ein bestimmtes Urteil
zu fällen. Dennoch -- das ist der Witz der natürlichen Dialektik -- überreden
wir nicht nur andere, sondern auch uns selbst, unserer Argumentation zu
vertrauen.

Wie entscheiden wir, ob ein Urteil gut begründet ist oder ob es sich um
Wunschdenken und Selbsttäuschung handelt? Offenbar gibt es zwei Punkte, an denen
wir ansetzen können: Wir können die irrationalen subjektiven Ursachen zu eliminieren
versuchen und wir können untersuchen, ob unsere Gründe tatsächlich objektiv sind.
Dazu müssen wir zunächst eruieren, was der \singlequote{wahre} Grund für eine
Überzeugung ist. Max muss ausschließen, dass der Grund für seine Urteile über
Peter in seinem Neid liegt. Wenn er erkennt, dass dies sein Grund ist, stellt er
fest, dass er keinen objektiven Grund hat und sein Urteil unbegründet ist. Wie
erkennt Max aber, was der Grund für sein Urteil ist und ob es sich um einen
objektiven Grund oder eine bloß subjektive Ursache handelt?

\authorfullcite{Stevenson:OpinionBelieforFaithandKnowledge2003} hält es
für naheliegend, die Frage, ob ein objektiver Grund vorliegt, als eine Frage der
\singlequote{Introspektion} aufzufassen. Wenn Max urteilt, dass Peter unsympathisch ist, und nun wissen
möchte, ob dieses Urteil auf einem objektiv zureichenden Grund beruht, dann
solle er sich fragen, welcher Grund ausschlaggebend für sein Urteil ist. Es
könnte ja sein, dass er entdeckt, neidisch auf Peter zu sein und aus diesem
Neid heraus zu urteilen. Aber das ist offensichtlich
unbefriedigend, denn dass Max eine mögliche irrationale Ursache in sich
entdeckt, mag ihn zur Vorsicht mahnen, aber es beweist noch lange nicht, dass
sein Urteil unbegründet ist. Es ist hilfreicher, wenn er die Gründe untersucht,
die für sein Urteil sprechen (und vielleicht auch die, die gegen sein Urteil
sprechen), und diese Gründe anderen zur Prüfung vorlegt. Und so sagt auch
\name[Immanuel]{Kant}, dass die Überprüfung des eigenen Urteils an dem Urteil
anderer die \emph{einzige} Möglichkeit sei, Überredung und Überzeugung zu
unterscheiden.\footnote{\enquote{Überredung demnach kann von der Überzeugung
subjektiv zwar nicht unterschieden werden, wenn das Subjekt das Fürwahrhalten,
bloß als Erscheinung seines eigenen Gemüts, vor Augen hat; der Versuch aber,
den man mit den Gründen desselben, die für uns gültig sind, an anderer
Verstand macht, ob sie auf fremde Vernunft eben dieselbe Wirkung tun, als auf
die unsrige, ist doch ein, obzwar nur subjektives, Mittel, zwar nicht
Überzeugung zu bewirken, aber doch die Privatgültigkeit des Urteils, d.\,i.
etwas in ihm, was bloße Überredung ist, zu entdecken} \mkbibparens{\cite[][B
849]{Kant:KritikderreinenVernunft2003}, \cite[][III:
532.17--24]{Kant:GesammelteWerke1900ff.}}. Darauf verweist schließlich auch
\textcite[vgl.][79]{Stevenson:OpinionBelieforFaithandKnowledge2003}.}

Erschwerend kommt hinzu, dass \name[Immanuel]{Kant} neben objektiven Gründen und
subjektiven Ursachen auch objektiv zureichende und subjektiv zureichende Gründe
erwähnt. Nun ließe sich \emph{prima facie} vermuten, dass die objektiven Gründe
einfach mit den objektiv zureichenden Gründen zu identifizieren seien und
\emph{a fortiori} die subjektiv zureichenden Gründe den subjektiven Ursachen
entsprächen. Ich halte dies für unbefriedigend, weil es das, was
\name[Immanuel]{Kant} \emph{Glauben} nennt, zu einer Wirkung kontingenter
Ursachen degradiert. Ein Grund ist nicht allein dadurch bereits subjektiv
\emph{zureichend}, dass er in uns ein Fürwahrhalten (egal ob vernünftiger oder
unvernünftiger Weise) \emph{verursacht}.
Die Überlegungen zu subjektiv zureichenden Gründen können nur dann
gewinnbringend sein, wenn ein subjektiv zureichender Grund Gewähr dafür leistet,
dass unser Fürwahrhalten \emph{vernünftig} ist.\footnote{Dagegen urteilt
\authorfullcite{Pasternack:KantonOpinion2014}: \enquote{It refers to the
psychological state of firmly holding a proposition to be true}
\parencite[][43]{Pasternack:KantonOpinion2014}. Allerdings schreibt er weiter:
\enquote{Conviction has the same extension as subjective sufficiency, but it
explicitly brings out a normative element}
\parencite[][46]{Pasternack:KantonOpinion2014}, womit doch eine Wertung aus der
Perspektive der Vernunft Einzug hält.
Tatsächlich scheint gerade der \singlequote{doktrinale Glaube} lediglich einen
psychologischen Zustand zu beschreiben (siehe Kapitel \ref{section:HandelnAufEpistemischDuennerGrundlage}); aber
wenn die subjektive Zulänglichkeit nichts anderes bezeichnete, ginge der Witz der Überlegungen
verloren.} Einen Grund als
subjektiv zureichend zu bezeichnen, heißt nicht, ihm eine kausale Funktion zuzuschreiben, sondern einen epistemischen Status. Mir scheint \name[Immanuel]{Kant} bei der thematischen Hinführung im Abschnitt \KapitelTitel{Vom Meinen, Wissen und Glauben} das subjektiv zureichende Fürwahrhalten noch nicht zu berücksichtigen.

Ich schlage zunächst vor, in einem ersten Schritt zu erläutern, wann ein Grund
zureichend ist und erst dann den Unterschied zwischen subjektiv und objektiv
zureichenden Gründen zu klären. Um zu verstehen, wann ein Grund (subjektiv oder
objektiv) zureichend ist, sollten wir davon ausgehen, dass ein Grund nur dann
ein zureichender Grund sein kann, wenn die Bildung eines Fürwahrhaltens auf
seiner Grundlage \emph{vernünftig} ist. Um diesem Gedanken gerecht zu werden,
schlage ich vor, den Begriff des \emph{zureichenden Grundes} unter Rückgriff auf
doxastische Grundsätze zu erläutern. Unter einem \emph{doxastischen Grundsatz}
verstehe ich eine Regel, die (bewusst oder unbewusst) unser Fürwahrhalten
leitet, indem sie als Grundsatz bestimmt, was wir als Grund für ein Urteil
akzeptieren.\footnote{Ich verwende den Ausdruck \enquote{doxastisch} in einem
allgemeinen Sinne, so dass er auch \enquote{epistemisch} mit umgreift und auf
alle Modi des Fürwahrhaltens -- Meinen, Glauben und Wissen -- bezogen werden
kann. Es ließe sich stattdessen auch der Ausdruck \enquote{epistemisch}
verwenden; jedoch wird dieser öfter in einem eingeschränkten Sinne verwendet
und lediglich auf Wissen bezogen, was hier die Folge hätte, dass der Ausdruck
\enquote{epistemischer Grundsatz} nur auf solche Grundsätze passte, die
objektive Gründe als solche ausweisen und Wissen konstituieren.}  Ein
Grund ist \emph{zureichend} genau dann, wenn der doxastische Grundsatz, der ihn als
Grund für ein Urteil ausweist, ein vernünftiger Grundsatz gemäß der positiven Bestimmung
des Selbstdenkens und \emph{a fortiori} kein Vorurteil ist. Der Grundsatz,
Informationen von anderen als Grundlage unseres Wissens zu akzeptieren, solange
nichts gegen deren Vertrauenswürdigkeit spricht, ist ein solcher Grundsatz
(siehe Kapitel \ref{section:autonomieunddaszeugnisanderer}).
Wenn Ingrid sich auf diesen Grundsatz berufen kann, dann liegt eine Überzeugung
und keine Überredung vor, auch wenn Peter lügt. Überzeugungen sind also die Folgen
von vernünftigen doxastischen Grundsätzen, Überredung ist die Folge eines
Vorurteils.

Die zentrale Forderung \enquote{sapere aude!} tritt in der \titel{Kritik der
reinen Vernunft} in Form der Unterscheidung von \emph{Überzeugung} und
\emph{Überredung} hervor. Selbstdenken heißt nach \name[Immanuel]{Kant}, sich im
Denken am Maßstab des eigenen Vernunftgebrauchs auszurichten, also nur das für
wahr zu halten, was wir selbst als nach vernünftigen doxastischen Grundsätzen
fundiert ansehen können. Ein Vorurteil wiederum ist ein unvernünftiger Grundsatz
des Denkens, also ein Urteil, dass selbst nicht hinreichend fundiert ist, uns
aber als Prinzip dient, auf dessen Grundlage wir weitere Urteile generieren.
Wenn Max glaubt, dass Peter unsympathisch ist, und Max dies glaubt, weil Peter
aus A-Dorf kommt, dann besteht sein Vorurteil darin, dass er ein unbegründetes
Inferenzschema akzeptiert: \emph{Wenn} Person $A$ aus A-Dorf kommt, \emph{dann}
ist $A$ unsympathisch. Sein Urteil, dass Peter unsympathisch ist, ist selbst
kein Vorurteil, sondern eine \emph{Folge} seines Vorurteils. Er könnte dies
entdecken, indem er seine Urteile an den Urteilen anderer Menschen überprüft;
diese werden Urteile, die nach dem genannten Inferenzschema begründet werden,
nicht akzeptieren, woraufhin Max erkennen kann, dass er manche seiner Urteile
vermutlich auf der Grundlage eines Vorurteils bildet. Er würde dann zunächst
versuchen, sein Urteil mit weiteren Gründen auszustatten und eine explizite
Grundlage suchen, die sein Vorurteil offensichtlich nicht bieten kann.

\name[Immanuel]{Kant}s Darstellung setzt voraus, dass unser Fürwahrhalten
überhaupt Regeln folgt, die wir als vernünftig oder unvernünftig bewerten können.
Wenn unser Fürwahrhalten aber nicht auf einem objektiven Grund, sondern
lediglich auf einer subjektiven Ursache beruht, dann ist denkbar, dass wir überhaupt keiner
Regel folgen. Für gewöhnlich sind es jedoch wiederkehrende Beeinflussungen wie
im Falle des Wunschdenkens oder der Sympathie und Antipathie, die uns dazu bringen, etwas
aufgrund subjektiver Ursachen statt objektiver Gründe für wahr zu halten. Und
wenn wir dem auch nur in Einzelfällen nachgeben, laufen wir Gefahr, die
entsprechende Urteilsweise zu habitualisieren\footnote{Dies ist ein
entscheidendes Argument bei
\textcite[vgl.][294]{Clifford:TheEthicsofBelief1877}.} und zu einer Regel zu
machen, der wir (unbewusst) folgen. Dass wir eine entsprechende Regel nicht als
solche unseres Denkens benennen können, heißt nicht, dass sie nicht als
Grundsatz unseres Denkens fungiert.

Wie \name[Immanuel]{Kant} in \titel{Was heißt: sich im Denken orientieren?}
ausführt, ist es eine Illusion, wenn jemand glaubt, er könne sein Denken
dauerhaft \emph{ohne} verbindliche Regeln gestalten. Er wird sein Denken
letztlich an den Regeln orientieren, die er \emph{nicht} frei
wählt.\footnote{\cite[Vgl.][A
327\,f.,]{Kant:Washeisst:SichimDenkenorientieren?1977}
\cite[][VIII: 145.6--35]{Kant:GesammelteWerke1900ff.}.} Urteile, die nicht
Ausdruck des Selbstdenkens sind, bilden wir nicht \emph{ohne} Regeln. Wir bilden
sie auf der Grundlage anderer Regeln, die wir nicht als vernünftig einsehen
können und die uns zumeist auch gar nicht bewusst sind.
Statt vernünftiger epistemischer Grundsätze regieren biografische Zufälligkeiten
wie Gewohnheiten und Neigungen unser Denken. Max' Urteil, dass Peter
unsympathisch ist, steht in Zusammenhang zu anderen Urteilen -- hier mit dem
Urteil, dass Peter aus A-Dorf kommt.
Und dieser Zusammenhang wird selbst Ursachen haben; vielleicht ist Max durch ein
traumatisierendes Erlebnis im Zusammenhang mit A-Dorf geprägt. Er folgt dann
immer noch einem Grundsatz in seinem Urteilen, aber einem Grundsatz, der nicht
Ausdruck seiner Vernunft, sondern Resultat biographischer Zufälligkeiten ist.
Wir können nur nach Regeln urteilen, doch können die entsprechenden Regeln
vernünftig sein oder unvernünftig. Vernünftig ist eine Regel, die wir ganz
bewusst zum allgemeinen Grundsatz unseres Denkens machen (und im Falle einer
Diskussion als zustimmungsfähigen Grundsatz benennen) können. Ein Vorurteil ist
eine Regel, der wir im Denken folgen, obwohl wir sie nicht als vernünftigen
Leitfaden ansehen können.

Wenn meine Analyse richtig ist, dann müssen wir bei der Beurteilung, ob ein
Grund zureichend ist, zunächst einen doxastischen Grundsatz identifizieren,
den es von den Bedingungen seiner Anwendung zu unterscheiden gilt. Mir scheint,
dass \name[Immanuel]{Kant} in den Abschnitten der transzendentalen Methodenlehre
lediglich auf diese doxastischen Grundsätze achtet und die Frage nach
Anwendungsbedingung nicht berücksichtigt. Die Frage, ob wir vernünftigen
doxastischen Grundsätzen folgen und dabei Erfolg haben,  oder ob wir mit diesen
vernünftigen Grundsätzen scheitern, ist zwar für eine Analyse des Begriffs
\enquote{Wissen} von zentraler Bedeutung, aber aus Sicht der \emph{ethics of
belief} nicht relevant. Ob in einem konkreten Fall Wissen vorliegt, mag von
Zufällen abhängen, die wir selbst nicht kontrollieren können. Aber ob wir mündig
und verantwortlich urteilen, das kann aus der Sicht \name[Immanuel]{Kant}s
wegen des ethischen Charakters der Fragestellung nicht davon abhängen, ob wir
\singlequote{Glück} haben -- man könnte hier von \emph{epistemic luck}
sprechen.\footnote{Der Ausdruck ist in Anlehnung an
\textcite[vgl.][]{Williams:MoralLuck1981} gebildet und findet sich etwa bei
\textcite[vgl.][]{Engel:IsEpistemicLuckCompatiblewithKnowledge1992} und
\textcite[vgl.][\pno~104\,f.]{Hills:MoralTestimonyandMoralEpistemology2009}.

Gegen diese Interpretation spricht \emph{prima facie}, dass
\name[Immanuel]{Kant} Urteile, die auf objektiv zureichenden Gründen beruhen, als \enquote{Wissen} bezeichnet.
\enquote{Wissen} wiederum ist ein eindeutig veridischer Begriff: Wir können eine
Überzeugung ausschließlich dann als Wissen bezeichnen, wenn sie tatsächlich wahr
ist. Wir stehen also vor der Wahl, entweder den Ausdruck \enquote{objektiv
zureichend} mit einer eigenen Wahrheitsbedingung zu versehen, die
Wahrheitsbedingung zusätzlich zum Vorliegen objektiv zureichender Gründe in den
Begriff des Wissens zu integrieren oder als zusätzliche epistemische Kategorie
neben \enquote{Wissen} auch \enquote{wahres Wissen}
zuzulassen \parencite[vgl.][330]{Chignell:BeliefinKant2007}.

Um das Sprechen von falschen Überzeugungen zu vermeiden, könnten wir auch sagen,
dass weder Überzeugung noch Überredung, sondern einfach eine Täuschung vorliegt.
Aber letztlich scheint mir \emph{Überzeugung} ein epistemischer Begriff zu sein,
der nicht von der Wahrheit des Urteils, sondern lediglich von den Gründen
abhängt, die den Urteilenden verfügbar sind. Als Beleg dafür, dass dies mit
\name[Immanuel]{Kant}s Wortgebrauch vereinbar ist, ließe sich folgende Bemerkung
anführen: \enquote{Die subjektive Zulänglichkeit heißt \ori{Überzeugung} (für
mich selbst), die objektive, \ori{Gewißheit} (für
jedermann)} \mkbibparens{\cite[][B 850]{Kant:KritikderreinenVernunft2003},
\cite[][III: 533.5--7]{Kant:GesammelteWerke1900ff.}}. Dass Überzeugung sich hier
nicht auf das nur subjektiv zureichende Fürwahrhalten bezieht, erhellt daraus,
dass Wissen, Glauben und Meinen alle als Stufen der \enquote{subjektive[n]
Gültigkeit des Urteils, in Beziehung auf die Überzeugung (welche zugleich
objektiv gilt)} \mkbibparens{\cite[][B 850]{Kant:KritikderreinenVernunft2003},
\cite[][III: 532.36--37]{Kant:GesammelteWerke1900ff.}} angesprochen
werden. Leider macht diese Bemerkung das Verständnis letztlich
schwieriger, da \name[Immanuel]{Kant} die Zusammenhänge nicht weiter
erläutert, sondern lapidar anmerkt, er werde sich \enquote{bei der Erläuterung
so faßlicher Begriffe nicht aufhalten} \mkbibparens{\cite[][B
850]{Kant:KritikderreinenVernunft2003}, \cite[][III:
533.7--8]{Kant:GesammelteWerke1900ff.}}.}

Meine Analyse kongruiert mit dem, was wir von einer \emph{ethics of belief}
vernünftigerweise erwarten sollten. Es geht um die Unterscheidung epistemisch
verantwortungsvoller (lobenswerter) Haltungen auf der einen und epistemisch
fahrlässiger (tadelnswerter) Ansichten auf der anderen Seite. Wir wollen aber
nicht denjenigen tadeln, der einer Täuschung anheim fiel, die er selbst nicht
entdecken konnte. Wenn Ingrid und Max sich in einer fremden Stadt nach dem Weg
erkundigen, aber das Pech haben, dass ein Witzbold ihnen die falsche Richtung
weist, dann ist dessen Lüge zu tadeln, nicht aber Ingrids und Max'
Gutgläubigkeit. Ingrid und Max erwerben eine Überzeugung, die nicht auf
subjektiven Ursachen beruht, aber auch nicht unbegründet ist. Und dies liegt daran, dass sie
vernünftigen epistemischen Grundsätzen folgen, wenngleich ihre Ausführung
unvorhergesehen (und unvorhersehbar) scheitert.

Anders verhält es sich bei falschen Grundsätzen. Wir tadeln ja gerade
diejenigen, die aus (beispielsweise) rassistischen Vorurteilen heraus urteilen
(und zwar ganz unabhängig davon, ob aus diesen Vorurteilen in Einzelfällen wahre
Urteile resultieren). Und wir tadeln denjenigen, der dem Grundsatz folgt, seine
politischen Urteile generell an denen seiner Eltern auszurichten. Wir sagen,
dass er unmündig sei, weil er hierin einem unvernünftigen epistemischen
Grundsatz folgt, der keine objektiven Gründe generiert, sondern geradezu dazu
aufruft, objektiv rationale Erkenntnisse subjektiv historisch zu erwerben. Der
Grundsatz ruft zum Verzicht auf Selbstdenken auf und ist als solcher Aufruf
tadelnswert, mögen aus diesem Verzicht auch wahre Urteile resultieren.

Ein zweiter Aspekt der Unterscheidung von Überzeugung und Überredung ist, dass
Überredung unbewusst stattfindet: Sie liegt vor, wenn jemand aus einer
subjektiven Zufälligkeit heraus urteilt, aber denkt, vernünftigen epistemischen
Grundsätzen zu folgen.\footnote{Allerdings suggeriert eine Bemerkung
\name[Immanuel]{Kant}s im selben Abschnitt, dass uns auch Überredungen manchmal
bewusst sein können: \enquote{Überredung kann ich für mich behalten, wenn ich
mich dabei wohl befinde}
\mkbibparens{\cite[][B 850]{Kant:KritikderreinenVernunft2003},
\cite[][III: 532.33--34]{Kant:GesammelteWerke1900ff.}}.} Max denkt in unserem
Beispiel vielleicht, er fälle ein erfahrungsbasiertes Urteil über Peter, doch in
Wahrheit folgt er einer zufällig erworbenen Abneigung gegen A-Dorf oder seinem
Neid gegenüber Peter. Sein Urteil ist Ausdruck eines Vorurteils, wenngleich er
selbst denkt, vernünftigen Grundsätzen zu folgen. Wir können in aller Regel wohl
davon ausgehen, dass der Grund darin liegt, dass derjenige, der einem Vorurteil
aufsitzt, selbst nicht weiß, welchem Grundsatz er folgt (sei es, dass er gar
keinen Grundsatz nennen könnte, sei es, dass er glaubt, einem Grundsatz zu
folgen, der in seinem Denken nicht maßgebend war und den er vielleicht sogar
verletzt).

In Anlehnung an \name[Immanuel]{Kant} bezeichne ich ein Urteil, bei dem der
Urteilende sich die Gründe seines Urteilens transparent gemacht hat und weiß,
welchen epistemischen Grundsätzen er folgt, als
\enquote{überlegt}.\footnote{\cite[Vgl.][B
316\,f.,]{Kant:KritikderreinenVernunft2003} \cite[][III:
214.33--215.31]{Kant:GesammelteWerke1900ff.}.} Aufklärung fordert von uns also,
überlegt zu urteilen.
Und das bedeutet, dass wir uns bewusst machen sollen, warum wir etwas für
wahr halten und ob es sich dabei um einen zureichenden Grund handelt.
Ein brauchbares Kriterium, dies zu erkennen, ist der in Kapitel
\ref{section:sensuscommunis} beschriebene \singlequote{Pluralismus} oder die
\singlequote{erweiterte Denkungsart}. Auf diese Weise können wir auch ohne
explizite Überlegung zu mündigen und vorurteilsfreien Urteilen gelangen. Denn
unsere bloß kontingenten Grundsätze verraten sich als
unvernünftig, wenn wir bei anderen keine Zustimmung finden. Freilich ist dieses
Kriterium dabei weder hinreichend -- die Menschen in unserer Umgebung mögen
dieselben Vorurteile hegen -- noch ist es stets zuverlässig -- mitunter können
wir mit einem vernünftigen Urteil auf vorurteilsbeladene Gesprächspartner
treffen.


Nun kann ein Grundsatz, der zureichende Gründe liefert, einerseits auf
die empirische Darstellung des Gegenstandes, andererseits auf Prinzipien der
Vernunft gestützt sein.\footnote{Siehe dazu das Zitat in Anmerkung
\ref{Anmerkung:KU90UeberredenUeberzeugen} auf S. \pageref{Anmerkung:KU90UeberredenUeberzeugen}.}
Wenn wir überlegen, auf welchen doxastischen Grundsätzen unser Urteil beruht,
dann ist nach \name[Immanuel]{Kant} genau die Frage relevant, die -- wie Kapitel
\ref{chapter:MuendigerErwerbTestimonialenWissens} zeigte -- seinen Umgang mit
testimonialem Wissen bestimmt:
\begin{quote}
Die erste Frage vor aller weitern Behandlung unserer Vorstellungen ist die: in
welchem Erkenntnisvermögen gehören sie zusammen? Ist es der Verstand, oder sind
es die Sinne, vor denen sie verknüpft, oder verglichen werden? Manches Urteil
wird aus Gewohnheit angenommen, oder durch Neigung geknüpft; weil aber keine
Überlegung vorhergeht, oder wenigstens kritisch darauf folgt, so gilt es für ein
solches, das im Verstande seinen Ursprung erhalten
hat.\footnote{\cite[][B 316]{Kant:KritikderreinenVernunft2003},
\cite[][III: 215.6--12]{Kant:GesammelteWerke1900ff.}.}
\end{quote}
Was hat die Frage nach dem Ursprung einer Erkenntnis mit der Frage zu tun, ob
sie der Vernunft oder bloßer Gewohnheit und Neigung entstammt? Mir scheint die
Antwort in folgendem zu liegen. Wir erwerben empirische Erkenntnis, indem wir
bestimmte Ereignisse stets verknüpft vorfinden. Eine solche Art des
Erkenntniserwerbs erinnert daher nicht grundlos an den Erwerb einer Gewohnheit,
weswegen es nicht absurd scheint, wenn
\authorfullcite{Hume:AnEnquiryConcerningHumanUnderstanding1964} die Grundlage
all unserer Schlüsse aus Erfahrung in der Gewohnheit (\emph{custom},
\emph{habit})
vermutet.\footnote{\cite[Vgl.][37]{Hume:AnEnquiryConcerningHumanUnderstanding1964}.}
Aber rationale Erkenntnisse dürfen wir nicht auf solche Art annehmen; hier ist
es gerade kein legitimer Maßstab, ob wir etwas regelmäßig in unserer Wahrnehmung
verknüpft finden.

Wenn wir nun aber nicht wissen, ob eine Erkenntnis eine
empirische oder eine rationale Erkenntnis ist (und sie vielleicht fälschlich
für empirisch halten), dann kann es uns passieren, dass wir sie als Erkenntnis
akzeptieren, weil wir sie in der Erfahrung bestätigt finden. Wir sehen die
Grundsätze des reinen Verstandes etwa in der Erfahrung bestätigt und akzeptieren
sie als vermeintlich empirisch fundiertes Wissen. Oder wir akzeptieren ein
moralisches Urteil, weil alle Menschen in unserer Umgebung beständig
entsprechend urteilen. In solchen Szenarien fällen wir Urteile aus bloßer
Gewohnheit, also aus Vorurteil, weil die Grundsätze, denen wir bei dem Fällen
unserer Urteile folgen, gar nicht die relevanten Grundsätze sind, die bei den
entsprechenden Urteilen anzulegen sind.

Um mündig und vorurteilsfrei zu denken, müssen wir nicht bei jedem Urteil
die Gründe und Gegengründe abwägen; aber wir müssen darüber Gewissheit erlangen,
auf welcher Grundlage wir urteilen, und speziell darüber, ob es sich um eine
(objektiv) rationale oder empirische Erkenntnis handelt und entsprechend
(subjektiv) rationale Erkenntnis gefordert ist oder historische Erkenntnis
legitim und ausreichend ist. \name[Immanuel]{Kant} sagt dazu, wir bedürften
nicht jederzeit einer \singlequote{Untersuchung}, wohl aber stets der
\singlequote{Überlegung} (\emph{reflexio}).\footnote{\enquote{Nicht alle Urteile
bedürfen einer \ori{Untersuchung}, d.\,i. einer Aufmerksamkeit auf die Gründe
der Wahrheit {\punkt}. Aber alle Urteile {\punkt} bedürfen einer
\ori{Überlegung}, d.\,i. einer Unterscheidung der Erkenntniskraft, wozu die
gegebenen Begriffe gehören}
\mkbibparens{\cite[][B 316\,f.,]{Kant:KritikderreinenVernunft2003}
\cite[][III: 215.12--18]{Kant:GesammelteWerke1900ff.}}.} Aufklärung und die
Forderung nach Selbstdenken erfordern keine tiefgreifenden Untersuchungen. Wenn
wir sehen, dass etwas der Fall ist, oder wenn uns jemand mitteilt, dass es der
Fall ist, dann brauchen wir nicht weiter zu untersuchen, ob dieser Grund denn
auch zuverlässig ist. Wir müssen etwa keine weiteren Informationen über die
Glaubwürdigkeit unseres Informanten einholen; der epistemische Grundsatz,
Mitteilungen Glauben zu schenken, ist völlig ausreichend.
Aber wir sollen doch in dem angesprochenen Sinne \emph{überlegt} urteilen, also
wissen, ob es sich um eine Erkenntnis handelt, bei welcher der entsprechende epistemische
Grundsatz Anwendung finden kann.

\subsection{Objektive Gründe}\label{subsection:ObjektiveGruende}
Im weiteren Verlauf diskutiert \name[Immanuel]{Kant} ausschließlich das
überlegte, aufgeklärte und vernünftige Fürwahrhalten. Dieses hat drei Stufen:
Meinen, Wissen und Glauben. Sie unterscheiden sich in ihrem Status, insofern
Wissen allein objektiv und subjektiv zureichend ist, Glauben immerhin noch
subjektiv zureichend, objektiv aber unzureichend, Meinen hingegen objektiv und
subjektiv unzureichend ist. Wissen liegt dort vor, wo wir über objektive Gründe
verfügen. Glaube ist ein Status, der zwar keine objektiv zureichenden Gründe zu
haben beanspruchen kann, aber subjektiv zureichende Gründe geltend macht.
Da es bei allen drei Arten des Fürwahrhaltens um überlegte Überzeugungen geht, täuscht
sich der Urteilende in allen Fällen nicht über den Status seiner Überzeugung.

\Revision[Pelletier]{Trotz der oberflächlichen Ähnlichkeit zu verbreiteten
Darstellungen beispielsweise bei
\authorcite{Wolff:Discursuspraeliminarisdephilosophiaingenere1996}\footnote{Siehe
\cite[][200--205]{Wolff:VernuenftigeGedankenvondenKraeftendesmenschlichenVerstandesundihremrichtigenGebraucheinErkenntnisderWahrheit1978},
sowie \cite[][\S\S~594--661]{Wolff:PhilosophiarationalissiveLogica1740}.} ist diese Systematik durchaus
originell. Autoren wie \authorcite{Wolff:Discursuspraeliminarisdephilosophiaingenere1996} und
\authorcite{Meier:AuszugausderVernunftlehre1752} verstehen den Glauben schlicht
als \singlequote{historischen Glauben}, also als testimoniales
Wissen,\footnote{\Revision[Pelletier]{\cite{Wolff:PhilosophiarationalissiveLogica1740}:
\enquote{\ori{Fides} dicitur assenus, quem praebemus propositioni propter
autoritatem dicentis, ipsamque propositionem \ori{credere} dicimur.}
\authorcite{Meier:AuszugausderVernunftlehre1752} schreibt: \enquote{Aus
anderer Leute Erfahrung werden wir, vermittelst des Glaubens, gewiss}
\mkbibparens{\cite[][\S~206]{Meier:AuszugausderVernunftlehre1752}, \cite[][XVI:
496.28--29]{Kant:GesammelteWerke1900ff.}}.}} welches \name[Immanuel]{Kant} hier
wiederum unter den Begriff des Wissens subsumiert, ohne ihm eine eigene
Kategorie zuzuweisen. Meinungen werden in aller Regel als unzureichend bewiesene
Behauptungen verstanden; ein Rekurs auf das Wissen um den Status des je eigenen
Fürwahrhaltens findet sich zumindest bei
\authorcite{Meier:AuszugausderVernunftlehre1752}, der zugleich zwischen gemeiner
(\singlequote{\emph{opinio vulgaris}}) und gelehrter oder philosophischer
Meinung (\singlequote{\emph{hypothesis
philosophica, erudita}})
unterscheidet.\footnote{\Revision[Pelletier]{\cite[Vgl.][\S~181]{Meier:AuszugausderVernunftlehre1752},
\cite[][XVI: 461.22--24]{Kant:GesammelteWerke1900ff.}: \enquote{\ori{Eine Meinung} (opinio) ist eine jede ungewisse Erkenntniss, in so ferne wir sie
annehmen, und zugleich erkennen, dass sie nicht gewiss sei.} Dagegen definiert
\authorcite{Wolff:Discursuspraeliminarisdephilosophiaingenere1996} schlichter:
\enquote{Propositio insufficienter probata dicitur \ori{Opinio}}
\parencite[][\S~602]{Wolff:PhilosophiarationalissiveLogica1740}.}} Was sich
meines Wissens nirgends vor \name[Immanuel]{Kant} findet, das ist die
Unterscheidung zwischen subjektiv und objektiv zureichendem Fürwahrhalten.}

Nun habe ich bisher nur gesagt, wann ein Grund zureichend ist, ohne auf den
Unterschied zwischen objektiv und subjektiv zureichenden Gründen einzugehen.
Ein Grund ist ein zureichender Grund, wenn es nach Maßgabe des
positiven Begriffs des Selbstdenkens vernünftig ist, auf seiner Grundlage einem
Urteil seine Zustimmung zu geben. Einen Zusammenhang zwischen beiden können wir jedoch bereits festhalten. Denn
wie aus der Einteilung in Wissen, Glauben und Meinen unmittelbar hervorgeht,
folgt aus der objektiven Zulänglichkeit die subjektive. Wenn ich einsehe, dass
ich über einen objektiven Grund $G$ verfüge zu urteilen, dass $p$, dann ist $G$
ein Grund, der auch subjektiv zureichend ist.\footnote{\name[Immanuel]{Kant} kennt kein Fürwahrhalten, das
objektiv zureichend, subjektiv aber unzureichend ist. Wenn Ingrid einen objektiv
zureichenden Grund hat, dann hat sie \emph{eo ipso} einen subjektiv zureichenden
Grund. \authorfullcite{Stevenson:OpinionBelieforFaithandKnowledge2003} verweist
darauf, dass objektive Zulänglichkeit subjektive Zulänglichkeit impliziert und sich so
die Dreiteilung in Meinen, Glauben und Wissen ergibt
\parencite[vgl.][78]{Stevenson:OpinionBelieforFaithandKnowledge2003}.
\authorfullcite{Chignell:BeliefinKant2007} hält die Annahme dieser Implikation
für unbegründet und spricht im Falle von objektiv zureichenden Gründen, die dem
Urteilenden nicht zugänglich sind, von \enquote{mere convictions}
\parencite[vgl.][\pno~331\,f.]{Chignell:BeliefinKant2007}. Es ist aber
eindeutig, dass \name[Immanuel]{Kant} eine entsprechende Kategorie nicht für
diskussionswürdig erachtet.}
Wann ist ein Grund nun aber auch \emph{objektiv}
und wann ist er nur \emph{subjektiv} zureichend? \name[Immanuel]{Kant}s einzige
Auskunft scheint zu sein: \enquote{Die subjektive Zulänglichkeit heißt
\ori{Überzeugung} (für mich selbst), die objektive, \ori{Gewißheit} (für jedermann).}\footnote{\cite[][B
850]{Kant:KritikderreinenVernunft2003}, \cite[][III:
533.5--7]{Kant:GesammelteWerke1900ff.}. \name[Immanuel]{Kant} fügt an, er werde
sich \enquote{bei der Erläuterung so faßlicher Begriffe nicht aufhalten}
\mkbibparens{\cite[][B 850]{Kant:KritikderreinenVernunft2003}, \cite[][III:
533.7--8]{Kant:GesammelteWerke1900ff.}}. Siehe auch \cite[][A
98\,f.,]{Kant:ImmanuelKantsLogik1977} \cite[][IX:
66.4--7]{Kant:GesammelteWerke1900ff.}: \enquote{Das gewisse Fürwahrhalten oder
die \ori{Gewißheit} ist mit dem Bewußtsein der Notwendigkeit verbunden; das
ungewisse dagegen oder die \ori{Ungewißheit} mit dem Bewußtsein der Zufälligkeit
oder der Möglichkeit des Gegenteils.}} Doch was heißt dies? Die Schwierigkeit in
der Interpretation liegt offensichtlich darin, dass auch die subjektive
Zulänglichkeit vernünftig sein soll und insofern scheinbar nicht nur für mich
gelten kann. Die Bedeutung dieser Unterscheidung ist nur durch die Darstellung
des Begriffs des Vernunftglaubens zu erhellen. Schließlich sind fast alle
Gründe, die wir kennen, entweder objektiv zureichende oder schlicht
unzureichende Gründe einer Überzeugungsbildung. Der Bereich möglicher subjektiv
zureichender und objektiv unzureichender Gründe ist auch nach
\name[Immanuel]{Kant} sehr begrenzt und es fragt sich (aus der Perspektive des
Philosophen, nicht des bloßen Interpreten), ob es solche Gründe tatsächlich
gibt. Sowohl die bisher erwähnten empirischen Erkenntnisse \emph{ex datis} als
auch die Vernunfterkenntnisse \emph{ex principiis} zählen zu den
Überzeugungen mit objektiv zureichender Grundlage.

\authorfullcite{Chignell:KantsConceptsofJustification2007} sagt, ein Grund sei
objektiv zureichend, wenn er die Wahrheit des Urteils hinreichend
\emph{wahrscheinlich} macht.\footnote{\enquote{A \ori{sufficient objective ground} is
one that renders the proposition in question moderately-to-highly likely}
\parencite[][42]{Chignell:KantsConceptsofJustification2007}. \enquote{A
\ori{sufficient objective ground} for assent to a proposition is one that
indicates to a moderate-to-high degree---though not always infallibly---that
the proposition is true} \parencite[][326]{Chignell:BeliefinKant2007}.} Dies
könne der Urteilende sicherlich nicht mit letzter Sicherheit ausmachen und darum
handle es sich um eine \emph{externalistische}
Position.\footnote{\cite[Vgl.][49]{Chignell:KantsConceptsofJustification2007}.}
Nach \authorcite{Chignell:KantsConceptsofJustification2007} können wir zwar
wissen, auf \emph{welchen} Gründen unser Urteil beruht, aber wir können nur
vermuten, ob sie objektiv zureichend sind.\footnote{\enquote{it is also worth
noting that despite the at bottom externalist character of the account, there is
an emphatic nod to internalist intuitions {\punkt}. Knowledge cannot merely be
based on sufficient objective grounds; rather, the subject must also be in a
position, on reflection, to cite those grounds, although she need not be
(and usually is not) able to determine that they are objectively
sufficient}
\parencite[][\pno~49\,f.]{Chignell:KantsConceptsofJustification2007}.} Ob ein
zureichender Grund objektiv zureichend ist, hänge also im Gegensatz zur Frage
nach subjektiv zureichenden Gründen doch von unserem \emph{epistemic luck} ab.

Dies ist aus mehreren Gründen unbefriedigend. Zum einen ist es unbefriedigend,
von einer externalistischen Position auszugehen, denn wenn wir eine Person für
ihre Überzeugungen und die Art ihrer Überzeugungsbildung verantwortlichen
machen, dann möchten wir auf Aspekte rekurrieren, die diese Person auch
überschauen und beeinflussen kann. Zum anderen konstituiert Wahrscheinlichkeit
kein Wissen. \enquote{Unter Wahrscheinlichkeit ist
ein Fürwahrhalten aus unzureichenden Gründen zu verstehen, die aber zu den
zureichenden ein größeres Verhältnis haben, als die Gründe des
Gegenteils.}\footnote{\cite[][A 126]{Kant:ImmanuelKantsLogik1977},
\cite[][IX: 81.23--26]{Kant:GesammelteWerke1900ff.}. Siehe dazu
\cite[][\nopp 2583]{Kant:Reflexionen1900ff.},
\cite[][XVI: 9--11]{Kant:GesammelteWerke1900ff.}.} Somit begründet
Wahrscheinlichkeit bloß Meinung, aber kein Wissen.

Wie im Falle der Überredung liegt bei der Meinung kein objektiv zureichender
Grund vor, aber der \singlequote{Meinende} lässt sich dadurch nicht täuschen,
sondern \emph{weiß}, dass er (noch) nicht über einen objektiven Grund verfügt,
und hält ihn daher auch subjektiv nicht für zureichend. Der Unterschied zwischen
legitimen und illegitimen vorläufigen Urteilen besteht primär darin, dass der
Urteilende bei legitimen vorläufigen Urteilen weiß, dass sie einer objektiven
Grundlage entbehren, und entsprechend mit ihnen verfährt.
Er wird keine Schlüsse aus ihnen ziehen beziehungsweise berücksichtigen, dass
alles, was er aus ihnen schließt, ebenfalls nicht objektiv begründet ist. Und er
wird sie niemandem als Wissen mitteilen, sondern sie als seine Vermutungen für
sich behalten oder sie zumindest explizit als Vermutungen kennzeichnen.

Allerdings bedarf es zumindest vereinzelter Anhaltspunkte, die für die Wahrheit
des Urteils sprechen; und dass diese Anhaltspunkte vorliegen, das muss wiederum
gewusst werden.\footnote{\enquote{Ich darf mich niemals unterwinden, zu \ori{meinen},
ohne wenigstens etwas zu \ori{wissen}, vermittelst dessen das an sich bloß
problematische Urteil eine Verknüpfung mit Wahrheit bekommt, die, ob sie
gleich nicht vollständig, doch mehr als willkürliche Erdichtung ist}
\mkbibparens{\cite[][B 850]{Kant:KritikderreinenVernunft2003},
\cite[][III: 533.9--12]{Kant:GesammelteWerke1900ff.}}.} Es mag sein, dass ich
nicht weiß, ob noch Schwarzbier im Kühlschrank ist. Ich erinnere mich vielleicht
nicht mehr genau, ob ich ein oder zwei Flaschen gekauft habe und wie viele ich
am Wochenende trank, und kann nicht ausschließen, dass sich mein Mitbewohner eine
nahm. Damit ich aber doch vermuten kann, dass noch Schwarzbier im Kühlschrank
ist, muss ich doch \emph{wissen}, dass ich Schwarzbier kaufte, dass ich die
Flaschen in den Kühlschrank lege und einiges mehr. Selbst wenn ich der Hoffnung,
es sei noch Schwarzbier im Kühlschrank, eine sehr geringe Wahrscheinlichkeit
beimesse, muss es doch Anhaltspunkte geben, die mehr als nur wahrscheinlich
sind, sonst ließe sich \emph{gar nicht} von Wahrscheinlichkeit sprechen.

\Revision[Pelletier]{Meinungen sind nicht generell abzulehnen. Zu ihnen gehören
nach traditioneller Auffassung auch
\emph{Hypothesen},\footnote{\Revision[Pelletier]{\cite[Vgl.][\S~606]{Wolff:PhilosophiarationalissiveLogica1740}.}}
worunter Meinungen verstanden werden, aus denen sich beobachtbare Geschehnisse
in der Welt erklären
lassen.\footnote{\Revision[Pelletier]{\cite[Vgl.][\S~181]{Meier:AuszugausderVernunftlehre1752},
\cite[][XVI: 461.24--29]{Kant:GesammelteWerke1900ff.}, sowie
\cite[][\S~126]{Wolff:Discursuspraeliminarisdephilosophiaingenere1996}.}}} Schon
\authorcite{Wolff:Discursuspraeliminarisdephilosophiaingenere1996} betont die
Bedeutung von vorläufigen Urteilen, legt aber zugleich Wert darauf, dass der
Status von \singlequote{Hypothesen} jederzeit als solcher berücksichtigt werden
muss.\footnote{\Revision[Pelletier]{\cite[Vgl.][\S\S~127--129]{Wolff:Discursuspraeliminarisdephilosophiaingenere1996},
sowie \cite[][\S\S~606--610]{Wolff:PhilosophiarationalissiveLogica1740}.}}
Sobald bloß vorläufige Urteile unter unseres Überzeugungen nicht mehr als solche
kenntlich sind, laufen wir Gefahr, sie zu Prämissen unserer Schlüsse zu machen
und damit die Zuverlässigkeit unseres gesamten Wissensgebäudes zu unterminieren.

Nach \name[Immanuel]{Kant} wiederum ist ein vorläufiges Urteil -- eine Meinung
-- ein Vorurteil, wenn es als Grundsatz unseres Schließens
gebraucht wird. Gerade daraus erhellt, warum Meinungen nicht Gegenstand von
Behauptungen sein dürfen: Wir verbreiten Vorurteile, wenn wir Meinungen im Modus
des Behauptens aussprechen und bezüglich einer Meinung \enquote{$p$} nicht
sagen, wir hielten es für möglich oder wahrscheinlich, dass $p$, sondern
schlicht behaupten, dass $p$. \Revision[Heidemann]{Alles, was Grundlage für
einen weiteren Erkenntniserwerb sein soll, muss gewusst werden, und zwar rationale Erkenntnisse
in rationaler und empirische Erkenntnisse in historischer Form. So lautet die
oberste Anweisung mündigen Denkens und Urteilens; wer dieser Anweisung nicht
nachkommt, der verfängt sich in Vorurteilen.}


\section{Von anderen
lernen}\label{subsubsection:EndlichesundUnendlichesErkennen}
Der wichtigste Unterschied zwischen Wissen und Meinen dürfte
unumstritten sein: Wissen kann und darf ich mitteilen, Meinungen soll ich für
mich behalten. Man könnte genauer sagen: Dann und nur dann wenn Person $A$
\emph{weiß}, dass $p$ (also \emph{objektiv gültige} Gründe für $p$ hat), dann
ist sie auch berechtigt, anderen mitzuteilen, dass $p$. Handelt es sich nicht um
Wissen, besitzt Person $A$ also keine objektiv zureichenden Gründe für $p$, dann
kann und darf sie natürlich auf ihre eigene Überzeugung verweisen. $A$ darf
sagen, dass sie vermutet, dass $p$, oder auch, sie sei sich sehr sicher, dass
$p$ (je nachdem, wie viele Anhaltspunkte sie dafür hat, dass $p$). Aber es
liegen Welten zwischen der einfachen Aussage, dass $p$, und der Aussage, man sei
sich sicher, dass $p$. \authorfullcite{Austin:OtherMinds1979} hat dies so
erläutert, dass er sagt, Behaupten sei wie das Geben von Versprechen; wenn ich
sage \enquote{Ich weiß, dass $p$}, dann \emph{verspreche} ich gleichsam die
Wahrheit von \enquote{$p$} und ein Anderer kann {$p$} auf \emph{meine}
Verantwortung hin glauben und weitersagen. Möchte ich diese Verantwortung nicht übernehmen, kann ich sagen,
dass ich \emph{mir sicher} bin (\name[Immanuel]{Kant} sagt -- im Kontext des
Glaubens -- \emph{ich} sei gewiss, statt \emph{es} sei
gewiss\footnote{\cite[Vgl.][B 857]{Kant:KritikderreinenVernunft2003},
\cite[][III: 537.1--2]{Kant:GesammelteWerke1900ff.}.}); der Andere mag es
immer noch auf meine Aussage hin glauben, aber er glaubt es dann auf
\emph{seine eigene}
Verantwortung.\footnote{\cite[Vgl.][98--103]{Austin:OtherMinds1979}.
\enquote{When I have said only that I am sure, and prove to have been mistaken,
I am not liable to be rounded on by others in the same way as when I have said
\enquote{I know}. I am sure \ori{for my part}, you can take it or leave it:
accept it if you think I'm an acute and careful person, that's your
responsibility} (\cite[][100]{Austin:OtherMinds1979}).}

Wer etwas behauptet und sagt, er wisse etwas, teilt nicht einfach mit, dass er
selbst die entsprechende wahre und adäquat begründete Überzeugung hat, sondern
übernimmt  Verantwortung für diese Überzeugung. Er \textit{verspricht}, eine
passende Begründung zu haben, \emph{verpflichtet} sich zur Angabe von Gründen
und gibt seinen Zuhörern die \textit{Berechtigung}, es \emph{auf seine
Verantwortung} hin zu wiederholen. \Revision[Heidemann]{Die Verantwortlichkeit
auf Seiten des Mitteilenden ist damit klar. Wie aber muss sich der Empfänger von
Informationen verhalten, um von dem Wissen anderer zu lernen und gleichzeitig
seiner eigenen Verantwortung für das \emph{eigene} Überzeugungssystem -- also
der \emph{ethics of belief} -- gerecht zu werden?}

\subsection{Gründe mitteilen}
Wir lernen von anderen, wenn sie uns ihr Wissen mitteilen. Das ist zumindest bei
empirischen Erkenntnissen für \name[Immanuel]{Kant} ganz unproblematisch. Wenn
Ingrid gesehen hat, dass etwas der Fall ist, und daraufhin Max mitteilt, dass es
der Fall ist, dann weiß Max hinterher genau so gut wie Ingrid, dass es der Fall
ist. Bei empirischen Erkenntnissen zählen eigene Wahrnehmungen und Mitteilungen
als objektive Gründe. Anders verhält es sich bei rationalen Erkenntnissen. Wenn
Ingrid aufgrund eigener Überlegungen \emph{weiß}, dass der Kompaktheitssatz aus
dem Endlichkeitssatz der Folgebeziehung folgt oder dass
\authorcite{Wolff:Discursuspraeliminarisdephilosophiaingenere1996}s Beweis der
Satzes vom zureichenden Grund aus dem Prinzip des ausgeschlossenen Widerspruchs
ungültig ist, dann hat sie objektiv zureichende Gründe für ihre Überzeugungen
und deswegen mathematisches beziehungsweise philosophisches Wissen. Wenn sie nun
Max mitteilt, dass der Kompaktheitssatz aus dem Endlichkeitssatz der
Folgebeziehung folgt und
\authorcite{Wolff:Discursuspraeliminarisdephilosophiaingenere1996}s Beweis
nicht gültig ist, dann zählt dies jedoch nicht als objektiv zureichend und Max
erwirbt kein Wissen (zumindest kein mathematisches und philosophisches,
sondern lediglich historisches). Kann Max nun dennoch von Ingrid lernen?

Wie schon bei \authorcite{Wolff:Psychologiaempirica1968} ist es auch im
Kontext der Systematik \name[Immanuel]{Kant}s nicht \emph{per se} verwerflich, historische
Erkenntnis der philosophischen Erkenntnis anderer zu
haben. Wenn wir uns mit der Geschichte der Philosophie befassen, dann sammeln
und systematisieren wir gerade solche Erkenntnisse. Damit verfügen wir natürlich
über philosophiehistorisches und nicht über philosophisches Wissen. In den
\titel{Prolegomena} schreibt \name[Immanuel]{Kant}: \enquote{Es gibt Gelehrte,
denen die Geschichte der Philosophie (der alten sowohl, als neuen) selbst ihre
Philosophie ist}\footnote{\cite[][A
3]{Kant:ProlegomenazueinerjedenkuenftigenMetaphysikdiealsWissenschaftwirdauftretenkoennen1977},
\cite[][IV: 255.5--6]{Kant:GesammelteWerke1900ff.}.}. Eine solche Haltung bleibt
hinter der Forderung der Aufklärung zurück. Zwar spricht nichts gegen den Erwerb
historischer Erkenntnisse, sie dürfen die rationalen Erkenntnisse jedoch nicht
als solche ersetzen; die Geschichte der Philosophie soll nicht mit der
Philosophie selbst verwechselt werden.

Unproblematisch ist natürlich das Lernen von anderen durch Kontrolle der
Vernünftigkeit eigener Gedanken. Angenommen, Max versucht, den Beweis des
Kompaktheitssatzes selbst zu führen und
\authorcite{Wolff:Discursuspraeliminarisdephilosophiaingenere1996}s Beweis
selbst zu widerlegen. Wenn er sich nun fragt, ob seine Gründe objektiv
zureichend sind, wendet er sich an Ingrid, die die Kompetenz besitzt, dies zu
entscheiden. Aber auf diese Weise würde er nie in größerem Umfang mathematisches
und philosophisches Wissen erwerben. Die Entdeckung solcher Beweise wie des
Kompektheitssatzes sind großartige Leistungen, die den wenigsten gelingen
können und die ein großes Maß an Originalität und Kreativität verlangen. Es geht
aber gerade nicht um Originalität; rationales Wissen kann ebenso derjenige
erlangen, der die Beweise von anderen lernt. Welchen Status mathematisches
Wissen hat, ob es sich um bloß historische Kenntnisse oder um Kompetenz und
Einsicht handelt, ist unabhängig davon, wie dieses Wissen ursprünglich erworben
wurde. Der Unterschied besteht in der aktuellen Kompetenz, nicht im vergangenen
Erwerb. Und auch bei philosophischem Wissen ist nicht der Ersterwerb
entscheidend, sondern die ausgebildete Kompetenz.

Max kann also Wissen erwerben, wenn Ingrid ihm nicht nur die Ergebnisse ihres
Nachdenkens mitteilt, sondern zugleich die (objektiv zureichenden) Gründe nennt,
die zeigen, dass ihre Überzeugungen wahr sind. Wenn sie Max die Beweise liefert
(und Max die Beweise versteht und nachvollzieht), dann erwirbt auch Max
entsprechendes Wissen. Und er mag den kategorischen Imperativ zunächst in
all seinen Formulierungen auswendig lernen, ohne ihn anwenden oder auch kritisch beurteilen
zu können. Später aber erwirbt er tieferes Verständnis und stellt fest, dass er
die Überzeugung, die er zunächst nur verbal reproduzieren konnte, weiterhin
teilt. Er ist dann selbst Philosoph geworden, wenngleich er die Überzeugungen,
die er hat, nicht selbst entdeckte. Somit kann eine Erkenntnis, die wir als
historische Erkenntnis erwerben, auch bei uns (subjektiv) zu einer eigenständig
philosophischen Erkenntnis werden.

Umgekehrt kann eine Überzeugung, die wir als eigenständige philosophische oder
mathematische Erkenntnis erworben haben, diesen Status auch wieder verlieren.
Denn mitunter verfügen wir nach einiger Zeit nicht mehr über die Kompetenz, die
Begründungen zu rekonstruieren. Auch wenn wir einst mathematische Theoreme
mitsamt ihrer Beweise gelernt haben sollten, muss uns der Beweis heute nicht
mehr verfügbar sein. Ingrid mag den Kompaktheitssatz selbst bewiesen haben. Aber
wenn sie sich nun nicht mehr an den
Beweis erinnert, hat sie dann ebenso lediglich historische Kenntnis wie Max?
\authorcite{Burge:ContentPreservation1993} nimmt an, dass eine objektiv
rationale Erkenntnis auch subjektiv rational ist, wenn jemandem die Gründe
nicht mehr einfallen.\footnote{\enquote{Although nothing in
\singlequote{Content Preservation} commits me on this matter, I believe that a
person clearly \ori{can be} entitled to believe a theorem she believes because
of preservative memory even if she cannot remember the proof she gave long ago,
and even if she cannot remember that she gave a proof. Most of what one is
entitled to believe from past reading, past interlocution, past reasoning, or
past empirical learning, derives from sources and warrants that one has
forgotten} \parencite[][38]{Burge:InterlocutionPerceptionandMemory1997}.}
\name[Immanuel]{Kant} scheint auf die gegenteilige Position festgelegt zu sein
und sagen zu müssen, dass nur subjektiv rationale Erkenntnis hat, wer die Gründe
dieser Erkenntnis zu nennen und zu explizieren vermag.

\subsection{Gründe und Kompetenzen}
Interessant ist hier  insbesondere folgende Überlegung zum Status der
Mathematik als Wissenschaft gegenüber der Philosophie: Beide sind keine
empirischen Erkenntnisse, also kein tradierbares Wissen, von dem es legitim ist,
bloß historisches Wissen zu haben. Einen beachtlichen Vorteil gegenüber der
Philosophie habe die Mathematik aber, insofern sie leichter \emph{lehr-} und
\emph{lernbar} sei und es bei ihr nicht -- oder nicht so oft -- den bedenklichen
Fall einer subjektiv bloß historischen Erkenntnis objektiv rationaler -- hier:
mathematischer -- Wahrheiten gebe. \Revision[Heidemann]{Es ist also im Bereich
der Mathematik leichter, Mündigkeit zu erwerben; mündiges mathematisches Urteilen wird durch
den Unterricht der Mathematik begünstigt.}

Nun sei es bei mathematischem Wissen viel leichter als bei philosophischen
Erkenntnissen, sie so mitzuteilen, dass der Rezipient tatsächlich rationales
Wissen erwirbt und nicht lediglich historische Kenntnis:
\begin{quote}
 Es ist aber doch sonderbar, daß das mathematische Erkenntnis, so wie man es
erlernet hat, doch auch subjektiv für Vernunfterkenntnis gelten kann {\punkt}.
Die Ursache ist, weil die Erkenntnisquellen, aus denen der Lehrer allein schöpfen
kann, nirgend anders als in den wesentlichen und echten Prinzipien der Vernunft
liegen, und mithin von dem Lehrlinge nirgend anders hergenommen, noch etwa
gestritten werden können, und dieses zwar darum, weil der Gebrauch der Vernunft
hier nur in concreto, obzwar dennoch a priori, nämlich an der reinen, und eben
deswegen fehlerfreien, Anschauung geschieht, und alle Täuschung und Irrtum
ausschließt.\footnote{\cite[][B~865]{Kant:KritikderreinenVernunft2003}, \cite[][III:
541.24--34]{Kant:GesammelteWerke1900ff.}.}
\end{quote}
Es gebe in der Mathematik keine anderen Erkenntnisquellen, auf die wir
verfallen können, als die Vernunft. Jedes Argument in der Mathematik sei
entweder objektiv gültig oder offensichtlich unzureichend.
Unmöglich hingegen sei es, dass andere Quellen sich unbemerkt einschleichen, die
Mathematik ist also nicht anfällig für Vorurteile. Aber warum ist sie nicht
anfällig für Vorurteile? Dies liegt -- so sagt \name[Immanuel]{Kant} -- daran,
dass die Vernunft zwar \emph{a priori} (wie die Philosophie), aber \emph{in
concreto} gebraucht werde. Und dies wiederum schließe Täuschung und Irrtum aus.
Wir müssen also auf den \emph{formalen} Unterschied zwischen Mathematik und
Philosophie achten, d.\,i. auf den Unterschied in der Art und Weise, wie
entsprechende Erkenntnisse generiert werden. Gerade dies nennt
\name[Immanuel]{Kant} ja die \emph{Form} einer Erkenntnis (siehe Kapitel
\ref{section:MuendigkeitundPhilosophie}).



Der Mathematik kommt in \name[Immanuel]{Kant}s Darstellung gegenüber der
Philosophie insofern ein besonderer Status zu, als er sie im expliziten Kontrast
zur Philosophie als Wissenschaft bezeichnet, obwohl sie keine Erfahrungswissenschaft ist.
Sie sei \enquote{von den frühesten Zeiten her {\punkt} den sichern Weg einer
Wissenschaft gegangen}\footnote{\cite[][B x]{Kant:KritikderreinenVernunft2003},
\cite[][III: 9.7--9]{Kant:GesammelteWerke1900ff.}.} und wegen dieser
Erfolgsgeschichte einer reinen Vernunftwissenschaft
mögliches Vorbild auf dem Weg zur Wissenschaftlichkeit auch der
Philosophie.\footnote{\cite[Vgl.][\S\S~5\,f.,]{Kant:ProlegomenazueinerjedenkuenftigenMetaphysikdiealsWissenschaftwirdauftretenkoennen1977}
\cite[][IV: 279.15--280.32]{Kant:GesammelteWerke1900ff.}, sowie
\cite[][B 14--17]{Kant:KritikderreinenVernunft2003},
\cite[][III: 36.14--38.24]{Kant:GesammelteWerke1900ff.}.} Damit referiert er
freilich die Hoffnung vieler Autoren der Neuzeit, die Wissenschaftlichkeit der
Philosophie durch Orientierung an der methodischen Strenge der Mathematik zu
gewährleisten.
\authorcite{Wolff:Discursuspraeliminarisdephilosophiaingenere1996} ist dabei
derjenige, den \name[Immanuel]{Kant} stets vor Augen hatte (siehe Kapitel
\ref{subsection:SelbstdenkenbeiKant}). Doch verwirft er gerade diese Hoffnung
und ersetzt sie durch die Forderung, die Vernunft gerade in Fragen der
Philosophie durch Intersubjektivität zu fundieren (Kapitel
\ref{section:sensuscommunis}).

Der primäre Grund für die Lehrbarkeit der Mathematik liegt nicht darin,
dass man nichts Falsches glaubt, weil die Mathematik eine mit besonderer
Gewissheit vorgehende Disziplin ist, sondern darin, dass die Grundlagen in den
\enquote{echten Prinzipien der Vernunft} liegen, die dann auch die je eigenen
sind. Das bloße Lernen einer Formel zur Berechnung ohne die entsprechenden
Herleitungen führt jedenfalls nach \name[Immanuel]{Kant} nicht zum adäquaten
Erlernen der Mathematik -- trotz ihrer Sicherheit, die dann möglicherweise
größer sein mag, als wenn ich selbst die Herleitung vorgenommen
habe.\footnote{\cite[Vgl.][A 105]{Kant:ImmanuelKantsLogik1977}, \cite[][IX:
69.8--10]{Kant:GesammelteWerke1900ff.}: \enquote{Mathematische
Vernunftwahrheiten kann man auf Zeugnisse zwar glauben, weil Irrtum hier teils
nicht leicht möglich ist, teils auch leicht entdeckt werden kann; aber man kann
sie auf diese Art doch nicht wissen.} Siehe auch
\cite[][\nopp 1631]{Kant:Reflexionen1900ff.},
\cite[][XVI: 52.2--3]{Kant:GesammelteWerke1900ff.}, sowie
\cite[][\nopp 2471]{Kant:Reflexionen1900ff.},
\cite[][XVI: 384.8--10]{Kant:GesammelteWerke1900ff.}.}


Dass Mathematik leichter lehrbar ist, liegt darin begründet, dass ein Unterricht
der Mathematik so eingerichtet sein könne und müsse, dass die lernende Teilnahme
darin gerade das selbständige Mitdenken der rationalen Inhalte sei. Wer
Mathematik lernt, memoriert nicht vorgefertigte Lösungswege, um sie im richtigen
Moment zu reproduzieren, sondern übt mit fortlaufend anderen, aber
ähnlichen Aufgaben, Lösungen selbst zu finden. Es ist natürlich nicht
garantiert, dass ein Unterricht der Mathematik ihrer Natur als rationaler
Wissenschaft adäquat ist, aber es sei doch leichter möglich, als in der
Philosophie. Ein Philosophieunterricht mag -- unter ungünstigen, aber häufigen
Umständen -- so eingerichtet sein, dass der Schüler bedeutende Argumente aus der
Philosophiegeschichte lernt, aber keine Methodenkompetenz erwirbt, selbständig
zu philosophieren. Aber ein Mathematikunterricht nimmt nicht die Form an, dass
ein Schüler nur Beweise auswendig lernt. Tut er dies doch, so erwirbt der
Schüler nur ein historisches Wissen von einem mathematischen Tatbestand.

Die Methodenorientierung ist freilich kein Alleinstellungsmerkmal der
Mathematik, sondern wünschenswert in \emph{jeder} wissenschaftlichen Disziplin,
auch den empirischen Wissenschaften. Der Physiker
soll die Gesetze der Physik nicht \emph{auswendig} kennen, sondern sie anwenden,
ihre Begründung nennen und sie bei widerstreitender Erfahrung entsprechend
modifizieren können. Er wird zum Forscher und Anwender ausgebildet, nicht zur
Formelsammlung.\footnote{Dasselbe gilt auch noch mit gewissen Einschränkungen
für die Vertreter der oberen Fakultäten, also Juristen, Mediziner und Theologen.
Auch sie sollen zur Anwendung ihrer Erkenntnisse ausgebildet werden, wenngleich
hier der Forschungsaspekt deutlich zurücktritt. Generell steht bei den oberen
Fakultäten der Anwendungs- und bei den Fächern der unteren Fakultät
(Naturwissenschaft, Mathematik, Philosophie) der Forschungsaspekt im
Vordergrund. \kantcite{Kant:DerStreitderFakultaeten1977}{Vgl.}{A 6--10}{VII:
18.30--20.11}.} Im Unterschied zur Mathematik kann es dabei in
empirischen Wissenschaften auch vorkommen, dass Inhalte gelehrt
werden, die schlicht als testimoniales Wissen gelernt werden,
weil sie auf Erfahrungen beruhen, die nicht jeder selbst machen kann oder muss.
In der Mathematik kommt solches empirisches Wissen wie in der (reinen)
Philosophie kaum\footnote{In beiden Fällen beinhaltet die Ausbildung notwendiger
Weise auch historisches Wissen um die Geschichte des eigenen Faches. Dieses mag
umfangreicher (Philosophie) oder geringer (Mathematik) ausfallen, aber
letztlich muss doch selbst der Mathematiker wenigstens die jüngere
Vergangenheit seines Faches zur Kenntnis nehmen, um auch nur relevante Fragen
der Forschung ausmachen zu können.} vor, weswegen der Unterricht in Mathematik
wie in Philosophie idealiter kein testimoniales Wissen generiert. Nur generiert
die Mathematik tatsächlich auch im Unterricht subjektiv rationales Wissen,
während die Philosophie dazu neigt, sich in historischer Erkenntnis zu
verfangen.

Wieso gelingt der Philosophie nicht, was in der Mathematik so
selbstverständlich ist? Der Grund dafür ist, dass
die Mathematik sich nach \name[Immanuel]{Kant} leichter auf eine gemeinsame
Grundlage beziehen kann, da in ihr Definitionen und Axiome und damit Beweise
möglich sind, die jeder entsprechend ausgebildete Zuhörer selbst nachvollziehen und kontrollieren
kann.\footnote{\cite[][B 754--766, insb. B
754\,f.,]{Kant:KritikderreinenVernunft2003} \cite[][III: 477.5--483.32, insb.\
477.5--8]{Kant:GesammelteWerke1900ff.}: \enquote{Die Gründlichkeit der Mathematik beruht auf Definitionen, Axiomen, Demonstrationen. Ich werde mcih
damit begnügen, zu zeigen: daß keines dieser Stücke in dem Sinne, darin sie der
Mathematiker nimmt, von der Philosophie könne geleistet, noch nachgeahmet
werden.}} Entscheidend ist, dass es seiner Darstellung nach nicht auf eine
vermeintliche Korrekturresistenz ankommt, sondern darauf, dass jeder ihre
Wahrheit unmittelbar \emph{selbst} einsehen könne. Und dies liege wiederum -- im
Falle der Axiome und Postulate -- an der Möglichkeit einer Darstellung in der
reinen Anschauung. Dass die Lehrbarkeit der Philosophie nicht in derselben Weise
gegeben ist, liegt also an der Unverfügbarkeit dessen, was \name[Immanuel]{Kant}
intellektuelle Anschauung nennt.

Philosophische Erkenntnis haben wir nur als \emph{diskursive} rationale
Erkenntnis; und in dieser Diskursivität unseres Erkennens besteht gerade die
Eigenart des endlichen Denkens.
Verfügten wir in der Philosophie über die Möglichkeit eines
\emph{intuitiven} rationalen Erkennens analog der reinen Anschauung in der
Mathematik, so wäre unser Denken
nicht endlich. Nun basiert \name[Immanuel]{Kant}s Definition mathematischer
Erkenntnis damit natürlich auf der in der transzendentalen Ästhetik der \titel{Kritik der reinen Vernunft}
entfalteten Theorie mathematischen Wissens, die heute kaum jemand ohne weiteres
zu akzeptieren bereit ist. In dem hier zu verhandelnden Zusammenhang
interessiert die mathematische Erkenntnis jedoch nur als Kontrastfolie, die die
Eigenart philosophischer Erkenntnis zu verdeutlichen helfen soll.\footnote{Wir
können dabei die Frage, ob \name[Immanuel]{Kant} eine überzeugende Theorie
mathematischen Erkennens anbietet, daher außer Acht lassen. Ob es so etwas wie
eine reine Anschauung (von Raum und Zeit) gibt und was unter einer solchen
Anschauung zu verstehen ist, das braucht uns nicht zu beschäftigen. Denn
philosophisches Wissen kann auf eine solche Anschauung nach
\name[Immanuel]{Kant} nicht zurückgreifen. Und nur dies interessiert; die reine
Anschauung der Mathematik könnten wir ebenso als bloß gedachten Kontrast zum
selben Zweck verwenden.}

Die Philosophie sei in dem Zustand, den \name[Immanuel]{Kant} vorzufinden
behauptet, keine Wissenschaft, weil ihr die gemeinsame Basis fehlt, so dass
philosophische Argumente auf Prämissen beruhen, die von der Anerkennung durch
uns und unsere Gesprächspartner abhängen. Sie verfährt daher viel stärker \emph{ad
hominem}, denn die Überlegungen, die auf unsicheren Axiomen aufbauen, sind nur
für denjenigen überzeugend, der die entsprechenden Grundüberzeugungen teilt. Der
Streit philosophischer Schulen und schließlich die Gegenreaktion des
Skeptizismus sind \name[Immanuel]{Kant}s Belege für diese Diagnose.

Wenn wir die Grundlagen philosophischen Vorgehens jedoch nicht nachvollziehen
können, erscheint uns jede Prämisse, die nicht weiter begründet wird, als
\singlequote{innere Eingebung}. Wollte Max von Ingrid lernen, und
verwendete Ingrid in ihren Argumentationen Prämissen, deren Gültigkeit Max nicht
selbst kontrollieren kann, so bliebe ihm nichts anderes übrig, als Ingrids Wort
als oberstes Kriterium der Wahrheit anzusehen. Die Tatsache, dass er nicht
selbst die Gültigkeit der Behauptungen kontrollieren kann, bedingt damit seine
Abhängigkeit von Ingrids Autorität und damit seine eigene Unmündigkeit. Das war
\authorcite{Wolff:Discursuspraeliminarisdephilosophiaingenere1996}s wichtige
Einsicht.\footnote{Siehe Kapitel
\ref{Abschnitt:WolffunddieWissenschaftlichkeitderPhilosophiemoregeometrico}.}
Daraufhin müssten wiederum -- sollte Ingrid schulbildend wirken --
\enquote{aus inneren Eingebungen durch Zeugnisse äußere bewährte Facta, aus
Traditionen, die anfänglich selbst gewählt waren, mit der Zeit aufgedrungene
Urkunden, mit einem Worte die gänzliche Unterwerfung der Vernunft unter Facta,
d.\,i. der \ori{Aberglaube} entspringen}\footnote{\cite[][A 327]{Kant:Washeisst:SichimDenkenorientieren?1977}, \cite[][VIII:
145.30--34]{Kant:GesammelteWerke1900ff.}.}. Das Problem der unzureichenden
Fundierung für die Lehrbarkeit der Philosophie besteht also nicht in der Gefahr,
falsche Sätze für wahr zu halten, sondern darin, dass es letzte Sätze gibt, die
nur durch die Autorität des Lehrers bezeugt sind. Es ist also die Unfähigkeit
des Lehrers, Gründe für alle seine Behauptungen anzugeben, die die Lehr- und
Lernbarkeit der Philosophie unterminieren.

Gerade darin liegt aber die Schwierigkeit in dem Bestreben, Philosophie als
auch subjektiv rationales Wissen zu erwerben. Weil sich die echten Quellen der
Vernunft nicht so leicht identifizieren lassen, verfallen wir schnell der
Versuchung, grundlegende Urteile (Prinzipien oder \singlequote{Grundsätze}) von
Autoritäten zu übernehmen.
Denn -- so lehrte uns schon \authorcite{Wolff:Psychologiaempirica1968}\footnote{Siehe oben auf
S.~\pageref{Stellenverweis:Wolff:SelbstaendigkeitnurdurchKompetenz}.} --  nur dadurch, dass wir
selbst zu fundierten, vernünftigen Einsichten gelangen, können wir vermeiden,
unser Fürwahrhalten an zufälligen Einflüssen statt an der Vernunft auszurichten.
Der bloße Anspruch auf Selbständigkeit führt nur zu einem Schein des
Selbstdenkens, \Revision[Heidemann]{zur Verwendung historischer Kenntnisse
philosophischer Urteile als Grundsätze des weiteren Erkenntniserwerbs. Die rationalen Grundlagen unseres
Denkens selbst zu verantworten macht aber gerade Aufklärung und Mündigkeit aus.
Solche Grundlagen selbst verantworten zu können heißt wiederum, über
entsprechende methodische Kompetenzen in der Philosophie zu verfügen.
Intellektuelle Tugendhaftigkeit äußert sich im Bemühen um den Erwerb
philosophischer Kompetenzen. Der bequemere Weg bestünde freilich darin, einfach
die Grundsätze anderer unkontrolliert zu übernehmen. So mag beispielsweise auch
derjenige, der von
\authorcite{Wolff:Discursuspraeliminarisdephilosophiaingenere1996},
\authorcite{Meier:AuszugausderVernunftlehre1752} und
\authorcite{Reimarus:DieVernunftlehrealseineAnweisungzumrichtigenGebrauchderVernunftinderErkenntnisderWahrheit1756}
die richtigen Grundsätze des Umgangs mit testimonialem Wissen erlernt hat, in
der Folge korrekt und kritisch urteilen. Aber aufgeklärt ist er erst, wenn er
über diese Grundsätze selbst kompetent urteilen und diskutieren kann, statt
sich nur auf die genannten Autoren oder eine \singlequote{\emph{best practice}}
zu berufen.}

Beiden -- Philosophie und Mathematik -- geht es letztlich nicht um den Erwerb
von Tatsachenwissen, sondern von philosophischen und mathematischen Kompetenzen.
Derjenige, der beispielsweise die Axiome der Klassischen Aussagenlogik bloß
benennen, aber nicht anwenden, geschweige denn ihre Korrektheit beweisen kann,
hat gar kein mathematisches Wissen. Und ebenso hat derjenige kein
philosophisches Wissen, der die verschiedenen Formulierungen des kategorischen
Imperativs zwar auswendig aufsagen, sie aber weder anwenden, noch vernünftig
darüber diskutieren kann, ob es sich um einen guten Vorschlag zur Grundlegung
der Ethik handelt. \Revision[Heidemann]{Und das ist eben der Kern der Auflösung
des vermeintlichen Widerspruchs zwischen der Forderung nach Selbstdenken und
unserer Abhängigkeit von anderen. \name[Immanuel]{Kant}s \emph{ethics of
belief} fordert nicht, von Anderen epistemisch und kognitiv unabhängig zu
werden. Sie fordert aber im Bereich rationaler Erkenntnisse die eigenständige
vernünftige Kontrolle über die Korrektheit unserer Überzeugungen zu erwerben.
Der Mathematik gelinge dies besser als der Philosophie, weil sie über eine
feste Grundlage verfüge und sich auf dieser Grundlage mathematische Kompetenzen
leichter vermitteln ließen, zumal diese in der Fähigkeit zur Befolgung einer
bestimmten Methode bestünden. Aufgrund des unterschiedlichen Charakters der
jeweiligen epistemischen Grundlagen könne die Mathematik aber nur scheinbar als
Vorbild für philosophisches Vorgehen dienen.}



\begin{comment}
\subsection{Vernunft und Gedächtnis}

\authorfullcite{Kripke:NameundNotwendigkeit1981} hält \name[Immanuel]{Kant}s
Unterscheidung von rationalen und empirischen Erkenntnissen für problematisch,
weil es doch sein könnte, dass wir von rationalen Erkenntnissen empirisches Wissen
erwerben. Wir nutzen beispielsweise einen Taschenrechner, um eine Rechnung
auszuführen, oder einen Computer, um zu ermittelt, ob eine bestimmte Zahl $k$
eine Primzahl ist. Damit beruht unser Wissen, dass $k$ eine Primzahl ist, jedoch
nicht auf rein rationalen Überlegungen, sondern auf empirischen Erkenntnissen
über das Funktionieren des
Computers.\footnote{\authorcite{Kripke:NameundNotwendigkeit1981} behauptet
explizit: \enquote{Wenn wir glauben, daß die Zahl eine Primzahl ist, glauben
wir es also aufgrund unseres Wissens von den Gesetzen der Physik, der
Konstruktion des Computers usw.}
\parencite[][45]{Kripke:NameundNotwendigkeit1981}.
\authorfullcite{Tymoczko:TheFour-ColorProblemanditsPhilosophicalSignificance1979}
schreibt: \enquote{The most natural interpretation {\punkt} is that
computer-assisted proofs introduce experimental methods into pure mathematics}
\parencite[][58]{Tymoczko:TheFour-ColorProblemanditsPhilosophicalSignificance1979}.
\authorfullcite{Burge:ContentPreservation1993} wiederum widerspricht und
behauptet, dass die Benutzung eines Computers nicht als Einsatz eines
empirischen Verfahrens zu verstehen sei, sondern als Erweiterung der eigenen rationalen Fähigkeiten. Die
Arbeitsweise eines solchen Instruments entspringe den mathematischen Kompetenzen
des Entwicklers und sei daher als Ausdruck vernünftiger Tätigkeit anzusehen
\parencite[vgl.][31]{Burge:ComputerProofAprioriKnowledgeandOtherMinds:TheSixthPhilosophicalPerspectivesLecture1998}.}
Wir sollten daher vielleicht gar nicht davon sprechen, ob eine Erkenntnis
\emph{a priori} ist, sondern davon, ob eine bestimmte Person etwas
\emph{a priori} oder \emph{a posteriori}
erkennt.\footcite[Vgl.][\pno~44\,f.]{Kripke:NameundNotwendigkeit1981}

Wir haben gesehen, dass \name[Immanuel]{Kant} beide Verwendungen kennt und
unterscheidet.

Nun wurde der Erwerb mathematischen Wissens durch Computer in den letzten
Jahrzehnten zu einem viel diskutierten Problem der Philosophie der Mathematik,
von dem \name[Immanuel]{Kant} freilich noch nichts wissen konnte. Die
Möglichkeit, mathematische Beweise von Computern führen zu lassen, bringt jedoch
eine Herausforderung an die hier dargestellt Position mit sich, deren Behandlung
durchaus erhellend ist.

Mathematische Beweise gehen in aller Regel schrittweise vor. Mitunter beweisen
wir erst einige Hilfssätze, bevor wir uns dem Beweis des eigentlichen Theorems
zuwenden. Und wenn das Theorem bewiesen ist, ist es häufig schwer, den gesamten
Beweis zu überblicken. Wir erinnern uns aber an die Ergebnisse der letzten
Beweisschritte und knüpfen an diese an.
\authorcite{Locke:TheWorksofJohnLocke1963} sagt daher, dass ein
Beweis, der in mehreren Schritten vorgeht, zweifelhafter und irrtumsanfälliger
sei, als eine \singlequote{intuitive} mathematische Gewissheit, also ein
einzelner Beweisschritt.

Nun argumentiert \authorfullcite{Chisholm:Erkenntnistheorie1979}, dass wir uns
bei der Rechtfertigung einer Behauptung mittels eines Beweises, der über mehrere
Beweisschritte geht, auch auf empirische Annahmen über unser Gedächtnis stützen
müssen und deswegen nicht sagen sollten, dass wir die Behauptung \emph{a priori}
gerechtfertigt
hätten.\footcite[Vgl.][\pno~72\,f.]{Chisholm:Erkenntnistheorie1979}
Nach \authorcite{Chisholm:Erkenntnistheorie1979} ist das Wissen um eine Tatsache
mittels der Erinnerung offenbar ein Fall inferenziellen Wissens: Wenn ich weiß,
dass ich den Herd ausgeschaltet habe, dann weiß ich zunächst, dass meine
Erinnerung mir sagt, ich hätte ihn ausgeschaltet. Mittels der Prämisse, dass
meine Erinnerung mir meistens Dinge sagt, die tatsächlich getan oder
wahrgenommen habe, \emph{schließe} ich dann darauf, dass ich den Herd
ausschaltete. Die Prämissen sind offenbar empirisch. Und wenn ich mich erinnere,
den Kompaktheitssatz bewiesen zu haben, dann verfahre ich analog und weiß
\emph{nun empirisch}, dass eine Aussagenmenge erfüllbar ist, wenn jede ihrer
endlichen Teilmengen erfüllbar ist. Vorhin -- als ich ihn bewies -- mag ich dies
\emph{a priori} gewusst haben. Nun weiß ich ihn (in \name[Immanuel]{Kant}s
Terminologie) bloß noch historisch.

Diese Konsequenz ist freilich nicht tragbar -- schon gar nicht aus
\name[Immanuel]{Kant}ischer Perspektive --, denn sie vereitelte jede
Möglichkeit, in nennenswertem Umfang rationales Wissen zu erlangen. Für
\authorfullcite{Hume:AnEnquiryConcerningHumanUnderstanding1964} ist die
Erinnerung neben der momentanen Wahrnehmung eine völlig unproblematische
Wissensquelle.\footnote{Die Leitfrage des \titel{Enquiry Concerning Human
Understanding} lautet entsprechend, wie gelangen wir zu fundierter Erkenntnis,
die über das momentane Zeugnis der Sinne und unser Gedächtnis hinausgeht:
\enquote{It may,therefore, be a subject worthy of curiosity, to enquire what is
the nature of that evidence, which assures us of any real existence and matter
of fact, beyond the present testimony of our senses, or the records of our
memory} \parencite[][23]{Hume:AnEnquiryConcerningHumanUnderstanding1964}.} Es
erscheint sogar geboten, sie gar nicht als genuine Wissens\emph{quelle} zu
bezeichnen, denn sie generiert kein Wissen, sondern bewahrt lediglich dasjenige
Wissen, welches wir zuvor erlangt hatten. \authorcite{Burge:ContentPreservation1993} bezeichnet die
Erinnerung als \enquote{content preserving}.
\footcite[Vgl.][]{Burge:ContentPreservation1993} Das Gedächtnis sei wie
ein logischer Schluss, insofern es das Fortschreiten der Vernunft ermöglicht,
ohne propositionalen Gehalt beizutragen. Es unterscheide sich von einem Schluss
darin, dass es kein Übergang und keine Bewegung ist, also auch kein Element
einer Begründung darstellt.

\authorfullcite{Burge:ContentPreservation1993} schlägt nun vor, die Weitergabe
einer Erkenntnis von einer Person zu einer anderen nicht in Analogie zu eigener
Wahrnehmung, sondern in Analogie zu der eigenen Erinnerung zu verstehen. Wenn
Ingrid gestern gesehen hat, dass Paul in der Stadt war, und sich heute daran
erinnert, dann bewahrt diese Erinnerung sowohl den Gehalt ihrer Überzeugung
(\enquote{Paul war in der Stadt}) als auch ihren Status als empirisches Wissen.
Wenn sie nun Max ihre Überzeugung mitteilt, dann geschehe dasselbe: Es werde
sowohl der Gehalt als auch der Status bewahrt. Die Mitteilung funktioniert also
wie ein Gedächtnis, das sich nicht auf eine Person beschränkt (und wir
sprechen ja auch von Büchern als dem Gedächtnis der Menschheit).

Und in der Tat unterscheiden wir nicht die Fälle, in denen Ingrid sich einen
Schritt in einem Beweis nur in ihr Gedächtnis einprägt, von solchen, in denen
sie sich die Zwischenergebnisse notiert. Es scheint absurd zu sein, darin einen
relevanten Unterschied zu sehen. Und warum sollte es dann relevant sein, wenn
nun nicht Ingrid, sondern Max sich an ihre Aufzeichnungen setzt und den Beweis
fortführt?

Wenn Ingrid gestern den Kompaktheitssatz bewiesen hat und sich heute daran
erinnert, dann verfügt sie über mathematisches Wissen. Es wäre absurd zu
erwarten, dass sie den Beweis in Gedanken noch einmal durchgehen müsste, um auch
heute mathematisches Wissen zu haben. In jedem längeren Beweis müssen wir uns
darauf verlassen, dass wir uns an frühere Beweisschritte korrekt erinnern. Wenn
das Ergebnis, das Ingrid sich vorhin notierte, jetzt nur noch als historische
Kenntnis zählt, dann wird niemand mathematisches Wissen erlangen. Doch wenn
Ingrids Wissen, das sich auf ihre Erinnerungen oder ihre gestrigen
Aufzeichnungen stützt, auch heute als rationales Wissen zählt, dann ist nicht
einzusehen, warum Max' Wissen, das dieser ebenso aus diesen Aufzeichnungen haben
kann, nicht als rational, sondern als historisch zählt.

Wir stehen damit vor einem Dilemma: Entweder wir sagen, das Aufschreiben von
Wahrheiten sei wie das Gedächtnis \singlequote{\emph{content preserving}}, dann
sind mitgeteilte mathematische Erkenntnis selbst subjektiv mathematisch (und
nicht historisch); oder wir bestehen darauf, dass eine mathematische Erkenntnis,
die ich gelesen habe, bloß historisch ist, dann wissen wir alle Ergebnisse
längerer Beweise bloß historisch.

\authorfullcite{Burge:ContentPreservation1993} entscheidet sich für die erste
Option und sagt, wenn der Rezipient eine Überzeugung auf eine Rechtfertigung
aufbaut, in die empirische Elemente als Prämissen eingehen, so könne seine Überzeugung nicht \emph{a priori} begründet sein.
Wenn jedoch ausschließlich \emph{a priori} begründete Prämissen eingehen, so
habe der Rezipient auch dann Wissen \emph{a priori}, wenn er von der Mitteilung eines anderen
abhängt.\footcite[Vgl.][\pno~486\,f.]{Burge:ContentPreservation1993}
\name[Immanuel]{Kant} kann sich nicht für diese Option entscheiden, ohne seine
gesamte Konzeption von historischen, empirischen und rationalen Erkenntnissen zu
gefährden. Er kann aber auch nicht die andere Option wählen, denn dann wäre die
Mathematik keine rationale Wissenschaft mehr.

Eine Lösung im Sinne \name[Immanuel]{Kant}s scheint mir jedoch naheliegend zu
sein: Ob eine Erkenntnis subjektiv historisch oder rational ist, hängt nicht
davon ab, wie sie ursprünglich erworben wurde, sondern davon, ob die Person über die Fähigkeit
verfügt, die Gründe der Erkenntnis zu benennen, zu verstehen und zu bewerten.
Wenn Max durch Ingrids Auskunft in die Lage versetzt wird, den Beweis des
Kompaktheitssatzes zu führen und zu erläutern, dann verfügt er über
mathematisches Wissen. Kann er ihn nur zitieren, dann ist sein Wissen bloß
historisch. Und dasselbe betrifft auch Ingrids Erinnerung: Nur weil sie
weiterhin ihre mathematischen Fähigkeiten besitzt, zählt ihr Wissen als
mathematisch. Kämen ihr plötzlich die Kompetenzen abhanden und verfügte sie nur
noch über die Erinnerung, dann wäre auch ihr Wissen bloß noch historisch. Dies
kongruiert auch mit \name[Immanuel]{Kant}s Darstellung des unmündigen
\authorcite{Wolff:Discursuspraeliminarisdephilosophiaingenere1996}ianers in der
\KapitelTitel{Architektonik der reinen Vernunft}. Dessen Unselbständigkeit zeigt
sich nicht darin, woher er seine Definitionen ursprünglich hatte, sondern darin,
dass er nicht die Kompetenz besitzt, selbst Definitionen
aufzustellen.\footnote{\cite[Vgl.][B 864]{Kant:KritikderreinenVernunft2003},
\cite[][III: 541.6--7]{Kant:GesammelteWerke1900ff.}.}
Auch damit knüpft er an
\authorcite{Wolff:Discursuspraeliminarisdephilosophiaingenere1996}s Überlegungen
zum Unterschied von \emph{cognitio philosophica} und \emph{cognitionis
philosophicae cognitio historica} an. Der Unterschied zwischen beiden liegt
nicht im ursprünglichen Erwerb, sondern in der aktuellen Kompetenz.

Es ist dann letztlich das Kriterium der Relevanz, für das uns die Überlegungen
aus Kapitel \ref{chapter:AufklaerungundWissenschaft} als Leitfaden dienen
können, welches anzeigt, in welchen Themenbereichen es nicht weiter schädlich
ist, historische Kenntnis rationaler Erkenntnisse zu haben.\footnote{\enquote{Bei einigen rationalen Erkenntnissen ist es schädlich, sie bloß historisch zu
wissen, bei andern hingegen ist dieses gleichgültig. So weiß z.\,B. der Schiffer
die Regeln der Schifffahrt historisch aus seinen Tabellen; und das ist für ihn
genug. Wenn aber der Rechtsgelehrte die Rechtsgelehrsamkeit bloß historisch
weiß: so ist er zum echten Richter und noch mehr zum Gesetzgeber völlig
verdorben} \mkbibparens{\cite[][A 21]{Kant:ImmanuelKantsLogik1977},
\cite[][IX: 22.21--26]{Kant:GesammelteWerke1900ff.}}.}
In der Mathematik wird es für die meisten Menschen unproblematisch sein. Wenn wir auf den Beweis
des Satzes des \singlename{Pythagoras} angesprochen werden, mögen wir sagen:
\enquote{Ich konnte ihn mal, aber ich habe vergessen, wie er ging.} Unser
ursprünglich rationales Wissen ist zur bloßen historischen Kenntnis geworden.
Dies ist aber auch nicht schlimm. Die Zuverlässigkeit mathematischer Erkenntnis
ist groß genug und in der Anwendung schadet es in der Regel nicht, die Beweise
nicht zu kennen. Anders verhält es sich in Fragen der Moral und Ethik unter
Einschluss politischer Fragen. Dort müssen uns die Gründe weiterhin verfügbar
sein, wenn wir mündig sein wollen.
\end{comment}


\section{Handeln auf epistemisch unzureichender
Grundlage}\label{section:HandelnAufEpistemischDuennerGrundlage}
Bis hierher fügt sich \name[Immanuel]{Kant}s \emph{ethics of belief} in seine
Aufklärungskonzeption, zumal sie die Überlegungen zu testimonialem Wissen aus
den Kapitel \ref{section:autonomieunddaszeugnisanderer} und
\ref{subsection:BewertungvonInformationennachihrerART} zu integrieren erlaubt
und eine eindeutig evidentialistische Position beschreibt.
Die klassische evidentialistische Position in den \emph{ethics of belief} -- wie
sie etwa von \authorfullcite{Clifford:TheEthicsofBelief1877} vertreten wird -- zeichnet sich
durch die Überzeugung aus, dass es \emph{immer} schlecht und tadelnswert ist,
eine unzureichend begründete Überzeugung zu
haben.\footnote{\enquote{To sum up: it is wrong always, everywhere, and for any
one, to believe anything upon insufficient evidence}
\parencite[][195]{Clifford:TheEthicsofBelief1877}.} \name[Immanuel]{Kant}s
Begriffe des Wissens und des Meinens bleiben dabei beide im Rahmen einer
evidentialistischen Position, denn auch im Falle des Meinens halten wir eine
Behauptung nicht schlechthin für wahr, sondern lediglich für möglich oder auch
wahrscheinlich. Nun gibt es auch Autoren, die behaupten, es sei doch zumindest
in manchen Situationen besser und sogar vernünftig, eine unzureichend begründete Überzeugung zu
haben.\footnote{Die klassische Reaktion findet sich bei William
\textcite[vgl.][]{James:TheWilltoBelieve1919}.} Eine Überzeugung zu haben sei
etwa dann vernünftig, wenn wir zwar keine hinreichenden Anhaltspunkte für ihre
Wahrheit haben, das Haben dieser Überzeugung jedoch positive Effekte zeitige.
Klassische Beispiele sind zum einen die \name[Blaise]{Pascal}'sche
Wette und zum anderen die Annahme, dass sich eine optimistische Einstellung bei
bestimmten Krankheiten positiv auf den Heilungserfolg
auswirke.\footcite[Vgl.][]{Chignell:TheEthicsofBelief2013}

\name[Immanuel]{Kant} scheint sich auf eine evidentialistische Position
festzulegen, denn er beschreibt die \enquote{innere} Lüge, also eine
Unaufrichtigkeit im Fürwahrhalten sich selbst gegenüber, als Verstoß gegen eine
unbedingte Pflicht, der auf einer Stufe mit der \singlequote{äußeren} Lüge
stehe.\footnote{\cite[Vgl.][A
85\,f.,]{Kant:DieMetaphysikderSitten1977Tugendlehre} \cite[][VI:
430.9--431.3]{Kant:GesammelteWerke1900ff.}.} Und die unbedingte Geltung des
Lügenverbots ohne Ansehen der Folgen steht spätestens durch
\name[Immanuel]{Kant}s Auseinandersetzung mit \name[Benjamin]{Constant} außer
Frage.\footnote{\cite[Vgl.][]{Kant:UebereinvermeintesRechtausMenschenliebezuluegen1977},
\cite[][VIII: 423--430]{Kant:GesammelteWerke1900ff.}.} Gerade weil er die
unbedingte Geltung des Gebots der Aufrichtigkeit nicht wie
\authorfullcite{Clifford:TheEthicsofBelief1877} konsequentialistisch über die
schlechten Folgen der Unaufrichtigkeit
begründet,\footnote{\authorcite{Clifford:TheEthicsofBelief1877} schreibt:
\enquote{No real belief, however trifling and fragmentary it may seem, is ever
truly insignificant; it prepares us to receive more of its like, confirms those
which resemble it before, and weakens others; and so gradually it lays a
stealthy train in our inmost thoughts, which may some day explode into overt
action, and leave its stamp upon our character for ever}
\parencite[][292]{Clifford:TheEthicsofBelief1877}. Er argumentiert dabei, dass
einzelne Urteilsakte, die zunächst harmlos erscheinen mögen, doch zur
Habitualisierung einer unkritischen Urteilsweise beitragen:
\enquote{Every time we let ouselves believe for unworthy reasons, we weaken our
powers of self-control, of doubting, of judicially and fairly weighing
evidence} \parencite[][294]{Clifford:TheEthicsofBelief1877}. Ein ähnliches
Argument findet sich bzgl. der Unaufrichtigkeit sich selbst gegenüber auch bei
\name[Immanuel]{Kant}
\mkbibparens{\cite[vgl.][A 86]{Kant:DieMetaphysikderSitten1977Tugendlehre},
\cite[][VI: 430.35--431.3]{Kant:GesammelteWerke1900ff.}}.} scheint es
\emph{prima facie} kaum denkbar, Gründe einzubeziehen, die nicht auf die
Wahrheit der Überzeugung, sondern ihre positiven oder negativen Folgen
rekurrieren. Denn er betont, dass es für die Bewertung einer Überzeugung nicht
relevant ist, ob sich aus ihr positive oder negative Folgen ergeben:
\begin{quote}
Es kann auch bloß Leichtsinn, oder gar Gutmütigkeit, die Ursache davon sein, ja
selbst ein wirklich guter Zweck dadurch beabsichtigt werden, so ist doch die
Art, ihm nachzugehen, durch die bloße Form ein Verbrechen des Menschen an seiner
eigenen Person, und eine Nichtswürdigkeit, die den Menschen in seinen eigenen
Augen verächtlich machen
muß.\footnote{\cite[][A 85]{Kant:DieMetaphysikderSitten1977Tugendlehre},
\cite[][VI: 430.4--8]{Kant:GesammelteWerke1900ff.}.}
\end{quote}
Gerade die \name[Blaise]{Pascal}'sche Wette kritisiert er, weil es unwürdig sei,
sich eine Überzeugung aus Hoffnung auf Belohnung oder Furcht vor Strafe selbst
einzureden.\footnote{\cite[Vgl.][A 85\,f.,]{Kant:DieMetaphysikderSitten1977Tugendlehre}
\cite[][VI: 430.19--23]{Kant:GesammelteWerke1900ff.}.}

Nun wird behauptet, \name[Immanuel]{Kant}s \emph{ethics of belief} seien gerade
deswegen von aktuellem Interesse, weil er mit dem Begriff des Glaubens die
evidentialistische Linie verlasse und sich der Position des späteren
Pragmatismus
annähere.\footnote{\cite[Vgl.][335]{Chignell:BeliefinKant2007}.}
Es gebe also andere Gründe für die Vernünftigkeit einer Überzeugung als solche,
die auf ihre Wahrheit hinzielen. Wir können Rechtfertigungen von
Überzeugungen, die die Wahrheit einer Überzeugung belegen, \enquote{epistemische
Rechtfertigungen} nennen und von \enquote{nicht-epistemischen Rechtfertigungen}
abgrenzen.\footcite[Vgl.][34]{Chignell:KantsConceptsofJustification2007}
\name[Immanuel]{Kant} spricht hier von \enquote{subjektiv zureichenden} Gründen,
die auch aus Sicht des Urteilenden nicht objektiv zureichend sind und die
Überzeugung ganz bewusst an ein eigenes \emph{Bedürfnis} rückbinden (es handelt
sich auch hier nicht um Überredung, sondern um Überzeugung oder ein
\emph{überlegtes} Urteil). Es handelt sich allem Anschein nach also bei objektiv
zureichenden Gründen um epistemische, bei bloß subjektiv zureichenden Gründen um
nicht-epistemische Gründe.

Ich stimme freilich der Aussage zu, dass \name[Immanuel]{Kant}
nicht-epistemische Rechtfertigungsgründe für Überzeugungen benennt, die eine
Überzeugung auch dann als vernünftig erscheinen lassen, wenn diese objektiv
nicht zureichend begründet werden kann und es sich auch nicht um die bescheidene
Position eines Meinens handelt. Allerdings denke ich nicht, dass es sich bei
\name[Immanuel]{Kant}s Glaubensbegriff um ein tragfähiges Konstrukt handelt.
Der Begriff des Glaubens ist schwierig, weil \name[Immanuel]{Kant}
darin die Legitimität bloß subjektiv zureichender Erkenntnisgründe behauptet.
Bis hierher schien es ja so zu sein, dass mündig und aufgeklärt derjenige
urteilt, der seine Zustimmung daran bemisst, ob etwas einen \emph{objektiv}
zureichenden Grund hat. Nun sagt \name[Immanuel]{Kant}, auch subjektiv
zulängliche Erkenntnisgründe können ein aufgeklärtes Fürwahrhalten
konstituieren, wenn der Urteilende nur \emph{weiß}, dass sie lediglich
subjektiv, nicht aber objektiv zureichend sind -- wenn also \emph{Überzeugung}
und keine \emph{Überredung} stattfindet.

Um in solchen Fragen nicht in Beliebigkeit zu verfallen, erwächst aus dieser
Konstellation -- die Fragen sind einerseits unvermeidlich, transzendieren aber
andererseits den Bereich unseres Wissens -- das Bedürfnis der Vernunft nach
Orientierung.\footnote{Siehe dazu
\cite[][A 309--311]{Kant:Washeisst:SichimDenkenorientieren?1977}, \cite[][VIII:
136.1--137.3]{Kant:GesammelteWerke1900ff.}.} Wie in Kapitel
\ref{section:KantalsliberalerAufklaerer} und zu Beginn von Kapitel
\ref{section:autonomieunddaszeugnisanderer} bereits aufgezeigt, ist es nach
\name[Immanuel]{Kant} kein gangbarer Weg, Fragen nach Übersinnlichem -- also
etwa die Frage nach der Existenz Gottes oder nach seinen Eigenschaften -- dem
privaten Belieben zu überantworten.
Sie können zwar nicht wissenschaftlich, ja nicht einmal im \emph{modus} des Wissens beantwortet
werden, aber sie sollen doch \emph{vernünftig} zu beantworten sein. Und dies
könne auch jeder, der sich nur von dem Vorurteil der eigenen Unfähigkeit und der
Überlegenheit anderer befreit. Hierzu wiederum leistet die Vernunftkritik einen
entscheidenden Beitrag:
\begin{quote}
Die Veränderung betrifft also bloß die arroganten Ansprüche der Schulen, die
sich gerne hierin (wie sonst mit Recht in vielen anderen Stücken) für die
alleinigen Kenner und Aufbewahrer solcher Wahrheiten möchten halten lassen, von
denen sie dem Publikum nur den Gebrauch mitteilen, den Schlüssel derselben aber
für sich behalten (quod mecum nescit, solus vult scire
videri).\footnote{\cite[][B xxxiii]{Kant:KritikderreinenVernunft2003},
\cite[][III: 20.28--33]{Kant:GesammelteWerke1900ff.}.}
\end{quote}
Der Vernunftkritik hat das Ziel, Aufklärung zu befördern, indem sie die
Ansprüche philosophischer Schulen beschränkt. Durch die Erkenntnis, dass es
keine wissenschaftliche Erkenntnis der \enquote{unvermeidlichen Aufgaben der
reinen Vernunft} gibt, braucht der Einzelne sich durch die Aussprüche und
Anmaßungen anderer nicht einschüchtern zu lassen, sondern kann den Mut finden,
seine eigene Vernunft zu gebrauchen. Auf der Grundlage der praktischen Vernunft
seien sie einfach zu beantworten. Dies ist zumindest dann nicht ganz unplausibel, wenn man
\name[Immanuel]{Kant}s moralisch-epistemischen Optimismus
voraussetzt.\footnote{Siehe oben, Kapitel
\ref{Abschnitt:moralischepistemischerOptimismus},
S.~\pageref{Abschnitt:moralischepistemischerOptimismus}--\pageref{Abschnitt:moralischepistemischerOptimismus-Ende}.}
Dieser ist die Grundlage dafür, dass sich jeder in Fragen der Metaphysik ganz
einfach selbst vernünftig orientieren könne, wenn er nur den Mut und die
Entschlusskraft dazu findet und sich von philosophischen Schulen nicht
einschüchtern lässt.

Jeder Glaube betrifft also etwas, das die Reichweite unseres Erkennens
übersteigt. Grundbedingung ist somit, dass wir nur dann sagen, wir
\emph{glauben}, dass $p$, wenn wir weder objektive Gründe {für} die
Behauptung, dass $p$, haben, noch solche dagegen. Der Glaube tritt nun wiederum
in dreierlei Gestalt auf, als \emph{doktrinaler Glaube}
(\ref{DoktrinalerGlaube}), als \emph{pragmatischer Glaube}
(\ref{PragmatischerGlaube}) und als \emph{moralischer Glaube}
(\ref{MoralischerGlaube}).
%
\begin{nummerierung}
\item\label{DoktrinalerGlaube} Der \emph{doktrinale Glaube} ist ein
theoretisches Fürwahrhalten in Fragen, die für uns keine direkte Relevanz haben.
Zu dieser Form des Glaubens zählt er insbesondere die Physikotheologie und gibt
als einzigen Unterschied zum Meinen die \singlequote{Festigkeit} an, die sich in
der Bereitschaft zeige, eine Wette einzugehen.\footnote{\cite[Vgl.][B
852--856]{Kant:KritikderreinenVernunft2003},
\cite[][III: 534.14--536.11]{Kant:GesammelteWerke1900ff.}.} Ein solches
Konstrukt ist freilich wenig überzeugend, spielt in \name[Immanuel]{Kant}s
Systematik aber auch keine weitere
Rolle. \authorfullcite{Chignell:BeliefinKant2007} behauptet, dass sich
auf Grundlage des doktrinalen Glaubens auch eine \singlequote{liberale
Metaphysik} errichten lasse, in der jeder nach seinen Überzeugungen über Dinge
an sich reden möge, solange diese Überzeugungen nicht mit objektivem Wissen
verwechselt werden. Dies sei möglich auf Grundlage der Einsicht in die
Unerkennbarkeit der Dinge an sich.\footnote{\enquote{Given this situation, we can and
should go ahead and build metaphysical arguments in all the usual ways, by
appealing to \enquote{intuitions} (of the Moorean rather tha the Kantian sort),
reflective equilibrium, inference to best explanation, simplicity, and so forth}
\parencite[][360]{Chignell:BeliefinKant2007}.} Eine Grundlage scheint mir diese
Interpretation in den Schriften \name[Immanuel]{Kant}s nicht zu haben (und
tatsächlich verweist \authorcite{Chignell:BeliefinKant2007} auch auf keine
textbasierte Evidenz). Auch scheint mir ein solches Unterfangen, dass dem
\enquote{frictionless spinning in the
void}\footcite[][11]{McDowell:MindandWorld1994} sehr nahe käme, nicht sehr
anziehend zu sein.
%
\item\label{PragmatischerGlaube} Der \emph{pragmatische Glaube} liegt vor, wenn
wir ein Ziel verfolgen und nur eine Bedingung wissen, unter der das Ziel erreicht werden kann. Nach
\name[Immanuel]{Kant} glauben wir dann vernünftiger Weise, dass die
entsprechende Bedingung erfüllt ist, da wir sonst unser Ziel nicht verfolgen
können. Sein eigenes Beispiel lautet: Ein Arzt versucht, einen Patienten zu
heilen, kann aber nicht aus objektiven Gründen beurteilen, welche Krankheit
dieser hat. Er vermutet, dass der Patient an der Schwindsucht leidet, und unter
dieser Annahme weiß er, was zu tun ist.

Nun scheint mir nicht einsichtig zu sein, warum \name[Immanuel]{Kant} nicht
einfach sagt, dass hier ein Fall von Meinen vorliegt, und es in manchen Fällen
nötig ist, bei unseren Entscheidungen Vermutungen über die zugrunde liegende
Situation (nicht über die moralischen Gebote!) einzubeziehen. Ein
verantwortungsvoller Arzt würde wohl urteilen, dass Schwindsucht die Krankheit
ist, die zu vermuten ist, und entsprechend handeln. Aber er wird weiter
berücksichtigen, dass es sich um eine Vermutung handelt; treten Erscheinungen
auf, die gegen diese Vermutung sprechen, wird er sie überdenken und
gegebenenfalls die Therapie ändern oder abbrechen. Vielleicht sollten wir
erstens sagen, dass es ein objektiver Grund ist, der dafür spricht, \emph{diese} Behauptung --
dass es die Schwindsucht und keine andere Krankheit ist -- für wahr zu halten,
wenngleich der Grund nicht zureichend ist und nur Wahrscheinlichkeit, aber kein
Wissen generiert. Und dann können wir zweitens sagen, dass ein subjektiver
Grund vorliegt, das Urteil nicht zu suspendieren, sondern trotz unzureichender
Grundlage \emph{sofort} zu urteilen; und dieser subjektive Grund ist der
eingetretene Handlungsdruck.

Ein recht eindrückliches Beispiel, welches \name[Immanuel]{Kant}s Position
entgegen kommt, liefert \authorfullcite{Chignell:BeliefinKant2007} mit folgendem
Szenario: Ein Bergsteiger befinde sich in einer Situation, in der er sein eigenes Leben nur
dadurch retten kann, dass er ungesichert über eine Kluft springt. Er hat keine
epistemische Grundlage, um objektiv entscheiden zu können, ob ihm dies gelingen
wird. Er \emph{weiß} also nicht, ob er über die Kluft springen kann. Nimmt man
nun aber an, dass die Chancen, eine entsprechende Leistung zu erzielen, von der
Festigkeit der Überzeugung abhängt, dass man es schaffen kann, so sei es
durchaus vernünftig, an die eigene Fähigkeit zu
glauben.\footcite[Vgl.][\pno~343\,f.]{Chignell:BeliefinKant2007}

Dieses Szenario verdeutlicht den Punkt, auf den es \name[Immanuel]{Kant}
anzukommen scheint: Wenn wir unser Handeln auf bloße Meinung gründen, dann
unterminiert dies unsere Motivation und Zuversicht. Es ändert zwar nichts daran,
was vernünftiger Weise zu tun ist, aber derjenige, der bewusst bloß vermutet,
wird verzagt handeln und damit den Handlungserfolg gefährden. Und dies mag eine
gewisse psychologische Plausibilität haben. Jeder, der sich sportlichen
Herausforderungen stellt, kennt die Bedeutung der eigenen Überzeugungen für die
eigene Leistungsfähigkeit.
%
\item\label{MoralischerGlaube} Bei dem \emph{moralischen Glauben} ist uns der
Zweck, den wir verfolgen, durch die Moral vorgegeben; des Weiteren wissen wir,
dass dieser Zweck nur erreicht werden kann, wenn eine bestimmte Bedingung
erfüllt ist. Der Zweck, der uns vorgegeben ist, besteht in dem höchsten Gut.
\begin{quote}
Glückseligkeit allein ist für unsere Vernunft bei weitem nicht das vollständige
Gut. Sie billigt solche nicht {\punkt}, wofern sie nicht mit der Würdigkeit,
glücklich zu sein, d.\,i. dem sittlichen Wohlverhalten, vereinigt ist.
Sittlichkeit allein, und, mit ihr, die bloße \ori{Würdigkeit}, glücklich zu
sein, ist aber auch noch lange nicht das vollständige Gut. Um dieses zu
vollenden, muß der, so sich als der Glückseligkeit nicht unwert verhalten hatte,
hoffen können, ihrer teilhaftig zu
werden.\footnote{\cite[][B 841]{Kant:KritikderreinenVernunft2003},
\cite[][III: 527.33--528.3]{Kant:GesammelteWerke1900ff.}.}
\end{quote}
Ob wir an die Existenz Gottes glauben oder nicht, das ändert nichts an dem
praktischen Wissen, das uns zur Verfügung steht. Insbesondere folgen keine
kategorischen oder moralischen Imperative, die ohne religiöse
Annahmen nicht zu begründen wären.
Und auch wenn Religion \enquote{das Erkenntnis aller unserer Pflichten als göttlicher
Gebote}\footnote{\cite[][B~229]{Kant:DieReligioninnerhalbderGrenzenderblossenVernunft1977}, \cite[][VI:
153.28--29]{Kant:GesammelteWerke1900ff.}.} sei, hängen doch unsere Pflichten
nicht von der Religion ab. Aber der Glaube stütze doch die Erfolgsaussichten
unseres Handelns gemäß den Geboten der Sittlichkeit. Denn er lasse uns
\emph{hoffen}, dass eine Lebensausrichtung, die Gebote der Sittlichkeit an
oberster und Ratschläge der Klugheit an zweiter Stelle ansiedelt, \emph{in
beidem} Erfolg haben könne, also sowohl zu Sittlichkeit, als auch zu
Glückseligkeit führe. Unglaube hingegen unterminiere die Motivation zu
moralischem Handeln, insofern er mit der Einsicht in die Möglichkeit (oder gar
Wahrscheinlichkeit) des Scheiterns verbunden sei. Dennoch -- das ist nicht
fraglich -- darf die Motivation zu moralischem Handeln nicht in der Erwartung
göttlicher Belohnung und Strafe liegen.
%

Bereits die \titel{Kritik der reinen Vernunft} artikuliert
die Hoffnung, \enquote{ob sich nicht in ihrer praktischen Erkenntnis Data
finden, jenen transzendenten Vernunftbegriff des Unbedingten zu
bestimmen}\footnote{\cite[][B xxi]{Kant:KritikderreinenVernunft2003},
\cite[][III: 14.16--18]{Kant:GesammelteWerke1900ff.}. Siehe auch
\cite[][B xxiv\,f.,]{Kant:KritikderreinenVernunft2003}
\cite[][III: 16.9--16]{Kant:GesammelteWerke1900ff.}.}.
Die \titel{Kritik der reinen Vernunft} beschneidet die Ansprüche der Schulen in
Fragen der Metaphysik, um \emph{auf eben diesem Gebiet} dem Einzelnen die
Freiheit des Selbstdenkens zu geben.
\begin{quote}
Nur solche Gegenstände sind Sachen des Glaubens, bei denen das Fürwahrhalten
notwendig frei, d.\,h. nicht durch objektive, von der Natur und dem Interesse
des Subjekts unabhängige, Gründe der Wahrheit bestimmt
ist.\footnote{\cite[][A 106]{Kant:ImmanuelKantsLogik1977},
\cite[][IX: 70.9--12]{Kant:GesammelteWerke1900ff.}.}
\end{quote}
Aber dies heißt gerade \emph{nicht}, Fragen der Religion, die nicht mit den
Mitteln der theoretischen Philosophie zu beantworten sind, dem privaten Belieben
und subjektiven Zufälligkeiten anheim zu stellen. Das Aufklärungsverständnis des
\enquote{sapere aude!} beruht wesentlich auf der Überzeugung, das diejenigen
Fragen, die für uns als Menschen von Bedeutung sind, einer \emph{vernünftigen}
und damit intersubjektiv kommunizierbaren und kritisierbaren Beantwortung
zugänglich sind. Der Indifferentismus, der verschiedene Antwortmöglichkeiten als
gleichrangige Optionen akzeptiert statt
sie dem Maßstab der Vernunft unterzuordnen, ist nach \name[Immanuel]{Kant} ja gerade
keine Option. Bester Beleg hierfür ist sein Begriff eines Vernunftglaubens.
\end{nummerierung}

Die Situation der Metaphysik, die \name[Immanuel]{Kant} so eindringlich in den
Vorreden der \titel{Kritik der reinen Vernunft} beschreibt, tritt zunächst
als Bedrohung der Aufklärung auf. Wenn Metaphysik ihrem Weltbegriff nach die
Fragen thematisiert, die wir beantworten müssen, um unserer Bestimmung gerecht
werden zu können, diese Fragen aber einer vernünftigen Antwort nicht zugänglich
zu sein scheinen, dann gefährdet dies die Möglichkeit, unser je eigenes Leben
vernünftig auszurichten. So erklärt sich die Emphase, mit der er das Schicksal
der menschlichen Vernunft beschreibt, durch unabweisbare und unbeantwortbare
Fragen belästigt zu
werden.\footnote{\cite[Vgl.][A vii]{Kant:KritikderreinenVernunft2003},
\cite[][IV: 7.2--6]{Kant:GesammelteWerke1900ff.}.}

Der anvisierte Ausweg lautet: Wir haben zwar keine theoretische
Ausgangsbasis für eine vernünftige Beantwortung der metaphysischen Fragen, aber eine
Ausgangsbasis in der praktischen Vernunft. Die praktische Vernunft sagt uns, was
wir tun sollen, und gibt uns einen notwendigen Zweck mit auf den Weg: das
höchste Gut in Gestalt einer \enquote{Glückseligkeit {\punkt} in dem genauen
Ebenmaße mit der Sittlichkeit der vernünftigen Wesen, dadurch sie derselben
würdig sind}\footnote{\cite[][B 842]{Kant:KritikderreinenVernunft2003},
\cite[][III: 528.13--14]{Kant:GesammelteWerke1900ff.}.} Da uns dieses höchste
Gut als notwendiger Zweck aufgegeben sei, sei es nur vernünftig, die Bedingungen
für gegeben zu halten, unter denen allein er realisierbar ist. Obwohl wir nicht
\emph{wissen} können, ob ein Gott existiert und ob es ein Leben nach dem Tod
gibt, können wir doch bestimmte Antworten als vernünftig ausmachen. Der Begriff
eines moralischen Glaubens und die Ethikotheologie sind Versuche, die
Aufklärungskonzeption vor den Resultaten der Vernunftkritik in Schutz zu nehmen.

\chapter{Schlussbemerkungen}\label{Schlussbemerkungen}


Unsere Endlichkeit steht der Forderung nach Selbstdenken entgegen,
insofern wir das Besondere gar nicht durch eigenes Nachdenken bestimmen können, sondern auf gegebene
Erkenntnisse angewiesen bleiben, und insofern wir in der vernünftigen
Ausrichtung unserer Handlungen stets mit der Abhängigkeit unseres Willens von
Neigungen als Hindernissen konfrontiert werden. Unsere Endlichkeit macht
Aufklärung aber nicht unmöglich, sondern erst notwendig. Ein nicht-endliches Wesen bedürfte nicht der Aufklärung, der nur
dasjenige Wesen unterworfen werden kann, dessen Vernunfttätigkeit von seiner
Sinnlichkeit abhängig ist, da diese seinem diskursiven Denken überhaupt erst
Inhalte bereitstellt. Der Verstand eines endlichen Wesens ist diskursiv,
insofern er nur mittelbaren Bezug zu den Gegenständen seines Denkens aufweist:
Denken ist Erkennen durch Begriffe; und Begriffe sind Vorstellungen, die sich
nur vermittelst allgemeiner Merkmale auf ihre Gegenstände beziehen.
Vorstellungen, die einer solchen Vermittlung nicht bedürfen, sondern sich
unmittelbar auf Gegenstände beziehen, nennt \name[Immanuel]{Kant}
\enquote{Anschauungen}. Ohne Vorstellungen, die sich unvermittelt auf
Gegenstände beziehen, könnte es aber gar keinen Gegenstandsbezug geben, unser
Denken wäre ein \enquote{frictionless spinning in the
void}\footnote{\cite[][11]{McDowell:MindandWorld1994}.}, wie
\authorfullcite{McDowell:MindandWorld1994} schreibt.


Ein unendliches Wesen müsste in seinem Denken nicht zwischen
rezeptivem und spontanem Erkenntnisvermögen unterscheiden: Sein Erkennen wäre
gänzlich spontan und seine Vernunft fände kein Hindernis an den eigenen
sinnlichen Antrieben. Damit wäre ein solches Wesen in einer Weise
selbstbestimmt, aktiv und \singlequote{spontan}, die sich von unserer Art des
Denkens und Erkennens nicht graduell, sondern qualitativ unterscheidet. Wir
endliche Wesen hingegen, deren oberes Erkenntnisvermögen diskursiv ist, sind stets der Gefahr ausgesetzt,
unsere Aktivität und Spontaneität hintanzustellen und uns der Passivität
hinzugeben. Dem entgegenzuwirken heißt Aufklärung.

Kant bestimmt Aufklärung als den Ausgang des Menschen aus selbst verschuldeter
Unmündigkeit und damit als Haltung, immer und überall selbst zu denken und die
eigene Vernunft als obersten Maßstab der Wahrheit seiner Überzeugungen gelten zu
lassen. Dabei nimmt er jedoch die Warnung
\authorcite{Wolff:Discursuspraeliminarisdephilosophiaingenere1996}s vor einem
Abgleiten in Flachheit und Beliebigkeit durch eine falsch verstandene Aufklärung
auf. Es kommen dabei zwei unterschiedliche Aufklärungsverständnisse in den
Blick: Selbstdenken kann zum einen als Resultat eines \emph{Entschlusses}, zum
anderen als Ausdruck einer \emph{Fähigkeit} verstanden werden.
\authorcite{Wolff:Discursuspraeliminarisdephilosophiaingenere1996}s Einsicht
ist: Nur durch die Ausbildung eigener intellektueller Kompetenzen lässt sich
Selbstdenken erreichen. \name[Immanuel]{Kant} vertritt nun allem Anschein nach
die Gegenposition, auch indem er sich von
\authorcite{Wolff:Discursuspraeliminarisdephilosophiaingenere1996}s
Methodenideal -- \Revision{der mathematischen Methode} -- explizit abgrenzt. Tatsächlich
verwirft \name[Immanuel]{Kant} den Gedanken, intellektuelle Kompetenz durch
bestimmte Standards oder eine bestimmte Methode beschreiben zu können. Er
verfällt jedoch nicht der naiven Vorstellung, Selbstdenken sei schlicht das
Resultat eines Entschlusses, der keiner weiteren Kompetenzen bedürfte.
Stattdessen bindet er sie zurück an die Ausbildung und Ausübung intellektueller
Kompetenz in Gemeinschaft und im Austausch mit anderen. Der aufgeklärt Denkende
ist daher nicht der isoliert von anderen jedes Urteil nur vor sich selbst
rechtfertigende Individualist, sondern derjenige, der sich nach dem Modell
einer republikanischen Vernunft selbst als gleichberechtigtes Mitglied einer
Gemeinschaft von Denkenden begreift. Die eigene Vernunft ist zu verstehen als
der je eigene \emph{Gebrauch} der \emph{allgemeinen} Menschenvernunft.

Aufklärung verlangt von uns nicht, alles zu wissen, sondern das zu
wissen, was für unser alltägliches Handeln wesentlich ist. \singlequote{Wesentliche}
Erkenntnisse sind wiederum solche, die sich auf die \singlequote{Bestimmung des
Menschen} -- ein zentrales Schlagwort der zweiten Hälfte der Aufklärung --
beziehen und den \emph{Weltbegriff der Philosophie} ausmachen. Ein solches
Wissen nennt Kant \enquote{pragmatisch} und versteht darunter
handlungsrelevantes Wissen, also einerseits Wissen darum, wie wir handeln
sollen, -- welches sich in den verschiedenen Arten von Imperativen (Regeln der
Geschicklichkeit, Ratschläge der Klugheit und Gebote der Sittlichkeit)
artikuliert -- sowie andererseits Wissen über die Situation, in der wir Menschen
uns als Menschen in der Welt befinden (die \emph{conditio humana}). Die
Herausforderung besteht dabei nicht darin, solches Wissen erst selbst zu
erwerben; gewöhnlich verfügen wir längst über dieses Wissen. Worum wir uns
hingegen erst bemühen müssen ist, das vorhandene Wissen in unserem Handeln
wirksam werden zu lassen.

Dennoch bleibt die Frage bestehen: Kann Mündigkeit vereinbar sein mit dem Erwerb
von Wissen durch die Auskunft Anderer? Selbstdenken wird verstanden als Selbstbestimmung im Denken, als
Versuch, unabhängig von Traditionen und Autoritäten unser Denken selbst zu
verantworten. Der Unaufgeklärte verzichtet darauf, selbst zu denken -- so
\name[Immanuel]{Kant}s \emph{prima facie} problematisches Postulat --, wenn er
Bücher kaufen kann, die das \enquote{verdrießliche Geschäft} des Denkens für
ihn übernehmen.\footnote{\cite[][A
482]{Kant:BeantwortungderFrage:WasistAufklaerung?1977}, \cite[][VIII:
35.13--16]{Kant:GesammelteWerke1900ff.}.} Wenn wir denken, dann tun wir jedoch
oft gut daran, uns nicht auf unsere eigenen oft schlecht ausgebildeten
Fähigkeiten und unsere eigenen mitunter schlecht fundieren Einsichten zu
verlassen, sondern auf das Wissen von solchen, die es besser wissen als wir. Wir
fragen Menschen, die sich aufgrund ihrer Ausbildung besser mit der Materie
auskennen (Experten), oder solche, die einen Sachverhalt besser beurteilen
können als wir, weil sie sich zu einer bestimmten Zeit an einem bestimmten
epistemisch günstigen Ort aufgehalten haben (Zeugen). Bereits im 18.
Jahrhundert waren sich Philosophen wie Kant der Tatsache bewusst, dass der
Erwerb von Wissen keine Leistung von Einzelkämpfern, sondern Resultat
gemeinschaftlicher Erkenntnisbemühungen ist. So besäßen wir gar
nicht in nennenswertem Umfang Wissen, wenn wir nicht über sogenanntes testimoniales
Wissen verfügten, also über solche Überzeugungen, die wir lediglich auf die
Autorität anderer Personen hin für wahr halten, insofern diese uns entsprechende
Informationen zur Verfügung stellen und uns an ihrem Wissen teilhaben lassen.
Die Notwendigkeit solchen Wissens scheint aber die Forderung der Aufklärung, in
allen Belangen selbst zu denken, \emph{ad absurdum} zu führen, auch dann, wenn
wir den Bereich dessen, was wir selbst kompetent beurteilen müssen, auf
pragmatisches Wissen einschränken.

Seit \authorcite{Descartes:OeuvresdeDescartes1983}, der einen der
folgenreichsten Angriffe auf die \singlequote{Büchergelehrsamkeit} führte,
bemühen sich Philosophen um eine sinnvolle Darlegung der Möglichkeit
testimonialen Wissens. \authorcite{Descartes:OeuvresdeDescartes1983}' Argumente
weisen in zwei unterschiedliche Richtungen: Zum einen argumentiert er gegen die
Verlässlichkeit der Auskünfte, die wir von anderen erhalten. Zum anderen
behauptet er, durch das Lesen von Büchern erwürben wir doch keine Wissenschaft,
sondern lediglich \singlequote{historische Kenntnisse}. Von weitaus größerer
Relevanz für das Projekt der Aufklärung ist die zweite Argumentationsrichtung,
weil sie die Entwicklung eigener intellektueller Kompetenzen betrifft. Wir sind
nicht unaufgeklärt, wenn wir auf Grundlage einer Mitteilung etwas für wahr
halten, was \emph{de facto} falsch ist, sondern dann, wenn wir uns passiv
verhalten und bloß rezipieren, was unser aktives Mitdenken und die Entwicklung
eigener intellektueller Kompetenzen verlangt.


Die Forderung, selbst die Verantwortung für die eigenen Überzeugungen zu
übernehmen, betrifft zunächst dasjenige Wissen, welches wir nicht aus der
Erfahrung, sondern aus dem Gebrauch unserer Vernunft haben: Mathematik und
Philosophie. Da die Mathematik wegen der Möglichkeit der Darstellung ihrer
Inhalte in der reinen Anschauung leichter adäquat gelehrt und gelernt werden
kann, bleibt als schwierige Herausforderung die adäquate Aneignung
philosophischen Wissens, also rationaler Erkenntnis aus Begriffen. Ein solches
rationales Wissen aus Begriffen nennt \name[Immanuel]{Kant} auch
\emph{Metaphysik}.\footnote{Die Bestimmungen des Selbstdenkens als der Forderung, selbständig metaphysische
Erkenntnisse zu generieren, harmonieren mit Kants Explikation des Begriffs des
Aberglaubens wie auch des Vorurteils. Vorurteile sind vorläufige, unbegründete
Urteile, insofern sie als Prinzipien unseres Schließens wirksam sind. Um
Vorurteile zu vermeiden müssten wir zunächst erkennen, welche Prinzipien unseren
Urteilen zugrunde liegen, und uns aktiv mit den begrifflichen Grundlagen unseres
Denkens auseinander
setzen \mkbibparens{\cite[vgl.][A~116\,f.,]{Kant:ImmanuelKantsLogik1977}
\cite[][IX: 75.24--76.12]{Kant:GesammelteWerke1900ff.}}. Vorurteilsbehaftet
bleibt derjenige, der die begrifflichen Grundlagen seines Denkens so hinnimmt, wie sie ihm gegeben wurden.
Aberglaube ist nach Kant das Vorurteil, die Welt so zu repräsentieren, als wäre
sie nicht notwendig den Gesetzen unseres Verstandes unterworfen. Es handelt sich
also nicht einfach um empirisch falsche Urteile, sondern um Vorurteile, da sie
auf einem falschen metaphysischen Grundsatz beruhen. Dem Aberglauben steht daher
die metaphysische Einsicht entgegen, dass alles, was geschieht, nach objektiven
Gesetzen geschieht. Deswegen ist er eine Form der Heteronomie im Denken: Wer
abergläubisch ist, der bewertet die gegebene Information höher als sein eigenes
Denken. Dabei spielt es keine Rolle, ob die Information von den eigenen Sinnen
zu kommen scheint oder von einem vermeintlichen Experten oder Zeugen.} Und damit bezieht sich die Forderung des
\enquote{Sapere aude!} zunächst und vor allem auf metaphysische Erkenntnisse, was aus der Sicht
manches Aufklärungsverständnisses überraschen dürfte. Nun also soll Aufklärung
\emph{nur} durch Metaphysik möglich sein, freilich durch eine Metaphysik, die
sich als \singlequote{kritische} Metaphysik von der Metaphysik seiner Vorgänger
unterscheidet. Doch das Grundproblem dieser Arbeit tritt nun neu hervor: Ist
Metaphysik aus der Perspektive endlicher Wesen überhaupt möglich?

Nach \name[Immanuel]{Kant} zeichnen sich obere Erkenntnisvermögen dadurch aus,
dass sie metaphysische Erkenntnisse generieren und dadurch Prinzipien enthalten,
die all unser Erkennen bestimmen. Zwar sind wir endlich und können daher nur
durch Rückgriff auf Sinnlichkeit gehaltvolle Gedanken haben. Aber dieses Denken,
welches nur durch seinen Bezug auf Erfahrung gehaltvoll ist, ist doch zugleich
autonom, d.\,h. es wird durch Grundsätze bestimmt, die selbst nicht der
Erfahrung entlehnt, auf diese aber notwendigerweise anwendbar sind.
Eine \emph{Ethics of Belief} im Sinne \name[Immanuel]{Kant}s und seines
Aufklärungsprogramms hat daher zur Kernforderung, nicht das passiv zu
übernehmen, was unser eigenes Denken regieren sollte. Der Grundsatz, den
obersten Probierstein der Wahrheit in seiner \emph{eigenen} Vernunft zu suchen,
den \name[Immanuel]{Kant} als Definition des Begriffs des Selbstdenkens anführt,
verweist in diesem Sinne auf die Autonomie der Erkenntnisvermögen. Wer
Erkenntnisse, die er \emph{a priori} (oder \singlequote{\emph{ex principiis}})
erwerben könnte, passiv aufnimmt, der lässt sich überreden statt überzeugen.

\Revision[Heidemann]{Damit ist die Frage nach einer \emph{ethics of belief} bei
\name[Immanuel]{Kant} im Grunde auch abschließend beantwortet. Selbstverständlich ließen sich viele
weitere und weitaus konretere Regeln benennen, die der mündige und aufgeklärte
Mensch befolgen sollte. Aber diese Regeln stellen keine Konkretisierung des
Begriffs Mündigkeit dar. Denn letztere besteht niemals in der bloßen Befolgung
solcher Regeln, die vielmehr selbst zum Thema selbständigen Philosophierens zu
machen sind. Mündig ist also nicht, wer epistemischen Regeln der Aufklärung
folgt, sondern wer die eigenen epistemischen Grundsätze reflektiert und
selbständig artikulieren, bewerten, verteidigen und gegebenenfalls revidieren
kann.}

\begin{comment}
Gerade in Fragen der Religion kommt zum Vorschein, wie pikant die Ergebnisse der
Vernunftkritik aus aufklärerischer Perspektive sind. Die Unmöglichkeit objektiv
gültiger Urteile bezüglich der Existenz Gottes befördert intellektuelle Freiheit
in Religionssachen entgegen dem ersten Anschein gerade nicht, sondern stellt
eine Gefährdung derselben dar. Der aufgeklärte Mensch benötigt einen allgemeinen
Maßstab der Vernunft, sonst bleibt ihm nur die Urteilsenthaltung. Um diesen Maßstab
zu entwickeln bedarf es der Theorie eines praktischen Vernunftglaubens. Diese
Konzeption eines Vernunftglaubens ist gerade Ausdruck der Spannung zwischen
Aufklärung und ihrer Forderung nach Selbstdenken und Mündigkeit auf der einen
und unserer Endlichkeit und der ihr geschuldeten Begrenzung unseres
Erkenntnisbereichs auf der anderen Seite.
\end{comment}



\backmatter\sloppy
\part*{Anhang}\addcontentsline{toc}{part}{Anhang}
\renewcommand{\thesection}{\Alph{section}}
\renewcommand{\thesubsection}{\alph{subsection}}
\defbibnote{Siglen}{Die Schriften Kants (und ausschließlich diese) werden
durchgängig nach Siglen zitiert. Diese orientieren sich an den Gepflogenheiten
der \emph{Kant-Studien}. In jedem Fall wird die zitierte Stelle
durch Angabe des Orts in der Akademie-Ausgabe (\emph{Kant's Gesammelte
Schriften}) kenntlich gemacht. Das gilt auch dann, wenn dem Zitat eine andere
Textausgabe zugrunde liegt.}

\newpage
\chapter{Literaturverzeichnis}
\section{Quellen}

\defbibheading{PQK}{\subsection{Schriften Kants (mit Siglen)}}
\defbibheading{PQS}{\subsection{Sonstige Quellen}}
\defbibheading{SQ}{\section{Weitere Literatur}}

\defbibfilter{PQK}{\keyword{Quellen}\and\keyword{Kantschrift}}
\defbibfilter{PQS}{\keyword{Quellen}\and\not\keyword{Kantschrift}}
\defbibfilter{SQ}{\not\keyword{Quellen}}

\printshorthands[prenote=Siglen,heading=PQK]
% \printbibliography[heading=PQK,filter=PQK]
\printbibliography[heading=PQS,filter=PQS]
\printbibliography[heading=SQ,filter=SQ]

% \renewcommand{\indexname}{Abbildungsverzeichnis}
%\cleardoublepage\phantomsection%
%\label{listoffigures}\chaptermark{\indexname}%
%\addcontentsline{toc}{chapter}{\indexname}%
%\listoffigures

\renewcommand{\indexname}{Namensverzeichnis}
\cleardoublepage\phantomsection%
\label{printindex}\chaptermark{\indexname}%
\addcontentsline{toc}{chapter}{\indexname}%
\printindex

\end{document}

