
Unter dem Titel einer \emph{Sozialen Erkenntnistheorie} florieren seit einigen
Jahrzehnten Überlegungen dazu, wie wir in unserem je eigenen Wissen und Erkennen
von der Gemeinschaft abhängig sind, in der wir leben und aufgewachsen
sind.\footnote{Einen Überblick geben \cite{Schmitt:Introduction2010},
\cite{Wilholt:SozialeErkenntnistheorie2007}, sowie
\cite{Scholz:DasZeugnisanderer2001}.} Dabei handelt es sich -- allen
anders lautenden Bekundungen zum Trotz\footnote{Siehe etwa
\cite[][529--531]{Grundmann:AnalytischeEinfuehrungindieErkenntnistheorie2008},
sowie \cite[][46]{Wilholt:SozialeErkenntnistheorie2007}.} -- keineswegs um eine
neue Stoßrichtung der Erkenntnistheorie.\footnote{\cite[Vgl.][46]{Scholz:DasZeugnisanderer2001}.}
Nicht nur \authorcite{Hume:AnEnquiryConcerningHumanUnderstanding1964} und
\authorcite{Reid:EssaysontheIntellectualPowersofMan2002}, sondern bereits
\authorcite{Descartes:OeuvresdeDescartes1983},
\authorcite{Spinoza:EthikingeometrischerOrdnungdargestellt2007}, verschiedene
Autoren der deutschen Aufklärung und nicht zuletzt \name[Immanuel]{Kant} setzen
sich mit den sozialen Grundlagen unseres Denkens und Erkennens
auseinander. Das Wissen und Können anderer -- Eltern, Lehrer, Traditionen -- sind
die ersten Quellen unseres je eigenen Wissens und Könnens. Dass wir bei diesem
ursprünglichen Wissenserwerb alles andere als mündig und selbständig sind, ist
der Ausgangspunkt von \authorcite{Descartes:OeuvresdeDescartes1983}'
Überlegungen zur Verlässlichkeit unserer
Überzeugungen.\footnote{\cite[Vgl.][VI:
13.1--12]{Descartes:OeuvresdeDescartes1983}.
Zur Bedeutung für die deutsche Aufklärungsphilosophie siehe
\cite[][104]{Schneiders:AufklaerungundVorurteilskritik1983}.} Dessen Behauptung,
wir könnten versuchen, das \emph{corpus} unseres Wissens auf
individualistischer Grundlage neu zu errichten, ist  auch im 17. und 18.
Jahrhundert nicht mehrheitsfähig. Selbst unser alltäglichstes Wissen,
beispielsweise unser Wissen um das eigene Geburtsdatum und die eigene Herkunft
sind -- so erinnert uns \authorcite{Spinoza:SpinozaOpera1972} im \titel{Tractatus de intellectus
emendatione} -- Dinge, die wir ausschließlich von anderen erfahren
können.\footnote{\cite[Vgl.][II: 10.22--24]{Spinoza:SpinozaOpera1972}.} Erst
auf der Grundlage solchen Wissens und von anderen erlernter Fähigkeiten beginnen
wir, selbständig  Wissen zu generieren. Und dass wir irgendwann anfingen, nur
noch solche Überzeugungen neu zu erwerben, von deren Wahrheit wir uns selbst
überzeugen können, ist weder realistisch noch vernünftig.
\name[Immanuel]{Kant} sagt, wir könnten uns einen solchen
\singlequote{historischen Unglauben} -- den Verzicht darauf, Überzeugungen
auf der Grundlage von Mitteilungen zu bilden -- gar nicht als vorsätzlich
vorstellen; zu absurd und realitätsfern wäre diese
Haltung.\footnote{\cite[Vgl.][A
328]{Kant:Washeisst:SichimDenkenorientieren?1977}, \cite[][VIII:
146.8--11]{Kant:GesammelteWerke1900ff.}.}

In der Neuzeit haben (mitunter dieselben) Philosophen gefordert, sich von
den Einflüssen durch andere zu emanzipieren, selbst zu denken, statt Lehrern und
Traditionen zu folgen. Mündig sollen wir sein, nicht autoritätshörig, im Denken
wie im Handeln dem \emph{eigenen} Verstand folgen, statt zu tun und für wahr zu
halten, wozu andere uns anhalten -- das war und ist eine Forderung, die sich mit
dem Etikett \enquote{Aufklärung} schmückt. Doch sie scheint in
Konflikt zu geraten mit der Einsicht in die sozialen Grundlagen unseres
Wissens. Deswegen müssen wir fragen, was es eigentlich
heißen soll, wir sollten \emph{selbst} denken, unseren \emph{eigenen} Verstand
und die \emph{eigene} Vernunft gebrauchen. Solange wir diese Fragen nicht
hinreichend beantworten können, bleibt jede Anknüpfung an \singlequote{die}
Aufklärung -- zumindest in der Form, die \name[Immanuel]{Kant} ihr in dem
berühmten Aufsatz gab -- leer und unbestimmt.

Wir verstehen uns als Erben der Aufklärung, verwenden sorglos das Adjektiv
\enquote{aufgeklärt} und wenden es ohne Zögern auf uns selbst und
unseren Kulturkreis an. Doch nur selten  fragen wir, worin \singlequote{die} Aufklärung
eigentlich besteht. Dabei ist unklar, ob es einen einheitlichen
Sinn gibt, in dem wir von Aufklärung und Mündigkeit sprechen, und ob diese
Ausdrücke ein vernünftiges Projekt
beschreiben.\footnote{\cite[Vgl.][41]{Stekeler-Weithofer:Denken2012}:
\enquote{Doch es ist erst noch einmal zu bedenken, ob wir einer derartigen
Selbstdarstellung der Aufklärung überhaupt folgen können. Denn was heißt hier
Denken und Selbstdenken? Gegen wen oder was richtet sich das
\singlequote{Selbst}?} Siehe auch \cite[][42]{Stekeler-Weithofer:Denken2012}:
\enquote{Wenn sich Aufklärung dann aber auch gegen eine \singlequote{Disziplinlosigkeit des Denkens}
richtet und Autonomie damit offenbar ähnlich wie die Kompetenz zu herrschen erst
einmal Disziplin und Ordnung, vielleicht sogar Abrichtung, voraussetzt, bemerken
wir, dass der scheinbar klaren Entgegensetzung von Autonomie und der Autorität
der Tradition eine viel komplexere Struktur zugrunde liegt.} \authorfullcite{Stuke:Aufklaerung1972}
behauptet, dass \name[Immanuel]{Kant} den Begriff Aufklärung uneinheitlich
verwende und seine Aufklärungskonzeption in sich inkonsistent
sei \parencite[vgl.][265--272]{Stuke:Aufklaerung1972}.} Möglicherweise
beschreibt der Terminus \enquote{Aufklärung} nur eine zurückliegende Epoche der
europäischen Geistesgeschichte, aber kein in sich schlüssiges Programm, auf das wir
uns heute noch berufen könnten. Und vielleicht sind die Forderungen, die sich
hinter Begriffen wie \enquote{Mündigkeit} und \enquote{Selbstdenken} verbergen,
angesichts unserer Lebenswirklichkeit und kognitiven Ausstattung viel zu
anspruchsvoll. Insbesondere der Konflikt zwischen der Forderung nach
Selbständigkeit und (epistemischer) Unabhängigkeit auf der einen und der
eingangs beschriebenen Abhängigkeit von Anderen auf der anderen Seite lässt die
Forderung der Aufklärung als unrealistisch
erscheinen.\footnote{\phantomsection\label{Anmerkung:KantundderDoxastischeVoluntarimus}\authorfullcite{Scholz:KantsAufklaerungsprogramm2009}
diskutiert des Weiteren den Einwand, Aufklärung setze eine Form des
\emph{doxastischen Voluntarismus}
voraus \mkbibparens{\cite[vgl.][38--40]{Scholz:KantsAufklaerungsprogramm2009}.}
Unter doxastischem Voluntarismus verstehen wir die Position, dass wir
willentlich kontrollieren können, welche Behauptungen wir für wahr halten. Wenn
wir uns in der Annahme von Überzeugungen auf bestimmte Art und Weise
verhalten sollen, dann -- so scheint es -- muss es uns möglich sein, selbst zu
entscheiden, welche Überzeugungen wir annehmen und welche nicht. Wir müssten
frei sein in der Entscheidung, was wir für wahr halten. Nun scheint der
doxastische Voluntarismus falsch zu sein, denn er widerspricht unserer
alltäglich Erfahrung: Wenn wir wahrnehmen, dass es regnet, dann können wir
nicht \emph{ad libitum} eine andere Überzeugung annehmen, etwa die, dass es
schneit. Der doxastische Voluntarismus wird aus diesem Grund auch von
\name[Immanuel]{Kant} zurückgewiesen \mkbibparens{\cite[vgl.][\nopp
2508]{Kant:Reflexionen1900ff.}, \cite[][XVI:
398.11--14]{Kant:GesammelteWerke1900ff.}. In der {\jaeschelogik} steht: \enquote{Unmittelbar hat der Wille keinen Einfluß
auf das Fürwahrhalten; dies wäre auch sehr ungereimt} \mkbibparens{\cite[][A
113]{Kant:ImmanuelKantsLogik1977}, \cite[][IX:
73.33--34]{Kant:GesammelteWerke1900ff.}}. Siehe dazu auch
\cite[][]{Cohen:KantontheEthicsofBelief2014}, sowie
\cite{Cohen:KantonDoxasticVoluntarismanditsImplicationsfortheEthicsofBelief2013}.
Dagegen behauptet \authorfullcite{Chignell:KantsConceptsofJustification2007},
dass \name[Immanuel]{Kant} zumindest bezüglich mancher Arten von Überzeugungen
Voluntarist sei
\parencite[vgl.][36]{Chignell:KantsConceptsofJustification2007}.}}

Diese kritischen Nachfragen stellen keine Perspektive dar, mit der erst wir heutigen Interpreten
uns der Aufklärung und ihrem Programm zuwenden: Neben der Frage, was derjenige,
der Aufklärung einklagt, von seinen Mitmenschen eigentlich fordert, stritten
Philosophen des 18. Jahrhunderts auch darüber, ob der Mensch mit seinen
eingeschränkten intellektuellen Fähigkeiten der Aufklärung überhaupt fähig ist.
Für die Einschränkungen unserer intellektuellen oder kognitiven
Leistungsfähigkeit, insofern sie uns nicht zufällig als dieses oder jenes
Individuum betreffen, sondern die Beschränktheiten meinen, denen wir \emph{als Menschen} unterliegen, steht der Begriff der \emph{Endlichkeit}.
Zu zeigen, dass die Aufklärungsformel nicht leer ist, beinhaltet, ein kantisches
Aufklärungsprogramm zu rekonstruieren, das auf der Forderung nach Mündigkeit
(nach einer aktiven, selbständigen und vorurteilsfreien Art zu denken) gründet,
dabei aber auch unsere Endlichkeit als zentralen Bestandteil der
\emph{conditio humana} berücksichtigt. Es geht daher darum, eine
\singlequote{\emph{ethics of belief}}\footnote{Der Ausdruck stammt von
\authorfullcite{Clifford:TheEthicsofBelief1877}, der ihn 1877 als Titel eines
Textes wählt, in dem er dafür argumentiert, dass wir für unsere Überzeugungen
auch in ethischer Hinsicht verantwortlich sind
\parencite[siehe][\pno~189\,f.]{Clifford:TheEthicsofBelief1877}.} zu entwickeln,
die zwischen der Forderung der Aufklärung nach Unabhängigkeit und
Selbständigkeit auf der einen und unserer Endlichkeit auf der anderen Seite
vermittelt.  Ich möchte also zeigen, dass aus dem \enquote{sapere aude!}
konkrete Regeln erwachsen, die zu befolgen sind, um mündig zu sein.

Wenn hier von der Endlichkeit des Menschen gesprochen wird, dann geht es um die
Endlichkeit in der Ausübung seiner kognitiven Vermögen, also um die Endlichkeit
des \emph{Verstandes} bzw. der (theoretischen wie praktischen) \emph{Vernunft}. Es
wird sich zeigen, dass wir nach \name[Immanuel]{Kant} endlich sind,
insofern wir in uns in unserem Denken, Erkennen
und Handeln als \emph{abhängig} erweisen.\footnote{\cite[Vgl.][B
72]{Kant:KritikderreinenVernunft2003}, \cite[][III: 72.29--73.4]{Kant:GesammelteWerke1900ff.}. Siehe auch
\cite[][\S~10]{Kant:Demundisensibilisatqueintelligibilisformaetprincipiis1968},
\cite[][II: 396.19--397.4]{Kant:GesammelteWerke1900ff.}. Nach
\name[Immanuel]{Kant}s Auskunft in der \titel{Kritik der reinen Vernunft} ist
unser Verstand endlich, insofern er nur denkt, nicht aber anschaut.
Er ist in diesem Sinne ein \singlequote{diskursiver}, kein intuitiver
Verstand. Man beachte aber, dass \name[Immanuel]{Kant} in der \titel{Kritik
der reinen Vernunft} kein einziges Mal von einem diskursiven oder intuitiven
Verstand spricht; erst in \S~57 der \titel{Prolegomena zu einer jeden künftigen
Metaphysik, die als Wissenschaft wird auftreten können} und in \S~77 der
\titel{Kritik der Urteilskraft} findet sich diese Bezeichnungsweise. Bis dahin sind es
bewusste objektive Vorstellungen, die als diskursiv (Begriffe) oder intuitiv (Anschauungen)
bezeichnet werden, sowie Erkenntnisse \emph{ex principiis}, die sich in
diskursive Vernunfterkenntnisse (Philosophie) und intuitive Vernunfterkenntnisse
(Mathematik) unterteilen. Die Bezeichnung unseres Verstandes als diskursiv ist
eine spätere Übertragung der Bedeutung, die aussagt, dass der Verstand eben ein
Vermögen der Begriffe, nicht der Anschauungen ist. Siehe dazu Kap.
\ref{subsection:DiskursiverVerstandundsinnlicheAnschauung}.} Ein unendliches Wesen
wäre insofern vollkommen unabhängig oder zeichnete sich -- wie
\name[Immanuel]{Kant} 1763 schreibt -- durch \singlequote{Allgenugsamkeit}
aus.\footnote{\cite[Vgl.][A
186\,f.,]{Kant:DereinzigmoeglicheBeweisgrundvomDaseinGottes1977}
\cite[][II: 154.4--19]{Kant:GesammelteWerke1900ff.}.} Unser Verstand ist
endlich,\footnote{Man ist versucht, von der Endlichkeit unseres Verstandes oder
auch unseres Denkens zu sprechen und dagegen Gott ein unendliches Denken zuzusprechen, doch dies ist im Rahmen der
sprachlichen Gepflogenheiten \name[Immanuel]{Kant}s mindestens ungenau.
\name[Immanuel]{Kant} sagt, unser Verstand sei endlich, insofern er denkt und
nicht anschaut. Es ist also der Verstand -- und nicht das Denken -- welcher
endlich ist, insofern Denken die Tätigkeit ist, die er auszuführen
vermag. Ein unendlicher Verstand dächte nicht, sondern schaute; ein
unendliches Denken kann es nicht geben, weil Denken gerade die Tätigkeit ist,
die unserem Verstand als endlichem
zukommt \mkbibparens{\cite[vgl.][B 71]{Kant:KritikderreinenVernunft2003},
\cite[][III: 72.10--16]{Kant:GesammelteWerke1900ff.}, wonach Denken stets das
Vorliegen von Schranken beweise}. Denken -- so ließe sich auch sagen -- ist
\emph{per se} endlich, der Ausdruck \enquote{Endlichkeit des Denkens}
beschreibt eine Tautologie. Eine unendliche (das hieße: intellektuelle)
Anschauung wiederum wäre die Erkenntnis eines Vermögens der
Spontaneität, welches nicht der Rezeptivität unserer Sinne bedürfte.}
weil er als von Rezeptivität abhängiger Verstand ohne Sinnlichkeit nichts zu erkennen
vermag. Damit beschreibt die Endlichkeit eine Spannung innerhalb des oberen
Erkenntnisvermögens: Als Verstand ist dieses selbsttätig oder
\singlequote{spontan}, nur aus sich selbst heraus handelnd und unabhängig von
äußeren Bestimmungen. Als endlich ist unser Verstand wiederum abhängig. In
dieser Beschreibung der Endlichkeit kommt damit der Widerstreit zweier
Grundbestimmungen der Aufklärungsprogrammatik wieder zum Vorschein: Unserer
Unabhängigkeit und Selbständigkeit steht unsere Abhängigkeit oder Endlichkeit entgegen.

\Revision[Pelletier]{Die Tatsache, dass wir auf epistemische Vorarbeiten
Anderer angewiesen sind, stellt eine Ausprägung oder Instantiierung unserer
allgemeinen Endlichkeit dar. Die Forderung der Aufklärung, die eigene Vernunft
zum obersten Kriterium der Wahrheit zu machen, provoziert die Frage, wie diese
Forderungen mit der Tatsache kompatibel ist, dass wir die meisten Erkenntnisse
nicht aus unserer Vernunft, sondern empirisch -- aus dem Gebrauch der Sinn --
erlangen. Unser Wissen muss uns in aller Regel \emph{gegeben} werden. Um diese
Frage zu beantworten ist es sinnvoll, sich mit derjenigen Ausprägung zu
befassen, die das Projekt der Aufklärung am offensichtlichsten anzugreifen
scheint: der Angewiesenheit auf Erkenntnisse, die wir lediglich auf die
Autorität anderer hin als Wissen anerkennen. Es wird sich zeigen, dass die
Auflösung dieser Frage auch allgemein auf den scheinbaren Konflikt zwischen
Aufklärungsforderung und Abhängigkeit von empirischer Erkenntnis anwendbar ist.}

\section{Inhaltliches Vorgehen}
In Kapitel \ref{section:KantalsliberalerAufklaerer} expliziere ich die
Grundstrukturen des kantischen Aufklärungsbegriffs und erarbeite die
mit ihm verbundenen Schwierigkeiten und Herausforderungen. Die
Endlichkeit des Menschen ist Thema des \ref{chapter:endlichkeitmenschlichendenkens}.
Kapitels. Hier zeige ich, dass \name[Immanuel]{Kant} in einem einheitlichen Sinne
von der Endlichkeit unseres Verstandes spricht, die sich sowohl in der
Abhängigkeit von rezeptiv gewonnenen Informationen, als auch in dem nötigenden
Charakter der praktischen Vernunft und schließlich in der in \S~77 der
\titel{Kritik der Urteilskraft} beschriebenen Besonderheit zeigt, dass
wir vom Analytisch-Allgemeinen (von Begriffen) ausgehend   zum
Besonderen gehen müssen und nicht von einem
Synthetisch-Allgemeinen (der Anschauung des Ganzen als eines solchen)
ausgehen können. Kapitel \ref{chapter:AufklaerungundWissenschaft}
zeigt auf, dass Aufklärung und Mündigkeit primär (aber nicht
ausschließlich) unseren praktischen 
Vernunftgebrauch betreffen und auf die Endlichkeit des \emph{Willens} -- d.\,i. der
praktischen Vernunft -- verweisen. Die Kapitel
\ref{section:autonomieunddaszeugnisanderer},
\ref{chapter:MuendigerErwerbTestimonialenWissens} und
\ref{Chapter:KantsSocialEpistemology} behandeln das epistemische
Grundproblem des Aufklärungsprogramms: die Spannung zwischen
emanzipatorischem Anspruch und sozialer Wirklichkeit des Denkens und
Erkennens. Kapitel \ref{section:autonomieunddaszeugnisanderer} wird dieses
Grundproblem als solches ausarbeiten, wie es heute in der
\enquote{Soziale Erkenntnistheorie} unter dem Stichwort \enquote{testimoniales Wissen} diskutiert
wird: Während Aufklärung von uns Unabhängigkeit verlangt, sind wir
doch darin  abhängig, dass wir darauf angewiesen sind, Wissen auf
die Autorität anderer hin zu übernehmen. Die Vereinbarkeit von
aufklärerischer Forderung nach Selbständigkeit und Abhängigkeit von
testimonialem Wissen wird die Themen der folgenden Kapitel bestimmen: Kapitel
\ref{chapter:MuendigerErwerbTestimonialenWissens} zeigt auf, wie zwei
verschiedene Perspektiven auf diese Frage in der deutschen Aufklärung zu
unterscheiden sind. Dies dient der Unterscheidung einer bei
\name[Immanuel]{Kant} und anderen Aufklärern vorherrschenden Perspektive von
derjenigen, die ebenso \name[David]{Hume}s Herangehensweise leitet wie
diejenige heutiger Philosophen. Kapitel \ref{Chapter:KantsSocialEpistemology}
erläutert schließlich, wie \name[Immanuel]{Kant}s an
\authorcite{Wolff:Discursuspraeliminarisdephilosophiaingenere1996} anschließende
Herangehensweise und Lösung aussieht, der es um die Unterscheidung bloß
historischer Kenntnisse von mündigem Wissen geht. Darin wird Kapitel
\ref{Absatz:AufklaerungundZugangsInternalismus} zeigen, dass
\name[Immanuel]{Kant}s Position mitnichten individualistisch ist.
Kapitel \ref{section:MuendigkeitundPhilosophie} fragt danach, wie die Forderung
nach Unabhängigkeit im Denken und Erkennen dennoch ihren Gehalt und ihre
Berechtigung hat, und zeigt auf, welche zentrale Rolle der Metaphysik
zukommt. Kapitel \ref{section:MetaphysikausderPerspektivedesMenschen} verbindet
\name[Immanuel]{Kant}s Metaphysikbegriff mit dem Begriff der Autonomie
(als Charakteristikum oberer Erkenntnisvermögen) und findet auf diesem
Weg Anschluss an die Überlegungen zur
Aufklärung. In Kapitel \ref{section:KantsEthicsofBelief} wird
abschließend auf der Grundlage von \name[Immanuel]{Kant}s
Unterscheidungen zwischen Überredung und Überzeugung die Grundstruktur
einer kantischen \singlequote{\emph{Ethics of Belief}} erarbeitet. 


\section{Forschungsstand}
Der Konflikt zwischen Aufklärungsprogramm und Endlichkeit ist der Forschung
freilich längst bekannt. Erstaunlicherweise wurde er aber nicht zum Gegenstand
eigenständiger Untersuchungen.
Als offenes Problem der Aufklärungsforschung benannt, aber nicht ausgearbeitet
findet er sich beispielsweise bei
\authorfullcite{Engfer:ChristianThomasius1989}, der den Konflikt folgendermaßen
beschreibt: Auf der einen Seite stehe eine aufklärerische Tradition, die dem
Menschen Selbstaufklärung und Selbstbestimmung abverlange, auf der anderen Seite
die der protestantischen Tradition entstammende Überzeugung, dass dem Menschen
als endlichem Wesen gerade die Fähigkeit hierzu
fehle.\footnote{\cite[Vgl.][36]{Engfer:ChristianThomasius1989}.} Als Gegenpol zu
der von \authorcite{Engfer:ChristianThomasius1989} vertretenen
Problemschilderung tritt \authorfullcite{Schnaedelbach:WirKantianer2005} auf,
der die Aufklärung als geradezu in der Einsicht in unsere Endlichkeit fundiert
begreift. Er zählt das Programm der Aufklärung und die Einsicht in unsere
Endlichkeit zu den bewahrenswerten und grundlegenden Aspekten der Philosophie
Kants. Als Philosoph der Endlichkeit verkörpere Kant zugleich die Aufklärung und
die moderne Kultur, denn die Einsicht in unsere Endlichkeit unterminiere die
Ansprüche einer überkommenen Metaphysik und gerade dies mache das Wesen der Aufklärung
aus.\footnote{\cite[Vgl.][passim]{Schnaedelbach:WirKantianer2005}. Siehe
auch \cite[][100]{Schnaedelbach:Vernunft2007}, sowie
\cite[][976]{Schnaedelbach:PhilosophiealsGespraech2012}: \enquote{Die Frage ist
natürlich, was es heißt, jetzt Kantianer zu sein? Seine Raumkonzeption, die
Synthesistheorie des Urteils, das ganze Verhältnis von Sinnlichkeit und Verstand
kann man sich so nicht mehr zu Eigen machen. Kantianer zu sein, ist dann fast
nur noch eine Frage des Stils, -- eines Stils, der sich an der Endlichkeit
unserer Vernunft orientiert.} Die Deutung \name[Immanuel]{Kant}s als des Philosophen
der modernen Kultur orientiert sich an
\textcite[vgl.][]{Rickert:KantalsPhilosophdermodernenKultur1924}.} Auch
\authorfullcite{Hinske:KantalsHerausforderungandieGegenwart1980} sieht eine
klare Verbindung zwischen der Betonung unserer Endlichkeit und der
Aufklärung.\footnote{\enquote{Philosophie der Aufklärung ist demgemäß ihrem
eigenen Impuls zufolge Philosophie der endlichen Vernunft. Sie lebt und denkt
aus der Einsicht, daß das Ganze, die Totalität der Wahrheit, auf die Vernunft
aus ist, dem Menschen nicht gegeben, sondern nur aufgegeben ist. Sie steht im
Horizont des Unbedingten, aber sie befindet sich nicht in seinem Besitz. Sie
leidet an dieser ihrer Endlichkeit, aber sie lügt sich nicht über sie hinweg.
Eben das aber gibt ihr zugleich auch die Offenheit, die Vernunft des Anderen:
die Vernunft jedes Anderen, als ein Stück der allgemeinen Vernunft zu begreifen
und ernstzunehmen}
\parencite[][38]{Hinske:KantalsHerausforderungandieGegenwart1980}.} Die These von einem
inneren Widerstreit zwischen der Einsicht in unsere Endlichkeit und dem
Aufklärungsdenken wird also durchaus bestritten -- bis hin zur Behauptung,
Aufklärung sei Ausdruck der Einsicht in unsere Endlichkeit.

Gemeinsam ist den genannten Arbeiten, dass sie den Begriff der Endlichkeit nicht
explizit zum Thema machen. Bereits \authorfullcite{Heidegger:KantunddasProblemderMetaphysik1965}
macht darauf aufmerksam, dass die Bedeutung der Aussage, der Mensch
sei endlich, und damit auch \name[Immanuel]{Kant}s Verständnis
unserer Endlichkeit klärungsbedürftig ist.\footnote{\enquote{Wie soll nach der Endlichkeit im
Menschen gefragt werden? Ist das überhaupt ein ernsthaftes Problem? Liegt die
Endlichkeit des Menschen nicht allerorts und jederzeit tausendfältig zutage?\\
So mag es schon genügen, Endliches am Menschen zu nennen, aus seinen
Unvollkommenheiten beliebige anzuführen. Auf diesem Wege gewinnen
wir allenfalls Belege dafür, daß der Mensch ein endliches Wesen
ist. Wir erfahren aber weder, worin das Wesen seiner Endlichkeit
besteht, noch gar, wie diese Endlichkeit den Menschen als das
Seiende, das er ist, von Grund aus im ganzen
bestimmt} \parencite[][198]{Heidegger:KantunddasProblemderMetaphysik1965}.} Bis
heute gibt es jedoch keine eigenständige Untersuchung der Frage, was die
Endlichkeit des Menschen nach \name[Immanuel]{Kant} ausmacht und wie sie sich
zum Projekt der Aufklärung verhält.\footnote{Nicht einschlägig ist hier die
Auffassung \name[Immanuel]{Kant}s von einem mathematischen Endlichen und
Unendlichen, wie es etwa in der Antithetik diskutiert wird. Dies werde ich in
Kap. \ref{subsection:QuantitativeundqualitativeUnendlichkeit} darlegen.}
Mitunter wird sie eher beiläufig und ohne nähere Auseinandersetzung
beantwortet,\footnote{So behauptet etwa
  \authorfullcite{Sandkuehler:KantsenquoteRevolutionderDenkungsart2005}:
\enquote{Was Kant letztlich mit der systematischen Kritik der Erkenntnis
anstrebt, ist Aufklärung -- der \enquote{Ausgang aus selbstverschuldeter
Unmündigkeit}. Es geht ihm um die Ermöglichung von Selbstdenken ohne Widerspruch
und mit Rücksicht auf andere}
\parencite[][93]{Sandkuehler:KantsenquoteRevolutionderDenkungsart2005}. Gründe
für diese Interpretation liefert er jedoch nicht.} damit aber gerade nicht als
drängendes Problem wahrgenommen. Dabei ist die Frage nach unserer Endlichkeit
gerade im Zusammenhang mit dem Programm der Aufklärung nicht nur
philosophiegeschichtlich höchst interessant, sondern betrifft eine Grundfrage
von bleibender Aktualität. Denn zum einen möchte auch heute niemand hinter die
Aufklärung zurück. Selbst die vielfachen Forderungen nach einer
\enquote{Aufklärung der Aufklärung}\footcite[Vgl.][]{Gutschmidt:AufklaerungderAufklaerung2012} oder die
Behauptungen einer \enquote{Dialektik der
Aufklärung}\footcite[Vgl.][]{Horkheimer:DialektikderAufklaerung1997}
beanspruchen ja eine Weiterentwicklung und keine Zurücknahme der Aufklärung
\emph{toto genere}. Auf der anderen Seite bestreitet niemand unsere grundlegende
Endlichkeit. Auch Hegelianer wie \authorfullcite{Stekeler-Weithofer:TheQuestionofSystem2006}
behaupten, unsere Endlichkeit lediglich besser verstehen zu wollen. Sie
entlasten Hegel von Behauptungen, die das Unendliche betreffen,
oder unterstellen ihm einen unverfänglichen Sinn von
Unendlichkeit.\footnote{\authorfullcite{Stekeler-Weithofer:TheQuestionofSystem2006}
schreibt über \name[Georg Wilhelm Friedrich]{Hegel}: \enquote{Grob gesagt schlägt er vor, die Rede über
Unendlichkeit im allgemeinen, über ein absolutes oder unendliches Wissen im
besonderen als ideale Rede zu begreifen. Sie ist Rede über die invariante Form
einer guten Entwicklung der je in ihrer Endlichkeit relativen Wahrheiten, des je
bestmöglichen Wissens und Bereifens}
\parencite[][186]{Stekeler-Weithofer:PhilosophiedesSelbstbewusstseins2005}.
Siehe auch \cite[][98]{Stekeler-Weithofer:TheQuestionofSystem2006}.
Ähnliches behauptet
\cite{Chiereghin:WozuHegelineinemZeitalterderEndlichkeit?1998}; dieser
Interpretation widerspricht wiederum \cite{Philipsen:NichtsalsKontexte2000}.}
Und \authorfullcite{Foerster:Die25JahrederPhilosophie2011}, der zuletzt durch
die an \name[Johann Wolfgang von]{Goethe} orientierte Behauptung auffiel, wir
verfügten über einen intuitiven Verstand, stellt zugleich klar, dieser
sei nicht mit einem unendlichen, göttlichen
Verstand zu identifizieren; vielmehr handle es sich auch bei dem Verstand, den
\name[Immanuel]{Kant} selbst in der \titel{Kritik der Urteilskraft} unserem
diskursiven Verstand entgegenstellt, um einen \enquote{\ori{endlichen}
intuitiven
Verstand}\footnote{\cite[][256]{Foerster:Die25JahrederPhilosophie2011}.}.

Obwohl die Grundfrage dieser Arbeit bisher keine eigenständige Untersuchung zu
Tage förderte, sind zentrale Themen freilich bereits in der
\name[Immanuel]{Kant}forschung diskutiert worden.
\authorfullcite{Engfer:MenschlicheVernunft2002},
\authorfullcite{Foerster:DieBedeutungvonSS7677deremphKritikderUrteilskraftfuerdieEntwicklungdernachkantischenPhilosophieTeil12002}
und \authorfullcite{Nuzzo:KritikderUrteilskraftSS76--772009} diskutieren die
\S\S~76 und 77 der \titel{Kritik der Urteilskraft}.
\authorcite{Engfer:MenschlicheVernunft2002} argumentiert dafür, dass zentrale
Ansätze der kritischen Philosophie von Annahmen über die Beschaffenheit unseres
menschlichen Erkenntnisvermögens abhängig
sind.\footcite[Vgl.][]{Engfer:MenschlicheVernunft2002}
\authorcite{Foerster:Die25JahrederPhilosophie2011} möchte nachweisen, dass
\name[Immanuel]{Kant} keinen einheitlichen Begriff unserer Endlichkeit
verwendet, sondern auf verschiedene Besonderheiten unseres Erkenntnisvermögens
verweist.\footnote{Siehe
\cite{Foerster:DieBedeutungvonSS7677deremphKritikderUrteilskraftfuerdieEntwicklungdernachkantischenPhilosophieTeil12002,Foerster:DieBedeutungvonSS7677deremphKritikderUrteilskraftfuerdieEntwicklungdernachkantischenPhilosophieTeil22002},
und \cite{Foerster:Die25JahrederPhilosophie2011}. Dasselbe behauptete
\textcite[vgl.][153--159]{McLaughlin:KantsKritikderteleologischenUrteilskraft1989}.}
\authorcite{Nuzzo:KritikderUrteilskraftSS76--772009} wiederum behauptet, dass es
sich in den \S\S~76 und 77 der \titel{Kritik der Urteilskraft} stets um
Beschreibungen derselben Besonderheit unseres Verstandes handelt -- der
Diskursivität -- und dass \name[Immanuel]{Kant} in diesen Passagen die Abgeschlossenheit der
Transzendentalphilosophie aufzeige.\footnote{Siehe
\cite{Nuzzo:KritikderUrteilskraftSS76--772009}, sowie
\cite[][348--353]{Nuzzo:KantandtheUnityofReason2005}. \enquote{\S\S~76--77 prove that Kant's transcendental philosophy
is already completed, and that it cannot be corrected without inaugurating an
utterly different paradigm}
\parencite[][146]{Nuzzo:KritikderUrteilskraftSS76--772009}.}
\authorfullcite{Duesing:DieTeleologieinKantsWeltbegriff1968} diskutiert die
Begriffe von diskursivem und intuitivem Verstand im Rahmen seiner Untersuchungen
zur Teleologie.\footnote{Siehe
\cite[][66--74]{Duesing:DieTeleologieinKantsWeltbegriff1968}, sowie
\cite[][144--147]{Duesing:NaturteleologieundMetaphysikbeiKantundHegel1990}.}
\authorfullcite{Allison:KantsTranscendentalIdealism2004}\footcite[Vgl.][]{Quarfood:DiscursivityandTranscendentalIdealism2012}
und
\authorfullcite{Quarfood:DiscursivityandTranscendentalIdealism2012}\footnote{\cite[Vgl.][]{Allison:KantsTranscendentalIdealism2004}.
Siehe dazu auch \cite{Pippin:IdealismandFinitude2008}.} thematisieren die
Diskursivität des Verstandes als Fundament des transzendentalen Idealismus.
\authorcite{Engfer:MenschlicheVernunft2002} sieht vor allem das Eingeständnis
expliziert, dass die Transzendentalphilosophie auf Voraussetzungen beruhe, die
sie selbst nicht in der Lage sei einzuholen.\footnote{\enquote{Was bedeutet
diese absichtsvoll wiederholte Berufung auf die besondere Beschaffenheit des
menschlichen Erkenntnisvermögens am Ende der Kritik der Urteilskraft systematisch
für das Ganze der kritischen Philosophie? Sie bedeutet
offenbar, daß \name[Immanuel]{Kant} jedenfalls an dieser Stelle die These vertritt, daß bestimmte
zentrale Ansätze und Gelenkstellen der kritischen Philosophie sich als
Konsequenzen spezifischer Eigentümlichkeiten der menschlichen Vernunft erweisen
und insofern trotz der scheinbar dagegen sprechenden Äußerungen \name[Immanuel]{Kant}s von
spezifisch menschlichen Voraussetzungen abhängig zu sein scheinen, die dann
trotz der entgegengesetzten Beteuerungen \name[Immanuel]{Kant}s die Basis für wesentliche
Aussagen der kritischen Philosophie sein könnten. Denn hier werden ja nicht etwa
marginale Bestimmungen, sondern zentrale und wesentliche Begriffe und
systematische Voraussetzungen aller drei Hauptschriften der kritischen
Philosophie selbst thematisiert}
\parencite[][\pno~272\,f.]{Engfer:MenschlicheVernunft2002}.} Zugleich zeigten
die Überlegungen aber auf, dass die Endlichkeit des Menschen auch einen Vorzug des
Menschen bedinge: Die Unterscheidungen von Möglichkeit und Wirklichkeit sowie
von Sein und Sollen artikulierten nicht nur Unvollkommenheiten des menschlichen
Verstandes, sondern seien zugleich Zeichen für die Freiheit des
Menschen.\footnote{\cite[Vgl.][283]{Engfer:MenschlicheVernunft2002}.}

Ähnlich wie mit dem Begriff der Endlichkeit verhält es sich mit dem der
Aufklärung; wobei hier der Vorteil vorliegt, dass \name[Immanuel]{Kant} einen
eigenständigen Text dazu verfasst hat, der schon wegen seiner Bekanntheit
Interpretationen provoziert.\footnote{Die Masse an Publikationen zu \name[Immanuel]{Kant}s
Beitrag in der \titel{Berlinischen Monatsschrift} ist, wie kaum überrascht,
unüberschaubar. Interessant sind hier v.\,a. diejenigen Beiträge, die auf die
Verbindung zu weiteren Aspekten seiner Philosophie, insbesondere der
Vernunftkritik, eingehen. So zeigt
\authorfullcite{Allison:KantsConceptionofemphAufklaerung2012}, dass
\name[Immanuel]{Kant} eine komplexe und detaillierte Aufklärungskonzeption
entwirft, die zu den grundlegendsten seiner philosophischen Überzeugungen in Verbindung steht
\parencite[vgl.][]{Allison:KantsConceptionofemphAufklaerung2012}.
\authorfullcite{Schmidt:WhatEnlightenmentWas1992} interpretiert
\name[Immanuel]{Kant}s Aufsatz als Reinterpretation des Aufklärungsprogramms auf
der Grundlage der Vernunftkritik
\parencite[vgl.][]{Schmidt:WhatEnlightenmentWas1992}.
\authorfullcite{Scholz:KantsAufklaerungsprogramm2009} möchte schließlich zeigen,
dass \name[Immanuel]{Kant}s Aufklärungsprogramm als Fluchtpunkt seiner gesamten
Philosophie angesehen werden kann
\parencite[vgl.][]{Scholz:KantsAufklaerungsprogramm2009,Scholz:BeantwortungderFrage:WasisteinaufgeklaerteWeltbuerger2011}.
Vor allem in den Arbeiten \authorfullcite{Hinske:ArtikelAufklaerung1985}s finden sich Argumente
für die Verbindung von Aufklärungsprogramm und Vernunftkritik
\parencite{Hinske:KantsVernunftkritik--FruchtderAufklaerungundoderWurzeldesDeutschenIdealismus1993,Hinske:ZwischenAufklaerungundVernunftkritik1998,Hinske:ZwischenAufklaerungundVernunftkritik1993}.
\authorfullcite{LaRocca:WasAufklaerungseinwird2004} thematisiert
Implikationen des Aufklärungsprogramms für den Begriff der Rationalität
\parencite[vgl.][]{LaRocca:WasAufklaerungseinwird2004}. Verbindungen zwischen
\name[Immanuel]{Kant}s Aufklärungsbegriff und anderen Aspekten seiner
Philosophie behauptet des weiteren auch Robert
\textcite[vgl.][]{Theis:KantetlAufklaerung2012}.
\Revision[Theis,
Pelletier]{\authorfullcite{Foucault:DieRegierungdesSelbstundderanderen2009}
nimmt \name[Immanuel]{Kant}s Aufklärungsaufsatz vielfach auf; siehe bspw.
\cite{Foucault:WasistAufklaerung1990}, und
\cite{Foucault:DieRegierungdesSelbstundderanderen2009}. Dabei sieht er die
Themen der drei Kritiken bereits in der Beschreibung der Unmündigkeit
angesprochen
\parencite[vgl.][\pno~49\,f.]{Foucault:DieRegierungdesSelbstundderanderen2009}.
Siehe weiter \cite{Foucault:WasistAufklaerung1990}, sowie
\cite{Foucault:WasistKritik?1992}. Dabei betont er selbst die Notwendigkeit,
eine andere Form unserer Abhängigkeit zu thematisieren: Demzufolge gehe es
heute nicht mehr um eine transzendentale Analyse von formalen Strukturen mit
universeller Bedeutung, sondern um die archäologische Aufdeckung all derjenigen
geschichtlichen Ereignisse und Umstände, durch die wir uns selbst erst
konstituieren und als die Subjekte unseres Handelns und Denkens ansehen können
\parencite[vgl.][]{Foucault:WasistAufklaerung1990}. Offensichtlich setzt sich
\name[Immanuel]{Kant} mit dieser Form von Abhängigkeit nicht auseinander; ob
dies ein ernsthaftes Manko ist, wird in vorliegender Arbeit jedoch offen
bleiben.}} Jedoch wird zumeist die Frage ausgeklammert, was die Forderung nach
Selbstdenken und eigenem Vernunftgebrauch von uns konkret fordert. Dass es bei
\name[Immanuel]{Kant} in diesem Sinne eine \emph{Ethics of Belief} zu
rekonstruieren gilt, ist erst in den letzten Jahren bewusst
geworden.\footnote{Dass die meisten Arbeiten zu \name[Immanuel]{Kant}s
Erkenntnistheorie seine \emph{ethics of belief} ignorieren, bedauert
\authorfullcite{Chignell:BeliefinKant2007}, der im Begriff des Glaubens einen
tragfähigen Ansatz zu einer Theorie nicht-epistemischer Rechtfertigungen sieht
\parencite[vgl.][]{Chignell:KantsConceptsofJustification2007}.
\authorcite{Chignell:KantsConceptsofJustification2007} möchte den Begriff des
Glaubens auch für eine \singlequote{liberale Metaphysik} nutzbar
machen, die durch Abschwächung der Geltungsbedingungen auch Aussagen über Dinge an sich auf Grundlage des
theoretischen Vernunftgebrauchs erlaube
\parencite[vgl.][335--360]{Chignell:BeliefinKant2007}: \enquote{Given
  this situation, we can and should go ahead and build metaphysical
  arguments in all the usual ways, by appealing to
  \enquote{intuitions} (of the \name[George Edward]{Moore}an rather than the
  \name[Immanuel]{Kant}ian sort), reflective equilibrium, inference to
best explanation, simplicity, and so forth} \parencite[][360]{Chignell:BeliefinKant2007}.
Wie \authorcite{Chignell:BeliefinKant2007} rekurriert
\authorfullcite{Stevenson:OpinionBelieforFaithandKnowledge2003} primär auf die
Überlegungen des Kapitels \titel{Vom Meinen, Wissen und Glauben} der
\titel{Kritik der reinen Vernunft}
\parencite[vgl.][]{Stevenson:OpinionBelieforFaithandKnowledge2003}.
\authorfullcite{Cohen:KantontheEthicsofBelief2014} verlässt den engen Rahmen
der Überlegungen zu Meinen, Wissen und Glauben und greift auf Überlegungen der
Moralphilosophie zurück, um \name[Immanuel]{Kant}s \emph{ethics of belief} zu
interpretieren \parencite[vgl.][]{Cohen:KantontheEthicsofBelief2014}.} Dazu
gehört insbesondere die Frage nach konkreten Regeln im Umgang mit dem sozialen
Charakter unseres Wissens und Erkennens. Die Diskussion um
\name[Immanuel]{Kant}s Position zu testimonialem Wissen -- um die wichtigste
Frage in diesem Zusammenhang herauszugreifen -- ist in einem 
Aufsatz von \authorfullcite{Schmitt:JustificationSocialityandAutonomy1987}
erstmalig diskutiert worden.
\authorcite{Schmitt:JustificationSocialityandAutonomy1987} sieht
\name[Immanuel]{Kant}s Aufklärungsprogrammatik dabei als Grund an, ihm eine
individualistische Position zu unterstellen, die mit unserer epistemischen
Wirklichkeit nicht kompatibel
ist.\footnote{\cite[Vgl.][46]{Schmitt:JustificationSocialityandAutonomy1987}.
Prominenten Ausdruck findet die Behauptung eines \index{Kant,
Immanuel}kantischen Individualismus bei Karl-Otto
\textcite[][passim]{Apel:DasAprioriderKommunikationsgemeinschaft1976}.} In den
letzten Jahren wurde dem widersprochen, insbesondere von
\authorfullcite{Scholz:DasZeugnisanderer2001}\footnote{Siehe
\cite{Scholz:AutonomieangesichtsepistemischerAbhaengigkeiten2001},
\cite{Scholz:DasZeugnisanderer2001},
\cite{Scholz:enquotedotsdenoberstenProbiersteinderWahrheitinsichselbstd.i.inseinereigenenVernunftsuchen2004},
und\cite{Scholz:Aufklaerung:VonderErkenntnistheoriezurPolitik2006}.} und
\authorfullcite{Gelfert:KantonTestimony2006}\footnote{Siehe \cite{Gelfert:KantonTestimony2006}, sowie
\cite{Gelfert:KantandtheEnlightenmentsContributiontoSocialEpistemology2010}.},
aber auch von weiteren Autoren wie
\authorfullcite{Mikalsen:TestimonyandKantsIdeaofPublicReason2010}.\footnote{\cite[Vgl.][]{Mikalsen:TestimonyandKantsIdeaofPublicReason2010}.
Die offenere Fragestellung, inwieweit \name[Immanuel]{Kant} Öffentlichkeit als
Notwendigkeit für unseren Vernunftgebrauch ansieht, stößt freilich auf mehr
Interesse in der Forschungslandschaft. Siehe dazu etwa
\cite{Hoeffe:EinerepublikanischeVernunft1996},
\cite{Deligiorgi:UniversalisabilityPublicitaandCommunication2002}, sowie
\cite{Deligiorgi:KantandtheCultureofEnlightenment2005}.}

\section{Anmerkungen zum methodischen Vorgehen}

Die Arbeit versteht sich weder als ideengeschichtlicher Beitrag, der die Autoren
fernab jeder Bewertung aus heutiger Sicht lediglich in ihrem historischen
Kontext darstellen möchte, noch als rationalisierende Interpretation, die sich
der Untersuchung historischer Umstände und Bezüge enthält. Mir geht es um ein
Verständnis der Position \name[Immanuel]{Kant}s, was sowohl erfordert, ihn aus
heutiger Perspektive -- mit heutigem Wissen und heutiger Terminologie -- zu
beschreiben, als auch, ihn in seiner Zeit zu betrachten.
Dabei schien es bei der Erstellung der Arbeit oft wenig ergiebig zu sein, der
Genese der Überlegungen in \name[Immanuel]{Kant}s diachroner Entwicklung nachzuspüren. Eine
umfangreichere Entwicklungsgeschichte des Aufklärungsdenkens oder des
Endlichkeitsbegriffs ist im Werk \name[Immanuel]{Kant}s nicht zu finden; eine
Entwicklungsgeschichte verwandter Themen wie der Metaphysik würde den Rahmen
der vorliegenden Arbeit sprengen. Einen größeren Gewinn verspricht der Versuch,
systematische Zusammenhänge zwischen den speziellen Darlegungen zu den hier
anvisierten Fragen auf der einen und den Grundstrukturen der kritischen
Philosophie auf der anderen Seite zu rekonstruieren.


Die Authentizität der Interpretation ist dabei nur dadurch zu gewährleisten,
dass relevante und zuverlässige Quellen ausgewertet werden. Die von \name[Immanuel]{Kant} selbst
veröffentlichten Werke genießen hierbei Vorrang
gegenüber unveröffentlichten Schriften wie Briefen, Vorlesungsmitschriften,
Reflexionen und dem \titel{Opus postumum}. Dabei ist neben der zeitlichen
Einordnung der Vorlesung und der Zuverlässigkeit der Mitschriften, von denen mitunter weder
das Datum noch der Mitschreiber bekannt sind, fraglich, inwieweit \name[Immanuel]{Kant} in diesen
Vorlesungen überhaupt eigene Ansichten referiert. Schließlich lag z.\,B.
seinen Metaphysikvorlesungen kein eigenständiges Konzept und keine eigene
Schrift, sondern die Metaphysik
\authorcite{Baumgarten:Metaphysica---Metaphysik2011}s zugrunde. Deswegen eignen
sich die Vorlesungsmitschriften nur bedingt als Maßstab der
Interpretation,\footnote{Dass man auch die Vorlesungsmitschriften gewinnbringend
anführen kann, wenn die Schwierigkeiten stets bewusst bleiben, demonstrieren
Autoren wie
\textcite[vgl.][passim]{Hinske:ZwischenAufklaerungundVernunftkritik1998}.} zumal
\name[Immanuel]{Kant} selbst gerade in den drei Kritiken hinreichend Auskunft zu den hier
interessierenden Fragen erteilt. Damit ist der Nutzen von Vorlesungsmitschriften
und Notizen \name[Immanuel]{Kant}s nicht in Abrede gestellt. Sie sind in ihrer Bedeutung von solchen
Schriften, die \name[Immanuel]{Kant} selbst publiziert und autorisiert hat, dennoch
grundlegend verschieden. Als
Maßstab der Interpretation dienen in erster Linie die \titel{Werke}
\name[Immanuel]{Kant}s, wie sie sich in der ersten Abteilung der
Akademieausgabe finden. Der neunte und letzte Band dieser Abteilung, der die
\titel{Logik}, die \titel{physische Geographie} und die \titel{Pädagogik}
enthält, nimmt dabei eine Sonderrolle ein, insofern es sich letztlich gar nicht
um von \name[Immanuel]{Kant} verantwortete und autorisierte Publikationen
handelt.\footnote{Nach
\authorfullcite{Stark:DieKant-AusgabederBerlinerAkademie--EineMusterausgabe2000}
ist dieser Band \enquote{[g]egen den Stand der Wissenschaft {\punkt}
erschienen}
\parencite[][216]{Stark:DieKant-AusgabederBerlinerAkademie--EineMusterausgabe2000},
die physische Geographie sei sogar \enquote{wissenschaftlich wertlos}
\parencite[][214]{Stark:DieKant-AusgabederBerlinerAkademie--EineMusterausgabe2000}.}
Dabei hat insbesondere die von \name[Gottlob Benjamin]{Jäsche} auf Geheiß
\name[Immanuel]{Kant}s hin erstellte {\jaeschelogik} in den letzten Jahren
vermehrt das Interesse der \name[Immanuel]{Kant}forschung geweckt, obwohl schon
seit langer Zeit Bedenken bezüglich der Zuverlässigkeit als Quelle für die
Positionen und Gedanken artikuliert
werden.\footnote{\phantomsection\label{Anmerkung:Einleitung:VorbehaltegegenueberderJaescheLogik}Die
{\jaeschelogik} wurde zwar auf der Grundlage von \name[Immanuel]{Kant}s
Durchschussexemplar von \authorfullcite{Meier:Vernunftlehre1752}s
\titel{Auszug aus der Vernunftlehre} und im Auftrag \name[Immanuel]{Kant}s, aber
letztlich selbständig von Gottlob Benjamin \name[Gottlob Benjamin]{Jäsche}
ausgearbeitet \mkbibparens{vgl.\ hierzu \name[Gottlob Benjamin]{Jäsche}s eigene
Auskunft in \cite{Kant:ImmanuelKantsLogik1977}, \cite[][IX:
3.2--4.14]{Kant:GesammelteWerke1900ff.}; siehe auch
\cite[][52--55]{Rameil:KantueberLogikalsVernunftwissenschaft2004}, und
\cite{Hinske:DieemphJaesche-LogikundihrbesonderesSchicksalimRahmenderAkademie-Ausgabe2000}}.
\authorfullcite{Reich:DieVollstaendigkeitderkantischenUrteilstafel1932} lehnt
die Beachtung der {\jaeschelogik} rundweg ab, weil es ihr ersichtlich an
Qualität mangle
\parencite[vgl.][21--24]{Reich:DieVollstaendigkeitderkantischenUrteilstafel1932}.
Von erheblicher Bedeutung ist aus der Sicht der vorliegenden Untersuchung, dass
\authorcite{Reich:DieVollstaendigkeitderkantischenUrteilstafel1932} zur
Begründung seines Urteils auf Unstimmigkeiten im IX. Abschnitt der Einleitung verweist, in der
unterschiedliche Formen des Fürwahrhaltens thematisiert werden, die hier in
Kapitel \ref{section:KantsEthicsofBelief} untersucht werden. Dem Urteil
\authorcite{Reich:DieVollstaendigkeitderkantischenUrteilstafel1932}s schließt
sich \authorfullcite{Stuhlmann-Laeisz:KantsLogik1976} an, der zusätzlich die
Unverständlichkeit der Ausführungen \name[Gottlob Benjamin]{Jäsche}s  zum
Verhältnis von hypothetischen und kategorischen Urteilen bemängelt
\parencite[vgl.][1]{Stuhlmann-Laeisz:KantsLogik1976}. Bedauerlich ist die
mangelnde Authentizität des Textes v.\,a., weil sich dort zusammenhängende
Überlegungen zu Form und Bildung von Begriffen finden, die man in den
eigenhändig verfassten Werken \name[Immanuel]{Kant}s vergeblich sucht. Diese
Überlegungen sind insbesondere im Rahmen der Analyse der Diskursivität in
Kapitel \ref{chapter:endlichkeitmenschlichendenkens} relevant. Ähnliche
Bedenken wie gegenüber der {\jaeschelogik} sind bezüglich der von Friedrich Theodor
\name[Friedrich Theodor]{Rink} bearbeiteten \enquote{physische Geographie}
und \enquote{Pädagogik} zu berücksichtigen. Beide sind auf ähnliche
Weise wie die {\jaeschelogik} entstanden \mkbibparens{\cite[siehe][IX:
  155.12--24, 439.5--9]{Kant:GesammelteWerke1900ff.}}.} Diese
Schriften werden hier dennoch nicht einfach ignoriert -- wie dies
\authorfullcite{Stuhlmann-Laeisz:KantsLogik1976} im Anschluss an
\authorfullcite{Reich:DieVollstaendigkeitderkantischenUrteilstafel1932}
beschließt\footnote{\cite[Vgl.][1]{Stuhlmann-Laeisz:KantsLogik1976}, sowie
\cite[][\pno~24, Anm.
11]{Reich:DieVollstaendigkeitderkantischenUrteilstafel1932}. Wenn letzterer den
Wert der {\jaeschelogik} für das Studium \name[Immanuel]{Kant}s betont, sie
aber nicht als selbständiges Beweisstück heranzieht, dann ist dies weitgehend
im Sinne dieser Arbeit. Die {\jaeschelogik} hat -- wie Reflexionen und
Vorlesungsnachschriften -- mehr Indizien- als Beweischarakter. Allerdings kann ich
das Vertrauen, welches
\authorcite{Reich:DieVollstaendigkeitderkantischenUrteilstafel1932} und
\authorcite{Stuhlmann-Laeisz:KantsLogik1976} auf der anderen Seite in die Vorlesungsnachschriften
setzen, nicht nachvollziehen. Für sie gilt m.\,E. dasselbe, was auch für die
{\jaeschelogik} gilt.} --, sondern mit der gebotenen Vorsicht herangezogen.
Letztlich soll in den von \name[Immanuel]{Kant} selbst
publizierten und \emph{autorisierten} Schriften das wichtigste Kriterium gesehen
werden.\footnote{Siehe zu diesem Ansatz auch
\textcite[][21--23]{Schwaiger:KategorischeundandereImperative1999}, der sich
damit von \textcite[vgl.][6--8]{Schmitz:WaswollteKant1989} abgrenzt.} Briefe,
Notizen, Reflexionen und Vorlesungsmitschriften sind gute Hilfsmittel bei der
Erstellung von Interpretationen. Aber die autorisierten Publikationen sind
letztlich das Tribunal, vor dem jede Interpretation sich rechtfertigen muss.

\phantomsection\label{Einleitung:AbschnittIdealistennachKant}
\name[Immanuel]{Kant}s Überlegungen zu unserer Endlichkeit und seine Konzeption
eines nicht\-dis\-kur\-si\-ven Verstandes und einer intellektuellen statt
sinnlichen Anschauung haben auch weite Teile der an ihn anschließenden Philosophien
geprägt. Viele dieser Diskussionen fanden in kritischer Auseinandersetzung mit
\name[Immanuel]{Kant}s Positionen und seinem Verhältnis zu Vorläufern wie
\authorfullcite{Leibniz:Meditationesdecognitioneveritateetideis1999} statt. Es
ist verlockend, in einer Darstellung der Philosophie
\name[Immanuel]{Kant}s auf diese Autoren einzugehen, schon um deren Verständnis
nutzbar zu machen. Jedoch steht dem entgegen, dass Autoren wie
\name[Salomon]{Maimon}\footnote{Siehe dazu
\cite{Atlas:SolomonMaimonsDoctrineofInfiniteReasonanditsHistoricalRelations1952},
sowie \cite[][326--361]{Kroner:VonKantbisHegel2007}, und
\cite{Ehrensperger:WeltseeleundunendlicherVerstand2006}.},
\authorcite{Fichte:DieBestimmungdesMenschen1800},
\name[Friedrich Wilhelm Joseph]{Schelling} und
\authorcite{Hegel:GesammelteWerke} in jeweils eigenständiger und mitunter
origineller, dabei aber nicht immer fairer Auseinandersetzung an
\name[Immanuel]{Kant} anschlossen. Ihre Einwände und Entgegnungen wären jeweils
ein eigenständiges Thema von gehörigem Umfang, weswegen in dieser Arbeit auf die
Auseinandersetzung mit solchen Autoren verzichtet wird. Der zu erwartende
Mehrwert läge sicherlich unter den beachtlichen Verlusten an Genauigkeit und
Tiefe im Durchdringen der Texte, die eine weitere thematische Verbreiterung mit
sich brächte.

Abschließend seien noch ein paar allgemeine Angaben zur Zitierweise gemacht.
Die Schriften \name[Immanuel]{Kant}s werden nach den im Literaturverzeichnis angegebenen
Ausgaben zitiert und durch Angabe der jeweiligen Schrift (als Siglum) und des
Ortes in der Akademieausgabe angegeben. Dabei wird auf die Akademieausgabe
in der Form \enquote{\cite{Kant:GesammelteWerke1900ff.} [Band]:
[Seite].[Zeile(n)]} mit Bandangabe in römischen sowie Seiten- und Zeilenangaben
in arabischen Ziffern und vorhergehendes Siglum verwiesen. Somit
verweist die Angabe \enquote{\cite{Kant:KritikderreinenVernunft2003},
\cite[][III: 108.16]{Kant:GesammelteWerke1900ff.}} auf die Stelle in der Kritik
der reinen Vernunft, die in der Akademieausgabe in Band III auf Seite 108 in der
16. Zeile zu finden ist. Wenn bei anderen Autoren eine mehrbändige Werkausgabe
vorliegt, verfahre ich analog.

