Es ist umstritten, in welchem Verhältnis \name[Immanuel]{Kant}s
Aufklärungsdenken zur Vernunftkritik und zur kritischen Philosophie insgesamt
steht.\footnote{\enquote{\name[Immanuel]{Kant}s Verhältnis zur Aufklärung
scheint also die verschiedensten Auslegungen zuzulassen, ja es scheint nicht weniger zu schillern
als der Begriff der Aufklärung selber}
\parencite[][31]{Hinske:KantalsHerausforderungandieGegenwart1980}.} Oliver
\authorcite{Scholz:DasZeugnisanderer2001} behauptet: \enquote{Kants gesamte
theoretische und praktische Philosophie dient direkt oder indirekt der
Begründung der Ideen der
Aufklärung.}\footnote{\cite[][30]{Scholz:KantsAufklaerungsprogramm2009}, siehe
auch
\cite{Scholz:enquotedotsdenoberstenProbiersteinderWahrheitinsichselbstd.i.inseinereigenenVernunftsuchen2004}
sowie \cite[][23]{Recki:KantunddieAufklaerung2006}: \enquote{Die in der
Aufforderung \enquote{Sapere aude} geforderte Selbstbefreiung des Menschen
dürfen wir als das Leitmotiv der gesamten -- theoretischen und praktischen --
Philosophie Kants ansehen.}. Auch
\authorfullcite{Wood:KantandtheProblemofHumanNature2003} scheint diese Ansicht
zu vertreten, wenn er -- m.\,E. stark übertreibend -- schreibt:
\enquote{Kant's critical philosophy as a whole is the greatest and most
characteristic product of the intellectual and social movement, known as
\enquote{the Enlightenment,} which remains the unique source in the world for
all progressive thought and action (at least insofar as it has its roots
anywhere in the Western tradition)}
(\cite[][56]{Wood:KantandtheProblemofHumanNature2003}).}
\Revision[Theis, Pelletier]{Ähnlich
\authorfullcite{Foucault:DieRegierungdesSelbstundderanderen2009} in der
Beschreibung der Unmündigkeit die Themen aller drei Kritiken angesprochen.
Wenngleich die Beziehung von \name[Immanuel]{Kant} selbst nicht explizit
angesprochen werde, wende er sich mit der \titel{Kritik der reinen Vernunft}
doch gerade gegen die Abhängigkeit von geistigen
Autoritäten.}\footnote{\Revision[Theis, Pelletier]{\enquote{Die kritischen
Grenzen zu überschreiten und sich der Autorität eines anderen zu unterstellen, das sind die beiden Seiten
dessen, wogegen \name[Immanuel]{Kant} sich in der \emph{Kritik} erhebt,
dasjenige, von dem der Prozeß der Aufklärung selbst uns befreien soll}
\parencite[][\pno~50\,f.]{Foucault:DieRegierungdesSelbstundderanderen2009}.
Siehe auch \cite[][41]{Foucault:WasistAufklaerung1990}: \enquote{In einem
gewissen Sinne ist die Kritik das Handbuch der in der \ori{Aufklärung} mündig
gewordenen Vernunft, und umgekehrt ist die \ori{Aufklärung} das Zeitalter der
Kritik.}}} Dagegen ist die Deutung verbreitet, dass
\name[Immanuel]{Kant} sich \emph{neben} seinen metaphysikkritischen
Untersuchungen von den Herausgebern der Berlinischen Monatsschrift \emph{auch
noch} zu einer Beschäftigung mit den Themen der Aufklärung hat hinreißen lassen.
Zumindest wird die Vernunftkritik oft genug außerhalb des Kontextes der
Aufklärungsphilosophie
gelesen.\footnote{\cite[Vgl.][47]{Scholz:BeantwortungderFrage:WasisteinaufgeklaerteWeltbuerger2011}.}
Mitunter wird die These vertreten, dass \name[Immanuel]{Kant}s Vernunftkritik
gerade den für die Aufklärung konstitutiven Erkenntnisoptimismus und damit die
Aufklärung selbst überwinde.\footnote{Siehe beispielsweise die allerdings stark
tendenziöse Betrachtung von Magnus
\textcite{Selling:DieUeberwindungderAufklaerung1942}.} Oder es wird auf den
synthetischen Charakter der Vernunftkritik gegenüber einer analytischen
Ausrichtung der Aufklärung als einer \enquote{Philosophie der Analysis}
verwiesen, um einen Bruch \name[Immanuel]{Kant}s mit dem Aufklärungsdenken durch
die Vernunftkritik aufzuweisen.\footnote{\enquote{[W]ährend die Philosophie der
Aufklärung, was \name[Immanuel]{Kant}s eigene Generation angeht, ihrer ganzen
Tendenz nach eine Philosophie der Analysis ist, analytische Philosophie,
versteht sich die \ori{Kritik der reinen
Vernunft} in bewußtem Gegensatz dazu als Philosophie der Synthesis}
\parencite[][34]{Hinske:KantalsHerausforderungandieGegenwart1980}.}
\authorfullcite{Schneiders:AufklaerungundVorurteilskritik1983} nennt das
Erscheinen der \titel{Kritik der reinen Vernunft} zusammen mit dem Ausgang der
Französischen Revolution als ein entscheidendes Ereignis, das den Übergang von
der Aufklärung zum Deutschen Idealismus eingeleitet
habe.\footnote{\enquote{Beide Ereignisse haben in Deutschland entscheidend zu
einem neuen, primär an der Idee der Freiheit orientierten Menschenbild
beigetragen und so u.\,a. zum sogenannten Deutschen Idealismus geführt}
\parencite[][263]{Schneiders:AufklaerungundVorurteilskritik1983}. An anderer
Stelle wiederum verzichtet er auf die Nennung der Vernunftkritik und sieht die
Französische Revolution als einzigen wichtigen Markstein auf dem Weg zu einem
Ende der Aufklärung
\parencite[vgl.][18]{Schneiders:DasZeitalterderAufklaerung2005}. Zur
\singlequote{Zwischenstellung} \name[Immanuel]{Kant}s und seiner Vernunftkritik
zwischen Aufklärung und Idealismus siehe auch
\cite{Hinske:KantsVernunftkritik--FruchtderAufklaerungundoderWurzeldesDeutschenIdealismus1993}.}
Der Bruch mit der Aufklärung wird so als ein solcher interpretiert, den
\name[Immanuel]{Kant} selbst weder intendierte noch bemerkte, den er selbst
möglicherweise nicht durchführte, zu dem er jedoch mit seiner kritischen
Philosophie den Grund
legte.\footnote{\cite[Vgl.][60]{Hinske:ZwischenAufklaerungundVernunftkritik1993}:
\enquote{Die Logikvorlesungen zeigen \name[Immanuel]{Kant} also als
entschiedenen Vertreter der Philosophie der deutschen Aufklärung und zugleich
als Urheber jener neuen, kritischen Philosophie, die den Bruch mit den
Grundideen und Grundüberzeugungen der deutschen Aufklärung ungewollt
vorbereitet.}}

Der Begriff \enquote{Aufklärung} birgt selbst schon erhebliche
Schwierigkeiten. Denn er beschreibt einerseits eine historische Epoche der
(europäischen) Geistesgeschichte, ist aber andererseits auch ein
Programmbegriff, also eine Bezeichnung für eine philosophische Herangehensweise,
für die sich Vertreter auch außerhalb der \enquote{Aufklärung} genannten Zeit
finden.\footnote{\phantomsection\label{Anmerkung:ScholzundderBegriffEinerBewegung}\cite[Vgl.][S.~28\,f.]{Scholz:KantsAufklaerungsprogramm2009}.
Nach \authorcite{Scholz:DasZeugnisanderer2001} bezeichnet der Terminus außerdem eine philosophische
\enquote{\emph{Bewegung}}, also ein Gesamt an philosophischen Denkern, Verbreitern und Förderern
des Aufklärungsdenkens innerhalb seiner Epoche. Unter einer \emph{Bewegung} ist
hier also das Gesamt an Bemühungen vieler Menschen in einem mehr oder weniger
langen zeitlichen Rahmen zu verstehen, ein bestimmtes Programm, das nicht immer klar umrissen sein muss,
umzusetzen. Beispiele für solche Bewegungen sind etwa die 68er, die
Friedensbewegung und die Widerstandsbewegung zur Zeit des Nationalsozialismus.}
Und wenngleich wir zwischen Aufklärung als historischer Epoche oder
Bewegung\footnote{Zum Begriff \enquote{Bewegung} siehe
Anm.~\ref{Anmerkung:ScholzundderBegriffEinerBewegung}.} und Aufklärung als noch
immer aktueller Programmatik unterscheiden können und sollten, sind beide Fragen
doch miteinander verwoben, was die Behandlung des Begriffs erneut
erschwert.\footnote{\cite[Vgl.][9]{Schneiders:HoffnungaufVernunft1990}:
\enquote{Aufklärung als Aktion oder Aktionsprogramm und Aufklärung als
geschichtliche Erscheinung oder Epoche scheinen nahezu unvermeidlich immer in
einem Atemzug genannt werden zu müssen, deskriptiv-historische und
philosophisch-systematische bzw.
normativ-programmatische Fragen scheinen sich unvermeidlich zu verknüpfen und zu
vermischen. Die historische Frage, was Aufklärung zu ihrer Zeit war bzw. bis
heute ist, präformiert die programmatische Frage, was Aufklärung auch heute und
in Zukunft noch sein könnte oder sein sollte.}} Auch
ein programmatischer Begriff ist nicht ohne historische Untersuchungen zu haben,
denn jeder Aufklärungsbegriff ist dem geschichtlichen Phänomen der Aufklärung
verpflichtet, wenn er nicht leer und beliebig werden
soll.\footnote{\cite[Vgl.][4]{Bubner:WaskannsollunddarfPhilosophie?1978}.} 
Umgekehrt muss jeder Versuch, einen historischen Aufklärungsbegriff zu
bestimmen, schon mit einem Vorverständnis beginnen, das sich aus systematischen Interessen
speist.\footnote{\textcite[][28]{Scholz:KantsAufklaerungsprogramm2009}
beschreibt die Bedeutungen von \enquote{Aufklärung} als Epoche oder Bewegung
gegenüber dem Programmbegriff als derivativ. Auch
\textcite[][\pno~9\,f.]{Schneiders:HoffnungaufVernunft1990} erklärt, dass
systematische und programmatische Fragen die historische Aufklärungsforschung
leiten, wenngleich er das Verhältnis eher als gegenseitig beschreibt.}

Wir können auf eine Definition \name[Immanuel]{Kant}s zurückgreifen, der den
Begriff der Aufklärung ausdrücklich zu explizieren versuchte, werden aber
feststellen müssen, dass \name[Immanuel]{Kant}s Bestimmung des
Aufklärungsbegriffs Fragen offen lässt. Besonders offensichtlich wird dies bei
dem Begriff des Selbstdenkens -- oder des \singlequote{eigenen
Verstandesgebrauchs}, den \name[Immanuel]{Kant} als Definiens des
Aufklärungsbegriffs heranzieht. So bemühen sich über das gesamte 18.
Jahrhundert hinweg (und auch bereits im 17. Jahrhundert) verschiedene Autoren
wie Christian \name[Christian]{Thomasius} und Christian
\authorcite{Wolff:Psychologiaempirica1968}, die darin schulbildend wirkten, mit
unterschiedlichen Ergebnissen um einen vernünftigen Begriff des Selbstdenkens --
wobei freilich der Ausdruck \enquote{Selbstdenken} erst spät entstand und zuvor
eher von \enquote{Eklektik} gesprochen
wurde.\footnote{\cite[Vgl.][92]{Albrecht:Thomasius--keinEklektiker?1989}, sowie
\cite[][241--243]{Albrecht:ChristianThomasius1999}.} Wir dürfen außerdem
nicht vergessen, dass in \name[Immanuel]{Kant}s \titel{Beantwortung der Frage:
was ist Aufklärung?} Beschreibungen eines übergreifenden Projekts des 18.
Jahrhunderts und seiner spezifisch eigenen Programmatik Hand in Hand gehen.

Zur \emph{historischen Epoche} der Aufklärung zählt man in Bezug auf den deutschen
Sprachraum für gewöhnlich die Zeit von 1690 bis etwa 1800, wobei es sich
anbietet, weiter zu unterteilen in (a) eine Frühaufklärung von 1690 bis 1720 mit
Christian \name[Christian]{Thomasius} als Leitfigur, (b) eine Hochaufklärung, welche
wiederum eine schulphilosophische Phase (1720 bis
1750) und eine popularphilosophische Phase (1750 bis 1780) durchläuft, und (c)
eine Spätaufklärung (1780 bis 1800), der sich auch Immanuel
\name[Immanuel]{Kant} zuordnen lässt.\footnote{\cite[Vgl.][33]{Schneiders:HoffnungaufVernunft1990}.
Siehe zur Epocheneinteilung auch
\cite[][311]{Hinske:WolffsStellunginderdeutschenAufklaerung1986}.} Darüber
hinaus bezieht sich der Begriff der Aufklärung natürlich auch auf philosophische
und wissenschaftliche Entwicklungen und Autoren außerhalb Deutschlands, etwa in
Frankreich, Großbritannien und den Niederlanden. Schon aufgrund der
unterschiedlichen Rahmenbedingungen musste sich die Aufklärung in den
verschiedenen Ländern und Zeiten unterschiedlich
entwickeln. Während -- wie
\authorfullcite{Schneiders:DasZeitalterderAufklaerung2005}
ausführt\footnote{Siehe dazu
\cite[][16--18]{Schneiders:DasZeitalterderAufklaerung2005} und passim.} -- für
alle Regionen dasselbe historische Ereignis (der Ausgang der Französischen
Revolution) als \emph{Endpunkt} der Aufklärung auszumachen sei,\footnote{An
anderer Stelle wiederum nennt
\authorcite{Schneiders:AufklaerungundVorurteilskritik1983} als weiteres
Ereignis neben dem Ausgang der Französischen Revolution, welches zur Auflösung
der Aufklärung beigetragen habe, das Erscheinen der \titel{Kritik der reinen Vernunft}
\parencite[vgl.][263]{Schneiders:AufklaerungundVorurteilskritik1983}.} begann
sie doch in Frankreich, Großbritannien und Deutschland zwar in etwa zeitgleich, aber doch unabhängig voneinander durch unterschiedliche historische Ereignisse: Die
\emph{Glorious Revolution} von 1688, der Aufhebung des \emph{Edikts von Nantes}
1685 und \name[Christian]{Thomasius}' Leipziger Vorlesungsankündigung von 1687.
In Großbritannien habe sich Aufklärung auch als Religionskritik weitgehend
ungestört entwickeln können, in Frankreich hingegen sei sie in Opposition zu
einem politisch starken Katholizismus geraten und in Deutschland sei sie
\enquote{im wesentlichen durch zwei Faktoren bestimmt: das weitgehend positive
Verhältnis zur christlichen Religion und zum absolutistischen Staat einerseits
sowie die institutionelle Bindung an die Universitäten
andererseits.}\footcite[][89]{Schneiders:DasZeitalterderAufklaerung2005}
Wie die Aufklärung sich in den verschiedenen Regionen entwickelte und in welchem
Verhältnis dann wiederum die deutsche Aufklärung zu den Entwicklungen in anderen
Gegenden stand, ist Gegenstand divergierender Deutungen und anhaltender
philosophiegeschichtlicher Forschung.\footnote{Zuletzt hat
\authorfullcite{CarbonciniGavanelli:DasParadoxderAufklaerung2007} sich gegen die
verbreitete Ansicht gewandt, eine Beeinflussung sei lediglich von Großbritannien
und Frankreich aus in Richtung Deutschland erfolgt.
\cite[Vgl.][\pno~73\,f.]{CarbonciniGavanelli:DasParadoxderAufklaerung2007}.}


Diese Schwierigkeiten, interne Differenzierungen der Epoche der Aufklärung
vorzunehmen und die verschiedenen Strömungen wiederum in Beziehung zueinander zu
setzen, erschwert dann natürlich auch das Verständnis des Aufklärungsdenkens
\name[Immanuel]{Kant}s.\footnote{Siehe hierzu die Beiträge in
\cite{Emundts:ImmanuelKantunddieBerlinerAufklaerung2000}.} Es gibt keine einheitliche
Definition des Aufklärungsbegriffs, die als Grundlage und Ausgangspunkt dienen
könnte; und es muss eine solche auch nicht geben. Es kann auch sein, dass wir
verschiedene Philosophen und Gedanken zur Aufklärung rechnen, die keine
gemeinsamen \distanz{Wesensmerkmale} besitzen, sondern eher in einer Weise
zusammengehalten werden, wie sie \name[Ludwig]{Wittgenstein} unter dem Titel
\enquote{Familienähnlichkeit}
beschreibt.\footcite[Vgl.][\S\S~65--67]{Wittgenstein:PhilosophischeUntersuchungen2003}
Oliver \authorcite{Scholz:DasZeugnisanderer2001} behauptet zwar bezüglich des
allgemeinen Programms der Aufklärung:
\enquote{Ohne die regionalen und nationalen Unterschiede herunterzuspielen, darf
man den Kern dieses Programms folgendermaßen fassen: Der Mensch soll sich
mittels des richtigen Gebrauchs seines Vernunftvermögens selbst befreien und
kognitiv, vor allem aber moralisch
vervollkommnen.}\footcite[][28]{Scholz:KantsAufklaerungsprogramm2009} Doch hier
hängt nicht nur alles an den Begriffen von Selbstbefreiung und Vervollkommnung,
die es zu klären gilt. Es lässt sich außerdem bezweifeln, erstens, dass diese
Beschreibung auf alle Aufklärer zutrifft, und zweitens, dass sie nicht auch ganz
andere philosophische Strömungen und Autoren ebenso treffend charakterisiert.
Man möchte bezweifeln, dass
\authorfullcite{Wolff:Discursuspraeliminarisdephilosophiaingenere1996}s
Philosophie durch den Begriff der Selbstbefreiung passend beschrieben ist. Und
\name[Immanuel]{Kant}s Philosophie in den Dienst der moralischen Vervollkommnung
zu stellen, ist zumindest stark erläuterungsbedürftig.\footnote{Damit ist nicht
gesagt, dass die moralische Vervollkommnung nicht als Ziel der
Aufklärungsbemühungen nach dem Verständnis \name[Immanuel]{Kant}s gesehen
werden kann. Zur Vervollkommnung äußert er sich in den \S\S~21\,f. der
\titel{Tugendlehre}
\mkbibparens{\cite[vgl.][A 113--115]{Kant:DieMetaphysikderSitten1977Tugendlehre},
\cite[][VI: 446.9--447.17]{Kant:GesammelteWerke1900ff.}}.
Siehe zu praktischen und moralischen Zielen der Aufklärung nach
\name[Immanuel]{Kant} auch Kapitel \ref{chapter:AufklaerungundWissenschaft}
dieser Arbeit.} Und wenn Norbert \name[Norbert]{Hinske} die Aufklärung mittels
ihrer Programmideen (Aufklärung, Selbstdenken, Mündigkeit), Kampfideen
(Vorurteile, Aberglaube, Schwärmerei) und Basisideen (Bestimmung des Menschen,
allgemeine Menschenvernunft)
charakterisiert,\footnote{\cite[Vgl.][392--398]{Hinske:ArtikelAufklaerung1985}.
Außerdem nennt \authorcite{Hinske:ArtikelAufklaerung1985} noch Toleranz und
Pressefreiheit als \singlequote{abgeleitete Ideen}
\parencite[vgl.][\pno~398\,f.]{Hinske:ArtikelAufklaerung1985}.} so ist damit das
Erläuterungsbedürfnis ebenfalls nur verschoben.\footnote{Damit soll der
Verdienst der Erläuterung nicht in Abrede gestellt werden.
\name[Norbert]{Hinske}s Unterscheidung zwischen drei Formen von Ideen der
Aufklärung erhellt durchaus die innere Struktur aufklärerischen Denkens,
freilich nur für denjenigen, der schon weiß, was Aufklärung ist.} Zusätzlich
bleibt natürlich fraglich, ob solche Charakterisierungen wirklich allgemein oder
auf bestimmte Teile der Aufklärung -- im Fall \name[Norbert]{Hinske}s: auf
\name[Immanuel]{Kant} -- zugeschnitten sind, und ob sie vielleicht im Dienste
bestimmter philosophischer Ansichten stehen. Dies ist zum Beispiel ganz klar der
Fall, wenn
\authorfullcite{Sternberg:AufklaerungKlassizismusundRomantikbeiKant1931} die
Aufklärung in offen hegelianisierender Manier als \enquote{Sache des
Verstandes}, als \enquote{Analysieren, \punkt{} Sichten und Auseinanderhalten}
beschreibt.\footcite[][\pno~31\,f.]{Sternberg:AufklaerungKlassizismusundRomantikbeiKant1931}
Dasselbe Vorgehen finden wir bei
\authorcite{Horkheimer:DialektikderAufklaerung1997}, wenn Aufklärung über die
Endlichkeit oder Diskursivität des Denkens definiert und dabei im Interesse des
Beweisziels verkürzt wird.\footnote{Der Begriff der Diskursivität kommt in der
Eindeutschung \enquote{fortschreitendes Denken} in der \titel{Dialektik der
Aufklärung} vor: \enquote{Seit je hat Aufklärung im umfassenden Sinn
fortschreitenden Denkens das Ziel verfolgt, von den Menschen die Furcht zu
nehmen und sie als Herren einzusetzen}
\parencite[][19]{Horkheimer:DialektikderAufklaerung1997}. Zur Einschätzung der
Arbeit von \authorcite{Horkheimer:DialektikderAufklaerung1997} siehe
\cite[][xiii--xv]{Hinske:Einleitung1990}.}
Die Heterogenität dessen, was wir heute als historisches Phänomen unter dem Namen
\enquote{Aufklärung} zusammenfassen, wird noch deutlicher, wenn wir
\authorfullcite{Israel:RadicalEnlightenment2001}s Unterscheidung von Radikalaufklärung und moderater
Aufklärung\footnote{\cite[Vgl.][3--13]{Israel:RadicalEnlightenment2001} und
passim. Die Radikalaufklärung \enquote{rejected all compromise with the past and
sought to sweep away existing structures entirely, rejecting the Creation as
traditionally understood in Judaeo-Christian civilization, and the intervention
of a providential God in human affairs, denying the possibility of miracles, and
reward and punishment in an afterlife, scorning all forms of ecclesiastical
authority, and refusing to accept that there is any God-ordained social
hierarchy, concentration of privilege or land-ownership in noble hands, or
religious sanction for monarchy}
\parencite[][\pno~11\,f.]{Israel:RadicalEnlightenment2001}. Dem stehe die
wirkmächtigere moderate Aufklärung gegenüber, der u.\,a.
Christian \name[Christian]{Thomasius}
und \authorfullcite{Wolff:Discursuspraeliminarisdephilosophiaingenere1996}
angehörten und die von zahlreichen Regierungen und Teilen der Kirchen
unterstützt wurde. \enquote{This was the Enlightenment which aspired to conquer
ignorance and superstition, establish toleration, and revolutionize ideas,
education, and attitudes by means of philosophy but in such a way as to preserve
and safeguard what were judged essential elements of the older structures,
effecting a viable synthesis of old and new, and of reason and faith}
\parencite[][11]{Israel:RadicalEnlightenment2001}.} oder
\authorfullcite{Hunter:RivalEnlightenments2001}s auf die deutsche Philosophie
bezogene Differenzierung von bürgerlicher und metaphysischer
Aufklärung\footnote{\cite[Vgl.][14--9]{Hunter:RivalEnlightenments2001} und
passim.} hinzu nehmen. Dies sollte uns davor warnen, allzu schnell
Verallgemeinerungen über eine ganze Epoche vorzunehmen und Zusammenhänge
zwischen Autoren zu postulieren, die erst noch zu erweisen sind.

Der Begriff von Aufklärung, den ich im folgenden entwickle und verwende, wird
somit nicht beanspruchen können, dem Gesamtphänomen dessen gerecht zu werden,
was sich retrospektiv \singlequote{die} Aufklärung nennen ließe, sondern primär
auf \name[Immanuel]{Kant} fokussieren. Ich werde dabei nicht auf solche Fragen
eingehen wie die, ob \name[Immanuel]{Kant} noch zur Aufklärung oder schon zu
einer \enquote{klassischen deutschen Philosophie} oder einem \enquote{deutschen
Idealismus} gehört, ob er einer \singlequote{Berliner Aufklärung}, einer
\singlequote{moderaten} oder einer \singlequote{metaphysischen} Aufklärung
zuzurechnen ist. Mich interessiert, ob sich die Endlichkeit des Denkens oder die
Vernunftkritik, die von der Endlichkeit menschlichen Denkens ausgeht, in
\name[Immanuel]{Kant}s Aufklärungsprogrammatik integrieren lässt, vielleicht
sogar einen wesentlichen Teil derselben ausmacht, oder ob die Vernunftkritik den
Rahmen aufklärerischen Denkens bereits verlässt.\footnote{Diese Frage werde ich
jedoch in diesem \ref{section:KantalsliberalerAufklaerer}. Kapitel mitnichten
abschließend beantworten. Hier soll zunächst der Aufklärungsbegriff entwickelt
werden, der einer Antwort zugrunde liegt.} Der zugrunde gelegte
Aufklärungsbegriff ist dabei der, den \name[Immanuel]{Kant} selbst entwickelt --
explizit etwa in der Berlinischen Monatsschrift.

\section{\enquote{Aufklärung} und \enquote{aufklären}}
Bereits im 18.\ Jahrhundert wurde darauf hingewiesen, dass der Begriff der
Aufklärung keinen klar umrissenen und allseits bekannten Gehalt hat. Man könnte
gar fragen, ob ihm \emph{überhaupt} ein Gehalt zukommt und ob er ein reales
Phänomen bezeichnet; schließlich kam er als Neologismus des 18.
Jahrhunderts recht spät zum
Vorschein\footnote{\phantomsection\label{Anmerkung:MosesMendelssohnZumNeologismus}So
schreibt Moses
\textcite[][3]{Mendelssohn:UeberdieFrage:washeisstaufklaeren?2008}: \enquote{Die
Worte \ori{Aufklärung, Kultur, Bildung} sind in unsrer Sprache noch neue
Ankömmlinge.} Dies verweise aber nicht darauf, dass entsprechende Phänomene erst
neu seien.} und wurde in den Jahrtausenden
zuvor allem Anschein nach auch nicht vermisst. Eine entsprechende Skepsis kommt in der Frage \authorfullcite{Zoellner:IstesrathsamdasEhebuendnissnichtfernerdurchdieReligionzusancieren?1783}s
zum Ausdruck, die schließlich die Antwortversuche \name[Immanuel]{Kant}s und
\name[Moses]{Mendelssohn}s provozierte: \enquote{Was ist Aufklärung? Diese
Frage, die beinahe so wichtig ist, als: was ist Wahrheit, sollte doch mal
beantwortet werden, ehe man aufzuklären anfinge! Und noch habe ich sie nirgends
beantwortet
gefunden}\footnote{\cite[][516]{Zoellner:IstesrathsamdasEhebuendnissnichtfernerdurchdieReligionzusancieren?1783}.
\authorcite{Zoellner:IstesrathsamdasEhebuendnissnichtfernerdurchdieReligionzusancieren?1783}
selbst bestimmte die Aufklärung später als Vorurteilskritik
\parencite[siehe
dazu][\pno~271\,f.]{Schneiders:AufklaerungundVorurteilskritik1983}.}.
Diese Frage explizit zu stellen ist wichtig, weil eine Position oder Behauptung
als unaufgeklärt zu bezeichnen einer finalen Zurückweisung gleichkommt, die
ebenso schlagend ist, wie der Nachweis der Falschheit einer Behauptung.
Damit dieses argumentative Vorgehen legitim ist, muss natürlich klar sein, worin
der enthaltene Vorwurf besteht, dem sich niemand ausgesetzt sehen möchte. Was --
so scheint
\authorcite{Zoellner:IstesrathsamdasEhebuendnissnichtfernerdurchdieReligionzusancieren?1783}
fragen zu wollen -- können wir von unserem Denken denn noch mehr oder anderes
verlangen, als dass es sich an der Wahrheit orientiere?

Man könnte versucht sein, auf
\authorcite{Zoellner:IstesrathsamdasEhebuendnissnichtfernerdurchdieReligionzusancieren?1783}s
Frage, was denn zur \emph{Orientierung} an der Wahrheit hinzukommen müsse, zu
antworten: ihr \emph{Besitz}! Im Sinne einer szientistisch verstandenen
Aufklärung ließe sich dann mutmaßen, es ginge um die Gewinnung und Verbreitung
sicheren, \enquote{wissenschaftlich fundierten} Wissens im Kontrast zu religiösen,
mystischen oder metaphysischen Erklärungen \distanz{unaufgeklärter} Zeitalter
wie des \distanz{dunklen
Mittelalters}. Man
könnte so den Begriff der Aufklärung in enger Anlehnung an die Entwicklung der
modernen Naturwissenschaften zu explizieren versuchen und Aufklärung in
Entgegensetzung gegen ein zuvor vorherrschendes religiöses oder metaphysisches
Weltbild verstehen. Unaufgeklärt wäre dann eine (falsche) Überzeugung, die im
Zuge der wissenschaftlichen Revolution und der Entwicklung eines
wissenschaftlichen Weltbildes widerlegt wurde.

Der Etymologie des Wortes \enquote{Aufklärung} gemäß -- und dies gilt ebenso für
die Entsprechungen \enquote{Enlightenment}, \enquote{Si{\`e}cle des
Lumi{\`e}res} und \enquote{illuminismo} -- soll \distanz{Licht in's Dunkel}
gebracht werden. Und dazu passt sicherlich die Rede von einem \distanz{dunklen
Mittelalter} mit seinen überkommenen Erkenntnissen. Diese Abgrenzungen gegen
eine scholastische Philosophie des Mittelalters mag verbreitet gewesen
sein.\footnote{\cite[Vgl.][42]{Stekeler-Weithofer:Denken2012}:
\enquote{Die Epoche der Aufklärung sieht in der Abhängigkeit von traditionalen
Vorbildern, wie sie etwa das \singlequote{Mittelalter} unter Einschluss der
Renaissance und Reformation angeblich oder wirklich prägen, eine Heteronomie,
die es aufzuheben gilt.}} Aber zur Beschreibung und Selbstabgrenzung gegen
falsche Erklärungen hätte der Begriff der Wahrheit durchaus zugereicht. Wozu
also dieser Neologismus des {18.} Jahrhunderts?

Nun lag die Unklarheit für
\authorcite{Zoellner:IstesrathsamdasEhebuendnissnichtfernerdurchdieReligionzusancieren?1783}
auch in dem Umstand begründet, dass es sich bei dem Wort \enquote{Aufklärung}
um einen Neuankömmling in der deutschen Sprache handelte.\footnote{Siehe
Anm. \ref{Anmerkung:MosesMendelssohnZumNeologismus} auf
S.~\pageref{Anmerkung:MosesMendelssohnZumNeologismus}.} Heute ist es eher der
inflationäre Gebrauch, der seinen Gehalt undeutlich werden lässt. Aber dieser häufige Gebrauch spricht doch auch dafür, dass er eine wichtige und einzigartige sprachliche Funktion
übernimmt, für die der Begriff der Wahrheit allein kein Äquivalent liefert.
Diese Funktion mag wiederum nicht einheitlich sein. Eine Vielfalt an Formen von
\enquote{Aufklärung} findet sich nicht nur historisch in der (zeitlichen und
regionalen) Heterogenität der Epoche, sondern auch systematisch in den
verschiedenen, noch immer verbreiteten Redeweisen von \enquote{aufklären} und
\enquote{aufgeklärt} wieder. Eltern klären ihre Kinder auf,
wenn sie sie mit der menschlichen Sexualität vertraut machen. Ärzte klären
Patienten auf, indem sie ihnen die Wirkungsweise, die Risiken und Nebenwirkungen
und auch die Chancen einer Therapie erläutern. Verbraucherschützer propagieren
Verbraucheraufklärung. Verbrechen werden aufgeklärt. Politiker sprechen mitunter
von \distanz{schonungsloser Aufklärung}. Das Militär kennt seine eigene Form der
\distanz{Aufklärung}, ebenso wie auch nichtmilitärische Geheimdienste
\distanz{Aufklärung} betreiben. All diese Redeweisen unterscheiden sich
erheblich von dem Projekt des 18.\ Jahrhunderts, sind aber doch durch dieses
geprägt.

Meines Erachtens sehen wir die Herkunft dieser Redeweisen am besten anhand einer
bestimmten Auswahl: Wo der Begriff im Sinne von \enquote{\emph{jemanden}
aufklären} (im Kontrast zu \enquote{\emph{etwas} aufklären}) verwendet wird,
geht es zwar \emph{auch} darum, jemanden zu informieren -- über den menschlichen
und damit auch den eigenen Körper, über eine Krankheit und was damit einhergeht,
über die In\-halts\-stof\-fe von Lebensmitteln oder die Rechte und Pflichten
eines Vertragspartners. Aber nicht jede Information zählt als Aufklärung,
sondern nur diejenige, die auf die Mündigkeit desjenigen abzielt, den sie
aufklärt. Wer jemanden aufklärt, möchte ihm ermöglichen, ein eigenes Urteil und
eigene Entscheidungen zu fällen -- der Aufgeklärte urteilt und handelt
selbstbestimmt.
So ist das Ziel der Sexualaufklärung der mündige und selbstbestimmte Umgang mit der eigenen
Sexualität. Und der Arzt, der einen Patienten aufklärt, versorgt ihn mit genau
den und so vielen Informationen, wie nötig sind, um den Patienten in die Lage zu
versetzen, sich \emph{selbst} für oder wider eine bestimmte Therapieform zu
entscheiden. Gerade darin erweist sich der Begriff der Aufklärung auch in seinen
derivativen Verwendungen als Abkömmling der europäischen Geistesgeschichte:
Aufklärung verbindet Information mit Freiheit, Selbstbestimmung und Mündigkeit.
Und gerade diese Verbindung macht ihn aus und sichert ihm seine Bedeutung unter
unseren Begriffen. Negierte man das zweite dieser Momente und setzte etwa
Aufklärung in die Verbreitung wahrer Erkenntnisse, so verlöre dieser Begriff
seinen spezifischen Reiz. Wir sollten dann lieber gleich von Wahrheit sprechen,
um Missverständnisse zu vermeiden.

\section{Selbstdenken als Autonomie der
Vernunft}\label{subsection:SelbstdenkenbeiKant}
In den 1780er Jahren artikuliert \name[Immanuel]{Kant} in Aufsätzen in der
Berlinischen Monatsschrift eine Form von Aufklärung, die sich nicht an
bestimmten Inhalten, sondern der Freiheit und Selbstbestimmung des einzelnen
Subjekts orientiert. Dies wird zunächst in der bekannten
Aufklärungsschrift\footnote{\cite[Vgl.][A~481]{Kant:BeantwortungderFrage:WasistAufklaerung?1977},
\cite[][VIII: 35.1--8]{Kant:GesammelteWerke1900ff.}:
\enquote{\ori{Aufklärung ist der Ausgang des Menschen aus seiner
selbstverschuldeten Unmündigkeit. Unmündigkeit} ist das Unvermögen, sich seines
Verstandes ohne Leitung eines anderen zu bedienen. \punkt\ Sapere aude! Habe
Mut, dich deines eigenen Verstandes zu bedienen! ist also der Wahlspruch der
Aufklärung.}} deutlich, die durchgängig die Relation der Erkenntnis zu
dem einzelnen Subjekt hervorhebt und keinerlei spezielle Inhalte als solche der
Aufklärung herausstellt.\footnote{Mit der Religion, die \name[Immanuel]{Kant}
als zentrales Thema der Aufklärung herausstellt \mkbibparens{\cite[vgl.][A
492]{Kant:BeantwortungderFrage:WasistAufklaerung?1977}; \cite[][VIII:
41.10--12]{Kant:GesammelteWerke1900ff.}}, ist eben bloß ein
\emph{Thema} als besonders relevant bezeichnet, aber keine bestimmte Behauptung.}
\name[Immanuel]{Kant} stellt durch Betonung des Selbstdenkens die intellektuelle
Freiheit in das Zentrum der Aufklärungsprogrammatik und verdeutlicht, dass es nicht um
den \emph{Inhalt} des Fürwahrgehaltenen geht, sondern um die
\enquote{Denkungsart}.\footnote{Siehe exemplarisch \cite[][A
484]{Kant:BeantwortungderFrage:WasistAufklaerung?1977},
\cite[][VIII: 36.28--33]{Kant:GesammelteWerke1900ff.}:
\enquote{Durch eine Revolution wird vielleicht wohl ein Abfall von
persönlichem Despotism und gewinnsüchtiger oder herrschsüchtiger Bedrückung,
aber niemals wahre Reform der Denkungsart zu Stande kommen; sondern neue
Vorurteile werden, eben sowohl als die alten, zum Leitbande des gedankenlosen
großen Haufens dienen.}} Insbesondere in der Schrift \titel{Was heißt:
sich im Denken orientieren?} betont er dies durch die Abgrenzung des eigenen
Aufklärungsverständnisses von Positionen, die Aufklärung an
\enquote{Kenntnissen}
ausrichten.\footnote{\cite[Vgl.][A~329]{Kant:Washeisst:SichimDenkenorientieren?1977},
\cite[][VIII: 146.31--32]{Kant:GesammelteWerke1900ff.}.} Dabei ist jedoch nicht
leicht zu erkennen, was \name[Immanuel]{Kant} hier als \enquote{Kenntnisse}
anspricht. Eine naheliegende Deutungsmöglichkeit scheint mir zu sein, dies als
Ausdruck für den \emph{Inhalt} unseres Wissens zu lesen, im Kontrast dazu, wie
er erworben und weiter gehandhabt wird, ob er Produkt des Selbstdenkens ist und
wie er in die Handlungsplanung des Subjekts eingeht. \enquote{Kenntnisse} steht
demnach für die Informationen, die jemand über die Welt haben oder nicht haben
kann, das gesamte Tatsachenwissen, wie wir es beispielsweise in Lexika
finden.\footnote{\cite[Vgl.][32]{Scholz:KantsAufklaerungsprogramm2009}:
\enquote{Aufklärung besteht also nicht in dem materialen Besitz von Kenntnissen,
in irgendwelchen spezifischen Denkinhalten, Lehren oder Kenntnissen, die man
etwa in einer Liste zusammenfassen und dann jedermann einpauken könnte}.
Mitunter scheinen mit \enquote{Kenntnisse} aber auch gewisse (intellektuelle) Fähigkeiten
angesprochen zu sein, die nicht hinlänglich zur Mündigkeit sind, weil sie der
notwendigen eigenen Einsicht in ihre Vernünftigkeit entbehren. Siehe
z.\,B. \cite[][A~26]{Kant:ImmanuelKantsLogik1977}, \cite[][IX:
25.21--26]{Kant:GesammelteWerke1900ff.}: \enquote{[O]hne Kenntnisse wird man
nie ein Philosoph werden, aber nie werden auch Kenntnisse allein den
Philosophen ausmachen, wofern nicht eine zweckmäßige Verbindung aller
Erkenntnisse und Geschicklichkeiten zur Einheit hinzukommt, und eine Einsicht
in die Übereinstimmung derselben mit den höchsten Zwecken der menschlichen
Vernunft.} Dieser Passus erinnert an die Unterscheidung zwischen Schul- und
Weltbegriff der Philosophie (siehe dazu weiter unten Kapitel
\ref{subsection:DieBestimmungdesMenschen}) und identifiziert, sollte diese
Analogie stimmig und intendiert sein, die \enquote{Kenntnisse} mit den
Fähigkeiten nach dem Schulbegriff.} In der \titel{Anthropologie in pragmatischer
Hinsicht} beschreibt \name[Immanuel]{Kant} die \enquote{Kenntnisse} als Ausdruck
von
\enquote{Büchergelehrsamkeit}\footnote{\cite[Vgl.][BA~166]{Kant:AnthropologieinpragmatischerHinsicht1977},
\cite[][VII: 228.14--16]{Kant:GesammelteWerke1900ff.}:
\enquote{Büchergelehrsamkeit vermehrt zwar die Kenntnisse, aber erweitert nicht
den Begriff und die Einsicht, wo nicht Vernunft dazu kommt.} Mir scheint, dass
\name[Immanuel]{Kant} implizit auf die \emph{nudae factorum notitiae} der
wolffschen Philosophie und
\authorcite{Wolff:Psychologiaempirica1968}s Warnung vor einer \emph{cognitio
cognitionis philosophicae historica} verweist, die ich nicht an dieser, sondern
einer späteren Stelle ausführlicher behandeln werde. Hier sei nur folgendes als
Erläuterung angeführt: Eine \emph{nuda facti notitia} oder historische Erkenntnis bei
\authorcite{Wolff:Psychologiaempirica1968} ist die Kenntnis einer Tatsache ohne Verständnis ihres Grundes
(\cite[vgl.][\S~7]{Wolff:Discursuspraeliminarisdephilosophiaingenere1996}) und
die Warnung gilt dementsprechend einem Wiedergebenkönnen von Wahrheiten ohne
philosophisches Verständnis derselben (\enquote{wo nicht Vernunft dazu kommt},
siehe Kapitel \ref{paragraph:wolffswarnung}).
Zur Differenz zwischen der Warnung in
\authorcite{Wolff:Psychologiaempirica1968}s und der in \name[Immanuel]{Kant}s
Fassung siehe Kapitel \ref{section:MuendigkeitundPhilosophie}.}, also einer Art
des unselbständigen Denkens, das sich von anderen leiten lässt. Gerade die
\emph{Art} des Denkens -- ob jemand selbst denkt oder sich von anderen leiten
lässt -- kann bei veränderten Inhalten unverändert bleiben (und umgekehrt kann
sich bei gleichbleibenden Inhalten die \emph{Art} des Denkens
ändern), weswegen nur
eine langsame Reform, aber keine Revolution im Denken der Aufklärung förderlich
sei.\footnote{\cite[Vgl.][A~484]{Kant:BeantwortungderFrage:WasistAufklaerung?1977},
\cite[][VIII: 36.28--33]{Kant:GesammelteWerke1900ff.}: \enquote{Durch eine
Revolution wird vielleicht wohl ein Abfall von persönlichem Despotismus und
gewinnsüchtiger oder herrschsüchtiger Bedrückung, aber niemals wahre Reform der
Denkungsart zustande kommen; sondern neue Vorurteile werden ebenso wohl als die
alten zum Leitbande des gedankenlosen großen Haufens dienen.}} Man könnte eine
solche Aufklärungskonzeption, die sich auf die \emph{Art} des Denkens, die
Relation der Inhalte des Denkens zu dem denkenden Subjekt bezieht, eine
\phantomsection\label{Benennung:LiberaleAufklaerung}\emph{liberale} Aufklärung
nennen und sie einer \emph{szientistischen} Aufklärung entgegensetzen, die bestimmte Inhalte -- etwa den Heliozentrismus
oder die Evolutionstheorie -- als solche der Aufklärung glaubt ausweisen zu
können.
\authorfullcite{Schneiders:DasZeitalterderAufklaerung2005} unterscheidet in eben
diesem Sinne einen rationalistischen Aufklärungsbegriff, dem zufolge es um die
Klärung von Begriffen und die Beseitigung von Unvernunft und Unwissenheit geht, von einem
emanzipatorischen Aufklärungsbegriff, der sich an der \enquote{Befreiung von Fesseln
aller Art}
orientiert.\footcite[Vgl.][7]{Schneiders:DasZeitalterderAufklaerung2005} Und
\authorfullcite{Stuke:Aufklaerung1972} spricht etwas allgemeiner von dem
Unterschied zwischen einem formalen Aufklärungsverständnis und einem solchen,
welches durch objektiv-materiale Kriterien konstituiert werde, und betont, dass
\name[Immanuel]{Kant} mit seiner Artikulation einer streng formalen (oder --wie
ich sie nenne -- liberalen) Aufklärung von dem vorherrschenden Sprachgebrauch
seiner Zeitgenossen
abweiche.\footnote{\cite[Vgl.][\pno~265\,f.]{Stuke:Aufklaerung1972} Dabei
verfahre \name[Immanuel]{Kant} jedoch inkonsequent, insofern mindestens in
seiner Religionsphilosophie bestimmte Wissensgehalte den Begründungszusammenhang
der Aufklärung umschrieben \parencite[vgl.][265--272]{Stuke:Aufklaerung1972}. Es
ist dabei sicherlich korrekt, dass \name[Immanuel]{Kant} mitunter auch bestimmte
Überzeugungen als solche voraussetzt, ohne die Aufklärung nicht möglich ist.
Warum er diese voraussetzen muss, wird erst auf Grundlage der Überlegungen in
Kapitel \ref{chapter:MuendigerErwerbTestimonialenWissens} zu beantworten sein.
Es läuft darauf hinaus, dass es von der Art der Erkenntnisse abhängt, was es je
konkret heißt, sie mündig zu erwerben. Ein Missverständnis bezüglich der Art
der vorliegenden Erkenntnisse macht es unmöglich, sich ihnen gegenüber kritisch zu
verhalten. Siehe dazu Kapitel \ref{section:KantsEthicsofBelief}.}
\name[Immanuel]{Kant} hat die liberale Aufklärung freilich nicht
erfunden, sondern in Anlehnung an Selbstdenker\footnote{Mitunter
findet sich statt des Ausdrucks \enquote{Selbstdenker} gerade in Bezug auf
\name[Christian]{Thomasius} auch die Bezeichnung \enquote{Eklektiker}. Siehe
dazu auch Anm. \ref{Anmerkung:BegriffderEklektik} auf S.
\pageref{Anmerkung:BegriffderEklektik}.} wie \name[Christian]{Thomasius} und
Autoren wie \authorcite{Lessing:EineDuplik1897} aufgegriffen.\footnote{Gerade
\authorcite{Lessing:EineDuplik1897} verleiht der geistigen Liberalität Ausdruck,
wenn er schreibt:
\phantomsection\label{Zitat:Lessing:EineDuplik}
  \enquote{Ein Mann, der Unwahrheit, unter entgegengesetzter Ueberzeugung, in
  guter Absicht, ebenso scharfsinnig als bescheiden durchzusetzen sucht, ist
  unendlich mehr werth, als ein Mann, der die beste edelste Wahrheit aus
  Vorurtheil, mit Verschreyung seiner Gegner, auf alltägliche Weise
  vertheidiget.
  \punkt\ Nicht die Wahrheit, in deren Besitz irgend ein Mensch ist, oder zu
  seyn vermeynet, sondern die aufrichtige Mühe, die er angewandt hat, hinter
  die Wahrheit zu kommen, macht den Werth des
  Menschen} \parencite[][S. 23\,f.]{Lessing:EineDuplik1897}.
Auf der anderen Seite reicht es nicht, Wahres zu denken, um frei von Vorurteilen
zu sein. Mit \authorcite{Meier:AuszugausderVernunftlehre1752} kommt die
Auffassung auf, dass auch \emph{wahre} Urteile \emph{Vor}urteile sein
können. \cite[Vgl.][89]{Schneiders:PraktischeLogik1980}: \enquote{Im
Gegensatz zu \name[Christian]{Thomasius}
und \authorcite{Wolff:Discursuspraeliminarisdephilosophiaingenere1996}
verzichtete \authorcite{Meier:AuszugausderVernunftlehre1752} ausdrücklich
darauf, die Vorurteile als falsche Urteile zu definieren; er betont vielmehr
nachdrücklich, daß sie als unbegründete Urteile dennoch wahre Urteile sein
könnten.}}

\subsection{Der Begriff des
Selbstdenkens}\label{subsection:DerBegriffdesSelbstdenkens}
In der Aufklärungsschrift skizziert \name[Immanuel]{Kant} nicht ein neues
Projekt, sondern bemüht sich um Artikulation eines Projekts, das aus seiner
Sicht längst besteht und dessen wichtigstes Merkmal das Selbstdenken ist. Doch
bleibt \name[Immanuel]{Kant} zunächst eine Bestimmung dessen schuldig, was es
heißt, selbst zu denken. Während er in der Aufklärungsschrift nur abstrakt von
dem Mut des eigenen Verstandesgebrauchs und der Mündigkeit als des Vermögens,
\enquote{sich seines Verstandes ohne Leitung eines anderen zu
bedienen}\footnote{\cite[][A~481]{Kant:BeantwortungderFrage:WasistAufklaerung?1977},
\cite[][VIII: 35.3]{Kant:GesammelteWerke1900ff.}.}, spricht, sieht er sich an
anderer Stelle genötigt, die Rede vom Selbstdenken näher zu erläutern. Er
schreibt:
\begin{quote}\label{def:selbstdenken}
 \ori{Selbstdenken} heißt den obersten Probirstein der Wahrheit in sich selbst
(d.\,i.\ in seiner eigenen Vernunft) suchen; und die Maxime, jederzeit selbst zu
denken, ist die \ori{Aufklärung}. Dazu gehört nun eben so viel nicht, als sich
diejenigen einbilden, welche Aufklärung in \ori{Kenntnisse} setzen: da sie
vielmehr ein negativer Grundsatz im Gebrauche seines Erkenntnißvermögens ist, und öfter
der, so an Kenntnissen überaus reich ist, im Gebrauche derselben am wenigsten
aufgeklärt
ist.\footnote{\cite[][A 329]{Kant:Washeisst:SichimDenkenorientieren?1977},
\cite[][VIII: 146.29--35]{Kant:GesammelteWerke1900ff.}.}
\end{quote}
\name[Immanuel]{Kant} erläutert nicht explizit, was er unter einem negativen
Grundsatz versteht. Nahe liegt der Gedanke, dass er glaubt, damit nur gesagt zu
haben, was der Selbstdenker nicht tut. Man denke etwa an den \enquote{negativen
Begriff} der Freiheit im dritten Abschnitt der \titel{Grundlegung}. Dort
expliziert \name[Immanuel]{Kant} die Freiheit des Willens als Unabhängigkeit von
bestimmenden Ursachen. Diese Unabhängigkeit reiche aber noch nicht aus, um
hinreichend zu sagen, was Freiheit ist,\footnote{\cite[Vgl.][BA~97]{Kant:GrundlegungzurMetaphysikderSitten1965},
\cite[IV: 446.13--15]{Kant:GesammelteWerke1900ff.}: \enquote{Die angeführte
Erklärung der Freiheit ist \ori{negativ}, und daher, um ihr Wesen einzusehen,
unfruchtbar; allein es fließt aus ihr ein \ori{positiver} Begriff derselben, der
desto reichhaltiger und fruchtbarer ist.}} es müsse noch ein positiver Begriff
hinzukommen: im Falle der Freiheit die \enquote{Autonomie, d.\,i.\ die
Eigenschaft des Willens, sich selbst ein Gesetz zu
sein}\footnote{\cite[][BA~98]{Kant:GrundlegungzurMetaphysikderSitten1965},
\cite[][IV: 447.1--2]{Kant:GesammelteWerke1900ff.}.}. Entsprechend lässt
\name[Immanuel]{Kant} auch in der Definition des Selbstdenkens einen entsprechenden
\phantomsection\label{positiverBegriffdesSelbstdenkens}\distanz{positiven}
Begriff folgen, freilich ohne ihn als solchen zu benennen:
\begin{quote}
  Sich seiner \ori{eigenen} Vernunft bedienen will nichts weiter sagen, als bei
  allem dem, was man annehmen soll, sich selbst fragen: ob man es wohl tunlich
  finde, den Grund, warum man etwas annimmt, oder auch die Regel, die aus dem,
  was man annimmt, folgt, zum allgemeinen Grundsatze seines Vernunftgebrauchs zu
  machen?\footnote{\phantomsection\label{Fussnote:positiverBegriffdesSelbstdenkens}\cite[A~329]{Kant:Washeisst:SichimDenkenorientieren?1977},
  \cite[VIII: 146.35--147.6]{Kant:GesammelteWerke1900ff.}.
  \authorfullcite{ONeill:AufgeklaerteVernunft1996} rechnet auch diese Bestimmung
  des Selbstdenkens zum negativen Grundsatz, obwohl sie die Parallele zur
  praktischen Philosophie erkennt
  (\cite[vgl.][218]{ONeill:AufgeklaerteVernunft1996}). Auch
  \authorfullcite{Deligiorgi:UniversalisabilityPublicitaandCommunication2002}
  bezeichnet den Grundsatz als negativ, weil er keine konkreten Inhalte benenne
  \parencite[vgl.][146]{Deligiorgi:UniversalisabilityPublicitaandCommunication2002}.
  Beide übersehen m.\,E.\ die Ähnlichkeit zum positiven Begriff der Freiheit.
  Siehe auch unten, Kapitel \ref{subsection:MetaphysikundAutonomie}.}
\end{quote}
Um selbständig zu denken sollen wir zweierlei beachten: Wir sollen erstens nach
\emph{Grundsätzen} urteilen und zweitens diese Grundsätze danach bewerten, ob sie als
\emph{allgemeine} Grundsätze akzeptiert werden können.
\authorfullcite{Cohen:KantontheEthicsofBelief2014} verweist auf die
Ähnlichkeit dieser Formulierung zur Universalisierungsformel des Kategorischen Imperativs --
und damit zu dem Gesetz, das der Wille als praktische Vernunft sich selbst
ist.\footnote{\cite[Vgl.][330]{Cohen:KantontheEthicsofBelief2014}.} Diese \enquote{Maxime
der \ori{Selbsterhaltung} der Vernunft} -- wie \name[Immanuel]{Kant} sie an
derselben Stelle bezeichnet -- werde ich weiter unten wieder aufgreifen.
Zunächst soll aber noch der \enquote{negative Grundsatz} weiter betrachtet
werden, der nach \name[Immanuel]{Kant}s Darstellung die Aufklärung
ausmacht.\footnote{\cite[Vgl.][\S~40]{Kant:KritikderUrteilskraft2009},
\cite[][V: 294.35--36]{Kant:GesammelteWerke1900ff.}.}

\phantomsection\label{falscheFaehrte:VernunftkritikalsWissensbegrenzung}
Innerhalb der theoretischen Philosophie verwendet \name[Immanuel]{Kant} das Wort
\enquote{positiv} dort, wo Wissen und Erkenntnis vermehrt werden,
\enquote{negativ} hingegen dort, wo es um die Vermeidung von Irrtümern geht.\footnote{Z.\,B.~in der \titel{Anthropologie}:
\enquote{Der Verstand ist positiv und vertreibt die Finsternis der Unwissenheit
-- die Urteilskraft mehr negativ zu Verhütung der Irrtümer aus dem dämmernden
 Lichte, darin die Gegenstände erscheinen.}
 (\cite[][BA~166]{Kant:AnthropologieinpragmatischerHinsicht1977},
 \cite[][VII: 228.10--12]{Kant:GesammelteWerke1900ff.}.)} In der Kritik der
 reinen Vernunft vergleicht er die Philosophie als negatives Instrument mit der
 Polizei, insofern sie die Menschen voreinander schütze und so den positiven
 Nutzen habe, dass \enquote{ein jeder seine Angelegenheit ruhig und sicher treiben
 könne.}\footnote{\cite[][B~xxv]{Kant:KritikderreinenVernunft2003}, \cite[][III:
16.29--30]{Kant:GesammelteWerke1900ff.}.} Sie ist \enquote{negativ}, weil sie
nicht (als \emph{Organon}) unser Wissen erweitert, sondern (als \emph{Kanon})
vermeintlich objektive Wissensansprüche zurückweist, um Freiräume für den
Glauben und je selbst zu verantwortende Überzeugungen zu gewähren. Der
\enquote{negative Grundsatz} scheint also deswegen die Aufklärung auszumachen,
weil sie einen Freiraum garantiert, innerhalb dessen jeder in dem, was er für
wahr hält und sagt, nur \enquote{seiner eigenen Vernunft} Rechenschaft schuldig
ist. Dies wird im allgemein  dahingehend gedeutet, dass \name[Immanuel]{Kant}
die je eigene Einsicht und das je eigene Sich-selbst-überzeugen anspricht.
\authorfullcite{Gerhardt:Selbstbestimmung1999} etwa interpretiert das
\enquote{sapere aude} sogar folgendermaßen:
\begin{quote}
\ori{Suche deine eigene Einsicht und folge ihr} -- das ist der Wahlspruch der
Philosophie. Jeder solle sich, so hat \name[Immanuel]{Kant} es für die
Aufklärung formuliert, \ori{seines eigenen Verstandes} bedienen. Wäre das Ich
nur insoweit gefragt, als es ohnehin bei jedem intellektuellen Akt beteiligt
ist, bedürfte es der Betonung des jeweiligen \ori{eigenen} Verstandes nicht.
Also lebt die Philosophie tatsächlich aus einer \ori{selbstbewußten Forcierung
der Individualität}, eine Besonderheit, die sie, wie \singlename{Platon} bereits
wußte, mit den Künsten
teilt.\footnote{\cite[][35]{Gerhardt:Selbstbestimmung1999}. Zur Verwendung des
\singlename{Horaz}-Zitats in der Aufklärung siehe
\cite{Venturi:ContributiadundizionariostoricoI:WasistAufklaerungSapereaude!1959,Firpo:Ancoraapropositiodienquotesapereaude!1960}.}
\end{quote}
Der negative Grundsatz des Selbstdenkens richtet sich somit gegen (die Berufung
auf) Autoritäten. Und dies wird in der \titel{Anthropologie} noch einmal
deutlich, wo \name[Immanuel]{Kant} der Beschreibung des Selbstdenkens als
negativ das \singlename{Horaz}-Zitat beigibt:
\enquote{nullius addictus iurare in verba
magistri}\footnote{\cite[][BA~167]{Kant:AnthropologieinpragmatischerHinsicht1977},
\cite[][VII: 228.35]{Kant:GesammelteWerke1900ff.}. Zum ursprünglichen Kontext
des Zitats in der antiken Eklektik im Umfeld von \singlename{Cicero} und
\singlename{Horaz} siehe \cite[][38--49]{Albrecht:Eklektik1994}.} -- auf das
Wort keines Lehrers sei man in seinem Urteil verpflichtet.

\phantomsection\label{Abschnitt:AufklaerungundMuendigkeitdurchKompetenz}
Man kann die Ablehnung von Autoritätshörigkeit gewiss als eine wichtige
Konstante in der deutschen Aufklärungsphilosophie bezeichnen, die zumindest viele Autoren
eint. Im Falle der Selbstdenker wie \name[Christian]{Thomasius} und der an ihn
anschließenden Strömungen der deutschen Frühaufklärung ist sie gar zum Programm erhoben und selbst
\authorcite{Wolff:Psychologiaempirica1968}, der der \name[Christian]{Thomasius}-Anhängerschaft vielleicht unverdächtigste
Philosoph der deutschen Aufklärung, lehnt die Berufung auf Autoritäten zumindest
für einen bestimmten Personenkreis (diejenigen, die Philosophie nach
philosophischer Methode lehren sollen) \emph{expressis verbis}
ab,\footcite[Vgl.][\S~156]{Wolff:Discursuspraeliminarisdephilosophiaingenere1996} wenngleich er natürlich weit davon entfernt ist, Gedankenfreiheit in einer Weise zu fordern, wie sie uns heute selbstverständlich ist und wenigstens
partiell \name[Immanuel]{Kant} und einigen anderen Aufklärern
vorschwebte.\footnote{\cite[Vgl.][S.~xliv]{Gawlick:Einleitung1996}.}
Gerade \authorcite{Wolff:Psychologiaempirica1968} erkennt aber auch die Gefahren eines unbestimmten Begriffs
des Selbstdenkens, wenn dieser unreflektiert zum Ideal und Leitbild erklärt
wird. Es ist nichts gewonnen, wenn Freiheit und Mündigkeit bloß zu
Oberflächlichkeit und Beliebigkeit
führen.\footcite[Vgl.][Dedicatio,
\pno~259]{Wolff:Discursuspraeliminarisdephilosophiaingenere1996}


Es ist der Geist der Gründlichkeit, die Vorrangstellung der (methodisch
disziplinierten) Vernunft und das Vertrauen auf ihre Wirksamkeit, welche
\name[Immanuel]{Kant} stets an \authorcite{Wolff:Psychologiaempirica1968}
rühmt,\footnote{\cite[Vgl.~z.\,B.][B~xxxv--xxxvii]{Kant:KritikderreinenVernunft2003},
\cite[][III: 21.24--22.23]{Kant:GesammelteWerke1900ff.}.} und welche dieser in
der Philosophie des {18.} Jahrhunderts
verankerte.\footnote{\cite[Vgl.][315]{Hinske:WolffsStellunginderdeutschenAufklaerung1986}:
\enquote{Wolff hat der deutschen Aufklärung nicht nur ihre methodische Strenge
und ihre systematische Weite geschenkt, sondern auch ihren Glauben an die Macht
der Vernunft, der durch ihn ein wesentliches Moment von Aufklärung geworden
ist.}} \authorcite{Wolff:Psychologiaempirica1968}s unbestrittenes Verdienst ist
es, die Forderung nach Gründlichkeit und den unhintergehbaren Führungsanspruch
der Vernunft innerhalb der Aufklärung gegen
alle Widersacher -- Fideisten wie Freigeister und Schwärmer -- erhoben und
letztlich durchgesetzt zu
haben.\footnote{\cite[Vgl.][242]{Kreimendahl:ChristianWolff:EinleitendeAbhandlungueberPhilosophieimallgemeinen1994}:
\enquote{An seiner Demonstrationswut und der penetranten Aufdringlichkeit seiner
Methode haben sich bereits die Zeitgenossen gestoßen, darunter auch der junge
Kant \punkt . Allein der damalige philosophische Betrieb der Universitäten
scheint einen Zuchtmeister wie Wolff nötig gehabt zu haben. Er insistiert auf der
Forderung, nur das anzuerkennen, von dessen Wahrheit man durch eigene Einsicht
überzeugt ist. Sein Pochen auf die Notwendigkeit eines vom Subjekt selbst zu
erbringenden Erweises einer Wahrheit leistet einen entscheidenden Beitrag zur
Emanzipation der Vernunft von der Vorherrschaft jedweder Autorität. Dadurch
wirkt er genuin aufklärerisch und prägt wesentlich das intellektuelle Klima, in
dem noch Kant aufwächst.} Seine Logik wollte allgemeiner Standard für alle
intellektuellen Unternehmungen sein; und als \authorcite{Baumgarten:Metaphysica---Metaphysik2011} und einige Andere
sich daran machten, diesen Standard über die Mathematik, Naturforschung und
Philosophie hinaus auch auf die Theologie anzuwenden, wurde die Aufklärung
brisant \parencite[vgl.][\pno~xvii\,f.]{Gawlick:Einleitung2011}.}
Der Genius des Freidenkers bleibt -- auch dank \authorcite{Wolff:Psychologiaempirica1968} -- die Ausnahme. Und
die Frage, wie Freiheit und Mündigkeit allgemein realisierbar sind, ohne in Beliebigkeit und
geistigen Niveauverlust zu münden, bleibt ein zentrales Thema der Aufklärung.
Sie soll nicht die Wahrheit zugunsten der Freiheit und Selbständigkeit
zurückstellen, sondern beide verbinden. Und genau hier beginnt die
philosophische Herausforderung einer Explikation von Aufklärung, Selbstdenken
und Mündigkeit.

Das Reden von einem  negativen Grundsatz, die Zurückweisung von Wissensansprüchen
und nicht zuletzt die Forderung, wir sollten den \enquote{Probirstein der
Wahrheit} in uns selbst suchen, suggerieren ein Aufklärungsverständnis,
welches sein Heil in Schwärmerei und Freigeisterei sucht. Schwärmerei ist laut
\name[Immanuel]{Kant} \enquote{die Maxime der Ungültigkeit einer zu oberst
gesetzgebenden
Vernunft}\footnote{\cite[][A~327]{Kant:Washeisst:SichimDenkenorientieren?1977},
\cite[VIII: 145.25--26]{Kant:GesammelteWerke1900ff.}.}, die schließlich in
Aberglaube und Unmündigkeit
umschlage.\footnote{\cite[Vgl.][A~327]{Kant:Washeisst:SichimDenkenorientieren?1977},
\cite[][VIII: 145.27--35]{Kant:GesammelteWerke1900ff.}.} Um Selbstdenken und
Mündigkeit vor dem Hintergrund eines möglichen Konflikts mit Wahrheit und
Vernunft zu explizieren, ist auf die bereits genannte positive Beschreibung zu
rekurrieren:
\begin{quote}
  Sich seiner \ori{eigenen} Vernunft bedienen will nichts weiter sagen, als bei
  allem dem, was man annehmen soll, sich selbst fragen: ob man es wohl tunlich
  finde, den Grund, warum man etwas annimmt, oder auch die Regel, die aus dem,
  was man annimmt, folgt, zum allgemeinen Grundsatze seines Vernunftgebrauchs zu
  machen?\footnote{\cite[A~329]{Kant:Washeisst:SichimDenkenorientieren?1977},
  \cite[VIII: 146.35--147.6]{Kant:GesammelteWerke1900ff.}.}
\end{quote}\enlargethispage{\baselineskip}
Wer sich daran halte, sei
nicht nur mündig, er werde auch \enquote{Aberglauben und Schwärmerei bei dieser
Prüfung alsbald verschwinden sehen, wenn er gleich bei weitem die Kenntnisse
nicht hat, beide aus objektiven Gründen zu
widerlegen.}\footnote{\cite[A~329]{Kant:Washeisst:SichimDenkenorientieren?1977},
  \cite[VIII: 147.6--9]{Kant:GesammelteWerke1900ff.}.} Im Falle der Freiheit
erklärt \name[Immanuel]{Kant} es zu einem Missverständnis anzunehmen, wer nicht durch
Gesetze der Natur bestimmt sei, unterliege \emph{keinen} Gesetzen. Wer frei ist,
unterliegt seinen \emph{eigenen} Gesetzen, und zwar nicht denen, die er sich aus
welchen Gründen auch immer selbst \emph{gibt}, sondern denen, die seinen
Willen als \emph{vernünftigen} Wille oder als \emph{praktische Vernunft}
allererst \emph{konstituieren}.\footnote{An dieser Stelle sei lediglich betont,
dass Autonomie der praktischen Vernunft nicht heißen kann, sich beliebige
Gesetze selbst zu geben. Wer autonom handelt, der folgt damit Gesetzen der
\emph{Vernunft}. Siehe dazu Kapitel \ref{subsection:MetaphysikundAutonomie}.}
Ebenso kann Selbstdenken nicht heißen, regellos zu verfahren oder sich die
Regeln des Denkens selbst nach Gutdünken zu geben, sondern den \emph{Gesetzen}
der \emph{Vernunft} gemäß zu denken.\footnote{Die Grundlage dieser Vorstellung
ist wohl bei Hermann Samuel
\authorcite{Reimarus:DieVernunftlehrealseineAnweisungzumrichtigenGebrauchderVernunftinderErkenntnisderWahrheit1756}
zu suchen. Dies behauptet jedenfalls
\textcite[][25]{Hinske:ReimaruszwischenWolffundKant1980}:
\enquote{Thema der Philosophie, Thema der Logik und Erkenntnistheorie sind nun
[d.\,i.\ bei
\authorcite{Reimarus:DieVernunftlehrealseineAnweisungzumrichtigenGebrauchderVernunftinderErkenntnisderWahrheit1756};
A.\,G.] nicht mehr die vorgegebenen Gesetze, an denen sich die Vernunft zu
orientieren hat, sondern eben jene Regeln, die sie von Hause aus selbst
mitbringt und in deren Rahmen sie sich daher auch zwangsläufig bewegt. Die
Vernunft wird damit zu einer Quelle eigener, in ihr selbst gründender,
apriorischer Gesetze, sie wird im wörtlichen Sinne autonom.} Siehe hierzu auch
\cite{Arndt:DieLogikvonReimarusimVerhaeltniszumRationalismusderAufklaerungsphilosophie1980}.}
Doch was sind diese Gesetze? Woher stammen sie? Solche Fragen sind naturgemäß
nicht leicht zu beantworten. Das 18.
Jahrhundert hält eine Fülle an verschiedenen Deutungen parat und auch
\name[Immanuel]{Kant}s Position ist deutlich komplexer, als die Auskünfte in der
Berlinischen Monatsschrift vermuten lassen. Und auch der soeben genannte
positive Grundsatz bedarf noch der Auslegung.

\subsection{Kompetenzen und Entscheidungen}\label{Abschnitt:WolffunddieWissenschaftlichkeitderPhilosophiemoregeometrico}
Für \authorcite{Wolff:Psychologiaempirica1968} ist die Vermeidung von Oberflächlichkeit und
Beliebigkeit eine Frage der Methode, wobei die \emph{mathematische} Methode auch
in der Philosophie anzuwenden
sei.\footnote{\cite[Vgl.][\S~139]{Wolff:Discursuspraeliminarisdephilosophiaingenere1996}:
\enquote{Identitas methodi philosophicae {\&} mathematicae}.} \Revision{Diese
Methode zeichne sich durch drei Merkmale aus:
\begin{nummerierung}
\item Alle Begriffe bedürfen einer deutlichen Explikation,
\item alle Behauptungen müssen gründlich bewiesen werden, und
\item alle Behauptungen müssen gemäß der logischen Struktur ordentlich
verknüpft werde.\footnote{\Revision{\enquote{Wenn ich alles auf das genaueste
    überlege, was in der mathematischen Lehr-Art vorkommet, so finde
    ich diese drey Haupt-Stücke, 1. daß alle \ori{Wörter, dadurch die
      Sachen angedeutet werden, davon man etwas erweiset, durch
      deutliche und ausführliche Begriffe erkläret werden; 2. daß alle
    Sätze durch ordentlich an einander hangende Schlüsse erwiesen
    werden; 3. daß kein Förder-Satz angenommen wird, der nicht vorher
    wäre ausgemacht worden, und solchergestalt die folgenden Sätze mit
  dem vorhergehenden verknüpfft
  werden}} \parencite[][\S~25]{Wolff:AusfuehrlicheNachrichtvonseineneigenenSchrifftendieerindeutscherSpracheherausgegeben1973}.
Die Darstellung des \titel{Discursus} ist ausführlicher, verzichtet
jedoch auf diese übersichtliche Zusammenfassung \parencite[vgl.][\S\S~115--139]{Wolff:Discursuspraeliminarisdephilosophiaingenere1996}.}}
\end{nummerierung}}
Die Orientierung an der Mathematik und ihrem disziplinierten Vorgehen soll den einzelnen Denker
in die Lage versetzen, allein auf sich gestellt der Wahrheit nachzuforschen,
ohne sich in ungewissen Meinungen zu verlieren. Die eigene methodische Kompetenz
ermöglicht so Selbständigkeit im Verhältnis zu
anderen.\footnote{\cite[Vgl.][\S\S~156--162]{Wolff:Discursuspraeliminarisdephilosophiaingenere1996}.}
Dabei sieht \authorcite{Wolff:Psychologiaempirica1968}, dass nicht auf allen
Gebieten des Wissens die Beachtung methodischer Vorgaben im gleichen Umfange
möglich und sinnvoll ist. Mitunter sei es nicht möglich, sich streng
an die mathematische Methode zu halten, und manchmal führte dies auch nicht zu
größerer Genauigkeit, sondern übermäßiger Weitläufigkeit des Gedankengangs.
Zumindest sei die Genauigkeit der mathematischen Methode in vielen anderen
Disziplinen \emph{noch nicht} realisierbar.\footnote{\cite[Vgl.][Cap. 7,
\S~2]{Wolff:VernuenftigeGedankenvondenKraeftendesmenschlichenVerstandesundihremrichtigenGebraucheinErkenntnisderWahrheit1978}.
Siehe ebenso
\cite[][\S~148]{Wolff:Cogitationesrationalesdeviribusintellectushumani1983}.
Leider gibt \authorcite{Wolff:Discursuspraeliminarisdephilosophiaingenere1996}
keine Beispiele für diese Behauptung an.}


Auch wenn die Genauigkeit, die wir aus der Mathematik kennen, nicht in allen
Bereichen realisierbar und wünschenswert sei, bleibe es doch die Disziplinierung
des Denkers durch Übung in der Mathematik, die ihn auch auf anderen Feldern zu
Vernunft und Erkenntnis der Wahrheit befähige -- selbst dort, wo die Anwendung
der Methode nicht vollständig erfolgen kann. Nun mag es sein, dass
\authorcite{Wolff:Psychologiaempirica1968} seine Betonung von Methodik und
Systematik in Abgrenzung gegen Bestrebungen auf Freiheit und Selbstdenken
entwickelt.\footnote{\cite[Vgl.][12]{Schneiders:Deusestphilosophusabsolutesummus1986}:
\enquote{Als Wolff 1707 nach Halle kam, fand er dort eine Denkweise vor, die
mehr an Freiheit und Selbstdenken, an Praxis und Popularität als an reiner
Erkenntnis und wissenschaftlicher Gewißheit (Richtigdenken) interessiert war.
Dies mußte seinen metaphysischen und methodischen Interessen zuwiderlaufen und
sie reaktiv verstärken.}} Es ist aber nicht \authorcite{Wolff:Psychologiaempirica1968}s Absicht, das Anliegen
der Eklektiker und Selbstdenker zurückzuweisen, sondern es in sein auf
Systematik ausgerichtetes Philosophieren zu integrieren.\footnote{Man beachte
auch, dass das \singlename{Horaz}-Zitat \enquote{Sapere aude} zunächst in der an
\authorcite{Wolff:Psychologiaempirica1968}s Philosophie orientierten Vereinigung der Alethophilen
Verwendung fand
\parencite[vgl.][\pno~255\,f.]{Bronisch:WasistAufklaerung?2011}.} Gründlichkeit
und Wissenschaftlichkeit gelten ihm nicht als Alternative zu Selbständigkeit und
Selbstdenken, sondern als deren Ausdruck. Wer über einen systematischen Verstand
verfügt, der kann selbst die Wahrheit oder Falschheit einer Behauptung begründet
einsehen und muss sich nicht auf die Autorität anderer verlassen. Der
Systematiker ist daher der wahre
Eklektiker.\footnote{\phantomsection\label{Stellenverweis:Wolff:SelbstaendigkeitnurdurchKompetenz}Vgl.
\cite[\S~16]{Wolff:Dedifferentiaintellectussystematici&nonsystematici2011}:
  \enquote{Qui intellectu systematico praediti sunt, ab autoritatis praejudicio
  immunes, {\&} eclecticos agere apti sunt. Qui enim intellectu polent
  systematico, iidem non admittunt, nisi quod per principia in
  systemate contenta demonstrari potest. Judicant adeo ex intrinsecis
  rationibus\punkt{} Enimvero autoritas inter rationes extrinsecas locum
  habet ad quas confugiunt, qui intrinsecas minime capiunt.}}

\authorcite{Wolff:Psychologiaempirica1968} ist zuversichtlich, den Forderungen der Eklektiker und Selbstdenker
-- soweit sie berechtigt sind -- durch Integration in sein systematisches
Philosophieren Genüge zu tun.\footcite[Vgl.][526--538]{Albrecht:Eklektik1994}
Gerade der Systematiker sei es, der den berechtigten Forderungen der Aufklärung
genüge, weil nur er die Kompetenzen hat, die ihn von Autoritäten emanzipieren.
Selbstdenken ist also etwas, dessen derjenige überhaupt erst fähig ist, der die
philosophische Methode beherrscht. Umgekehrt ist Autoritätsgläubigkeit keine
schlechte Angewohnheit oder Neigung, die sich einfach durch eine Entscheidung
ablegen ließe. Sie ist das notwendige Resultat einer Schwäche oder Inkompetenz:
Gerade derjenige, der eine Wahrheit nicht streng beweisen kann, muss bei
schlechten  Gründen wie Autoritäten seine Zuflucht suchen.
\authorcite{Wolff:Psychologiaempirica1968} spricht von \enquote{äußeren Gründen}, also solchen, die
nicht vom Ziel der Wahrheit
ausgehen.\footnote{\cite[Vgl.][\S~155]{Wolff:Discursuspraeliminarisdephilosophiaingenere1996}.}
Wer nicht über entsprechende Erkenntnisfähigkeiten verfügt, der versuche
vergeblich, Vorurteile zu
vermeiden.\footnote{\cite[Vgl.][28--31]{Wolff:OratiodeSinarumphilosophiapractica1988}.}
Und so kann \authorcite{Wolff:Psychologiaempirica1968} die \emph{libertas philosophandi} auch auf
diejenigen beschränken, die Philosophie nach philosophischer Methode zu lehren
haben.\footcite[Vgl.][\S~166]{Wolff:Discursuspraeliminarisdephilosophiaingenere1996}
Bei ihm findet sich nichts von der Liberalität, die dem Programm und der
didaktischen Selbstverpflichtung eines \name[Christian]{Thomasius} zumindest in
dessen frühen Jahren noch
anhing.\footnote{\cite[Vgl.][241--243]{Albrecht:ChristianThomasius1999}. Zur
Ausbildung eines zunehmend pessimistischen Bildes vom Menschen bei
\name[Christian]{Thomasius}, das dessen Liberalität Abbruch tat, siehe
\cite{Engfer:ChristianThomasius1989}.} Aus \name[Christian]{Thomasius}'
Sicht ist \emph{jeder} Mensch -- unabhängig von Herkunft und Geschlecht -- in
vergleichsweise kurzer Zeit fähig, soweit gelehrt zu werden, dass er selbständig
und ohne Anleitung weiter studieren oder sein Wissen nutzbringend anwenden
könne.\footnote{\cite[Vgl.][34--36]{Thomasius:ChristianThomasiuseroeffnetDerStudirendenJugendzuLeipzigineinemDiscoursWelcherGestaltmandenenFrantzoseningemeinemLebenundWandelnachahmensolle?1994}.}
\authorcite{Wolff:Psychologiaempirica1968} traut dies allem Anschein nach nur seinem eigenen Stande des
Universitätsprofessors zu. Wer die hehren Ansprüche eines
wolffschen Systematikers erfüllen soll, muss mehr als ein paar
Semester Philosophie studiert haben. Die methodische Strenge und Systematik
wissenschaftlichen Denkens -- und das heißt bei ihm: \Revision{nach
  dem Vorbild der Mathematik} -- ist es
also, was \authorcite{Wolff:Psychologiaempirica1968} zum Ausdruck von Selbstdenken und Mündigkeit erklärt und
welche wir als seine Konzeption der Aufklärung bezeichnen können.\footnote{Ich
werde \authorcite{Wolff:Psychologiaempirica1968}s Konzeption weiter unten noch in einigen Punkten erweitern
(siehe Kapitel \ref{paragraph:wolffswarnung}), um genauer angeben zu können,
wie \name[Immanuel]{Kant} sich ihm anschließt (Kapitel
\ref{section:MuendigkeitundPhilosophie}). Hier soll aber zunächst der
systematische Rahmen zu \authorcite{Wolff:Psychologiaempirica1968}s und v.\,a.
\name[Immanuel]{Kant}s Überlegungen in diese Richtung erarbeitet werden.}

An dieser Stelle wird ein systematisch relevanter Unterschied sichtbar: Man kann
Selbstdenken einerseits als Ausdruck einer \emph{Fähigkeit} oder aber andererseits als
Ausdruck eines \emph{Entschlusses} auffassen. \authorcite{Wolff:Psychologiaempirica1968} verteidigt die
erste Sichtweise, wonach derjenige ein Selbstdenker sei, der die nötigen
Kompetenzen hat. Weder bedarf es nach dieser Ansicht eines Entschlusses zur
Unabhängigkeit, noch kann ein solcher Entschluss zum Selbstdenken führen, wo die entsprechenden
Kompetenzen nicht vorhanden sind. \name[Immanuel]{Kant} hingegen scheint die
gegenteilige Ansicht zu befördern, wenn er Aufklärung als den Ausgang aus
\enquote{selbstverschuldete[r] Unmündigkeit} bestimmt, deren \enquote{Ursache
\punkt{} nicht am Mangel des Verstandes, sondern der Entschließung und des Mutes
liegt}\footnote{\cite[][A~481]{Kant:BeantwortungderFrage:WasistAufklaerung?1977},
\cite[][VIII: 35.4--5]{Kant:GesammelteWerke1900ff.}}. Einige Bemerkungen
erinnern aber auch in \name[Immanuel]{Kant}s Schriften an die Grundzüge der Sichtweise
\authorcite{Wolff:Psychologiaempirica1968}s, dass Kompetenzen die Grundlage des Selbstdenkens bilden.
Dazu gehört, dass \name[Immanuel]{Kant}  den Kreis der Adressaten des Aufklärungsprogramms auf die
\enquote{Gelehrten} einschränkt, denen allein der freie öffentliche
Vernunftgebrauch zuzugestehen sei.\footnote{\cite[Vgl.][A
485]{Kant:BeantwortungderFrage:WasistAufklaerung?1977}, \cite[][VIII:
37.11--13]{Kant:GesammelteWerke1900ff.}: \enquote{Ich verstehe aber unter dem
öffentlichen Gebrauche seiner eigenen Vernunft denjenigen, den jemand als
Gelehrter von ihr vor dem ganzen Publikum der Leserwelt macht.}} Und auch im
Aufklärungsaufsatz betont er die Entwicklung der intellektuellen Fähigkeiten,
die sich aber erst auf Grundlage des Entschlusses und Mutes durch Öffentlichkeit
erwerben
ließen.\footnote{\cite[Vgl.][]{Kant:BeantwortungderFrage:WasistAufklaerung?1977},
\cite[][VIII: 36.10--15]{Kant:GesammelteWerke1900ff.}: \enquote{Wer sie
[d.\,i. die Fußschellen der Unmündigkeit; A.\,G.] auch abwürfe, würde dennoch
auch über den schmalsten Graben einen nur unsicheren Sprung tun, weil er zu
dergleichen freier Bewegung nicht gewöhnt ist. Daher gibt es nur Wenige, denen
es gelungen ist, durch eigene Bearbeitung ihres Geistes sich aus der
Unmündigkeit heraus zu wickeln und dennoch einen sicheren Gang zu tun.}}
\authorcite{Wolff:Psychologiaempirica1968} und \name[Immanuel]{Kant} -- so wird sich zeigen lassen
-- unterscheiden sich nicht bezüglich der Frage, \emph{ob} Selbstdenken in
intellektuellen Kompetenzen gründet, sondern hinsichtlich der Frage, \emph{wie}
solche Kompetenzen \emph{zu erwerben} sind und um \emph{welche} Kompetenzen es
sich konkret handelt.

\phantomsection\label{Abschnitt:AufklaerungundMuendigkeitdurchKompetenz-Ende}
\name[Immanuel]{Kant} schätzt den Vorteil einer Orientierung an der äußeren Form
der Mathematik schon 1763 als gering
ein\footnote{\enquote{Der Gebrauch, den man in der Weltweisheit von der Mathematik machen kann, bestehet
entweder in der Nachahmung ihrer Methode, oder in der wirklichen Anwendung ihrer
Sätze auf die Gegenstände der Philosophie. Man siehet nicht, daß der erstere
bis daher von einigem Nutzen gewesen sei, so großen Vorteil man sich auch
anfänglich davon versprach}
(\cite[][A~i]{Kant:VersuchdenBegriffdernegativenGroessenindieWeltweisheiteinzufuehren1977},
\cite[][II: 167.2--6]{Kant:GesammelteWerke1900ff.}).} und führt dies 1764 weiter
aus\footnote{\cite[Vgl.][A
71--79]{Kant:UntersuchungueberdieDeutlichkeitderGrundsaetzedernatuerlichenTheologieundderMoral1977},
\cite[][II: 276.1--283.9]{Kant:GesammelteWerke1900ff.}.}. In der \titel{Kritik
der reinen Vernunft} widmet er der Kritik der Anwendung der mathematischen Methode ein Kapitel der Methodenlehre.\footnote{\cite[Vgl.][B~740-766]{Kant:KritikderreinenVernunft2003},
\cite[][III: 468.22--483.32]{Kant:GesammelteWerke1900ff.}: \enquote{Die
Disziplin der reinen Vernunft im dogmatischen Gebrauche}.} Die Methode der \Revision{Mathematik} sei
zwar der Mathematik angemessen, für die Philosophie aber völlig unbrauchbar,
weil diese nicht mit Definitionen beginnen könne, sondern bei ihnen erst ende;
außerdem seien Axiome nur durch die reine Anschauung in der Mathematik
erhältlich. Der endliche Verstand habe außerhalb der Mathematik gar keine
solchen Sätze zur Verfügung, mit denen er es dem mathematischen Denken gleich tun
könnte.\footnote{\name[Immanuel]{Kant}s Zurückweisung der Identifizierung von
philosophischer und mathematischer Methode ist gut erforscht. Siehe z.\,B.
\cite[][26--67]{Engfer:PhilosophiealsAnalysis1982}, sowie
\cite[][42--101]{Wolff-Metternich:DieUeberwindungdesmathematischenErkenntnisideals1995}.
Siehe zur Bedeutung der Differenz mathematischen und philosophischen Erkennens
für die Themen \enquote{Aufklärung} und \enquote{Endlichkeit des Denkens} auch
unten Kapitel \ref{subsubsection:EndlichesundUnendlichesErkennen}.}

Nun redet \name[Immanuel]{Kant} selbstverständlich nicht Disziplinlosigkeit,
Schwärmerei und Geniekult das Wort.\footnote{Man beachte etwa den emphatischen Appell, die
eigene Freiheit des Denkens und der Feder nicht zu missbrauchen, mit dem er den
Aufsatz \titel{Was heißt: sich im Denken Orientieren?} schließt: \enquote{Freunde des
Menschengeschlechts und dessen, was ihm
am heiligsten ist! Nehmt an, was euch nach sorgfältiger und aufrichtiger Prüfung
am glaubwürdigsten scheint, es mögen nun Facta, es mögen Vernunftgründe sein;
nur streitet der Vernunft nicht das, was sie zum höchsten Gut auf Erden macht,
nämlich das Vorrecht ab, der letzte Probierstein der Wahrheit zu sein.
Widrigenfalls werdet ihr, dieser Freiheit unwürdig, sie auch sicherlich
einbüßen, und dieses Unglück noch dazu dem übrigen schuldlosen Teile über den
Hals ziehen, der sonst wohl gesinnt gewesen wäre, sich seiner Freiheit
\ori{gesetz}mäßig und dadurch auch zweckmäßig zum Weltbesten zu bedienen!}
(\cite[][A~329\,f.]{Kant:Washeisst:SichimDenkenorientieren?1977},
\cite[][VIII: 146.23-147.4]{Kant:GesammelteWerke1900ff.})} Eine
\enquote{Freiheit im Denken, wenn sie so gar unabhängig von Gesetzen der
Vernunft verfahren will,} zerstöre \enquote{endlich sich
selbst.}\footnote{\cite[][A~328]{Kant:Washeisst:SichimDenkenorientieren?1977},
\cite[][VIII: 146.21--22]{Kant:GesammelteWerke1900ff.}.} Nur ist es bei
\name[Immanuel]{Kant} nicht mehr die eine Methode des \authorcite{Wolff:Psychologiaempirica1968}ianismus, die das Denken
bestimmen soll; stattdessen bleibt zunächst nur der abstrakte Appell an einen
durch \authorcite{Wolff:Psychologiaempirica1968} etablierten \enquote{bisher noch nicht
erloschenen Geist \punkt{} der
Gründlichkeit}\footnote{\cite[][B~xxxvi]{Kant:KritikderreinenVernunft2003},
\cite[III: 22.9]{Kant:GesammelteWerke1900ff.}.}.
\phantomsection\label{Terminus:methodischerNaturalismus}\name[Immanuel]{Kant}
verwirft die Ansicht, die er die \enquote{naturalistische Methode} nennt,
{d.\,i.} die Auffassung, jeder brächte (speziell in der Metaphysik) die nötigen
Kompetenzen in seinem Urteilen von Haus aus mit, ohne einer speziellen Ausbildung zu
bedürfen.\footnote{\cite[Vgl.][B~883\,f.]{Kant:KritikderreinenVernunft2003},
\cite[][III: 551.30--552.13]{Kant:GesammelteWerke1900ff.}.} Der mündige Denker
ist der -- zumindest in gewissen Ausmaßen -- methodisch ausgebildete Denker, der
in seiner Ausbildung zwar nicht notwendig bestimmte Inhalte gelernt, wohl aber
gewisse Kompetenzen erworben haben muss.\footnote{Gewisse Einschränkungen wird
diese Behauptung in Kapitel \ref{chapter:AufklaerungundWissenschaft} erfahren.
Speziell Kapitel \ref{subsection:DieBestimmungdesMenschen} wird sich mit
\singlequote{pragmatischem} Wissen auseinandersetzen, welches für Aufklärung und
Mündigkeit relevant ist. Es handelt sich dabei primär um Wissen bzgl. der
\emph{conditio humana}.} Dies wird dadurch gestützt, dass das Selbstdenken als
\enquote{Maxime der \ori{vorurteilsfreien} \punkt{} Denkungsart}, die sich an
den Verstand richte, in der \titel{Kritik der Urteilskraft} sowie in der
\titel{Anthropologie in pragmatischer Hinsicht} durch zwei weitere Maximen
flankiert wird: Einerseits durch die Maxime der \emph{erweiterten} oder
\emph{liberalen} \enquote{Denkungsart} der Urteilskraft, andererseits durch die
Maxime einer \enquote{\ori{konsequenten} Denkungsart} der
Vernunft.\footnote{\cite[Vgl.][\S~40]{Kant:KritikderUrteilskraft2009}, \cite[V:
294.14--295.19]{Kant:GesammelteWerke1900ff.};
\cite[BA~166\,f.]{Kant:AnthropologieinpragmatischerHinsicht1977}, \cite[VII:
228.27--229.2]{Kant:GesammelteWerke1900ff.}.} Es ist die dritte Maxime (der
konsequenten Art des Denkens), auf die \authorcite{Wolff:Psychologiaempirica1968} sich konzentrierte, indem er
sie in der Methodik der Mathematik paradigmatisch realisiert
sah.\footnote{Katerina
\textcite[vgl.][151]{Deligiorgi:UniversalisabilityPublicitaandCommunication2002}
sieht in der dritten Maxime nicht die Forderung nach Konsistenz oder
methodischer Stimmigkeit artikuliert, sondern speziell die Forderung danach, die
beiden ersten Maximen in Übereinstimmung zu bringen. Dennoch schließt sie die
Forderung nach der methodischen Stimmigkeit der Erkenntnisse selbst natürlich
mit ein.} Auch \name[Immanuel]{Kant} denkt an den freien Gebrauch einer
\emph{disziplinierten} Vernunft, aber er gibt sich
nicht der Illusion hin, dass hierzu schlicht bestimmte Regeln der Vernunft
befolgt werden müssten, die wir als solche in einem Lehrbuch niederschreiben und
lernen könnten. Die Orientierung an der
Vernunftform lasse sich nur auf der Grundlage der beiden anderen Maximen (der
vorurteilsfreien und der erweiterten Art des Denkens)
erreichen.\footnote{\cite[Vgl.][\S~40]{Kant:KritikderUrteilskraft2009}, \cite[V:
295.14--17]{Kant:GesammelteWerke1900ff.}: \enquote{Die dritte Maxime, nämlich
die der \ori{konsequenten} Denkungsart, ist am schwersten zu erreichen und kann auch
nur durch die Verbindung beider ersten und nach einer zur Fertigkeit gewordenen
öfteren Befolgung derselben erreicht werden.}}
Konsequent (der Form oder den Gesetzen der Vernunft gemäß) zu denken, setze also
voraus, selbst zu denken. Und selbst zu denken verlange, konsequent zu denken,
wie wir soeben dem Appell der Orientierungsschrift entnehmen konnten.
Beide Forderungen kommen darin überein, dass die Regeln, denen die Vernunft
unterworfen ist, gerade als Ausdruck und Konkretisierung der Freiheit
verstanden werden. Eine solche Konzeption, die Freiheit und Unterwerfung unter
Regeln nicht durch Entgegensetzung bestimmt, die die Gesetze nicht als äußere
Beschränkungen versteht, sondern als ihre innere Verwirklichung, bezeichnet man
mit dem Namen
\enquote{Autonomie}.\footnote{\cite[Vgl.][7]{Khurana:ParadoxienderAutonomie2011}.}
Und auch \name[Immanuel]{Kant} selbst nennt die Vernunft \enquote{das Vermögen, nach der
Autonomie, {d.\,i.} frei (Prinzipien des Denkens überhaupt gemäß) zu
urteilen}\footnote{\cite[A~25]{Kant:DerStreitderFakultaeten1977}, \cite[VII:
27.30--31]{Kant:GesammelteWerke1900ff.}.}. Ebenso wie ein freier Wille und ein
Wille unter sittlichen Gesetzen einerlei
sind\footnote{\cite[Vgl.][BA~98]{Kant:GrundlegungzurMetaphysikderSitten1965},
\cite[IV: 447.6--7]{Kant:GesammelteWerke1900ff.}.}, so ist auch eine freie
-- aufgeklärte oder mündige -- Vernunft eine Vernunft unter Vernunftgesetzen.
Und diese Idee der Autonomie gilt es als Autonomie des Denkens zu explizieren.

\subsection{Regeln mündigen Denkens}
Welche Regeln sind dies also, denen das Denken als freies Denken unterworfen
ist? Zumindest an einer Stelle -- am Schluss des Aufsatzes \titel{Was
heißt: sich im Denken orientieren?} -- gibt \name[Immanuel]{Kant} auf diese
Frage eine konkrete Antwort. Ich zitiere die Stelle erneut, um sogleich an einem Beispiel eine
Deutung zu versuchen:
\begin{quote}
  Sich seiner \ori{eigenen} Vernunft bedienen will nichts weiter sagen, als bei
  allem dem, was man annehmen soll, sich selbst fragen: ob man es wohl tunlich
  finde, den Grund, warum man etwas annimmt, oder auch die Regel, die aus dem,
  was man annimmt, folgt, zum allgemeinen Grundsatze seines Vernunftgebrauches
  zu machen?\footnote{\cite[][A
  329]{Kant:Washeisst:SichimDenkenorientieren?1977},
  \cite[][VIII: 146.35--37, 147.5--7]{Kant:GesammelteWerke1900ff.}.}
\end{quote}
\name[Immanuel]{Kant}s Formulierung ähnelt offenkundig der des kategorischen
Imperativs: \enquote{[H]andle nur nach derjenigen Maxime, durch die du
zugleich wollen kannst, daß sie ein allgemeines Gesetz
werde.}\footnote{\cite[][BA 52]{Kant:GrundlegungzurMetaphysikderSitten1965},
\cite[][IV:421.7--8]{Kant:GesammelteWerke1900ff.}. In der \titel{Kritik der
praktischen Vernunft} lautet die Formulierung: \enquote{Handle so, daß die
Maxime deines Willens jederzeit zugleich als Princip einer allgemeinen Gesetzgebung gelten könne}
(\cite[][\S~7]{Kant:KritikderpraktischenVernunft1974}, \cite[][V:
30.38--39]{Kant:GesammelteWerke1900ff.}).} Zunächst setzt dies voraus, dass wir
nicht willkürlich im Einzelfall ohne jede allgemeine Orientierung entscheiden,
was wir tun wollen, sondern nach \distanz{Maximen} oder
\singlequote{allgemeinen Grundsätzen} handeln.\footnote{\name[Immanuel]{Kant}
sagt nirgends ausdrücklich, dass wir nach Maximen handeln, aber er setzt dies erstens in seinen Formulierungen des KI
voraus, sagt zweitens ausdrücklich, dass Menschen \emph{nach Regeln} handeln,
und sagt außerdem, dass Maximen subjektive Regeln oder Prinzipien sind, also
Regeln, die je mir eigen sind.} Und entsprechend fordert die Explikation des
Begriffs des Selbstdenkens eine Orientierung an allgemeinen Grundsätzen des
Denkens: Selbstdenken heißt hier also, sich nach allgemeinen Grundsätzen des
Denkens zu richten, \emph{die man selbst als vernünftig ansieht}. Die letzte
Einschränkung ist nötig, weil letztlich jedes Denken -- \emph{qua} Denken --
sich an Grundsätzen orientieren muss. Auch und gerade das vorurteilsbehaftete
Denken folgt Grundsätzen, denn gerade in bestimmten Grundsätzen des Denkens
bestehen Vorurteile: \enquote{Vorurteile sind vorläufige Urteile, \ori{in so
ferne sie als Grundsätze angenommen werden}.}\footnote{\cite[][A
116]{Kant:ImmanuelKantsLogik1977}, \cite[][IX:
75.24--25]{Kant:GesammelteWerke1900ff.}. Siehe dazu die Parallelstelle in
\cite[][\nopp 2538]{Kant:Reflexionen1900ff.},
\cite[][XVI: 409.5]{Kant:GesammelteWerke1900ff.}.} Es wäre eine Illusion zu
glauben, wir könnten im Denken auf Grundsätze schlechthin verzichten; wir können
lediglich beeinflussen, \emph{welchen} Regeln wir unser Denken unterwerfen -- ob
wir eigenen (Autonomie) oder fremden (Heteronomie) Regeln folgen.\footnote{Die
ist die Pointe des Gedankengangs, dass ein Denken, welches sich \emph{keinen} Regeln zu
unterwerfen gedenkt, letztlich fremden Regeln unterworfen wird
\mkbibparens{\cite[vgl.][A
326\,f.,]{Kant:Washeisst:SichimDenkenorientieren?1977}, \cite[][VIII:
145.6--35]{Kant:GesammelteWerke1900ff.}}.}

\name[Immanuel]{Kant} spricht hier bewusst nicht von der Übereinstimmung der
Vernunft mit einem externen Maßstab, sondern mit Gesetzen, die sie sich selbst
gibt.\footnote{\cite[Vgl.][A 326]{Kant:Washeisst:SichimDenkenorientieren?1977},
\cite[][VIII: 145.6--7]{Kant:GesammelteWerke1900ff.}.}
Es geht also nicht darum, Standards vernünftigen Überlegens zu finden, die der Vernunft
von außen angetragen werden. Bezüglich der Moral fragt \name[Immanuel]{Kant},
wie es sein könne, dass unser Wille Gesetzen unterworfen ist, und beantwortet
dies damit, dass diese Gesetz eben nicht von außen an unseren Willen
herangetragen werden, sondern er nur der eigenen, aber allgemeinen Gesetzgebung
unterworfen sei. Nur durch die Vorstellung der Autonomie seien Pflicht und
Verbindlichkeit denkbar.\footnote{\cite[Vgl.][BA
73\,f.,]{Kant:GrundlegungzurMetaphysikderSitten1965} \cite[][IV:
432.25--433.11]{Kant:GesammelteWerke1900ff.}.} Ebenso wäre es ganz
verwunderlich, wenn unsere Vernunft Gesetzen unterworfen würde, die ihr
äußerlich sind. Wir müssten fragen -- und könnten diese Frage vermutlich nicht
beantworten --, was diesen Gesetzen die Autorität verleiht, unser Denken zu
reglementieren. Aber dadurch, dass es sich um \emph{interne} Normen handelt, die
unser Denken selbst ausmachen, verschwindet diese Merkwürdigkeit. Es ist die
Vernunft selbst, die sich die Gesetze des Denkens gibt und daher als Vernunft
ihnen auch unterworfen ist.


Wenn wir uns selbst fragen, ob unser Gedanke vernünftig
ist, sollten wir die Frage also in folgender Form stellen: Folge ich in meinem
Urteil einer epistemischen Regel, die ich nicht nur hier anwenden möchte,
sondern die ich \emph{stets} anzuwenden bereit bin, wenn sich ein Anwendungsfall
dieser Regel findet? Es mag beispielsweise die Frage im Raume
stehen, ob etwas für wahr gehalten werden sollte, weil es seit geraumer Zeit im
Dorf erzählt wird. Wer sich um Selbständigkeit und Mündigkeit bemüht, ist
gehalten, sich nun zu fragen, ob er es richtig fände, der allgemeinen Regel zu
folgen, für wahr zu halten, was im Dorf erzählt wird. Da dort so manche
Despektierlichkeit im Umlauf ist, wird der mündige Denker dies für keine gute
Idee halten und also sein Urteil auch in diesem konkreten Fall nicht auf das
Gerücht im Dorf gründen. Wir haben also zwar keine konkreten Regeln an der Hand,
wohl aber einen Verallgemeinerungstest für mögliche Grundsätze unseres Denkens,
der deutlich an die Universalisierungsformel des Kategorischen Imperativs
erinnert. Leider lässt diese Regel Vieles offen. Sollte ich es mir etwa zur
Regel machen, Informationen von Wikipedia zu vertrauen? Hier können verschiedene
Antworten gegeben werden, von denen sich keine unmittelbar als unvernünftig
erweist. Warum sollte jemand nicht den allgemeinen Grundsatz annehmen,
bestimmten Autoritäten (etwa seinem Pfarrer) \emph{alles} zu glauben? Es ist
nicht (zumindest nicht offensichtlich) widersprüchlich, es sich selbst zum
Grundsatz zu machen, unmündig zu bleiben. Und \emph{a fortiori} reicht es zum
Selbstdenken nicht, bloß logisch konsistent zu denken -- womit auch diese
Interpretation des \singlequote{positiven} Begriffs des Selbstdenkens ausgeschlossen ist.

Wie deutlich wird, erlaubt die Formulierung dessen, was Selbstdenken heißt, noch
eine Vielzahl möglicher Grundsätze des
Denkens.\footnote{\cite[Vgl.][224]{ONeill:AufgeklaerteVernunft1996}:
\enquote{Das Denkprinzip, im Denken nur solche Grundprinzipien zu verwenden, die
universell anerkannt werden können, beschränkt wohl die Kategorien des Denkens,
die Urteils- und Schlußweisen, die wir rationaliter anerkennen können, bestimmt
sie aber nicht vollständig. Kant bietet keinen Algorithmus für den vernünftigen
Erwerb von Überzeugungen.}} Dass wir nach universellen Grundsätzen urteilen
sollen, schließt somit nicht die Möglichkeit aus, dass die Beliebigkeit nun in
die Wahl der Grundsätze verlegt wird. \name[Immanuel]{Kant}s
Aufklärungsauffassung läuft somit trotz der eigenen Beschwichtigungen zumindest
\emph{prima facie} Gefahr, ein beliebiges Meinen und Fürwahrhalten auf Grund
persönlicher Neigungen und Vorlieben zu protegieren.
Er vertritt zwar gewiss nicht einen \enquote{gesetzlose[n]} Gebrauch
der Vernunft als \enquote{die Denkungsart \punkt , die man \ori{Freigeisterei}
nennt, d.\,i.\ den Grundsatz, gar keine Pflicht mehr zu
erkennen}\footnote{\cite[][A 328]{Kant:Washeisst:SichimDenkenorientieren?1977},
\cite[][VIII: 146.15--16]{Kant:GesammelteWerke1900ff.}.}. Aber ebenso wenig
schließt er sich der Konzeption
\authorcite{Wolff:Discursuspraeliminarisdephilosophiaingenere1996}s an, die
Freigeisterei und Beliebigkeit durch Rückbindung des Selbstdenkens an die
\Revision{mathematische} Methode zu verhindern sucht. Es gibt nach \name[Immanuel]{Kant}
keinen allgemeinen Standard, der vernünftige Überlegungen von unvernünftigen
unterscheidet. Dann jedoch gilt es, einen anderen Weg zu finden, die
Vernünftigkeit je eigenen Denkens sicherzustellen.


In der \titel{Kritik der Urteilskraft} im Kontext einer
Exposition des \enquote{\emph{sensus communis}}, der nicht nicht die
Berufung auf allgemein geteilte und als plausibel gewertete Vorannahmen eines
\distanz{common sense} oder einen als unkultiviert gedachten \distanz{gemeinen
Menschenverstand} bezeichnet, sondern thematisch werden lässt, dass Denken die
Tätigkeit sozial lebender Wesen ist, behandelt \name[Immanuel]{Kant} drei
Maximen eines an der Aufklärung orientierten Denkens.
Nur wenn wir in der Lage sind, in unseren Urteilsakten unser Denken nicht an
subjektiven Privatbedingungen und beliebigen Wünschen und Einfällen, sondern an
der allgemein Menschenvernunft zu orientieren, können wir überhaupt objektive
Urteile fällen.\footnote{\cite[Vgl.][\S~40]{Kant:KritikderUrteilskraft2009}, \cite[][V:
293.30--36]{Kant:GesammelteWerke1900ff.}: \enquote{Unter dem \ori{sensus
communis} aber muß man die Idee eines \ori{gemeinschaftlichen} Sinnes, d.\,i.\
eines Beurteilungsvermögens verstehen, welches in seiner Reflexion auf die
Vorstellungsart jedes anderen in Gedanken (a priori) Rücksicht nimmt, um
gleichsam an die gesamte Menschenvernunft sein Urteil zu halten und dadurch der
Illusion zu entgehen, die aus subjektiven Privatbedingungen, welche leicht für
objektiv gehalten werden könnten, auf das Urteil nachteiligen Einfluß haben
würde.}} Dieser \emph{sensus communis}, der auf die gemeinschaftliche Grundlage und
Ausrichtung des Denkens verweist, konkretisiert sich also in drei Maximen des
gemeinen Menschenverstandes, die angeben, worin die \emph{Kultivierung} des
Denkens im Sinne einer liberalen Aufklärung besteht. Die Maximen des gemeinen Menschenverstandes
sind folgende\footnote{\cite[Vgl.][\S~40]{Kant:KritikderUrteilskraft2009},
\cite[][V: 294.14--295.19]{Kant:GesammelteWerke1900ff.}. Nach
\authorfullcite{Cohen:KantontheEthicsofBelief2014} handelt es sich um Analoga zu
den drei Formulierungen des \emph{Kategorischen Imperativs}
\parencite[vgl.][330]{Cohen:KantontheEthicsofBelief2014}.}:
\begin{nummerierung}
 \item Die \emph{Maxime der vorurteilsfreien Denkungsart} ist die Maxime des
 Verstandes und einer niemals passiven Vernunft, welche wir bereits in der
 Form des \enquote{Sapere aude! Habe Mut, dich deines \ori{eigenen} Verstandes zu
 bedienen!}\footnote{\cite[][A
 481]{Kant:BeantwortungderFrage:WasistAufklaerung?1977},
 \cite[][VIII: 35.6--7]{Kant:GesammelteWerke1900ff.}.} kennen.
 \item Die \emph{Maxime der erweiterten Denkungsart} ist die Maxime der
 Urteilskraft. Sie fordert von uns einen \emph{Perspektivenwechsel} im Denken
 und steht im Kontrast zur bornierten Art zu denken. Sie fordert uns dazu auf,
 an der Stelle jedes anderen zu denken, also bei unseren Urteilen und
 Gedankengängen zu berücksichtigen, ob andere diese ebenfalls als vernünftig
 ansehen können.
 \item Die \emph{Maxime der konsequenten Denkungsart} ist die Maxime der
 Vernunft. Sie fordert von uns, dass wir mit uns selbst einstimmig denken. Das
 heißt, wir sollen nicht beliebige Gedanken und Behauptungen aneinander reihen,
 sondern uns in unserem Denken an den Regeln unserer eigenen Vernunft
 orientieren. Es ist also gerade die Maxime, die zu bewahren zentrales Anliegen
 \name[Immanuel]{Kant}s im Anschluss an
 \authorcite{Wolff:Discursuspraeliminarisdephilosophiaingenere1996} ist. Diese
 Maxime sei \enquote{am schwersten zu erreichen und} könne \enquote{auch nur
 durch die Verbindung beider ersten und nach einer zur Fertigkeit gewordenen
 öfteren Befolgung derselben erreicht
 werden.}\footnote{\cite[][\S~40]{Kant:KritikderUrteilskraft2009}, \cite[][V:
 295.15--17]{Kant:GesammelteWerke1900ff.}.}
\end{nummerierung}
Interessant ist hier die Angabe, dass sich Vernunft im Denken nur durch die
Verbindung der beiden anderen Maximen erreichen lasse. Selbstdenken und
erweiterte Denkungsart scheinen also konstitutiv zu sein für ein Denken, das
weder der Beliebigkeit anheim fällt noch heteronom wird, sondern sich einzig
und allein denjenigen Regeln unterwirft, die unsere Vernunft ausmachen.

\section{Selbstdenken als Denken in Gemeinschaft}\label{section:sensuscommunis}
Dass wir selbst denken, unsere eigene Vernunft als obersten Probierstein der
Wahrheit ansehen und uns keinen Autoritäten beugen sollen, ist die
zentrale Maxime der Aufklärung. Aber Aufklärung fordert von uns auch, uns am
Maßstab der Vernunft zu orientieren; und dies ist nicht einfach eine
weitere Maxime, sondern eine direkte Folgerung der ersten Maxime, wie
\authorfullcite{Wolff:Discursuspraeliminarisdephilosophiaingenere1996} betont:
Selbst denken kann nur, wer sich selbst kompetent am Maßstab der Vernunft
orientiert. Deswegen fallen Unabhängigkeit und Vernunft wesentlich zusammen.
Nach \authorcite{Wolff:Discursuspraeliminarisdephilosophiaingenere1996} besteht der
Maßstab der Vernunft in der mathematischen Methode, die auch die Methode der
Philosophie sei, gerade weil sie nichts anderes als das Wesen der Vernunft zum
Ausdruck
bringe.\footnote{\cite[Vgl.][\S~161]{Wolff:Discursuspraeliminarisdephilosophiaingenere1996}.}
Nun verneint \name[Immanuel]{Kant} die
Möglichkeit, einen solchen allgemeinen Standard für vernünftiges Denken im Sinne
einer einheitlichen Methode anzugeben. Wie ist es dann aber möglich zu
gewährleisten, dass die Befolgung der ersten Maxime nicht in die Vernachlässigung der zweiten Maxime
mündet -- dass Selbstdenken nicht zu \singlequote{Freidenkerei} und Beliebigkeit
führt?

\subsection{Logischer Egoismus versus Pluralismus}
Wir haben bereits gesehen, dass sich die Vernunft an Grundsätzen orientieren
muss und dass diese Grundsätze wiederum auf ihre Vernünftigkeit hin befragt
werden müssen. Grundsätze des Vernunftgebrauchs sind entweder solche, die der
Vernunft selbst entstammen (Autonomie) oder solche, die ihr von außen angetragen
werden (Heteronomie). Wie entscheiden wir aber, ob ein Grundsatz ein solcher der
Vernunft oder ein Vorurteil ist?

In der \titel{Kritik der Urteilskraft} ebenso wie in der \titel{Anthropologie in
pragmatischer Hinsicht}\footnote{\cite[Siehe][B
166\,f.,]{Kant:AnthropologieinpragmatischerHinsicht1977} \cite[][VII:
228.10--229.2]{Kant:GesammelteWerke1900ff.}.} verweist \name[Immanuel]{Kant} auf
die Maxime der erweiterten oder liberalen Art zu denken. Er versteht darunter
die Maxime, \enquote{sich über die subjektiven
Privatbedingungen des Urteils, wozwischen so viele andere wie eingeklammert
sind, weg[zu]setzen \punkt{} und aus einem \ori{allgemeinen Standpunkte} (den
[man] nur dadurch erreichen kann, daß [man] sich in den Standpunkt anderer
versetzt) über sein eigenes Urteil [zu]
reflektier[en].}\footnote{\cite[\S~40]{Kant:KritikderUrteilskraft2009},
\cite[][V: 295.10--14]{Kant:GesammelteWerke1900ff.}.} Hiernach sollen wir nicht
nur unsere Grundsätze einem Verallgemeinerungstest unterstellen, sondern unsere
Urteile auch aus einer anderen Perspektive (aus der Sicht eines anderen)
betrachten. In der Anmerkung am Ende von \titel{Was heißt: sich im Denken
orientieren?} übergeht \name[Immanuel]{Kant} zwar diese auf die Gemeinschaft mit
anderen verweisende \distanz{mittlere} Maxime zwischen dem Selbstdenken und der Unterwerfung unter
Gesetze der Vernunft.\footnote{Claudio
\textcite[][133]{LaRocca:WasAufklaerungseinwird2004} weist darauf hin, dass die
zitierte Anmerkung die Differenz zwischen der ersten und der dritten Maxime
quasi mit einem Schlag überbrücke.} Er insistiert aber kurz zuvor im Haupttext
auf die soziale Eingebundenheit unseres Denkens.\footnote{\cite[Vgl.][A
325]{Kant:Washeisst:SichimDenkenorientieren?1977}, \cite[][VIII:
144.17--22]{Kant:GesammelteWerke1900ff.}: \enquote{Zwar sagt man:
die Freiheit zu sprechen, oder zu schreiben, könne uns zwar durch obere Gewalt,
aber die Freiheit zu denken durch sie gar nicht genommen werden. Allein, wie
viel und mit welcher Richtigkeit würden wir wohl denken, wenn wir nicht
gleichsam in Gemeinschaft mit anderen, denen wir unsere und die uns ihre
Gedanken mitteilen, dächten!}} Er sieht Intersubjektivität für
einen wesentlichen Gesichtspunkt bei der Vereinbarkeit von Selbstdenken und
Richtigdenken an, der sich daraus ergibt, dass Vernunftmaßstäbe nicht in der
Form einer allgemeinen Methode wie der mathematischen zu haben
sind.\footcite[Vgl.][\pno~214\,f.]{ONeill:AufgeklaerteVernunft1996}

Man kann an dieser Stelle noch mit Gründen bezweifeln, dass dies bereits das
allgemeine Vorurteil über die \index{Kant, Immanuel}kantische Philosophie aus den Angeln
hebt, diese sei in ihrem Kern solipsistisch und trage einen monologischen
Charakter. Denn \name[Immanuel]{Kant} behauptet, dass man sich \emph{dadurch}
von den subjektiven Privatbedingungen emanzipiere und aus einem allgemeinen
Standpunkt reflektiere, dass
\begin{quote}
  man sein Urteil an anderer nicht sowohl wirkliche, als vielmehr bloß mögliche
  Urteile hält und sich in die Stelle jedes anderen versetzt, indem man bloß von
  den Beschränkungen, die unserer eigenen Beurteilung zufälligerweise anhängen,
  abstrahiert; welches wiederum dadurch bewirkt wird, daß man das, was in dem
  Vorstellungszustande Materie, {d.\,i.} Empfindung ist, soviel möglich wegläßt
  und lediglich auf die formalen Eigentümlichkeiten seiner Vorstellungen oder
  seines Vorstellungszustandes
  achthat.\footnote{\cite[\S~40]{Kant:KritikderUrteilskraft2009}, \cite[V:
  294.1--8]{Kant:GesammelteWerke1900ff.}.}
\end{quote}
Auch wenn man einmal von den Schwierigkeiten absieht, die die Forderung mit sich
bringt, von der Materie eines Urteils abzusehen und nur auf das Formale zu
achten, drängt sich doch eine Beobachtung auf: \name[Immanuel]{Kant} betrachtet
zumindest an dieser Stelle die gedankliche Operation, in der wir uns in den Standpunkt anderer
versetzen, nicht als Akt der Kommunikation und des gedanklichen Austausches, sondern als die einsame
Überlegung eines Denkers, der bloß von seinen eigenen Eigentümlichkeiten
absieht. Er verweist nicht auf den öffentlichen Vernunftgebrauch der
Aufklärungsschrift, sondern auf das Abstraktum einer allgemeinen
Menschenvernunft.\footnote{\Revision[Thei, Pelletier]{Dass wir auch abhängig
sind von einer Vorgeschichte, die es erst ermöglicht, dass wir
uns als autonome Subjekte konstituieren, ist eine Überlegung, die sich bei
\name[Immanuel]{Kant} noch nicht findet (und sich bei ihm vielleicht noch gar
nicht finden kann). \authorfullcite{Foucault:WasistAufklaerung1990} wird aus
solchen Gründen fordern, philosophische Kritik müsse in ihrem Bemühen um
Aufklärung heute nicht mehr transzendental, sondern archäologisch verfahren und
statt vermeintlich zeitloser formaler Strukturen die für uns (für unser Denken
und Handeln) konstitutiven Ereignisse und Umstände aufsuchen
\parencite[vgl.][]{Foucault:WasistAufklaerung1990}.}} Schließlich -- so ließe
sich eine entsprechende Position im Sinne der Aufklärung verteidigen -- muss die
Vernünftigkeit von Grundsätzen des Denkens von jeder faktischen Gemeinschaft und
ihren geteilten Überzeugungen in gewisser Hinsicht unabhängig sein, wenn der
Vernunft nicht doch der Beigeschmack von Heteronomie anhaften soll. Jedenfalls
kann ein Grundsatz nicht allein darum vernünftig oder unvernünftig sein, weil
die existierenden Mitglieder einer Kommunikationsgemeinschaft ihn als solchen
anerkennen oder ablehnen.\footnote{Siehe dazu auch
\cite[][20]{ONeill:ConstructionsofReason1989}: \enquote{At first thought the
idea of modeling reason on free debate may seem to add nothing. First, we may
suspect, this account too will only be negative instruction: Debates do not
usually produce agreement; hence this image adds nothing to that of the
\enquote{tribunal} of reason. Second, debates presuppose reason, so we cannot
draw on the notion of debate to explain the authority of reason. Third, we may doubt
that the prospects for uncoerced debate are any rosier that those for tribunals
that do not rest on power relations.}} Rassistische Vorurteile bleiben ja auch
dann Vorurteile, wenn wir ausschließlich von Rassisten umgeben sind. Dies mag
wie eine Leerstelle in \name[Immanuel]{Kant}s System aussehen, sich letztlich
jedoch als korrekte Einsicht erweisen. Es bleibt daher mit Birgit
\name[Birgit]{Recki} erst zu fragen: \enquote{Gibt es bei Kant Anknüpfungspunkte
für die Einsicht in die Angewiesenheit der menschlichen Erkenntnis auf Sprache,
in die Notwendigkeit der Mitteilung, des Austausches und der Auseinandersetzung in dem, worauf sich das
Interesse vernünftiger Wesen richtet -- in den dialogischen Charakter von
Vernunftleistungen?}\footnote{\cite[114]{Recki:enquoteAnderStelle[je]desanderendenken2006}.
Sie stellt daher explizit die Frage, ob diese zweite Maxime des gemeinen
Menschenverstandes überhaupt etwas mit Kommunikation zu tun hat
\parencite[vgl.][116]{Recki:enquoteAnderStelle[je]desanderendenken2006}.} Sollte
die Maxime der erweiterten Denkungsart nicht auf tatsächliche Akte der
Kommunikation, sondern auf das Abstraktum einer allgemeinen Menschenvernunft
verweisen, wäre sie mit einer individualistischen Hintergrundtheorie vereinbar.

\phantomsection\label{Abschnitt:KantunddieOeffentlichkeitderVernunft}Der
Interpretation \name[Immanuel]{Kant}s als eines solipsistischen und
monologischen Denkers ist bereits mehrfach ausführlich und mit überzeugenden
Argumenten widersprochen
worden.\footnote{\cite[Siehe][]{Hinske:PluralismusundPublikationsfreiheitimDenkenKants1986},
\cite{Hoeffe:EinerepublikanischeVernunft1996},
\cite{Recki:enquoteAnderStelle[je]desanderendenken2006} sowie \cite[][41,
56]{Wood:KantandtheProblemofHumanNature2003}, und
\cite[][325--328]{Pieper:EthikalsVerhaeltnisvonMoralphilosophieundAnthropologie1978},
und ausführlich
\cite{Keienburg:ImmanuelKantunddieOeffentlichkeitderVernunft2011}.} Es
gibt viele aussagekräftige Textbelege gegen diese Interpretation, von denen mir
eine Passage aus der Anthropologie am deutlichsten zu sein scheint. Die liberale oder
erweiterte Art des Denkens richtet sich nicht gegen den Selbstdenker überhaupt,
sondern gegen den \enquote{logische[n] Egoist[en]}, wie \name[Immanuel]{Kant} im
Anschluss an Georg
\authorcite{Meier:Vernunftlehre1752}\footnote{\cite[Vgl.][\S~202]{Meier:Vernunftlehre1752}
und \cite[][\S~170]{Meier:AuszugausderVernunftlehre1752} (\cite[][XVI:
413.29]{Kant:GesammelteWerke1900ff.}).
Nach \textcite[64]{Hinske:ZwischenAufklaerungundVernunftkritik1993} liegt der
Ursprung dieses Ausdrucks bei \authorcite{Meier:Vernunftlehre1752}.} denjenigen
nennt, der \enquote{es für unnötig [erachtet], sein Urteil auch am Verstande
anderer zu
prüfen}\footnote{\cite[][BA~6]{Kant:AnthropologieinpragmatischerHinsicht1977},
\cite[VII: 128.31--32]{Kant:GesammelteWerke1900ff.}. Zum Begriff des Egoismus in
der Philosophie des 18.\ Jahrhunderts siehe
\cite[][200--227]{Halbfass:DescartesFragenachderExistenzderWelt1968}. Einen
 Überblick über die Verwendung von \enquote{Egoismus} als Vorlage für
 \name[Immanuel]{Kant} liefert
 \cite[][15--46]{Heidemann:KantunddasProblemdesmetaphysischenIdealismus1998},
 der dabei jedoch ebenso wie \authorcite{Halbfass:DescartesFragenachderExistenzderWelt1968} \name[Immanuel]{Kant}s
 Vorstellungen von dogmatischem und problematischem Idealismus fokussiert und sich
weniger für den \emph{logischen} Egoismus interessiert.}. Dabei sei gerade aus
diesem Umstand die Freiheit öffentlichen Meinungsaustauschs so nötig, weil
Selbstdenken nur demjenigen möglich sei, der in Gemeinschaft mit anderen
denkt.\footnote{\cite[Vgl.][BA~6]{Kant:AnthropologieinpragmatischerHinsicht1977},
\cite[VII: 128.33--129.3]{Kant:GesammelteWerke1900ff.}: \enquote{Es ist aber so
gewiß, daß wir dieses Mittel, uns der Wahrheit unseres Urteils zu versichern,
nicht entbehren können, daß es vielleicht der wichtigste Grund ist, warum das
gelehrte Volk so dringend nach der \ori{Freiheit der Feder} schreit; weil, wenn
diese verweigert wird, uns zugleich ein großes \ori{Mittel} entzogen wird, die
Richtigkeit unserer eigenen Urteile zu prüfen, und wir dem Irrtum preis gegeben
werden.}} Die Möglichkeit der Mitteilung ist auch nach der \titel{Kritik der
reinen Vernunft} das entscheidende äußere Indiz der Wahrheit und Vernünftigkeit
einer Überzeugung, nämlich ob es auf einem objektiven Grund oder bloß auf der
subjektiven Beschaffenheit des Subjekts
beruht.\footnote{\cite[Vgl.][B 848\,f.,]{Kant:KritikderreinenVernunft2003}
\cite[][III: 532.3--16]{Kant:GesammelteWerke1900ff.}.} In der Aufklärungsschrift
erklärt er, dass der Ausgang aus selbst verschuldeter Unmündigkeit nur schwerlich den
Einzelnen, sondern nur einem Publikum möglich sei, dem ein Mindestmaß an Presse-
und Meinungsfreiheit gegeben
ist.\footnote{\cite[Vgl.][A~482--484]{Kant:BeantwortungderFrage:WasistAufklaerung?1977},
\cite[][VIII: 36.4--37]{Kant:GesammelteWerke1900ff.}.}


Den Ausdruck \enquote{logischer Egoismus} musste \name[Immanuel]{Kant} nicht neu
erfinden; er konnte auf eine breite Verwendungsweise in der Philosophie des 18.
Jahrhunderts zurückgreifen. Nach
\authorcite{Hinske:PluralismusundPublikationsfreiheitimDenkenKants1986} kam der
Begriff des \enquote{Egoismus} in der Bedeutung von \enquote{Solipsismus} durch
\authorcite{Wolff:Discursuspraeliminarisdephilosophiaingenere1996} in die
deutsche
Sprache.\footnote{\cite[Vgl.][39]{Hinske:PluralismusundPublikationsfreiheitimDenkenKants1986}.
In der Tat scheinen Ausdrücke \enquote{Egoismus} und \enquote{Solipsismus} ihre
Bedeutung seitdem vertauscht zu haben, insofern \enquote{Solipsismus} zunächst
in der Bedeutung von Selbstsucht und Selbstgefälligkeit verwendet wird
\parencite[vgl.][224--227]{Halbfass:DescartesFragenachderExistenzderWelt1968}.}
Dieser schreibt:
\begin{quote}
 Die Monisten sind abermahl von zweyerley Gattung, entweder \ori{Idealisten}
 oder \ori{Materialisten}. Jene geben blosse Geister oder auch solche Dinge zu,
 welche nicht aus Materie bestehen\punkt ; halten aber die Welt und die darinnen
 befindelichen Cörper für blosse Einbildungen \punkt\ und sehen sie nicht anders
 als einen regulirten Traum an. \punkt\ Endlich die Idealisten geben entweder
 mehr als ein Wesen zu, oder halten sich für das einige würckliche Wesen. Jene
 werden \ori{Pluralisten}; diese hingegen \ori{Egoisten}
 genennet.\footnote{\cite[][Vorrede zu der andern Auflage (nicht
 paginiert)]{Wolff:VernuenftigeGedankenvondenKraeftendesmenschlichenVerstandesundihremrichtigenGebraucheinErkenntnisderWahrheit1978}.}
\end{quote}
Nach der Position des Egoismus gibt es nur mich und nichts und niemanden sonst,
keine körperliche Welt, keine anderen \distanz{Geister}, weder die Leiber anderer Menschen, noch meinen
eigenen. Ich muss in der Lage sein, mein Denken und Erkennen
vollständig selbst zu generieren -- es ist in der Folge somit eine Vorstellung
von geistiger Autarkie, nicht nur von Autonomie vonnöten. Und dieses Moment der geistigen
Autarkie findet sich auch im Begriff des  des \emph{logischen} Egoisten, der
wohl auf Georg \authorcite{Meier:Vernunftlehre1752}
zurückgeht\footnote{\cite[Vgl.][64]{Hinske:ZwischenAufklaerungundVernunftkritik1993}.
Nach \authorcite{Hinske:PluralismusundPublikationsfreiheitimDenkenKants1986}
beginnt sich der Begriff des Egoismus genau an dieser Stelle entscheidend in
Richtung seiner heutigen Bedeutung zu entwickeln
\parencite[vgl.][39]{Hinske:PluralismusundPublikationsfreiheitimDenkenKants1986}.}.
\authorcite{Meier:Vernunftlehre1752} schreibt in der \titel{Vernunflehre}:
\begin{quote}
Die Egoisten glaubten, daß sie allein würklich wären, und sie führten sich
selbst beständig im Munde. Weil nun die logischen Egoisten allemal sich selbst zum
Grunde anführen, warum ist das oder das wahr? weil ichs sage; warum ist das oder
das falsch? weil ichs sage: so haben sie daher diesen Namen bekommen. Dieses
Vorurteil ist so unverschämt und pedantisch, daß es keiner Widerlegung bedarf,
und gleichwohl werden die meisten Gelehrten durch dieses Vorurteil
regieret.\footcite[][\S~202]{Meier:Vernunftlehre1752}
\end{quote}
\authorcite{Meier:Vernunftlehre1752} zählt \enquote{die \ori{logische Egoisterey}} zu den logischen
Vorurteilen, die in einem jeweils zu großen Zutrauen oder Misstrauen bestehen,
hier also in einem \enquote{gar zu grosse[n] Vertrauen, durch Hochmuth und
Eigenliebe verblendet, auf sich selbst und auf die Stärke seines Verstandes}%
\footnote{\Cite[][\S~202]{Meier:Vernunftlehre1752}.}. Das Bemühen um vernünftige
Selbständigkeit im Denken ist dabei immer die Suche nach einem Mittelweg
zwischen der blinden Autoritätshörigkeit und dem Aufgehen in einer
philosophischen \distanz{Schule} auf der einen und dem ebenso unvernünftigen
Vertrauen auf die eigenen Meinungen und das eigene Fürwahrhalten auf der anderen
Seite. Und so stehen die erste und die zweite Maxime des Verstandes in einem
Spannungsverhältnis.

Wir benötigen den Austausch mit anderen, um selbst denken zu
können. Ohne diesen Austausch fehlte uns mindestens die Selbstsicherheit in der
Ausübung der Vernunft. Darauf macht \name[Immanuel]{Kant} wiederholt aufmerksam. Und in
der Vorrede zur ersten Auflage der \titel{Kritik der reinen Vernunft} verweist
er auf die öffentliche Prüfung als Kriterium der Wissenschaftlichkeit von
Erkenntnissen: nur was einer öffentlichen Prüfung standhält, das ist zumindest
\emph{prima facie} vernünftig und vertrauenswürdig.\footnote{\cite[Vgl.][A
xi]{Kant:KritikderreinenVernunft2003}, \cite[][IV:
9.33--38]{Kant:GesammelteWerke1900ff.}: \enquote{Unser Zeitalter ist das
eigentliche Zeitalter der Kritik, der sich alles unterwerfen muss.
\ori{Religion} durch ihre \ori{Heiligkeit} und \ori{Gesetzgebung} durch ihre
\ori{Majestät} wollen sich gemeiniglich derselben entziehen. Aber alsdann
erregten sie gerechten Verdacht wider sich, und können auf unverstellte Achtung
nicht Anspruch machen, die die Vernunft nur demjenigen bewilligt, was ihre freie
und öffentliche Prüfung hat aushalten können.}}
Der logische Egoist in \name[Immanuel]{Kant}s Darstellung glaubt, er könne die Richtigkeit
und Qualität seines Denkens und Erkennens selbst kontrollieren. Urteilen ist
jedoch eine Tätigkeit, die aus der gemeinsamen Praxis unter gegenseitiger
Kontrolle hervorgeht und erst danach zurückgezogen und privat betrieben werden
kann. Wobei wir aber doch die prinzipielle Möglichkeit haben müssen, unsere
Urteile an denen anderer zu vergleichen. Dies trifft sogar -- wie
\name[Immanuel]{Kant} explizit
betont\footnote{\cite[Vgl.][\S~2]{Kant:AnthropologieinpragmatischerHinsicht1977},
\cite[][VII: 129.3--8]{Kant:GesammelteWerke1900ff.}.} -- auch und gerade auf die
Mathematik zu, die wegen ihrer methodischen Strenge und Exaktheit ein Absehen
von intersubjektiver Kontrolle bei geübten Menschen in konkreten Einzelfällen
noch am ehesten erlaubt und daher die Illusion erzeugt, sie sei von Natur aus privat
zu betreiben. Eine solche Illusion verkennt nicht nur die Mittel, die zu
vernünftigen Urteilen führen, sondern das Wesen der Vernunft selbst. Denn
Vernunft kann es nur geben, wenn es eine gemeinsam kontrollierte Praxis des
Bewertens gibt.\footnote{\cite[Vgl.][B
766f.,]{Kant:KritikderreinenVernunft2003} \cite[][III:
484.10--14]{Kant:GesammelteWerke1900ff.}: \enquote{Auf dieser Freiheit beruht
sogar die Existenz der Vernunft, die kein diktatorisches Ansehen hat, sondern
deren Ausspruch jederzeit nichts als die Einstimmung freier Bürger ist, deren
jeglicher seine Bedenklichkeiten, ja sogar sein veto, ohne Zurückhaltung muß
äußern können.}} Möglicherweise ließe sich sogar
eine weitaus stärkere These vertreten:
Zumindest auf den ersten Blick klingt es nach einer provokanten Behauptung, wenn
\name[Immanuel]{Kant} in der \titel{Kritik der reinen Vernunft} über die Meinungs- und
Publikationsfreiheit schreibt:
\begin{quote}
  Auf dieser Freiheit beruht sogar die \myemph{Existenz} der Vernunft, die kein
  diktatorisches Ansehen hat, sondern deren Ausspruch jederzeit nichts als die
  Einstimmung freier Bürger ist, deren jeglicher seine Bedenklichkeit, ja sogar
  sein veto, ohne Zurückhalten muß äußern
  können.\footnote{\cite[][B~766\,f.,]{Kant:KritikderreinenVernunft2003}
  \cite[III: 484.10--14]{Kant:GesammelteWerke1900ff.}, \myherv .}
\end{quote}
Dieser Textpassage zufolge geht es nicht nur darum, dass wir durch den Austausch
mit anderen in unserem Urteilen und Schließen selbstsicherer werden und \emph{besser} zu denken
lernen. Die Vernunft selbst ist grundsätzlich kommunikativ
verfasst.\footnote{\cite[Vgl.][120]{Recki:enquoteAnderStelle[je]desanderendenken2006}:
\enquote{Wir dürfen daraufhin behaupten, daß die Vernunft, auch wenn Kant nicht
ausdrücklich sagt, sie vollziehe sich im stetigen Austausch und Abgleich mit
anderen Vernünften, intern kommunikativ, ja: dialogisch konzipiert ist.} Mir
hingegen scheint \name[Immanuel]{Kant} dies an der zitierten Stelle sehr ausdrücklich zu
sagen.}

\name[Immanuel]{Kant} fordert nicht nur eine in gewisser Hinsicht offene,
nämlich den freien Meinungsaustausch zulassende bürgerliche Gesellschaft; er
fordert auch von jedem Einzelnen eine \enquote{pluralistische} Haltung,
\enquote{d.\,i. die Denkungsart: sich nicht als die ganze Welt in seinem Selbst
befassend, sondern als einen bloßen Weltbürger zu betrachten und zu
verhalten.}\footnote{\cite[BA 8]{Kant:AnthropologieinpragmatischerHinsicht1977},
\cite[VII: 130.12--14]{Kant:GesammelteWerke1900ff.}.} \singlequote{Pluralismus}
meint bei \name[Immanuel]{Kant} jedoch nicht das Lob von
Meinungs\emph{vielfalt}, sondern das Bemühen, in freier Kommunikation zu
\emph{einer gemeinsamen} Überzeugung zu gelangen.
\begin{comment}
Unser \distanz{moderner} Pluralismus mit seiner Forderung nach
Toleranz\footnote{\name[Immanuel]{Kant} nennt den Namen der Toleranz
\enquote{hochmütig}
\mkbibparens{\cite[][A 491]{Kant:BeantwortungderFrage:WasistAufklaerung?1977},
\cite[][VIII: 40.30]{Kant:GesammelteWerke1900ff.}}. Siehe dazu
\cite{Weidemann:VonenquotebisweilenunvermeidlicherGeringschaetzung2010}.} und
Offenheit gründet in der Annahme, dass entsprechende Urteile -- bezüglich des
\distanz{richtigen} Lebensstils, der Religion etc.\ -- keine epistemisch
fundierte Antwort erlauben, sei es, weil sie nicht wahrheitsfähig sind
(\enquote{de gustibus non est disputandum}), sei es, weil wir die Wahrheit nicht
ausmachen (Gott nicht erkennen) können. Es gehört zu unserem modernen,
aufgeklärten Selbstverständnis, dass die Wahl des religiösen Bekenntnisses und
die Ansichten über den \distanz{Sinn des Lebens} und die \distanz{Bestimmung des
Menschen} dem subjektiven Belieben anheim gestellt sind. Religiöse Toleranz,
sexuelle Selbstbestimmung und Gleichwertigkeit unterschiedlicher Lebensentwürfe
gehören zu dem, was wir ganz selbstverständlich als Erbe der Aufklärung und
Errungenschaften der Moderne ansehen. Dieser Begriff von Pluralismus gründet in
einer (heute vielleicht dominanten) Konzeption von Autonomie, die diese als
wesentlich individualistisch und egozentrisch begreift. Diese Konzeption trennt
Autonomie und Vernunft, insofern Autonomie darin besteht, unsere je eigenen
Präferenzen zu verfolgen, und der Vernunft die nachrangige Aufgabe zukommt, das
Verfolgen dieser Präferenzen effizient zu gestalten und mit den Präferenzen
anderer zu koordinieren.\footnote{Siehe dazu auch
\cite[][\pno~216\,f.]{ONeill:AufgeklaerteVernunft1996}.} Ein Dialog als
Grundlage unserer Vernunft und vernünftiger Einsichten und Entscheidungen kann
sich in \name[Immanuel]{Kant}s Konzeption hingegen nicht mit dem Anspruch
begnügen, einen \emph{modus vivendi} zur friedlichen Koexistenz verschiedener Gruppen auszuhandeln. Stattdessen erweist
sich die Vernünftigkeit der beteiligten Überzeugungen selbst erst im
Versuch ihrer Mitteilung, also der Überzeugung anderer. Auch die von uns
verfolgten Ziele sind dabei Gegenstand vernünftiger Erwägung und damit der
Mitteilbarkeit zugänglich.\footnote{Ich werde später zeigen, inwiefern
\name[Immanuel]{Kant} dennoch der Tatsache Rechnung trägt, dass es eine
Vielfalt an unterschiedlichen Präferenzen gibt, die je individuell zu verfolgen
ebenso zu unseren vernünftigen Rechten gehört. Das schließt nicht aus, dass wir
sie als vernünftig oder unvernünftig bewerten können und dass dies auch auf der
Möglichkeit der Kommunikation mit anderen beruht. Siehe dazu die Überlegungen
zur \enquote{Klugheit} in Kapitel \ref{subsection:aufklaerungundpraxis}.}
\end{comment}
Im Gegensatz zu neueren um Forderungen nach
Toleranz\footnote{\name[Immanuel]{Kant} nennt den Namen der Toleranz
\enquote{hochmütig}
\mkbibparens{\cite[][A 491]{Kant:BeantwortungderFrage:WasistAufklaerung?1977},
\cite[][VIII: 40.30]{Kant:GesammelteWerke1900ff.}}. Siehe dazu
\cite{Weidemann:VonenquotebisweilenunvermeidlicherGeringschaetzung2010}.} und
Offenheit zentrierten Pluralismusbegriffen wie
\authorcite{Rawls:TheLawofPeoples1999}' \enquote{\emph{reasonable pluralism}} meint Pluralismus bei
\name[Immanuel]{Kant} nicht die Anerkennung unüberwindlicher Differenzen
hinsichtlich religiöser und philosophischer Überzeugungen oder -- wie
\authorcite{Rawls:TheLawofPeoples1999} sagt -- \enquote{comprehensive
doctrines}.\footnote{\cite[Vgl.][136]{Rawls:TheLawofPeoples1999}: \enquote{The
fact of reasonable pluralism {\punkt} means that the differences between
citizens arising from their comprehensive doctrines, religious and nonreligious,
may be irreconcilable.} \cite[Siehe auch][\pno~11, 12, 31, 124,
131\,f.]{Rawls:TheLawofPeoples1999}.
\authorcite{Rawls:TheLawofPeoples1999} wertet den vernünftigen Pluralismus als zentralen
Bestandteil liberaler Demokratien; \cite[vgl.][124]{Rawls:TheLawofPeoples1999}: \enquote{A
basic feature of liberal democracy is the fact of reasonable pluralism---the
fact that a plurality of conflicting reasonable comprehensive doctrines, both
religious and nonreligious (or secular), is the normal result of the culture of
its free institutions.}} \name[Immanuel]{Kant}s Pluralismuskonzeption beschreibt
anderes als der \emph{reasonable pluralism} des 20. Jahrhunderts; er bezieht sich direkt auf die
erweiterte Denkungsart der zweiten Maxime des gemeinen Menschenverstandes. Diese
erweiterte Denkungsart nennt \name[Immanuel]{Kant} \enquote{Pluralismus} und setzt sie dem \enquote{logischen Egoismus}
entgegen.\footnote{\cite[Vgl.][\S~2]{Kant:AnthropologieinpragmatischerHinsicht1977},
\cite[][VII: 128.21--130.21]{Kant:GesammelteWerke1900ff.}. Zum Begriff des
Egoismus in der Philosophie des 18.\ Jahrhunderts siehe
\cite[][200--227]{Halbfass:DescartesFragenachderExistenzderWelt1968}. Einen
 Überblick über die Verwendung von \enquote{Egoismus} als Vorlage für
 \name[Immanuel]{Kant} liefert
 \textcite[vgl.][15--46]{Heidemann:KantunddasProblemdesmetaphysischenIdealismus1998},
 der dabei jedoch ebenso wie \authorcite{Halbfass:DescartesFragenachderExistenzderWelt1968} \name[Immanuel]{Kant}s
 Vorstellungen von dogmatischem und problematischem Idealismus -- den
 sogenannten \emph{metaphysischen} Egoismus -- fokussiert und sich weniger für
 den \emph{logischen} Egoismus interessiert.}

\subsection{Aufklärung der
Urteilskraft}\label{subsection:AufklaerungderUrteilskraft}
Dass es keinen einfachen Algorithmus gibt, der vernünftige
Prinzipien von Vorurteilen unterscheidet, ist vielleicht einfach eine Tatsache,
mit der wir immer umgehen müssen. Sie macht es so schwer, sich gänzlich von
Vorurteilen zu befreien, und verhindert, dass aus dem Zeitalter der Aufklärung
ein aufgeklärtes Zeitalter wird.\footnote{\authorfullcite{ONeill:AufgeklaerteVernunft1996} schlägt vor,
\name[Immanuel]{Kant} so zu lesen, dass zwar die Prinzipien der Vernunft nicht
sozial konstruiert sind, sie aber so verfasst sein müssen, dass sie die
\enquote{Bedingungen für die \ori{Konstruktion} von Intersubjektivität
erfüllen} \parencite[][213]{ONeill:AufgeklaerteVernunft1996}. Insofern sind sie
nicht davon abhängig, welche Überzeugungen und Prinzipien faktisch Zustimmung
finden; sie seien aber daran gebunden, Kommunikation möglich zu
machen. \cite[Vgl.][219]{ONeill:AufgeklaerteVernunft1996}:
\enquote{Kant gründet Vernunft nicht auf tatsächlichen Konsens oder die
Übereinstimmung und die Standards irgendeiner historischen Gemeinschaft; er
gründet sie auf die Zurückweisung aller Prinzipien, die die Möglichkeit von
strukturell unbegrenzt offener Kommunikation und Interaktion ausschließen.}
Auch dies stelle freilich nur eine Einschränkung der Menge möglicher Grundsätze
dar, ohne uns auf bestimmte Grundsätze festzulegen.}
Es gibt keinen Algorithmus, kein einfach anzuwendendes
Entscheidungsverfahren zur Vermeidung von Vorurteilen. Es bedarf zum
Selbstdenken nicht einfach einer an der Mathematik orientierten
Methodik\footnote{Wie \name[Immanuel]{Kant} nachdrücklich hervorhebt, ist die
Mathematik und ihre Methodik selbst in der \emph{gemeinsamen} Vernunftausübung
gegründet (\cite[vgl.][BA~6]{Kant:AnthropologieinpragmatischerHinsicht1977},
\cite[VII: 129.3--8]{Kant:GesammelteWerke1900ff.}).}, sondern einer gereiften
Urteilskraft,\footnote{In diese Richtung geht auch die Interpretation in
\cite{Enskat:BedingungenderAufklaerung2008}.} die sich naturgemäß nicht mit ein
paar Regeln hinreichend beschreiben lässt, sondern im \emph{gemeinsamen} Umgang
mit unseren Erkenntnissen (mit Aussagen und Begriffen) eingeübt und ausgebildet
werden muss. Deswegen ist Aufklärung nicht einzelnen Denkern, sondern nur einem
Publikum möglich. Somit unterscheidet sich \name[Immanuel]{Kant} von
\authorcite{Wolff:Psychologiaempirica1968} zwar nicht hinsichtlich der Ansicht,
dass freies Denken ein kompetentes Denken ist, wohl aber hinsichtlich des
Erwerbs der Kompetenzen: Während \authorcite{Wolff:Psychologiaempirica1968} den
Mathematikunterricht in seiner Methodik zum Vorbild nehmen kann und muss, steht
aus \name[Immanuel]{Kant}s Sicht die gemeinsame Ausbildung der Urteilskraft
Pate.\phantomsection\label{Abschnitt:KantunddieOeffentlichkeitderVernunft-Ende}

Die aufgeklärte \emph{Urteilskraft} ist es dann auch, die zur Krise der
Metaphysik führt, und zwar gerade weil das Kriterium der freien Einstimmung zu
dem Schluss führt, dass Metaphysik auf keinem vernünftigen Fundament erbaut sein
kann. Die der Vernunftkritik vorgängige Gleichgültigkeit in Fragen der
Metaphysik sei \enquote{offenbar die Wirkung nicht des Leichtsinns, sondern der
gereiften \ori{Urteilskraft} des Zeitalters, welches sich nicht länger durch
Scheinwissen hinhalten läßt}\footnote{\cite[][A
xi]{Kant:KritikderreinenVernunft2003}, \cite[][IV:
9.2--4]{Kant:GesammelteWerke1900ff.}.}. Die Vorreden zur \titel{Kritik der
reinen Vernunft} weisen die Notwendigkeit einer Vernunftkritik als
Anwendungsfall des Pluralismus aus, denn nur auf Grundlage der Maxime der
erweiterten Denkungsart lässt sich von der Tatsache fehlender Einhelligkeit
unter Metaphysikern auf die Unwissenschaftlichkeit der Disziplin schließen.

Dass Wissen mitteilbar ist, ist ihm nicht äußerlich. Es ist keine Forderung, die
wir von außen als plausibel oder für unsere epistemische Praxis notwendig dem
Wissen erst nachträglich abverlangen. Möglicherweise sagt sie uns sogar
etwas darüber, was Wissen ist; sie formuliert eine Beschränkung dessen, was wir
als Wissen akzeptieren können. Die Mitteilbarkeit ist als Prüfstein für die
Wahrheit eines Urteils aber auch bedeutsam für die (aus Sicht der Aufklärung
enorm wichtige) Unterscheidung von Überzeugung und Überredung.\footnote{\enquote{Der Probierstein des Fürwahrhaltens, ob es Überzeugung oder bloße Überredung
  sei, ist also äußerlich die Möglichkeit, dasselbe mitzuteilen und das
  Fürwahrhalten für jedes Menschen Vernunft gültig zu befinden; denn alsdenn ist
  wenigstens eine Vermutung, der Grund der Einstimmung aller Urteile, unerachtet
  der Verschiedenheit der Subjekte untereinander, werde auf dem
  gemeinschaftlichen Grunde, nämlich dem Objekte beruhen, mit welchem sie daher
  alle zusammenstimmen und dadurch die Wahrheit des Urteils beweisen
  werden} \mkbibparens{\cite[][B 488\,f.,]{Kant:KritikderreinenVernunft2003}
  \cite[][III: 532.9--16]{Kant:GesammelteWerke1900ff.}}.}
Überzeugungen machen die aktive, aufgeklärte Art des
Überzeugungserwerbs aus, Überredungen die passive und unmündige Art, etwas von
anderen zu übernehmen. Ich werde hierauf in Kapitel
\ref{section:KantsEthicsofBelief} eingehen.

Wir verfügen über Wissen, wenn unser Urteil mit dem Objekt, welches wir
beurteilen, übereinstimmt. Dies ist zunächst die \enquote{Namenerklärung der
Wahrheit}\footnote{\cite[][B 82]{Kant:KritikderreinenVernunft2003}, \cite[][III:
79.9]{Kant:GesammelteWerke1900ff.}.}. Aber sie verschafft uns kein einheitliches
Verfahren, die Wahrheit eines Urteil zu beurteilen. Die einzige sinnvolle
Prüfung, der wir unsere Urteile unterwerfen können, ist der Vergleich mit den
Urteilen anderer.\footnote{\enquote{Erkenntnisse und Urteile müssen sich, samt
der Überzeugung, die sie
  begleitet, allgemein mitteilen lassen; denn sonst käme ihnen keine
  Übereinstimmung mit dem Objekt zu; sie wären insgesamt ein bloß subjektives
  Spiel der Vorstellungskräfte, gerade so wie es der Skeptizism verlangt}
  \mkbibparens{\cite[][\S~21]{Kant:KritikderUrteilskraft2009}, \cite[][V:
  238.19--23]{Kant:GesammelteWerke1900ff.}.}.} Wenn wir Wissen zu haben
  beanspruchen, dann sollte es möglich sein, dieses Wissen mitzuteilen. Wissen
  zu haben ist eine genuin soziale Angelegenheit. Um
Wissen kann es sich daher nur bei \emph{mitteilbaren} Erkenntnissen handeln;
wenn wir etwas wissen, so ist es möglich, dass jemand anderes dieses Wissen
erlangt, indem wir es ihm mitteilen.

Eine weitere Bedeutung der pluralistischen Denkweise findet sich in
\name[Immanuel]{Kant}s Anmerkungen in seinem Handexemplar von
\authorcite{Baumgarten:Metaphysica---Metaphysik2011}s \titel{Metaphysica}:
\begin{quote}
 Allein in ansehung des bescheidnen Urtheils über den Werth seiner eignen
 wissenschaft und der Mäßigung des Eigendünkels und \ori{egoismus}, den eine
 Wissenschaft giebt, wenn sie allein im Menschen residirt, ist etwas nöthig, was
 dem gelehrten humanitaet gebe, damit er nicht sich selbst verkenne und seinen
 Kräften zu viel Zutraue.\\ Ich nenne einen solchen Gelehrten einen Cyclopen. Er
 ist ein egoist der Wissenschaft, und es ist ihm noch ein Auge nöthig, welches
 macht, daß er seinen Gegenstand noch aus dem Gesichtspunkte anderer Menschen
 ansieht. Hierauf gründet sich die humanitaet der Wissenschaften, d.\,i.\ die
 Leutseeligkeit des Urtheils, dadurch man es andrer Urtheil mit unterwirft, zu
 geben. \punkt\ Nicht die Stärke, sondern das einäugigte macht hier den Cyclop.
 Es ist auch nicht gnug, viel andre Wissenschaften zu wissen, sondern die
 Selbsterkenntnis des Verstandes und der Vernunft. \ori{Anthropologia
 transscendentalis}.\footnote{\cite[][\nopp 903]{Kant:Reflexionen1900ff.},
 \cite[][XV: 394.23--395.8,395.29--32]{Kant:GesammelteWerke1900ff.}. Nach
 \name[Erich]{Adickes} stammt dieser Eintrag aus den Jahren 1776-1778.}
\end{quote}
Es geht an dieser Stelle nicht um die Gewissheit und Verlässlichkeit einer
Erkenntnis, sondern um die korrekte Bewertung ihrer Bedeutung. Über eine reife
Urteilskraft verfügt, wer weiß, worauf es ankommt, wie \name[Immanuel]{Kant} an
anderer Stelle notiert.\footnote{\enquote{Die obere Erkentniskrafte. 1.) Er
[versteht] weiß, was er will; 2) weiß, worauf es ankommt; 3. er sieht ein,
worauf es hinausläuft. 1.) richtiger Verstand. 2) reife Urtheilskraft. 3.
geläuterte Vernunft} \mkbibparens{\cite[][\nopp 455]{Kant:Reflexionen1900ff.},
\cite[][XV: 188.6--8]{Kant:GesammelteWerke1900ff.}}.}
Wir neigen dazu, die Erkenntnisse derjenigen Disziplin, an deren Förderung wir
jeweils selbst mitwirken, für bedeutsamer zu halten, als sie es tatsächlich aus
einer allgemeinen Perspektive sind. Dies ist einerseits verständlich und fast
unvermeidlich, andererseits aber auch ganz einfach zu korrigieren: Wir müssen
lediglich die von unserer Einschätzung unterschiedene Bewertung durch andere
ernst nehmen. Dann können wir die Fehleinschätzung bereits ohne tiefgehende
Auseinandersetzung mit dem Wert wissenschaftlicher Forschung vermeiden. Der
Gelehrte dürfe dazu auch nicht zu eitel sein, sein Urteil auch an dem Urteil von
Menschen zu prüfen, die nicht seinem Stand angehören. Der Laie habe nicht
\emph{per se} ein geringeres Vermögen, die Relevanz wissenschaftlicher
Erkenntnisse zu bewerten. Er hat sogar einen Vorteil, und zwar den der
Unparteilichkeit.\footnote{\cite[Vgl.][A 66]{Kant:ImmanuelKantsLogik1977},
\cite[][IX: 48.13--16]{Kant:GesammelteWerke1900ff.}: \enquote{Die Schule hat
ihre Vorurteile so wie der gemeine Verstand. Eines verbessert hier das andre. Es ist daher
wichtig, ein Erkenntnis an Menschen zu prüfen, deren Verstand an keiner Schule
hängt.}}

Das Unternehmen einer \emph{anthropologia transscendentalis} als einer
Selbsterkenntnis des Verstandes und der Vernunft kann einerseits auf solche
Überlegungen, andererseits natürlich auch auf das Unternehmen einer Kritik der
reinen Vernunft verweisen. Letztlich fallen beide Sichtweisen zusammen, wenn man
die Vernunftkritik vor eben diesem Hintergrund eines
\singlequote{Weltbegriffs}\footnote{Siehe hierzu
Kap.~\ref{chapter:AufklaerungundWissenschaft} dieser Arbeit.} der Philosophie
liest, dem es um die Humanität der Wissenschaften geht.
In einer Erläuterung der zitierten Stelle jedenfalls klingt der Verweis auf die
Vernunftkritik deutlich an:
\begin{quote}
 Das zweyte Auge ist also das der Selbsterkentnis der Menschlichen Vernunft,
 ohne welches wir kein Augenmaas der Größe unserer Erkentnis haben. \punkt\ Der
 \ori{egoismus} rührt daher, weil sie den Gebrauch, welchen sie von der Vernunft
 in ihrer Wissenschaft machen, weiter ausdehnen und auch in anderen Feldern vor
 hinreichend halten.\footnote{\cite[][\nopp 903]{Kant:Reflexionen1900ff.},
 \cite[][XV: 395.17--19,24--27]{Kant:GesammelteWerke1900ff.}.}
\end{quote}
Wer ernsthaft auf die Einschätzungen anderer hört, der stelle fest, dass mit
aller Wissenschaft die wahren Fragen aus der Perspektive der Humanität noch
nicht beantwortet sind. Dies scheint mir das Ergebnis zu sein, das sich nach
\name[Immanuel]{Kant} einstellt, wenn der Metaphysiker auf das Urteil anderer --
in diesem Fall der Indifferentisten -- achtet.

\section{Zusammenfassung}

\name[Immanuel]{Kant}s Aufklärungsprogramm mit seiner Forderung, sich aus
selbst verschuldeter Unmündigkeit zu befreien, betont die intellektuelle Freiheit
und Selbständigkeit im Gebrauch des oberen Erkenntnisvermögens. Selbstdenken
bedeutet dabei den je eigenen Gebrauch der Vernunft und ist daher an den Erwerb
intellektueller \emph{Kompetenzen} gebunden, wie
\authorcite{Wolff:Discursuspraeliminarisdephilosophiaingenere1996} unermüdlich
hervorhob. Wer diese Kompetenzen nicht hat, der wird in seinem Urteil keine
andere Chance haben, als sich fremden Autoritäten zu unterwerfen, denen er ein
kompetentes Urteil zutraut. Wer hingegen selbst sicher Urteilen kann, wird auch
seine Vernunft selbst gebrauchen. Während
\authorcite{Wolff:Discursuspraeliminarisdephilosophiaingenere1996} jedoch die
Beherrschung einer bestimmten Methodik in das Zentrum seiner Überlegungen schob,
verwirft \name[Immanuel]{Kant} den Gedanken, es gebe eine solche einheitliche
Methode der Vernunft und verweist stattdessen auf den allgemeinen Standpunkt
einer \singlequote{erweiterten Denkungsart} und die damit verbundene
intersubjektive Kontrolle des je eigenen Denkens.


Aufklärung ist in weiten Teilen eine Frage der Ausbildung der Urteilskraft
im Austausch mit anderen.
Daher ist intellektuelle Selbständigkeit nicht als Autarkie misszuverstehen, da
vernünftige Freiheit nur in Gemeinschaft möglich ist, für die uns
\name[Immanuel]{Kant} selbst Metaphern aus dem Bereich der Politik anbietet,
wenn er vom Ausspruch der Vernunft als der \enquote{Einstimmung freier
Bürger}\footnote{\cite[][B 766]{Kant:KritikderreinenVernunft2003}, \cite[][III:
484.12]{Kant:GesammelteWerke1900ff.}.} spricht. Sie lässt sich mit Adjektiven
wie \enquote{republikanisch} oder \enquote{demokratisch}
umschreiben,\footnote{\cite[Vgl.][]{Hoeffe:EinerepublikanischeVernunft1996}.}
wenngleich dies die Gefahr von Missverständnissen fördern kann, wenn man die
Metaphern allzu wörtlich nimmt. Der Selbstdenker ist nicht derjenige, der sich
um die Urteile seiner Mitmenschen nicht schert, sondern derjenige, der sich
selbst als gleichberechtigtes Mitglied einer Kommunikationsgemeinschaft ansieht
und verhält. Das wiederum heißt nach der Maxime des Selbstdenkens, wir sollen in
unserem Denken und Erkennen nicht passiv, sondern aktiv sein. Statt Erkenntnisse
von außen aufzunehmen, sollen wir sie selbst generieren. In welchem Umfange wir
dies können, wird nun im \ref{chapter:endlichkeitmenschlichendenkens}. Kapitel
zu untersuchen sein.



