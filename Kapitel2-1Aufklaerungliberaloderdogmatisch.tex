\section{Aufklärung und Wissenschaft}\label{Abschnitt:EpistemischeArbeitsteilung}
Aufklärung ist eng mit der Entwicklung der modernen Wissenschaften verbunden.
Zwischen diesen Wissenschaften und ihren Organisationsformen auf der einen und der
Forderung der Aufklärung nach Mündigkeit und epistemischer Selbstbestimmung auf
der anderen Seite besteht jedoch ein kompliziertes Spannungsverhältnis. Die
unüberschaubar große Masse an wissenschaftlichen Erkenntnissen und die große
Bandbreite an universitären Disziplinen mit ihrer Vielfalt an Methoden und
zugehörigen Kompetenzen erlaubt es nicht, Experte auf jedem Gebiet zu sein.
Ebenso wenig können wir auf diese Bandbreite unseres Wissens verzichten
und uns auf das Wissen beschränken, welches uns ohne Rückgriff auf das
Wissen anderer zugänglich ist.\footnote{In der \titel{Idee zu einer
allgemeinen Geschichte in weltbürgerlicher Absicht} betont
\name[Immanuel]{Kant}, die Vernunft wirke \enquote{selbst nicht instinktmäßig,
sondern} bedürfe \enquote{Versuche, Übung und Unterricht, um von einer Stufe der
Einsicht zur andern allmählich fortzuschreiten.} Sie bedürfe gar \enquote{einer
vielleicht unabsehlichen Reihe von Zeugungen, deren eine der andern ihre
Aufklärung überliefert} \mkbibparens{\cite[][A
389]{Kant:IdeezueinerallgemeinenGeschichteinweltbuergerlicherAbsicht1977},
\cite[][VIII: 19.7--8]{Kant:GesammelteWerke1900ff.}}.} Dass wir auf unsere
arbeitsteilige Wissensgesellschaft gerade auch in unserem Bemühen um Aufklärung und Mündigkeit
angewiesen sind, zeigt sich ja darin, dass Aufklärung mit der Entwicklung der
modernen, kooperativen und arbeitsteiligen Wissenschaften entstand und durch
diese vorangetrieben wurde.


Wissenschaft ist nach \name[Immanuel]{Kant} ein Generationen übergreifendes
Gemeinschaftsunternehmen, das ein Einzelner in Zusammenarbeit mit anderen immer nur ein kleines Stück voranbringt,
nachdem er aufgreift und verinnerlicht, was andere vor ihm
bewerkstelligten.\footnote{\cite[Vgl.][A
388\,f.,]{Kant:IdeezueinerallgemeinenGeschichteinweltbuergerlicherAbsicht1977}
\cite[VIII: 18.29--19.16]{Kant:GesammelteWerke1900ff.}, sowie \cite[A
322\,f.,]{Kant:AnthropologieinpragmatischerHinsicht1977} \cite[VII:
325.30--326.9]{Kant:GesammelteWerke1900ff.}.} Er
beschreibt die Wissenschaften in Analogie zu Gewerbe und Handwerk
\emph{expressis verbis} als
arbeitsteilig.\footnote{\phantomsection\label{Fussnote:EpistemischeArbeitsteilungGMS}\cite[Vgl.][BA
v--vii]{Kant:GrundlegungzurMetaphysikderSitten1965}, \cite[][IV:
388.15--389.4]{Kant:GesammelteWerke1900ff.}.
\cite[Vgl.][67]{Brandt:UniversitaetzwischenSelbst-undFremdbestimmung2003}.} Der
Versuch, diese Aufteilung in Disziplinen mit ihren jeweiligen Experten
rückgängig zu machen, könne nur Stümperei (\name[Immanuel]{Kant} sagt: \enquote{Barbarei})
hervorbringen.\footnote{\cite[Vgl.][BA
vi]{Kant:GrundlegungzurMetaphysikderSitten1965}, \cite[][IV:
388.19--21]{Kant:GesammelteWerke1900ff.}: \enquote{Wo die Arbeiten so nicht
unterschieden und verteilt werden, wo jeder ein Tausendkünstler ist, da liegen
die Gewerbe noch in der größten Barbarei.}} Schließlich ermöglicht erst die
Aufteilung der epistemischen Arbeit in sich gegenseitig ergänzende Teilschritte
die Entstehung entsprechender Spezialisten, die für die nötige Qualität der
einzelnen Arbeitsschritte sorgen können. Eine solche
arbeitsteilige Wissenschaft und mit ihr die Qualität und Quantität ihrer
Ergebnisse gibt es aber nur durch das Vertrauen auf das Wissen von
Experten auf verschiedenen Gebieten. Ein naives Verständnis der Mündigkeits- und
Selbständigkeitsforderung, wie sie sich in dem Bemühen zeigte, \emph{jede}
Behauptung nur dann zu akzeptieren, wenn wir ihre Wahrheit unabhängig von
anderen kontrollieren können, brächte uns um die Früchte der modernen
Wissenschaften. Es kann also nicht um die \emph{Vermeidung} epistemischer
Abhängigkeiten gehen, sondern nur um deren \emph{Vereinbarung} mit der Forderung
nach epistemischer Selbständigkeit: Es gilt die Frage aufzuwerfen, wie
\enquote{Autonomie angesichts epistemischer Abhängigkeiten}\footnote{So der
Titel von \cite{Scholz:AutonomieangesichtsepistemischerAbhaengigkeiten2001}.} aussehen
kann. \name[Immanuel]{Kant} sagt nicht, sich seiner eigenen Vernunft zu bedienen heiße, alles
selbst erkennen zu wollen und sich generell nicht auf das Wissen anderer zu
berufen. Wir müssen, sollen und wollen Erkenntnisse auch von Anderen übernehmen,
aber wir müssen uns dabei mündig verhalten.

Die genannten Spannungen hinter dem Streben nach Autonomie und Authentizität im
Denken betreffen ausschließlich die Bestrebungen, die ich vorhin als
\emph{liberale} Aufklärung bezeichnete,\footnote{Siehe Kapitel
\ref{Benennung:LiberaleAufklaerung},
S.~\pageref{Benennung:LiberaleAufklaerung}.} nicht aber eine
\distanz{szientistische} Aufklärung, die die Ersetzung eines veralteten
(religiösen oder metaphysischen) Weltbildes durch ein neues, wissenschaftlich
fundiertes Weltbild anstrebt. Die szientistische Aufklärung fordert
Wissenschaftlichkeit und Objektivität, nicht Autonomie und Authentizität. Kritik
und Selbstdenken gelten ihr zwar als Forschungsmaximen \emph{innerhalb} der
Wissenschaften, die beispielsweise in der Forderung nach Überprüfbarkeit der
Ergebnisse zum Ausdruck kommen. Sie sollen aber nicht das Verhältnis des Laien
zu den Wissenschaften bestimmen. Und dies scheint sinnvoll und geboten, gerade
weil -- wie vorhin unter Rekurs auf
\authorfullcite{Wolff:Discursuspraeliminarisdephilosophiaingenere1996} betont
wurde -- Selbstdenken und Mündigkeit Kompetenzen voraussetzt, die der Laie
\emph{per definitionem} nicht besitzt. Weil liberale Aufklärung ganz wesentlich
darin besteht, das einzelne Subjekt zu Mündigkeit und Selbständigkeit, zu
Unabhängigkeit von Autoritäten und Traditionen zu erziehen, gibt es einen
Konflikt, der sich auch als Konflikt zwischen Aufklärung und Wissenschaft
beschreiben lässt. In offener Konfrontation gegen Wissenschaftlichkeit und
Objektivität wird Aufklärung freilich nicht bestehen können. Auch \name[Immanuel]{Kant}
verteidigt die Vorrangstellung der Wissenschaft gegenüber dem, was man einen
\enquote{gesunden Menschenverstand} nennen könnte.\footnote{Siehe oben, Kap.
\ref{subsection:SelbstdenkenbeiKant} dieser Arbeit.} Und deswegen ist liberale Aufklärung nur
dann eine ernstzunehmende Option, wenn es gelingt, zu einer Vermittlung von Aufklärung und Wissenschaft zu
gelangen.\footnote{\cite[Vgl.][837]{Schnaedelbach:WirKantianer2005}:
\enquote{Aufklärung durch Wissenschaft allein verfehlt ihr Ziel und endet
notwendig im Dogmatismus, wenn sich die Wissenschaft nicht über
sich selbst aufklärt; umgekehrt bekommt die Aufklärung, wenn sie sich von der
Wissenschaft fernzuhalten sucht, keinen Boden unter die Füße und verheddert
sich unvermeidlich in ihren skeptischen Argumenten.}}

Möglicherweise soll die Vernunftkritik den Weg weisen, Aufklärung und Wissenschaft in ein
zuträgliches Verhältnis zueinander zu setzen. Dies ist der Vorschlag
\authorfullcite{Schnaedelbach:WirKantianer2005}s, der auf den in der \titel{Geschichte der
reinen Vernunft} als einzigen Ausweg genannten kritischen
Weg\footnote{\cite[Vgl.][B~884]{Kant:KritikderreinenVernunft2003},
\cite[III: 552.19]{Kant:GesammelteWerke1900ff.}.} verweist, den \name[Immanuel]{Kant}, vor
genau dieses Dilemma gestellt,
propagiere.\footnote{\cite[Vgl.][837]{Schnaedelbach:WirKantianer2005}:
  \enquote{Das ist dann auch der Grund, warum alle ernst zu nehmenden
    Philosophen seit Kant zunächst einmal den kritischen Weg
    einschlagen mussten, und dies auch und gerade dann, wenn sie
    vorhatten, sich von Kant abzuwenden.}} Die vom Ausgangspunkt der
Endlichkeit menschlichen Denkens aus verfasste \titel{Kritik der reinen
Vernunft} sei gerade deswegen Kernstück der Aufklärung, weil die Einsicht in
unsere Endlichkeit und deren Konsequenzen die Vermittlung von Wissenschaft auf
der einen und Selbständigkeit auf der anderen Seite ermögliche. Aber auch wenn
dem so sein sollte, bleibt eine große Bandbreite an Möglichkeiten, dieses
Verhältnis von Aufklärung und Wissenschaft und die Rolle der Vernunftkritik bei
diesem Prozess zu interpretieren. Aus der Sicht \name[Herbert]{Schnädelbach}s geht es
primär um Metaphysikkritik; und eine Passage aus der Vorrede zur ersten Auflage
der \titel{Kritik der reinen Vernunft}, die eine Verbindung zwischen
Aufklärungsdenken und Vernunftkritik zumindest zu erahnen erlaubt, mag diese
Vermutung stützen. \name[Immanuel]{Kant} bezieht sich darin auf die Gleichgültigkeit, die
sein Zeitalter der Metaphysik entgegenbringe, und schreibt:
\begin{quote}
  Sie [die Gleichgültigkeit gegenüber Fragen der Metaphysik; A.\,G.] ist
  offenbar die Wirkung nicht des Leichtsinns, sondern der gereiften
  \ori{Urteilskraft} des Zeitalters, welches sich nicht länger durch
  Scheinwissen hinhalten läßt und eine Auffoderung an die Vernunft, das
  beschwerlichste aller ihrer Geschäfte, nämlich das der Selbsterkenntnis aufs
  neue zu übernehmen und einen Gerichtshof einzusetzen, der sie bei ihren
  gerechten Ansprüchen sichere, dagegen aber alle grundlose Anmaßungen, nicht
  durch Machtsprüche, sondern nach ihren ewigen und unwandelbaren Gesetzen,
  abfertigen könne und dieser ist kein anderer als die \ori{Kritik der reinen
  Vernunft} selbst.\footnote{\cite[][A xi-xii]{Kant:KritikderreinenVernunft2003},
  \cite[][IV: 9.1--10]{Kant:GesammelteWerke1900ff.}.}
\end{quote}
Ob etwas als Wissenschaft taugt, erkennt man nach \name[Immanuel]{Kant} daran,
ob es als \emph{gemeinsames} Erkenntnisprojekt betrieben werden kann. Denn Vernunft
ist an die Möglichkeit eines allgemeinen Standpunkts und des
intellektuellen Austauschs gebunden, wie sich in Kapitel
\ref{section:sensuscommunis} bei der Betrachtung der Maxime der erweiterten
Denkungsart ergab. Was als Kampfplatz einzelner Denker auftritt, die sich nicht
auf eine gemeinsame Grundlage und einen gemeinsamen Forschungsstand verständigen
können, ist (allem Vermuten nach) keine Wissenschaft.\footnote{\enquote{Ob die
Bearbeitung der Erkenntnisse, die zum Vernunftgeschäfte gehören, den sicheren
Gang einer Wissenschaft gehe oder nicht, das läßt sich bald aus dem Erfolg beurteilen.
Wenn \punkt{} es nicht möglich ist, die verschiedenen Mitarbeiter in der Art,
wie die gemeinschaftliche Absicht erfolgt werden soll, einhellig zu machen: so
kann man immer überzeugt sein, daß ein solches Studium bei weitem noch nicht den
sicheren Gang einer Wissenschaft eingeschlagen, sondern ein bloßes Herumtappen
sei} (\cite[][B vii]{Kant:KritikderreinenVernunft2003},
\cite[][III: 7.2--4, 7--11]{Kant:GesammelteWerke1900ff.}).} Die Vernunftkritik hat
die Aufgabe, echte Wissenschaft von bloß angemaßter Wissenschaftlichkeit
zu scheiden, wobei sie dem Bereich dessen, was nicht zur Wissenschaft taugt,
durchaus Achtung zollt und seine Bedeutung betont. Schließlich soll sie
\enquote{das \ori{Wissen} aufheben, um zum \ori{Glauben Platz} zu
bekommen}\footnote{\cite[B xxx]{Kant:KritikderreinenVernunft2003},
\cite[III: 19.6]{Kant:GesammelteWerke1900ff.}.}. Schafft Vernunftkritik also
einen Freiraum für aufgeklärte Religion, indem sie einer aufgeklärten
Selbständigkeit individueller Denker auf der einen und gemeinschaftlich
betriebener Wissenschaft auf der anderen Seite je eigene Bereiche zuweist?

Diese erste und naheliegende, dennoch aber falsche Deutung verweist
auf die Begrenzung von
Geltungsansprüchen.\footnote{\authorfullcite{LaRocca:WasAufklaerungseinwird2004}
spricht hingegen statt von \emph{Begrenzung} von der \emph{Differenzierung} von
Geltungsansprüchen: \enquote{Für das Projekt Aufklärung ist die kritische
Differenzierung von Geltungsbereichen wesentlicher als die bloße Gegenüberstellung
Vernunft/Aberglaube} \parencite[135]{LaRocca:WasAufklaerungseinwird2004}. Dies
entspricht der Deutungsrichtung, die \name[Immanuel]{Kant} als Philosophen der modernen Kultur liest; so z.\,B.
Heinrich \textcite[][141]{Rickert:KantalsPhilosophdermodernenKultur1924}:
\enquote{Kant hat als erster Denker in Europa die \ori{allgemeinsten theoretischen Grundlagen}
geschaffen, die wissenschaftliche Antworten auf spezifisch moderne
Kulturprobleme überhaupt \ori{möglich} machen, und insbesondere läßt sich dartun: sein
Denken, wie es sich in seinen drei großen Kritiken darstellt, ist in dem Sinn
\enquote{kritisch}, d.\,h. \ori{scheidend} und \ori{Grenzen ziehend} gewesen,
daß es dadurch im Prinzip dem Prozeß der \ori{Verselbständigung} und \ori{Differenzierung} der Kultur
entspricht, wie er sich seit dem Beginn der Neuzeit faktisch vollzogen, aber in
der Philosophie vor Kant noch keinen theoretischen Ausdruck gefunden hatte.}
Ähnliches behauptet im Anschluss an \authorcite{Rickert:KantalsPhilosophdermodernenKultur1924} auch
\textcite[vgl.][838]{Schnaedelbach:WirKantianer2005}.} Ihre Vertreter könnten
etwa folgendes anführen:
\begin{quote}
\enquote{Wir können, dürfen und sollen uns auf Autoritäten, Traditionen und unsere
Mitmenschen verlassen, aber nur dort, wo objektiv gültiges Wissen vorliegt oder
zumindest nach unserem Kenntnisstand vorliegen könnte. Bei den
Naturwissenschaften und der Mathematik ist dies der Fall -- deswegen ist niemand
unmündig, wenn er den Lehrbüchern von Physik, Chemie, Biologie oder Algebra
Glauben schenkt. Die Vernunftkritik beschränkt den Bereich möglichen Wissens auf das, was Gegenstand
möglicher Erfahrung ist. Wissenschaft im eigentlichen Sinne ist eine
mathematisch fundierte\footnote{\cite[Vgl.][A
viii]{Kant:MetaphysischeAnfangsgruendederNaturwissenschaften1977}, \cite[IV:
470.13--15]{Kant:GesammelteWerke1900ff.}: \enquote{Ich behaupte aber, daß in
jeder besonderen Naturlehre nur so viel eigentliche Wissenschaft angetroffen
werden könne, als darin Mathematik anzutreffen ist.}} Erfahrungserkenntnis nach
dem Vorbild \name[Isaac]{Newton}s. In der Metaphysik -- und damit auch in der
natürlichen Theologie -- gibt es daher keine gültigen Wissensansprüche, also ist
unmündig, wer Aussagen zur Metaphysik Glauben schenkt.}
\end{quote}
Eine solche Konzeption versucht, die von der Aufklärung geforderte Selbständigkeit
dadurch mit der Autorität der Wissenschaft zu vereinbaren, dass sie beiden
Seiten jeweils eigene Bereiche zuweist.


\phantomsection\label{ThomasiusZuPrivatheitreligioesenBekenntnisses}\name[Christian]{Thomasius}
begründet die Tatsache, dass Häresie nicht justiziabel ist, mit
dem Mangel an Wissenschaftlichkeit religiöser Behauptungen. Gerade weil es bei
vielen Fragen im Bereich der Religion keine allgemein verbindlichen und
wissenschaftlich beglaubigten Wahrheiten gebe, könne, dürfe und solle jeder
seinem je eigenen Urteil in Fragen der Religion
folgen.\footnote{\cite[Vgl.][\pno~244\,f.]{Albrecht:ChristianThomasius1999}.}
Und in dieselbe Richtung scheint es zu weisen, wenn \name[Immanuel]{Kant} der
Vernunftkritik die Aufgabe zuweist, das Wissen zu begrenzen, um dem Glauben
Platz zu verschaffen: Weil eben niemand ein fundiertes Expertenvotum bezüglich
der Wahrheit in Religionsfragen, die er zum Zentrum seines Aufklärungsdenkens
erklärt\footnote{\cite[Vgl.][A
492\,f.,]{Kant:BeantwortungderFrage:WasistAufklaerung?1977} \cite[][VIII:
41.10--22]{Kant:GesammelteWerke1900ff.}.}, abgeben könne, deshalb solle jeder
sich ein eigenes Urteil bilden.\footnote{Belege \emph{für} diese Deutung liefern
Stellen wie die folgende: \enquote{Ich kann also nur sagen: \ori{Ich} sehe mich
durch meinen Zweck nach Gesetzen der Freiheit genötiget, ein höchstes Gut in der
Welt als möglich anzunehmen, aber \ori{ich kann keinen andern durch Gründe
nötigen} (der Glaube ist \ori{frei})} (\cite[][A
104]{Kant:ImmanuelKantsLogik1977}, \cite[][IX:
69.22--25]{Kant:GesammelteWerke1900ff.}).} Denn hörten wir bei solchen Dingen
auf andere, so billigten wir ihnen eine Autorität zu, die sich nicht durch ihr
Wissen als legitim ausweisen lässt. Wir vertrauten auf unechte, nur angemaßte
Experten. Und dasselbe gelte schließlich auch für die Philosophie, die (noch)
keine Wissenschaft sei und die man deshalb nicht lernen
könne.\footnote{\cite[Vgl.][B 866]{Kant:KritikderreinenVernunft2003}, \cite[III:
542.12--14]{Kant:GesammelteWerke1900ff.}: \enquote{Bis dahin kann man keine
Philosophie lernen; denn, wo ist sie, wer hat sie im Besitze, und woran läßt sie
sich erkennen?}}

Eine solche Lösung scheint auf den ersten Blick naheliegend zu sein und dabei
zugleich einen möglichen Zusammenhang von Endlichkeit und
Aufklärungsprogrammatik aufzuzeigen: Wie in Kapitel
\ref{subsubsection:UnterscheidungvonDenkenundErkennen} gezeigt, ist die
Unterscheidung von Möglichkeit und Wirklichkeit der zentrale Ausdruck unserer
Endlichkeit. Und diese Unterscheidung wird mitunter als Grundlage unserer
(individuellen) Freiheit und der Aufklärung
interpretiert.\footnote{\cite[Vgl.][283]{Engfer:MenschlicheVernunft2002}.} Im Denken frei wären wir dann dort, wo wir
verschiedene Möglichkeiten \emph{denken}, aber ihre Wahrheit oder Falschheit
nicht \emph{erkennen} können. Das \emph{Denken} soll also gänzlich frei und
ungebunden sein, wohingegen das \emph{Erkennen} sich nach
objektiven Regeln der Wissenschaftlichkeit richtet und auf die Erfahrung als
Aktualisierung eines \emph{rezeptiven} Erkenntnisvermögens angewiesen bleibt.

Auf den zweiten Blick verliert sich der Reiz dieser Konzeption und sie erweist
sich als äußerst unbefriedigend. In Wahrheit expliziert sie nämlich gar keinen
Begriff autonomen und authentischen Wissens, sondern nennt zwei Bereiche, wobei
wir in dem einen Bereich Wissen erwerben und in dem anderen vielleicht
authentisch sind, aber weder in dem einen noch in dem anderen Bereich handelt es sich um
Autonomie. Denn nach diesem Vorschlag sollen in einem Bereich Regeln der
Vernunft walten (Wissenschaft), während wir in einem \emph{anderen} Bereich frei
sein dürfen (Religion und Metaphysik). Autonomie aber verlangt, dass Freiheit
durch Regeln der Vernunft konkretisiert wird und die Regeln der Vernunft Regeln der
Freiheit sind. Freiheit, Unterwerfung unter Regeln der Vernunft und damit die
Möglichkeit der kollektiven Verständigung müssen \emph{in demselben
Erkenntnisbereich} vorliegen, soll sinnvoll von Autonomie gesprochen werden
können.
 
Die genannte Deutung ist  nicht nur systematisch unbefriedigend, sondern auch
als \name[Immanuel]{Kant}interpretation fragwürdig. Denn \name[Immanuel]{Kant}
akzeptiert keine zügellose Freiheit außerhalb der Wissenschaft, sondern fordert
auch von Religion und Ethik -- zwei Bereichen, die er eindeutig \emph{nicht} den
Erfahrungswissenschaften zuweist -- Vernünftigkeit und Mitteilbarkeit. Wenn
verschiedene Moralphilosophien vertreten werden und Menschen verschiedenen
Religionen und Konfessionen angehören, wie es \emph{de facto} der Fall ist, dann
zeigt dies die Unzulänglichkeit der (meisten) eingenommenen Positionen und
religiösen Überzeugungen. Schließlich gebe es nur eine einzige \emph{korrekte}
Moralphilosophie und nur eine einzige \emph{wahre} Religion für alle
Menschen.\footnote{Zu letzterem siehe z.\,B. \cite[][A
45]{Kant:DerStreitderFakultaeten1977}, \cite[VII:
36.26--32]{Kant:GesammelteWerke1900ff.}.} Dass \name[Immanuel]{Kant} den Bereich
der religiösen Überzeugungen nicht der Beliebigkeit anheim stellt, kann zu dem Urteil
verleiten, er sei in seiner Religionskritik in aufklärerischer Absicht auf
halbem Wege stehen geblieben, insofern er versucht, religiöse Inhalte aus reiner
Vernunft zu
generieren.\footnote{\cite[Vgl.][135--137]{LaRocca:WasAufklaerungseinwird2004}.
\authorcite{Horkheimer:DialektikderAufklaerung1997} werten es als Inkonsequenz des
Aufklärungsdenkens, überhaupt Inhalte (außer der reinen Selbsterhaltung) als
vernünftig ausweisen zu wollen. Der einzig konsequente \singlequote{Aufklärer}
sei daher der \name[Donatien-Alphonse-Fran{\c{c}}ois]{Marquis de Sade} gewesen;
\cite[vgl.][101]{Horkheimer:DialektikderAufklaerung1997}: \enquote{Jedes
inhaltliche Ziel, auf das die Menschen sich berufen mögen, als sei es eine
Einsicht der Vernunft, ist nach dem strengen Sinn der Aufklärung Wahn, Lüge,
\singlequote{Rationalisierung}, mögen die einzelnen Philosophen sich auch die
größte Mühe geben, von dieser Konsequenz hinweg aufs menschenfreundliche Gefühl zu lenken.}}
Es ist aber keine Inkonsequenz, wenn \name[Immanuel]{Kant} es gerade nicht für Ziel und
Ergebnis einer aufgeklärten Religionskritik ansieht, \enquote{jede religiöse
Erfahrung als gleichrangige Möglichkeit zu
berücksichtigen}\footcite[][136]{LaRocca:WasAufklaerungseinwird2004}.
Dies ergibt sich vielmehr zwingend aus der Darstellung der Aufklärungsforderung
in der Form der drei Maximen des Denkens, die nicht nur von Wissenschaften,
sondern von \emph{jedem} vernünftigen Gedanken fordern, dass er der Vernunft und
der erweiterten Denkungsart gemäß ist.\footnote{Siehe oben, Kap.
\ref{section:KantalsliberalerAufklaerer}.}

Daraus ergibt sich auch die Bestimmtheit, mit der er auf Offenbarung gegründete
religiöse Überzeugungen kritisiert; ein nennenswertes Beispiel findet sich im
\titel{Streit der Fakultäten}, wo \name[Immanuel]{Kant} sich nicht scheut,
\singlename{Abraham}s Vertrauen in die Stimme Gottes -- welche ihm befiehlt,
seinen eigenen Sohn als Opfer darzubringen -- der Irrationalität zu
schelten.\footnote{\enquote{\singlename{Abraham} hätte auf diese vermeinte
göttliche Stimme antworten müssen: \enquote{daß ich meinen guten Sohn nicht töten solle, ist ganz
gewiß; daß aber du, der du mir erscheinst, Gott sei, davon bin ich nicht gewiß,
und kann es auch nicht werden, wenn sie auch vom (sichtbaren) Himmel
herabschallete}.} \mkbibparens{\cite[][A
102\,f.]{Kant:DerStreitderFakultaeten1977}; \cite[][VII:
63.34--38]{Kant:GesammelteWerke1900ff.}}. Siehe hierzu auch Kapitel
\ref{Beispiel:AbrahamOpfertSeinenSohn}, ab
S.~\pageref{Beispiel:AbrahamOpfertSeinenSohn}.} Jede religiöse Überzeugung --
auch und gerade Offenbarungserlebnisse -- sollen sich innerhalb der Grenzen der
bloßen Vernunft bewegen. Gerade weil er Aufklärung und Mündigkeit in
Religionsfragen fordert, kann er keine Beliebigkeit zulassen, sondern muss die Möglichkeit verbindlicher
Vernunftwahrheiten voraussetzen. Und deswegen gibt es laut \name[Immanuel]{Kant}
auch nur \emph{eine} Religion, die freilich in verschiedenen Ausprägungen
auftreten kann. Doch ist diese Varietät als ein Mangel anzusehen, den es zu
überwinden gilt.\footnote{\cite[Vgl.][B
167--183]{Kant:DieReligioninnerhalbderGrenzenderblossenVernunft1977}, \cite[VI:
115.1--124.5]{Kant:GesammelteWerke1900ff.}.}

\phantomsection\label{Absatz:UniversalitaetderAufklaerung} Ebenso wenig
akzeptiert er Unmündigkeit in Bereichen, deren Behandlung der
Erfahrungswissenschaft zusteht, denn \name[Immanuel]{Kant} betont stets,
Aufklärung sei die Maxime, \emph{jederzeit} selbst zu
denken,\footnote{\cite[Vgl.][A
229]{Kant:Washeisst:SichimDenkenorientieren?1977}, \cite[VIII:
146.30--31]{Kant:GesammelteWerke1900ff.}.} und die Maxime einer vorurteilsfreien
Denkungsart sei die einer \emph{niemals} passiven
Vernunft\footnote{\cite[Vgl.][\S~40]{Kant:KritikderUrteilskraft2009}, \cite[V:
294.20]{Kant:GesammelteWerke1900ff.}.}. Die Aufforderung, unseren eigenen
Verstand zu gebrauchen, betrifft jeden Erkenntnisbereich und alle Urteile, nicht
eine Auswahl aus diesen. Wir können keine Vereinbarkeit von individueller
Selbstbestimmung im Denken mit der Autorität wissenschaftlicher Erkenntnisse
herbeiführen, indem wir die Geltungsansprüche der letzteren beschneiden.

Gewiss bleibt an dieser Stelle noch die Deutungsmöglichkeit, Wissenschaft und
Aufklärung durch die \emph{methodische Orientierung} (nicht wie bei
\authorcite{Wolff:Psychologiaempirica1968} an der Mathematik, sondern) \emph{an den
Naturwissenschaften} zu vereinbaren. Selbstdenken sei dann also gerade in den
Wissenschaften und \emph{nicht} im Bereich der Religion zu verorten.
\name[Immanuel]{Kant} sagt, dass Aufklärung und Mündigkeit demjenigen leicht
fielen, der \enquote{das, was über seinen Verstand ist, nicht zu wissen
verlangt}\footnote{\cite[\S 40]{Kant:KritikderUrteilskraft2009}, \cite[V:
294.32--33]{Kant:GesammelteWerke1900ff.}.}. Der aufgeklärte Mensch wäre
derjenige, der die kritische, methodisch disziplinierte und auf stete Kontrolle
an der Erfahrung ausgerichtete Art naturwissenschaftlichen Denkens erlernt und
verinnerlicht hat und sich von \distanz{Spekulationen} über Unerkennbares fern
hält. Religion und Metaphysik wären dann gerade keine Bereiche, die freiem und
mündigem Denken offen stehen --  es sei denn es gelänge, sie auf eine feste
methodische Basis zu stellen. In Kapitel
\ref{section:KantalsliberalerAufklaerer} habe ich nur hervorgehoben, dass
\name[Immanuel]{Kant} \authorcite{Wolff:Psychologiaempirica1968} nicht darin
folgt, die Vereinbarkeit dadurch zu gewährleisten, dass er die Methode der
Mathematik überall zugrunde legt. Aber es könnte noch sein, dass er die Methodik
der Naturwissenschaften -- die stete Rückbindung an die Erfahrung oder auch die
experimentelle Methode -- für die geeignete Grundlage hält. Zumindest in
\name[Immanuel]{Kant}s \distanz{vorkritischer} Phase finden sich in seinen
Schriften Belege für die zweite Deutungsmöglichkeit.
\phantomsection\label{anm:jungerkantundaufklaerung}In den 1750er Jahren betrieb
\name[Immanuel]{Kant} Aufklärung aus der Perspektive einer
\name[Isaac]{Newton}'schen Naturphilosophie, indem er theologisch besetzte
Themen von moralphilosophischer Bedeutung -- z.\,B.\ das Seebeben von Lissabon
1755 in drei Abhandlungen aus dem darauf folgenden
Jahr\footnote{Siehe
\cite{Kant:VondenUrsachenderErderschuetterungenbeiGelegenheitdesUnglueckswelchesdiewestlicheLaendervonEuropagegendasEndedesvorigenJahresbetroffenhat1910},
\cite[][I: 417--427]{Kant:GesammelteWerke1900ff.},
\cite{Kant:GeschichteundNaturbeschreibungdermerkwuerdigstenVorfaelledesErdbebenswelchesandemEndedes1755stenJahreseinengrossenTheilderErdeerschuetterthat1910},
\cite[][I: 429--461]{Kant:GesammelteWerke1900ff.},
\cite{Kant:FortgesetzteBetrachtungderseiteinigerZeitwahrgenommenenErderschuetterungen1910}, \cite[vgl.][I:
463--472]{Kant:GesammelteWerke1900ff.}.} -- nach den zu seiner Zeit aktuellen
Standards der Wissenschaften untersuchte.  In dieser Hinsicht erweist sich
\name[Immanuel]{Kant} bis zu Beginn der 1760er Jahre tatsächlich als
Vertreter einer naturwissenschaftlich geprägten Aufklärung.\footnote{\name[Immanuel]{Kant}
sucht dabei nicht für die Theologie nach einer geeigneten Methodik, sondern für eine Grundlegung der Metaphysik.
Diese sollte auf Grundlage sicherer Erfahrungssätze möglich sein, wie er 1763
schreibt. \cite[Vgl.][A
69f.]{Kant:UntersuchungueberdieDeutlichkeitderGrundsaetzedernatuerlichenTheologieundderMoral1977},
\cite[][II: 275.8--24]{Kant:GesammelteWerke1900ff.}.}

Doch diese \distanz{szientistische} Vorstellung von Aufklärung unterscheidet
sich ganz offensichtlich von dem, was \name[Immanuel]{Kant} in der 1780er Jahren
in der Berlinischen Monatsschrift propagiert. Aushängeschild hierfür
ist seine Betonung der Religion als Themenbereich, der Mündigkeit und
Selbstdenken verlange und zulasse, was zusammen mit der Trennung von Wissen und
Glauben in der \titel{Kritik der reinen Vernunft} deutlich macht, dass
Selbstdenken nicht durch eine Orientierung an der
newtonschen Physik zu verstehen sei. \name[Immanuel]{Kant}s spätes
Aufklärungsverständnis ist das Ergebnis einer Entwicklung zwischen den 1760er
und den 1780er Jahren. In einer Anmerkung in seinem Durchschussexemplar der
\titel{Beobachtungen über das Schöne und Erhabene} relativiert \name[Immanuel]{Kant} in
Anlehnung an \name[Jean-Jacques]{Rousseau} den allgemeinen Wert von Wissenschaft und
Forschung, welche nur im Kontext der Herstellung der \enquote{[R]echte der
Menschheit} einen Wert
erhielten.\footnote{\phantomsection\label{Anmerkung:Rousseauhatmichzurechtgebracht}\cite[Vgl.][XX:
44.8-16]{Kant:GesammelteWerke1900ff.}: \enquote{Ich selbst bin aus Neigung ein
Forscher. Ich fühle den gantzen Durst nach Erkenntnis u.\ die begierige Unruhe
darin weiter zu kommen oder auch die Zufriedenheit bey jedem Erwerb. Es war eine
Zeit da ich glaubte dieses allein könnte die Ehre der Menschheit machen u.\ ich
verachtete den Pöbel der von nichts weis. \ori{\name[Jean-Jacques]{Rousseau}} hat mich zurecht
gebracht. Dieser verblendete Vorzug verschwindet, ich lerne die Menschen ehren
u.\ ich würde mich unnützer finden wie den gemeinen Arbeiter wenn ich nicht
glaubete daß die Betrachtung allen übrigen einen Werth ertheilen könne, die
[R]echte der Menschheit herzustellen}. Zur Problematik der Datierung siehe
\cite[][67--71]{Schwaiger:KategorischeundandereImperative1999}.} Dies markiert
die Abkehr von dem szientistischen
Aufklärungsverständnis. \Revision{\authorcite{Macor:DieBestimmungdesMenschen1748--18002013}
schreibt, \name[Immanuel]{Kant} nehme fortan \enquote{entschieden
  gegen intellektualistische Positionen
  Stellung}\footcite[][203]{Macor:DieBestimmungdesMenschen1748--18002013}
und sehe den Wert des Menschen in der Moral, die -- so wird sich noch
als bedeutsam herausstellen --, mitnichten von besonderen, nur wenigen
vorbehaltenen Kenntnissen und Kompetenzen abhängig sei.} Diese Abkehr
drückt sich eben auch in dem Aufsatz über den Aufklärungsbegriff aus, wenn
\name[Immanuel]{Kant} die Religion und nicht die empirischen Wissenschaften
thematisiert und dies auch explizit hervorhebt. Religion ist aber nach
\name[Immanuel]{Kant} ebenso wie die Ethik nicht mit den Mitteln empirischer
Forschung anzugehen -- ganz im Gegenteil: Eine vernünftige Religion erwächst aus
reiner Vernunft, wie \name[Immanuel]{Kant} beispielsweise in der \titel{Kritik
der Urteilskraft} verdeutlicht. Sie baut ihre Fundamente auf der
Moralphilosophie auf, nicht auf der (teleologischen) Betrachtung der Natur, und
ist infolge dessen nicht die Weiterführung empirischer Naturforschung
(Physikotheologie), sondern apriorischer Ethik
(Ethikotheologie).\footnote{\cite[Vgl.][\S~85\,f.,]{Kant:KritikderUrteilskraft2009}
\cite[V: 436.3--447.13]{Kant:GesammelteWerke1900ff.}.}

Wenn \name[Immanuel]{Kant} die Aufklärung vornehmlich in Fragen der
Religion setzt, dann kann die Vereinbarkeit von Selbständigkeit und Wissenschaft
nicht in der Übernahme einer Methodik der Naturwissenschaften bestehen. Zu den
Themen der Aufklärung gehören gerade auch solche, die den Bereich der Erfahrung
transzendieren. Auch zeitlich kongruiert die Akzentverschiebung
\name[Immanuel]{Kant}s mit der Entwicklung seines Aufklärungsprogramms.
So schreibt \authorfullcite{Kreimendahl:Kant1990}: \enquote{Man darf \punkt{}
festhalten, dass Kants Aufklärungsbegriff in den 1750er und 1760er Jahren noch
in keiner wie
auch immer gearteten Verbindung zum Programm des Selbstdenkens, zu Mündigkeit
oder Autonomie des Subjekts
steht.}\footnote{\cite[][134]{Kreimendahl:KantsvorkritischesProgrammderAufklaerung2009}}
Erst in dem \distanz{stillen Jahrzehnt} zwischen 1770 und 1781 taucht die
Verbindung der Begriffe \enquote{Aufklärung} und \enquote{Autonomie} oder
\enquote{Selbstdenken} erstmalig im handschriftlichen Nachlass
auf\footnote{\cite[Vgl.][\pno~134\,f.]{Kreimendahl:KantsvorkritischesProgrammderAufklaerung2009}}
Damit fällt die Entwicklung des Aufklärungsbegriffs zwar \emph{zeitlich}
mit der Entstehung der kritischen Philosophie zusammen, aber der
\emph{inhaltliche} Zusammenhang -- wenn ein solcher überhaupt besteht -- wird
zunehmend undurchsichtig.


\section{Die Bestimmung des Menschen und die \emph{conditio
humana}}\label{subsection:DieBestimmungdesMenschen} Die Forderung nach
Mündigkeit und epistemischer Selbstbestimmung wäre einfacher zu erfüllen, wenn
sie sich nur auf einen eingeschränkten Themenbereich bezöge; damit wäre eine
Vereinbarkeit mit der Ansicht gegeben, wir könnten uns zumindest in den meisten
Bereichen auf das Wissen von Experten verlassen, ohne dadurch unsere Mündigkeit
zu gefährden. Einer solchen Einschränkung widersprechen aber eindeutige
Äußerungen \name[Immanuel]{Kant}s. Die Verbindung von Aufklärung und
Selbstdenken verträgt sich nicht mit einer thematischen
\emph{Einschränkung} der Aufklärung; sie geht aber mit einer
\emph{Akzentuierung} -- und zwar in der Aufklärungsschrift: der Religion --
einher. Vielleicht mögen Akzentsetzungen helfen, die Forderung nach Mündigkeit
und Selbständigkeit erfüllbar zu halten: Niemand ist in der Lage, in
\emph{allen} Erkenntnisbereichen kompetent zu sein. Aber wenn jeder nur in
Bereichen kompetent sein soll, die von besonderer Relevanz sind, muss die
Forderung nach Mündigkeit und Selbständigkeit nicht überschwänglich
wirken.\footnote{\cite[Vgl.][\S~40]{Kant:KritikderUrteilskraft2009}, \cite[V:
294.29--37]{Kant:GesammelteWerke1900ff.}.}

\subsection{Die Ordnung der Erkenntnisse}
\name[Immanuel]{Kant} begründet seine Akzentuierung der Religion im Rahmen der
Aufklärungsprogrammatik mit dem Mangel an Aufmerksamkeit der Herrschenden auf
die Wissenschaften.\footnote{\cite[Vgl.][A
492]{Kant:BeantwortungderFrage:WasistAufklaerung?1977}, \cite[][VIII:
41.10--14]{Kant:GesammelteWerke1900ff.}.} In der Tat scheint es für eine
Regierung kaum von Interesse zu sein, in die naturwissenschaftliche Forschung
und Lehre einzugreifen, denn ein ungehindertes und erfolgreiches Forschen kann
ihr dort schon wegen der Entwicklung technischer Möglichkeiten nur von Vorteil
sein. Es ist -- um ein prominentes und sicherlich auch \name[Immanuel]{Kant}
bekanntes Beispiel zu verwenden -- kaum vorstellbar, dass Christian
\authorcite{Wolff:Psychologiaempirica1968} vom preußischen König für seine Lehr-
und Publikationstätigkeit sanktioniert worden wäre, wenn er sich auf Themen der
angewandten Mathematik beschränkt hätte.
Nur bei Lehre sowie Vortrags- und Publikationstätigkeit im Bereich der Religion
-- wie in \authorcite{Wolff:Psychologiaempirica1968}s Prorektoratsrede über die
praktische Philosophie der
Chinesen\footnote{\cite[Vgl.][]{Wolff:OratiodeSinarumphilosophiapractica1988}.
Zur Bedeutung der Prorektoratsrede für sein weiteres Schicksal siehe
\cite[][\pno~xlvi--liii]{Albrecht:Einleitung1988}.} -- wurde ein Eingreifen
denkbar, weil die Regierung hier ein eigenes Interesse haben konnte. Ein solches
Interesse der Regierungen an der Beeinflussung von Lehre und
Publikationstätigkeit nennt \name[Immanuel]{Kant} an anderer Stelle als
Kennzeichen der oberen Fakultäten gegenüber der freien unteren Fakultät. Es
handelt sich bei den oberen Fakultäten um diejenigen Disziplinen, mit denen sich
eine Regierung Einfluss auf die Verhaltensweise des Volkes verschaffen
kann.\footnote{\cite[Vgl.][A 6--8]{Kant:DerStreitderFakultaeten1977},
\cite[][VII: 18.30--19.20]{Kant:GesammelteWerke1900ff.}.} Dagegen ist die untere
(philosophische) Fakultät diejenige, die ihre Lehren nicht auf Befehl der
Regierung annimmt, sondern die Interessen einer aufgeklärten Mündigkeit
vertritt.\footnote{\cite[Vgl.][A~~25]{Kant:DerStreitderFakultaeten1977},
\cite[][VII: 27.32-35]{Kant:GesammelteWerke1900ff.}: \enquote{Also wird die
philosophische Fakultät, darum, weil sie für die \ori{Wahrheit} der Lehren, die
sie aufnehmen, oder auch einräumen soll, stehen muß, in so fern als frei und nur
unter der Gesetzgebung der Vernunft, nicht aber der Regierung stehend gedacht
werden müssen.}} Zwischen diesen Fakultäten sei kein dauerhafter Friedenszustand
möglich, sondern ein Aufbrechen des Konflikts jederzeit zu
befürchten,\footnote{\cite[Vgl.][A~35-42]{Kant:DerStreitderFakultaeten1977},
\cite[VII: 32.13-35.27]{Kant:GesammelteWerke1900ff.}.} was der Behauptung
korrespondiert, es gebe kein aufgeklärtes Zeitalter, sondern nur ein Zeitalter
der Aufklärung.\footnote{\cite[Vgl.][A
491]{Kant:BeantwortungderFrage:WasistAufklaerung?1977}, \cite[][VIII:
40.17--19]{Kant:GesammelteWerke1900ff.}: \enquote{Wenn den nun gefragt wird:
Leben wir jetzt in einem \ori{aufgeklärten} Zeitalter? so ist die Antwort: Nein,
aber wohl in einem Zeitalter der \ori{Aufklärung}.}} Die
oberen Fakultäten seien beliebte Ansprechpartner für all diejenigen, die sich in ihrer Unmündigkeit eingerichtet haben, weil sie kein Interesse an ihrer eigenen Freiheit nehmen.\footnote{\cite[][A~29-30]{Kant:DerStreitderFakultaeten1977}, \cite[VII:
29.34--30.9]{Kant:GesammelteWerke1900ff.}: \enquote{Nun wird der Streit der
Fakultäten um den Einfluß aufs Volk geführt, und diesen Einfluß können sie nur bekommen, so fern jede derselben das Volk glauben machen kann, daß sie das Heil desselben am besten zu befördern verstehe, dabei aber doch in der Art, wie sie dieses auszurichten gedenken,
  einander gerade entgegengesetzt sind.\\
  Das Volk aber setzt sein Heil zu oberst nicht in der Freiheit, sondern in
  seinen natürlichen Zwecken, also in diesen drei Stücken: nach dem Tode selig,
  im Leben unter andern Mitmenschen des Seinen, durch öffentliche Gesetze
  gesichert, endlich des physischen Genusses des Lebens an sich selbst (d.\,i.
  der Gesundheit und langen Lebens) gewärtig zu
  sein.}}

Im Falle der Religion stellt die untere Fakultät dem biblischen Theologen,
insofern dieser \enquote{von dem verschrienen Freiheitsgeist der Vernunft und
Philosophie noch nicht angesteckt
ist}\footnote{\cite[][A~18]{Kant:DerStreitderFakultaeten1977}, \cite[VII:
24.26--27]{Kant:GesammelteWerke1900ff.}.} und sich nicht auf Vernunft, sondern
auf Offenbarung beruft, eine Vernunftreligion
entgegen.\footnote{\cite[Vgl.][A~15-18]{Kant:DerStreitderFakultaeten1977},
\cite[VII: 23.9--24.30]{Kant:GesammelteWerke1900ff.}.} Nun betrachtet er
die Auseinandersetzung zwischen Vernunft und Offenbarung nur als einen
Teilbereich in einer Auseinandersetzung der Vernunft mit Vertretern der
oberen Fakultäten. Es gibt drei obere Fakultäten und somit kann nicht nur
die Theologie mit einer Vernunftreligion in Konflikt geraten, sondern auch
Jurisprudenz und Medizin geraten in einen Widerstreit mit der unteren Fakultät.
Also findet auch hier eine Auseinandersetzung um Aufklärung und Mündigkeit
statt. \name[Immanuel]{Kant} stellt in der Aufklärungsschrift aber nicht
Religion, Gesundheit und Recht in den Mittelpunkt, sondern ausschließlich die Religion,
wofür es also einen weiteren Grund geben muss. Dieser lautet, dass Religion eine
besondere Bedeutung für uns Menschen habe: Darin sei Unmündigkeit \enquote{so
wie die schädlichste, also auch die entehrendste unter
allen}\footnote{\cite[Vgl.][A~492]{Kant:BeantwortungderFrage:WasistAufklaerung?1977},
\cite[VIII: 41.14--15]{Kant:GesammelteWerke1900ff.}.}. Aber was ist es, was
Religion zu einem so zentralen Thema macht? Warum nennt er Unmündigkeit in
Fragen der Religion als besonders schädlich und nicht Unmündigkeit bezüglich
naturwissenschaftlicher, historischer, rechtlicher, politischer oder
medizinischer Ansichten? Der Verweis auf die Aufmerksamkeit der Herrschenden
stellt keine befriedigende Antwort dar, weil sie selbst der Begründung bedürfte:
Warum ist es für die Herrschenden so interessant, gerade die öffentliche Meinung
zu Religionsfragen zu regulieren?

Nach \name[Immanuel]{Kant} gibt es (auf Seiten der unteren Fakultät) eine
natürliche Hierarchie von Erkenntnissen, wobei an der Spitze zunächst nicht die
Religion, sondern die Philosophie (nach ihrem Weltbegriff) steht:
\begin{quote}
  Der Mathematiker, der Naturkündiger, der Logiker sind, so vortrefflich die
  ersteren auch überhaupt im Vernunfterkenntnisse, die zweiten besonders im
  philosophischen Erkenntnisse Fortgang haben mögen, doch nur Vernunftkünstler. Es
  gibt noch einen Lehrer im Ideal, der alle diese ansetzt, sie als Werkzeuge
  nutzt, um die wesentlichen Zwecke der menschlichen Vernunft zu befördern. Diesen
  allein müßten wir den Philosophen
  nennen\dots\footnote{\cite[][B~867]{Kant:KritikderreinenVernunft2003}, \cite[][III:
  542.33--543.2]{Kant:GesammelteWerke1900ff.}.}
\end{quote}
Der \enquote{Vernunftkünstler} betreibt Philosophie bloß nach ihrem
Schulbegriff, wonach \enquote{sie nur als eine von den Geschicklichkeiten zu
gewissen beliebigen Zwecken angesehen
wird.}\footnote{\cite[B~867]{Kant:KritikderreinenVernunft2003},
\cite[III: 543.33--34]{Kant:GesammelteWerke1900ff.}.} In diesem Sinne gibt es
eine mit wissenschaftlichem Anspruch auftretende Philosophie, die sich anhand ihres
\distanz{Handwerkszeugs} erkennen lässt. Die logische Strenge der
mathematischen Methode oder heutzutage die methodischen Maßstäbe, welche die Analytische
Philosophie sich selbst zugute hält, sind beispielsweise deutliche Erkennungsmarken einer
solchen Philosophie nach dem Schulbegriffe. Nach dem Schulbegriff sei
\enquote{Philosophie {\punkt} das System der philosophischen Erkenntnisse oder
der Vernunfterkenntnisse aus Begriffen.}\footnote{\cite[][A
23]{Kant:ImmanuelKantsLogik1977}, \cite[][IX:
23.30--31]{Kant:GesammelteWerke1900ff.}.} Dies beschreibt die Art von
Erkenntnissen, die \name[Immanuel]{Kant} ihrer Form nach -- als weder empirische
(sondern Vernunft-) Erkenntnisse, noch solche aus der Konstruktion von Begriffen
(wie in der Mathematik) --  als philosophisch ansieht. Ich werde auf diesen
Schulbegriff, insofern er die Philosophie als System der \emph{philosophischen}
oder \emph{Vernunft}erkenntnisse charakterisiert, in Kapitel
\ref{subsection:BewertungvonInformationennachihrerART} im Rahmen einer
Diskussion testimonialen Wissens zurückkommen. Zunächst ist der -- für
\name[Immanuel]{Kant} wichtigere -- \enquote{Weltbegriff} zu erläutern.
\name[Immanuel]{Kant} schreibt: \enquote{Nach dem \ori{Weltbegriffe} ist
sie [d.\,i. die Philosophie; A.\,G.] die Wissenschaft von den letzten Zwecken
der menschlichen Vernunft.}\footnote{\cite[][A~23]{Kant:ImmanuelKantsLogik1977}, \cite[][IX:
23.32--33]{Kant:GesammelteWerke1900ff.}. \cite[Siehe
auch][B~866]{Kant:KritikderreinenVernunft2003}, \cite[III:
542.3--24]{Kant:GesammelteWerke1900ff.}.} Ihrem Weltbegriffe nach ist sie keine
kontingent betriebene Wissenschaft, auf die wir auch verzichten könnten, sondern
notwendiges Thema der Vernunft. Das Wissen um die Bestimmung des Menschen und
die Beziehung aller Erkenntnisse auf die \enquote{notwendigen und wesentlichen
Zwecke der Menschheit} nennt \name[Immanuel]{Kant} dann auch
\enquote{Weisheit}.\footnote{\cite[][B~878]{Kant:KritikderreinenVernunft2003},
\cite[][III: 549.13--24]{Kant:GesammelteWerke1900ff.}. Zur Rolle des Begriffs
der \enquote{Weisheit} im Rahmen der \index{Kant, Immanuel}kantischen Aufklärungsphilosophie
siehe \cite{Trawny:DasIdealdesWeisen2008}.}

Was sind nun die \enquote{wesentlichen Zwecke}, um die es der Philosophie ihrem
Weltbegriffe nach geht? \name[Immanuel]{Kant} nennt den höchsten Zweck allen
Vernunftgebrauchs und damit auch dasjenige, was im Zentrum der Aufklärung steht, den Endzweck der Vernunft, der
nur ein einziger sein könne und auf den sich alle anderen Zwecke der Vernunft,
als solche, bezögen. Er ist also das Ziel letztlich \emph{aller} Wissenschaft. Und zwar
handle es sich um \enquote{die ganze Bestimmung des
Menschen}.\footnote{\cite[][B~868]{Kant:KritikderreinenVernunft2003}, \cite[][III:
543.11]{Kant:GesammelteWerke1900ff.}.
Dies spiegelt sich auch in \name[Immanuel]{Kant}s Zusammenfassung der
Philosophie zu der Frage \enquote{Was ist der Mensch?} wieder
\mkbibparens{\cite[vgl.][A 25]{Kant:ImmanuelKantsLogik1977}, \cite[][IX:
25.3--6]{Kant:GesammelteWerke1900ff.}}, die an der Parallelstelle in der
\titel{Kritik der reinen Vernunft}
\mkbibparens{\cite[vgl.][B 833]{Kant:KritikderreinenVernunft2003},
\cite[][III: 522.32--34]{Kant:GesammelteWerke1900ff.}} jedoch nicht vorkommt.
Norbert \name[Norbert]{Hinske} zufolge handelt es sich bei der Bestimmung des Menschen
um eine der \enquote{Basisideen} der Aufklärung
\parencite[vgl.][398]{Hinske:ArtikelAufklaerung1985}.}
\name[Immanuel]{Kant} schreibt in der \titel{Kritik der reinen Vernunft}:
\begin{quote}\label{Zitat:EndzweckalsganzeBestimmungdesMenschen}
  Wesentliche Zwecke sind darum noch nicht die höchsten, deren (bei
  vollkommener systematischer Einheit der Vernunft) nur ein einziger sein kann.
  Daher sind sie entweder der Endzweck, oder subalterne Zwecke, die zu jenem als
  Mittel notwendig gehören. Der erstere ist kein anderer, als die ganze
  Bestimmung des Menschen, und die Philosophie über dieselbe heißt
  Moral.\footnote{\phantomsection\label{Anmerkung:GanzeBestimmung}\cite[B~868]{Kant:KritikderreinenVernunft2003},
  \cite[III: 543.7--12]{Kant:GesammelteWerke1900ff.}.}
\end{quote}
Der Weltbegriff der Philosophie konstituiert eine Hierarchie unter den
philosophischen Wissenschaften, an deren Spitze die Moralphilosophie steht.
Auch bei unserem theoretischen Vernunftgebrauch, der selbst eine Form
zielgerichteten Handelns darstellt, gibt es vielfältige Verbindungen von Zwecken
und Mitteln. Menschen lösen Rechenaufgaben, um Prüfungen zu bestehen, die
Entfernung eines Gewitters oder die Tiefe eines Brunnens zu berechnen. Und oft
dienen unsere Erkenntnisse der Gewinnung weiterer Erkenntnisse, beispielsweise
wenn ein mathematischer Hilfssatz bewiesen wird, der später in den Beweis eines
wichtigen Theorems eingehen soll. \name[Immanuel]{Kant} spricht hier von
\singlequote{\emph{formaler} objektiver
Zweckmäßigkeit}.\footnote{\cite[Vgl.][\S~62]{Kant:KritikderUrteilskraft2009},
\cite[V: 362.6--364.2]{Kant:GesammelteWerke1900ff.}.} Je nachdem, welchen
Handlungen oder welchen weiteren Erkenntnisgewinnen eine Erkenntnis dienen kann,
nennen wir sie \enquote{nützlich} oder auch \enquote{unnütz}. Wir haben gesehen,
dass \name[Immanuel]{Kant} wissenschaftliche Erkenntnisse im Anschluss an
\name[Jean-Jacques]{Rousseau} nicht als Selbstzweck ansieht, sondern sie in den
Dienst der \enquote{Rechte der Menschheit} gestellt wissen
möchte.\footnote{Siehe oben, Anm.
\ref{Anmerkung:Rousseauhatmichzurechtgebracht} auf S.
\pageref{Anmerkung:Rousseauhatmichzurechtgebracht}.} Als \enquote{Endzweck}
bezeichnet \name[Immanuel]{Kant} einen \enquote{Zweck, der keines anderen als
Bedingung seiner Möglichkeit
bedarf}\footnote{\cite[\S~84]{Kant:KritikderUrteilskraft2009}, \cite[V:
434.7--8]{Kant:GesammelteWerke1900ff.}.}, und unterscheidet ihn von letzten Zwecken.

Man kann das Verhältnis der verschiedenen Zweckbegriffe folgendermaßen
illustrieren: Wenn mich jemand fragt, warum ich Mehl siebe, könnte ich
antworten, dass ich einen Teig herstelle. Das Sieben des Mehls ist dann das
Mittel zu einem Zweck, der Herstellung des Teigs. Wenn es nun möglich ist,
weiter zu fragen, warum ich einen Zweck verfolge, dann verfolge ich keinen
Endzweck. Es könnte beispielsweise jemand fragen, warum ich einen Teig herstelle
-- Teigherstellung ist kein Endzweck. Wenn ich auf eine solche Frage antworte,
gebe ich einen höheren Zweck an. Ich sage zum Beispiel: \enquote{Ich backe einen
Obstkuchen.} Und damit gebe ich einen Zweck an, von dem der niedere
Zweck -- die Teigherstellung -- abhängig ist; denn wollte ich keinen Obstkuchen
backen, bräuchte ich auch keinen Teig herstellen. Wenn ich nun einen Zweck
nennen kann, der selbst nicht mehr Mittel zu einem anderen Zweck ist,
so finde ich einen \emph{letzten Zweck}. Der Wunsch, Obstkuchen zu essen, mag
ein solcher letzter Zweck sein. Aber es drückt doch keinen Endzweck aus, denn er
ist davon abhängig, dass ich Obstkuchen essen möchte und mir diesen Zweck
\emph{setze}. Zu einer anderen Zeit habe ich diesen Zweck nicht und es ist auch
gerade jetzt zufällig, dass ich diesen Zweck verfolge. Im Falle eines Endzwecks
muss es sich von selbst verstehen, dass es sich um einen Zweck handelt, den
anzustreben richtig oder gut ist; ein Endzweck ist daher -- wie
\name[Immanuel]{Kant} sagt -- \enquote{in der Ordnung der Zwecke von keiner
anderweitigen Bedingung als bloß seiner Idee
abhängig}\footnote{\cite[\S~84]{Kant:KritikderUrteilskraft2009}, \cite[V:
435.13--14]{Kant:GesammelteWerke1900ff.}.}.


Wir kennen nur einen einzigen Endzweck, so behauptet
\name[Immanuel]{Kant} in der \titel{Kritik der Urteilskraft}: den Menschen als
\emph{Noumenon}. Denn nur von diesem -- dem Menschen als eines moralischen
Wesens -- könne nicht weiter gefragt werden, wozu er da
sei.\footnote{\cite[Vgl.][\S~84]{Kant:KritikderUrteilskraft2009}, \cite[V:
435.15--436.2]{Kant:GesammelteWerke1900ff.}.} Wenn ich auf die Frage nach dem
Grund meines Handelns sage: \enquote{Ich helfe jemandem, der in Not geraten
ist!}, so lässt sich nicht sinnvoll weiterfragen, warum ich das tue.
Jedenfalls könnte (dürfte) ich eine solche Frage als unsinnig zurückweisen und
sagen: Wir brauchen keinen weiteren Grund, um so zu handeln; es handelt sich um
einen Imperativ, der \emph{kategorisch} gebietet und nicht von weiteren Zwecken
abhängig ist. Was Pflicht ist, das ist nach \name[Immanuel]{Kant} ein
Endzweck.\footnote{Zumindest gilt dies dann, wenn derjenige, dem ich helfe,
nicht durch eigenes unmoralisches Verhalten in eine Notsituation geriet, wenn ich dadurch also nicht
sein unmoralisches Verhalten billige und fördere. Helfe ich hingegen einem
rechtmäßig verurteilten Mörder bei der Flucht, so ist die Frage nach einem Warum
weiterhin angebracht. Mein Zweck ist dann aber auch nicht ein Mensch als
\emph{moralisches} Wesen, denn seine Notsituation (die Haft) war gerade
moralisch geboten. Dies ist damit bezeichnet, dass der Mensch als
\emph{Noumenon} Endzweck der Vernunft sei; denn der Ausdruck \enquote{Mensch
als \emph{Noumenon}} bezeichnet ihn insbesondere als moralisches Wesen.
Und auch die \singlequote{Selbstzweck-Formel} des Kategorischen Imperativs betont dies, wenn
es heißt, man solle \singlequote{die Menschheit} in der Person eines jeden (und
nicht etwa einfach jeden Menschen) immer auch als Zweck betrachten
\mkbibparens{\cite[vgl.][BA
66\,f.,]{Kant:GrundlegungzurMetaphysikderSitten1965} \cite[][IV: 429.10--12]{Kant:GesammelteWerke1900ff.}. Trotz dieser ersten
Deutung dieser Formulierung ist diese Redeweise \name[Immanuel]{Kant}s
problematisch, insofern er keine Erläuterung des Ausdrucks \enquote{Menschheit
in der Person von\dots} angibt.}} So bestimmt
die Moralphilosophie den Endzweck unseres Handelns und wird von
\name[Immanuel]{Kant} als die Wissenschaft angeführt, die die Bestimmung des
Menschen untersucht.\footnote{Siehe dazu die Anm.
\ref{Anmerkung:GanzeBestimmung}, S.
\pageref{Anmerkung:GanzeBestimmung}.}

Wird Philosophie nicht nach dem Schul-, sondern nach ihrem Weltbegriff
verstanden, so bezieht sie unsere Erkenntnisse auf \distanz{wesentliche} Zwecke
unserer Vernunft -- das ist die Forderung des Weltbegriffs der Philosophie. Was
aber sind wesentliche Zwecke? Wesentliche Zwecke unserer Vernunft sind erstens
der Endzweck der Vernunft und zweitens diejenigen niederen Zwecke, die als
Mittel zur Umsetzung des Endzwecks unentbehrlich
sind.\footnote{\phantomsection\label{Anmerkung:wesentlicheZwecke}\cite[Vgl.][B
868]{Kant:KritikderreinenVernunft2003}, \cite[III:
543.7--10]{Kant:GesammelteWerke1900ff.}: \enquote{Wesentliche Zwecke sind
\punkt{} entweder der Endzweck, oder subalterne Zwecke, die zu jenem als Mittel
notwendig gehören.}} \name[Immanuel]{Kant} setzt voraus, dass die Vernunft zu vollkommener
systematischer Einheit fähig ist.\footnote{Siehe auch
\cite[A~xiii]{Kant:KritikderreinenVernunft2003}, \cite[IV:
10.11--16]{Kant:GesammelteWerke1900ff.}: \enquote{In der Tat ist auch reine
Vernunft eine so vollkommene Einheit: daß, wenn das Prinzip derselben auch nur
zu einer einzigen aller der Fragen, die ihr durch ihre eigene Natur aufgegeben
sind, unzureichend wäre, man dieses immerhin nur wegwerfen könnte, weil es
alsdenn auch keiner der übrigen mit völliger Zuverlässigkeit gewachsen sein
würde.}} Daher kann er fordern, dass ein \emph{einziger} höchster Zweck
angegeben werde. Dieser höchste Zweck sei die Bestimmung des Menschen und
die Wissenschaft, die ihn erforscht, sei die Moralphilosophie. Die Bestimmung
des Menschen muss einen Zweck anzeigen, der von keiner Bedingung abhängig ist
und sich von selbst versteht. Sie ist erstens nicht weiter begründungsbedürftig und stellt zweitens den Dreh- und
Angelpunkt aller unserer Erkenntnisse dar. Von ihrer Beziehung auf die
Bestimmung des Menschen erhalten diese erst ihren je eigenen inneren
Wert.\footnote{\cite[Vgl.][A 23]{Kant:ImmanuelKantsLogik1977}, \cite[IX:
23.33--24.2]{Kant:GesammelteWerke1900ff.}. Von dem Weltbegriffe aus habe die
Philosophie sogar einen absoluten Wert, d.\,i.\ Würde. Daneben können
Wissenschaften (nach ihrem Schulbegriff) natürlich auch einen \distanz{äußeren},
monetären \emph{Preis} haben. Nach dem Preis bewertet, liegt die
Moralphilosophie, der der höchste innere Wert zukommt, freilich weit abgeschlagen. 
Siehe auch die Parallele in \cite[][BA~77]{Kant:GrundlegungzurMetaphysikderSitten1965},
\cite[][IV: 434.31--435.4]{Kant:GesammelteWerke1900ff.}}
Wenn nun aber die Moral die Wissenschaft vom Endzweck der menschlichen
Vernunft ist, warum verweist \name[Immanuel]{Kant} dann auf die Bestimmung des Menschen
und nicht auf die Pflichten des Menschen?

\subsection{Aufklärung und
Anthropologie}\label{subsection:AufklaerungundAnthropologie} Es handelt sich bei
der Bestimmung des Menschen um ein Thema der Aufklärungstheologie des 18.~Jahrhunderts, was andeutet, dass dies die richtige Fährte ist, die Relevanz
der Religion für \name[Immanuel]{Kant}s Aufklärungsprogrammatik zu klären. Mehrfach wurde
behauptet, dass es sich um einen der zentralen Topoi der zweiten Hälfte der
Aufklärung
handelt.\footnote{\Revision{Zum Begriff der Bestimmung des Menschen siehe
  \cite{Macor:DieBestimmungdesMenschen1748--18002013}, und die dort
  ausführlich zitierte und besprochene
  Literatur.} Siehe außerdem
\cite[][476]{Zoeller:DieBestimmungderBestimmungdesMenschenbeiMendelssohnundKant2001}:
\enquote{Wohl kaum eine Wendung oder Formel dürfte so geeignet sein, das
philosophische Projekt der deutschen Spätaufklärung in der zweiten Hälfte des
18. Jahrhunderts zu bezeichnen wie die von der \enquote{Bestimmung des
Menschen}.} Auch Reinhard \name[Reinhard]{Brandt} sieht in
\authorcite{Spalding:BetrachtungueberdieBestimmungdesMenschen1749}s Buch
\enquote{die Programmschrift der zweiten Phase der deutschen Aufklärung}
\parencite[][61]{Brandt:DieBestimmungdesMenschenbeiKant2007}. Zur
Wirkungsgeschichte des Begriffs \enquote{Bestimmung des Menschen} siehe
\cite{DAlessandro:DieWiederkehreinesLeitworts1999}.
Vgl.\ außerdem \cite{Hinske:EineantikeKatechismusfrage1999},
\cite[][179--211]{Tippmann:DieBestimmungdesMenschenbeiJohannJoachimSpalding2011},
und allgemein zur Vorstellung einer Bestimmung des Menschen und ihrer
Erkennbarkeit mittels natürlicher Vernunft die Textsammlung in
\cite[][47--103]{Ciafardone:LIlluminismotedesco1983} (Kapitel über
\enquote{l'uomo e la sua destinazione}).
\Revision{\textcite{Macor:DieBestimmungdesMenschen1748--18002013}
  bespricht \name[Immanuel]{Kant}s Behandlung dieser Thematik und ihre
  Auswirkungen auf den Seiten 199--310.} Zur Wirkung
\authorcite{Spalding:BetrachtungueberdieBestimmungdesMenschen1749}s
auf \name[Immanuel]{Kant} siehe außerdem
 \cite[][189--198]{Tippmann:DieBestimmungdesMenschenbeiJohannJoachimSpalding2011},
 und die dort angeführte Literatur sowie
 \cite[][7--138]{Brandt:DieBestimmungdesMenschenbeiKant2007}.}
Mit der Bestimmung des Menschen als Thema der Moral und Endzweck der Vernunft
gibt \name[Immanuel]{Kant} aussagekräftige Auskunft über seine Selbstverortung innerhalb
der Aufklärung: Sie reflektiert die dominante Stellung der Moral
gegenüber der Religion und die \index{Kant, Immanuel}kantische Auffassung des
Verhältnisses von Vernunft und Glaube.


Der Ausdruck \enquote{Bestimmung des Menschen} war im 18.\ Jahrhundert bekannt
durch eine Schrift des Aufklärungstheologen und Predigers Johann Joachim
\authorcite{Spalding:BetrachtungueberdieBestimmungdesMenschen1749}, dessen auf
1748 datierte\footnote{Bei der auf dem Titelblatt angegebenen Jahreszahl handelt
es sich wohl um eine Vordatierung.
\cite[Vgl.][\pno~17,
Fn.~50]{Schwaiger:ZurFragenachdenQuellenvonSpaldingsemphBestimmungdesMenschen1999}.}
\titel{Betrachtung über die Bestimmung des Menschen} etliche Auflagen erlebte
und bis in den Deutschen Idealismus hinein breit rezipiert
wurde.\footnote{\Revision{Damit soll nicht gesagt sein, dass
    \authorcite{Spalding:BetrachtungueberdieBestimmungdesMenschen1749}
    den Begriff neu in die philosophische Sprache eingeführt habe, er
    hat aber sicherlich die bekannteste Schrift publiziert, welche die
    Bestimmung in ihrem Titel trägt und zum Aushängeschild dieser
    Thematik wurde. Siehe hierzu insbesondere die beeindruckende Studie
  von \authorfullcite{Macor:DieBestimmungdesMenschen1748--18002013}, die insgesamt 40 Auflagen von Spaldings Schrift
  zählt \parencite[vgl.][33]{Macor:DieBestimmungdesMenschen1748--18002013}
  und die Ursprünge des Begriffs der Bestimmung des Menschen bis in
  das 16. Jahrhundert
  zurückverfolgt \parencite[vgl.][36--109]{Macor:DieBestimmungdesMenschen1748--18002013}. \enquote{Weit
  davon entfernt, eine Neuschöpfung des 18. Jahrhunderts zu sein, läßt
sich das Wort \singlequote{Bestimmung} schon im 16. Jahrhundert und in
einigen Fällen bereits im letzten Jahrzehnt des 15. Jahrhunderts
verzeichnen} \parencite[vgl.][37]{Macor:DieBestimmungdesMenschen1748--18002013}.}
\cite[Für
  das Hineinreichen der Wirkungsgeschichte in die Philosophie des
  Deutschen
Idealismus steht][]{Fichte:DieBestimmungdesMenschen1800}. Siehe hierzu auch
\cite{Zoeller:BestimmungzurSelbstbestimmung1995} sowie
\cite[][477--482]{Zoeller:DieBestimmungderBestimmungdesMenschenbeiMendelssohnundKant2001}.}
\authorcite{Spalding:BetrachtungueberdieBestimmungdesMenschen1749} versucht in
dieser Schrift, eine Erkenntnis dessen, was vernünftige Ziele menschlichen
Lebens ist, ohne Berufung auf Offenbarungswissen oder andere übernatürliche
Erkenntnisquellen zu
etablieren.\footnote{\cite[Vgl.][3]{Spalding:BetrachtungueberdieBestimmungdesMenschen1749}.
Die Unabhängigkeit seiner Argumentation von Offenbarung zwingt ihn in der
dritten Auflage dazu, in einem Anhang den Vorwurf des Deismus abzuwehren, der
die natürliche Theologie gegen das offenbarte Christentum in Stellung bringe,
\cite[vgl.][26--32]{Spalding:BetrachtungueberdieBestimmungdesMenschen1749}.
Zu den Grundlagen und Ursprüngen von \authorcite{Spalding:BetrachtungueberdieBestimmungdesMenschen1749}s Abhandlung vgl.
\cite{Schwaiger:ZurFragenachdenQuellenvonSpaldingsemphBestimmungdesMenschen1999},
wo \authorcite{Spalding:BetrachtungueberdieBestimmungdesMenschen1749} als
methodischer Anhänger \authorcite{Wolff:Psychologiaempirica1968}s rekonstruiert wird, der den
Anstoß zu seinem \enquote{Appell an das Selbstdenken in Sachen Bestimmung des
Menschen}
\parencite[][13]{Schwaiger:ZurFragenachdenQuellenvonSpaldingsemphBestimmungdesMenschen1999}
und insbesondere das Bekenntnis zu der Lehre vom moralischen Gefühl
\parencite[vgl.][15]{Schwaiger:ZurFragenachdenQuellenvonSpaldingsemphBestimmungdesMenschen1999}
der Lektüre \name[Anthony Ashley-Cooper]{Shaftesbury}s entnimmt.} So schreibt er
zu Beginn des Buches:
\begin{quote}
  Ich sehe, daß ich die kurze Zeit, die ich auf der Welt zu leben habe, nach
  ganz verschiedenen Grundregeln zubringen kann, deren Wehrt und Folgen daher
  auch unmöglich einerley seyn können. Da ich nun unläugbar eine Fähigkeit zu
  wählen, und in meinen Entschliessungen eines dem andern vorzuziehen an mir
  finde, so muß ich auch hiebey nicht blindlings zufahren, sondern vorher nach
  meinem letzten Vermögen auszumachen suchen, welcher Weg der sicherste,
  anständigste und vortheilhafteste sey. \punkt{} Es ist doch einmal der Mühe
  wehrt, zu wissen, warum ich da bin, und was ich vernünftiger Weise seyn
  soll.\footcite[][4]{Spalding:BetrachtungueberdieBestimmungdesMenschen1749}
\end{quote}
Der Titel \enquote{Bestimmung des Menschen} verweist also zunächst auf die Sorge
um das gute, \distanz{glückende} Leben, wie sie genuiner Bestandteil der
Religion ist.\footnote{\Revision{Gewiss eröffnet
\authorcite{Spalding:BetrachtungueberdieBestimmungdesMenschen1749} kein
neues Thema, sondern greift eines auf, welches schon bei früheren
Autoren unter diesem und anderen Titeln
behandelt wird \parencite[siehe dazu
ausführlich][36--139]{Macor:DieBestimmungdesMenschen1748--18002013}.} \textcite[vgl.][47--103]{Ciafardone:LIlluminismotedesco1983}
lässt in seinem Kapitel über \enquote{l'uomo e la sua destinazione} nicht
\authorcite{Spalding:BetrachtungueberdieBestimmungdesMenschen1749} selbst zu
Wort kommen, sondern hebt die Präsenz des Themas bei \name[Gottfried
Wilhelm]{Leibniz}, \name[Christian]{Thomasius}, \authorcite{Wolff:Psychologiaempirica1968},
\authorcite{Crusius:Anweisungvernuenftigzuleben1744}, \name[Moses]{Mendelssohn}
\name[Immanuel]{Kant}, \authorcite{Lessing:EineDuplik1897} und
\name[Friedrich]{Schiller} hervor.} Seine Abhandlung ist dabei aber stets auch
ein \enquote{Appell an das Selbstdenken in Sachen Bestimmung des
Menschen}\footnote{\cite[][13]{Schwaiger:ZurFragenachdenQuellenvonSpaldingsemphBestimmungdesMenschen1999}.}
und damit ein Aufruf, der jedem Menschen eigenen Freiheit in der
Lebensgestaltung aufgrund jeweils eigener Einsicht gerecht zu werden. Und damit
nähern wir uns erkennbar einer Antwort auf die Frage nach dem Grund der Relevanz
der Religion als Thema der Aufklärung: Mündig zu sein bedeutet primär,
Verantwortung für die eigene Lebensgestaltung zu übernehmen.
 
 
\name[Immanuel]{Kant} schreibt keinen eigenständigen Text über die Bestimmung des
Menschen, verwendet diesen Ausdruck aber an etlichen Stellen. Häufig geschieht dies eher
beiläufig, wie an der zitierten Stelle in der \titel{Kritik der reinen
Vernunft}, wo sich weder eine Erläuterung des Begriffs findet noch die Verbindung zur
Moralphilosophie verdeutlicht
wird.\Revision{\footnote{\Revision{\textcite[vgl.][31]{Macor:DieBestimmungdesMenschen1748--18002013}
      behauptet (allerdings ohne eine Begründung zu liefern), dass
      alle drei Kritiken \name[Immanuel]{Kant}s die Bestimmung des 
      Menschen zu ihrem (im Titel allerdings nicht 
  genannten) Thema hätten.}}} Etwas größeren Raum nehmen Äußerungen über
die \enquote{Bestimmung des Menschen} oder die \enquote{Bestimmung des
Menschengeschlechts}\footnote{\cite[][B
330]{Kant:AnthropologieinpragmatischerHinsicht1977}, \cite[][VII:
331.28]{Kant:GesammelteWerke1900ff.}. Darin kommt zum Ausdruck, dass eine Bestimmung des Menschen eine Bestimmung ist, die ihm \emph{als Mensch} zukommt
(und nicht als dieses oder jenes Individuum). Dass \name[Immanuel]{Kant} in
gleicher Bedeutung von einer Bestimmung \emph{des Menschen} und einer Bestimmung
\emph{des Menschengeschlechts} sprechen kann, liegt aber sicherlich auch in
seiner Überzeugung begründet, dass der Mensch seiner Bestimmung nur als Gattung gerecht
werden kann; \cite[vgl.][A
388]{Kant:IdeezueinerallgemeinenGeschichteinweltbuergerlicherAbsicht1977},
\cite[][VIII: 18.29--32]{Kant:GesammelteWerke1900ff.}. In der \titel{Pädagogik}
wiederum wird die Menschheit selbst als Bestimmung des Menschen genannt;
\cite[vgl.][A 3]{Kant:UeberPaedagogik1977}, \cite[][IX:
442.3--4]{Kant:GesammelteWerke1900ff.}.} im zweiten Teil der
\titel{Anthropologie in pragmatischer Hinsicht} ein, wo es um den
\enquote{Charakter} des Menschen geht. Der Charakter eines Lebewesens sei
\enquote{das, woraus sich seine Bestimmung zum voraus erkennen
läßt.}\footnote{\cite[B~326]{Kant:AnthropologieinpragmatischerHinsicht1977},
\cite[VII: 329.14--15]{Kant:GesammelteWerke1900ff.}.} Somit geht es der
\enquote{Charakteristik} -- so heißt der zweite Teil der \titel{Anthropologie}
-- wenigstens indirekt um die Bestimmung des Menschen (und zwar als Person,
Geschlecht, Volk und
Gattung)\footnote{\cite[Vgl.][B~253]{Kant:AnthropologieinpragmatischerHinsicht1977},
\cite[VII: 285.1--3]{Kant:GesammelteWerke1900ff.}.}, was eine
philosophie-historisch interessante Entwicklung anspricht:
Im Laufe des 18.\ Jahrhunderts entwickelt sich aus dem Topos einer Bestimmung
des Menschen die philosophische Disziplin der
Anthropologie\footnote{\cite[Vgl.][]{DAlessandro:DieWiederkehreinesLeitworts1999},
sowie
\cite[][138--140]{Tippmann:DieBestimmungdesMenschenbeiJohannJoachimSpalding2011}.}.
Genauer müsste man sagen:
\authorcite{Spalding:BetrachtungueberdieBestimmungdesMenschen1749}s Abhandlung
ist eine von zwei wichtigen Quellen der neu entstehenden Anthropologie; die
andere Quelle ist die mit \authorcite{Wolff:Psychologiaempirica1968} anhebende
empirische Psychologie.\footnote{Siehe dazu
\cite{Bae:DieEntstehungderKantischenAnthropologieundihreBeziehungzurempirischenPsychologiederWolffschenSchule1994},
sowie
\cite{Hinske:WolffsempirischePsychologieundKantspragmatischeAnthropologie1999}.}

\name[Immanuel]{Kant} hielt beginnend mit dem Wintersemester 1772/1773 in jedem
Winterhalbjahr eine Vorlesung über Anthropologie, die sich an einen breiteren
Hörerkreis wendete. Wie im Falle der physischen Geographie handelt es sich um
Wissen, das jeden angehe, was darauf hindeutet, dass es aufklärungsrelevant ist.
Aus diesen Vorlesungen ging dann Ende der 1790er Jahre die Schrift
\titel{Anthropologie in pragmatischer Hinsicht} hervor.\footnote{Zur Entwicklung
der Anthropologie bei \name[Immanuel]{Kant} siehe
\cite{Stark:HistoricalNotesandInterpretiveQuestionsaboutKantsLecturesonAnthropology2003}.
\name[Immanuel]{Kant}s Anthropologie ist in den letzten Jahren immer st{"a}rker
zum Thema von Publikationen geworden, siehe etwa
\cite{Zoeller:DieBestimmungderBestimmungdesMenschenbeiMendelssohnundKant2001},
\cite{Zammito:KantHerderandtheBirthofAnthropology2002},
\cite{Brandt:TheGuidingIdeaofKantsAnthropologyandtheVocationoftheHumanBeing2003},
\cite{Wood:KantandtheProblemofHumanNature2003},
\cite{Wilson:KantsPragmaticAnthropology2006},
\cite{Cohen:KantandtheHumanSciences2009}, sowie
\cite{Sturm:KantunddieWissenschaftenvomMenschen2009}.}
Ursprünglich basierten diese Vorlesungen über Anthropologie auf den Abschnitten
über empirische Psychologie in
\authorcite{Baumgarten:Metaphysica---Metaphysik2011}s \titel{Metaphysica}.
Dennoch ist \name[Immanuel]{Kant}s Anthropologie nicht einfach eine Etappe in
der Entwicklung der empirischen Psychologie. Sie schließt zwar an diese an, aber
sie wandelt deren Erkenntnisinteresse merklich ab, gerade indem sie ihren
Ursprung in der empirischen Psychologie mit Fragestellungen bezüglich der
Bestimmung des Menschen
verbindet.\footcite[Vgl.][]{Brandt:TheGuidingIdeaofKantsAnthropologyandtheVocationoftheHumanBeing2003}
Und gerade dies macht die Aufklärungsrelevanz solchen Wissens aus. Eine solche
Verschmelzung ursprünglich getrennter Disziplinen lässt zunächst die Frage
aufkommen, was das Thema der Anthropologie ist, wenn sie überhaupt ein
einheitliches Thema hat und kein bloßes Konglomerat -- oder wie
\name[Immanuel]{Kant} sagen würde: Aggregat -- von Kenntnissen ist. Lautet die
leitende Frage \enquote{Was ist der Mensch?}, wie es in der Jäsche-Logik
heißt?\footnote{\cite[Vgl.][A~25]{Kant:ImmanuelKantsLogik1977},
\cite[IX: 25.1-10]{Kant:GesammelteWerke1900ff.}: \enquote{Das Feld der
Philosophie in dieser weltbürgerlichen Bedeutung läßt sich auf folgende Fragen bringen:
  \begin{nummerierung}
\item Was kann ich wissen?
\item Was soll ich thun?
\item Was darf ich hoffen?
\item Was ist der Mensch?
  \end{nummerierung}
  Die erste Frage beantwortet die Metaphysik, die zweite die Moral,
  die dritte die Religion und die vierte die Anthropologie. Im
  Grunde könnte man aber alles dieses zur Anthropologie rechnen, weil sich
  die drei ersten Fragen auf die letzte beziehen.} Dieser Deutung schließt sich
  z.\,B.\ \textcite{Alpheus:WasistderMensch1968} an.} Reinhardt \name[Reinhard]{Brandt}
  weist den Vorschlag, die Frage \enquote{Was ist der Mensch?} als Leitfrage der
  Anthropologie in pragmatischer Hinsicht anzusehen, explizit zurück und nennt
  stattdessen die \emph{Bestimmung} des Menschen als deren eigentliches
  Thema.\footcite[Vgl.][86-7]{Brandt:TheGuidingIdeaofKantsAnthropologyandtheVocationoftheHumanBeing2003}
  Dagegen nimmt \authorfullcite{Wood:KantandtheProblemofHumanNature2003} an,
  dass es der Anthropologie um die
\emph{Natur} des Menschen
gehe.\footcite[Vgl.][passim]{Wood:KantandtheProblemofHumanNature2003} Geht es
der Anthropologie also um den Menschen, um die Natur des Menschen oder um die
Bestimmung des Menschen? Oder meint dies vielleicht dasselbe? Was ist
\distanz{die Natur} oder \distanz{die Bestimmung} von etwas? Und um welche Art
von Fragen handelt es sich? Sind es empirische Fragen an eine Naturwissenschaft
vom Menschen? Oder handelt es sich um philosophische, vielleicht sogar
\singlequote{metaphysische} Fragen?

Solche Fragen sind auch deshalb von Bedeutung, weil es gute Gründe gibt, einer
philosophischen Disziplin, die sich mit der Natur oder der Bestimmung
des Menschen befasst, skeptisch gegenüber zu stehen. So wirft \authorcite{Wood:KantandtheProblemofHumanNature2003}
die Frage auf, ob wir heute nicht davon ausgehen sollten, dass es eine allgemeine und
bei allen gleiche Natur des Menschen gar nicht gebe (und dass sich dies auch und
gerade aus \name[Immanuel]{Kant}s aufklärerischer Perspektive einsehen
lasse).\footnote{\cite[Vgl.][38--39]{Wood:KantandtheProblemofHumanNature2003}:
\enquote{Kant was reluctant to address the most fundamental question. As we
shall see later, this reluctance anticipates some of the issues (about human
freedom and about the historical variability of human ways of life) that have
led others since Kant's time to declare that there is no such thing as
\enquote{human nature} uniformly and equally determining all human beings at all
times and places.}} Zudem scheint \name[Immanuel]{Kant}s \titel{Anthropologie
in pragmatischer Hinsicht} ein Sammelplatz von sexistischen, ethnischen und
weiteren Vorurteilen zu sein, den als zentrales Element von Aufklärung zu
bezeichnen so manchem Leser Magenschmerzen verursachen
dürfte.\footnote{So wirft er allen Angehörigen nichteuropäischer Nationen in
einer Nebenbemerkung Borniertheit -- \enquote{die Eingeschränktheit
aller übrigen [Völker; A.\,G.] an Geist}
\mkbibparens{\cite[][A 300]{Kant:AnthropologieinpragmatischerHinsicht1977},
\cite[][VII: 312.28--29]{Kant:GesammelteWerke1900ff.}} -- vor und stellt dann
allein Franzosen, Engländer und Deutsche als kosmopolitisch heraus:
\enquote{Die Eingeschränktheit des Geistes aller Völker, welche die
uninteressierte Neubegierde nicht anwandelt, die Außenwelt mit eigenen Augen
kennen zu lernen, noch weniger sich dahin (als Weltbürger) zu verpflanzen, ist
etwas Charakteristisches an denselben, wodurch sich Franzosen, Engländer und
Deutsche vor anderen vorteilhaft unterscheiden}
\mkbibparens{\cite[][A 306]{Kant:AnthropologieinpragmatischerHinsicht1977},
\cite[][VII: 316.33--37]{Kant:GesammelteWerke1900ff.}}. Zum Charakter des
Geschlechts bemerkt er u.\,a.: \enquote{Der Mann ist leicht zu erforschen, die
Frau verrät ihr Geheimnis nicht; obgleich anderer ihres (wegen ihrer
Redseligkeit) schlecht bei ihr verwahrt ist}
\mkbibparens{\cite[][A 285\,f.,]{Kant:AnthropologieinpragmatischerHinsicht1977}
\cite[][VII: 303.35--304.2]{Kant:GesammelteWerke1900ff.}}. Die Reihe an
Beispielen ließe sich fast beliebig fortsetzen.
Vgl.~\cite[271]{Boehme:AnthropologieinpragmatischerHinsicht1985}, sowie
\cite[79]{Louden:TheSecondPartofMorals2003}, und
\cite[173]{Schmidt:KantsTranscendentalEmpiricalPragmaticandMoralAnthropology2007}.}
\name[Immanuel]{Kant}s Behauptung, es läge den einzelnen Nationen eine je
gemeinsame Abstammung und ein erblicher Nationalcharakter
zugrunde,\footnote{\cite[Vgl.][A~297--301]{Kant:AnthropologieinpragmatischerHinsicht1977},
\cite[VII: 311.6--313.16]{Kant:GesammelteWerke1900ff.}.} ist ebenso Ausdruck
vorurteilsbehafteten Denkens wie seine Ausführungen über den jeweiligen
Charakter von Frauen und Männern. Und der Umstand, dass er die Erläuterung des Charakters der verschiedenen
\distanz{Menschenrassen} ausspart, lässt sich mit Gernot \name[Gernot]{Böhme}
wohl nur als großes Glück
bezeichnen.\footnote{\cite[Vgl.][271]{Boehme:AnthropologieinpragmatischerHinsicht1985}.
Auch \authorfullcite{Doerflinger:DieEinheitderMenschheitalsTiergattung2001},
der \name[Immanuel]{Kant}s Rassebegriff in der physischen Anthropologie sehr
aufgeschlossen gegenüber steht, konzidiert, dass \name[Immanuel]{Kant} auch in
großem Umfang Stereotypen und Vorurteile seine Zeit wiederhole
\parencite[vgl.][349]{Doerflinger:DieEinheitderMenschheitalsTiergattung2001}.}
\name[Immanuel]{Kant}s Anthropologie ist darüber hinaus in ihren mutigen
Verallgemeinerungen in weiten Teilen ein methodisch fragwürdiges Unternehmen, das in
der heutigen Wissenschaftslandschaft keinen Platz mehr hätte und von methodisch
disziplinierten empirischen Wissenschaften abgelöst
wurde.\footcite[Vgl.][167-173]{Schmidt:KantsTranscendentalEmpiricalPragmaticandMoralAnthropology2007}
Dazu trägt bei, dass viele von \name[Immanuel]{Kant}s Grundannahmen über
Verankerungen von Eigenschaften und Verhaltensweisen in der jeweiligen
(biologischen) Natur des Menschen inzwischen der \enquote{soziologischen
Aufklärung} erlegen sind.\footcite[Vgl.][272]{Boehme:AnthropologieinpragmatischerHinsicht1985}

Die Lage verschlechtert sich weiter, wenn und insofern es nicht nur um eine
\emph{Natur}, sondern um eine \emph{Bestimmung} des Menschen geht. Denn diese verweist nicht
bloß auf eine Definition oder das \distanz{Wesen} des Menschen, sondern bezieht
sich auf ein Wozu, einen Zweck oder ein Ziel menschlichen
Lebens.\footcite[Vgl.][86--87,
93]{Brandt:TheGuidingIdeaofKantsAnthropologyandtheVocationoftheHumanBeing2003}
Das Projekt einer pragmatischen Anthropologie stützt sich in Übereinstimmung mit einer solchen Interpretation durchgängig auf teleologische
Betrachtungen.\footnote{\cite[Vgl.][65--67]{Cohen:KantandtheHumanSciences2009}.}
Und teleologische Behauptungen, sofern sie nicht die Handlungen von Menschen,
sondern Produkte der Natur betreffen, sind heute durchgängig in Verruf geraten.
Dem aufgeklärten Zeitgeist gänzlich indiskutabel erscheint -- um ein eindeutiges
Beispiel aus der Tugendlehre anzuführen -- \name[Immanuel]{Kant}s Versuch, eine Pflicht
zur sexuellen Enthaltsamkeit über die Annahme eines Naturzwecks der
Fortpflanzung und die Rede von einer \enquote{Begierde wider den Zweck der
Natur} zu motivieren.\footnote{\cite[][\S~7]{Kant:DieMetaphysikderSitten1977Tugendlehre}, \cite[VI:
424.12--426.32]{Kant:GesammelteWerke1900ff.}.}

\phantomsection\label{Abschnitt:TwoConceptsofLiberty}
Es geht hier letztlich um die Frage, ob eine Philosophie, welche die
\distanz{Bestimmung des Menschen} zu ihrem Ausgangspunkt macht, mit einem
liberalen Freiheitsverständnis vereinbar ist,
welches die je individuelle Souveränität des Einzelnen über die Ausgestaltung
seiner Freiheit als Ausgangspunkt berücksichtigt, oder notwendig ein
\singlequote{positives} Freiheitsverständnis verlangt, bei dem von vornherein
festgelegt ist, was als vernünftige (und schützenswerte) Ausübung der Freiheit
zählt.\footnote{Die Unterscheidung negativer und positiver Freiheitsbegriff
stammt meines Wissens nach von
\authorfullcite{Berlin:TheProperStudyofMankind1997}. Negative Freiheit
orientiere sich an der Frage \enquote{What is the area within which the subject
-- a person or group of persons -- is or should be left to do or be what he is able to do or
be, without interference by other persons?}
\parencite[][194]{Berlin:TheProperStudyofMankind1997}. Positive Freiheit
hingegen orientiere sich an der Leitfrage \enquote{What, or who, is the source
of control or interference that can determine someone to do, or be, this
rather than that?} \parencite[][194]{Berlin:TheProperStudyofMankind1997} Frei im
negativen Sinne ist jeder, der in seinen Handlungen nicht von anderen Menschen
absichtlich behindert oder gelenkt wird, er kann seine eigenen Entscheidungen
treffen und ausführen, ganz gleich welchen Ursprungs diese Entscheidungen sind.
Der Vertreter positiver Freiheit  entwickelt eine anspruchsvolle Theorie davon,
was es heißt, sein eigener Herr zu sein. Wer frei ist und frei handelt folgt
nach dieser Auffassung einer bestimmten, als Autorität angesehenen Instanz.}
Verkehrt sich eine Berufung auf Freiheit und Mündigkeit nicht selbst in ihr
Gegenteil, wenn sie zugleich sagt, was Ausdruck der Freiheit ist und was nicht,
den Begriff der Freiheit also mit positiven Inhalten
füllt?\footnote{\cite[Vgl.][191-242]{Berlin:TheProperStudyofMankind1997}. Eine
Entgegnung findet sich bei Charles
\textcite[vgl.][]{Taylor:WhatsWrongWithNegativeLiberty2005}.} Ist die Teleologie
hinter der Bestimmung des Menschen mit der Aufklärung vereinbar?
\begin{comment}
Wir
sollten genau hinschauen, wenn \name[Immanuel]{Kant} glaubt, eine Natur
oder gar Bestimmung des Menschen aufweisen zu können. Möglicherweise handelt es
sich bloß um halbherzige Säkularisierungen religiöser Dogmen, die eine auf
Freiheit ausgerichtete Aufklärung besser ganz eliminieren sollte. Sobald man aber
versucht, von teleologischen Momenten innerhalb der Anthropologie abzusehen,
erhält man eine (veraltete) Etappe in der Entwicklung der empirischen
Psychologie, die von der Anknüpfung an die Bestimmung des Menschen entlastet
ist, damit aber auch ihre Relevanz für die Aufklärungsphilosophie
verliert.\end{comment}

Können wir trotz dieser Bedenken für einen liberalen Aufklärungsbegriff von dem
Grundgedanken einer pragmatischen Anthropologie lernen? Was erforscht eine
philosophische Anthropologie? Was ist ihr Zweck? (Dieser Zweck muss nach allem,
was wir bisher wissen, mit dem Endzweck der Vernunft -- der Moral oder dem
Menschen als \emph{Noumenon} -- in Zusammenhang stehen.) Als Anthropologie ist
ihr Thema zunächst der Mensch.
Aber in welcher Hinsicht und mit welchen Mitteln soll der Mensch untersucht
werden? Schließlich gibt es viele Möglichkeiten, den Menschen zum Thema zu
machen, etwa in der Biologie, der Psychologie oder der Moralphilosophie. Und
entsprechend viele \distanz{Anthropologien} sind denkbar.


\authorfullcite{Schmidt:KantsTranscendentalEmpiricalPragmaticandMoralAnthropology2007}
nennt vier verschiedene Hinsichten (transzendental, empirisch, moralisch und
pragmatisch), in denen der Mensch Thema anthropologischer Untersuchungen sein
könne und die sich in der \titel{Anthropologie in pragmatischer Hinsicht}
fänden.\footnote{\cite[Vgl.][]{Schmidt:KantsTranscendentalEmpiricalPragmaticandMoralAnthropology2007}.
Vor allem Reinhard \name[Reinhard]{Brandt} wendet sich stets gegen eine
Identifizierung dieser verschiedenen Hinsichten. \cite[Vgl.
z.\,B.][92]{Brandt:KritischerKommentarzuKantsenquoteAnthropologieinpragmatischerHinsicht1999}.}
\name[Immanuel]{Kant} nennt die relevante Hinsicht bereits im Titel des Buches \enquote{pragmatisch}. Er erläutert die
pragmatische Hinsicht 1775 in einer Vorlesungsankündigung, die nicht die Anthropologie, sondern die
physische Geographie betrifft, und die auch deswegen interessant ist, weil hier
der Begriff der Bestimmung vorkommt:
\begin{quote}
  Die physische Geographie, die ich hiedurch ankündige, gehört zu einer Idee,
  welche ich mir von einem nützlichen akademischen Unterricht mache, den ich:
  die Vorübung in der \ori{Kenntnis der Welt} nennen kann. Diese Weltkenntnis
  ist es, welche dazu dient, allen sonst erworbenen Wissenschaften und
  Geschicklichkeiten das \ori{Pragmatische} zu verschaffen, dadurch sie nicht
  bloß vor die \ori{Schule} sondern vor das \ori{Leben} brauchbar werden, und
  wodurch der fertig gewordene Lehrling auf den Schauplatz seiner Bestimmung
  nämlich in die \ori{Welt} eingeführet
  wird.\footnote{\cite[A~12]{Kant:VondenverschiedenenRassenderMenschen1977},
  \cite[II: 443.12--19]{Kant:GesammelteWerke1900ff.}.}
\end{quote}
Physische Geographie und pragmatische Anthropologie sind die beiden Felder der
Weltkenntnis.\footnote{\enquote{Weltkenntnis setzt sich zusammen aus den
Erfahrungen mit der Natur und dem Umgang mit Menschen. Die Erkenntnis der
Natur ist Aufgabe und Gegenstand der physischen Geographie, die Kenntnis des
Menschen lehrt die Anthropologie -- und das Leben selbst}
\parencite[][185]{Boehr:PhilosophiefuerdieWelt2003}.} Insofern eine
Anthropologie in pragmatischer Hinsicht abgefasst ist, hat sie also tatsächlich
die Bestimmung des Menschen zum Thema. Dies reflektiert zunächst die auf
Integration in die Praxis orientierte Wissenschaftsauffassung:
Wissenschaftliches Erkennen ist kein Selbstzweck, sondern hat seinen Zweck und
sein Ziel immer in den Bedürfnissen und Interessen der Menschen. Eine auf ihre
Tauglichkeit in Handlungen und Praxisformen hin orientierte Erkenntnis nennt
\name[Immanuel]{Kant} \enquote{Weltkenntnis}. Und diese Weltkenntnis
korrespondiert dem Welt\emph{begriff}, der ebenso auf die Nützlichkeit oder
Brauchbarkeit des Wissens
verweist.\footnote{\cite[Vgl.][A~23]{Kant:ImmanuelKantsLogik1977}, \cite[IX:
24.6--8]{Kant:GesammelteWerke1900ff.}: \enquote{In dieser scholastischen
Bedeutung des Worts geht Philosophie nur auf Geschicklichkeit; in Beziehung auf
den Weltbegriff dagegen auf die Nützlichkeit.}}

Es war von Anfang an ein Charakteristikum der deutschen
Aufklärung, die handlungsorientierende Funktion und den Praxisbezug des Wissens
herauszustellen.\footnote{\cite[Vgl.][xviii]{Berndt:PraxisundProgramm2012}:
\enquote{Im aktivistischen Programm der Aufklärer ist die Theorie vielmehr
zugleich ein Teil der Praxis, denn die Theorie zielt hier letztlich immer auf
die praktische Umgestaltung der Welt und dient als deren Instrument.
Aufklärerische Theorie wird nicht um ihrer selbst willen betrieben, sondern als
ein – dirigierender, reflektierender, kontrollierender – Teil der Praxis
begriffen, so dass man von einem Re-Entry sprechen kann.}}
Christian \name[Christian]{Thomasius} hat die Ausrichtung des Wissens auf seinen
außeruniversitären Nutzen und seine Anwendbarkeit am Hofe und in der Welt in
seiner \titel{Introductio ad philosophiam aulicam} schon im Titel zum Programm
werden lassen.\footnote{Siehe zu Grundausrichtung der Philosophie bei
\name[Christian]{Thomasius}
\cite[][]{Schneiders:300JahreAufklaerunginDeutschland1989},
sowie
\cite{Ciafardone:UeberdasPrimatderpraktischenVernunftvordertheoretischenbeiThomasiusundCrusiusmitBeziehungaufKant1982}.}
Er betrachtet in selbstbewusster Abgrenzung von der aristotelischen Tradition
die Logik aus der Perspektive eines ethischen Nutzens, womit er die Tradition
aufklärerischer Logiklehrbücher einleitet, die nicht nur formale
Schlussverfahren thematisieren, sondern als praktische Logiken dem Leser zu
Mündigkeit und Selbständigkeit in der Wissenschaft wie auch im Leben verhelfen
wollen.\footnote{\phantomsection\label{Fussnote:PraktischeLogiken}\cite[Vgl.][248]{Albrecht:ChristianThomasius1999},
und zur praktischen Ausrichtung der Logik in der deutschen Aufklärung
\cite{Schneiders:PraktischeLogik1980}.} Und auch Christian \authorcite{Wolff:Psychologiaempirica1968}, der
Antipode zu \name[Christian]{Thomasius}, betont in der \titel{Praefatio} zum
\titel{Discursus praeliminaris de philosophia in genere} zunächst zwei Aspekte philosophischen
Wissens: Gewissheit und
Nützlichkeit.\footnote{\cite[Vgl.][262]{Wolff:Discursuspraeliminarisdephilosophiaingenere1996}:
\enquote{Duo inprimis sunt, quae in omni philosophia hactenus desiderantur.
Deest illa evidentia, quae sola assensum gignit certum atque immotum, nec, quae
in ea traduntur, usui vitae respondent.}} Später fügt er hinzu, dass die
Sicherheit unserer Erkenntnis kein Selbstzweck ist, sondern gerade im Dienste
der Nützlichkeit
steht.\footnote{\cite[Vgl.][\S~139]{Wolff:Discursuspraeliminarisdephilosophiaingenere1996}:
\enquote{Nos certam consequi studemus cognitionem, non vanitati litantes, sed
progressui scientiarum {\&} utilitati in vita intenti.}}  Und
\authorcite{Baumgarten:Metaphysica---Metaphysik2011} schreibt: \enquote{Keine Wahrheit darff bey Vernünfftigen
gäntzlich Brache liegen. \punkt\ Was wir lernen, muß nützlich seyn. Was nützlich
seyn soll, muß gebraucht werden. Was gebraucht wird, hat in Thun und Lassen
seinen
Einfluß.}\footcite[][\S~9]{Baumgarten:GedanckenvomVernuenfftigenBeyfallaufAcademien2008}
Es geht daher nicht ausschließlich um den aufgeklärten Wissenserwerb, sondern
gerade um den eigenständigen, kompetenten Umgang mit unserem Wissen und dessen
Anwendung im Leben oder -- wie die deutsche Aufklärungsphilosophie sich
ausdrückt -- in der \emph{Welt}. Dies spiegelt sich auch in der Verwendung des
Begriffs der \singlequote{\emph{Weltweisheit}} als Übersetzung des lateinischen
\enquote{philosophia} wieder. So soll Weltweisheit als Medium der Aufklärung
bewirken, dass die Menschen auch
\enquote{der Freiheit zu handeln nach und nach fähiger}
werden.\footnote{\cite[][A
493\,f.,]{Kant:BeantwortungderFrage:WasistAufklaerung?1977}
\cite[][VIII: 41.36]{Kant:GesammelteWerke1900ff.}.} Die Vorsilbe \enquote{Welt-}
deutet oft auf eine Ausrichtung der Wissenschaft auf erfolgreiche
Praxis hin, ebenso wie das Adjektiv \enquote{pragmatisch}.\footnote{\name[Immanuel]{Kant}
artikuliert dies mit den Worten, es gehe der Weltkenntnis darum, Welt zu \emph{haben}. Siehe
\cite[BA~vii]{Kant:AnthropologieinpragmatischerHinsicht1977}, \cite[VII:
120.9--11]{Kant:GesammelteWerke1900ff.}: \enquote{Noch sind die Ausdrücke:
die Welt \ori{kennen} und Welt \ori{haben} in ihrer Bedeutung ziemlich weit
auseinander; indem der eine nur das Spiel \ori{versteht}, dem er zugesehen
hat, der andere aber \ori{mitgespielt} hat.}}


\phantomsection\label{Abschnitt:AufklaerungunddieNuetzlichkeit}
Während Philosophen der Aufklärung ganz selbstverständlich die Nützlichkeit
intellektueller Anstrengungen einforderten -- freilich mit der Warnung, nicht
vorschnell darüber zu urteilen, ob etwas Nutzen abwerfe\footnote{So schreibt
{z.\,B.} \name[Immanuel]{Kant}, dass \enquote{kein Vorwitz der Erweiterung unserer
Erkenntnis nachteiliger sei, als der, so den Nutzen jederzeit zum voraus wissen
will, ehe man sich den mindesten Begriff von diesem Nutzen machen könnte, wenn
derselbe auch vor Augen gestellt würde}
\mkbibparens{\cite[][B 296]{Kant:KritikderreinenVernunft2003}, \cite[][III:
203.25--29]{Kant:GesammelteWerke1900ff.}}.} --, war es
\authorcite{Hegel:GesammelteWerke}, der in der \titel{Phänomenologie des
Geistes} die Nützlichkeit als letzten gemeinsamen Nenner der Aufklärung
brandmarkte.\footnote{\cite[Vgl.][IX: 310.22--315.11]{Hegel:GesammelteWerke}.} Die
Vernachlässigung (oder gar das Bestreiten) verbindlicher \distanz{Werte} und
vernünftiger Handlungsziele sind gängige Themen einer Bildungskritik, die sich
auch bei \name[Immanuel]{Kant} findet, wenn dieser gegen eine lediglich auf
Zweckrationalität zielende Ausbildung gewendet schreibt:
\begin{quote}
  Weil man in der frühen Jugend nicht weiß, welche Zwecke uns im Leben aufstoßen
  dürften, so suchen Eltern vornehmlich ihre Kinder recht \ori{vielerlei} lernen
  zu lassen, und sorgen für die \ori{Geschicklichkeit} im Gebrauch der Mittel zu
  allerlei \ori{beliebigen} Zwecken, von deren keinem sie bestimmen können, ob
  er nicht etwa wirklich künftig eine Absicht ihres Zöglings werden könne, wovon
  es indessen doch \ori{möglich} ist, daß er sie einmal haben möchte, und diese
  Sorgfalt ist so groß, daß sie darüber gemeiniglich verabsäumen, ihnen das
  Urteil über den Wert der Dinge, die sie sich etwa zu Zwecken machen möchten,
  zu bilden und zu
  berichtigen.\footnote{\cite[BA~41--2]{Kant:GrundlegungzurMetaphysikderSitten1965},
  \cite[IV: 415.20--17]{Kant:GesammelteWerke1900ff.}.}
\end{quote}
Ein Wissen, das sich als Geschicklichkeit für beliebige Zwecke nutzen lässt,
nennt \name[Immanuel]{Kant} technisch. Und eine Wissensvermittlung, die sich in diesem
Sinne als technische Ausbildung versteht, vermittelt zwar die nötige
Mittelkompetenz, die jemand zur Umsetzung beliebiger Zwecke benötigt. Aber sie vernachlässigt
dabei doch die nicht minder nötige Zielkompetenz, also die Fähigkeit, sich
selbst vernünftige Zwecke und Ziele vorzunehmen.

Der Mensch hat die Fähigkeit, sich für unterschiedliche Handlungen, Grundsätze
und Lebensentwürfe zu entscheiden, und sucht nach Orientierung; und diese Suche
nach Orientierung ist nicht nur bei
\authorcite{Spalding:BetrachtungueberdieBestimmungdesMenschen1749} klar als
Thema seiner Abhandlung ausgesprochen. Auch \name[Immanuel]{Kant}s Anthropologie
ist als Orientierungswissen für das je eigene Leben
gedacht.\footcite[Vgl.][105--108]{Cohen:KantandtheHumanSciences2009} Zu den
notwendigen Zielen gehört, was \name[Immanuel]{Kant} die Bestimmung des Menschen
nennt und was die Vorgaben der Moral zu seinem Zentrum hat. Vernunft, als
anzustrebender Leitfaden des eigenen Lebens und Ziel von Bildung, ist bei
\name[Immanuel]{Kant} keine Zweckrationalität oder instrumentelle Vernunft,
schon weil sie Moral und Ethik enthält.\footnote{Ähnliches behauptet schon
Christian August \authorcite{Crusius:Anweisungvernuenftigzuleben1744}:
\enquote{Zu einem vernünftigen Leben gehört nicht nur, daß man klug, sondern
auch hauptsächlich, daß man tugendhaft lebe, worauf die gereinigte Vernunft am
allermeisten dringet} \parencite[][Vorrede,
unpaginiert]{Crusius:Anweisungvernuenftigzuleben1744}.} Wenn
\name[Immanuel]{Kant} nun sagt, dass die Moral die Wissenschaft von der
\emph{ganzen} Bestimmung sei\footnote{Siehe das Zitat zu Anm.
\ref{Anmerkung:GanzeBestimmung} auf S. \pageref{Anmerkung:GanzeBestimmung},
wo \name[Immanuel]{Kant} den Endzweck mit der ganzen Bestimmung des Menschen
identifiziert.}, dann sieht man, dass Moral nicht eine Zielvorgabe neben anderen
innerhalb der Weltkenntnis sein kann, sondern an zentraler Stelle auch die weiteren Zielvorgaben \emph{bestimmt} oder zumindest \emph{einschränkt}.

Es gibt nach \name[Immanuel]{Kant} zwei Themen der pragmatischen Weltkenntnis:
die Natur und den Menschen; und diesen entsprechen die physische Geographie und
die Anthropologie.\footnote{\authorfullcite{Cohen:KantandtheHumanSciences2009}
identifiziert die physische Geographie, insofern sie ebenfalls den Menschen
thematisiert, mit der
physiologischen Anthropologie. \enquote{The object of pragmatic anthropology is
the human being considered as a free rational being, whilst physical geography
studies him as one \enquote{thing} on earth, independent of his intentionality}
\parencite[63]{Cohen:KantandtheHumanSciences2009}. Danach sähe \name[Immanuel]{Kant} die
physische Geographie zumindest 1798 nicht (mehr) als pragmatische Wissenschaft
an. Dem widerspricht \name[Immanuel]{Kant} aber innerhalb der
\titel{Anthropologie in pragmatischer Hinsicht} in einer Anmerkung zur Vorrede
(\cite[vgl.][BA~xiii--xiv]{Kant:AnthropologieinpragmatischerHinsicht1977},
\cite[][VII: 122.8--15]{Kant:GesammelteWerke1900ff.}).} Unter diesen beiden
jedoch nehme die Anthropologie eine Sonderrolle ein, weil sie sich auf den
Menschen konzentriere. In der Vorrede zur \titel{Anthropologie in pragmatischer
Hinsicht} schreibt \name[Immanuel]{Kant}:
\begin{quote}
  Alle Fortschritte in der Kultur, wodurch der Mensch seine Schule macht, haben
  das Ziel, diese erworbenen Kenntnisse und Geschicklichkeiten zum Gebrauch für
  die Welt anzuwenden; aber der wichtigste Gegenstand in derselben, auf den er
  jene verwenden kann, ist \ori{der Mensch}: weil er sein eigener letzter Zweck
  ist.\footnote{\cite[][BA~iii]{Kant:AnthropologieinpragmatischerHinsicht1977},
  \cite[][VII: 119.2--6]{Kant:GesammelteWerke1900ff.}.}
\end{quote}
Der Mensch ist ein letzter Zweck, wenngleich noch nicht der Endzweck, welcher
er erst als \emph{Noumenon} sein kann. Weil Weltkenntnis immer auf den Menschen
als ihren Zweck ausgerichtet ist, verdiene besonders die Erkenntnis des Menschen den Namen
Weltkenntnis.\footnote{\cite[Vgl.][BA~iii--iv]{Kant:AnthropologieinpragmatischerHinsicht1977},
\cite[][VII: 119.6--8]{Kant:GesammelteWerke1900ff.}.} 
Eine pragmatische Anthropologie ist also zunächst eine auf Brauchbarkeit im Leben angelegte Lehre vom
Menschen, die sich nicht nur an Fachphilosophen, sondern an ein breites Publikum richtet.

In der Vorrede konkretisiert \name[Immanuel]{Kant} die pragmatische Hinsicht mit Bezug auf den Gegenstand
der Untersuchung:
\begin{quote}
  Eine Lehre von der Kenntnis des Menschen, systematisch abgefaßt
  (Anthropologie), kann es entweder in \ori{physiologischer} oder in
  \ori{pragmatischer} Hinsicht sein. -- Die physiologische Menschenkenntnis geht
  auf die Erforschung dessen, was die \ori{Natur} aus dem Menschen macht, die
  pragmatische auf das, was \ori{er}, als freihandelndes Wesen, aus sich selber
  macht, oder machen kann und
  soll.\footnote{\cite[][BA~iv]{Kant:AnthropologieinpragmatischerHinsicht1977},
  \cite[][VII: 119.9--14]{Kant:GesammelteWerke1900ff.}.}
\end{quote}
Mit der Frage danach, was der Mensch aus sich selbst machen \emph{kann} und
\emph{soll}, finden die Frage nach der Bestimmung des Menschen und die
Moralphilosophie Eingang in die Anthropologie. Denn die Anthropologie behandelt
auch \emph{Ziele} menschlichen Lebens, nicht bloß Mittel zur Erreichung
kontingenter Zwecke. Ebenso wichtig ist aber die Angabe, dass die pragmatische
Anthropologie untersucht, was der Mensch aus sich selbst \emph{macht}, und
nicht, was er von Natur aus \emph{ist}.


Der Mensch ist als biologisches Wesen Gegenstand empirischer Wissenschaften, die
ihn hinsichtlich der Eigenschaften untersuchen, die er von Natur aus hat. Aber
dies ist weder die einzige Art der Herangehensweise an den Menschen, noch aus
philosophischer Sicht besonders aufschlussreich, denn sie ist für
vernunftbegabte und darum freie Wesen nicht ausreichend. Dies heißt nicht, dass
sie zu vernachlässigen wäre; vielmehr ist sie zu ergänzen. (Möglicherweise prägt
dabei die pragmatische Sichtweise auf den Menschen die Art und Weise, wie er
Gegenstand der Naturwissenschaft werden kann und soll.) \name[Immanuel]{Kant}
ändert also die Herangehensweise an die
Untersuchung des Menschen in einer entscheidenden Hinsicht: Es geht nicht mehr
(primär) darum, was der Mensch \emph{ist}, sondern darum, was er aus sich und
der Welt um ihn herum
\emph{macht}.\footnote{\cite[Vgl.][61]{Cohen:KantandtheHumanSciences2009}:
\enquote{Kant's method entails a decisive re-evaluation of traditional enquiries
into human nature: it redirects the question, \enquote{what is the human being?}
from his passive essence to his active relationship with the world -- from what
he \ori{is} to what he \ori{does}.} Siehe
\cite[35--61]{Cohen:KantandtheHumanSciences2009}, sowie \cite{Mengusoglu:DerBegriffdesMenschenbeiKant1966}.
Siehe außerdem
\cite[][275]{Boehme:AnthropologieinpragmatischerHinsicht1985}:
\enquote{Der ganze erste Teil der \ori{Anthropologie} hat dann die pragmatische
Funktion, dem Leser an der Entwicklung der anthropologischen Strukturen
klarzumachen, daß diese Würde nicht gegeben ist, sondern durch Leistung errungen
und stabilisiert werden muß. Der Mensch ist eben erst Mensch, insofern er etwas
aus sich macht.}}


Nicht als Naturwesen ist der Mensch also Thema der pragmatischen Anthropologie,
sondern als eines, das frei handelt. Was aber soll die Anthropologie dann noch
als \distanz{Natur} des Menschen beschreiben? Kann es dann überhaupt noch eine
adäquate und zugleich allgemeine Beschreibung dessen geben, was
\name[Immanuel]{Kant} den Charakter des Menschen nennt? Sofern es um den
Charakter des Menschen als eines Vernunftwesens geht, kann nur auf dessen
Freiheit verwiesen werden, sich diesen Charakter selbst zu schaffen:
\begin{quote} 
  Es bleibt uns also, um dem Menschen im System der lebenden Natur seine Klasse
  anzuweisen und so ihn zu charakterisieren, nichts übrig, als: daß er einen
  Charakter hat, den er sich selbst schafft; indem er vermögend ist, sich nach
  seinen von ihm selbst genommenen Zwecken zu perfektionieren; wodurch er, als
  mit \ori{Vernunftfähigkeit} begabtes Tier (animal rationabile), aus sich
  selbst ein \ori{vernünftiges} Tier (animal rationale) machen
  kann[.]\footnote{\cite[A~315]{Kant:AnthropologieinpragmatischerHinsicht1977},
  \cite[VII: 321.29--35]{Kant:GesammelteWerke1900ff.}.}
\end{quote}
Der Mensch schafft sich seinen Charakter selbst. Nun gebe es im Kontext
einer pragmatischen Anthropologie zwei Bedeutungen von \enquote{Charakter}:
Dieses Wort meine zum einen den natürlichen (\enquote{physischen}) Charakter,
den jemand von Natur aus mitbringt. \name[Immanuel]{Kant} denkt, dass uns
Eigenschaften wie das Temperament angeboren sind. Darüber hinaus bezeichne das
Wort aber auch den \enquote{moralischen} Charakter, den \name[Immanuel]{Kant}
\enquote{Denkungsart} nennt. Der Ausdruck \enquote{Denkungsart} ist wiederum
zentral bei der Bestimmung des Begriffs der Aufklärung, der es ja gerade um eine
Änderung der Denkungsart zu tun sei.\footnote{Siehe oben Kapitel
\ref{subsection:SelbstdenkenbeiKant}.} Und dies ist der Charakter, den sich ein
Mensch selbst schafft und der genuiner Ausdruck von Freiheit
ist.\footnote{\cite[Vgl.][A~255-6]{Kant:AnthropologieinpragmatischerHinsicht1977},
\cite[VII: 285.6-21]{Kant:GesammelteWerke1900ff.}.} Wie schon bei
\authorcite{Spalding:BetrachtungueberdieBestimmungdesMenschen1749} geht es
darum, was ein frei handelnder Mensch aus sich selbst und seinem eigenen Leben
macht.

\phantomsection\label{Abschnitt:MaximenHandlungenFreiheit}
Einen (moralischen) Charakter hat überhaupt erst derjenige, der nach
\emph{Maximen}\footnote{Siehe zum Begriff der Maxime den Forschungsüberblick in
\cite{Gressis:RecentWorkonKantianMaxims2010,Gressis:RecentWorkonKantianMaximsI:EstablishedApproaches2010}.}
handelt, d.\,i. nach Grund\-sät\-zen, denen er Autorität über seine eigenen
Handlungen beimisst und die in seinem Handeln wirksam
werden.\footnote{\cite[Vgl.][A~256]{Kant:AnthropologieinpragmatischerHinsicht1977},
\cite[VII: 285.13--15]{Kant:GesammelteWerke1900ff.}: \enquote{Der Mann von
Grundsätzen, von dem man sicher weiß, wessen man sich, nicht etwa von seinem
Instinkt, sondern von seinem Willen zu versehen hat, hat einen Charakter.} Siehe
auch \cite[A~266]{Kant:AnthropologieinpragmatischerHinsicht1977}, \cite[VII:
292.6--9]{Kant:GesammelteWerke1900ff.}: \enquote{Einen Charakter aber
schlechthin zu haben, bedeutet diejenige Eigenschaft des Willens, nach welcher
das Subjekt sich selbst an bestimmte praktische Prinzipien bindet, die er sich
durch seine eigene Vernunft unabänderlich vorgeschrieben hat.}} Nun ist es
umstritten, ob \name[Immanuel]{Kant}s Handlungstheorie überhaupt ein
Handeln zulässt, welches \emph{nicht} durch Maximen geleitet ist.\footnote{Siehe
dazu
\cite[][89--92]{Schuessler:KantsethischesLuegenverbot--derSonderfallderLuegeausFurcht2013},
sowie die dort angeführte Literatur.} Seine Auskünfte zum moralischen Charakter
jedenfalls stützen die Ansicht, wonach dies möglich ist. Ich schließe
mich hier den von
\authorfullcite{Schuessler:KantsethischesLuegenverbot--derSonderfallderLuegeausFurcht2013}
angeführten Gründen für diese Auffassung an und gehe davon aus, dass
Menschen zwar aus Maximen handeln können, aber mitunter auch gegen ihre Maximen
und manchmal sogar ganz ohne Maximen handeln. Maximen sind -- grob gesprochen --
allgemeine Lebensregeln, die aus unterschiedlichsten Gründen aufgestellt werden
können und die wir als für uns jeweils selbst verbindlich
anerkennen.\footnote{Die Deutung von Maximen als \singlequote{Lebensregeln}
findet sich u.\,a. bei Rüdiger \textcite{Bittner:Maximen1974}. Siehe dazu
kritisch \cite[][62--67]{Schwartz:DerBegriffderMaximebeiKant2006}.}


Damit, dass jemand nach Maximen handelt, ist noch nicht gesagt, dass die
entsprechenden Maximen moralisch wertvoll sind. Auch derjenige, dessen Maximen den objektiven
Gesetzen der Moral widersprechen, hat Charakter und ist daher -- so behauptet
\name[Immanuel]{Kant} -- höher zu schätzen als derjenige, der sich ohne feste
Grundsätze nur von seinen Instinkten und kurzfristigen Wünschen treiben
lässt.\footnote{\cite[Vgl.][A~266]{Kant:AnthropologieinpragmatischerHinsicht1977},
\cite[VII: 292.10--14]{Kant:GesammelteWerke1900ff.}: \enquote{Ob nun diese
Grundsätze auch bisweilen falsch und fehlerhaft sein dürfte, so hat doch das
Formelle des Wollens überhaupt, nach festen Grundsätzen zu handeln (nicht wie in
einem Mückenschwarm bald hiehin bald dahin abzuspringen), etwas Schätzbares und
Bewundernswürdiges in sich; wie es denn auch etwas Seltenes ist.} Siehe ferner
\cite[][A~269]{Kant:AnthropologieinpragmatischerHinsicht1977}, \cite[VII:
293.14--23]{Kant:GesammelteWerke1900ff.}.} 
Nach festen Grundsätzen muss nicht nur derjenige handeln, der sich an der Moral
orientiert, sondern gerade auch derjenige, der sein eigenes Glück im Sinne eines
langfristigen und beständigen Zustands der Zufriedenheit erreichen
möchte. \name[Immanuel]{Kant} spricht hier von (Privat-) \emph{Klugheit}. Wenn jemand
seine Handlungen in Abhängigkeit von subjektiven Grundsätzen oder Maximen
bestimmt, so handelt er frei. Und daher ist sowohl moralisches Handeln, als auch
kluges Handeln Ausdruck von Freiheit, insofern es ein Handeln nach Grundsätzen
ist. Der Mensch, der nach Maximen handelt, wird ein Subjekt, das nicht mehr bloß
seiner Natur unterworfen ist, sondern sich seine eigene
Seinsweise in gewisser Weise selbst geschaffen hat.\footnote{\cite[Vgl.][A
267]{Kant:AnthropologieinpragmatischerHinsicht1977}, \cite[VII:
292.15--18]{Kant:GesammelteWerke1900ff.}: \enquote{Es kommt hiebei nicht auf das
an, was die Natur aus dem Menschen, sondern was dieser \ori{aus sich selbst
macht}; denn das erstere gehört zum Temperament (wobei das Subjekt großenteils
passiv ist) und nur das letztere gibt zu erkennen, daß er einen Charakter
habe.}} Die Entwicklung der Vernunft und die Ausgestaltung der je eigenen
Freiheit sind zunächst der gesuchte Endzweck. Die Bestimmung des Menschen ist
eine Bestimmung zur Freiheit.

Nun stellt sich die Frage, wozu \name[Immanuel]{Kant} bei der Untersuchung einer
Bestimmung des Menschen auf die empirische Anthropologie zurückgreift, wo es
doch um eine Bestimmung zur Freiheit geht, der Mensch also gerade nicht als
Naturwesen, sondern als \emph{Noumenon} in den Blick kommt. Die Antwort lautet:
Die Frage, welche Eigenschaften der Mensch \distanz{von Natur aus}, also ohne
sein Zutun hat (sein physischer Charakter), interessiert zunächst als sekundäre
Frage wegen der Bedeutung für die Handlungsmöglichkeiten, die der Menschen als
Rahmenbedingungen seines freien Handelns vorfindet. Wie wir im nächsten
Abschnitt sehen werden, versucht die Anthropologie damit die \emph{Endlichkeit} des
Menschen als Hintergrundfolie seines Handelns zu berücksichtigen. Die Welt als
das, was wir erfahren und empirisch erforschen können, gibt dem Menschen seine
Bestimmung nicht vor -- dies vermag nur die Vernunft --, aber sie ist doch der
\enquote{Schauplatz seiner Bestimmung}\footnote{\cite[A~12]{Kant:VondenverschiedenenRassenderMenschen1977},
\cite[II: 443.18]{Kant:GesammelteWerke1900ff.}.}, den es in seiner
Beschaffenheit zu berücksichtigen gilt. Eine Anthropologie hat also
zunächst nicht wegen der Handlungsziele, sondern wegen der Rahmenbedingungen
freien Handelns mit Erfahrung und nicht nur mit Vernunft und Freiheit zu tun.
Sonst benötigte sie keine Erfahrung und müsste nicht empirisch vorgehen, sondern
fiele mit der Moralphilosophie (und eventuell einer praktischen Logik) zusammen.
Zu diesem Schauplatz rechnet \name[Immanuel]{Kant} nun nicht nur das, was uns
als Natur und Welt umgibt, sondern auch das, was wir von Natur aus selbst sind.



Eine pragmatische Anthropologie befasst sich auch mit natürlichen Eigenschaften
des Menschen, immer aber aus der Perspektive desjenigen, der sich als
vernunftbegabtes Wesen frei zu entwickeln trachtet. Und aus dieser Perspektive
muss Anthropologie insbesondere auch die Hindernisse und
Widrigkeiten in den Blick nehmen, die unserer Vernunft und
Freiheit gerade im Wege stehen. Was der Mensch von Natur aus \emph{ist} -- im
Gegensatz zu dem, was er aus freien Stücken aus sich selbst \emph{macht} --
betrachtet \name[Immanuel]{Kant} primär als ein \emph{Hindernis} der Freiheit.
Freilich sind gerade diese
Hindernisse im Interesse einer realistischen Umsetzung unserer Freiheit damit
auch Thema der Philosophie, speziell der pragmatischen
Anthropologie.\footcite[Vgl.][68]{Cohen:KantandtheHumanSciences2009} Sie
betrachtet diese einerseits natürlich mit dem Ziel, die anderweitig vorgegebenen
Ziele effektiv \emph{umsetzen} zu können. Andererseits haben diese Betrachtungen aber
auch Auswirkungen auf die Ziele selbst: Was von uns realistischerweise nicht
umgesetzt werden kann, sollte vernünftigerweise auch nicht als Ziel verfolgt
werden. Es ist beispielsweise
unvernünftig, Ziele zu verfolgen, zu denen man kein Talent mitbringt. Wenn aber
die Entwicklung der eigenen Talente eine Forderung der Moral ist (wofür
\name[Immanuel]{Kant} sie hält), muss man natürlich wissen, welche Talente man
denn besitzt. (Freilich ist die Kenntnis der je eigenen Talente kein Thema einer
allgemeinen philosophischen Anthropologie. Aber sie erläutert doch recht
anschaulich, wie sich \name[Immanuel]{Kant} die Relevanz empirischen Wissens
auch für die moralische Zielsetzung denkt.) Kompetenz in der eigenen Zielsetzung
setzt voraus, über das Erreichbare und die Grenzen der eigenen
Handlungsmöglichkeiten Bescheid zu wissen.\footnote{Ähnlich beschreibt bereits
\authorfullcite{Doering:UeberKantsLehrevonBegriffundAufgabederPhilosophie1885}
die Aufgabe der Philosophie nach \name[Immanuel]{Kant}: \enquote{Die Bestimmung der Aufgabe der Philosophie [gemeint
ist \name[Immanuel]{Kant}s Bestimmung derselben; A.\,G.] ist in ihrer
Allgemeinheit richtig; nach ihr bildet die Philosophie das Mittel- und Bindeglied zwischen der reinen theoretischen
Wissenschaft als Erkenntniß des Thatsächlichen im weitesten Umfange und ohne
Nebengedanken und den praktischen Disciplinen als Anleitungen zur Verwirklichung
des dem Menschen Wichtigen und Werthvollen, indem sie lehrt, was das für den
Menschen Werthvolle ist und in welchem Umfange es nach Lage der gegebenen
Welteinrichtung verwirklicht werden kann}
\parencite[][479]{Doering:UeberKantsLehrevonBegriffundAufgabederPhilosophie1885}.}
Außerdem ergeben sich besondere Zwecke aus den besonderen Umständen, in denen
sich der Mensch als Mensch stets befindet.


Zu den auffälligsten Kennzeichen des
natürlichen Charakters des Menschen gehört nach \name[Immanuel]{Kant} die
\enquote{ungesellige Geselligkeit}:
\begin{quote}
  \punkt wobei aber das Charakteristische der Menschengattung, in Vergleichung
  mit der Idee möglicher vernünftiger Wesen auf Erden überhaupt, dieses ist: daß die
  Natur den Keim der \ori{Zwietracht} in sie gelegt und gewollt hat, daß ihre
  eigene Vernunft aus dieser diejenige \ori{Eintracht}, wenigstens die
  beständige Annäherung zu derselben,
  herausbringe\punkt\footnote{\cite[B~313--314]{Kant:AnthropologieinpragmatischerHinsicht1977},
  \cite[VII: 322.3--8]{Kant:GesammelteWerke1900ff.}. In der \titel{Idee zu
  einer allgemeinen Geschichte in weltbürgerlicher Absicht} schreibt er:
  \enquote{Ich verstehe hier unter dem Antagonism die \ori{ungesellige Geselligkeit}
  der Menschen; d.\,i. den Hang derselben, in Gesellschaft zu treten, der doch
  mit einem durchgängigen Widerstande, welcher diese Gesellschaft beständig zu
  trennen droht, verbunden ist}
  \mkbibparens{\cite[][A
  392]{Kant:IdeezueinerallgemeinenGeschichteinweltbuergerlicherAbsicht1977},
  \cite[][VIII: 20.30--33]{Kant:GesammelteWerke1900ff.}}.}
\end{quote}
Die Aufforderung an den Menschen, einen Zustand der Eintracht und des
friedlichen Miteinander herzustellen, -- die Bestimmung zur Errichtung einer
bürgerlichen Gesellschaft --  ist nur vor dem Hintergrund der Tatsache
verständlich, dass der Mensch von Natur aus auf Gemeinschaft angewiesen, hierzu
aber nur bedingt geeignet ist.\footnote{Von Vertretern einer
\singlequote{Kritischen Theorie} in der Traditionslinie von
\authorcite{Horkheimer:DialektikderAufklaerung1997} wird dieser Punkt mitunter
als bürgerliche Ideologie angesehen und behauptet, der Antagonismus liege nicht
in der Natur des Menschen, sondern resultiere aus gesellschaftlichen
Verhältnissen, die es zu überwinden gelte.
\cite[Vgl.][passim]{Staedtler:KantunddieAporetikmodernerSubjektivitaet2011}.}
Dass wir zu einem friedlichen und einträchtigen Leben verpflichtet sind, wissen
wir \emph{a priori}; aber dass wir als Menschen zur Zwietracht neigen,
können wir nur aus Erfahrung wissen. Wenn empirische Erkenntnisse also in die
Generierung notwendiger Handlungsziele eingehen, dann geschieht dies hier analog
zu dem Fall, in dem ich als moralisch notwendiges Ziel erkenne, einem Menschen
zu helfen, der in Not geraten ist. Dass ich zur Hilfe verpflichtet bin, weiß ich
zwar \emph{a priori}, aber dass jemand diese Hilfe benötigt, erkenne ich eben \emph{a posteriori}.

\name[Immanuel]{Kant}s pragmatische Anthropologie thematisiert die Grundlagen
und Bedingungen unseres Menschseins -- sie greift damit mit dem Ziel der
Förderung unserer Freiheit aus dem Humanismus die Thematik einer \emph{conditio
humana}
auf.\footcite[Vgl.][476]{Zoeller:DieBestimmungderBestimmungdesMenschenbeiMendelssohnundKant2001}
Dabei mögen \name[Immanuel]{Kant}s Vorgehen und seine Behauptungen im Einzelnen
fragwürdig erscheinen; zu würdigen ist zunächst das Projekt als solches. Es geht
darum, empirisches Wissen mit Blick auf seine Relevanz für die je eigene
selbstbestimmte Lebensführung und Persönlichkeitsentwicklung aufzuarbeiten.
Deswegen lässt sich pragmatische Anthropologie nicht durch Moral ersetzen. Sie
ist auch keine angewandte Ethik im Sinne der praktischen Anthropologie, von der
\name[Immanuel]{Kant} in der \titel{Grundlegung zur Metaphysik der Sitten}
spricht.\footnote{\cite[Vgl.][BA v]{Kant:GrundlegungzurMetaphysikderSitten1965},
\cite[][IV: 388.9--14]{Kant:GesammelteWerke1900ff.}.} Das Wissen um die
\emph{conditio humana} ist eine notwendige Bedingung zur Umsetzung der
Bestimmung des Menschen zur Freiheit und gehört damit zu den wesentlichen
Zwecken der Vernunft.

\begin{comment}
Ich hatte oben einen ersten Einwand formuliert und gefragt, wie sich die
Thematisierung einer Bestimmung des Menschen zu dem Begriff der Freiheit
verhält, ob sie sich mit einem negativen Freiheitsbegriff verträgt oder einen
positiven Freiheitsbegriff zugrunde legt, und wie sich dies auf das Projekt
einer liberalen Aufklärung auswirkt.\footnote{Siehe S.
\pageref{Abschnitt:TwoConceptsofLiberty}.} Eine naheliegende Antwort lautet,
dass verbindliche Vorgaben, wie die je individuelle Freiheit auszufüllen sei,
ausschließlich der Moral entstammen. Wenn es einen Sinn gibt, in dem
\name[Immanuel]{Kant} einen \singlequote{positiven} Freiheitsbegriff verficht,
dann ist es danach genau der Sinn, den
\authorcite{Berlin:TheProperStudyofMankind1997} bereits ohne Erwähnung der
Anthropologie oder einer Bestimmung des Menschen der Moralphilosophie
\name[Immanuel]{Kant}s attestiert. Eine Anthropologie in pragmatischer Hinsicht
versucht nicht, die Freiheit des einzelnen mit weiteren Inhalten zu füllen und
den Begriff der Freiheit damit -- aus liberaler Sicht -- \emph{ad absurdum} zu
führen. Dennoch gibt es Maßstäbe der Vernunft, anhand derer wir unser Denken und
Handeln bewerten können.

\phantomsection\label{Abschnitt:DialektikderAufkaerungEinwaende}
Ein zweiter, dem ersten diametral entgegengesetzter Einwand gegen die Betonung
\singlequote{pragmatischer} Wissenschaften war: Damit reduziere Aufklärung unser Denken auf die bloße
Nützlichkeit und lasse kein vernünftiges Ziel mehr bestehen. Diesen Vorwurf
machen \authorcite{Horkheimer:DialektikderAufklaerung1997} im Ausgang von
\authorcite{Hegel:GesammelteWerke} stark.\footnote{Siehe
S.~\pageref{Abschnitt:AufklaerungunddieNuetzlichkeit}. Kürzlich wurde dieser
Einwand ausführlich von
\textcite[vgl.][passim]{Staedtler:KantunddieAporetikmodernerSubjektivitaet2011}
aufgegriffen.} Er ist die unmittelbare Kehrseite der Antwort auf den letzten
Einwand, denn er argumentiert, dass ein rein negativer Freiheitsbegriff den
einzelnen Menschen ohne Orientierung lasse und letztlich jede Bestimmung
vernünftiger Ziele jenseits der bloßen Selbsterhaltung unterminiere. Im nächsten
Abschnitt zeige ich auf, welche Erkenntnisse wir nach \name[Immanuel]{Kant}
benötigen, um unser Handeln an der Vernunft auszurichten.
\end{comment}

\section{Aufklärung des endlichen Willens}
\subsection{Arten handlungsorientierenden
Wissens}\label{subsection:aufklaerungundpraxis} Wenn die Bestimmung eines
Menschen also das ist, wovon sein Charakter der Ausdruck ist, dieser Charakter aber frei und die Bestimmung eine Bestimmung zur Freiheit sein soll, dann bleibt als
Inhalt einer allgemeinen Bestimmung des Menschen nichts übrig als die
Entwicklung und Konkretisierung der Freiheit selbst vor dem Hintergrund der
Situation, mit der sich der Mensch \emph{qua} Mensch in dieser Welt konfrontiert
sieht (der \emph{conditio humana}). Die \emph{conditio humana} betrifft dabei insbesondere
auch die eigene Verfassung des Menschen als eines Naturwesens. Und genau dies
findet sich bei genauerem Hinsehen innerhalb der \titel{Anthropologie in
pragmatischer Hinsicht} als Angabe der Bestimmung des Menschen:
\begin{quote}
  \phantomsection\label{Zitat:KultivierungZivilisierungMoralisierungalsBestimmungdesMenschen}Die
  Summe der pragmatischen Anthropologie in Ansehung der Bestimmung des Menschen
  und die Charakteristik seiner Ausbildung ist folgende. Der Mensch ist durch
  seine Vernunft bestimmt, in einer Gesellschaft mit Menschen zu sein, und in
  ihr sich durch Kunst und Wissenschaft zu \ori{kultivieren}, zu
  \ori{zivilisieren} und zu \ori{moralisieren}; wie groß auch sein tierischer
  Hang sein mag, sich den Anreizen der Gemächlichkeit und des Wohllebens, die er
  Glückseligkeit nennt, \ori{passiv} zu überlassen, sondern vielmehr
  \ori{tätig}, im Kampf mit den Hindernissen, die ihm von der Rohigkeit seiner
  Natur anhängen, sich der Menschheit würdig zu
  machen.\footnote{\cite[A~321]{Kant:AnthropologieinpragmatischerHinsicht1977},
  \cite[VII: 324.33--325.4]{Kant:GesammelteWerke1900ff.}.}
\end{quote}
In der \titel{Pädagogik}, die ebenfalls auf die Bestimmung des Menschen
ausgerichtet ist,\footnote{\cite[Vgl.][A 17]{Kant:UeberPaedagogik1977},
\cite[IX:
447.30--33]{Kant:GesammelteWerke1900ff.}: \enquote{Kinder sollen nicht dem
gegenwärtigen, sondern dem zukünftig möglich bessern Zustande des menschlichen
Geschlechts, das ist: der Idee der Menschheit und deren ganzer Bestimmung
angemessen, erzogen werden.} Zu pädagogischen Ansprüchen des
Aufklärungsprogramms siehe \cite[][\pno~513\,f.]{Theis:KantetlAufklaerung2012}.}
nennt \name[Immanuel]{Kant} als erste Stufe die
\emph{Disziplinierung}, die in der \titel{Kritik der Urteilskraft} zur Kultur (der \enquote{Zucht}) und in der \titel{Anthropologie in pragmatischer Hinsicht}
zur Zivilisierung gezählt
wird,\footnote{\cite[Vgl.][B~392]{Kant:KritikderUrteilskraft2009}, \cite[V:
432.3--12]{Kant:GesammelteWerke1900ff.}; \cite[A
319]{Kant:AnthropologieinpragmatischerHinsicht1977}, \cite[VII:
323.26]{Kant:GesammelteWerke1900ff.}.} behält ansonsten aber das Schema aus
Kultivierung, Zivilisierung und Moralisierung
bei.\footnote{\cite[Vgl.][A~22f.]{Kant:UeberPaedagogik1977}, \cite[IX:
449.27--450.14]{Kant:GesammelteWerke1900ff.}.}

Wenn man mit \authorcite{Baumgarten:Metaphysica---Metaphysik2011}, von dem
ausgehend \name[Immanuel]{Kant} seine Position entwickelt\footnote{Vgl.
\cite[][5, 29--65]{Schwaiger:KategorischeundandereImperative1999}, sowie
\cite[][152]{Schwaiger:KlugheitbeiKant2002}.}, Zielkompetenz als Weisheit und
Mittelkompetenz als Klugheit
bezeichnet,\footnote{\cite[Vgl.][\S~882]{Baumgarten:Metaphysica---Metaphysik2011},
\cite[XVII: 172.22--29]{Kant:GesammelteWerke1900ff.}: \enquote{\ori{Sapientia}
nexus finalis \ori{generatim} est perspicientia, et quidem finium
\ori{sapientia speciatim}, remediorum \ori{prudentia}.} \authorcite{Baumgarten:Metaphysica---Metaphysik2011}
selbst schlägt vor, \enquote{sapientia} mit \enquote{Weisheit} und
\enquote{prudentia} mit \enquote{Klugheit} zu übersetzen. Siehe dazu
\cite[][152]{Schwaiger:KlugheitbeiKant2002}.} so liegt es nahe, für die Bildung
neben Klugheit auch Weisheit als Ziel zu fordern. Aber wenn man dann denkt, man müsste
nur die technische Ausbildung um die Vermittlung ethischer Erkenntnisse
erweitern, also die Klugheit mit der Kultur und die Weisheit mit der Moral
identifiziert, bleibt man hinter der Systematik
\name[Immanuel]{Kant}s zurück. Weisheit umfasst mehr als das Wissen um das moralisch
Richtige; und Klugheit geht nicht in technischem Wissen und Können auf. Um
besser sehen zu können, welche Kompetenzen und welches Wissen zu einem
selbstbestimmten Leben nötig sind, lohnt daher eine genauere Untersuchung von
Begriffen wie \enquote{pragmatisch}, \enquote{klug}, \enquote{technisch} und
\enquote{praktisch}. Die leitende Frage ist: Welche Funktionen können
Erkenntnisse bei unserer Handlungsorientierung übernehmen und welche Relevanz
hat dies für \name[Immanuel]{Kant}s Aufklärungsprogramm?

\authorfullcite{Meier:Vernunftlehre1752} unterscheidet zwischen praktischen,
spekulativen und theoretischen Erkenntnissen, um ihre Einbindung in unser
Handeln zu thematisieren. Zunächst heißt eine Erkenntnis nach \authorcite{Meier:Vernunftlehre1752} praktisch (im
Gegensatz zu spekulativ), wenn sie überhaupt geeignet ist, eine
handlungsorientierende Funktion zu übernehmen.
\begin{quote}
  \ori{Eine Erkenntniss ist praktisch} (cognitio practica), in so ferne sie uns
  auf eine merkliche Art bewegen kann, eine Handlung zu thun oder zu lassen.
  Alle vollkommenere Erkenntniss, die nicht praktisch ist, wird \ori{eine
  speculativische Erkenntniss} (cognitio speculativa, speculatio) genennet. Alle
  gelehrte Erkenntniss ist demnach entweder praktisch oder
  speculativisch.\footnote{\cite[][61]{Meier:AuszugausderVernunftlehre1752},
  \cite[][XVI: 516.20--22, 517.23--24]{Kant:GesammelteWerke1900ff.}.}
\end{quote}
Er verwendet den Ausdruck \enquote{praktisch} aber auch in einem anderen
Sinne, nämlich als Gegenbegriff zu \enquote{theoretisch}; und hier heißt
\enquote{praktisch} so viel wie \enquote{vorschreibend}:
\begin{quote}
  Eine Erkenntniss, in welcher wir uns vorstellen, dass etwas gethan oder
  gelassen werden solle, wird auch praktisch genannt, in so ferne sie \ori{der
  theoretischen Erkenntniss} (cognitio theoretica, theoria) entgegengesetzt
  wird, der Erkenntniss, die uns nicht vorstellt, dass etwas gethan oder
  gelassen werden solle. Alle gelehrte Erkenntniss ist entweder praktisch oder
  theoretisch, und beide Arten gehören entweder zu der praktischen oder
  spekulativischen
  Erkenntniss[.]\footnote{\cite[][\pno~61\,f.]{Meier:AuszugausderVernunftlehre1752},
  \cite[][XVI: 517.25--31]{Kant:GesammelteWerke1900ff.}.}
\end{quote}
Da diese doppelte Bedeutung von \enquote{praktisch} nicht ohne Härte ist (es
gibt dann praktische Erkenntnisse, die wiederum nicht praktisch sind), können
wir von \enquote{pragmatisch} als Gegenbegriff zu \enquote{spekulativ} sprechen.
Diesen Vorschlag macht \name[Immanuel]{Kant},\footnote{Siehe die Randbemerkung
zur Begriffseinführung in \authorcite{Meier:Vernunftlehre1752}s Lehrbuch in
\cite[][\nopp 2795]{Kant:Reflexionen1900ff.}, \cite[][XVI:
516.8-9]{Kant:GesammelteWerke1900ff.}.} aber dennoch dürfen wir uns nicht darauf
verlassen, dass es sich um eine terminologische Festlegung handelt, die in
seinen Schriften Bestand hat.\footnote{Zum Wandel der Begriffsverwendungen bei
\name[Immanuel]{Kant} siehe
\cite[][113--141]{Schwaiger:KategorischeundandereImperative1999}.} Zumindest
handelt es sich nicht um die einzige Bedeutung dieses Ausdrucks bei
\name[Immanuel]{Kant}; wir können sie die weite Bedeutung nennen und werden
gleich noch eine engere Bedeutung
finden.\footnote{\phantomsection\label{Fussnote:DoppelteBedeutungvonPragmatisch}Alix
\textcite[][69]{Cohen:KantandtheHumanSciences2009} macht eine solche doppelte
Bedeutung des Wortes \enquote{pragmatisch} aus, wohingegen Allen
\textcite[39--42]{Wood:KantandtheProblemofHumanNature2003} sogar vier
verschiedene Bedeutungen unterscheidet. Auf eine vergleichbare Doppeldeutigkeit
von \enquote{praktisch} verweist
\textcite[vgl.][77]{Louden:TheSecondPartofMorals2003}. Ich werde im folgenden
von \enquote{pragmatisch im weite(re)n Sinne} sprechen, wenn es allgemein um
handlungsorientierende Erkenntnisse geht, und von \enquote{pragmatisch im
enge(re)n Sinne} bei Erkenntnissen der Klugheit.}


Eine Erkenntnis heißt also pragmatisch im weiteren Sinne, wenn ich meine
Handlungen an ihr ausrichte bzw.
sie dazu geeignet ist, dass ich das tue. Da es unseren Erkenntnissen in gewisser
Hinsicht äußerlich ist, ob jemand sein Handeln an ihnen ausrichtet, sagt
\name[Immanuel]{Kant}, dass die Unterscheidung \enquote{pragmatisch}/\enquote{spekulativ}
nicht die Erkenntnisse, sondern ihren Gebrauch
beschreibe.\footnote{\cite[Vgl.][\nopp 2802]{Kant:Reflexionen1900ff.},
\cite[][XVI:
519.15--17]{Kant:GesammelteWerke1900ff.}: \enquote{Erkenntnis ist
(\textsuperscript{g} Satze sind) entweder practisch oder theoretisch. Der
Gebrauch der Erkenntnis entweder \sout{theoretisch} practisch oder speculativ.}} Als Handlungsorientierungen
kommen Aussagen unterschiedlichster Art in Betracht, beispielsweise
\enquote{Mein Wecker klingelt}, \enquote{Ich habe Hunger} oder \enquote{Der
Termin rückt näher}, aber auch \enquote{Beeile dich!}, \enquote{Du solltest ihn
nicht verletzen!} oder \enquote{Wenn du einen Pizzateig ansetzen möchtest,
solltest du erst das Mehl in eine Schüssel sieben}. Die Aussagen der
pragmatischen Anthropologie (und auch der physischen Geographie) sollen stets
zur Handlungsorientierung geeignet sein. Aber sie haben kein Monopol auf
Brauchbarkeit, sondern bilden nur den geeigneten Ausgangspunkt zur vernünftigen
und aufgeklärten Anwendung weiterer Erkenntnisse.\footnote{Philosophie ist
\enquote{die Wissenschaft von der Beziehung \myemph{aller} Erkenntnis auf die
wesentlichen Zwecke der menschlichen Vernunft}
\mkbibparens{\cite[][B 867]{Kant:KritikderreinenVernunft2003},
\cite[][III: 542.26--28]{Kant:GesammelteWerke1900ff.}}. Siehe auch
\cite[][A 23]{Kant:ImmanuelKantsLogik1977}, \cite[][IX:
23.30--24.2]{Kant:GesammelteWerke1900ff.}.} Auch und gerade technisches Wissen
ist handlungsorientierend.
Spekulative Erkenntnisse auf der anderen Seite sind müßig, ohne Relevanz für
unser Leben, in Wahrheit aber auch sehr selten. Denn letztlich kann fast jede
Erkenntnis eine mehr oder minder stark ausgeprägte
handlungsorientierende Funktion übernehmen, wenn wir auch manchmal
dazu neigen, diese zu übersehen.\footnote{\authorcite{Meier:Vernunftlehre1752} verwendet die etwas
hölzern klingende Form \enquote{speculativisch}, die ich hier dem
Sprachgebrauch \name[Immanuel]{Kant}s folgend in \enquote{spekulativ} umwandle.
\cite[Vgl.][\S~219]{Meier:AuszugausderVernunftlehre1752}, \cite[][XVI:
520.14--17]{Kant:GesammelteWerke1900ff.}: \enquote{Keine wahre gelehrte
Erkenntniss ist ihrer Natur nach speculativisch, sondern nur um des Mangels der
Einsicht eines Gelehrten willen, welcher ihren Zusammenhang mit dem Verhalten
des Menschen nicht einsehen kann, oder nicht einsehen will. In dem letzten Fall
beschimpft sich der Gelehrte selbst.}} Dabei können Erkenntnisse aus
unterschiedlichsten Gründen und in verschiedensten Hinsichten pragmatisch sein,
beispielsweise wenn sie zur Erlangung oder Erhaltung
unserer Zufriedenheit dienen oder Einfluss auf unser (gutes)
Verhalten
haben.\footnote{\cite[Vgl.][\S\S~222--225]{Meier:AuszugausderVernunftlehre1752},
\cite[XVI: 520--522]{Kant:GesammelteWerke1900ff.}.}

Nun sollten wir vielleicht eher von präskriptiven und deskriptiven Urteilen
und Äußerungen sprechen, um \authorcite{Meier:Vernunftlehre1752}s Einteilung in praktische und
theoretische Erkenntnisse zu artikulieren. Als präskriptiv zählen beispielsweise
Imperative der Ethik oder des Rechts oder auch göttliche Gebote und Befehle. Es versteht sich,
dass der Bereich des Präskriptiven viel kleiner ist als der des Pragmatischen
(im weiten Sinne) und dass es keinen Mangel einer Erkenntnis darstellt, nicht
präskriptiv zu sein. Die meisten unserer Urteile und Äußerungen sind es nicht
und es ist offensichtlich, dass die Klasse handlungsorientierender Erkenntnisse
nicht mit der Klasse präskriptiver Äußerungen zusammenfällt.

Eine begriffliche Neuerung \name[Immanuel]{Kant}s gegenüber
\authorcite{Meier:Vernunftlehre1752} wird für die hier interessierende
Fragestellung wichtig werden: Mit \name[Immanuel]{Kant} wird der Begriff des
Imperativs zum zentralen Mittel der Beschreibung von Handlungsweisen endlicher
Wesen (\authorcite{Meier:Vernunftlehre1752} spricht zwar von einem
\singlequote{Sollen}, aber noch nicht von
\singlequote{Imperativen}).\footnote{\cite[Vgl.][164]{Schwaiger:KategorischeundandereImperative1999}:
\enquote{Zu Kants erfolgreichsten Neuschöpfungen auf dem Gebiet der ethischen,
ja der philosophischen Terminologie überhaupt, bei der wiederum die
Auseinandersetzung mit Baumgartens Lehrbüchern eine entscheidende Rolle gespielt
hat, zählt der Begriff \singlequote{Imperativ}.}} \name[Immanuel]{Kant} greift
dabei \authorcite{Baumgarten:Metaphysica---Metaphysik2011}s Unterscheidung von
Notwendigkeit (\singlequote{\emph{necessitas}}) und Nötigung
(\singlequote{\emph{necessitatio}})\footnote{\cite[Vgl.][]{Baumgarten:Metaphysica---Metaphysik2011},
\S~102 (\enquote{\emph{necessitas}}),
\S~701 (\enquote{\emph{necessitatio}}).} auf, wenn er den Imperativ als etwas
bestimmt, dem nur endliche Wesen mit ihrer Diskrepanz zwischen Sollen und Wollen
unterliegen.\footnote{\cite[Vgl.][164--167]{Schwaiger:KategorischeundandereImperative1999}.
\enquote{Baumgarten seinerseits hat letzteren Begriff gegenüber Wolff neu
eingeführt und damit den zwingenden Charakter moralischer Vorschriften
wesentlich stärker zur Geltung gebracht}
\parencite[][167]{Schwaiger:KategorischeundandereImperative1999}.} Moralisches
Sollen gilt für alle Wesen mit Notwendigkeit, aber es nötigt nur endliche Wesen.
Es zeichnet uns als endliche Wesen aus, dass wir nicht automatisch gemäß
vernünftiger Einsicht handeln, weil die vernünftige Einsicht stets in Konkurrenz
zu unseren unmittelbaren Neigungen steht.\footnote{Siehe oben, Kap.
\ref{subsubsection:DieEndlichkeitdesWillens}.} Was
\authorcite{Meier:Vernunftlehre1752} als praktische Erkenntnisse den
theoretischen entgegenstellt, nennt \name[Immanuel]{Kant} Imperative, die uns
als nötigend gegenüberstehen.

Bei \name[Immanuel]{Kant} sind die \enquote{spekulativen} Erkenntnisse neben den
\enquote{Naturerkenntnissen} eine Untergruppe der theoretischen
Erkenntnisse,\footnote{\cite[Vgl.][B 662f.]{Kant:KritikderreinenVernunft2003}, \cite[][III:
422.16--20]{Kant:GesammelteWerke1900ff.}:
\enquote{Eine theoretische Erkenntnis ist \ori{spekulativ}, wenn sie auf einen
Gegenstand, oder solche Begriffe von einem Gegenstande, geht, wozu man in keiner
Erfahrung gelangen kann. Sie wird der \ori{Naturerkenntnis} entgegengesetzt,
welche auf keine anderen Gegenstände oder Prädikate derselben geht, als die in
einer möglichen Erfahrung gegeben werden können.}} während bei
\authorcite{Meier:Vernunftlehre1752} auch spekulative Erkenntnisse denkbar sind,
die nicht theoretisch
sind\footnote{\cite[Vgl.][62]{Meier:AuszugausderVernunftlehre1752}, \cite[][XVI:
517.29--31]{Kant:GesammelteWerke1900ff.}.}.
Die Einteilung in theoretische und praktische Erkenntnisse scheint er zu
übernehmen, insofern er zumindest an manchen Stellen ebenfalls das \emph{Sollen}
als Charakteristikum praktischer Erkenntnis herausstellt.\footnote{\cite[Siehe
z.\,B.][B 661]{Kant:KritikderreinenVernunft2003}, \cite[][III:
421.17--19]{Kant:GesammelteWerke1900ff.}.} \emph{Prima facie} ließe sich daraufhin
vermuten, dass \name[Immanuel]{Kant} die praktische Erkenntnis mit dem
Moralischen identifizierte, zumal das Praktische bei \name[Immanuel]{Kant} von
Imperativen und einem Sollen
handelt.\footnote{\enquote{Daß diese Vernunft
nun Kausalität habe, wenigstens wir uns eine dergleichen an ihr vorstellen, ist
aus den \ori{Imperativen} klar, welche wir in allem Praktischen den ausübenden
Kräften als Regeln aufgeben. Das \ori{Sollen} drückt eine Art von Notwendigkeit
und Verknüpfung mit Gründen aus, die in der ganzen Natur sonst nicht vorkommt}
\mkbibparens{\cite[][B 575]{Kant:KritikderreinenVernunft2003},
\cite[III: 371.15--17]{Kant:GesammelteWerke1900ff.}}.}

Doch auch wenn die praktische Vernunft uns endlichen Wesen generell in der Form
von Imperativen entgegentritt, darf \enquote{praktisch} nicht mit
\enquote{moralisch} identifiziert werden, sondern umfasst auch technische und
(in einem noch zu erläuternden engeren Sinne) pragmatische
Erkenntnisse.\footnote{\cite[Vgl.][B~828]{Kant:KritikderreinenVernunft2003}, \cite[III: 520.1-16]{Kant:GesammelteWerke1900ff.}. Mit den moralischen Gesetzen
koextensiv sind nicht die praktischen, sondern die \emph{reinen} praktischen
Gesetze.} In der \titel{Kritik der reinen Vernunft} schreibt
\name[Immanuel]{Kant}: \enquote{Praktisch ist alles, was durch Freiheit möglich
ist.}\footnote{\cite[B~828]{Kant:KritikderreinenVernunft2003}, \cite[III:
520.1]{Kant:GesammelteWerke1900ff.}. Siehe zur Unterscheidung von praktischer
und theoretischer Vernunft:
\cite{Engstrom:KantsDistinctionbetweenTheoreticalandPracticalKnowledge2002}.}
Und durch Freiheit ist auch aus \name[Immanuel]{Kant}s Sicht nicht nur
moralisches Handeln möglich.\footnote{Siehe oben, Kapitel \ref{Abschnitt:MaximenHandlungenFreiheit},
S.~\pageref{Abschnitt:MaximenHandlungenFreiheit}f.} Und in der \titel{Kritik
der Urteilskraft} heißt es:
\begin{quote}
[A]lles, was als durch einen Willen möglich (oder notwendig) vorgestellt wird,
heißt praktisch-möglich (oder -notwendig); zum Unterschiede von der physischen
Möglichkeit oder Notwendigkeit einer Wirkung, wozu die Ursache nicht durch
Begriffe (sondern, wie bei der leblosen Materie, durch Mechanismen und bei
Tieren durch Instinkt) zur Kausalität bestimmt
wird.\footnote{\cite[][B xii\,f.,]{Kant:KritikderUrteilskraft2009}
\cite[][V: 172.6--11]{Kant:GesammelteWerke1900ff.}.}
\end{quote}
Die Prinzipien, die die praktische Möglichkeit und Notwendigkeit bestimmen,
unterteilen sich daraufhin in \emph{technisch-praktische} und
\emph{moralisch-praktische}
Prinzipien -- je nachdem, ob ein Naturbegriff oder ein Freiheitsbegriff die
\singlequote{Kausalität} unseres Willens bestimmt. Die technisch-praktischen Prinzipien
beziehen sich naheliegenderweise auf hypothetische, die moralisch-praktischen
Prinzipien auf kategorische Imperative. Nur die moralisch-praktischen Prinzipien
gehören zur praktischen Philosophie oder \enquote{Sittenlehre}; die
technisch-praktischen Prinzipien hingegen gehören nach Auskunft der
\titel{Kritik der Urteilskraft} der theoretischen Philosophie
an.\footnote{\cite[Vgl.][B
xiii]{Kant:KritikderUrteilskraft2009}, \cite[][V:
172.14--22]{Kant:GesammelteWerke1900ff.}. Es ist daher zu ungenau, wenn
\authorfullcite{Fonnesu:KantspraktischePhilosophieunddieVerwirklichungderMoral2004}
behauptet, dass \enquote{das Praktische für \name[Immanuel]{Kant} mit der
Moral zusammen[fällt}
\parencite[][49]{Fonnesu:KantspraktischePhilosophieunddieVerwirklichungderMoral2004}.
Die praktische \emph{Philosophie} fällt mit der Moral zusammen, nicht aber der
Gesamtbereich der praktischen Erkenntnisse.}

\name[Immanuel]{Kant} nennt in der \titel{Grundlegung zur Metaphysik der Sitten}
nicht drei Arten von handlungsorientierenden Erkenntnissen, die er als
Imperative bezeichnet: problematische Imperative als \enquote{\ori{Regeln} der
Geschicklichkeit}, assertorische Imperative als \enquote{\ori{Ratschläge} der Klugheit} und
kategorische Imperative\footnote{\name[Immanuel]{Kant} spricht kategorische
Imperative an verschiedenen Stellen auch im Plural an, was zunächst verwirrend
erscheinend mag, aber recht plausibel ist, wenn man beachtet, dass \emph{der}
kategorische Imperativ (im Singular) Maximen als notwendig (geboten) ausweist,
die selbst wieder Imperative sind (uns nötigen) und kategorisch gelten.} als
\enquote{\ori{Gebote (Gesetze)} der Sittlichkeit.}\footnote{\cite[][BA
40--43]{Kant:GrundlegungzurMetaphysikderSitten1965}, \cite[IV:
414.32--416.20]{Kant:GesammelteWerke1900ff.}.}

Der kategorische Imperativ ist unabhängig von einem Zweck, da er \enquote{eine
Handlung als für sich selbst, ohne Beziehung auf einen andern Zweck, als
objektiv-notwendig
vorstellt[.]}\footnote{\cite[][BA~39]{Kant:GrundlegungzurMetaphysikderSitten1965},
\cite[][IV: 414.16--17]{Kant:GesammelteWerke1900ff.}.}. Das heißt nicht, dass
von Zwecken nicht die Rede sein könne, denn erstens fordert die
Selbstzweckformel des Kategorischen Imperativs, die \singlequote{Menschheit} in
der Person eines jeden Menschen als Zweck zu
betrachten,\footnote{\cite[Vgl.][BA~66f.]{Kant:GrundlegungzurMetaphysikderSitten1965},
\cite[][IV: 429.10--12]{Kant:GesammelteWerke1900ff.}.} und zweitens ist uns
durch die Moral das höchste Gut in der Welt als moralisch-notwendiger Zweck
vorgestellt\footnote{Siehe bspw. \cite[][B
842--847]{Kant:KritikderreinenVernunft2003}, \cite[][III:
528.13--531.23]{Kant:GesammelteWerke1900ff.};
\cite[][A 219--223]{Kant:KritikderpraktischenVernunft1974}, \cite[][V:
122.4--16]{Kant:GesammelteWerke1900ff.}.}.
Der kategorische Imperativ \emph{bestimmt} Zwecke, aber er \emph{setzt} keine Zwecke
\emph{voraus}.


Die problematischen und assertorischen Imperative wiederum fasst
\name[Immanuel]{Kant} zu der Gattung der hypothetischen Imperative zusammen. Sie
\enquote{stellen die praktische Notwendigkeit einer möglichen Handlung als
Mittel, zu etwas anderem, was man will (oder doch möglich ist, daß man es
wolle), zu gelangen, vor.}\footnote{\cite[][BA
39]{Kant:GrundlegungzurMetaphysikderSitten1965}, \cite[][IV:
414.13--15]{Kant:GesammelteWerke1900ff.}.} \name[Immanuel]{Kant}
implementiert zur Unterteilung wie so oft Vokabular, das eigentlich in der Logik
beheimatet ist: Sind die Zwecke bloß möglich, heißt der Imperativ
\enquote{problematisch} und bezeichnet eine Regel der
Geschicklichkeit, sind sie tatsächlich bei jedem Menschen vorhanden, dann heißt
er \enquote{assertorisch} und bezeichnet eine Regel der Klugheit. Der einzige
Zweck wiederum, den man als bei jedem Menschen vorhanden voraussetzen kann, ist
die \enquote{Glückseligkeit}.\footnote{\enquote{Es ist gleichwohl ein Zweck, den
man bei allen vernünftigen Wesen (so fern Imperative auf sie, nämlich als
abhängige Wesen, passen) als wirklich voraussetzen kann, und also eine Absicht,
die sie nicht etwa bloß haben \ori{können}, sondern von der man sicher
voraussetzen kann, daß sie solche insgesamt nach einer Naturnotwendigkeit
\ori{haben}, und das ist die Absicht auf \ori{Glückseligkeit}}
\mkbibparens{\cite[][BA 42]{Kant:GrundlegungzurMetaphysikderSitten1965};
\cite[][IV: 415.28--33]{Kant:GesammelteWerke1900ff.}}.}


Regeln der Geschicklichkeit heißen \emph{technisch} und gehören zur
\emph{Kunst}, Gebote der Sittlichkeit heißen \emph{moralisch} und die Ratschläge der Klugheit sind
diejenigen Imperative, die im engeren Sinne\footnote{Siehe Fußnote
\ref{Fussnote:DoppelteBedeutungvonPragmatisch} aus Seite
\pageref{Fussnote:DoppelteBedeutungvonPragmatisch}. Mit diesem Verweis auf die
Klugheit glaubt \name[Immanuel]{Kant} den Gebrauch des Wortes
\enquote{pragmatisch} charakterisieren zu können.
\cite[Vgl.][BA~4]{Kant:GrundlegungzurMetaphysikderSitten1965}, \cite[IV:
417.32--37]{Kant:GesammelteWerke1900ff.}: \enquote{Mich deucht, die eigentliche
Bedeutung des Worts \ori{pragmatisch} könne so am genauesten bestimmt werden.
{\punkt} Pragmatisch ist eine \ori{Geschichte} abgefaßt, wenn sie \ori{klug}
macht, d.\,i.\ die Welt belehrt, wie sie ihren Vorteil besser, oder wenigstens
eben so gut, als die Vorwelt, besorgen könne.}} \emph{pragmatisch} genannt
werden und zur \singlequote{\emph{Wohlfahrt}}
gehören.\footnote{\cite[Vgl.][BA~44]{Kant:GrundlegungzurMetaphysikderSitten1965},
\cite[IV: 416.28--417.2]{Kant:GesammelteWerke1900ff.}: \enquote{Man könnte die
ersteren Imperative auch \ori{technisch} (zur Kunst gehörig), die zweiten
\ori{pragmatisch} (zur Wohlfahrt), die dritten \ori{moralisch} (zum freien
Verhalten überhaupt, d.\,i.\ zu den Sitten gehörig) nennen.}} Die technischen
Regeln der Geschicklichkeit als Paradigmen der hypothetischen Imperative und die
moralischen Gebote der Sittlichkeit als Instanzen des kategorischen Imperativs
sind weithin bekannt. Anders verhält es sich mit den pragmatischen Ratschlägen
der Klugheit; dabei sind sie von erheblicher Bedeutung für
\name[Immanuel]{Kant}s
Entwicklung.\footnote{\cite[Vgl.][149]{Schwaiger:KlugheitbeiKant2002}.}
\phantomsection\label{Absatz:Weltklugheit}Was also ist Klugheit?
\name[Immanuel]{Kant} schreibt:
\begin{quote}
  Das Wort Klugheit wird in zwiefachem Sinn genommen, einmal kann es den Namen
  Weltklugheit, im zweiten den der Privatklugheit führen. Die erste ist die
  Geschicklichkeit eines Menschen, auf andere Einfluß zu haben, um sie zu seinen
  Absichten zu gebrauchen. Die zweite die Einsicht, alle diese Absichten zu
  seinem eigenen daurenden Vorteil zu vereinigen. Die letztere ist eigentlich
  diejenige, worauf selbst der Wert der erstern zurückgeführt wird, und wer in
  der erstern Art klug ist, nicht aber in der zweiten, von dem könnte man besser
  sagen: er ist gescheut und verschlagen, im ganzen aber doch
  unklug.\footnote{\cite[BA~42]{Kant:GrundlegungzurMetaphysikderSitten1965},
  \cite[IV: 416.30--37]{Kant:GesammelteWerke1900ff.}.}
\end{quote}
Der grundlegende Begriff ist also der der Privatklugheit, welche auf
Glückseligkeit geht. Ihr Ziel ist durch unser je eigenes Streben nach einer
dauerhaften und anhaltenden Zufriedenheit bestimmt. \name[Immanuel]{Kant}
spricht auch von \enquote{praktischen Klugheitsregeln nach dem Prinzip der
Selbstliebe}\footnote{\cite[][\S~91]{Kant:KritikderUrteilskraft2009}, \cite[][V:
470.9--10]{Kant:GesammelteWerke1900ff.}.} und der Glückselig als der
\enquote{Zufriedenheit mit seinem Zustande, sofern man der Fortdauer derselben
gewiß ist}\footnote{\cite[][A 16]{Kant:DieMetaphysikderSitten1977Tugendlehre},
\cite[][VI: 387.26--27]{Kant:GesammelteWerke1900ff.}. Siehe auch
\cite[][BA 1\,f.,]{Kant:GrundlegungzurMetaphysikderSitten1965}
\cite[][IV: 393.14--16]{Kant:GesammelteWerke1900ff.};
\cite[][A 168\,f.,]{Kant:DieMetaphysikderSitten1977Tugendlehre}
\cite[][VI: 480.23--25]{Kant:GesammelteWerke1900ff.};
\cite[][A 80]{Kant:DieReligioninnerhalbderGrenzenderblossenVernunft1977},
\cite[][VI: 67.20--23]{Kant:GesammelteWerke1900ff.}.}.
Es scheint mir hilfreich zu sein, den Begriff \enquote{Glückseligkeit} bei \name[Immanuel]{Kant} durch einen solchen Ausdruck wie \enquote{lang anhaltende (nicht nur kurzfristige) umfassende Zufriedenheit} zu erläutern. Es geht um einen Zustand, der erstens durch empirische Zufriedenheit bestimmt ist, aber zweitens nicht die
kurzfristige Befriedigung von Begierden meint, sondern auf die Ausgestaltung
längerer Lebensabschnitte (letztlich das gesamte Leben) zielt.

Damit ist der Begriff \enquote{Glückseligkeit} klarer, aber auch enger als der
Begriff der Eudaimonia bei \singlename{Aristoteles}\footnote{Hier bezogen auf
das erste Buch der \titel{Nikomachischen Ethik}, siehe
\cite[][1--12]{Aristoteles:NikomachischeEthik1972}.}, insofern
\name[Immanuel]{Kant} die Glückseligkeit eindeutig auf das somatische Wohlbefinden einschränkt. (Im Gegenzug stellt sie allerdings nicht mehr das höchste Gut dar, welches nun in der Verbindung von Glückseligkeit und Sittlichkeit
besteht.\footnote{\cite[Vgl.][B 841\,f.]{Kant:KritikderreinenVernunft2003};
\cite[][III: 527.33--528.15]{Kant:GesammelteWerke1900ff.}.}) Zu bestimmen und
erfolgreich anzuwenden sind nur noch die Mittel zum Erreichen dieses Ziels.
Zumindest der späte \name[Immanuel]{Kant} scheint den Ansatz zu verwerfen,
wonach die Klugheit sowohl die Inhalte des Glücks als auch die Mittel, diese zu
erreichen, bestimmt, weil durch das allen gemeinsame Ziel einer anhaltenden
\singlequote{Glückseligkeit} die Frage nach dem Inhalt bereits beantwortet
ist.\footnote{\cite[Vgl.][\pno~185f.]{Schwaiger:KategorischeundandereImperative1999}.
Nach \authorcite{Schwaiger:KategorischeundandereImperative1999} stellt sich
diese Doppelfrage beim späten \name[Immanuel]{Kant} nur noch hinsichtlich des
Glücks als \emph{summum bonum}.} Allerdings
sei es
\begin{quote}
  ein Unglück, daß der Begriff der Glückseligkeit ein so unbestimmter ist, daß,
  obgleich jeder Mensch zu dieser zu gelangen wünscht, er doch niemals bestimmt
  und mit sich selbst einstimmig sagen kann, was er eigentlich wünsche und
  wolle.\footnote{\cite[BA~46]{Kant:GrundlegungzurMetaphysikderSitten1965},
  \cite[IV: 418.1--4]{Kant:GesammelteWerke1900ff.}.}
\end{quote}
So könnte man auch sagen, dass durch die Unbestimmtheit des Begriffs der
Glückseligkeit die Frage nach dem Inhalt des Glücks weiter besteht. Dass
\name[Immanuel]{Kant} von der Glückseligkeit als einem
\singlequote{unbestimmten} Begriff spricht, lässt zunächst vermuten, dass er
doch den Gehalt des Begriffs und damit seine Inhalte für fraglich hält. Sachlich
ist der Punkt jedoch klar: Wenn wir auch die Mittel, dauerhafte Zufriedenheit zu
erlangen, als \emph{Inhalte} der Glückseligkeit bezeichnen können, steht doch
deren Ausrichtung auf die dauerhafte Zufriedenheit längst fest. Dieses
langfristige Wohlbefinden \emph{ist} die Glückseligkeit, wenngleich wir dazu
neigen, die je individuell als geeignet angesehenen Mittel als Inhalte der
Glückseligkeit anzusprechen.

Das Problem, mit dem wir stets konfrontiert sind, besteht darin, dass wir
verschiedene Zwecke zu verfolgen willens sind, die sich gegenseitig behindern. Wir wollen etwa
\emph{sowohl} beruflich erfolgreich sein, \emph{als auch} viel Freizeit haben
und stellen fest, dass beides zugleich nicht realisierbar ist. In diesem Fall --
so scheint es -- müssen wir nach dem Inhalt des \singlequote{Glücks} oder der
Glückseligkeit fragen. Allerdings ließe sich ebenso sagen, dass bloß die Mittel
fraglich sind, mit denen wir dauerhafte Zufriedenheit erlangen können. Wir
wissen eben nicht im Vorhinein, ob wir nach dem jahrelangen zielstrebigen
Verfolgen unserer Karriere zufrieden sind oder ausgebrannt. Noch weniger wissen
wir jemals, ob der andere Weg als der, den wir wählten, uns mehr an
Zufriedenheit eingebracht hätte. Diese dauerhafte Zufriedenheit ist letztlich
das einzige Ziel, zu dem die Klugheit die Mittel sucht.\footnote{\cite[Vgl.][185]{Schwaiger:KategorischeundandereImperative1999}: \enquote{[U]m
die Zeit der \ori{Grundlegung} ist der Gedanke, daß die Klugheitslehre neben den Mitteln auch
den Zweck der Glückseligkeit bestimmen müsse, völlig zurückgetreten. Die
Möglichkeit eines Irrtums bei der inhaltlichen Füllung des Glücksbegriffs gerät
aus dem Blick; ein bloßes Scheinglück wird mit keinem Wort mehr erwähnt. Die
Klugheit macht nur noch die Mittel ausfindig; allein der Sittlichkeit kommt es
zu, auch den Zweck zu bestimmen.}} Diese Mittel wiederum lassen sich als
(subalterne) Zwecke ansehen. Wir verfolgen sie nicht um ihrer selbst willen,
sondern weil wir sie für geeignete Mittel zur dauerhaften Zufriedenheit ansehen.
Dauerhaftes Glück jedoch wird um seiner selbst willen angestrebt -- es ist ein
letzter Zweck.



Glückseligkeit hat einen Sonderstatus unter unseren Zwecken, weil jeder Mensch
als endliches Vernunftwesen danach strebt. Denn jeder Mensch versucht, dauerhaft
glücklich zu sein, wenn auch die Vorstellungen auseinandergehen, wie dieser Zustand zu
erreichen ist. Dies liegt daran, dass der jeweils geeignete Weg zur
Glückseligkeit von biographischen Zufälligkeiten und der jeweils eigenen Wunsch-
und Bedürfnisstruktur abhängig ist. Was den einen Menschen glücklich macht, das
verursacht bei dem anderen negative Gefühle wie Stress oder Langeweile.
Der eine fühlt sich eben beim Angeln wohl, der andere auf einer anspruchsvollen
Hochtour. Selbstredend besitzen wir auch Wissen darüber, was Menschen ganz
allgemein glücklich oder unglücklich macht, und insofern ist es durchaus
möglich, allgemeines pragmatisches \emph{Wissen} zu haben, welches sich in
Ratschlägen artikuliert.\footnote{\cite[Vgl.][98]{Brandt:KlugheitbeiKant2005}:
\enquote{Obwohl das Glücksziel am Horizont des menschlichen Lebens, für dessen Erreichung die
Klugheit ihre Ratschläge in hypothetischen Imperativen liefert, zwar notwendig,
aber nur unbestimmt ist und bei jedem Menschen wechselt, gibt es ein weites
Klugheitsfeld im privaten und öffentlichen Handeln, das \name[Immanuel]{Kant} in
seiner \singlequote{pragmatischen Anthropologie} durchleuchtet; sie ist eine empirische
Klugheitslehre, die wegen ihrer Fundierung in der Natur des Menschen, also der
bloßen Erfahrung, nicht zur eigentlichen kritischen Philosophie als einer
Ver\-nunft\-er\-kennt\-nis aus Begriffen gehört; aber immerhin, Kant bezeichnet
sie als Wissenschaft.}} Der Verzicht auf harte Drogen ist beispielsweise für jeden
Menschen im Interesse seiner eigenen langfristigen Zufriedenheit empfehlenswert,
da wir begründeter Weise davon
ausgehen, dass der aus längerem Drogenkonsum resultierende Zustand für niemanden
wünschenswert ist.\footnote{Dieser Ratschlag ist nicht zu verwechseln mit einem
möglicherweise zu begründenden moralischen Verbot, sich selbst in einen
Rauschzustand zu versetzen. Ein solches Verbot ließe sich etwa dadurch
begründen, dass man darauf verweist, dass Drogenkonsum die Fähigkeit zu
moralischem Handeln reduziert, indem die Abhängigkeit zu
Beschaffungskriminalität verleitet oder der Rausch unmoralisches oder gar
kriminelles Handeln hervorruft. \name[Immanuel]{Kant} äußert sich dazu in der
\titel{Metaphysik der Sitten} \mkbibparens{\cite[vgl.][A
80--82]{Kant:DieMetaphysikderSitten1977Tugendlehre};
\cite[][VI: 427.1--428.26]{Kant:GesammelteWerke1900ff.}}.
Auffällig ist, dass zuverlässige Ratschläge oft negativ formuliert
sind, uns also sagen, was im Interesse unserer dauerhaften Zufriedenheit zu
unterlassen oder \singlequote{\emph{unklug}}
ist. \cite[Vgl.][190]{Schwaiger:KategorischeundandereImperative1999}:
\enquote{Im allgemeinen werden Ratschläge für das Glück eher negativ als positiv gehalten
sein müssen.} Ratschläge im positiven Sinne -- die nicht Unglück verhindern,
sondern Glück herbeiführen sollen -- sind weitaus schwerer zu finden. Dennoch
gibt es sie natürlich; man denke etwa an den Ernährungshinweis,
abwechslungsreiche Nahrung mit Obst und Gemüse zu sich zu nehmen. Medizinische
Ratschläge sind in der Regel Ratschläge der Klugheit, da uns unsere Gesundheit
stets im Interesse langfristiger Glückseligkeit am Herzen liegt.}


Weil die Überzeugungskraft von pragmatischen Imperativen der Klugheit von der
Übereinstimmung mit (mehr oder minder kontingenten) subjektiven Wünschen
abhängig ist, die jeder selbst beurteilen muss, handelt es sich bei positiven
wie negativen Ratschlägen der Klugheit nicht um Gebote, sondern eben nur um
Ratschläge.\footnote{\enquote{Die \ori{Ratgebung} enthält zwar Notwendigkeit,
die aber bloß unter subjektiver gefälliger Bedingung, ob dieser oder jener
Mensch dieses oder jenes zu seiner Glückseligkeit zähle, gelten kann} (\cite[BA
44]{Kant:GrundlegungzurMetaphysikderSitten1965}, \cite[IV:
416.23-26]{Kant:GesammelteWerke1900ff.}). \cite[Vgl.
auch][189]{Schwaiger:KategorischeundandereImperative1999}:
\enquote{Nach unserem alltäglichen Verständnis gehört aber offenbar wesentlich
zu einem Ratschlag, daß er selbst dann, wenn er gut und richtig ist, nicht
übernommen werden muß, sondern verworfen werden kann. Ratschläge lassen dem
anderen die Freiheit, über sein eigenes, individuelles Glück in letzter Instanz
selbst zu entscheiden.}} Es liegen zwischen den drei Arten von Imperativen
Unterschiede in der Art der Nötigung
vor,\footnote{\enquote{Das Wollen nach diesen dreierlei Prinzipien wird auch
durch die \ori{Ungleichheit} der Nötigung des Willens deutlich unterschieden}
\mkbibparens{\cite[][BA 43]{Kant:GrundlegungzurMetaphysikderSitten1965};
\cite[][IV: 416.15--16]{Kant:GesammelteWerke1900ff.}}.} wenngleich alle drei als
Imperative uns als nötigend begegnen. Der Rat des Arztes, weniger Alkohol zu trinken, hat nicht die Verbindlichkeit eines moralischen Gebotes. Wer es für wahrscheinlich hält, dass ein Leben in geselliger Bierlaune ihn glücklicher machen wird als lang anhaltende Gesundheit, der möge diesen Weg für sich einschlagen. Eine Erfolgsgarantie hat er dabei freilich so wenig wie derjenige,
der auf feucht-fröhliche Abende um der Gesundheit willen verzichtet.
Aber er hat doch das Recht, den von ihm für richtig befundenen Weg in seinem
eigenen Leben einzuschlagen und zu versuchen, nach seiner eigenen Fa\c{c}on
glücklich zu werden.


Gegenüber Kindern freilich, denen wir keine Mündigkeit zusprechen, artikulieren
wir Imperative der Klugheit als Anweisungen, denen sie Folge zu leisten haben.
Es steht nicht in ihrer eigenen Verantwortung,
sich zwischen schulischem Erfolg und Freizeit zu entscheiden. Aber wenn jemand
mündig zu sein beansprucht, muss er im Bereich der Klugheit Ratschläge auf
eigene Verantwortung annehmen oder ablehnen.
Dadurch unterscheiden sich aus der Perspektive mündiger Menschen Ratschläge der
Klugheit von moralischen Vorschriften. Unmündig ist mindestens derjenige, der
sich in der eigenen Lebensgestaltung ausschließlich am Beispiel anderer
orientiert, der sich von Freunden, Eltern oder ganz abstrakt \singlequote{der
Gesellschaft} zu einer bestimmten Lebensweise genötigt sieht, weil er sich
Selbständigkeit nicht zutraut.

Das heißt freilich nicht, dass der mündige Mensch sich von anderen Ratschläge
der Klugheit generell verbitten könnte. Das je eigene Leben klug einzurichten,
ist eine Sache der Vernunft, schon weil es um die langfristige Zufriedenheit
geht. Zwar müssen wir oft schlicht darauf achten, was uns selbst Freude bereitet
-- wir müssen auf unser eigenes Gefühl der Lust und Unlust hören. Dennoch gibt
es vernünftige und unvernünftige Wege, mit den eigenen Präferenzen und Neigungen
umzugehen, die als solche der intersubjektiven Überprüfbarkeit und Bewertbarkeit
zugänglich sein müssen. Denn damit, dass wir sie als vernünftig oder
unvernünftig bewerten, sind wir wieder auf die erweiterte Denkungsart verwiesen.
Wir sollen uns also auch in Fragen der Klugheit, des individuellen Strebens nach
Glückseligkeit nicht als logische Egoisten, sondern als Pluralisten verhalten.
Das kompetente Verfolgen je eigener Lebensentwürfe ist eine Sache der Vernunft
und daher auf den intersubjektiven Austausch mit anderen angewiesen. In diesem
Austausch erwerben und bewahren wir die Kompetenz der eigenen vernünftigen
Lebensführung.

Klugheit als die Kompetenz der je eigenen Lebensgestaltung rückt die
alltäglichste Form unserer individuellen Freiheit in den Mittelpunkt des theoretischen
Interesses.\footnote{\cite[Vgl.][158]{Schwaiger:KlugheitbeiKant2002}:
\enquote{Mit der Forderung, daß ein kluger Mensch divergierende bis
konfligierende Absichten zu seinem dauerhaften persönlichen Vorteil vereinigen
müsse, rückt nun die Freiheit des einzelnen hinsichtlich dessen, was er aus sich
selber machen möchte, in den Mittelpunkt.}} Als Provokation kann die Implikation
verstanden werden, dass neben der Moralität und der Glückseligkeit, die als
der empirische Zustand je eigener dauerhafter Zufriedenheit verstanden wird,
kein weiterer \singlequote{Wert} des eigenen Lebens zugelassen wird. Es gibt -- so
\name[Immanuel]{Kant} -- neben Recht und Moral keine verbindlichen Maßstäbe
unserer Lebensgestaltung. Die je eigene Glückseligkeit und Moralität -- und mit
der Moralität auch die \emph{fremde} Glückseligkeit -- sind die einzigen
Maßstäbe, anhand derer sich unsere je eigene Lebensgestaltung messen lässt.


Das Wissen und die Fähigkeit, sich selbst Ziele zu
stecken und erfolgreich zu verfolgen, die nicht kurzfristige Befriedigung,
sondern anhaltendes Glück gewähren, ist die Privatklugheit. Weltklugheit ist
dagegen \enquote{die Geschicklichkeit eines Menschen, auf andere Einfluß zu
haben, um sie zu seinen Absichten zu
gebrauchen.}\footnote{\cite[BA 42]{Kant:GrundlegungzurMetaphysikderSitten1965},
\cite[IV: 416.32--33]{Kant:GesammelteWerke1900ff.}. Entgegen einem möglichen
ersten Anschein konfligiert dies nicht mit der Forderungen der Moral, andere
\enquote{jederzeit zugleich als Zweck, niemals bloß als
Mittel} \mkbibparens{\cite[BA
66\,f.,]{Kant:GrundlegungzurMetaphysikderSitten1965} \cite[IV:
429.11--12]{Kant:GesammelteWerke1900ff.}, \ohio}, zu gebrauchen, da es nicht
verlangt, jemanden als \emph{bloßes} Mittel zu betrachten. Fast jede Kooperation
beinhaltet, andere Menschen \emph{auch} als Mittel zu betrachten, aber als freie
Kooperation beinhaltet sie eben auch, dass der andere zugleich Zweck ist --
insofern weder Zwang noch Unehrlichkeit die Grundlage bilden.}
Wer Weltklugheit besitzt, der -- so könnten wir heute sagen -- ist hinreichend
sozialkompetent, um sich sicher und zielgerichtet in der menschlichen
Gesellschaft bewegen zu können. Auch hier soll es sich um eine grundlegende
Kompetenz handeln, die nicht einfach die Anwendung einer anderen Fähigkeit
beschreibt. Insbesondere von technischem Können (\singlequote{Geschicklichkeit})
ist sie unterschieden, insofern Geschicklichkeit auf Sachen, Weltklugheit jedoch
auf Personen
geht.\footnote{\cite[Vgl.][128]{Schwaiger:KategorischeundandereImperative1999}.}
Gerade bei freien Kooperationen, in denen unsere Kooperationspartner zugleich
als Zwecke angesehen werden, die weder durch Zwang noch durch Täuschung zur
Mithilfe gebracht werden, sind beide grundlegend unterschieden.\footnote{Dies
macht die Aufgabe der \emph{Zivilisierung} aus: Wer zivilisiert ist, kann seine
eigenen Ziele in Zusammenarbeit mit seinen Mitmenschen verfolgen.
\cite[Vgl.][A~23]{Kant:UeberPaedagogik1977}, \cite[IX:
450.3-5]{Kant:GesammelteWerke1900ff.}: \enquote{Muß man darauf sehen, daß der
Mensch auch \ori{klug} werde, in die menschliche Gesellschaft passe, daß er
beliebt sei, und Einfluß habe. Hierzu gehört eine gewissen Art von Kultur, die
man \ori{Zivilisierung} nennet.} Siehe auch
\cite[A~319]{Kant:AnthropologieinpragmatischerHinsicht1977},
\cite[VII: 323.21-25]{Kant:GesammelteWerke1900ff.}: \enquote{\ori{Die
pragmatische Anlage} der Zivilisierung durch Kultur, vornehmlich der Umgangseigenschaften und der
natürliche Hang seiner Art, im gesellschaftlichen Verhältnisse aus der Rohigkeit
der bloßen Sachgewalt herauszugehen und ein gesittetes (wenn gleich noch nicht
sittliches), zur Eintracht bestimmtes, Wesen zu werden, ist nun eine höhere
Stufe.}}


Aber auch Weltklugheit generiert keine neuen Ziele, sondern setzt eine
Orientierung an Glückseligkeit (und Sittlichkeit) bereits voraus. Daher nützt
sie nur demjenigen, der zugleich die nötige Privatklugheit besitzt. Wer
sozialkompetent ist und andere zu seinen Zielen gebrauchen kann, aber nicht
selbständig sein Leben auf das Ziel der Glückseligkeit auszurichten vermag
(sondern hierin von anderen abhängig bleibt), wird durch Weltklugheit nicht
mündig. Daher ist Weltklugheit in ihrem Nutzen von der Privatklugheit
abhängig.\footnote{\cite[Vgl.][BA 42]{Kant:GrundlegungzurMetaphysikderSitten1965}, \cite[][IV:
416.34--37]{Kant:GesammelteWerke1900ff.}.}\phantomsection\label{Absatz:Weltklugheit-ENDE}


Wissen, welches uns zur Orientierung im Handeln dient, nennt
\name[Immanuel]{Kant} \emph{pragmatisch} (im weiten Sinne). Zu diesem zählen 
praktische Erkenntnisse, die uns endlichen Wesen als \emph{Imperative} begegnen.
Aber auch theoretisches Wissen zählt zu den pragmatischen Erkenntnissen,
insofern wir es für unser Handeln benötigen. Wie im letzten Kapitel gesehen,
zählt dazu insbesondere das Wissen der \titel{Anthropologie in pragmatischer
Hinsicht}, welches ich als Wissen um die \emph{conditio humana} umschrieb. Aus
dem Verhältnis zwischen dem, was wir als Menschen \emph{sollen} (praktische
Erkenntnis), und dem, was wir als Menschen \emph{sind} (\emph{conditio humana}),
ergibt sich die Bestimmung des Menschen. Diese Zusammenhänge verdeutlicht
Abbildung \ref{abbildung:ErkenntnisArtenHandlungsorientierungKant}.
\begin{figure}[htb]
\begin{minipage}[t]{\textwidth}
\centering
\begin{tikzpicture}[edge from parent fork down,
level 1/.style={sibling distance=8.5cm, level distance=1.5cm},
level 2/.style={sibling distance=4.2cm, level distance=2.5cm},
level 3/.style={sibling distance=3cm, level distance=2.5cm},
level 4/.style={sibling distance=3cm, level distance=4.5cm},
every node/.style={rectangle,draw=black,fill=gray!25, thin, inner sep=0.5em, minimum size=0.5em, align=center},
edge from parent/.style={thin,draw},
mylabel/.style={draw=none, fill=none, text width=5cm,text centered, inner sep=0.5em, anchor=base} ]
\node {pragmatische Erkenntnisse (i.\,w.\,S.)}
	child {node {praktische Erkenntnisse}
		child {node[text width=3cm] {kategorische(r) Imperativ(e)}
			child {node[text width=2cm] (gebote) {Gebote der Sittlichkeit}}}
		child {node[text width=3cm] {hypothetische Imperative}
			child {node[text width=2cm] (ratschlaege) {Ratschläge der Klugheit}}
			child {node[text width=2cm] (regeln) {Regeln der Geschicklichkeit}
				child {node (bestimmung) {Bestimmung des Menschen} edge from
				parent[draw=none]}}}}
	child {node {theoretische Erkenntnisse}
	    child {node[text width=3cm] {pragmatische Anthropologie}
	    	child {node[text width=2cm, fill=none] (ch) {\emph{conditio
	    	humana}}}} child {node[text width=3cm]
	    	{physische Geographie}}} ;
\draw [->,thick,decorate,decoration={snake,post
length=1mm,amplitude=.4mm,segment length=2mm}] (ch.south) to
node[below,sloped,draw=none,fill=none] {\tiny Antagonismus} (bestimmung.north);
\draw [->] (ratschlaege.south) to
node[above,sloped,draw=none,fill=none] {\tiny zivilisieren} (bestimmung.north);
\draw [->] (regeln.south) to
node[below,sloped,draw=none,fill=none] {\tiny kultivieren} (bestimmung.north);
\draw [->] (gebote.south) to
node[above,sloped,draw=none,fill=none] {\tiny moralisieren} (bestimmung.north);
\end{tikzpicture}
  \caption{Erkenntnisarten nach Handlungsorientierung bei
  \name[Immanuel]{Kant}}\label{abbildung:ErkenntnisArtenHandlungsorientierungKant}
\end{minipage}
\end{figure}

Da es drei Arten von Imperativen gibt und die Bestimmung des Menschen der
Gegenüberstellung von praktischen Erkenntnissen und \emph{conditio humana}
entspringt, entspricht es auch der sich einstellenden Erwartungshaltung, wenn
\name[Immanuel]{Kant} drei Aspekte der Bestimmung des Menschen als Konsequenzen
aus der Anthropologie nennt. Zu Beginn dieses Kapitels hatte ich Kultivierung,
Zivilisierung und Moralisierung als Ziele benannt, die sich aus der Bestimmung
des Menschen ergeben.\footnote{Siehe
S.~\pageref{Zitat:KultivierungZivilisierungMoralisierungalsBestimmungdesMenschen}.}
In der \titel{Pädagogik} verbindet \name[Immanuel]{Kant} nun die Kultur mit der
technischen Anlage des Menschen zur \emph{Geschicklichkeit}, die Zivilisierung
mit der pragmatischen Anlage zur \emph{Klugheit} und die Moralisierung mit der
moralischen Anlage nach dem
\emph{Freiheitsprinzip}\footnote{\cite[Vgl.][A~316-321]{Kant:AnthropologieinpragmatischerHinsicht1977},
\cite[VII: 322.13--325.10]{Kant:GesammelteWerke1900ff.};
\cite[A 22\,f.,]{Kant:UeberPaedagogik1977} \cite[IX:
449.27--450.14]{Kant:GesammelteWerke1900ff.}.}, also mit den drei Arten von
Imperativen.\footnote{Die Systematisierung mag sich an unterschiedlichen Stellen
geringfügig anders darstellen: In der \titel{Anthropologie} hingegen umfasst die Kultur neben der
Entwicklung der technischen auch die Entwicklung der pragmatischen Anlage
\mkbibparens{\cite[vgl.][A 319]{Kant:AnthropologieinpragmatischerHinsicht1977},
\cite[VII: 323.21--324.11]{Kant:GesammelteWerke1900ff.}}. Man könnte somit
Kultur und Moral als die beiden entscheidenden Bildungsziele bezeichnen. Und in der \titel{Kritik
der Urteilskraft} heißt es: \enquote{Die Hervorbringung der Tauglichkeit eines
vernünftigen Wesens zu beliebigen Zwecken überhaupt (folglich in seiner
Freiheit) ist die \ori{Kultur}}
\mkbibparens{\cite[\S~83]{Kant:KritikderUrteilskraft2009}, \cite[V:
431.28--30]{Kant:GesammelteWerke1900ff.}}. Doch letztlich bleibt der
durchgängige Bezug auf die drei Arten von Imperativen bestehen.} Moral, Klugheit und
Geschicklichkeit sind die wesentlichen Zwecke der Vernunft, die zu einer freien
und selbstbestimmten Gestaltung des eigenen Lebens notwendig sind. Somit sind es
die \emph{praktischen Erkenntnisse}\footnote{In der \titel{Kritik der
Urteilskraft} hebt \name[Immanuel]{Kant} hervor, dass technische Imperative
auf Naturbegriffen beruhen und daher nicht zur praktischen, sondern zur
theoretischen Philosophie zählen. Dennoch nennt er sie
\singlequote{technisch-praktische Erkenntnisse}. Die Einteilung der Philosophie in eine
theoretische und eine praktische Philosophie stimmt also nicht mit der
Einteilung der Erkenntnisse in theoretische und praktische überein. Die
praktische Philosophie beschränkt sich auf die Behandlung der
moralisch-praktischen Erkenntnisse unter Auslassung der Imperative der
Klugheit und der Geschicklichkeit 
\mkbibparens{vgl. \cite[][xi---xii]{Kant:KritikderUrteilskraft2009},
\cite[][V: 171.4--172.22]{Kant:GesammelteWerke1900ff.}}.}, über die wir verfügen
müssen, um Bescheid zu wissen über unsere Bestimmung als Menschen. Unsere
Bestimmung ist es, gerade in Fragen der Moral und der Klugheit kompetent
urteilen und so ein selbstbestimmtes, vernunftgeleitetes Leben führen zu können.

Welche technischen Regeln uns zu wissen obliegen, ist nun tatsächlich
individuell verschieden, denn es hängt von der Verfolgung von Zwecken ab, die in zweierlei
Hinsicht kontingent sind. Sie sind erstens kontingent in dem Sinne, dass wir bei
ihnen im Gegensatz zu dem Ziel der Glückseligkeit eine tatsächliche Absicht, sie
zu erreichen nicht \emph{voraussetzen} können. Sie sind außerdem kontingent, insofern
wir bei ihnen im Gegensatz zu den Zwecken, die zu verfolgen uns die Moral
vorschreibt, nicht \emph{fordern} können, sie zu verfolgen. Die Regeln, die wir
kennen und anwenden können müssen, um etwa ein Flugzeug zu konstruieren oder am
offenen Herzen zu operieren, müssen uns nur dann interessieren, wenn wir
Ingenieure oder Ärzte sind. Wir sind nicht unmündig, wenn wir sie nicht kennen.
Pragmatische Ratschläge und moralische Gebote hingegen gehen jeden von uns an,
denn sie hängen von keinen kontingenten Zwecken ab. Die Ratschläge der Klugheit
hängen von einem Zweck ab, der nicht kontingent ist, und die Gebote der
Sittlichkeit sind von gar keinen Zwecken abhängig, sondern schreiben Zwecke vor.
Über \emph{beide} müssen wir also im Interesse mündiger Lebensgestaltung je
selbst verfügen. Wer sich die Moral oder die Ratschläge der Klugheit von anderen
vorgeben lässt, der ist unmündig. Mündigkeit in der Lebensführung lässt sich
nicht auf die Fähigkeit zu moralischem Handeln allein oder klugem Handeln allein
reduzieren. \emph{Beides} geht \emph{alle} Menschen an; denn der Mensch kann
seine Bestimmung nur erfüllen, wenn er sich in beiden Bereichen hinreichend auskennt.


\subsection{Die Fehlbarkeit der praktischen
Vernunft}\label{section:AufklaerungdesendlichenWillens}
Ich möchte im folgenden zeigen, dass einer
selbständigen, vernünftigen und mündigen Lebensführung, die sich an der
Bestimmung des Menschen orientiert, keine epistemischen Schwierigkeiten
entgegenstehen. Stattdessen finden die Hindernisse einer aufgeklärten Lebensführung
ihren Niederschlag in \name[Immanuel]{Kant}s Konzeption der Endlichkeit unseres
Willens, wie sie in Kapitel \ref{subsubsection:DieEndlichkeitdesWillens}
beschrieben wurde. Unser Wille oder unsere praktische Vernunft ist endlich,
insofern praktische Erkenntnisse in der Form von Imperativen auftreten. Sie
\emph{nötigen} uns zu einem bestimmten Verhalten, welches wir als endliche
Vernunftwesen, deren Handeln wesentlich von Neigungen bestimmt wird, nicht
ohnehin ausführen. Gerade in den aufklärungsrelevanten Bereichen praktischer und
pragmatischer Erkenntnisse ist es nicht die Endlichkeit unseres Verstandes im
Erkennen, die Mündigkeit erschwert, sondern die Endlichkeit des Willens in der
Ausführung entsprechender Handlungsvorschriften.

\name[Immanuel]{Kant} versteht unter einem Imperativ die Formel für einen
Grundsatz, der für einen Willen nötigend
ist.\footnote{\cite[Vgl.][BA 37]{Kant:GrundlegungzurMetaphysikderSitten1965},
\cite[][IV: 413.9--11]{Kant:GesammelteWerke1900ff.}.} Nur dort, wo es zu
Diskrepanzen zwischen Prinzipien und unseren tatsächlichen Handlungen kommen
kann, die darin begründet sind, dass wir unseren Neigungen folgen, sind
Imperative denkbar. Deshalb sind Gebote der Moral paradigmatische, aber nicht
die einzigen Fälle von Imperativen. Denn nicht nur im Falle des kategorischen
Imperativs gibt es solche Diskrepanzen, sondern auch bei den hypothetischen Imperativen kann es sein, dass
Prinzipien bei uns nicht handlungswirksam werden, weil antagonistische Neigungen
dies verhindern.\footnote{\cite[Vgl.][114]{Paton:TheCategoricalImperative1948}:
\enquote{The principles of goodness thus appear in our finite human condition as
principles of obligation. This is true even where the principle in question is
one of skill or rational self-love and not of morality. Men are not wholly
rational in the pursuit of happiness or even in the adoption of means to ends.}}
Wer einen Halbmarathon unter 90 Minuten laufen möchte, muss dafür mehrfach in
jeder Woche trainieren. Ein hypothetischer Imperativ besagt daher:
\enquote{Trainiere mehrfach in jeder Woche, wenn du einen Halbmarathon unter 90
Minuten laufen möchtest!} Da es ein kontingentes Ziel ist, einen Halbmarathon
unter 90 Minuten zu laufen, handelt es sich um einen problematischen Imperativ
oder eine Regel der Geschicklichkeit. Sie gilt als objektives Prinzip für
diejenige, die dieses Ziel verfolgt. Und wie jeder weiß, können Neigungen die
Ausführung dieses objektiven Prinzips punktuell oder auch dauerhaft verhindern
--  sie verhindern damit, dass ein objektives Prinzip zugleich zu einem
subjektiven Prinzip wird. Uns tritt ein solches Prinzip
daher  als \emph{nötigend} entgegen. Selbst wenn wir oft
ohne zu zögern das Training beginnt, benötigen wir doch mitunter Disziplin oder
gar Überwindung. Und hier merken wir, dass Regeln der Geschicklichkeit uns
nötigen und daher Imperative sind. Ganz analog lautet ein Ratschlag der
Klugheit: \enquote{Rauche nicht!} Da Zigarettenrauch gesundheitsschädlich ist,
wirkt er langfristig der Glückseligkeit entgegen! Deswegen ist dieser Imperativ
einer der Klugheit. Nun weiß jeder, der selbst einmal rauchte oder noch immer
raucht, wie schwer es fällt, die zu diesem Imperativ
antagonistischen Begierden zu überwinden. Der Imperativ der Klugheit ist nötigend für den, der sich an ihm
orientieren möchte, so frei er ihn auch gewählt hat. Ein Imperativ verliert
seinen nötigenden Charakter nicht dadurch, dass andere ihn nur als Ratschlag
anbringen.
Die praktische Vernunft des Menschen ist also endlich, insofern er trotz
richtiger Einsicht dabei scheitert, vernünftig zu \emph{handeln}. Zwar
\emph{wissen} wir \name[Immanuel]{Kant} zufolge gut genug, was zu tun ist, um moralisch, klug und
geschickt zu handeln; aber dennoch scheitern wir in der Ausführung, weil wir
nicht die nötige Disziplin aufbringen.
\begin{comment}
Die Fehlbarkeit endlicher Wesen in der Ausübung der praktischen Vernunft besteht
oft nicht darin, dass die entsprechenden Erkenntnisse nicht verfügbar wären, sondern
darin, dass trotz korrekter Einsicht die Ausführung unterbleibt. Den
paradigmatischen Fall finden wir bei moralischen Verfehlungen, bei denen wir
davon ausgehen können, dass kein Mangel an moralischer Einsicht vorliegt,
sondern an korrekter Umsetzung vorhandener Einsichten. Ein nicht-endlicher
(ein \singlequote{heiliger}) Wille unterliegt denselben moralischen Gesetzen,
die auch uns bekannt sind (er ist uns kognitiv möglicherweise bloß gleichwertig);
aber wenngleich er dieselben moralischen Gesetze \emph{erkennt}, so \emph{handelt}
darüber hinaus auch stets nach ihnen:
\begin{quote}
  Ein vollkommen guter Wille würde also eben sowohl unter objektiven Gesetzen
  (des Guten) stehen, aber nicht dadurch als zu gesetzmäßigen Handlungen
  \ori{genötigt} vorgestellt werden können, weil er von selbst, nach seiner
  subjektiven Beschaffenheit, nur durch die Vorstellung des Guten bestimmt 
  werden kann. Daher gelten für den \ori{göttlichen} und überhaupt für einen
  \ori{heiligen} Willen keine Imperativen; das \ori{Sollen} ist hier am
  unrechten Orte, weil das \ori{Wollen} schon von selbst mit dem Gesetz
  notwendig einstimmig
  ist.\footnote{\cite[][BA 39]{Kant:GrundlegungzurMetaphysikderSitten1965},
  \cite[][IV: 414.1--8]{Kant:GesammelteWerke1900ff.}. Siehe auch
  \cite[][\S~76]{Kant:KritikderUrteilskraft2009},
  \cite[][V: 403.20--404.16]{Kant:GesammelteWerke1900ff.}.}
\end{quote}
Imperative gibt es nur für endliche Wesen, weil ein Wesen mit einem
nicht-endlichen Willen zur Befolgung vernünftiger Grundsätze nicht aufgefordert
werden müsste.\footnote{Auch in Vorlesungen hat \name[Immanuel]{Kant} diesen
Zusammenhang regelmäßig hervorgehoben. Siehe etwa
\cite{Kant:MoralphilosophieCollins1974}, \cite[][XXVII:
256.14--37]{Kant:GesammelteWerke1900ff.}: \enquote{Der göttliche Wille ist in
Ansehung der Moralitaet nothwendig, aber der menschliche Wille ist nicht
nothwendig sondern genöthigt. Also ist die practische Nothwendigkeit in Ansehung
des höchsten Wesens keine Obligation, das höchste Wesen handelt moralisch
nothwendig, aber hat keine Obligation. Warum sage ich nicht: Gott ist verbunden,
wahrhaftig heilig zu seyn? {\punkt} Also in Ansehung eines vollkommenen Willens,
bey dem die moralische Nothwendigkeit nicht allein objectiv sondern subjectiv
nothwendig ist, da findet keine Neceßitation und Obligation statt. Es müßen
demnach die sittlichen Handlungen nur zufällig seyn, wenn sie eine Nöthigung
haben sollen, und die einen moralisch unvollkommenen Willen haben, stehen unter
der Verbindlichkeit, und das sind Menschen.} Siehe außerdem
\cite{Kant:MetaphysikderSittenVigilantus1975}, \cite[][XXVII:
481.14--18]{Kant:GesammelteWerke1900ff.}: \enquote{Die Gesetze der Freiheit sind
nun {\punkt} 1. entweder blos nothwendige oder objective mere necessariae leges.
Diese finden allein bei Gott statt. 2. oder nöthigende, necessitantes. Diese finden bei Menschen Statt, und
sind objective necessaria, subjective aber zufällig.} Und schließlich schreibt
er in \cite{Kant:NaturrechtFeyerabend1979}, \cite[][XXVII:
1323.12--20]{Kant:GesammelteWerke1900ff.}: \enquote{Bei Gott ist sein guter
Wille nicht zufällig; daher findet auch bei ihm kein imperatives Gesetz statt,
um ihn zum guten Willen zu nöthigen. Denn das wäre überflüßig. Neceßitatio
einer an sich zufälligen Handlung durch objective Gründe ist praktische
Neceßitatio, das ist von praktischer Neceßitaet
unterschieden. Bei Gott sind auch Gesetze, aber die haben praktische
Nothwendigkeit. -- Praktische Neceßitation ist imperativ, ein Geboth. Ist der
Wille an sich selbst gut, so darf ihm gar nicht gebothen werden. Daher findet
bei Gott kein Geboth statt. Die objective praktische Nothwendigkeit ist bei Gott
auch subjective praktische Nothwendigkeit.}} Und wenngleich das Paradigma
endlicher Vernunft die \emph{moralische} Verfehlung ist, so macht doch die
Verbindung des Begriffs des Imperativs mit dem der Nötigung deutlich, dass
dieser Zusammenhang sich auf alle drei Arten von Imperativen
bezieht.\footnote{Siehe hierzu auch
\cite[][113]{Paton:TheCategoricalImperative1948}: \enquote{The principles
of goodness thus appear in our finite human condition as principles of
obligation. This is true even where the principle in question is one of skill or
rational self-love and not of morality. Men are not wholly rational in the
pursuit of happiness or even in the adoption of means to ends.}} Die
Endlichkeit der praktischen Vernunft bedingt, dass wir zwar wissen, wie zu
handeln klug wäre, aber dennoch
unklug handeln, weil wir auf kurzfristiges Glück setzen, obwohl wir wissen, dass
unser langfristiges Glück anderes erforderte. Und auch gegen die Regeln der
Geschicklichkeit handelt der endliche
Wille, der erkennt, was zu tun ist, und sich dennoch durch kurzfristige
Begierden von der Verfolgung eines Zieles abbringen lässt. Man denke an die
soeben beschriebene Langstreckenläuferin. Die Endlichkeit der menschlichen
Vernunft bezieht sich hier also nicht auf einen Mangel an Einsicht. Sie zeigt
sich gerade dort, wo uns die Einsicht leicht fällt.
\end{comment}


Zwischen Geschicklichkeit in verschiedenen Bereichen, Moralität und Klugheit
besteht hinsichtlich der jeweiligen epistemischen Zugänglichkeit ein
beachtlicher Unterschied.
Auf der einen Seite wissen wir nur sehr wenig darüber, wie Glückseligkeit -- die eigene
wie fremde -- zu erreichen ist. Denn Glückseligkeit meint nicht temporäre
Befriedigung, sondern langfristiges Glück. Und sie verlangt, die
unterschiedlichen und sich widerstreitenden (aktuellen wie zukünftigen)
Bedürfnisse in Einklang zueinander zu bringen oder gegeneinander abzuwägen. Wie
sich verschiedene Handlungsweisen und Entscheidungen langfristig bezahlt machen
werden, ist schwer abzuschätzen. \name[Immanuel]{Kant} neigt bezüglich unserer
Möglichkeiten, das je eigene langfristige Glück auf vernünftigen (und
moralkonformen) Wegen zu verfolgen, mitunter zum Pessimismus. Der
Mensch \emph{muss} sich zur Erreichung seiner Glückseligkeit seiner Vernunft
bedienen, hat damit aber ein denkbar schlechtes Instrument zur
Verfügung.\footnote{Man beachte hierzu die Ausführungen zur Misologie der
Vernunft in \cite[BA 4--8]{Kant:GrundlegungzurMetaphysikderSitten1965},
\cite[IV: 395.4--396.37]{Kant:GesammelteWerke1900ff.}.} Dennoch sind wir alle
nicht völlig ratlos. Wir können unser je eigenes Leben mit Blick auf anhaltendes
Glück kompetent führen, wenngleich eine gewisse Unsicherheit über den Ausgang
immer bleibt; und wir sind auch in begrenztem Umfang in der Lage, allgemeine
Ratschläge zu geben. Zumindest aber können wir sagen, dass niemand ohne unsere
Mithilfe sagen kann, wie wir selbst zur Glückseligkeit gelangen können. Wegen
ihrer Abhängigkeit von biographischen und anderen individuellen Zufälligkeiten
wie den eigenen Bedürfnissen und Neigungen, über die in aller Regel wir selbst
am zuverlässigsten Auskunft geben können, können Experten uns möglicherweise bei
eigenständigem Urteilen helfen, aber sie können uns dieses Urteil nicht
abnehmen.

Anders verhält es sich freilich mit den technischen Regeln der Geschicklichkeit.
Als Regeln, die sich auf zufällige Zwecke beziehen, sind sie prädestiniert für
die Ausbildung eines Expertenwesens. Wir werden jeweils über die meisten
technischen Regeln nicht selbst kompetent urteilen können und uns darauf
beschränken müssen, Kompetenzen bezüglich derjenigen Geschicklichkeiten zu
erwerben, deren Zwecke wir zu den eigenen machen wollen. Jede spezifische
Ausbildung liefert hierfür ein Beispiel.

\phantomsection\label{Abschnitt:moralischepistemischerOptimismus}
Schließlich befindet sich in Fragen der Moralität jeder in einer
epistemisch günstigen Position, weil \emph{jeder} über die entsprechende
Urteilskompetenz verfügt. Die These von der Aufklärungsrelevanz
ethischen Wissens ist bei \name[Immanuel]{Kant} mit der weiteren These korreliert, dass
jeder zu einem eigenständigen ethischen Urteil auch \emph{fähig} sei. In der
\titel{Grundlegung zur Metaphysik der Sitten} schreibt er:
\begin{quote}
  Was ich also zu tun habe, damit mein Wollen sittlich gut sei, darzu brauche
  ich gar keine weit ausholende Scharfsinnigkeit. Unerfahren in Ansehung des
  Weltlaufs, unfähig, auf alle sich eräugnende Vorfälle derselben gefaßt zu
  sein, frage ich mich nur: Kannst du auch wollen, daß deine Maxime ein
  allgemeines Gesetz
  werde?\footnote{\cite[BA 19--20]{Kant:GrundlegungzurMetaphysikderSitten1965},
  \cite[IV: 403.18--22]{Kant:GesammelteWerke1900ff.}.}
\end{quote}
In dieser These \name[Immanuel]{Kant}s liegt eine der Grundlagen des bekannten Primats des
Praktischen, der sich nicht erst aus dem Ergebnis der Vernunftkritik ergibt, sondern der Aufklärung weitgehend
programmatisch zugrunde
liegt.\footnote{\cite[Vgl.][127--130]{Ciafardone:UeberdasPrimatderpraktischenVernunftvordertheoretischenbeiThomasiusundCrusiusmitBeziehungaufKant1982}.
Wie \authorcite{Ciafardone:UeberdasPrimatderpraktischenVernunftvordertheoretischenbeiThomasiusundCrusiusmitBeziehungaufKant1982} darlegt, widerspricht dies dem fundamentalen
Missverständnis, welches in der \index{Kant, Immanuel}vorkantischen deutschsprachigen
Philosophie erstlinig den Hort einer \distanz{rationalistischen} Metaphysik
sieht. Vielmehr zeigt sich -- exemplarisch bei \name[Christian]{Thomasius} -- ein Vorbehalt
gegenüber metaphysischen Überlegungen, der sich einerseits aus
Relevanzüberlegungen und andererseits aus methodischen Vorbehalten speist.}
Es bestehe ein grundlegender Unterschied zwischen theoretischer und praktischer
Vernunft darin, dass wir in der theoretischen Vernunft anfällig für Irrtümer,
Aporien und Widersprüche in unseren Ansichten sind, sobald wir den Bereich der
Erfahrung verlassen und uns in die Metaphysik begeben. Die praktische Vernunft
hingegen habe es gerade dort besonders
leicht.\footnote{\cite[Vgl.][BA 21]{Kant:GrundlegungzurMetaphysikderSitten1965},
\cite[IV: 404.13--19]{Kant:GesammelteWerke1900ff.}: \enquote{In dem letzteren
[dem theoretischen Beurteilungsvermögen; A.\,G.], wenn die Vernunft es wagt, von
den Erfahrungsgesetzen und den Wahrnehmungen der Sinne abzugehen, gerät sie in
lauter Unbegreiflichkeiten und Widersprüche mit sich selbst, wenigstens in ein
Chaos von Ungewißheit, Dunkelheit und Unbestand. Im praktischen aber fängt die
Beurteilungskraft denn eben allererst an, sich recht vorteilhaft zu zeigen, wenn
der gemeine Verstand alle sinnliche Triebfedern von praktischen Gesetzen
ausschließt.}} Einfach ist freilich nicht das Aufsuchen des Prinzips der Moralität, also des
kategorischen Imperativs, sondern dessen Anwendung, welche unseren moralischen
Urteilen immer schon zugrunde liege. Dabei mögen sich auch die besten Köpfe
darin täuschen können, worin moralische Urteile fundiert sind --  so wie sich
nach \name[Immanuel]{Kant} etwa \name[David]{Hume} täuschte, als er unseren
Urteilen ein moralisches Gefühl zugrunde legte.
Die Anwendung hingegen erfordert kein Expertenwissen. Denn um zu wissen, welche Handlung
moralisch geboten oder verboten ist, muss ich nirgends nachschauen, ich muss
weder ein Buch (etwa einen Beichtspiegel) zur Hand nehmen noch einen Experten
befragen, sondern verlasse mich einfach auf den Maßstab, den ich in meiner
Vernunft finde. Der Kategorische Imperativ beschreibt die Grundlage unseres
moralischen Urteilens und Handelns, so wie sie uns auch vor jedem Philosophieren
längst vertraut sein
kann.\footnote{\cite[Vgl.][BA~20-1]{Kant:GrundlegungzurMetaphysikderSitten1965},
\cite[IV: 404.1-7]{Kant:GesammelteWerke1900ff.}: \enquote{Es wäre hier leicht zu
zeigen, wie sie [d.\,i.\ die gemeine Menschenvernunft; A.\,G.], mit diesem
Kompasse in der Hand, in allen vorkommenden Fällen sehr gut Bescheid wisse, zu
unterscheiden, was gut, was böse, pflichtmäßig, oder pflichtwidrig sei, wenn
man, ohne sie im mindesten etwas Neues zu lehren, sie nur, wie Sokrates tat, auf
ihr eigenes Prinzip aufmerksam macht, und daß es also keiner Wissenschaft und
Philosophie bedürfe, um zu wissen, was man zu tun habe, um ehrlich und gut, ja
sogar, um weise und tugendhaft zu sein.}} Nur handelt es sich freilich in der
Regel um ein implizites Wissen. Erst die Philosophie expliziere, was unsere
Praxis längst zum Ausdruck bringe, und zwar nicht, um neues Wissen zu
generieren, sondern um die Möglichkeit zu eliminieren, die jedem zugänglichen
ethischen Standards mittels ausgeklügelter Scheinargumente an unsere subjektiven
Bedürfnisse und Wünsche anzupassen -- die \emph{natürliche
Dialektik}.\footnote{\cite[Vgl.][BA
22\,f.]{Kant:GrundlegungzurMetaphysikderSitten1965},
\cite[][IV: 404.37--405.19]{Kant:GesammelteWerke1900ff.}.}


\subsection{Mündige Lebensführung und endliche
Vernunft}\label{section:MuendigeLebensfuehrung}
\begin{comment}
An verschiedenen stellen beschreibt \name[Immanuel]{Kant} unseren Willen als
endlich, insofern er nicht -- wie ein \singlequote{heiliger} Wille -- den
Vorschriften der praktischen Vernunft von sich aus genügt, sondern mit einem
Gegensatz von Wollen und Sollen konfrontiert ist. Weil wir über antagonistische
Neigungen verfügen, begegnen uns Imperative der Vernunft, die einen nötigenden
Charakter haben.\footnote{Siehe
\cite[][BA 40, 86\,f.,]{Kant:GrundlegungzurMetaphysikderSitten1965} \cite[][IV:
414.26--31, 439.28--440.13]{Kant:GesammelteWerke1900ff.};
\cite[][A 56--58, 145--150]{Kant:KritikderpraktischenVernunft1974}, \cite[][V:
32.1--33.5, 81.20--84.21]{Kant:GesammelteWerke1900ff.};
\cite[][\S~76]{Kant:KritikderUrteilskraft2009}, \cite[][V:
403.20--404.16]{Kant:GesammelteWerke1900ff.}.} Die Endlichkeit oder
Fehlbarkeit der praktischen Vernunft oder des Willens bezeichnet nach
\name[Immanuel]{Kant} den Sachverhalt, dass wir wider bessere Einsicht zu
handeln geneigt sind, weil unsere unmittelbaren Neigungen und Bedürfnisse der
Orientierung an vernünftigen Grundsätzen des Handelns -- praktischen
Erkenntnissen -- im Wege stehen. Wie ich im verbleibenden Teil dieses Kapitels
zeigen werde, ist es diese (praktische) Endlichkeit, die unserer Mündigkeit
zunächst im Weg steht, nicht eine vermeintliche (kognitive) Endlichkeit, der
zufolge wir die richtigen Einsichten nicht unabhängig vom Rekurs auf das Wissen
anderer zu generieren vermögen.
\end{comment}

\name[Immanuel]{Kant} weist die Behauptung zurück, Autonomie und Mündigkeit
seien mit Beliebigkeit verbunden, insofern sein Begriff der Mündigkeit und
Autonomie nicht besagt, dass wir Inhalte frei auswählen, sondern dass wir sie
eigenverantwortlich \emph{erkennen} sollen. Zu Beginn von Kapitel
\ref{subsection:DieBestimmungdesMenschen} artikulierte ich die Vermutung, dass
eine Akzentuierung bestimmter Erkenntnisse helfen könnte, Mündigkeit zu einem
realistischen Anspruch zu machen. Ginge es der Aufklärung nur um Mündigkeit in
bestimmten Fragen, wäre ihrem Anspruch in epistemischer Hinsicht leichter zu
genügen. Zumindest eine Akzentuierung ließ sich ausmachen: Aufklärung fokussiert
Wissen, welches ich vorhin als handlungsorientierend oder pragmatisch (im
weiteren Sinne) bezeichnete. In seinem Zentrum stehen Erkenntnisse, die wir
praktisch nennen, insofern sie nötigend sind und Imperative artikulieren. Eine
besondere Bedeutung kommt somit den drei Arten von Imperativen zu: Regeln der
Geschicklichkeit, Ratschlägen der Klugheit und Geboten der Sittlichkeit.

\name[Immanuel]{Kant}s moralisch-epistemischer
Optimismus\footnote{\phantomsection\label{Fussnote:moralischepistemischerOptimismus}Diesen
Ausdruck verwendete Martin \name[Martin]{Sticker} in einem Vortrag über
\titel{Educating the Common Agent -- Kant on Common Rational Capacities and the
Varieties of Moral Education} am 02.\,02.\,2013 an der Universität Wien. Dem
moralisch-epistemischen Optimismus \name[Immanuel]{Kant}s stehen zwei Formen des
Pessimismus in Fragen der Moral gegenüber: Erstens denkt \name[Immanuel]{Kant}
nicht, dass Menschen auch häufig moralisch (\enquote{aus Pflicht} und nicht nur
\enquote{pflichtmäßig}) handeln, und zweitens seien wir auch kaum fähig, die
eigene oder fremde Motivation zu erkennen, können also nie mit Gewissheit sagen,
\emph{ob} jemand (wir selbst oder andere) moralisch oder selbstsüchtig
handelten \mkbibparens{siehe dazu etwa
\cite[][A
222]{Kant:UeberdenGemeinspruch:dasmaginderTheorierichtigseintaugtabernichtfuerdiePraxis1977},
\cite[][VIII: 284.21--28]{Kant:GesammelteWerke1900ff.}}.} kommt der
Forderung nach Ausgang aus der selbst verschuldeten Unmündigkeit offensichtlich
entgegen, insofern ihm zufolge Mündigkeit bezüglich der Gebote der Sittlichkeit
zumindest insofern vorausgesetzt werden kann, dass wir die Gebote
\emph{erkennen} (wenngleich wir sie möglicherweise nicht befolgen).
Wir sind nicht nur \emph{berufen}, uns ein je eigenes Urteil in moralischen Fragen zu
bilden, sondern auch \emph{befähigt}. Der in der \titel{Kritik der reinen
Vernunft} gescholtene Naturalist der reinen Vernunft ist nicht derjenige, der
sich auf sein wenig geschultes (aber am Urteil anderer kontrolliertes!)
\emph{moralisches} Urteil verlässt. Denn wer sich in Fragen der Moral und Ethik
auf die gemeine (im Sinne von \emph{communis}, nicht von \emph{vulgaris}),
methodisch nicht speziell geschulte Vernunft verlässt, macht damit nichts falsch. Der Weg zur moralischen
Kompetenz führt über die Ausbildung der Urteilskraft im Austausch mit anderen,
nicht über den Erwerb besonderer Methoden. Lediglich bei den Fragen der
Metaphysik, die \name[Immanuel]{Kant} in der \titel{Kritik der reinen Vernunft}
diskutiert, kann nur kompetent urteilen, wer methodisch geschult ist. Und
deswegen bezieht \name[Immanuel]{Kant} die Ablehnung des Naturalismus der
Vernunft gerade auf \enquote{die erhabensten Fragen, die die Aufgabe der
Metaphysik ausmachen}\footnote{\cite[][B 883]{Kant:KritikderreinenVernunft2003};
\cite[][III: 551.36]{Kant:GesammelteWerke1900ff.}.}. In der Moral ist Mündigkeit
primär eine Frage des Handelns, nicht des Erkennens.

\begin{comment}
Mit Blick auf die Moral wird auch die Betonung der Religionsfragen in der
Aufklärungsschrift plausibel. Eine wichtige Eigenschaft von
\name[Immanuel]{Kant}s Ethik ist die Zurückweisung aller Formen von Heteronomie,
zu denen er insbesondere auch Versuche zählt, Moral in der Religion oder
Theologie zu fundieren. Die praktische Philosophie gibt die Bestimmung des
Menschen vor und prägt damit auch die Anthropologie, deren säkularer Charakter
gegenüber \authorcite{Spalding:BetrachtungueberdieBestimmungdesMenschen1749}s
Suche nach einer Bestimmung des Menschen trotz der Kontinuität gewiss offensichtlich
ist.\footnote{\cite[Vgl.][13]{Brandt:DieBestimmungdesMenschenbeiKant2007}.}
Dennoch ist es nun einfach, die Sonderstellung der Religion in
\name[Immanuel]{Kant}s Aufklärungsschrift zu erläutern. \name[Immanuel]{Kant}
stellt zunächst dem biblischen Theologen der oberen Fakultät einen reinen
Vernunftglauben gegenüber; und dieser sei ausreichend, der Bestimmung des
Menschen gemäß sich im eigenen Leben orientieren zu können:
\begin{quote}
  Ein reiner Vernunftglaube ist also der Wegweiser oder Kompaß, wodurch der
  spekulative Denker sich auf seinen Vernunftstreifereien im Felde
  übersinnlicher Gegenstände orientieren, der Mensch von gemeiner doch
  (moralisch) gesunder Vernunft aber seinen Weg, so wohl in theoretischer als
  praktischer Absicht, dem ganzen Zwecke seiner Bestimmung völlig angemessen
  vorzeichnen kann; und dieser Vernunftglaube ist es auch, der jedem anderen
  Glauben, ja jeder Offenbarung, zum Grunde gelegt werden
  muß.\footnote{\cite[A~320--1]{Kant:Washeisst:SichimDenkenorientieren?1977},
  \cite[VIII: 142.1--8]{Kant:GesammelteWerke1900ff.}.}
\end{quote}
Dies beruht auf wenigen, aber weitreichenden Voraussetzungen bezüglich des
Verhältnisses von Religion und Moral. Zu den einschlägigsten Voraussetzungen gehört
etwa, dass beide thematisch deckungs\-gleich
seien\footnote{\cite[A 44\,f.,]{Kant:DerStreitderFakultaeten1977} \cite[VII:
36.18--26]{Kant:GesammelteWerke1900ff.}: \enquote{Nicht der Inbegriff gewisser
Lehren als göttlicher Offenbarungen (denn der heißt Theologie), sondern der
aller unserer Pflichten überhaupt als göttlicher Gebote (und subjektiv der
Maxime, sie als solche zu befolgen) ist Religion.
  Religion unterscheidet sich nicht der Materie, d.i. dem Objekt nach in irgend
  einem Stücke von der Moral, denn sie geht auf Pflichten überhaupt, sondern ihr
  Unterschied von dieser ist bloß formal, d.\,i. eine Gesetzgebung der Vernunft,
  um der Moral durch die aus dieser selbst erzeugten Idee von Gott auf den
  menschlichen Willen zu Erfüllung aller seiner Pflichten Einfluß zu geben.}
  \cite[Des weiteren:][B 229--231]{Kant:DieReligioninnerhalbderGrenzenderblossenVernunft1977},
\cite[][VI: 153.28--154.5]{Kant:GesammelteWerke1900ff.}: \enquote{\ori{Religion}
ist (subjektiv betrachtet) das Erkenntnis aller unserer
 Pflichten als göttlicher Gebote. Diejenige, in welcher ich vorher wissen muß,
 daß etwas ein göttliches Gebot sei, um es als meine Pflicht anzuerkennen, ist
 die \ori{geoffenbarte} (oder einer Offenbarung benötigte) Religion: dagegen
 diejenige, in der ich zuvor wissen muß, daß etwas Pflicht sei, ehe ich es für
 ein göttliches Gebot anerkennen kann, ist die \ori{natürliche Religion}.}} und
 sich daher -- ähnlich wie in \authorcite{Lessing:EineDuplik1897}s Ringparabel -- religiöse
Überzeugungen ausschließlich anhand ihrer Zu- oder Abträglichkeit bezüglich der
Moralität der Gläubigen bewerten ließen.\footnote{\cite[Vgl.][A
103]{Kant:DerStreitderFakultaeten1977}, \cite[VII:
63.18--64.2]{Kant:GesammelteWerke1900ff.}: \enquote{Die Beglaubigung der Bibel
nun, als eines in Lehre und Beispiel zur Norm dienenden
evangelisch-messianischen Glaubens, kann nicht aus der Gottesgelahrtheit ihrer
Verfasser \punkt , sondern muß aus der Wirkung ihres Inhalts auf die Moralität
des Volks, von Lehrern aus diesem Volk selbst, als Idioten (im
Wissenschaftlichen), an sich, mithin als aus dem reinen Quell der allgemeinen,
jedem gemeinen Menschen beiwohnenden Vernunftreligion geschöpft, betrachtet
werden, die, eben durch diese Einfalt, auf die Herzen desselben den
ausgebreitetsten und kräftigsten Einfluß haben mußte.}} Dabei seien es die
\emph{a priori} erkennbaren Gesetze der Moral, die das uneingeschränkte Sagen
haben, während die Religion sich innerhalb des von der Moral gesteckten Rahmens
zu bewegen habe. Der biblische Theologe tritt nun aber als Experte in Fragen der Moral auf
und verweist auf schriftlich fixierte \enquote{Satzungen und Formeln} statt auf
die Vernunft.



Etwas anders als in Angelegenheiten der Moral verhält es sich mit Fragen der
Gerechtigkeit und der Bewertung von Politik. Geht es um das positive Recht oder um volkswirtschaftliche Fragen, so
fragen wir einen Juristen oder einen Ökonomen oder verweisen in Diskussionen auf
deren Aussagen; denn wir selbst haben normalerweise nicht die entsprechenden
Kenntnisse. Sich bei der Bewertung dessen, was positives Recht ist, auf einen
Maßstab der Gerechtigkeit hin, an einen Experten zu halten, wäre der Aufklärung
zuwider und eklatanter Ausdruck von Unmündigkeit.\footnote{Ich
sage \enquote{auf einen Maßstab der Gerechtigkeit hin}, weil es auch andere Maßstäbe gibt --
beispielsweise den der volkswirtschaftlichen Vernunft --, zu denen wir durchaus
Experten befragen sollten. Welche Folgen eine bestimmte Fiskalpolitik haben
wird, wie sich Monopolbildungen verhindern lassen usw., sind Fragen, die der
Ökonom (als Ökonom) beantworten kann, nicht der Staatsbürger als solcher.
In der Politik gibt es Bereiche, die ein technisches Wissen oder auch ein
besonderes Maß an Weltklugheit erfordern, das die Kompetenzen der meisten
Menschen übersteigt.} Wenn es in Diskussionen zu Meinungsverschiedenheit darüber
kommt, ob ein Zustand oder eine politische Entscheidung gerecht ist, reicht es
daher auch nicht aus, auf eine bestimmte Institution oder einen anerkannten
Experten zu verweisen. Wir müssen unsere eigenen Ansichten mit \emph{Argumenten}
untermauern, die unsere Gesprächspartner selbst als gültig einsehen können. Wir
sind also ebenso wie in Fragen der Moral berufen, unser eigenes kompetentes
Urteil nicht an \singlequote{Experten} zu delegieren -- zumindest wenn wir
beanspruchen, mündige Bürger eines demokratischen Gemeinwesens zu sein. Aber wir
sind in einer epistemisch schlechteren Position, denn Mündigkeit in Fragen
politischer Entscheidungen benötigt ein größeres Maß an Weltklugheit
als ethische Alltagsentscheidungen; man denke nur an die Feinheiten von
Diplomatie und Außenpolitik. Die allgemeine Zielsetzung von Politik gehört
jedoch nicht in die Hände von Experten, höchstens deren Umsetzung.\footnote{Dem widerspricht
\authorfullcite{Wilson:PoliticsandExpertise1971} unter Rückgriff auf eine These
\singlename{Platon}s: \enquote{I mean \punkt\ that there are people better
equipped than others to decide what is right, in the context of ends as well as means,
for a society or a state: the thesis maintained but inadequatly defended in
Plato's \ori{Republic}}
\parencite[][34]{Wilson:PoliticsandExpertise1971}. Allerdings scheint
\authorcite{Wilson:PoliticsandExpertise1971} dies letztlich auf die These einzuschränken, dass manche
Menschen charakterlich eher dazu geeignet sind, politische Mandate in einer repräsentativen Demokratie
einzunehmen, als andere. Schwieriger ist seine These, es gebe Güter, die für
alle rationalen Menschen erstrebenswert seien, zumal er explizit darauf
verzichtet, Argumente anzugeben. \authorcite{Wilson:PoliticsandExpertise1971} behauptet im Grunde:
\emph{Wenn} es eine allgemeingültige und einsehbare, rein rationale und apriorische Theorie
der Gerechtigkeit \emph{gäbe}, \emph{dann} spräche nichts gegen Experten zu
ihrer Umsetzung.} Wir können hier sagen, dass wir auf der einen Seite
\emph{berufen} sind, uns ein kompetentes Urteil in Fragen politischer
Entscheidungen zu bilden, dass wir uns auf der anderen Seite jedoch um unsere
\emph{Befähigung} stets zu bemühen haben.

Gewiss finden sich bei \name[Immanuel]{Kant} kaum Überlegungen dazu, was
den Menschen als mündigen Staatsbürger
ausmacht;\footnote{\name[Immanuel]{Kant}s Auskünfte beschränken
sich auf Angaben zur wirtschaftlichen Selbständigkeit, die den aktiven
Staatsbürger, dem allein das Wahlrecht zustehe, vom passiven unterscheide. Dabei
betont er, dass die Verweigerung der bürgerlichen Mündigkeit \enquote{der
Freiheit und Gleichheit derselben als Menschen}
\mkbibparens{\cite[][\S~46]{Kant:DieMetaphysikderSitten1977Rechtslehre},
\cite[][VI: 315.7--8]{Kant:GesammelteWerke1900ff.}} nicht entgegenstehe
\mkbibparens{\cite[vgl.][\S~46]{Kant:DieMetaphysikderSitten1977Rechtslehre},
\cite[][VI: 314.17--315.22]{Kant:GesammelteWerke1900ff.}}.} die politische
Emanzipation ist nicht sein Thema -- wie schon aus seiner ablehnenden Haltung
der Demokratie gegenüber hervorgeht\footnote{In \titel{Zum ewigen Frieden}
betont \name[Immanuel]{Kant}, ein Staatswesen müsse \emph{republikanisch}, aber
nicht unbedingt \emph{demokratisch} sein
\mkbibparens{vgl. \cite[][B 24--29]{Kant:ZumewigenFrieden1900ff.},
\cite[][VIII: 351.21--353.18]{Kant:GesammelteWerke1900ff.}}. Letztlich schließe
die Forderung nach einer republikanischen Verfassung die Demokratie sogar aus,
weil letztere eine wirksame Gewaltenteilung nicht zulassen. Dass Konzept einer
repräsentativen (parlamentarischen) Demokratie mit der Gewaltenverschränkung,
wie es uns heute als politische Wirklichkeit vertraut ist, stand
\name[Immanuel]{Kant} noch nicht zur Verfügung. Die demokratische Regierungsform
sei gerade nicht \singlequote{repräsentativ}, \enquote{weil alles da Herr sein
will} \mkbibparens{\cite[][B 27]{Kant:ZumewigenFrieden1900ff.},
\cite[][VIII: 353.1]{Kant:GesammelteWerke1900ff.}}.}.
so dass wir zunächst bemerken müssen, dass es sich bei der politischen
Mündigkeit um eine Forderung und eine Herausforderung handelt, die nicht die
Aufklärung, sondern höchstens die konsequente Weiterführung der Aufklärung
betrifft. Dass es ein zentrales Thema der Aufklärung sein sollte, ergibt sich
daraus, dass die Errichtung eines bürgerlichen Gemeinwesens zur Bestimmung des
Menschen gehört, die sich aus der anthropologischen Tatsache seiner
\singlequote{ungeselligen Geselligkeit} ergibt.
\end{comment}

Völlig anders verhält es sich mit den Regeln der Geschicklichkeit, die
paradigmatisch für Expertenwissen sind. Wir können nicht in allen Bereichen
selbst Kompetenzen erwerben, die für willkürlich zu wählende Zwecke nützlich
sein könnten. Es ist dies für den Ausgang aus selbst verschuldeter Unmündigkeit
offensichtlich auch nicht nötig. Niemand käme auf die Idee, einer Person die
Mündigkeit abzusprechen, weil sie sich in Fragen des Flugzeugbaus oder der
Elektrotechnik nicht auskennt. Ob wir solches Wissen benötigen, hängt von
unserem Lebensweg und den zufällig gewählten beruflichen Entscheidungen ab. Aber
selbst dann scheint es keine Frage der Mündigkeit zu betreffen. Ein Ingenieur,
der in Fragen der Mechanik nicht selbst kompetent urteilen kann, sondern immer
seine Kollegen fragen muss, ist kein unmündiger Mensch, sondern ein schlechter
Ingenieur. Regeln der Geschicklichkeit sollten also für Belange der Aufklärung
nicht ausschlaggebend sein.


Komplizierter ist die Situation bezüglich der Erkennbarkeit von Erkenntnissen
der Klugheit in ihrer Bedeutung für unsere je eigene Mündigkeit. Dass die
Ratschläge der Klugheit auch Thema des Aufklärungsaufsatzes sind, wird deutlich,
wenn man beachtet, dass medizinische Ratschläge solche der Klugheit sind. Auch
als Beispiel für Unmündigkeit wählt \name[Immanuel]{Kant} eines aus der Medizin:
\enquote{Habe ich {\punkt} einen Arzt, der für mich die Diät beurteilt, {\punkt}
so brauche ich mich ja nicht selbst zu bemühen.}\footnote{\cite[][A
482]{Kant:BeantwortungderFrage:WasistAufklaerung?1977}, \cite[][VIII:
35.13--16]{Kant:GesammelteWerke1900ff.}. Diese Forderung kritisiert
\textcite[vgl.][258]{ONeill:Therhetoricofdeliberation2002}.} Heißt dies nun
aber, dass wir Ratschläge von Experten generell zurückweisen sollten? Sollte dem
so sein, wäre durch die Akzentuierung wohl wenig gewonnen. Selbst den Anspruch
auf medizinische Kompetenz anzumelden, ist bestenfalls \singlequote{heroisch},
eher leichtfertig. \authorfullcite{ONeill:Therhetoricofdeliberation2002} nennt
das Beispiel eines besorgten Verbrauchers, der sich angesichts des BSE-Skandals
Gedanken über den Zusammenhang von BSE und Creutzfeld-Jakob-Krankheit macht.
Offensichtlich kann er ohne Hilfe von Experten hier nichts ausrichten. Er kann
-- wie \authorcite{ONeill:Therhetoricofdeliberation2002} verdeutlicht -- nicht
einmal auf der Grundlage eigenen Wissens beurteilen, welchem Experten er
vertrauen sollte.\footcite[Vgl.][257]{ONeill:Therhetoricofdeliberation2002} Wir
vertrauen in aller Regel lediglich darauf, dass staatliche
Zulassungsbestimmungen zuverlässig sind; und diese wiederum sehen Prüfungen der
Kompetenz durch andere Experten vor. Es ist in vielen Fällen so, dass die
Expertise, die jemand auf einem bestimmten Gebiet zu haben beansprucht, nur von
denjenigen bewertet werden kann, die selbst über sie verfügen, also von anderen
Experten. In medizinischen Fragen nicht auf Experten zu hören, weil man seine
Mündigkeit dadurch gefährdet sähe, ist freilich Unsinn.
Und wenn \name[Immanuel]{Kant} sich mit seiner Forderung genau hierauf festlegte, dann
wäre diese Forderung zu verwerfen.

Plausibler wird die Forderung nach Mündigkeit in medizinischen Fragen, wenn wir
nicht auf den \emph{Erwerb} medizinischen Wissens schauen, sondern auf dessen
\emph{Anwendung}. In vielen, wenn auch bei weitem nicht in allen Fällen ist uns
allen bekannt, was wir aus medizinischer Sicht tun sollten. (Wir sollten
beispielsweise nicht zu viel Zucker und zu viel Fett zu uns nehmen, nicht rauchen,
Alkohol nur in Maßen konsumieren und uns regelmäßig bewegen.) Hier liegt kein
epistemisches Problem vor, sondern eines der Anwendung:
Unsere kurzfristigen Begierden verhindern, dass entsprechende Ratschläge der
Klugheit auch als subjektive Prinzipien handlungswirksam werden.

Es geht \name[Immanuel]{Kant} nicht um die Zurückweisung
von Expertenwissen im Rahmen der Klugheit; unmündig ist nicht, wer in
medizinischen Fragen auf den Rat von Experten hört. Unmündig ist vielmehr, wer
wider besseres Wissen handelt und -- statt selbst die Verantwortung dafür zu
übernehmen -- hinterher auf das Wissen von Experten setzt.  Wir
suchen medizinischen Rat, der uns sagt, wie wir \emph{dennoch} -- trotz
unserer ungesunden Lebensweise -- dauerhaft gesund bleiben
können.\footnote{\cite[Vgl.][A 31]{Kant:DerStreitderFakultaeten1977},
\cite[][VII: 30.24--30]{Kant:GesammelteWerke1900ff.}: \enquote{Was ihr
\ori{Philosophen} da schwatzet, wußte ich längst von selbst; ich will aber von euch als Gelehrten
wissen: wie, wenn ich auch \ori{ruchlos} gelebt hätte, ich dennoch kurz vor
Torschlusse mir ein Einlaßbillett ins Himmelreich verschaffen, wie, wenn ich
auch \ori{Unrecht} habe, ich doch meinen Prozeß gewinnen, und wie, wenn ich
auch meine körperlichen Kräfte nach Herzenslust benutzt und \ori{mißbraucht}
hätte, ich doch gesund bleiben und lange leben könne.}} Statt auf \emph{solche}
Unterstützung von Experten zu bauen soll man laut \name[Immanuel]{Kant} zunächst
das längst vorhandene Wissen auch zur Anwendung bringen, wofür wir jedoch
Disziplin aufbringen müssen.\footnote{\name[Immanuel]{Kant} sagt, man solle
  \enquote{sich mäßig im Genusse und duldend in Krankheiten und dabei
  vornehmlich auf die Selbsthülfe der Natur rechnend {\punkt} verhalten; zu welchem allem es
  freilich nicht eben großer Gelehrsamkeit bedarf, wobei man dieser aber
  größtenteils entbehren kann, wenn man nur seine Neigungen bändigen und seiner
  Vernunft das Regiment anvertrauen wollte, was aber, als Selbstbemühung, dem
  Volk gar nicht gelegen ist} \mkbibparens{\cite[][A
  30\,f.,]{Kant:DerStreitderFakultaeten1977} \cite[][VII:
  30.14--19]{Kant:GesammelteWerke1900ff.}}.
Diesen Ratschlag sollten wir gewiss nicht \emph{en detail} übernehmen.
Möglicherweise ist es dem medizinischen Fortschritt zu verdanken, dass es uns töricht
erscheint, uns in Krankheiten \singlequote{duldend} zu verhalten; möglicherweise
war dies aber auch schon zu \name[Immanuel]{Kant}s Zeiten kein besonders kluger
Rat.} Das Moment der Unmündigkeit liegt dabei aber nicht in der Berufung auf
Experten, sondern darin, dass vorhandene eigene Wissen nicht zur Anwendung zu
bringen. Die Unmündigkeit ist also zunächst Folge der Endlichkeit unseres
Willens, nicht des Verstandes.\footnote{Auch
\authorfullcite{Mikalsen:TestimonyandKantsIdeaofPublicReason2010} behauptet,
dass sich die Bemerkung \name[Immanuel]{Kant}s in der Aufklärungsschrift primär
gegen unsere Bequemlichkeit richtet und deswegen keine \singlequote{heroischen}
Erkenntnisbemühungen fordert: \enquote{I do not think that Kant
is concerned with the relation between experts and non-experts in this passage. Nor do I think what he says has
any of the questionable implications discussed above. Rather than making a call
for foolhardy heroism, Kant points to a main obstacle to enlightenment: the
convenience of immaturity}
\parencite[][28]{Mikalsen:TestimonyandKantsIdeaofPublicReason2010}. Anders als
meine Interpretation sieht er die Bequemlichkeit wiederum im \emph{Erkennen}
und nicht in der \emph{Anwendung} der Erkenntnis. Ich sehe nicht, wie dies die Kritik
\authorcite{ONeill:Therhetoricofdeliberation2002}s entkräften soll.}

Möglicherweise überschätzt \name[Immanuel]{Kant} in seinen Texten (speziell im
\titel{Streit der Fakultäten}) unsere -- und auch seine eigenen -- medizinischen
Kompetenzen. Und er unterschätzt die Möglichkeiten der Schulmedizin, wie aus
heutiger Perspektive besonders deutlich wird. Dennoch ist es korrekt zu sagen,
dass es in der Medizin viele Beispiele für Ratschläge der Klugheit gibt, die
zeigen, dass wir eher in der Anwendung als in der Erkenntnis fehlgehen, und dass
diese Fehler in der Anwendung der Tatsache geschuldet sind, dass wir zu bequem
sind, unsere Vernunft gegenüber unseren Neigungen zu
behaupten.\footnote{Ganz analog
sieht \name[Immanuel]{Kant} zwei weitere Fälle: Wir alle wissen uns unseren
Mitmenschen gegenüber gerecht zu verhalten, ohne stets die Hilfe von Experten in
Anspruch nehmen zu müssen. Nur ist dies oft damit verbunden, persönliche
Begierden zurückzustellen. Wenn uns hingegen daran gelegen ist, ungerechte
Ansprüche vor Gericht durchzusetzen, seien wir auf Experten angewiesen. Und
während jeder Gläubige wissen könne, wie er ein gottgefälliges Leben führt
(indem er sich moralisch verhält), suche so mancher Rat bei seiner Kirche, um
sich unmoralisch verhalten und dennoch -- durch entsprechende Zeremonien -- die
Gnade Gottes erkaufen zu können. Der Ablasshandel ist ein anschauliches
Beispiel. In allen drei Fällen ist es kein Mangel an Wissen, der unserer
Mündigkeit entgegensteht, sondern die Verlockung durch kurzfristige Wünsche
und Begierden, \emph{gegen} dieses bessere Wissen zu handeln.}
Dies ist ein systematisch entscheidender Punkt bei der Frage, wie die Forderung
nach Mündigkeit und Selbstbestimmung mit unserer arbeitsteiligen
Wissensgesellschaft vereinbar sein kann. Aufklärung und Mündigkeit verlangen von
uns nicht, Experten zu werden. Sie verlangen lediglich ein gewisses Basiswissen,
etwa über eine gesunde Lebensführung, und vor allem die \emph{Anwendung} dieses
Basiswissens im Leben. Hier hat auch die pragmatische Anthropologie mit ihrem
Wissen um die \emph{conditio humana} ihren systematischen Ort: Sie versammelt
gerade zentrales Wissen über den Menschen, welches dieser benötigt, um sein
eigenes Leben kompetent führen zu können. Aber dieses Wissen ist kein
Expertenwissen, das einer breiteren Bevölkerung notwendig verschlossen bliebe.
Wer freilich auch über rudimentärste medizinische Kenntnisse oder basales Wissen
über die menschliche Natur nicht verfügt, den können wir in der Tat nicht als
mündig und aufgeklärt bezeichnen.

Auch dies ist eine Deutungsmöglichkeit der Aussage, diejenigen, die besonders
reich an Kenntnissen sind, seien mitunter \enquote{im Gebrauche derselben am
wenigsten aufgeklärt}\footnote{\cite[][A
329]{Kant:Washeisst:SichimDenkenorientieren?1977}, \cite[][VIII:
146.34]{Kant:GesammelteWerke1900ff.}.}, die dadurch gestützt wird, dass
\name[Immanuel]{Kant} anmerkt, die Unterscheidung der Erkenntnisse in
pragmatische und spekulative Erkenntnisse betreffe nicht die
Erkenntnisse selbst, sondern eben ihren
\emph{Gebrauch}.\footnote{\cite[Vgl.][\nopp 2802]{Kant:Reflexionen1900ff.},
\cite[][XVI: 519.15--17]{Kant:GesammelteWerke1900ff.}.} Aufklärung heißt nicht
nur, vernünftige Erkenntnisse zu gewinnen, sondern die eigenen Handlungen an
vernünftigen Erkenntnissen auszurichten. Es ist nicht nur unser eingeschränktes
Erkenntnisvermögen, welches Aufklärung zur Herausforderung für endliche Wesen
macht. Es reicht nicht zu wissen, wie eine gesunde Ernährung aussieht und welche
Vorteile sie uns bringt, wenn man nicht in der Lage ist, am Süßwarenregal vorbei zu
gehen. Als endliche Wesen müssen wir erst lernen, unsere längerfristigen
Intentionen gegen kurzfristige Neigungen zu behaupten, um vernünftig zu handeln.
Dies zu lernen, nennt \name[Immanuel]{Kant} \enquote{\emph{Disziplin}} oder auch \enquote{Kultur der
Zucht}.\footnote{\cite[Vgl.][\S~83]{Kant:KritikderUrteilskraft2009}, \cite[][V:
432.3--5]{Kant:GesammelteWerke1900ff.}.}

Auf den ersten Blick scheinen Disziplinierung und Zucht gerade der Freiheit
entgegengesetzt zu sein, insofern sie sich darauf zu richten scheinen, einen
bestimmten \singlequote{asketischen} Typus von Persönlichkeit, der frei ist von
Begierden und Neigungen, gewaltsam hervorzubringen. Sie richten sich aber nicht
gegen Begierden und Neigungen \emph{per se} (ohne Begierden und Neigungen gäbe
es auch keine langfristige Glückseligkeit), sondern gegen die Unfähigkeit,
einmal als richtig eingesehene Handlungsgrundsätze (\singlequote{Maximen}) auch
dann weiter zu verfolgen, wenn ihnen kurzfristige Begierden entgegenstehen.
Und sie dienen entsprechend nicht nur der Moral, sondern ermöglichen auch die
Befolgung technischer und pragmatischer Imperative der Geschicklichkeit und
Klugheit. Der konkrete Inhalt oder die Ausgestaltung der je eigenen Freiheit ist
noch immer mit den je konkreten Begierden und Neigungen verbunden, zu denen der
Handelnde sich nun jedoch bewusst und willentlich verhalten kann. Endliche Wesen
erlangen Mündigkeit daher nur auf dem Wege der Disziplinierung, die als
Grundlage von Kultivierung, Zivilisierung und Moralisierung fungiert, indem sie
uns befähigt, unserer kurzfristigen Neigungen Herr zu werden.



\begin{comment}
Ich hatte oben auf \name[Immanuel]{Kant}s Antinaturalismus im Gefolge
\authorcite{Wolff:Psychologiaempirica1968}s verwiesen\footnote{Siehe Kapitel
\ref{Terminus:methodischerNaturalismus}, insb.
S.~\pageref{Terminus:methodischerNaturalismus}.} und kann die gemachten Bemerkungen nun konkretisieren:
Genau wie \authorcite{Wolff:Psychologiaempirica1968} sieht \name[Immanuel]{Kant} die Unmündigkeit und
Autoritätsgläubigkeit nicht generell als das Resultat mangelnder
Entschlusskraft, sondern mitunter auch als Folge seiner Inkompetenz in vielen
Bereichen der Wissenschaft an. Unmündigkeit ist dort nicht selbst verschuldet,
wo die Urteilskompetenz fehlt, weil naturgemäß nur Experten über sie
verfügen. Und diese unverschuldete Unmündigkeit bleibt ohne negative
Folgen, wenn der der Wissenschaft im allgemeinen oder auch einer bestimmten
Wissenschaft Unkundige sich einfach auf das beschränkt, was zu wissen ihm
obliegt und auch möglich ist. Und dies sind Klugheit und Moral, bei denen er
gerade nicht unkundig, sondern zu einem eigenständigen Urteil fähig sein soll.
\end{comment}

Vorläufig zusammenfassend seien daher zwei Momente hervorgehoben, die wir zu
einer mündigen Lebensführung dringender benötigen als methodische Kenntnisse und
Fachwissen in verschiedenen Wissensbereichen: Nach den Überlegungen am Ende des
\ref{section:KantalsliberalerAufklaerer}. Kapitels benötigen wir eine geübte
Urteilskraft; nun zeigt sich, dass wir über Disziplin verfügen müssen.
Wer sich darauf beschränke, bezüglich dessen, was er tun soll und wie er sein eigenes Leben
einzurichten habe, eigenständig zu urteilen, dem falle dies auch nicht schwer.
Er muss lediglich seine Urteilskraft im Austausch mit anderen ausbilden und
Herr über seine eigenen kurzfristigen Begierden werden. Die Endlichkeit des
Menschen macht dies nötig, aber nicht unmöglich. Nur wer sich noch darüber
hinaus mit den heiklen Fragen der Metaphysik -- oder genauer: der Metaphysik der
Natur -- befasst, verfällt leicht der Versuchung, sich an Andere zu wenden, weil
ihm ein eigenständiges Urteil einfach nicht gelingen will. In der \titel{Kritik
der Urteilskraft} schreibt \name[Immanuel]{Kant}:
\begin{quote}
  Man sieht bald, daß Aufklärung zwar in Thesi leicht, in Hypothesi aber eine
  schwere und langsam auszuführende Sache sei: weil mit seiner Vernunft nicht
  passiv, sondern jederzeit sich selbst gesetzgebend zu sein, zwar etwas ganz
  Leichtes für den Menschen ist, der nur seinem wesentlichen Zwecke angemessen
  sein will und das, was über seinen Verstand ist, nicht zu wissen verlangt;
  aber da die Bestrebung zum Letzteren kaum zu verhüten ist, und es an anderen,
  welche diese Wißbegierde befriedigen zu können mit vieler Zuversicht
  versprechen, nie fehlen wird, so muß das bloß Negative (welches die
  eigentliche Aufklärung ausmacht) in der Denkungsart (zumal der öffentlichen)
  zu erhalten oder herzustellen sehr schwer
  sein.\footnote{\cite[][\S~40]{Kant:KritikderUrteilskraft2009}, \cite[][V:
  294.29--37]{Kant:GesammelteWerke1900ff.}.  Zur Verwendung des Begriffspaares
  \enquote{in thesi}/\enquote{in hypothesi} vgl.\
  \cite[][\nopp 5696]{Kant:Reflexionen1900ff.}, \cite[][XVIII:
  329.1--2]{Kant:GesammelteWerke1900ff.}, wo \name[Immanuel]{Kant} \enquote{in
  thesi} als der logischen (begrifflichen) Möglichkeit nach, \enquote{in hypothesi} als der realen
  Möglichkeit nach bedeutend erläutert. In \titel{Über den Gemeinspruch\ldots}
  wiederum sagt \name[Immanuel]{Kant}, mit der Wendung, ein Satz gelte zwar in
  thesi, nicht aber in hypothesi, meine man oft, dass es in der Theorie ganz
  gut und richtig, in der Praxis aber unbrauchbar sei
  \mkbibparens{\cite[vgl.][A
  204]{Kant:UeberdenGemeinspruch:dasmaginderTheorierichtigseintaugtabernichtfuerdiePraxis1977},
  \cite[][VIII: 276.9--18]{Kant:GesammelteWerke1900ff.}}. Eine ähnliche
  Verwendung findet sich in \titel{Zum ewigen Frieden}
  \mkbibparens{\cite[vgl.][BA 38]{Kant:ZumewigenFrieden1900ff.},
  \cite[][VIII: 357.12--13]{Kant:GesammelteWerke1900ff.}; siehe auch
  \cite[][B 22]{Kant:DieReligioninnerhalbderGrenzenderblossenVernunft1977},
  \cite[][VI: 29.24--30]{Kant:GesammelteWerke1900ff.}}.}
\end{quote}
Aufklärung ist da vonnöten, wo es um unsere je eigenen wesentlichen Zwecke geht.
Dass der Mensch \enquote{seinem wesentlichen Zwecke angemessen} sein solle,
bezieht sich auf die Bestimmung des Menschen, denn wesentliche Zwecke sind der
Endzweck, der in der Bestimmung des Menschen bestehe, und subalterne Zwecke,
die von diesem Endzweck abhängen.\footnote{\cite[Vgl.][B
868]{Kant:KritikderreinenVernunft2003}; \cite[][III:
543.7--12]{Kant:GesammelteWerke1900ff.}.}

\begin{comment}
Die Regeln der Geschicklichkeit verhalten sich neutral gegenüber den
wesentlichen Zwecken. Sie brauchen uns nur zu interessieren, wenn wir
entsprechende Zwecke verfolgen. Verfolgen wir die Zwecke nicht, dann sind wir
frei, uns des Urteils zu enthalten; wir urteilen dann weder unverantwortlich
noch fremdbestimmt, weil wir gar nicht urteilen.
Die Ratschläge der Klugheit und die Gebote der Sittlichkeit artikulieren
hingegen notwendige und vernünftige Zwecke des Handelns, die uns ebenso
angehen, wie die Mittel, mit deren Hilfe wir sie erreichen können.
Klugheit als freie Sorge um das je eigene Wohlergehen ist in der Tat ein
wichtiger Aspekt von Mündigkeit. Aber dieser Aspekt steht nicht alleine da,
sondern ist integriert in die Sorge um das moralisch Richtige. Und wenn
Vertreter der Aufklärung von Nützlichkeit sprechen, dann ist damit nicht bloß
gemeint, dass etwas für kontingente Zwecke oder Befriedigung je eigener
zufälliger Begierden brauchbar ist. Gerade die
notwendigen -- etwa moralisch zwingenden -- Ziele des Handelns \emph{respective}
ihre Erkenntnis gelten dem 18. Jahrhundert als \enquote{nützlich}. Da es im Falle der technischen
Regeln der Geschicklichkeit zufällig ist, ob wir ihrer bedürfen, und wir somit
auf diese verzichten können, lasse ich sie außen vor und konzentriere mich auf
Moral und Klugheit.

\end{comment}

Wenn die Hochaufklärung um \authorcite{Wolff:Psychologiaempirica1968} die
Problematik des Widerstreits von Endlichkeit und Selbständigkeitsforderung durch
einen allgemeinen Erkenntnisoptimismus
überging,\footcite[Vgl.][36]{Engfer:ChristianThomasius1989} dann ist eine
ähnliche Strategie mit Blick auf die Moralphilosophie auch \name[Immanuel]{Kant}
zuzuschreiben. Die Akzentuierung praktischer Erkenntnisse
hat eine gewisse Erleichterung gebracht, insofern die Forderung nach Mündigkeit
von uns dort etwas fordert, was uns nach \name[Immanuel]{Kant} zumindest keine
epistemischen Probleme bereitet. Mit der Aufklärung ist eine \emph{Akzentuierung} der Moral und mit ihr verwandter
Disziplinen verbunden, wenngleich sie keine \emph{Einschränkung} der
Forderung nach Selbstbestimmung auf bestimmte Themenbereiche erlaubt.
\phantomsection\label{Abschnitt:moralischepistemischerOptimismus-Ende}
Nun können wir uns in vielen Fragen unseres Urteils enthalten, wir dürfen
nur nicht unverantwortlich und heteronom urteilen. In Fragen, die unsere
Lebensausrichtungen und unser notwendiges Handeln und Entscheiden betreffen --
Fragen zu unserer \singlequote{Bestimmung} als Menschen --, ist eine solche
Urteilsenthaltung freilich nicht möglich, hier \emph{müssen} wir urteilen. Wie
die \titel{Kritik der reinen Vernunft} zeige, sei aber auch hier jeder
gleichermaßen \emph{fähig} zu urteilen. Denn ihr Ergebnis beinhalte
\begin{quote}
daß die Natur, in dem, was Menschen ohne Unterschied angelegen ist, keiner
parteiischen Austeilung ihrer Gaben zu beschuldigen sei, und die höchste
Philosophie in Ansehung der wesentlichen Zwecke der menschlichen Natur es nicht
weiter bringen könne, als die Leitung, welche sie auch dem gemeinsten Verstande
hat angedeihen lassen.\footnote{\cite[][B
859]{Kant:KritikderreinenVernunft2003}, \cite[][III:
538.11--16]{Kant:GesammelteWerke1900ff.}.}
\end{quote}
\name[Immanuel]{Kant} muss dabei nicht behaupten, dass Aufklärung leicht und
ohne Mühsal wäre.\footnote{So richtet sich der Aufklärungsaufsatz an den
\singlequote{Gelehrten} \mkbibparens{\cite[vgl.][A
485]{Kant:BeantwortungderFrage:WasistAufklaerung?1977}, \cite[][VIII:
37.11--13]{Kant:GesammelteWerke1900ff.}}.} Aufklärung sei \enquote{in Thesi}
leicht, \enquote{in Hypothesi} aber schwer auszuführen. Das heißt, dass sich,
erstens, leicht angeben lässt, was Aufklärung von uns fordert und was sie ihrem
Begriff nach (\enquote{in Thesi}) ist, dass aber, zweitens, ihre Ausführung und
Umsetzung (\enquote{in Hypothesi}) viel schwerer ist, gerade weil sie die
Ausbildung unserer Urteilskraft fordert. Aber die Vernunftkritik helfe doch,
Aufklärung als realistische Option auszuweisen, indem sie zeige, dass
tief schürfende Wissenschaft die wesentlichen Zwecke nicht besser verfolgen kann,
als eine im Austausch mit anderen kultivierte Urteilskraft.

\section{Zusammenfassung und Ausblick}
Wir haben nun gesehen, dass wir unser Augenmerk auf zwei Aspekte
richten müssen: Die \emph{Erkenntnis} dessen, was wir tun sollen, und seine
\emph{Durchführung}.
In Ansehung derjenigen Fragen, die uns als Menschen besonders betreffen,
gefährden keine epistemischen Defizite unsere mündige Lebensführung. Wir
sind in aller Abhängigkeit und Endlichkeit, bei aller Einschränkung unserer
kognitiven Vermögen doch in der Lage, hinreichend viel zu \emph{wissen} und
\emph{kompetent zu beurteilen}, was unsere je eigene Lebensführung anbelangt.
Doch die Endlichkeit unserer \emph{praktischen} Vernunft -- unser endlicher
Wille -- unterminiert mitunter die Durchführung, also die konsequente
Orientierung an der (zweifellos vorhandenen) vernünftigen Einsicht.

Die Fehlbarkeit endlicher Wesen in der Ausübung der praktischen Vernunft besteht
oft nicht darin, dass die entsprechenden Erkenntnisse nicht verfügbar wären, sondern
darin, dass trotz korrekter Einsicht die Ausführung unterbleibt. Den
paradigmatischen Fall finden wir bei moralischen Verfehlungen, bei denen wir
davon ausgehen können, dass kein Mangel an moralischer Einsicht vorliegt,
sondern an korrekter Umsetzung vorhandener Einsichten. Ein nicht-endlicher
(ein \singlequote{heiliger}) Wille unterliegt denselben moralischen Gesetzen,
die auch uns bekannt sind (er ist uns kognitiv möglicherweise bloß gleichwertig);
aber wenngleich er dieselben moralischen Gesetze \emph{erkennt}, so \emph{handelt}
darüber hinaus auch stets nach ihnen.

Doch damit wird das grundlegende Problem des Verweisens auf intellektuelle
Freiheit und Selbständigkeit zwar geschmälert, nicht jedoch gelöst. Wir
verfügen über genügend Wissen, um selbstbestimmt leben und Entscheidungen
treffen zu können. Aber dieses Wissen haben wir in der Regel nicht selbst
generiert, sondern von anderen übernommen. Man denke an medizinisches Wissen,
welches im Rahmen der Klugheit von Bedeutung ist. Dass Zigaretten und Alkohol
ungesund, Obst und Gemüse aber gesund sind, wissen wir, weil wir es -- in der
Regel von unseren Eltern -- gelernt haben. Aber wir können wir uns als frei und
selbständig verstehen, wenn wir lediglich das Wissen in unserem Handeln wirksam
werden lassen, welches wir von anderen übernommen haben? Dieser Frage nach der
Vereinbarkeit von intellektueller Freiheit und Selbständigkeit auf der einen und
dem Erwerb von Wissen durch das Lernen von anderen gehen die folgenden Kapitel
\ref{section:autonomieunddaszeugnisanderer}--\ref{Chapter:KantsSocialEpistemology}
nach. Kapitel \ref{section:autonomieunddaszeugnisanderer} wird dabei den Status
solchen Wissens in der Philosophie der Neuzeit mit Blick auf klassische Ansätze
bei \authorcite{Descartes:OeuvresdeDescartes1983} \name[David]{Hume}, \name[Thomas]{Reid} und
\authorcite{Crusius:WegzurGewissheitundZuverlaessigkeitdermenschlichenErkenntniss1965}
thematisieren, Kapitel \ref{chapter:MuendigerErwerbTestimonialenWissens} zwei
unterschiedliche Perspektiven auf dieses Problem differenzieren und Kapitel
\ref{Chapter:KantsSocialEpistemology} schließlich auf der Grundlage des zuvor
erarbeiteten die Position \name[Immanuel]{Kant}s eruieren.


