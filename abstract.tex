\selectlanguage{english}
\section*{Abstract}\addcontentsline{toc}{chapter}{Abstract}\markboth{}{}
It is often maintained that German Enlightenment philosophy---and especially the
Enlightenment program that Immanuel \name[Immanuel]{Kant} articulated in his
famous article in the \titel{Berlinische Monatsschrift}---is not able to do
justice to the role of testimonial knowledge and hence is an obstacle to a down-to-earth
epistemology because of its essential individualistic stance. I
argue that this is a misunderstanding and that \name[Immanuel]{Kant} like other
Enlightenment thinkers actually recognizes these social factors and their impact on our
knowing and thinking. I show that \name[Immanuel]{Kant}'s philosophy can be
reconstructed as an attempt to show the compatibility of Enlightenment's demand
for independence and the essential dependency of finite cognizers. His mature philosophy aims at
developing a kind of Enlightenment Ethics of Belief that can be found in the
final sections of his \titel{Critique of Pure Reason}.

The central task of (a \name[Immanuel]{Kant}ian conception of) Enlightenment
points to intellectual independence, the \singlequote{thinking for oneself} and
the spontaneous activity of our own intellectual faculties that is not ruled by
others but only by ourselves. Now this activity should consist of the
actualization of the common faculty of reason and not in arbitrary
decision making to believe one thing over another. Thinking for oneself
thus seems to be the outcome of a competence rather than the result of a mere
decision. This is an insight which was famously defended by
\authorfullcite{Wolff:Discursuspraeliminarisdephilosophiaingenere1996}. While
\authorcite{Wolff:Discursuspraeliminarisdephilosophiaingenere1996} claims that
this competence is to be understood as being able to follow some special
method---the \emph{mathematical} method---\name[Immanuel]{Kant} points to the social
character of thinking. Thinking for oneself means competently reasoning in
accordance with rules that can only be followed and assessed through
conversational practice.

How is this demand for maturity affected by our finiteness? To answer this
question it is necessary to give a clear concept of our finiteness. According to
\name[Immanuel]{Kant}, the core of our finiteness as human beings  consists in
our dependency on being passively affected: on cognizing things with our senses
instead of through pure reason or by what we are told. We cannot cognize things
without being affected by them; all that we know by pure reason alone are very
general truths (for instance,  \enquote{things happen in accordance to natural
laws}). The \emph{terminus technicus} for this concept is
\singlequote{discursivity} which is the property of
concepts that distinguishes them from intuitions. A concept is a representation
that is discursive which means that it is mediately related to its objects via
universal marks and is hence a universal representation. Intuitions on the other
hand are immediately related to their objects and are hence singular
representations. Our understanding is discursive because it is a faculty of
thinking which cognizes objects through concepts that are dependent on a
faculty of intuitions to relate to objects. Without such
\singlequote{intuitive}, i.\,e. immediate relations our understanding would have
no content at all.

Most work on testimonial knowledge focuses on the trustworthiness and competence
of the speaker. Although \name[Immanuel]{Kant} addresses such topics,
these are rather secondary reflections unsuited to be an adequate foundation for
understanding of the compatibility of intellectual independence and the
dependency on others. In freely following Wolff Kant distinguishes in a first
step between rational and historical cognitions. A cognition is rational if it
is generated by the cognizer's own use of his pure reason, otherwise it is
historical, i.\,e.
it is given to him by his senses or the words of others. In a second step he
makes a difference between rational and empirical cognitions, the latter being
cognitions that can only be known historically. While the distinction between
rational and \emph{empirical} cognitions signifies a difference in their very
character, the difference between rational and \emph{historical}
cognitions means a difference in our own personal access to them. (In a
further step he distinguishes between \singlequote{discursive} rational
cognitions in philosophy and \singlequote{intuitive} rational cognitions in
mathematics, but this is of lesser importance than the first distinction.) The
point is that different kinds of cognitions are to be treated in different ways.
Having a critical stance toward testimonial knowledge means that we should not accept
rational cognitions on the grounds of being told.

Knowingly diverging from his predecessors \name[Immanuel]{Kant} defines
metaphysics as the system of rational cognitions, i\,e. the systematically
ordered body of knowledge we gained through pure reason. Thus metaphysics is the
discipline that does not allow for just historical cognitions. Now to
\name[Immanuel]{Kant} philosophy is not---as it has been for
\authorcite{Wolff:Discursuspraeliminarisdephilosophiaingenere1996} and
others---the whole body of science besides mathematics, but is rather identical
with metaphysics. Summarizing all these results we can assign a reason for why
we should learn philosophizing rather than philosophy: It is because metaphysics
is a rational discipline that we cannot learn by acquiring mere historical
knowledge but only by learning to use our own faculty of reason.
\selectlanguage{ngerman}
