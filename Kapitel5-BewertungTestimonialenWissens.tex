
Wir müssen die Möglichkeit testimonialen Wissens anerkennen und einräumen, dass
weder ein testimonialer Reduktionismus noch ein testimonialer Skeptizismus tragfähig sind.
Es liegt aber auch auf der Hand, dass Selbstdenken im Gegensatz zu
Leichtgläubigkeit steht und es möglich sein muss zu
erläutern, was einen leichtfertigen Umgang mit testimonialem Wissen von einem
kritischen Umgang unterscheidet.

So verwundert nicht, dass sich in der deutschen Aufklärungsphilosophie
Auseinandersetzungen mit der Frage finden, wie Informationen zu bewerten sind.
Zunächst fallen dabei diejenigen Ansätze auf, die die Informationsquelle
\emph{respective} den Prozess der Informationsübermittlung fokussieren. Es
bietet sich an, dabei zunächst an
\authorcite{Crusius:WegzurGewissheitundZuverlaessigkeitdermenschlichenErkenntniss1965}
und seine Theorie einer \singlequote{historischen Präsumtion} anzuschließen.
Immerhin spricht auch \name[Immanuel]{Kant} von einer Präsumtion, dass sich
Unwahrheiten selbst verraten.\footnote{\cite[Vgl.][\nopp
2589]{Kant:Reflexionen1900ff.}, \cite[][XVI: 430.6]{Kant:GesammelteWerke1900ff.}.} Neben
\authorcite{Crusius:WegzurGewissheitundZuverlaessigkeitdermenschlichenErkenntniss1965}
werde ich vor allem auf \authorfullcite{Meier:AuszugausderVernunftlehre1752} und
\authorfullcite{Reimarus:DieVernunftlehrealseineAnweisungzumrichtigenGebrauchderVernunftinderErkenntnisderWahrheit1756}
eingehen (Kapitel \ref{subsection:BewertungvonInformationenanhandihrerQuellen}), um schließlich zu
zeigen, dass \name[Immanuel]{Kant} entsprechende Ideen zwar thematisiert, sie
aber letztlich nicht hinreichen um zu beschreiben, wie ein mündiger Umgang mit
Informationen aus zweiter Hand beschaffen ist. Aussichtsreicher ist hingegen die
Bewertung der Informationen anhand der \emph{Art} der mitgeteilten Erkenntnis.
Entsprechende Überlegungen knüpfen direkt an
\authorcite{Descartes:OeuvresdeDescartes1983}' drittes Argument gegen die
Büchergelehrsamkeit an, indem sie explizieren, worin der Unterschied zwischen
(bloßer) historischer Kenntnis  und Wissenschaft besteht. Diese Herangehensweise untersuche
ich zunächst bei \authorfullcite{Wolff:Discursuspraeliminarisdephilosophiaingenere1996} (Kapitel
\ref{subsection:BewertungvonInformationennachihrerART}). Sie wird
sich später als Ausgangspunkt der Überlegungen \name[Immanuel]{Kant}s erweisen
(Kapitel \ref{Chapter:KantsSocialEpistemology} und darin insbesondere Kapitel
\ref{section:MuendigkeitundPhilosophie}).

\section[Bewertung anhand
der Quellen]{Bewertung von Informationen anhand
der Quellen}\label{subsection:BewertungvonInformationenanhandihrerQuellen}
\subsection{Kompetenz und
Aufrichtigkeit}\label{subsubsection:GeorgFriedrichMeier}
\authorcite{Crusius:Anweisungvernuenftigzuleben1744} gibt ausführliche Hinweise
dazu, wie testimoniale Erkenntnisse zu evaluieren sind.\footnote{In den {\S\S}
615--626 seiner Schrift \titel{Weg zur Gewißheit und Zuverläßigkeit der
menschlichen Erkenntniss} nennt er 22 konkrete Regeln im Umgang mit
Informationen aus zweiter Hand. \authorcite{Crusius:Anweisungvernuenftigzuleben1744} nummeriert die Regeln von 1 bis 21, jedoch ist die Nummer 17 doppelt vergeben;
\cite[vgl.][\S\S~623\,f.]{Crusius:WegzurGewissheitundZuverlaessigkeitdermenschlichenErkenntniss1965}.}
Sie beruhen auf allgemeineren Überlegungen zur Bewertung von Informationen, die
er diesen Regeln vorausschickt, und die für meine Belange interessanter sind, als
die konkreten Regelausformulierungen. Die Betonung der Seite des Gegenstandes
der Information, welche \authorcite{Crusius:Anweisungvernuenftigzuleben1744}
inhaltlich mit \name[David]{Hume} verbindet, stellt  eine dieser Überlegungen
dar. Relevant bei der Bewertung der
Bonität von Informanten und Informationen sind dann auf der Seite des
Informanten zunächst dessen Kompetenz und Aufrichtigkeit.
\authorcite{Crusius:Anweisungvernuenftigzuleben1744} schreibt:
\begin{quote}
  Bey den Zeugnissen können vier Umstände in Betrachtung kommen: ob
  einer ein wahres Zeugniß hat ablegen können; ob er es auch hat ablegen wollen;
  ob und wiefern er hat betrügen können; und wiefern sein Zeugniß mit andern
  Zeugnissen übereinstimmet.\footnote{\Cite[][\S~611]{Crusius:WegzurGewissheitundZuverlaessigkeitdermenschlichenErkenntniss1965}.}
\end{quote}
Der Bonitätspräsumtion folgend dürfen wir immer zunächst davon ausgehen, dass
unser Informant \enquote{ein wahres Zeugniß hat ablegen können} -- dass er etwa
\emph{kompetent} genug ist -- und dass \enquote{er es auch hat ablegen wollen} -- dass
er \emph{aufrichtig} ist.

Überlegungen zur Bonität von Informanten, wie wir sie nun als einen Aspekt der
Bewertung von Informationen bei
\authorcite{Crusius:Anweisungvernuenftigzuleben1744} fanden, lassen sich auch
bei \name[Immanuel]{Kant} belegen\footnote{Oliver
\textcite[][839\,f.]{Scholz:AutonomieangesichtsepistemischerAbhaengigkeiten2001}
sieht hierin den Schlüssel zu einem Verständnis, was Mündigkeit bei
testimonialem Wissen nach \name[Immanuel]{Kant} sein kann. Auch Thomas
\textcite[][143--146]{Kater:PolitikRechtGeschichte1999} betont diese
Überlegungen.}; Thomas \name[Thomas]{Kater} führt beispielsweise vier Stellen
aus Mitschriften zu \name[Immanuel]{Kant}s Logikvorlesungen
an.\footnote{\cite[Vgl.][143--146]{Kater:PolitikRechtGeschichte1999}. Die
einschlägigen Passagen finden sich in \cite[][]{Kant:LogikPoelitz1966}, \cite[][XXIV:
242.4--250.9]{Kant:GesammelteWerke1900ff.},
\cite[][]{Kant:LogikPhilippi1966}, \cite[][XXIV:
448.20--450.35]{Kant:GesammelteWerke1900ff.}, \cite{Kant:LogikPoelitz1966},
\cite[][XXIV: 561.34--563.25]{Kant:GesammelteWerke1900ff.}, und
\cite{Kant:WienerLogik1966}, \cite[][XXIV: 893.24--900.32]{Kant:GesammelteWerke1900ff.}.} In diesen
Vorlesungen geht auch \name[Immanuel]{Kant} von einer \emph{prima
facie} anzunehmenden Bonität von Informanten und Informationen aus und verweist
auf Kompetenz und Aufrichtigkeit als entscheidende Kriterien bei der Bewertung von Informanten.
Soweit stimmen seine Auskünfte mit denen von \authorcite{Crusius:Anweisungvernuenftigzuleben1744} überein. Doch der
Ausgangspunkt seiner Überlegungen ist ein anderer: Der Textaufbau
sowie die übernommene Paragrapheneinteilungen weisen die Überlegungen
\name[Immanuel]{Kant}s zunächst als Kommentierungen der \S\S~206--214 des
\titel{Auszugs aus der Vernunftlehre} von Georg
\authorcite{Meier:Vernunftlehre1752} aus, der darin selbst wiederum zumindest
der Tendenz nach der Position \authorcite{Wolff:Psychologiaempirica1968}s in den
\titel{Vernünftige[n] Gedanken von den Kräften des menschlichen Verstandes und
ihrem richtigen Gebrauche in Erkenntnis der Wahrheit} folgt\footnote{Zu
\authorcite{Wolff:Psychologiaempirica1968}s Position zum Umgang mit testimonialem Wissen siehe
\cite[][200--203]{Wolff:VernuenftigeGedankenvondenKraeftendesmenschlichenVerstandesundihremrichtigenGebraucheinErkenntnisderWahrheit1978},
sowie
\cite[][\S\S~149--161]{Wolff:Cogitationesrationalesdeviribusintellectushumani1983}.}.
\authorcite{Meier:Vernunftlehre1752}s Position zur Mündigkeit im Umgang mit
testimonialen Erkenntnissen (der \enquote{vernünftige Glaube}) liest sich wie
folgt:
\begin{quote}
  \ori{Ein Zeuge ist glaubwürdig} (testis fide dignus), wenn man auf eine
  gelehrte Art wenigstens wahrscheinlich erkennen kann, dass er genugsames
  Ansehen habe; das Zeugnis eines solchen Zeugen ist \ori{ein glaubwürdiges
  Zeugniss} (testimonium fide dignum). \ori{Der vernünftige} oder \ori{sehende
  Glaube} (fides oculata, rationalis) ist die Fertigkeit nur glaubwürdigen
  Zeugen zu
  glauben.\footnote{\cite[][\S~214]{Meier:AuszugausderVernunftlehre1752},
  \cite[][XVI: 509.20--24]{Kant:GesammelteWerke1900ff.}.}
\end{quote}
Einen unvernünftigen Glauben nennt \authorcite{Meier:Vernunftlehre1752} wie \authorcite{Wolff:Psychologiaempirica1968}
\enquote{Leichtgläubigkeit}.\footnote{\cite[Vgl.][\S~213]{Meier:AuszugausderVernunftlehre1752},
\cite[][XVI: 508.26--27]{Kant:GesammelteWerke1900ff.};
\textcite[][201]{Wolff:VernuenftigeGedankenvondenKraeftendesmenschlichenVerstandesundihremrichtigenGebraucheinErkenntnisderWahrheit1978}
schreibt ähnlich wie \authorcite{Meier:Vernunftlehre1752}: \enquote{Jedoch, damit wir nicht
leichtgläubig sind, und uns betrügen lassen; so müssen wir uns zweyer Dinge erst versichern: 1)
daß derjenige, welcher etwas zeuget, die Sache recht habe erkennen können,
damit er sich nicht selbst betrogen: 2) daß er die Sache so erzehlet, wie er
sie erkandt hat, damit er nicht den Vorsatz habe, andere zu betrügen. Mit
einem Worte, ich muß versichert seyn, daß der Zeuge klug und aufrichtig genug
sey.} Was bei \authorcite{Wolff:Psychologiaempirica1968} gegenüber \authorcite{Meier:Vernunftlehre1752} fehlt, ist lediglich der
Verweis auf die sprachliche Kompetenz, das korrekt Erkannte auch verständlich
auszudrücken.} Vernünftig ist es, die Zustimmung zu einer Aussage, die
Gegenstand einer Mitteilung ist, von dem \enquote{Ansehen} des Informanten
abhängig zu machen. Das Ansehen wiederum bestehe -- ähnlich wie bei
\authorcite{Crusius:Anweisungvernuenftigzuleben1744} -- in der Kompetenz oder
\enquote{Tüchtigkeit} (\authorcite{Wolff:Psychologiaempirica1968} spricht stattdessen von
\enquote{Klugheit}\footnote{\cite[Vgl.][201]{Wolff:VernuenftigeGedankenvondenKraeftendesmenschlichenVerstandesundihremrichtigenGebraucheinErkenntnisderWahrheit1978}.})
und der \enquote{Aufrichtigkeit} des Informanten, also darin, dass er zum
einen uns nicht absichtlich irre führt (Aufrichtigkeit), und zum anderen selbst
die nötigen intellektuellen und epistemischen Voraussetzungen mitbringt, um das,
worüber er Auskunft gibt, auch zuverlässig beurteilen zu
können.\footnote{\phantomsection\label{Fussnote:MeierzurBonitaetdesZeugen}\cite[Vgl.][\S~207]{Meier:AuszugausderVernunftlehre1752},
\cite[][XVI: 504.21--30]{Kant:GesammelteWerke1900ff.}} Die
\enquote{Tüchtigkeit} oder epistemischen Voraussetzungen eines Informanten
bestimmt \authorcite{Meier:Vernunftlehre1752} des weiteren wie folgt:
\begin{quote}
  1) er muss bei der Sache gegenwärtig sein, die er bezeugt; 2) er muss im Stande sein, eine richtige
Erfahrung zu bekommen; 3) er muss ein gutes und treues Gedächtnis haben, oder
seine Erfahrung alsobald aufschreiben; 4) er muss die Gabe besitzen, seine
eigene Erkenntniss auf eine richtige und hinlängliche Art zu
bezeichnen.\footnote{\cite[][\S~209]{Meier:AuszugausderVernunftlehre1752},
\cite[][XVI: 505.24--29]{Kant:GesammelteWerke1900ff.}.}
\end{quote}
Von der Kompetenz des Informanten ist also die notwendige Erfahrungssituation
nochmal zu unterscheiden, was besonders bei Zeugen im alltäglichen Sinn
dieses Wortes ersichtlich ist. Ein guter Zeuge ist, wer sich in einer
bestimmten Situation befand, um eine Erfahrung machen zu können, die andere
nicht gemacht haben (\enquote{er muß bei der Sache gegenwärtig sein}). Dies
betrifft in mehr oder weniger starker Ausprägung alle Informanten. Nur aus
bestimmten Positionen im Verhältnis zu einem Fußballfeld lassen sich
beispielsweise Abseitspositionen erkennen; wer falsch steht, wird in der Regel
zugeben, nicht erkannt zu haben, ob Abseits vorlag. Diese Wahrnehmungsposition
begründet eine epistemische Asymmetrie im Falle von Mitteilungen. Genauso
spielt die Kompetenz des Informanten eine Rolle. Wüsste der
Schiedsrichterassistent nicht, was eine Abseitsposition ist, dann nützte sein
Signal der Schiedsrichterin nicht bei ihrer Entscheidung. Die
Wahrnehmungsposition macht in der Regel die epistemische Asymmetrie im Falle von
Zeugen, die Kompetenz die Asymmetrie im Falle von Experten aus. Beide
Aspekte -- Wahrnehmungsposition und Kompetenz -- werden in der Regel
zusammengefasst zu der Kompetenz oder Fähigkeit, in einem konkreten Fall eine
bestimmte Aussage begründet tätigen zu können, was bei
\authorcite{Meier:Vernunftlehre1752} \enquote{Tüchtigkeit} heißt und der
Aufrichtigkeit gegenübergestellt wird.\footnote{Siehe z.\,B.
\cite[][108]{Goldman:KnowledgeinaSocialWorld1999}: \enquote{There are two types
of skepticism receivers might have about a speaker or her report. First,
receivers might worry about the source's competence to make an accurate
observation or interpretation of the alleged state of affairs, especially in a
technical subject matter. {\punkt} The second category of
possible skepticism concern's the reporter's honesty rather than competence.} An
einer anderen Stelle unterscheidet \authorcite{Goldman:KnowledgeinaSocialWorld1999} dann die genannten
\emph{drei} Momente: \enquote{I propose a {\punkt} classification with three elements: (A)
the reporter’s \ori{competence}, (B) the reporter’s \ori{opportunity}, and (C)
the reporter’s \ori{sincerety} or \ori{honesty}}
(\cite[][123]{Goldman:KnowledgeinaSocialWorld1999}).}
\authorcite{Scholz:DasZeugnisanderer2001} (der sich jedoch nicht auf
\authorcite{Meier:Vernunftlehre1752}, sondern auf \name[Immanuel]{Kant} beruft)
spricht von den Bedingungen der \enquote{\emph{Aufrichtigkeit}} und der
\enquote{\emph{Kompetenz}},\footnote{\cite[Vgl.][361]{Scholz:DasZeugnisanderer2001}:
\enquote{Für das Zeugnis anderer sind die Gültigkeitsbedingungen: (i) dass die bezeugende
Person bei ihrer Äußerung aufrichtig ist, d.\,h., dass sie glaubt, was sie
behauptet (Bedingung der \ori{Aufrichtigkeit}), und (ii) dass sie (bei der
Gelegenheit $O$) kompetent bezüglich des Sachverhalts $p$ ist, um den es in
ihrer Äußerung geht -- und zwar in dem starken Sinne, dass die durch die
Behauptung ausgedrückte Überzeugung wahr ist oder in einem geeigneten schwächeren Sinne (Bedingung der
\ori{Kompetenz}). An den beiden Bedingungen zusammengenommen bemißt [sic] sich
die Glaubwürdigkeit der bezeugenden Person.}} wobei vergessen zu werden droht,
dass zur \singlequote{Tüchtigkeit} neben der Kompetenz auch die richtige
Positionierung im Verhältnis zum Gegenstand des Erkennens gehört. Es ist aber
letztlich unbedeutend, ob wir die richtige Positionierung als Aspekt
\emph{neben} der Kompetenz anführen oder ob wir den Ausdruck \emph{Kompetenz} so
verstehen, dass er diese Positionierung mit enthält. Im folgenden wird dieses
Detail in der Regel keine nennenswerte Funktion übernehmen, sodass ich der
Einfachheit halber von \enquote{Kompetenz} spreche, ohne damit die
Positionierung auszuschließen.

\name[Immanuel]{Kant} folgt \authorcite{Meier:Vernunftlehre1752} in der obersten
Zweiteilung und fasst die Kriterien der Bewertung zu den beiden genannten zusammen: \enquote{Die
Tüchtigkeit des Zeugen besteht darin, daß er hat \ori{können} die Wahrheit
sagen. \punkt{} Die
Aufrichtigkeit des Zeugen, daß er hat \ori{wollen} die Wahrheit
sagen.}\footnote{\cite{Kant:WienerLogik1966}, \cite[][XXIV: 898.3--4,
20--21]{Kant:GesammelteWerke1900ff.}.} Es greift also zu kurz, wenn Thomas
\name[Thomas]{Kater}  nur auf die
Aufrichtigkeit oder \enquote{Wahrhaftigkeit}
abhebt\footnote{\cite[Vgl.][145]{Kater:PolitikRechtGeschichte1999}:
\enquote{Wahrhaftigkeit des Bezeugenden ist also das wesentliche Kriterium, das
die Beurteilung von Zeugnissen leitet.}}; der Informant muss nicht nur die
Wahrheit sagen \emph{wollen}, er muss dies auch \emph{können}. Doch wie zu
sehen sein wird reicht es nicht, die beiden Aspekte Aufrichtigkeit und
Tüchtigkeit anzuführen; mitunter sind ganz andere Überlegungen relevant, die
sich nicht bei \authorcite{Meier:Vernunftlehre1752}, sondern erst bei
\authorcite{Reimarus:DieVernunftlehrealseineAnweisungzumrichtigenGebrauchderVernunftinderErkenntnisderWahrheit1756}
finden.

Schaut man bei \name[Immanuel]{Kant} lediglich auf die Ausführungen, die aus
seinen Logikvorlesungen überliefert sind, so finden sich genau die Bemerkungen,
die bei allen genannten Autoren ebenfalls zu finden sind. Man solle -- so warnt
\name[Immanuel]{Kant} -- nur \enquote{aufrichtigen} und \enquote{tüchtigen}
Informanten, nicht aber dem \enquote{gemeinen Mann}
glauben\footnote{Vgl. \cite{Kant:LogikPoelitz1966},
\cite[][XXIV: 562.30--38]{Kant:GesammelteWerke1900ff.},
\cite{Kant:WienerLogik1966}, \cite[][XXIV: 898.1--22]{Kant:GesammelteWerke1900ff.}.}, eher das als wahr
annehmen, was von vielen unabhängigen Informanten mitgeteilt
wird\footnote{Vgl. \cite{Kant:LogikPhilippi1966}, \cite[][XXIV:
450.23--28]{Kant:GesammelteWerke1900ff.}, sowie \cite{Kant:LogikPoelitz1966},
\cite[][XXIV: 563.15--17]{Kant:GesammelteWerke1900ff.}.}, und
\enquote{Augenzeugen}, das sind Informanten, die etwas aus erster Hand erfahren
haben, von \enquote{Ohrenzeugen} unterscheiden, die selbst nur wissen, was
andere ihnen
erzählten\footnote{\phantomsection\label{Anmerkung:AugenzeugenundOhrenzeugen}Vgl.
\cite{Kant:LogikPhilippi1966}, \cite[][XXIV:
450.20--28]{Kant:GesammelteWerke1900ff.}. An dieser Stelle erläutert
\name[Immanuel]{Kant} den \S~208 des \titel{Auszugs aus der Vernunftlehre}, an
der \authorcite{Meier:Vernunftlehre1752} von
\enquote{Augenzeugen} und \enquote{Hörenzeugen} spricht
(\cite[vgl.][\S~208]{Meier:AuszugausderVernunftlehre1752}, \cite[][XVI:
505.20--23]{Kant:GesammelteWerke1900ff.}),
\authorcite{Reimarus:DieVernunftlehrealseineAnweisungzumrichtigenGebrauchderVernunftinderErkenntnisderWahrheit1756}
hingegen von \enquote{Hauptzeugen} oder \enquote{ersten Zeugen} und
\enquote{Mittelzeugen}
(\cite[vgl.][\S\S~249--251]{Reimarus:DieVernunftlehrealseineAnweisungzumrichtigenGebrauchderVernunftinderErkenntnisderWahrheit1756}).}.
Auch solle man darauf achten, dass ein Zeuge von Falschauskünften keinen Nutzen
habe.\footnote{Vgl. \cite{Kant:LogikPhilippi1966}, \cite[][XXIV:
450.3--4]{Kant:GesammelteWerke1900ff.}.} Aber welchen Stellenwert haben diese Überlegungen innerhalb der kantischen
Positionierung zur Verträglichkeit von Mündigkeit und testimonialem Wissen?

Es ist auf Grundlage des über
\authorcite{Crusius:Anweisungvernuenftigzuleben1744} und
\authorcite{Meier:Vernunftlehre1752} Gesagten offensichtlich, dass
\name[Immanuel]{Kant} damit lediglich Gemeingut seiner Zeit wiedergibt, ohne viel an eigenen Gedanken
hinzuzufügen.\phantomsection\label{AbschnittzuCrusiusundKritischemJournalismus}
Er bleibt in der Ausführlichkeit und systematischen Ausgestaltung sogar
hinter deren Darstellungen zurück.
Gerade bei der Behandlung der subordinierten Informanten zeigt sich, wie weit er
hinter dem etwa bei \authorcite{Crusius:Anweisungvernuenftigzuleben1744}
erreichten Reflexionsniveau zurückbleibt. Auch dieser sieht natürlich, dass
\singlequote{Augenzeugen} einen \emph{prima facie}-Vorrang vor
\singlequote{Ohrenzeugen}
haben.\footnote{\cite[Vgl.][\S~621]{Crusius:WegzurGewissheitundZuverlaessigkeitdermenschlichenErkenntniss1965}:
\enquote{Ein Zeuge, welcher etwas selbst gesehen  hat, hat bey sonst gleichen
Umständen einen Vorzug vor einem, welcher etwas aus den Nachrichten anderer
hat.} (\ohio)} Aber dieser Vorrang gilt nicht in allen Fällen.
\authorcite{Crusius:Anweisungvernuenftigzuleben1744} schreibt:
\begin{quote}
  Jedoch können auch Fälle vorkommen, da die Glaubwürdigkeit einer Begebenheit
  nicht so wohl von der Versicherung eines Augenzeugen abhanget, weil nemlich
  derselbe unbekannt ist, und keine Präsumtion vor sich hat; sondern da ihr mehr
  von dem Zeugnisse eines mittelbaren Zeugen zuwächset, dessen Verstand und
  Glaubwürdigkeit bekannt ist, und welcher die Präsumtion vor sich hat, daß er
  seine Nachricht auf das Zeugniß tüchtiger Zeugen gegründet, und dieselben wohl
  geprüfet
  habe.\footnote{\Cite[][\S~621]{Crusius:WegzurGewissheitundZuverlaessigkeitdermenschlichenErkenntniss1965},
  \ohio}
\end{quote}
Das hier entwickelte Szenario besagt: Wir haben von einem Ereignis zwei
Berichte. Der erste stammt von einem unmittelbaren Zeugen des Geschehens, von
dem wir allerdings nichts oder sehr wenig wissen. Wir können selbst seine
Bonität daher schwer einschätzen. Der zweite Bericht stammt von einem
Informanten, der selbst auf andere Informanten angewiesen ist, also nur über
testimoniales Wissen von dem Geschehen verfügt. Wir wissen aber auch, dass
dieser Informant über die nötige Kompetenz in der Bewertung von Informanten und
Informationen verfügt. Hier -- sagt
\authorcite{Crusius:Anweisungvernuenftigzuleben1744} -- ist die über mehrere
subordinierte Informanten vermittelte Information glaubwürdiger als diejenige
Information, die wir direkt von einem unmittelbaren Zeugen des Geschehens
erhalten haben.

\name[David]{Hume}s Behauptung, die Gewissheit testimonialer
Erkenntnis könne über die Zeit -- durch die Vermehrung der Zwischenstationen -- nur abnehmen, ist
somit schlicht falsch. Letztlich ändert sich die Gewissheit nicht im Laufe der
Zeit, sonst müsste eine historische Tatsache der Antike im Mittelalter mit viel
größerer Gewissheit für wahr gehalten worden sein als in unseren Tagen. Und dies ist
offensichtlicher Unsinn. Nur wenn sich neue Gegenbelege oder \emph{defeater}
finden, kann sich der Grad der Gewissheit
verringern.\footnote{\cite[Vgl.][\S~626]{Crusius:WegzurGewissheitundZuverlaessigkeitdermenschlichenErkenntniss1965}:
\enquote{Die historische Wahrscheinlichkeit nimmt durch die Länge der Zeit nicht
ab, es wäre denn, daß die Beweisgründe derselben in folgenden Zeiten durch neue
Gründe entkräftet oder ausdrücklich widerleget werden könten; oder daß die
Erkenntnißgründe selbst, und zwar dergestalt verloren giengen, daß die übrig
bleibenden zu einer Zuverläßigkeit oder moralischen Gewißheit nicht mehr
hinlänglich wären.} (\ohio)} Die Qualität einer Information
kann sich sogar erhöhen, wenn sie vermittelt über eine weitere Station
akquiriert wird, und zwar dann, wenn dieser mittlere Informant sich uns gegenüber durch eine höhere
Kompetenz auszeichnet, die Qualität der ursprünglichen Information zu bewerten. Ein
Beispiel liefert uns die Idee eines kritischen Journalismus'. Natürlich
ist es uns möglich, Pressemitteilungen von Parteien selbst zu lesen. Aber es stimmt nicht, dass wir
dann besser informiert wären, als wenn wir die Aufbereitung der Informationen in
einer Tageszeitung rezipieren: Ein guter Journalist weiß, welchen Informationen
er trauen kann, welche zu überprüfen sind und wie er im Zweifelsfall nachhaken
sollte. Ein allgemeiner Vorzug kurzer Informationsketten lässt sich also
mitnichten rechtfertigen. Die Forderung nach Mündigkeit sollte also nicht als
Forderung nach kurzen Informationsketten missverstanden werden.
\singlequote{Kritischer Journalismus} fördert gerade Mündigkeit -- und zwar
dadurch, dass er der Informationskette ein weiteres Glied hinzufügt.

\name[Immanuel]{Kant} verwendet wenig Energie auf die Ausarbeitung einer
systematischen und tragfähigen Darstellung der Bewertung von Informanten,
sondern versorgt lediglich die Zuhörer seiner Vorlesungen mit groben
Informationen über längst bekannte Einsichten und Überlegungen. Kompetenz oder
\singlequote{Tüchtigkeit} und Aufrichtigkeit sind gewiss wichtige Aspekte in der
Beurteilung testimonialer Erkenntnisse, aber sie sind nicht die Gesichtspunkte,
denen in \name[Immanuel]{Kant}s Philosophie systematische Relevanz zukommt. Dass
\name[Immanuel]{Kant} in seinen Vorlesungen gerade diese Aspekte anspricht, ist
möglicherweise nur der Tatsache geschuldet, dass Georg Friedrich
\authorcite{Meier:Vernunftlehre1752}, dessen Lehrbuch in diesen Punkten wenig ausführlich ist, gerade
die beiden behandelt. Für \name[Immanuel]{Kant}s Philosophie relevant
sind Überlegungen, die er selbst in seinen Logikvorlesungen nicht ausführlich
thematisiert, wie etwa die Hermeneutik und Quellenkritik.


\subsection{Hermeneutik und
Quellenkritik}\label{subsubsection:HermannSamuelReimarus} Hermann Samuel
\authorcite{Reimarus:DieVernunftlehrealseineAnweisungzumrichtigenGebrauchderVernunftinderErkenntnisderWahrheit1756}
hat eine systematisch äußerst ausgefeilte Darstellung davon erarbeitet, wann
Informationen oder Informanten glaubwürdig
sind.\footnote{\cite[Vgl.][\S\S~239--258]{Reimarus:DieVernunftlehrealseineAnweisungzumrichtigenGebrauchderVernunftinderErkenntnisderWahrheit1756}.}
Dabei ist zu beachten, dass
\authorcite{Reimarus:DieVernunftlehrealseineAnweisungzumrichtigenGebrauchderVernunftinderErkenntnisderWahrheit1756}
die Bonität nicht nur des \emph{Informanten}, sondern der
\emph{Information}\footnote{Der Ausdruck \enquote{Information} ist hier
systematisch mehrdeutig, insofern er sprachlich einerseits auf den gesamten
Prozess der Übermittlung eines Informationsgehalts durch einen Informanten und
ein Medium der Informationsübertragung verweisen kann, andererseits aber auch
auf den bloßen Informationsgehalt. Hier ist die erste Bedeutung angesprochen,
für die ich auch von Informationsprozess sprechen werde -- unter der Maßgabe,
dass der Informationsgehalt mit inbegriffen ist.} oder des
\singlequote{Zeugnisses} -- also des \emph{Gesamts} einer Nachricht, ihres
Überbringers und des Prozesses ihrer Überbringung --
thematisiert,\footnote{\cite[Vgl.][\S~239]{Reimarus:DieVernunftlehrealseineAnweisungzumrichtigenGebrauchderVernunftinderErkenntnisderWahrheit1756}:
\enquote{\ori{Glaubwürdig} heißt ein Zeugniß, wenn es wegen seiner Wahrheit
werth ist, in die Stelle unserer eigenen Erfahrung gesetzt zu werden.}} wovon
wiederum die Bonität des Informanten nur einen Teil beschreibt und nicht wie bei
\authorcite{Meier:Vernunftlehre1752} die der Information bereits
garantiert\footnote{\cite[Vgl.][\S~214]{Meier:AuszugausderVernunftlehre1752},
\cite[][XVI: 509.22--23]{Kant:GesammelteWerke1900ff.}.}.


Zu den beiden Aspekten Kompetenz und Aufrichtigkeit, welche diese Bonität des
Informanten beschreiben, kommen bei der Bewertung der Bonität des Zeugnisses
zwei weitere Aspekte hinzu. Mit ihrer Hilfe lassen sich weitere
Überlegungen auch bei \name[Immanuel]{Kant} identifizieren, die insbesondere seine
Religionsphilosophie prägen.\footnote{Das Interesse an Überlegungen zu
testimonialem Wissen innerhalb der Philosophie des 18.~Jahrhunderts speist sich
zu einem erheblichen Teil aus Fragen bezüglich der Zuverlässigkeit und Tragkraft
heiliger Schriften, speziell natürlich des Neuen Testaments. Bei David
\name[David]{Hume} finden sich einschlägige Überlegungen entsprechend in
Abschnitt 10 (\enquote{On Miracles}) des \titel{Enquiry Concerning Human
Understanding}. Und
\authorfullcite{Reimarus:DieVernunftlehrealseineAnweisungzumrichtigenGebrauchderVernunftinderErkenntnisderWahrheit1756},
dessen Bekanntheit sich primär den von \authorcite{Lessing:EineDuplik1897}
anonymisiert herausgegeben \singlequote{Fragmenten} verdankt und der uns
hauptsächlich wegen seiner Religionsphilosophie und Offenbarungskritik bekannt
ist, liefert mit die ausführlichsten Überlegungen zu testimonialem Wissen in der
deutschen Aufklärungsphilosophie.
\cite[Vgl.][\S\S~239--258]{Reimarus:DieVernunftlehrealseineAnweisungzumrichtigenGebrauchderVernunftinderErkenntnisderWahrheit1756}.
In allen mir bekannte Fällen geht es in irgendeiner Form um Offenbarungskritik.}
\authorcite{Reimarus:DieVernunftlehrealseineAnweisungzumrichtigenGebrauchderVernunftinderErkenntnisderWahrheit1756}
nennt als wichtige Aspekte im Umgang mit testimonialem Wissen:
\begin{quote}
  \begin{enumerate}
  \item Ob des Zeugen Erfahrung einerley sey mit dem, was wirklich geschehen
  ist? welches auf seine dargelegte Geschicklichkeit in der Erfahrung ankömmt.
  \item Ob seine Nachricht davon einerley sey mit dem, was er sich selbst
  vorstellet erfahren zu haben? welches von seiner willkürlichen Aufrichtigkeit
  im Berichte abhängt.
  \item Ob der Verstand, welchen wir aus seiner Nachricht gezogen, einerley sey
  mit dem, was er hat bezeugen wollen? das gehöret für die Erklärungskunst.
  \item Ob der vermeynte Zeuge eben die Person sey, welche die Nachricht
  gegeben? das gehöret für die
  Kritik.\footnote{\Cite[][\S~240]{Reimarus:DieVernunftlehrealseineAnweisungzumrichtigenGebrauchderVernunftinderErkenntnisderWahrheit1756}.}
  \end{enumerate}
\end{quote}
Die ersten beiden Teilaspekte der Bonitätsfrage entsprechen dem, was
wir mit \authorcite{Meier:Vernunftlehre1752} bereits als
\singlequote{Tüchtigkeit} oder Kompetenz (hier: \enquote{Geschicklichkeit}) und
Aufrichtigkeit beschrieben haben. Die beiden anderen Kriterien sind neu und ihre Beachtung
dadurch bedingt, dass nicht mehr der Informant, sondern der gesamte
Informationsprozess betrachtet wird. Während uns im Alltag zunächst Fragen nach
Kompetenz und Aufrichtigkeit des Informanten beschäftigen, sind Heuristik und
Kritik gerade bei der Beurteilung historischer Dokumente wie alter Urkunden oder
heiliger Schriften von Bedeutung.


Es ergeben sich vier Stichworte zum mündigen Umgang mit testimonialem Wissen:
(1) die \emph{Kompetenz} des Informanten, (2) die \emph{Aufrichtigkeit} des
Informanten, (3) \emph{Hermeneutik} oder die Frage nach dem korrekten
Verständnis des Zeugnisses auf der Seite des Informationsempfängers und (4)
\emph{Kritik} oder die Identifizierung des Informanten.
Freilich sind hermeneutische Überlegungen in diesem Zusammenhang nicht neu;
\authorcite{Meier:Vernunftlehre1752} thematisiert sie zwar nur am Rande
und sieht die Verständigung in der Verantwortung des Informanten, aber gerade
\authorcite{Crusius:Anweisungvernuenftigzuleben1744} setzt sich direkt im
Anschluss an seine Überlegungen zu testimonialem Wissen ausführlich mit solchen
Fragen auseinander.\footnote{\cite[Vgl.][\S\S~628--656]{Crusius:WegzurGewissheitundZuverlaessigkeitdermenschlichenErkenntniss1965}.
Siehe dazu auch \cite[][44--51]{Scholz:VerstehenundRationalitaet1999}.} Doch
\authorcite{Reimarus:DieVernunftlehrealseineAnweisungzumrichtigenGebrauchderVernunftinderErkenntnisderWahrheit1756}
thematisiert sie explizit als Aspekte der Bewertung von Informationen durch
einen mündigen Informationsempfänger.


Die Hermeneutik behandelt zunächst nicht Fähigkeiten und Neigungen der Urheber
testimonialer Erkenntnisse (der Informanten), sondern Kompetenzen und
Tätigkeiten der Empfänger oder Adressaten von Informationen. Während
\authorcite{Meier:Vernunftlehre1752} die Sorge um die rechte Verständlichkeit
von Mitteilungen Sache des Mitteilenden und seiner sprachlichen Kompetenzen sein
ließ, nimmt
\authorcite{Reimarus:DieVernunftlehrealseineAnweisungzumrichtigenGebrauchderVernunftinderErkenntnisderWahrheit1756}
den Empfänger der Mitteilungen in die Pflicht. Gerade Fragen der Hermeneutik fordern nach
\authorcite{Reimarus:DieVernunftlehrealseineAnweisungzumrichtigenGebrauchderVernunftinderErkenntnisderWahrheit1756}
das aktive und kompetente Mitdenken des Rezipienten, denn nur
derjenige kann verstehen, was ihm mitgeteilt wird, der die nötige Sprach-, aber
auch Fachkompetenz besitzt und diese im konkreten Einzelfall aktiv einsetzt.
Sprach- und Sachkompetenz sind dabei als zusammenhängend zu betrachten. Denn
erstens ist die Bedeutung sprachlicher Ausdrücke und ihr Verständnis nicht
unabhängig von \mbox{(Welt-)} Wissen über entsprechende
Gegenstände.\footnote{\cite[Vgl.][\S~256]{Reimarus:DieVernunftlehrealseineAnweisungzumrichtigenGebrauchderVernunftinderErkenntnisderWahrheit1756}:
\enquote{Die Sprachkunde ist mit der Sachenkunde genau verbunden, weil niemand
ein Wort verstehen kann, ohne von der Sache einen Begriff zu haben.}
\authorcite{Reimarus:DieVernunftlehrealseineAnweisungzumrichtigenGebrauchderVernunftinderErkenntnisderWahrheit1756} verwirft hier die oft als selbstverständlich unterstellte
Trennung zwischen begrifflichem Wissen und Weltwissen. Ähnliche Aussagen finden
sich im 20. Jahrhundert bei \textcite{Quine:TwoDogmasofEmpiricism1951} und
\cite{Davidson:OntheVeryIdeaofaConceptualScheme1974}.} Deswegen ist
\authorcite{Meier:Vernunftlehre1752}s Unterscheidung zwischen der Kompetenz,
eine richtige Erfahrung zu machen, und der Fähigkeit, diese korrekt mitzuteilen,
nicht unproblematisch; denn wer beispielsweise über den Begriff der Abseitsstellung
nicht verfügt, kann auch nicht die Erfahrung machen, dass Daniel im Abseits
steht, als Dominik ihm den Ball zuspielt. Zweitens sind sprachliche Ausdrücke in
ihrer Bedeutung
kontextsensitiv.\footnote{\cite[Vgl.][\S~257]{Reimarus:DieVernunftlehrealseineAnweisungzumrichtigenGebrauchderVernunftinderErkenntnisderWahrheit1756}:
\enquote{Bey vielerley bedeutenden Worten kann die eine wirkliche Bedeutung
jeder Stelle nicht anders, als durch den Zusammenhang der Wörter und Sachen,
bestimmt seyn.}} Man täuscht sich, wenn man glaubt von Informanten verlangen zu
können, sie sollten ihre Äußerungen so tätigen, dass der Äußerungskontext von
den Rezipienten nicht mitberücksichtigt werden müsste. Die unvermeidbare
Kontextsensitivität sprachlicher Äußerungen nimmt immer den Rezipienten in die
Pflicht. Und drittens sind auch die Gedanken der Informanten nicht von Anfang an
hinreichend
bestimmt.\footnote{\cite[Vgl.][\S~254]{Reimarus:DieVernunftlehrealseineAnweisungzumrichtigenGebrauchderVernunftinderErkenntnisderWahrheit1756}:
\enquote{Alle Menschen denken aber nicht gleich klar und deutlich: oder sie
bestimmen ihre Gedanken nicht so genau\dots}} Es gibt immer
Verständnismöglichkeiten, die wir im Zweifelsfall durch selbständiges Mitdenken
als unvernünftig ausschließen müssen, ohne dass dies durch die Mitteilung
explizit vorgegeben wäre oder dem Mitteilenden anderweitig
vorgeschwebt hätte.\footnote{Man denke an folgendes Beispiel
\name[Ludwig]{Wittgenstein}s: \enquote{Jemand sagt mir: \enquote{Zeige den
Kindern ein Spiel!} Ich lehre sie, um Geld würfeln, und der Andere sagt mir
\enquote{Ich habe nicht so ein Spiel gemeint}. Mußte ihm da, als er mir den
Befehl gab, der Ausschluß des Würfelspiels vorschweben?}
\parencite[][\S~70]{Wittgenstein:PhilosophischeUntersuchungen2003}. Siehe hierzu
auch \cite[][]{Kambartel:VersuchueberdasVerstehen1991}.}



\enquote{Kritik} steht hier für die Frage, ob eine Mitteilung auch tatsächlich
aus der Quelle stammt, die als solche ausgegeben wird; es handelt sich um
Untersuchungen, für die sich inzwischen der Ausdruck \enquote{Quellenkritik}
als \emph{terminus technicus} eingebürgert hat. Wenig verwunderlich ist, dass es sich
um einen Ausdruck handelt, der fast nur in speziellen Kontexten -- etwa in der
Geschichtswissenschaft oder in der Theologie --, nicht aber im Alltag beheimatet
ist. Wenn Jasmin von Peter erfährt, dass es regnet, dann stellt sich die Frage,
ob es tatsächlich Peter ist, der spricht, nicht. Im Alltag und gerade im mündlichen Gespräch können wir den
Absender einer Nachricht in der Regel mühelos identifizieren. Anders verhält es sich bei schriftlichen Mitteilungen und vor
allem historischen Dokumenten und Urkunden, die sich durchaus als Fälschungen
erweisen können.


Eine Form von Quellenkritik, die sich bei \name[Immanuel]{Kant} findet, ist das
Bestreiten der Möglichkeit, Gott als Quelle testimonialen Wissens zu
identifizieren.\footnote{Zu \name[Immanuel]{Kant}s Zurückweisung von
Offenbarung zugunsten einer Vernunftreligion siehe
\cite{Doerflinger:UeberdenaufgeklaertenUmgangmitGottesWort2009}.} Es gebe
keinerlei Möglichkeit zu erkennen, dass Gott der Urheber einer Mitteilung sei,
wohl aber Möglichkeiten, ihn als Urheber auszuschließen.
Das einzige zulässige, aber auch ein hinreichendes Kriterium sei die
Vereinbarkeit der Mitteilung oder Aufforderung mit dem moralischen
Gesetz.\footnote{\cite[Vgl.][A 102]{Kant:DerStreitderFakultaeten1977},
\cite[][VII: 63.9--17]{Kant:GesammelteWerke1900ff.}.} \name[Immanuel]{Kant}s
Beispiel im \titel{Streit der Fakultäten} macht seine Haltung mehr als deutlich:
\begin{quote}
  \phantomsection\label{Beispiel:AbrahamOpfertSeinenSohn}Zum Beispiel kann die
  Mythe von dem Opfer dienen, das Abraham auf göttlichen Befehl durch Abschlachtung und Verbrennung seines einzigen Sohnes -- (das
  arme Kind trug unwissend noch das Holz hinzu) -- bringen wollte. Abraham hätte
  auf diese vermeinte göttliche Stimme antworten müssen: \enquote{Daß ich
  meinen guten Sohn nicht töten solle, ist ganz gewiß; daß aber du, der du mir
  erscheinst, Gott sei, davon bin ich nicht gewiß und kann es auch nicht
  werden}, wenn sie auch vom (sichtbaren) Himmel
  herabschallte.\footnote{\cite[][A 102\,f.,]{Kant:DerStreitderFakultaeten1977}
  \cite[][VII: 63.32--38]{Kant:GesammelteWerke1900ff.}.}
\end{quote}
\name[Immanuel]{Kant}s Verhältnis zu heiligen Schriften ist nicht durch detaillierte
quellenkritische Analysen geprägt, sondern durch dieses eine globale Argument
gegen die Möglichkeit von als solchen identifizierbaren Offenbarungen Gottes:
Gott ist als Quelle einer Mitteilung nicht anders zu identifizieren als über die
moralische Beschaffenheit des Mitgeteilten. \emph{Wenn} es sich um eine
Mitteilung handelt, die moralisch Gebotenes als solches anweist oder doch
wenigstens mit unseren moralischen Einsichten kompatibel ist, \emph{dann} ist es
zumindest \emph{möglich}, dass Gott tatsächlich Urheber der Mitteilung ist.
Handelt es sich hingegen  um eine Information, die gegen unsere moralischen
Einsichten verstößt, dann können wir Gott als Urheber ausschließen.

Auf hermeneutische Überlegungen greift \name[Immanuel]{Kant} zurück, um zu
erläutern, wie aus der Perspektive der Aufklärung mit Schriften umzugehen sei,
deren Urheber nicht Gott ist, sondern ein Mensch, der selbst an eine Inspiration
seines Schreibens durch Gott oder eine direkte Offenbarung glauben mag. Weil der
Inhalt heiliger Schriften mit \name[Immanuel]{Kant}s religionsphilosophischer
Haltung nicht immer harmoniert, muss die Vereinbarkeit auf dem Wege
der Hermeneutik erfolgen. Und deswegen sei es Aufgabe des Lesers, den
\emph{symbolischen} Sinn der Aussagen herauszufinden, statt die Schriften
wörtlich auszulegen:
\begin{quote}
  Daß alle Völker der Erde mit dieser Vertauschung angefangen haben, und daß,
  wenn es darum zu tun ist, was ihre Lehrer selbst, bei Abfassung ihrer heiligen
  Schriften wirklich gedacht haben, man sie alsdann nicht symbolisch, sondern
  \ori{buchstäblich} auslegen müsse, ist nicht zu streiten; weil es unredlich
  gehandelt sein würde, ihre Worte zu verdrehen. Wenn es aber nicht bloß um die
  \ori{Wahrhaftigkeit} des Lehrers, sondern auch, und zwar wesentlich, um die
  \ori{Wahrheit} der Lehre zu tun ist, so kann und soll man diese, als bloße
  symbolische Vorstellungsart, durch eingeführte Förmlichkeit und Gebräuche jene
  praktischen Ideen zu begleiten, auslegen; weil sonst der intellektuelle Sinn,
  der den Endzweck ausmacht, verloren gehen
  würde.\footnote{\cite[][BA~108]{Kant:AnthropologieinpragmatischerHinsicht1977},
  \cite[][VII: 192.7--17]{Kant:GesammelteWerke1900ff.}.}
\end{quote}
Diese Vertauschung aufzuheben bezeichnet \name[Immanuel]{Kant} explizit als
\enquote{Aufklärung}.\footnote{\cite[Siehe][BA
107\,f.,]{Kant:AnthropologieinpragmatischerHinsicht1977} \cite[][VII:
191.36--192.5]{Kant:GesammelteWerke1900ff.}.} \singlequote{Heilige Schriften}
haben einen \singlequote{intellektuellen Sinn}, der nicht den Zweck der
Verfasser dieser Schriften ausmacht, sondern \emph{unseren}
\singlequote{Endzweck}. Damit spielt \name[Immanuel]{Kant} auf ein Thema des
Kapitels \ref{Zitat:EndzweckalsganzeBestimmungdesMenschen} an: Der
\singlequote{Endzweck} war nichts geringeres \enquote{als die ganze Bestimmung
des Menschen}\footnote{\cite[][B 868]{Kant:KritikderreinenVernunft2003},
\cite[][III: 543.11]{Kant:GesammelteWerke1900ff.}; siehe oben Seite
\pageref{Zitat:EndzweckalsganzeBestimmungdesMenschen} in Kapitel
\ref{Zitat:EndzweckalsganzeBestimmungdesMenschen}.}. Aufgeklärte Religion
interpretiert heilige Schriften so, dass genau dieser Inhalt dort hineingelesen
werden kann. Und sie ist darin so frei, gegen jede -- und sei sie noch so
offensichtliche -- Autorintention zu
verstoßen.\footnote{\cite[Vgl.][134]{Doerflinger:UeberdenaufgeklaertenUmgangmitGottesWort2009}.}

Ein solches Vorgehen hat freilich mit dem üblichen Umgang mit
testimonialen Erkenntnissen wenig gemein. Wir sollten sogar sagen, dass es sich
überhaupt nicht um einen Umgang mit testimonialen Erkenntnissen handelt, denn
die Mitteilung ist nicht die Grundlage unserer Erkenntnis. Wir sollen den
\singlequote{heiligen Schriften} ja nicht den von ihren Autoren intendierten
Literalsinn entnehmen und auf Grundlage des Ansehens ihrer Verfasser das
Gesagte für wahr halten. Stattdessen sollen wir von Anfang an davon ausgehen,
dass die Verfasser irren, und das in ihrem literal gemeinten Texten als
symbolisch ausgedrückt erkennen, was wir unabhängig von ihren Texten bereits
wissen. Es gleicht eher den Fällen, in denen uns jemand auf etwas aufmerksam
macht, was wir dann selbst sehen, oder in denen wir einen mathematischen Beweis vorgeführt
bekommen, den wir im Anschluss selbst kontrollieren können. Das Ansehen des
Informanten ist in beiden Fällen irrelevant, weswegen es sich nicht um
\emph{genuin} testimoniales Wissen handelt.\footnote{Siehe oben, Kapitel
\ref{Abschnitte:ZweiThemenSozialerErkenntnistheorie}, ab
S.~\pageref{Abschnitte:ZweiThemenSozialerErkenntnistheorie}.} Und so nützt es
auch nicht als allgemeiner Vorschlag zur Vereinbarung von epistemischer
Autonomie und testimonialem Wissen, weil es sich bei dem Ergebnis der
Bibelauslegung im Sinne \name[Immanuel]{Kant}s einfach nicht um testimoniales
Wissen handelt.

\section{Historische Kenntnis versus
Wissenschaft}\label{subsection:BewertungvonInformationennachihrerART}

Nach \name[Immanuel]{Kant} ist die Maxime einer niemals passiven Vernunft
durchaus mit dem Erwerb von Wissen durch das Zeugnis anderer verträglich, weil
man sich auch bei bei dem Wissenserwerb über andere mündig (aktiv) oder unmündig
(passiv) verhalten kann. Eine erste, sich ganz natürlich einstellende Idee ist
in diesem Zusammenhang, die Mündigkeit in die Bewertung der Bonität des
Informanten zu setzen. Vor dem Hintergrund einer solchen Überlegung wird
mitunter auch \name[Immanuel]{Kant} interpretiert, etwa von
\authorfullcite{Kater:PolitikRechtGeschichte1999} und
\authorfullcite{Scholz:DasZeugnisanderer2001}. Wie in Kapitel
\ref{subsection:BewertungvonInformationenanhandihrerQuellen} zu sehen war,
wird der von \authorcite{Wolff:Psychologiaempirica1968} und \authorcite{Meier:Vernunftlehre1752} aufgestellte kurze Kriterienkatalog, den
\name[Immanuel]{Kant} in seinen Vorlesungen erläutert, von ihm ebenso wenig
negiert wie die beiden zusätzlichen Kriterien, die
\authorcite{Reimarus:DieVernunftlehrealseineAnweisungzumrichtigenGebrauchderVernunftinderErkenntnisderWahrheit1756}
systematisiert und die in \name[Immanuel]{Kant}s Religionsphilosophie und
Offenbarungskritik so bedeutsam werden. Aber mit diesen vier Kriterien ist noch
keine befriedigende Darstellung dessen erreicht, wie \name[Immanuel]{Kant}
Aufklärung mit der Möglichkeit testimonialen Wissens vereinbart. Tatsächlich
sind der grundlegende Punkt noch gar nicht berührt.

\subsection{Vernunftwahrheiten und
Erfahrungstatsachen}\label{subsection:VernunftwahrheitenUndErfahrungstatsachen}
\name[Immanuel]{Kant} sagt es sei egal, ob wir eine Überzeugung auf der
Grundlage auf der Grundlage eigener oder fremder Erfahrung annehmen. Darin kommt
nicht nur zum Ausdruck, dass testimoniales Wissen kein von anderen Wissensarten
abgeleitetes oder gar defizientes Wissen ist. Die Informationen, die andere uns
geben, sind der eigenen Wahrnehmung gleichrangig. Es kommt aber implizit auch
zum Ausdruck, dass die Differenzierung von Erkenntnissen in
solche aus Erfahrung (empirisches Wissen) und solche aus Vernunft
(Vernunftwahrheiten) wichtig ist. Es gibt Erkenntnisse, bei denen wir Andere als
Autoritäten ansehen und ihr Zeugnis als hinreichend annehmen können, und es gibt
Erkenntnisse, bei denen das nicht so ist. In der {\jaeschelogik} lesen wir:
\begin{quote}\phantomsection\label{Zitat:Kant:TestimonialesWissenVernunftErfahrung}
Wenn wir in Dingen, die auf Erfahrungen und Zeugnissen beruhen, unsre
Erkenntnis auf das Ansehen andrer Personen bauen: so machen wir uns dadurch
keiner Vorurteile schuldig; denn in Sachen dieser Art muß, da wir nicht alles
selbst erfahren, und mit unserm eigenen Verstande umfassen können, das Ansehen
der Person die Grundlage unsrer Urteile sein. -- Wenn wir aber das Ansehen
anderer zum Grunde unseres Fürwahrhaltens in Absicht auf Vernunfterkenntnisse
machen: so nehmen wir diese Erkenntnisse auf bloßes Vorurteil
an.\footnote{\cite[][A~120]{Kant:ImmanuelKantsLogik1977}, \cite[][IX:
77.31--78.5]{Kant:GesammelteWerke1900ff.}. Für dieses Zitat gibt es -- soweit
ich sehen kann -- in den Notizen \name[Immanuel]{Kant}s keine direkte Vorlage.
Sein Inhalt wird sich im Verlauf der Überlegungen jedoch bestätigen.}
\end{quote}
Die Gewissheit ist bei eigenen Erfahrungen (Erfahrungswissen) und bei
Mitteilungen (testimoniales Wissen) dieselbe, beides ist fehlbar, aber dennoch
\emph{Wissen}.\footnote{\cite[Vgl.][A~111,
Anm.]{Kant:ImmanuelKantsLogik1977}\protect{,} \cite[][IX:
72.24]{Kant:GesammelteWerke1900ff.}.} Daher genügt es selbst den pedantischsten
Ansprüchen an die Sprache zu sagen, man \emph{wisse}, dass es eine Stadt
namens Rom gibt, dass \name[Immanuel]{Kant} in Königsberg lebte und dergleichen, ohne dies
jemals selbst überprüft zu
haben.\footnote{\cite[Vgl.][A~319]{Kant:Washeisst:SichimDenkenorientieren?1977},
\cite[][VIII:
141.13--17]{Kant:GesammelteWerke1900ff.}. Dagegen spricht jedoch der von
\name[Immanuel]{Kant} vorgeschlagene Sprachgebrauch in
\cite[][A~v]{Kant:MetaphysischeAnfangsgruendederNaturwissenschaften1977},
\cite[][IV:
468.17--19]{Kant:GesammelteWerke1900ff.}:
\enquote{\ori{Eigentliche} Wissenschaft kann nur diejenige genannt werden, deren
Gewißheit apodiktisch ist; Erkenntnis, die bloß empirische Gewißheit enthalten
kann, ist ein nur uneigentlich so genanntes \ori{Wissen}.}} \name[Immanuel]{Kant} unterscheidet
Erkenntnisse dieser Art, von denen testimoniales Wissen zu haben möglich und
legitim ist, von Vernunftwahrheiten, von denen testimoniales Wissen zu haben
unmöglich ist oder denen eine solche Form zumindest unangemessen ist.


\name[Immanuel]{Kant} hat die hier zugrunde liegende Unterscheidung nicht erst
eingeführt. Während der gesamten Neuzeit wurden Unterscheidungen zwischen zwei
Arten von Erkenntnissen artikuliert, die \name[Immanuel]{Kant}s Unterscheidung
von Erfahrungswissen und Vernunftwahrheiten ähneln.
\authorcite{Leibniz:Meditationesdecognitioneveritateetideis1999}' Unterscheidung
von \emph{v{\'e}rit{\'e}s de raison} und \emph{v{\'e}rit{\'e}s de
fait}\footnote{\cite[Vgl.][\S~33]{Leibniz:Lamonadologie2002}.}
sowie \authorcite{Hume:AnEnquiryConcerningHumanUnderstanding1964}s Unterscheidung von
\emph{relations of ideas} und \emph{matters of
fact}\footnote{\cite[Vgl.][20--23]{Hume:AnEnquiryConcerningHumanUnderstanding1964}.}
sind vielleicht die bekanntesten Darstellungen.
Angewandt auf die Frage nach dem Umgang mit testimonialem Wissen ergibt sich
jedoch eine nicht ganz so einfach sichtbare Verbindung zu
\authorcite{Descartes:OeuvresdeDescartes1983}, wenn dieser die
Büchergelehrsamkeit als Ausdruck bloß \enquote{historischer Kenntnisse}
kritisiert. \authorcite{Descartes:OeuvresdeDescartes1983} kritisiert, dass wir
keine Wissenschaft, sondern lediglich historische Kenntnisse erwürben, wenn wir
Wissen aus Büchern erwerben.\footnote{Siehe oben Kapitel
\ref{Abschnitt:DescartesundhistorischeKenntnisse},
S.~\pageref{Abschnitt:DescartesundhistorischeKenntnisse}.} Die sich
anschließende Frage ist, was den Unterschied zwischen Wissenschaft
und historischen Kenntnissen ausmacht. Diesen -- so werde ich argumentieren --
verstehen wir nur vor dem Hintergrund der neuzeitlichen Unterscheidung zweier
Erkenntnisarten und ihrer Relevanz für Fragen der Sozialen
Erkenntnistheorie.

Einen guten Ansatz zum Verständnis des Zusammenhangs der Unterscheidung beider
Erkenntnisarten mit der Frage nach der Legitimität des Fürwahrhaltens aufgrund
von Mitteilungen finden wir bei Thomas \name[Thomas]{Hobbes}, der schreibt:
\begin{quote}
\Revision[Selbst]{Es gibt zwei Arten von \emph{Wissen}
(\enquote{\emph{knowledge}}):
Das eine ist \emph{Tatsachenwissen} (\enquote{\emph{knowledge of fact}}), das andere ist das
\emph{Wissen vom Schließen von einer Behauptung auf eine andere}. Das erste ist
nichts anderes als sinnliche Wahrnehmung und Erinnerung und ist \emph{absolutes
Wissen} (\enquote{\emph{absolute knowledge}}); wie wenn wir sehen, dass etwas getan
wird, oder uns erinnern, dass es getan wurde. Und dieses Wissen verlangen wir von einem Zeugen.
Das andere Wissen heißt \emph{Wissenschaft} (\enquote{\emph{science}}) und ist
\emph{konditional} (\enquote{\emph{conditional}}); so wissen wir, dass,
\emph{wenn die gezeigte Figur ein Kreis ist, dass ihn dann jede gerade Linie
durch seinen Mittelpunkt in zwei gleich große Teile teilt}. Und dieses Wissen
erwarten wir von einem Gelehrten (\enquote{\emph{philosopher}}), also von
jemandem, der sich auf den Gebrauch der Vernunft
beruft.}\footnote{\Revision[Selbst]{\enquote{\ori{There} are of \ori{knowledge}
two kinds; whereof one is \ori{knowledge of fact:} the other \ori{knowledge of the consequence of one affirmation to another}. The former is nothing else, but sense and memory, and is \ori{absolute knowledge;} es when we see a fact doing, or remember it done:
an this is knowledge required in a witness. The latter is called \ori{science;}
and is \ori{conditional;} as when we know, that, \ori{if the figure shown be a
circle, then any straight line through the centre shall divide it into two equal
parts.} And this is the knowledge required in a philosopher; that is to say, of
him that pretends to reasoning} \parencite[][Chapter IX,
S.~71]{Hobbes:Leviathan1962}.}}
\end{quote}\enlargethispage{\baselineskip}
Was wir von einem Zeugen verlangen, ist \emph{absolutes} im Gegensatz zu
\emph{konditionalem} Wissen. Dabei liegt der entscheidende Unterschied zwischen
beiden Arten nicht etwa darin, dass sich der Zeuge bei absolutem Wissen
besonders sicher sein müsste -- \enquote{absolut} hat nicht das Geringste mit
\enquote{unfehlbar} zu tun, genauso wenig wir \enquote{konditional} mit
\enquote{fehlbar} oder \enquote{vorläufig}\footnote{Ich vermeide aus naheliegenden Gründen die
Übersetzungsmöglichkeiten \enquote{hypothetisch} oder \enquote{bedingt} für
\enquote{conditional} und bleibe bei der etwas sperrig klingenden Eindeutschung
\enquote{konditional}.} --, sondern in der inneren Struktur
der Erkenntnis selbst. Konditionales Wissen ist Wissen um den
\emph{Zusammenhang} von (Behauptungen über) Tatsachen, während absolutes Wissen
einfach einzelne Tatsachen beschreibt, die (kontingenter Weise) geschehen
sind.\footnote{Logisch scheint diese Unterscheidung schwierig zu artikulieren zu
sein, insofern sich jede \distanz{absolute} Aussage $p$ ganz einfach in die
\distanz{konditionale} Aussage $ \top \supset p$ umwandeln ließe, wenn $ \top $
hier für eine beliebige Tautologie steht. Umgekehrt kann jedes Konditional $p
\supset q$ auch nicht-konditional in der Form $\neg (p \land \neg q)$ dargestellt
werden.} So zählt als absolutes Wissen, dass 1755 vor Lissabon ein Seebeben
stattfand, dass Rom die Hauptstadt Italiens ist und 753 v.\,u.\,Z.\ gegründet wurde oder dass
\name[Immanuel]{Kant} in Königsberg lehrte. Überwiegend konditional ist hingegen
das Wissen der Mathematik oder der Newtonschen Physik. Es klärt uns nicht über
einzelne Tatsachen auf, sondern über mathematische oder physikalische
Zusammenhänge, die zwischen einzelnen Tatsachen bestehen.
Der Satz des \singlename{Pythagoras} sagt uns: \emph{Wenn} in einem
rechtwinkligen Dreieck die Katheten die Längen $a$ und $b$ (z.\,B.\ $\unit[3]{cm}$ und
$\unit[4]{cm}$) haben, \emph{dann} hat die Hypotenuse die Länge $c = \sqrt{a^2 +
b^2}$ ($\unit[5]{cm}$). Konditionales, nicht
absolutes Wissen zählt dabei im eigentlichen Sinne als
Wissenschaft\footnote{\enquote{\ori{Science} is the knowledge of consequences,
and dependence of one fact upon another}
\parencite[][35]{Hobbes:Leviathan1962}.} oder (dem Sprachgebrauch der deutschen
Aufklärung näher) Philosophie, welche dadurch der Geschichte
(\enquote{\emph{history}}) oder dem Tatsachenwissen entgegengesetzt
wird.\footnote{Tatsachenwissen sei --
wenn es aufgezeichnet wurde (\enquote{[t]he register of \ori{knowledge of
fact}}) -- \emph{Geschichte}, und zwar zum einen \emph{Naturgeschichte}, zum
anderen \emph{Kultur-} oder \emph{Menschheitsgeschichte} (\enquote{\ori{civil history;}
which is the history of the voluntary actions of men in commonwealths})
\parencite[vgl.][71]{Hobbes:Leviathan1962}.}

Interessant für meine Fragestellung ist \authorcite{Hobbes:Leviathan1962}, weil
er die beiden Erkenntnisarten unterscheidet als Wissen, welches wir von Zeugen
verlangen, auf der einen Seite und Wissen, welches ein Gelehrter haben soll, auf
der anderen Seite. Testimoniales Wissen erwerben wir in der Regeln bei
\singlequote{absolutem} Wissen oder Tatsachenwissen. Der Zeuge soll nicht allgemeine
Wahrheiten über Zusammenhänge explizieren, sonder verraten, was geschehen
ist. Es soll also für gewöhnlich kein konditionales Wissen oder
wissenschaftliche Erkenntnisse vortragen. Absolutes und
konditionales Wissen entsprechen weitgehend dem, was wir von einem \emph{Zeugen}
auf der einen und einem \emph{Sachverständigen} oder \singlequote{Experten} auf
der anderen Seite erwarten. Und dieser Unterschied ist relevant für die Frage
nach einem verantwortlichen Umgang mit testimonialem Wissen.

An \authorcite{Hobbes:Leviathan1962}' Darstellung sei ein weiterer Punkt
hervorgehoben, der auch bei
\authorcite{Wolff:Discursuspraeliminarisdephilosophiaingenere1996} und
\name[Immanuel]{Kant} wichtig sein wird: Das \enquote{Wissen vom Schließen von
einer Behauptung auf eine andere} (\enquote{knowledge of the consequence of one
affirmation to another}) bezeichnet ein \emph{know how}, kein \emph{know that},
weswegen die direkte Übertragung in die deutsche Sprache entsprechend sperrig
klingt. Es ist kein Wissen im eigentlichen Sinn dieses Wortes, sondern ein
\emph{Können} oder eine Kompetenz. Wer dieses \singlequote{Wissen} im Sinne von
\emph{know how} besitzt, kann nicht bloß diesen wahren Satz äußern, dass eine
gerade Linie durch den Mittelpunkt eines Kreises diesen in gleich große Teile
teilt. Er kann diesen Satz in verschiedenen Situationen \emph{anwenden}. Bei Tatsachenwissen mögen
ebenfalls Kompetenzen des Informanten vorausgesetzt sein (siehe Kapitel
\ref{subsubsection:GeorgFriedrichMeier}), aber das Tatsachenwissen selbst
\emph{besteht} doch nicht in einer solchen Kompetenz. Von einem Gelehrten
hingegen erwarten wir eine entsprechende Kompetenz; wir erwarten von einem
Mathematiker, dass er Beweise \emph{führen}, und nicht, dass er sie auswendig
aufsagen kann.

Ich hatte schon in Kapitel \ref{subsection:SelbstdenkenbeiKant} herausgestellt,
dass die Frage nach der Kompetenz für die Aufklärung bei
\authorcite{Wolff:Discursuspraeliminarisdephilosophiaingenere1996} und
\name[Immanuel]{Kant} entscheidend sein werde. Gerade deswegen schließen beide
an diesen Gedanken einer Unterscheidung von Wissensarten an, den wir in einer
sehr einfachen, aber klaren Ausgestaltung bei \authorcite{Hobbes:Leviathan1962}
finden. \name[Immanuel]{Kant}s direkte Vorlage für die Ausarbeitung seiner
eigenen Überlegungen ist
\authorcite{Wolff:Discursuspraeliminarisdephilosophiaingenere1996} Konzeption
dieser
Unterscheidung\footcite[vgl.][\pno~12\,f.]{Albrecht:KantsKritikderhistorischenErkenntnis--einBekenntniszuWolff?1982},
die nun dargestellt werden soll.

\subsection{Philosophische und historische Erkenntnis bei
Christian Wolff}\label{paragraph:wolffswarnung}
\authorfullcite{Wolff:Psychologiaempirica1968} unterscheidet in seinem
\titel{discursus praeliminaris de philosophia in genere} drei Arten der
Erkenntnis: (1) die historische Erkenntnis (\enquote{\emph{cognitio historica}},
\S~3) als bloße unzusammenhängende Kenntnis dessen, was ist oder geschieht
(\enquote{\emph{nuda facti notitia}}, \S~7), (2) die philosophische Erkenntnis
(\enquote{\emph{cognitio philosophica}}, \S~6) als eine \emph{rationale}, Gründe
und Ursachen erforschende (\enquote{\emph{in} [\emph{cognitione}; A.\,G.]
\emph{philosophica reddimus rationem}}, \S~17) Erkenntnis dessen, was ist oder
geschieht, und (3) die mathematische Erkenntnis (\enquote{\emph{cognitio
mathematica}}, \S~14) als Erkenntnis von
Quantitäten.\footnote{\cite[Vgl.][Cap.I.
De triplici cognitione
humana]{Wolff:Discursuspraeliminarisdephilosophiaingenere1996}. Siehe auch die
weniger ausführlichen Bemerkungen in
\cite[][Vorbericht]{Wolff:VernuenftigeGedankenvondenKraeftendesmenschlichenVerstandesundihremrichtigenGebraucheinErkenntnisderWahrheit1978},
sowie
\cite[][Praefatio]{Wolff:Cogitationesrationalesdeviribusintellectushumani1983}.}
Da er die historische Erkenntnis des weiteren in die gemeine (\emph{communis})
und die verborgene (\emph{arcana}) historische Erkenntnis unterteilt, ergibt
sich die in Abbildung \ref{abbildung:ZeichnungErkenntnisartennachWolff.pdf}
dargestellte Einteilung.

\begin{figure}[htb]
\begin{minipage}[t]{\textwidth}
\centering
\begin{tikzpicture}[edge from parent fork down,
level 1/.style = {level distance=1.5cm, sibling distance=3.5cm},
level 2/.style = {level distance=1.5cm, sibling distance=2.5cm},
every node/.style={rectangle,draw=black,fill=gray!25, thin, inner sep=0.5em, minimum size=0.5em, align=center},
edge from parent/.style={thin,draw},
mylabel/.style={draw=none, fill=none, text width=5cm,text centered, inner sep=0.5em, anchor=base} ]
\node {\emph{cognitio}}
	child {node {\emph{historica}}
		child {node {\emph{communis}}}
		child {node {\emph{arcana}}}}
	child {node {\emph{philosophica}}}
	child {node {\emph{mathematica}}}
;
\end{tikzpicture}
  \caption{Einteilung der Erkenntnisarten nach
  \authorfullcite{Wolff:Psychologiaempirica1968}}\label{abbildung:ZeichnungErkenntnisartennachWolff.pdf}
\end{minipage}
\end{figure}

\authorcite{Wolff:Psychologiaempirica1968}s Darstellung unterscheidet sich von
Hobbes' Systematik am augenscheinlichsten durch die weitere Kategorie der
mathematischen Erkenntnis, die \name[Immanuel]{Kant} im Anschluss an
\authorcite{Wolff:Discursuspraeliminarisdephilosophiaingenere1996} mit der
philosophischen zur rationalen Erkenntnis zusammenfassen
wird.\footnote{\cite[Vgl.][B 863--865]{Kant:KritikderreinenVernunft2003},
\cite[][III: 540.32--33, 541.18--20]{Kant:GesammelteWerke1900ff.}. Siehe dazu
weiter unten Kapitel \ref{section:MuendigkeitundPhilosophie}.} Ich stelle die
mathematische Erkenntnis der Einfachheit und Übersichtlichkeit halber zunächst
zurück und behandle sie als Sonderfall in der Auseinandersetzung mit
\name[Immanuel]{Kant}s Konzeption.\footnote{Siehe Kapitel
\ref{subsubsection:EndlichesundUnendlichesErkennen}.} Hier expliziere ich
zunächst den Unterschied zwischen historischer und philosophischer
Erkenntnis (Kapitel \ref{paragraph:wolffswarnung}), um dann zu
erörtern, wie sich mit \authorcite{Wolff:Discursuspraeliminarisdephilosophiaingenere1996} die
Behauptung \authorcite{Descartes:OeuvresdeDescartes1983}' konkretisieren lässt,
durch das Lesen von Büchern erwerbe man nur historische Kenntnisse statt echter
Wissenschaft (Kapitel \ref{subsubsection:HistorischeKenntnisDerPhilosophie}). Im
letzten Abschnitt gehe ich einem Einwand gegen meine
\authorcite{Wolff:Discursuspraeliminarisdephilosophiaingenere1996}interpretation
nach und zeige auf, dass es sich bei der Differenzierung von historischen und
philosophischen Erkenntnissen nicht, wie etwa
\authorfullcite{Kambartel:ErfahrungundStruktur1968} behauptet, um einen
Unterschied zwischen singulären und allgemeinen Urteilen handelt
(Kapitel \ref{subsubsection:HistorischeErkenntnisallgemeinerTatsachen}).

Die drei Erkenntnisarten werden im ersten Kapitel des \titel{Discursus
praeliminaris de philosophia in genere} jeweils so vorgestellt, dass zunächst
die Grundlage der jeweiligen Erkenntnisart beschrieben (\S 1:
\enquote{\emph{Fundamentum cognitionis historicae}}; \S 4:
\enquote{\emph{Fundamentum cognitionis philosophicae}}; \S 13: \enquote{\emph{Fundamentum
cognitionis mathematicae}}), danach die Definition derselben angeben (\S 3:
\enquote{\emph{cognitio historica}}; \S 6: \enquote{\emph{cognitio philosophica}}; \S 14:
\enquote{\emph{cognitio mathematica}}) und schließlich ihr Verhältnis zu den zuvor
besprochenen Erkenntnisarten diskutiert wird. Bezüglich der historischen und philosophischen
Erkenntnis wird darüber hinaus zwischen der Beschreibung ihrer Grundlage und der
Angabe ihrer Definition zusätzlich die Frage nach ihrer Reichweite angesprochen,
aber nicht abschließend beantwortet (\S 2: \enquote{\emph{Cur ejus [i.\,e.~cognitionis
historicae; A.\,G.] hic limites non constituantur}}; \S 5: \enquote{\emph{Cur cognitionis
philosophicae non constituantur limites}}). Interessant und aussagekräftig sind
jeweils Grundlage und Definition, die nur zusammen ein Verständnis der
jeweiligen Erkenntnisart ermöglichen. Im Falle der historischen Erkenntnissen
lesen sich diese wie folgt:
\begin{quote}
  \ori{Wir erkennen mit Hilfe der Sinne die Dinge, die in der materiellen
  Welt sind und geschehen, und der Geist ist sich der Veränderungen bewußt, die
  in ihm selbst
  geschehen}.\footnote{\Cite[][\S~1]{Wolff:Discursuspraeliminarisdephilosophiaingenere1996}:
  \enquote{\ori{Sensuum beneficio cognoscimus, quae in mundo materiali sunt atque
  fiunt, {\&} mens sibi conscia est mutationum, quae in ipsa accidunt}.}}\\
  Die \ori{Erkenntnis} der Dinge, die sind und geschehen, sei es dass sie in der
  materiellen Welt, sei es dass sie in den immateriellen Substanzen, wird von
  uns \ori{historisch}
  genannt.\footnote{\Cite[][\S~3]{Wolff:Discursuspraeliminarisdephilosophiaingenere1996}:
  \enquote{\ori{Cognitio} eorum, quae sunt atque fiunt, sive in mundo materiali,
  sive in substantiis immaterialibus accidant, \ori{historica} a nobis appellatur.}}
\end{quote}
Die historische Erkenntnis ist also die Erkenntnis dessen, was ist oder
geschieht, und ihre Grundlage ist die sinnliche Wahrnehmung. Als Beispiele für
historische Erkenntnisse nennt
\authorcite{Wolff:Discursuspraeliminarisdephilosophiaingenere1996}, dass jemand
aus Erfahrung wisse, dass die Sonne morgens auf- und abends untergehe und
dass Tiere sich durch Zeugung fortpflanzen.

Noch 1709 in den \titel{Vernünftige[n] Gedanken von den Kräften des menschlichen
Verstandes und ihrem richtigen Gebrauche in Erkenntnis der Wahrheit}
identifiziert \authorcite{Wolff:Psychologiaempirica1968} die historische
Erkenntnis mit der \emph{cognitio
communis}.\footnote{\cite[Vgl.][\S~6]{Wolff:VernuenftigeGedankenvondenKraeftendesmenschlichenVerstandesundihremrichtigenGebraucheinErkenntnisderWahrheit1978}:
\enquote{Hierdurch wird die gemeine Erkäntniß von der Erkäntniß eines
Welt-Weisen unterschieden.} Dass er die gemeine Erkenntnis mit der historischen
identifiziert wird einerseits dadurch deutlich, dass ihr Begriff als
Gegenbegriff zur Erkenntnis eines Weltweisen verwendet wird, andererseits
erhellt es aus der 1740 publizierten Übersetzung der \titel{Deutsche[n] Logik}
in die lateinische Sprache; \cite[vgl.][\S~6]{Wolff:Cogitationesrationalesdeviribusintellectushumani1983}:
\enquote{Atque hoc ipso \ori{cognitio philosophica a communi} (quam {\&} \ori{historicam} dicimus) differt.}}
Im \titel{Discursus praeliminaris} bildet die \emph{cognitio communis} eine Untergruppe
der historischen Erkenntnis. Diese -- die \emph{cognitio historica} --
zerfalle in die \emph{cognitio (historica) communis} und die
\emph{cognitio (historica) arcana}, die \singlequote{verborgene} Erkenntnis:
\begin{quote}
  Die \ori{gemeine historische Erkenntnis} [(\emph{cognitio historica
  communis})] ist die Erkenntnis der Tatsachen der Natur (auch der rationalen), die offenbar sind, die \ori{verborgene
  historische Erkenntnis} [(\emph{cognitio historica arcana})] aber ist die
  Erkenntnis der Tatsachen der Natur (auch der rationalen), die verborgen
  sind.\footnote{\Cite[][\S~21]{Wolff:Discursuspraeliminarisdephilosophiaingenere1996}:
  \enquote{\ori{Cognitio historica communis} est cognitio factorum naturae,
  etiam rationalis, quae patent: \ori{Cognitio} autem \ori{historica arcana
  est cognitio factorum naturae, etiam rationalis, quae latent.}}}
\end{quote}
Den Herausgebern \authorfullcite{Gawlick:Einleitung1996} folgend bezeichne ich
die \emph{cognitio communis} als \enquote{gemeine (historische) Erkenntnis} und die
\emph{cognitio vulgaris} als \enquote{gewöhnliche Erkenntnis}. Gemeine
historische Erkenntnis liegt dort vor, wo wir lediglich hinschauen (oder
hinhören, fühlen etc.) müssen, um zu erkennen, was ist oder geschieht. So fühlen
wir etwa, dass die Sonne uns wärmt, ohne erst einen Schluss auf diese Tatsache
vollziehen zu müssen. Bei der gemeinen historischen Erkenntnis handelt es sich
also um \emph{nicht-inferentielles Wissen von Tatsachen} auf der Grundlage von
Erfahrung. Dagegen liegt verborgene historische Erkenntnis dort vor, wo wir
ausgehend von einer Erkenntnis, die uns direkt zugänglich ist, auf eine Tatsache
erst schließen müssen. Eine verborgene historische Erkenntnis besteht
beispielsweise in dem Wissen, dass weißes Licht polychromatisch ist. Das
bekannte Experiment mit einem Prisma zeigt, dass weißes Licht in seine
Spektralfarben zerlegt werden kann und aus einer Mischung von monochromatischem
Licht verschiedener Farben besteht.\footnote{So lautet das Beispiel in
\cite[][\S~20]{Wolff:Discursuspraeliminarisdephilosophiaingenere1996}.} Auf
diese Tatsache müssen wir ausgehend von unserer Beobachtung bei dem Experiment
jedoch erst \emph{schließen}, wir sehen dies nicht einfach. Es handelt sich um
\emph{inferentielles Erfahrungswissen von Tatsachen}, also Wissen, das
inferentiell ist, aber letztlich auf (nicht-inferentiellen) Wahrnehmungsurteilen
aufbaut und von Tatsachen
handelt.\footnote{\phantomsection\label{Anmerkung:CognitioArcanaCognitioCommunis}
Mit Hilfe von Experimenten und dem Einsatz technischer Geräte könne jedoch die \emph{cognitio arcana} in eine \emph{cognitio communis} verwandelt werden \parencite[vgl.][\S~24]{Wolff:Discursuspraeliminarisdephilosophiaingenere1996}.
Im Alltag ist -- um
\authorcite{Wolff:Discursuspraeliminarisdephilosophiaingenere1996}s Beispiel
aufzugreifen -- die Elastizität der Luft nicht zu merken, das Wissen darum ist
verborgen und müsste aus anderen Erkenntnissen geschlossen werden.
Nimmt man jedoch eine Luftpumpe und verschließt die Ventile, dann merkt man beim Hineindrücken
des Kolbens sehr leicht, dass die enthaltene Luft komprimiert wird und sich
wieder ausdehnt, sobald wir die Kraft von außen auf den Kolben reduzieren. Ein
neueres Beispiel für die Transformation inferentiellen in nicht-inferentielles
Wissen ist die Nebelkammer, mit deren Hilfe sich Teilchen, die sonst nicht
sichtbar sind, anhand einer \singlequote{Nebelspur} sichtbar werden. Es ist
nicht leicht zu entscheiden, wie überzeugend diese Auffassung
\authorcite{Wolff:Discursuspraeliminarisdephilosophiaingenere1996}s ist. Ob es
sich dabei dann tatsächlich um nicht-inferentielles Wissen (\emph{cognitio
communis}), dass ein entsprechendes Teilchen die Nebelkammer passiert, oder ob
es sich noch immer um inferentielles Wissen (\emph{cognitio arcana}) handelt,
ist ein Streitpunkt in der Diskussion zwischen
\authorfullcite{McDowell:MindandWorld1994} und
\authorfullcite{Brandom:MakingItExplicit1994}, wobei
\authorcite{Brandom:MakingItExplicit1994} die Position
\authorcite{Wolff:Discursuspraeliminarisdephilosophiaingenere1996}s einnimmt
\parencite[vgl.][141]{McDowell:BrandomonObservation2010}.}


Der Verweis auf rationale Erkenntnis ist an dieser Stelle nicht
erkenntnistheoretisch zu verstehen, sondern verweist darauf, dass es sich auch
hier (wie allgemein bei historischen Erkenntnissen) sowohl um Erkenntnis dessen
handelt, was in der materiellen Welt geschieht, als auch um
Erkenntnis mentaler Vorgänge; letztere -- es ist nicht ganz klar, ob die mentalen Vorgänge oder die
Erkenntnis von ihnen -- nennt er
\singlequote{rational}.\footnote{\enquote{Addo vocem \ori{rationalis}, ut intelligitur, ad facta naturae hic
quoque referri ea, quae in substantiis immaterialibus finitis, veluti mentibus
nostris, accidunt, {\&} cognitionis historicae non minus objectum sunt, quam
quae in mundo materiali contingunt (\S~3)}
\parencite[][\S~21]{Wolff:Discursuspraeliminarisdephilosophiaingenere1996}.}
Auch diese Vorgänge können sowohl offensichtlich als auch verborgen sein.


Historische Erkenntnis ist (inferentielles oder nicht-inferentielles)
Erfahrungswissen davon, was ist oder geschieht. Philosophische Erkenntnis geht
über dieses Erfahrungswissen hinaus, insofern sie nach den \emph{Gründen} dessen
fragt, was ist oder geschieht. Während das einfache Volk sich mit dem
historischen Wissen zufrieden gebe, dass Wasser siedet, wenn es über eine offene
Flamme gesetzt wird, frage der Philosoph (oder derjenige, der nach
philosophischer Erkenntnis strebt), warum dies
geschieht.\footnote{\cite[Vgl.][\S~23]{Wolff:Discursuspraeliminarisdephilosophiaingenere1996}.}
Ich zitiere \authorcite{Wolff:Discursuspraeliminarisdephilosophiaingenere1996}
wiederum mit der Grundlage und der Definition dieser Erkenntnisart:
\begin{quote}
  \ori{Die Dinge, die sind oder geschehen, ermangeln nicht ihres Grundes, aus
  dem erkannt wird, warum sie sind oder
  geschehen.}\footnote{\Cite[][\S~4]{Wolff:Discursuspraeliminarisdephilosophiaingenere1996}:
  \enquote{\ori{Ea, quae sunt vel fiunt, sua non destituuntur ratione, unde
  intelligitur, cur sint, vel fiant.}}}\\
  Die \ori{Erkenntnis} des Grundes der Dinge, die sind oder geschehen, werden
  \ori{philosophisch}
  genannt.\footnote{\Cite[][\S~6]{Wolff:Discursuspraeliminarisdephilosophiaingenere1996}:
  \enquote{\ori{Cognitio} rationis eorum, quae sunt, vel fiunt,
  \ori{philosophica} dicitur.}}
\end{quote}
Philosophische Erkenntnis ist die Erkenntnis des Grundes, aus dem zu erkennen
ist, warum etwas ist oder geschieht, und ihre Grundlage wird durch den Satz des
zureichenden Grundes beschrieben. Weil alles, was ist, einen zureichenden Grund
hat, warum es ist oder geschieht, lässt sich auch jederzeit nach diesem Grund
fragen. Zu jeder historischen Erkenntnis lässt sich somit -- sollte der Satz vom
zureichenden Grunde allgemein gelten\footnote{Allerdings sei es in unserem
Zusammenhang nicht nötig, die streng allgemeine Geltung des Satzes vom
zureichenden Grund vorauszusetzen und zu beweisen
\parencite[vgl.][\S~5]{Wolff:Discursuspraeliminarisdephilosophiaingenere1996}.}
-- eine entsprechende philosophische Erkenntnis finden, die den Grund dessen
enthält, wovon die historische Erkenntnis handelt. Wer etwa weiß, dass ein Stück
Holz auf Wasser schwimmt, während ein Stück Eisen untergeht, hat historische Erkenntnis
dieser Tatsachen. Wer darüber hinaus weiß, dass Holz auf Wasser schwimmt, weil
seine Dichte geringer als die des Wassers ist, während Eisen untergeht, weil
seine Dichte wiederum größer als die des Wassers ist, hat philosophische
Erkenntnis. Zumindest gilt dies \emph{prima facie} unter einer gleich zu
besprechenden Einschränkung (siehe Abschnitt
\ref{subsubsection:HistorischeKenntnisDerPhilosophie} ab
S.~\pageref{subsubsection:HistorischeKenntnisDerPhilosophie}).

Wer über philosophisches Wissen verfügt, ist demjenigen, der lediglich
historische Kenntnisse besitzt, bei der Anwendung seines Wissens überlegen.
Nehmen wir an, Ingrid und Max wissen beide, dass Holz auf Wasser schwimmt und dass Steine
untergehen. Sie verfügen also beide über historisches Wissen bezüglich dieser
beiden Tatsachen. Im Gegensatz zu Max weiß Ingrid aber auch, \emph{warum} dies
so ist -- sie weiß, dass Holz schwimmt, weil seine Dichte geringer ist als die
Dichte des Wassers, und dass Steine untergehen, weil ihre Dichte größer ist als
die des Wassers. Nun sagt
\authorcite{Wolff:Discursuspraeliminarisdephilosophiaingenere1996}, dass die
\emph{Anwendung} der philosophischen Erkenntnis sicherer sei als die der
historischen Kenntnis. (Er sagt explizit nicht, dass das philosophische
\emph{Wissen} sicherer sei, sondern seine \emph{Anwendung}.) Natürlich können
beide -- Ingrid und Max -- ihr Wissen anwenden, indem sie ein Floß aus Holz
bauen und ihre Sachen am Strand mit Steinen beschweren, damit sie bei Flut und
Wellen nicht davongetrieben werden können. Aber Ingrid weiß auch, dass dies
unter der Bedingung gilt, dass Holz eine geringere und Steine eine größere
Dichte haben als Wasser. Sie wird daher ihr Floß nicht aus Azob{\'e} bauen und
ihre Sachen nicht mit Bimsstein
beschweren.\footcite[Vgl.][\S~41]{Wolff:Discursuspraeliminarisdephilosophiaingenere1996}
Außerdem sei die philosophische Erkenntnis in der Anwendung breiter als die
historische: Sollten keine Steine zur Hand sein, wird Ingrid wissen, dass auch
jeder andere hinreichend schwere Gegenstand die Aufgabe der Steine übernehmen
kann. Und ihr wird bewusst sein, dass nicht nur Holz zur Konstruktion eines
Floßes taugt, sondern auch leere Fässer oder Materialien mit geringer
Dichte.\footcite[Vgl.][\S~42]{Wolff:Discursuspraeliminarisdephilosophiaingenere1996}
In der Regel erklärt philosophische Erkenntnis die Tatsachen, von denen wir
historische Erkenntnis haben, indem sie sie unter einen allgemeineren Begriff
subsumiert und eine allgemeinere Tatsache als Erklärung anführt. Durch diese
größere Allgemeinheit benötigen wir weniger Erkenntnisse, um über mehr Fälle
Bescheid zu
wissen.\footcite[Vgl.][\S~43]{Wolff:Discursuspraeliminarisdephilosophiaingenere1996}
Es hat also ganz praktische Vorteile, von der historischen Erkenntnis zur
philosophischen fortzuschreiten.


Während das erste Kapitel des \titel{Discursus praeliminaris de philosophia in
genere} die drei Erkenntnisarten \emph{cognitio historica}, \emph{cognitio
philosophica} und \emph{cognitio mathematica} unterscheidet und in Relation
zueinander setzt, exponiert das zweite Kapitel den Begriff
\enquote{\emph{philosophia}}. Die Übersetzung mit dem deutschen
\enquote{Philosophie} wird in vielen Situationen unpassend klingen, denn
\authorcite{Wolff:Discursuspraeliminarisdephilosophiaingenere1996}s
Philosophiebegriff unterscheidet sich durch seine Weite grundlegend von unserer
heutigen Verwendung. Es wäre zu überlegen, andere Ausdrücke -- etwa
\enquote{Gelehrter} für \enquote{\emph{philosophus}} und \enquote{Weltweisheit}
für \enquote{philosophia} -- zu verwenden, was jedoch terminologische
Folgeprobleme mit sich brächte. Ich bleibe daher bei Ausdrücken
\enquote{Philosophie} und \enquote{Philosoph}, die ich mitunter in einfache
Anführungszeichen setze, wenn betont werden soll, dass es sich um \authorcite{Wolff:Discursuspraeliminarisdephilosophiaingenere1996}s
Terminologie handelt, und eine Verwechslung erhebliche Missverständnisse zur
Folge hätte.


Philosophie sollte dem Wort nach sicherlich diejenige Wissenschaft sein, die
philosophische Erkenntnis zu ihrem Inhalt hat. Und als Philosophen
bezeichnen wir denjenigen, der über philosophisches Wissen verfügt oder nach ihm
strebt. Doch \authorcite{Wolff:Discursuspraeliminarisdephilosophiaingenere1996}s
Definition von \enquote{\emph{philosophia}} ist etwas komplexer aufgebaut, als
zu erwarten wäre. Ich zitiere die Definition zusammen mit den beiden
anschließenden Paragraphen, die für ein Verständnis der Definition nötig sind:
\begin{quote}
  {[\S~29:]} \ori{Philosophie} ist die Wissenschaft der möglichen
  Dinge, wie und warum sie möglich
  sind.\footnote{\label{Anmerkung:UebersetzungDefinitioPhilosophieWolffDiscursus}\Cite[][\S~29]{Wolff:Discursuspraeliminarisdephilosophiaingenere1996}:
  \enquote{\ori{Philosophia} est scientia possibilium, quatenus esse
  possunt.} \authorcite{Gawlick:Einleitung1996} übersetzen stattdessen:
  \enquote{\ori{Philosophie} ist die Wissenschaft des Möglichen, insofern es
  sein kann.} Sie bleiben also näher am Wortlaut. Ich habe die Übersetzung
  gewählt, die bereits
  \authorcite{Stiebritz:ErlaeuterungenderVernuenftigenGedanckenvondenKraefftendesmenschlichenVerstandesWolffs1977}
  als solche angibt
  \parencite[vgl.][\S~29]{Stiebritz:ErlaeuterungenderVernuenftigenGedanckenvondenKraefftendesmenschlichenVerstandesWolffs1977},
  weil sie zum einen den Zusammenhang von philosophischer Erkenntnis und
  Philosophie transparenter macht und zum anderen auch
  \authorcite{Wolff:Discursuspraeliminarisdephilosophiaingenere1996} selbst
  vorzuschweben scheint \parencite[vgl.][Vorbericht, \S~1]{Wolff:VernuenftigeGedankenvondenKraeftendesmenschlichenVerstandesundihremrichtigenGebraucheinErkenntnisderWahrheit1978}:
  \enquote{Die Welt-Weisheit ist eine Wissenschaft aller möglichen Dinge, wie
  und warum sie möglich sind.} Der Text des \titel{Discursus praeliminaris} ist
  wörtlich übernommen in die Übersetzen dieses Zitats; \cite[vgl.][Prolegomena,
  \S~1]{Wolff:Cogitationesrationalesdeviribusintellectushumani1983}:
  \enquote{\ori{Philosophia} est scientia possibilium, quatenus esse
  possunt.} Die erste Erwähnung findet diese Definition der Philosophie bereits
  1709 in:
  \cite[][Praefatio]{Wolff:AerometriaeelementainquibusaliquotAerisviresacproprietatesjuxtamethodumGeometrarumdemonstrantur1981}:
  \enquote{Philosophiam ego definire soleo per rerum possibilium, qua talium,
  scientiam.} Es handelt sich nach allgemeiner Auffassung um
  \authorcite{Wolff:Discursuspraeliminarisdephilosophiaingenere1996}s eigene
  Neuschöpfung, die er 1705 fand und nach einer in Briefen geführten Diskussion
  mit Caspar \name[Caspar]{Neumann} 1709 in den \titel{A{\"e}rometriae elementa} erstmal
  öffentlich kommunizierte
  \mkbibparens{\cite[vgl.][\S~29]{Wolff:Discursuspraeliminarisdephilosophiaingenere1996},
  \cite[][\S~29]{Stiebritz:ErlaeuterungenderVernuenftigenGedanckenvondenKraefftendesmenschlichenVerstandesWolffs1977},
  sowie \cite[][xxvii]{Gawlick:Einleitung1996}}.}\\{}
  {[\S~30:]} Unter \ori{Wissenschaft} verstehe ich hier die Fertigkeit, seine
  Behauptungen zu beweisen, das heißt, sie aus gewissen und unerschütterlichen Grundsätzen
  durch gültigen Schluß
  herzuleiten.\footnote{\Cite[][\S~30]{Wolff:Discursuspraeliminarisdephilosophiaingenere1996}:
  \enquote{Per \ori{Scientiam} hic intelligo habitum asserta demonstrandi,
  hoc est, ex principiis certis {\&} immotis per
  legitimam consequentiam inferendi.} \cite[Vgl.][Vorbericht, \S~2]{Wolff:VernuenftigeGedankenvondenKraeftendesmenschlichenVerstandesundihremrichtigenGebraucheinErkenntnisderWahrheit1978}:
  \enquote{Durch die Wissenschaft verstehe ich eine Fertigkeit des
  Verstandes, alles, was man behauptet, aus unwidersprechlichen Gründen
  unumstößlich darzutun.} Der Text des \titel{Discursus praeliminaris} ist
  wiederum fast wörtlich in der Übersetzen dieses Zitats übernommen;
  \cite[vgl.][Prolegomena, \S 2]{Wolff:Cogitationesrationalesdeviribusintellectushumani1983}:
  \enquote{Per \ori{scientiam} intelligo habitum asserta demonstrandi, hoc
  est, ex principiis certis {\&} immotis per legitimam consequentiam
  inferendi.}}\\{}
  {[\S~31:]} \ori{In der Philosophie ist der Grund anzugeben, warum die
  möglichen Dinge Wirklichkeit erlangen können.}
  Philosophie nämlich ist die Wissenschaft der
  möglichen Dinge, wie und warum sie möglich sind (\S~29). {\punkt} Wer aber
  beweist, warum etwas geschehen kann, der gibt den Grund an, warum dies geschehen kann: Ein
  Grund ist nämlich dasjenige, woraus verstanden wird, warum etwas anderes
  ist.\footnote{\Cite[][\S~31]{Wolff:Discursuspraeliminarisdephilosophiaingenere1996}:
  \enquote{\ori{In philosophia reddenda est ratio, cur possibilia actum
  consequi possint.} Philosophia enim est scientia possibilium, quatenus esse
  possunt (\S~29). {\punkt} Enimvero qui demonstrat, cur aliquid fieri possit,
  is rationem reddit, cur id fieri queat: ratio enim id est, unde
  intelligitur, cur alterum sit.} \cite[Siehe
  auch][Praefatio]{Wolff:AerometriaeelementainquibusaliquotAerisviresacproprietatesjuxtamethodumGeometrarumdemonstrantur1981}:
  \enquote{Philosophi igitur est, non solum nosse, quae fieri possint, quae non;
  sed {\&} rationes perspicere, ob quas aliquid fieri potest, vel esse nequit.}}
\end{quote}
Das Definiens hat die nicht ganz einfache Form \enquote{ein $X$
(Wissenschaft) aller $Y$ (möglichen Dinge), insofern sie $Z$
(möglich/wirklich werdend) sind}. Klar ist, dass $X$ hier das \emph{genus}
angibt, welches in \S~30 weiter erläutert wird, und dass zu diesem \emph{genus}
eine \emph{differentia specifica} zu identifizieren ist.
Dabei ist es ratsam, mit
\authorfullcite{Stiebritz:ErlaeuterungenderVernuenftigenGedanckenvondenKraefftendesmenschlichenVerstandesWolffs1977}
in dieser Definition zwischen einer \emph{differentia specifica
materialis} (den Gegenstandsbereich, $Y$, \enquote{\emph{possibilium}}) und
einer \emph{differentia specifica formalis} ($Z$, \enquote{\emph{quatenus esse
possunt}}) zu unterscheiden und beides getrennt zu
erläutern.\footnote{\cite[Vgl.][\S~29]{Stiebritz:ErlaeuterungenderVernuenftigenGedanckenvondenKraefftendesmenschlichenVerstandesWolffs1977}.}
Es ist die \emph{differentia specifica formalis}, die in \S~31 nähere
Erläuterung erfährt und sich auf den Begriff der \emph{cognitio philosophica}
bezieht. Ich erläutere die drei Momente der Definition im einzelnen:

\begin{nummerierung}

\item Das \emph{genus} der Philosophie ist \enquote{Wissenschaft}
(\enquote{\emph{scientia}}), wobei Wissenschaft als Fähigkeit oder Kompetenz
(\enquote{habitus}) bestimmt wird. Nicht eine Menge wahrer Erkenntnisse zählt
als Wissenschaft -- sie ist keine (auch keine systematisch geordnete) Sammlung wahrer Sätze --, sondern die je individuelle
Kompetenz zu wissenschaftlichem Arbeiten. \Revision{Die wissenschaftliche
Tätigkeit wird hier als \emph{beweisen} (\enquote{\emph{demonstrare}})
näher beschrieben, was die Fundierung in \emph{letzten Gründen} -- und nicht
bloß die Prüfung (\enquote{\emph{probare}}) durch Kontrolle der nächsten vorausgehenden
Gründe bezeichnet. Der Beweis (\enquote{\emph{demonstratio}}) lässt im
Unterschied zur Prüfung (\enquote{\emph{probatio}}) ausschließlich
Definitionen, Axioma, unzweifelhafte Erfahrungssätze und bereits bewiesene
Aussagen gelten.}\footnote{\Revision{\enquote{\ori{Probatio igitur
probabilis a demonstratione non differt nisi principiis.} Nam demonstrationis
principia sunt definitiones, axiomata, experientiae indubitatae {\&}
propositiones jam demonstratae}
\parencite[][\S~588]{Wolff:PhilosophiarationalissiveLogica1740}. Siehe auch
\cite[][\S~562]{Wolff:PhilosophiarationalissiveLogica1740}.}} Diese Kompetenz
bestimmt \authorcite{Wolff:Discursuspraeliminarisdephilosophiaingenere1996} freilich über eine einheitliche Methodik: \Revision{die mathematische Methode}.\footnote{Siehe hierzu Kapitel \ref{Abschnitt:WolffunddieWissenschaftlichkeitderPhilosophiemoregeometrico}, ab Seite
\pageref{Abschnitt:WolffunddieWissenschaftlichkeitderPhilosophiemoregeometrico}.}
Wie wir in Kapitel \ref{subsection:SelbstdenkenbeiKant} sahen, ist damit die
Grundbedingung des Selbstdenkens nach
\authorcite{Wolff:Discursuspraeliminarisdephilosophiaingenere1996} angegeben:
Ein wirklicher Selbstdenker (\enquote{\emph{eclecticus}}) kann nur sein, wer
über eine methodisch geschulte Vernunft, also über Wissenschaft
(\emph{scientia}) im Sinne eigener Kompetenz (nicht im Sinne des Kennens vieler
wahrer Aussagen) verfügt. Da \emph{scientia} das \emph{genus} ist, ist
Philosophie eine, wenngleich nicht die einzige Wissenschaft.\footnote{Daraus
erhellt schon, dass der Begriff \enquote{\emph{philosophia}} nicht mit dem der Wissenschaft
(\enquote{scientia}) zusammenfällt; auch die Mathematik ist Wissenschaft, aber keine
Philosophie. \authorfullcite{Schneiders:Deusestphilosophusabsolutesummus1986}
behauptet dagegen, nach \authorcite{Wolff:Psychologiaempirica1968} zähle streng
genommen nur die Philosophie als Wissenschaft: \enquote{Die wahre Wissenschaft
beginnt erst mit der Ursachenerkenntnis, mit der cognitio philosophica; ja
Philosophie und Wissenschaft sind sogar mehr oder weniger identisch.} Dagegen
würden \enquote{die beiden anderen Erkenntnisarten zu vorwissenschaftlichem
Faktenkennen bzw. zu bloßer Methodologie {\punkt} degradier[t]}
\parencite[vgl.][15]{Schneiders:Deusestphilosophusabsolutesummus1986}.
Allerdings kann ich nicht erkennen, wie
\authorcite{Schneiders:Deusestphilosophusabsolutesummus1986} dieses Urteil
fundiert. Gerade die Mathematik scheint zwar ihre Relevanz nach
\authorcite{Wolff:Psychologiaempirica1968} von der Philosophie zu erhalten, aber
vor allem methodisch das Paradigma wissenschaftlichen Erkennens darzustellen.}

\item Dass Philosophie die Wissenschaft aller möglichen Dinge
(\enquote{\emph{scientia possibilium}}) ist, bestimmt ihren Gegenstandsbereich.
\authorcite{Wolff:Discursuspraeliminarisdephilosophiaingenere1996}
bestimmt das Mögliche als das, was keinen Widerspruch in sich oder zu etwas anderem,
von dem wir wissen, dass es ist oder geschieht, enthält.\footnote{\enquote{Möglich nenne ich alles, was seyn kan, es mag entweder würcklich da
seyn, oder nicht} \parencite[][Vorbericht,
\S~3]{Wolff:VernuenftigeGedankenvondenKraeftendesmenschlichenVerstandesundihremrichtigenGebraucheinErkenntnisderWahrheit1978}. \enquote{Weil
nichts zugleich seyn und nicht seyn kan (\S . 10.); so erkennet man, daß etwas
unmöglich sey, wenn es demjenigen widerspricht, davon wir bereits wissen, daß es
ist oder seyn kan\punkt{} Woraus man ferner ersiehet, daß \ori{möglich} sey, was
nichts widersprechendes in sich enthält, das ist, nicht
allein selbst neben andern Dingen, welche sind oder seyn können, bestehen kan,
sondern auch nur dergleichen in sich enthält, so neben einander bestehen kan}
\parencite[][\S~12]{Wolff:VernuenftigeGedanckenvonGottderWeltundderSeeledesMenschenauchallenDingenueberhauptDeutscheMetaphysik1983}.
\authorcite{Wolff:Psychologiaempirica1968}s Ausdifferenzierung des Begriffs des Möglichen ist von
\authorfullcite{Baumgarten:Metaphysica---Metaphysik2011} aufgegriffen worden,
der zwischen einem \enquote{in
sich Möglichen} (\emph{possibile in se}) und einem \enquote{hypothetisch
Möglichen} (\emph{possibile hypothetice}) unterscheidet
\mkbibparens{\cite[Vgl.][\S\S~15--18]{Baumgarten:Metaphysica---Metaphysik2011};
\cite[][XVII: 29.16--24, 30.8--23]{Kant:GesammelteWerke1900ff.}}.} Einen
Widerspruch in sich können wir inneren Widerspruch, einen Widerspruch zu etwas
anderem, von dem wir wissen, dass es ist oder geschieht, einen äußeren
Widerspruch nennen. Es ist beispielsweise unmöglich, dass Max sowohl Junggeselle
als auch verheiratet ist (innerer Widerspruch); und ebenso ist es unmöglich,
dass Max mit Ingrid in Urlaub ist, wenn wir bereits wissen, dass Ingrid im
Nachbarraum eine Lehrveranstaltung abhält (äußerer Widerspruch). Dass Max
Junggeselle ist, ist möglich (sofern uns nichts gegenteiliges bekannt ist). Und
dass Max mit Ingrid im Urlaub ist, ist ebenso möglich, insofern wir Ingrid schon
länger nicht gesehen haben (und sie uns vielleicht sogar von ihren Urlaubsplänen
mit Max erzählte). Da wir wissen, dass Pferde nicht fliegen können und Geräte
ohne Energiezufuhr nicht ewig in Bewegung bleiben können, stehen solche Dinge in
Widerspruch zu diesen Tatsachen, von denen wir Erkenntnis haben. Sie sind also
unmöglich. Entsprechend gehören fliegende Pferde und \emph{perpetua mobilia}
nicht zum Gegenstandsbereich der Philosophie.

Dabei stellt die Angabe, Philosophie beziehe sich auf sämtliche
\emph{possibilia}, eine Betonung der Weite des Gegenstandsbereichs in zwei
Hinsichten dar. \emph{Erstens} sagt dies, dass sie über alle wirklichen Dinge
hinausgeht und sich auch auf das bezieht, was nur möglich ist.
Denn -- wie \authorcite{Wolff:Discursuspraeliminarisdephilosophiaingenere1996}
betont -- betrachtet Philosophie nicht nur dasjenige, was tatsächlich ist oder
geschieht, sondern auch das, was sein oder geschehen \emph{kann}, wenngleich die
entsprechende Möglichkeit nicht realisiert ist.\footnote{\cite[Vgl.][sectio II,
caput I,
\S~3]{Wolff:RatiopraelectionumWolfianaruminMathesinetPhilosophiamuniversametOpusHugonisGrotiideJurebellietpacis1735}:
\enquote{Est nempe mihi philosophia scientia omnium possibilium qua talium, ita
ut ad obiectum philosophia referri debeant res omnes, qualescunque esse possunt,
\myemph{sive existant sive non}.}} Wenn sich die Physik etwa mit dem freien Fall
oder der Flugbahn einer Kanonenkugel beschäftigt, dann ist weder vorausgesetzt noch auch
nur von Interesse, ob tatsächlich gerade ein Gegenstand fällt oder eine
Kanonenkugel abgeschossen wurde. Die unmittelbare und sicherlich gewollte Folge
ist, dass sie die Dinge nicht im einzelnen, sondern im allgemeinen
betrachtet.\footnote{\enquote{Unde in philosophia res considerantur in
universali non in singulari} \parencite[][\S~3,
\pno~107\,f.]{Wolff:RatiopraelectionumWolfianaruminMathesinetPhilosophiamuniversametOpusHugonisGrotiideJurebellietpacis1735}.}
\emph{Dieser bestimmte} Flug \emph{dieser bestimmten} Kanonenkugel kommt -- wenn überhaupt --
nur als Instantiierung allgemeiner physikalischer Gesetze in Betracht.

\emph{Zweitens} ist sie nicht auf den Gegenstandsbereich einer einzelnen
Disziplin wie der Physik oder der Psychologie eingeschränkt, sondern umfasst
schließlich den Gesamtbereich unseres Wissens. Explizit zählt
\authorcite{Wolff:Discursuspraeliminarisdephilosophiaingenere1996}
Logik\footcite[Vgl.][\S~61]{Wolff:Discursuspraeliminarisdephilosophiaingenere1996},
Ethik und
Politik\footcite[Vgl.][\S\S~63--65]{Wolff:Discursuspraeliminarisdephilosophiaingenere1996},
Ökonomie\footcite[Vgl.][\S~66\,f.]{Wolff:Discursuspraeliminarisdephilosophiaingenere1996},
Naturrecht\footcite[Vgl.][\S~68]{Wolff:Discursuspraeliminarisdephilosophiaingenere1996},
Technologie\footcite[Vgl.][\S~71]{Wolff:Discursuspraeliminarisdephilosophiaingenere1996},
Ontologie\footcite[Vgl.][\S~73]{Wolff:Discursuspraeliminarisdephilosophiaingenere1996},
Physik\footcite[Vgl.][\S
76]{Wolff:Discursuspraeliminarisdephilosophiaingenere1996},
Kosmologie\footcite[Vgl.][\S
77]{Wolff:Discursuspraeliminarisdephilosophiaingenere1996}, Pneumatik und
Metaphysik\footcite[Vgl.][\S~79]{Wolff:Discursuspraeliminarisdephilosophiaingenere1996},
Meteorologie\footcite[Vgl.][\S~80]{Wolff:Discursuspraeliminarisdephilosophiaingenere1996}
und einige andere auf. \emph{Jede} Erkenntnis und
\emph{jede} handwerkliche Fähigkeit kann philosophisch betrachtet und dadurch
verbessert werden. Philosophie wird damit -- wie
\authorcite{Stiebritz:ErlaeuterungenderVernuenftigenGedanckenvondenKraefftendesmenschlichenVerstandesWolffs1977}
erläutert -- in einem weiteren Sinne betrachtet, insofern sie eine Eigenschaft
oder Fähigkeit des Subjekts bezeichnet (\enquote{\emph{philosophia subiective et
habitualiter considerata}}), nicht eine Menge von Erkenntnissen über einen
bestimmten Gegenstandsbereich (\enquote{\emph{philosophia obiective et
systematice
considerata}}\footnote{\label{Anmerkung:StiebritzZuSubiectiveundObiective}\cite[][\S~44]{Stiebritz:ErlaeuterungenderVernuenftigenGedanckenvondenKraefftendesmenschlichenVerstandesWolffs1977}.}).
Die \emph{differentia specifica materialis} bezieht sich damit auf die
Bestimmung des \emph{genus} als einer Fähigkeit. Wegen ihres Umfang sei es -- so
\authorcite{Wolff:Discursuspraeliminarisdephilosophiaingenere1996} -- auch
niemandem vergönnt, in allen Belangen Philosoph zu
sein;\footcite[Vgl.][\S~48]{Wolff:Discursuspraeliminarisdephilosophiaingenere1996}
es gibt den Universalgelehrten nicht, der zu allen Fragen der Philosophie
kompetent urteilen kann. Betrachtete
\authorcite{Wolff:Discursuspraeliminarisdephilosophiaingenere1996} hingegen die
Philosophie als durch ihren Gegenstandsbereich bestimmt, dann ergäbe sich ein
Begriff von Philosophie, der dem Weltbegriff bei \name[Immanuel]{Kant}s ähnelte,
insofern die so bestimmte Philosophie dasjenige Wissen akkumulierte, welches
jeder Mensch zur erfolgreichen Gestaltung seines eigenen Lebens (zur
Glückseligkeit)
benötigt.\footnote{\cite[Vgl.][\S~44]{Stiebritz:ErlaeuterungenderVernuenftigenGedanckenvondenKraefftendesmenschlichenVerstandesWolffs1977}:
\enquote{Erwäget man aber die Welt-Weisheit obiective, systematice, und also in
engern Verstande: so verstehet man darunter, nach dem angenommenen Gebrauch der
in Reden, nichts anders, als eine durch die blose Vernunft erlangte
Wissenschaft, worinnen einem Menschen deutlich vorgetragen wird, was ihm in
ieden Stande zu seiner Glückseligkeit zu wissen und zu thun nöthig ist.} Siehe
hierzu oben Kapitel \ref{subsection:DieBestimmungdesMenschen}.}
Was \authorcite{Wolff:Discursuspraeliminarisdephilosophiaingenere1996} in seiner
Definition im \titel{Discursus praeliminaris} vorschwebt, ist eine Art
Schulbegriff von Philosophie, wonach sie \enquote{nur als eine von den
Geschicklichkeiten zu gewissen beliebigen Zwecken angesehen
wird.}\footnote{\cite[][B 867]{Kant:KritikderreinenVernunft2003}; \cite[][III:
543.33--34]{Kant:GesammelteWerke1900ff.}.}

Die Definition von Möglichkeit als Widerspruchsfreiheit
suggeriert, dass der Gegenstandsbereich der Philosophie (die \emph{possibilia})
durch bloße logische Konsistenz definiert ist. Doch dann verstehen wir die
\emph{differentia specifica formalis} nicht, wonach Philosophie nach Gründen für
diese Möglichkeit sucht.\footnote{Siehe dazu auch
\cite[][23]{Schneiders:Deusestphilosophusabsolutesummus1986}: \enquote{Allem Anschein nach versucht er [d.\,i.
\authorcite{Wolff:Discursuspraeliminarisdephilosophiaingenere1996}; A.\,G.], mit
seiner Definition der Philosophie als Wissenschaft vom Möglichen zwei Dinge
miteinander zu vereinbaren. Einerseits möchte er die Philosophie (trotz des
betonten Ausgangs von der Erfahrung) von der Fesselung an die Wirklichkeit
befreien. Das Denken muß sich in den Raum des logisch Möglichen
als des widerspruchsfrei Denkbaren erheben können -- gegebenenfalls sogar unter
Vernachlässigung der Wirklichkeit. Andererseits soll die Philosophie gerade
erklären, wie das Wirkliche möglich (ermöglicht) ist, d.\,h. nach seinen Gründen
oder Bedingungen fragen -- wobei Wolff kaum deutliche Unterschiede zwischen
physischer und metaphysischer Ursache macht.}
\authorcite{Schneiders:Deusestphilosophusabsolutesummus1986} fragt allgemein,
ob der Möglichkeitsbegriff bei \authorcite{Wolff:Psychologiaempirica1968}
überhaupt als Grundlage eines Verständnisses seines Philosophiebegriffs dienen
kann, zumal es fraglich sei, \enquote{ob die Erörterung des
Möglichkeitsbegriffs zu klaren Resultaten geführt hat}
\parencite[][9]{Schneiders:Deusestphilosophusabsolutesummus1986}.} Dies
verstehen wir nur, wenn alles, was möglich ist, einen Grund seiner Möglichkeit
hat. Möglich ist dann aber nicht einfach dasjenige, was widerspruchsfrei denkbar
ist, sondern erst das \singlequote{real Mögliche}, also dasjenige, dessen
Verwirklichung in dieser Welt geschieht oder doch geschehen kann -- dasjenige,
von wir wissen können, unter welchen Bedingungen, die ihrerseits
\emph{realiter} möglich sind, es tatsächlich wirklich würde. Der Entwurf, den
die Ent\-wick\-lungs\-ab\-tei\-lung eines Automobilkonzerns vorlegt, handelt von
etwas, das zunächst nicht wirklich, aber (hoffentlich) in diesem Sinne
\emph{realiter} möglich ist. Die Entwickler wissen zugleich, wie es möglich ist -- welche Arbeiten
unternommen werden müssen, um es zur Wirklichkeit zu bringen. Es handelt sich um
eine reale Möglichkeit, insofern beschrieben werden kann, wie ein Prototyp zu
bauen ist. Mit diesem Prototypen tritt die reale Möglichkeit über zur
Wirklichkeit.\footnote{Auf diese Weise trägt
\authorfullcite{Seidl:ArtikelenquoteMoeglichkeit1984} die Unterscheidung von
logischer und realer Möglichkeit als Bestandteil von
\authorcite{Wolff:Psychologiaempirica1968}s wie auch von
\authorcite{Baumgarten:Metaphysica---Metaphysik2011}s Philosophie vor. Die
äußere -- bei \authorcite{Baumgarten:Metaphysica---Metaphysik2011}:
hypothetische
\mkbibparens{\cite[Vgl.][\S~16]{Baumgarten:Metaphysica---Metaphysik2011},
\cite[][XVII: 30.8--12]{Kant:GesammelteWerke1900ff.}} -- Möglichkeit sei eine
solche, die von Ursachen abhänge
\parencite[vgl.][86]{Seidl:ArtikelenquoteMoeglichkeit1984}.
\cite[Siehe auch][17]{Schneiders:Deusestphilosophusabsolutesummus1986}:
\enquote{Möglichsein wird also mit Bezug auf Wirklichsein verstanden, als das
Möglichsein eines zunächst nur möglichen, aber möglicherweise dann auch
wirklichen Dinges: die res possibilis ist als essentia zunächst potentia.}}
Scheinbar möchte
\authorcite{Wolff:Discursuspraeliminarisdephilosophiaingenere1996} behaupten,
dass diese reale Möglichkeit bereits damit garantiert ist, dass der Begriff
(oder eine Konzeption) von etwas keinen allgemeinen (Natur-) Gesetzen
widerspricht (äußere Möglichkeit).

\item Der dritte Aspekt nach \emph{genus} und \emph{differentia specifica
materialis} beantwortet die Frage, in welcher Hinsicht die möglichen Dinge
wissenschaftlich untersucht werden. Der \emph{differentia specifica formalis}
nach ist es Aufgabe der Philosophie zu erläutern, \emph{warum} etwas
\emph{möglich} ist oder sein kann (\enquote{quatenus [possibilia] esse
possunt}). Dabei liegt die Grundlage der philosophischen Erkenntnis in dem Satz
vom zureichenden Grunde, der sagt, dass alles, was \emph{ist}, seinen zureichenden
Grund hat, warum es ist. Aber gilt dies auch von allem, was möglich ist?
\authorcite{Baumgarten:Metaphysica---Metaphysik2011} behauptet, dass jedes
Mögliche als solches einen zureichenden Grund
habe.\footnote{\cite[Vgl.][\S~20]{Baumgarten:Metaphysica---Metaphysik2011},
\cite[][XVII: 31.9--12]{Kant:GesammelteWerke1900ff.}.} Und so erläutert auch
\authorcite{Stiebritz:ErlaeuterungenderVernuenftigenGedanckenvondenKraefftendesmenschlichenVerstandesWolffs1977}
\authorcite{Wolff:Discursuspraeliminarisdephilosophiaingenere1996}s
Definition.\footcite[Vgl.][\S~32]{Stiebritz:ErlaeuterungenderVernuenftigenGedanckenvondenKraefftendesmenschlichenVerstandesWolffs1977}
Nun scheint \authorcite{Wolff:Discursuspraeliminarisdephilosophiaingenere1996}
den Satz vom zureichenden Grunde jedoch stets nur auf \emph{wirkliche} Dinge zu
beziehen. Weder in der Fassung des
\titel{Discursus}\footnote{\enquote{\ori{Ea, quae sunt vel fiunt, sua non
destituuntur ratione, unde intelligitur, cur sint, vel fiant}}
\parencite[][\S~4]{Wolff:Discursuspraeliminarisdephilosophiaingenere1996}.
Der \titel{Discursus} spricht also von den Dingen, die sind oder geschehen
(\enquote{\emph{ea, quae sunt vel fiunt}}) und orientiert sich damit an die
Formulierung der Definition philosophischer Erkenntnis.} noch in der
\titel{Deutschen
Metaphysik}\footnote{\enquote{[S]o muß auch alles, was ist, seinen zureichenden
Grund haben, warum es ist}
\parencite[][\S~30,
\ohio]{Wolff:VernuenftigeGedanckenvonGottderWeltundderSeeledesMenschenauchallenDingenueberhauptDeutscheMetaphysik1983}.} noch
auch in der \titel{Philosophia prima sive ontologia}\footnote{Die lateinische
Metaphysik formuliert das Prinzip negativ: \enquote{nil sit sine ratione
sufficiente, cur potius sit, quam non sit}
\parencite[][\S~71]{Wolff:Philosophiaprimasiveontologia1977}.} ist von
\emph{möglichen} Dingen die Rede.
Aber dennoch scheint die Deutung, dass er auch auf mögliche Dinge anwendbar ist,
nicht ganz abwegig zu sein; zumindest ist es auch die Interpretation, die
\authorcite{Stiebritz:ErlaeuterungderWolffschenVernuenfftigenGedanckenvonGottderWeltundderSeeledesMenschenauchallenDingenueberhaupt1999}
zu \authorcite{Wolff:Discursuspraeliminarisdephilosophiaingenere1996}s
Metaphysik anführt: Der Satz vom zureichenden Grund spreche von allem, was
\emph{ist}, nicht von allem, was \emph{wirklich} ist. Und dieses \enquote{ist}
lasse sich auf mögliche so gut wie auf wirkliche Dinge beziehen.\footnote{\enquote{Der Satz heisset:
alles, was ist, hat seinen zureichenden Grund, warum es ist. Was da ist, ist entweder möglich
allein, oder auch wircklich. Darum gehet dieser Satz beydes auf mögliche, als
auch auf wirckliche Dinge. Und von beyden wird bekräftiget, daß iederzeit etwas
da seyn müste, woraus ich bey einem möglichen verstehen kan, woraus es vielmehr
möglich, als unmöglich; vielmehr auf die Art, als auf eine iede andre Art
möglich sey}
\parencite[][\S~75]{Stiebritz:ErlaeuterungderWolffschenVernuenfftigenGedanckenvonGottderWeltundderSeeledesMenschenauchallenDingenueberhaupt1999}.}
Und in der Tat spricht
\authorcite{Wolff:Discursuspraeliminarisdephilosophiaingenere1996} in seinen
Definitionen dieses Satzes zwar nicht von möglichen Dingen, grenzt dessen
Anwendungsbereich aber auch nicht explizit auf die wirklichen Dinge
ein.\footnote{Er erläutert den Satz vom zureichenden Grund zwar in der
\titel{Deutschen Metaphysik} als
\enquote{es muß allezeit etwas seyn, daraus man verstehen kan, warum es würcklich werden kan}, aber dabei ist offensichtlich nicht von \enquote{wirklich sein}, sondern von \enquote{wirklich werden können}
die Rede, was nicht voraussetzt, dass etwas bereits wirklich \emph{ist}.} Somit
bliebe die Option bestehen,
\authorcite{Wolff:Discursuspraeliminarisdephilosophiaingenere1996} so zu
interpretieren, dass das \enquote{alles, was
ist,}\footcite[][\S~30]{Wolff:VernuenftigeGedanckenvonGottderWeltundderSeeledesMenschenauchallenDingenueberhauptDeutscheMetaphysik1983},
von dem das \emph{principium rationis sufficientis} sagt, es habe einen
zureichenden Grund, sich auf alle wirklichen und möglichen Dinge
bezieht. Dies mag zunächst etwas gezwungen erscheinen, hat aber
den Vorteil, dass seine Definition von Philosophie als der Wissenschaft der möglichen Dingen, wie und
warum sie möglich sind, verständlich wird. \Revision{Und tatsächlich sagt
\authorcite{Wolff:Discursuspraeliminarisdephilosophiaingenere1996} in der
\titel{Philosophia prima sive Ontologia}, dass der Begriff des Seienden
(\enquote{\emph{ens}}) nicht nur das Wirkliche, sondern ebenso das Mögliche
umfasst.\footnote{\Revision{\enquote{[\ori{Q}]\ori{uod possibile est, ens
est}} \parencite[][\S~135]{Wolff:Philosophiaprimasiveontologia1977}. Siehe auch
\cite[][\S\S~102, 133\,f.]{Wolff:Philosophiaprimasiveontologia1977}.}} Und
insofern ist das Prinzip vom zureichenden Grund ebenso auf die
\emph{possibilia} anwendbar.}

\end{nummerierung}


\begin{comment}
Solche Fragen nach dem \enquote{Warum?} machen deutlich, dass
\authorcite{Wolff:Psychologiaempirica1968} in der Philosophie primär an
(letztlich empirisches) Wissen über Ursachen denkt. Tatsächlich ist die
Bandbreite der Disziplinen, die \authorcite{Wolff:Psychologiaempirica1968} zur
Philosophie zählt, jedoch noch größer:Somit transzendiert sein Philosophiebegriff sowohl unseren heutigen Begriff
genuin \emph{philosophischen} Wissens, als auch den der an der kausalen
Erklärung des Geschehenes in der Welt interessierten Naturwissenschaft.
Dies funktioniert in \authorcite{Wolff:Psychologiaempirica1968}s System, weil dieser
nicht -- wie nach ihm \name[Immanuel]{Kant} -- zwischen empirischen und
Vernunftwissenschaften unterscheidet, sondern jede Wissenschaft letztlich
empirisch fundiert sieht. Man beachte, dass nach
\authorcite{Wolff:Psychologiaempirica1968} (wie auch nach
\authorcite{Baumgarten:Metaphysica---Metaphysik2011}) selbst die Logik eine
empirische Grundlage (in der Psychologie) hat. Erst \name[Immanuel]{Kant} setzt
hier eine klare Trennung durch, die in der Folge auch verhindert,
den Begriff der Philosophie so breit zu
fassen. Denn es handelt sich um ganz unterschiedliche Fragen, die sich
empirische Psychologie und Logik bezüglich des menschlichen Denkens
stellen.\footnote{\cite[Vgl.][A 6]{Kant:ImmanuelKantsLogik1977}.} Während die
(empirische) Psychologie das \enquote{Warum?} als Frage nach den kausalen
Ursachen mentaler Prozesse deutet, versteht die Logik das \enquote{Warum?} als
Frage nach dem Grund der Berechtigung eines geistigen Aktes wie beispielsweise
eines Schlusses. Bei \authorcite{Wolff:Psychologiaempirica1968} ist
\enquote{philosophia} oder \enquote{Weltweisheit} allgemein der Oberbegriff zu
all denjenigen Wissenschaften, die sich um die Beantwortung von
\enquote{Warum}-Fragen bemühen, wobei dieses \enquote{warum?} zumindest aus
unserer nach-kantischen Perspektive in unterschiedlicher Bedeutung genommen zu
werden scheint, insofern wir einerseits von \enquote{Ursache}, anderseits von
\enquote{Grund} sprechen. Auch \authorcite{Wolff:Psychologiaempirica1968} ist
sich dessen bewusst, entscheidet sich aber ganz bewusst für die Verwendung von
\enquote{Grund} mit ausgeweiteter Bedeutung.\footnote{\cite[Vgl.][\S
13]{Wolff:VernuenftigeGedanckenvonGottderWeltundderSeeledesMenschenauchallenDingenueberhauptandererTheilbestehendinausfuehrlichenAnmerkungen1983}:
\enquote{Das Wort \ori{Raison} oder \ori{Ratio} ist allgemeiner, als das Wort
\ori{Causa} oder Ursache, und hat etwas mehrers, als dieses, zu sagen. Ich gebe
also dem Worte \ori{Ursache} die Bedeutung, welche die Weltweisen dem Worte
\ori{Causa} gegeben haben, und, weil ich demnach ein anderes dazu nöthig gehabt,
wenn ich in den Fällen Teutsch reden will, wo der Franzose \ori{Raison} fordert,
der Lateiner \ori{Rationem} wissen will, so nehme ich das Wort \ori{Grund} dazu,
weil dasselbe in dergleichen Fällen gebraucht wird.}}

Ein Missverständnis wäre es jedoch anzunehmen, die
philosophische Erkenntnis sei der historischen gegenüber überlegen, weil
letztere die Erfahrung als Grundlage habe und daher nicht so gewiss sei, wie
eine reine Vernunftwissenschaft. Besonders deutlich wird dies dort, wo
\authorcite{Wolff:Psychologiaempirica1968} die \enquote{\emph{psychologia
empirica}}, die nach ihm wie nach
\authorcite{Baumgarten:Metaphysica---Metaphysik2011} zur Metaphysik
gehört,\footnote{\cite[Vgl.][\S
99]{Wolff:Discursuspraeliminarisdephilosophiaingenere1996}, wo die Psychologie
schlechthin (und \emph{a fortiori} auch die empirische Psychologie) zur
Metaphysik gezählt wird;
\cite[][\S\S 501--503]{Baumgarten:Metaphysica---Metaphysik2011}.} sowie die
\enquote{\emph{psychologia rationalis}} einführt und ihr Verhältnis zueinander bestimmt. Denn von der \emph{rationalen} Psychologie sagt er:
\begin{quote}
  Weil dies ein neues und der vorgefaßten Meinung entgegengesetztes Unterfangen
  ist, Neues aber anfangs von den meisten ungern zugestanden wird, war das der
  Hauptgrund dafür, daß ich die rationale Psychologie von der empirischen
  trennte, damit nicht psychologische Erkenntnis ohne Unterschied zurückgewiesen
  würde. {\punkt} Die praktische Philosophie ist von größter Bedeutung: was
  daher von größter Bedeutung ist, wollten wir nicht auf solche Grundsätze
  aufbauen, die bestritten werden können. Aus diesem Grund bauen wir die
  Wahrheiten der praktischen Philosophie nur auf solche Grundsätze auf, die in
  der empirischen Psychologie durch Erfahrung evident festgestellt
  werden.\footnote{\cite[][\S
  112]{Wolff:Discursuspraeliminarisdephilosophiaingenere1996}:
  \enquote{Novus cum sit ausus {\&} praejudicatae opinioni adversus, nova vero
  ab initio a plerisque aegre admittantur; praegnans maxime ratio fuit, cur
  Psychologiam rationalem ab empiricam discernerem, ne psychologica promiscue
  rejicerentur. {\punkt} Philosophia practica est maximi momenti: quae igitur
  maximi sunt momenti, istiusmodi principiis superstruere noluimus, quae in
  disceptationem vocantur. Ea de causa veritates philosophiae practicae non
  superstruimus nisi principiis, quae per experientiam in Psychologia empirica
  evidenter stabiliuntur.}}
\end{quote}
Die historische Erkenntnis liefert gerade dadurch, dass sie der Erfahrung
entspringt, \enquote{eine feste und unerschütterliche
Grundlage.}\footnote{\cite[][\S
11]{Wolff:Discursuspraeliminarisdephilosophiaingenere1996}:
\enquote{\dots\unkern firmo ac inconcusso nititur fundamento.}} Gerade deswegen
sollen wir nach \authorcite{Wolff:Psychologiaempirica1968} die \singlequote{Ehe}
zwischen historischer und philosophischer Erkenntnis
heiligen.\footnote{\cite[Vgl.][\S
12]{Wolff:Discursuspraeliminarisdephilosophiaingenere1996}:
\enquote{\dots\unkern{}nobis per omnem philosophiam sanctum est utriusque
connubium.}} Denn beide stehen nicht unverbunden nebeneinander, sondern
philosophische Erkenntnis basiert auf historischer Erkenntnis. \enquote{\ori{Wenn durch Erfahrung dasjenige festgestellt wird, woraus sich anderes, was ist und geschieht oder geschehen
kann, begründen lässt, liefert die historische Erkenntnis die Grundlage der
philosophischen.}}\footnote{\Cite[][\S
10]{Wolff:Discursuspraeliminarisdephilosophiaingenere1996}:
\enquote{\ori{Si per experientiam stabiliuntur ea, ex quibus aliorum, quae sunt
atque fiunt, vel fieri possunt, ratio reddi potest, cognitio historica
philosophicae fundamentum praebet.}}} Wir wissen beispielsweise aus Erfahrung,
dass massive Gegenstände, die in Wasser gelegt werden, auf diesem schwimmen,
während andere massive Gegenstände untergehen. Unsere historische Erkenntnis
besagt also, dass massive Gegenstände schwimmen können. Aus der Erfahrung wissen
wir nun vielleicht, dass genau diejenigen massiven Gegenstände schwimmen, deren
Material eine Dichte aufweist, die kleiner als die des Wassers ist. Auf
Grundlage dieser historischen Erkenntnis erkennen wir, \emph{warum} es
möglich ist, dass massive Gegenstände auf Wasser schwimmen: \emph{weil} ihre
Dichte geringer ist als die des Wassers. Wir haben nun eine philosophische
Erkenntnis der Tatsache, dass massive Gegenstände auf Wasser schwimmen können.
\end{comment}

Während also in
\authorcite{Wolff:Discursuspraeliminarisdephilosophiaingenere1996}s Definition
der Philosophie das \emph{genus} die Philosophie als Wissenschaft auszeichnet,
legen die \emph{differentia specifica materialis} (\enquote{\emph{possibilium}})
ihre Gegenstandsbereich (die \emph{possibilia}) und die \emph{differentia
specifica formalis} die anzustrebende Erkenntnisart (\emph{cognitio
philosophica}) fest. Die \emph{possibilia} sind all diejenigen Dinge und
Geschehnisse, die \emph{realiter} möglich sind. Dass diese \emph{possibilia} in
der Philosophie dahingehend untersucht werden, \enquote{\emph{quatenus esse
possunt}}, heißt, dass die Philosophie philosophische Erkenntnis von ihnen
generiert. Sie ist die Wissenschaft, der es darum geht, von dem, was
\emph{realiter} möglich ist, \emph{Gründe} und \emph{Ursachen}\footnote{Dabei
ist sich \authorcite{Wolff:Discursuspraeliminarisdephilosophiaingenere1996}
durchaus der Tatsache bewusst, dass es einen Unterschied zwischen Gründen und
Ursachen gibt \parencite[vgl.][\S~71]{Wolff:Philosophiaprimasiveontologia1977}. Er nutzt
selbst das Wort \enquote{Grund} als Übersetzung des lateinischen
\enquote{\emph{ratio}} und \enquote{Ursache} für \enquote{\emph{causa}}
\parencite[vgl.][\S~13]{Wolff:VernuenftigeGedanckenvonGottderWeltundderSeeledesMenschenauchallenDingenueberhauptDeutscheMetaphysik1983}.
Siehe dazu auch
\cite[][\S~74]{Stiebritz:ErlaeuterungderWolffschenVernuenfftigenGedanckenvonGottderWeltundderSeeledesMenschenauchallenDingenueberhaupt1999}.}
seines möglichen Seins oder Geschehens anzugeben. Da Wissenschaft als Fähigkeit
zu wissenschaftlich strengem methodischem Vorgehen bestimmt wurde, ist sie damit
die Kompetenz, unter Wahrung der Methode der Vernunft -- \Revision{der
mathematischen Methode} -- philosophische Erkenntnis möglicher Dinge und Sachverhalte zu
erwerben und darzustellen.


\subsection{Historische Kenntnis der
Philosophie}\label{subsubsection:HistorischeKenntnisDerPhilosophie}
Der größte Teil unseres Wissen besteht nach
\authorcite{Wolff:Psychologiaempirica1968} jedoch nicht in philosophischer,
sondern in historischer Erkenntnis.
Das \singlequote{Volk} (\emph{vulgus}) besitze ausschließlich historische
Erkenntnis, welche darum auch die \singlequote{gewöhnliche Erkenntnis}
(\emph{cognitio vulgaris}) genannt werde. Die Philosophie ist nur wenigen
vorenthalten. Allerdings benötigten die meisten Menschen im Leben auch keine
andere Erkenntnis als die historische, denn bei allem Nutzen, den uns die
philosophische Erkenntnis bringt, sei im Alltag doch die historische Erkenntnis
vollends
ausreichend.\footnote{\cite[Vgl.][\S~23]{Wolff:Discursuspraeliminarisdephilosophiaingenere1996},
siehe zum Nutzen der historischen Erkenntnis auch
\cite[][\S~13]{Wolff:Discursuspraeliminarisdephilosophiaingenere1996}.} Sie wird
dadurch nicht herabgewürdigt und kann auch nicht durch eine andere
Erkenntnis ersetzt werden. Die historische Erkenntnis liefert gerade dadurch,
dass sie der Erfahrung entspringt, der Philosophie eine sichere
Basis.\footnote{\cite[Vgl.][\S~11]{Wolff:Discursuspraeliminarisdephilosophiaingenere1996}:
\enquote{\dots\unkern firmo ac inconcusso nititur fundamento.}} Gerade deswegen
sollen wir nach \authorcite{Wolff:Psychologiaempirica1968} die \singlequote{Ehe}
zwischen historischer und philosophischer Erkenntnis
heiligen.\footnote{\cite[Vgl.][\S~12]{Wolff:Discursuspraeliminarisdephilosophiaingenere1996}:
\enquote{\dots\unkern{}nobis per omnem philosophiam sanctum est utriusque
connubium.}} Denn beide stehen nicht unverbunden nebeneinander, sondern
philosophische Erkenntnis basiert auf historischer Erkenntnis und \emph{ergänzt}
dies bei dem gebildeten Teil der Menschheit.

Ich habe in Kapitel \ref{subsection:DescartesKritikantestimonialemWissen}
behauptet, dass \authorcite{Descartes:OeuvresdeDescartes1983}' Unterscheidung
von Wissenschaft (\emph{scientia}) und historischer Kenntnis (\emph{historia})
für die deutsche Aufklärungsphilosophie von Bedeutung sein werde und gerade für
die Frage nach einem mündigen Umgang mit Informationsquellen wichtig werde.
\authorcite{Wolff:Discursuspraeliminarisdephilosophiaingenere1996} liefert uns
eine erste Explikation des Begriffs der \singlequote{historischen Kenntnis}.
Doch was gewinnen wir für die Frage nach einem mündigen Umgang mit
testimonialem Wissen? Was wird aus
\authorcite{Descartes:OeuvresdeDescartes1983}' Kritik der
\singlequote{Büchergelehrsamkeit}? Ich vertrete im folgenden die Auffassung,
dass die Entgegensetzung von \emph{scientia} und \emph{historia} bei
\authorcite{Wolff:Discursuspraeliminarisdephilosophiaingenere1996} -- wie später
bei \name[Immanuel]{Kant} -- als Differenz von \emph{philosophia} und
\emph{cognitio(nes) historica(e) philosophiae} rekonstruiert wird.

\authorcite{Descartes:OeuvresdeDescartes1983} sagt, durch das Lesen von Büchern
erwürben wir keine Wissenschaft, sondern bloß historische Kenntnisse. Doch wie
ist dies zu verstehen? Jeder weiß, dass bei Gewitter der Donner mit einiger
Verzögerung auf den Blitz folgt. Es handelt sich um eine einfache historische
Erkenntnis. Die Wissenschaft erklärt dies nun dadurch, dass die
Schallgeschwindigkeit deutlich geringer ist als die Lichtgeschwindigkeit, und
bewahrt dieses philosophische Wissen in Büchern auf. Wenn ich nun ein solches
Buch aufschlage und lese, dass der Blitz dem Donner vorausgeht, weil die
Lichtgeschwindigkeit größer als die Schallgeschwindigkeit ist, dann -- so sollte
man doch annehmen, erwerbe ich eben jenes philosophische Wissen und damit einen
Teil jener Wissenschaft. Doch diese Annahme ist vorschnell. Wie
\authorcite{Wolff:Discursuspraeliminarisdephilosophiaingenere1996} darlegt,
erwerbe ich auf diese Weise möglicherweise gar keine philosophische Erkenntnis,
sondern historische Kenntnis der Philosophie.

Nach \authorcite{Wolff:Discursuspraeliminarisdephilosophiaingenere1996} können
wir sowohl von der philosophischen Erkenntnis anderer als auch von der
Philosophie historische Erkenntnis
haben.\footnote{\cite[Vgl.][\S\S~8,
50]{Wolff:Discursuspraeliminarisdephilosophiaingenere1996}.} Dass jemand eine
philosophische Erkenntnis hat, ist eine Tatsache.
Und wenn ich weiß, dass etwa \name[Isaac]{Newton} das Fallen schwerer Gegenstände
über die wechselseitige Anziehung von Körpern erklärte, dann habe ich historische Erkenntnis von dieser
philosophischen Erkenntnis (\enquote{\emph{cognitionis philosophicae cognitio
historica}}) \name[Isaac]{Newton}s. Und es ist ebenso eine Tatsache, dass das
Folgen von Donner auf Blitz in der Philosophie durch die unterschiedlichen
Geschwindigkeiten von Licht und Schall erklärt wird. Wer das
weiß, der hat also eine historische Erkenntnis der Philosophie
(\enquote{\emph{cognitio historica philosophiae}}). Jede Einführung in eine
Wissenschaft nimmt ihren Weg anfänglich über solche Erkenntnisse und auch der
fertig ausgebildete Wissenschaftler kommt ohne solche Erkenntnisse nicht aus,
insofern er die Forschungsergebnisse seiner Kollegen rezipiert oder sich in
einen neuen Themenbereich einarbeitet.

Zu den bekanntesten Aspekten von \authorcite{Wolff:Psychologiaempirica1968}s
Philosophie gehört seine Warnung vor einer bloßen \emph{historica
cognitio cognitionis
philosophicae} oder \emph{philosophiae}
\footnote{\cite[Vgl.][\S\S~9,
51]{Wolff:Discursuspraeliminarisdephilosophiaingenere1996}.}, die von
\name[Immanuel]{Kant} gerade in Anspielung an die Schulbildung im Gefolge
\authorcite{Wolff:Discursuspraeliminarisdephilosophiaingenere1996}s
aufgegriffen\footnote{\cite[Vgl.][61--3]{Hinske:ZwischenAufklaerungundVernunftkritik1993}.
\enquote{Es gibt kaum einen Gedanken \authorcite{Wolff:Discursuspraeliminarisdephilosophiaingenere1996}s, den Kant mit größerer Zustimmung
und Leidenschaft aufgenommen und weitergedacht hätte, als \authorcite{Wolff:Discursuspraeliminarisdephilosophiaingenere1996}s Warnung vor
einer bloßen cognitio philosophiae historica, durch die der ursprüngliche
Charakter der philosophischen Erkenntnis in sein blankes Gegenteil verkehrt
wird} \parencite[][62]{Hinske:ZwischenAufklaerungundVernunftkritik1993}.
Siehe \cite[][B~864]{Kant:KritikderreinenVernunft2003}, \cite[][III:
540.37--541.12]{Kant:GesammelteWerke1900ff.}: \enquote{Daher hat der, welcher
ein System der Philosophie, z.\,B.\ das \ori{wolffische}, eigentlich
\ori{gelernt} hat, \punkt\ doch keine andere als vollständige \ori{historische}
Erkenntnis der wolffischen Philosophie\punkt\ Er hat gut gefaßt und behalten,
d.\,i.\ gelernet, und ist ein Gipsabdruck von einem lebenden Menschen.}} und als
zentrales Element in seiner Aufklärungskonzeption integriert
wird. Doch wie sollen wir diese Warnung verstehen?
Denn angenommen, wir haben uns die Inhalte eines Lehrbuchs angeeignet. Besitzen
wir dann nicht philosophische Erkenntnisse? Wir wissen doch dann, wie man die
Phänomene des entsprechenden Teils der Philosophie erklärt. Wenn Ingrid Max
sagt, dass Holz schwimmt, weil seine Dichte geringer ist als die des Wassers,
dann teil sie ihm doch philosophische Erkenntnis mit. Max kann daraufhin auf die
Frage, warum Holz schwimmt, sagen: \enquote{Nun, es schwimmt, weil seine Dichte
geringer ist als die des Wassers. Wäre sie größer, ginge es unter.} Er scheint
nun auch über \wolffsprech{philosophisches} Wissen zu verfügen und in dieser (wenn auch
überschaubar großen) Hinsicht \wolffsprech{Philosoph} zu sein.

Es ließe sich darauf verweisen, dass die Definition der Philosophie sich auf
einen Begriff von Wissenschaft als \emph{genus} stützte, der diese nicht als
Inbegriff gewisser Lehren (\emph{obiective considerata}), sondern als
methodische Kompetenz des Subjekts (\emph{subiective considerata}) bestimmt. Wer
eine Wissenschaft nur durch das Lesen von Büchern erwirbt, der erwirbt damit
nicht die entsprechenden Fähigkeiten. Und diese (inzwischen auch von
\authorcite{Hobbes:Leviathan1962} bekannte)\footnote{Siehe Kapitel
\ref{subsection:VernunftwahrheitenUndErfahrungstatsachen}.} Strategie ist durchaus korrekt.
Doch um welche Kompetenzen geht es hier?


Es wäre zu kurz gegriffen, nur auf eine allgemeine Methodik zu verweisen und zu
sagen, man müsse die mathematische Methode
(\authorcite{Wolff:Discursuspraeliminarisdephilosophiaingenere1996})
beherrschen oder analog diejenige Methode, die führende
Wissenschaftstheoretiker als \singlequote{die} Methode der Wissenschaften
auszumachen glauben. Die Beherrschung allgemeiner Methoden wissenschaftlichen
Arbeitens ist \emph{ein Teil} der Kompetenzen, die wir erwerben müssen, um über
eine Wissenschaft zu verfügen, es ist nicht das Gesamt der Kompetenzen, zu dem
nur noch die Inhalte hinzugefügt werden müssen. Schließlich würde diese Deutung
zwar erklären, warum Max zwar nur eine historische Kenntnis der
\wolffsprech{Philosophie} erwirbt, wenn er sich Ingrids Erklärung merkt. Aber
sie erklärte nicht, warum er bloß historische Kenntnis der philosophischen
Erkenntnis Ingrids erwirbt und nicht einfach philosophische Erkenntnis; denn in
der Definition der \emph{cognitio philosophica} war von methodischer Strenge und
Wissenschaft keine Rede.

Wenn Max von Ingrid erfährt, dass Holz wegen seiner geringen Dichte schwimmt,
dann -- so behauptet
\authorcite{Wolff:Discursuspraeliminarisdephilosophiaingenere1996} -- erwirbt er
möglicherweise nur historische Kenntnis der Philosophie, aber kein
philosophisches Wissen, weil ihm eine bestimmte \emph{Kompetenz} fehlt. Und
\authorcite{Wolff:Discursuspraeliminarisdephilosophiaingenere1996} bestimmt
auch, welche Kompetenz ihm in diesem Fall fehlt: Max kann zwar den Grund dafür,
dass Holz schwimmt, \emph{benennen}. Um über philosophische Erkenntnis zu
verfügen, müsse man darüber hinaus \emph{beweisen} können, dass dies der Grund
ist: \enquote{{Wenn einer nicht beweisen kann, daß der von einem anderen
angeführte Grund einer Tatsache wirklich ihr Grund ist, hat er keine
philosophische Erkenntnis dieser
Tatsache.}}\footnote{\enquote{{Si quis demonstrare non noverit, rationem facti ab altero
allegatam esse ejus rationem, is cognitione philosophica ejusdem facti
destituitur}} \parencite[][\S~9,
\ohio]{Wolff:Discursuspraeliminarisdephilosophiaingenere1996}.}


Ingrid und Max wissen beide, dass Holz schwimmt, weil seine Dichte geringer ist
als die des Wassers, aber im Gegensatz zu Ingrid besitzt Max nur historische
Kenntnis dieses Sachverhalts. Dies liegt nicht daran, dass Max dieses
Wissen nur aus zweiter Hand hat und Ingrid dies selbst festgestellt hätte.
Ingrid mag dies etwa aus einem Lehrbuch der Physik und damit selbst nur aus
zweiter Hand wissen. Es muss möglich sein, \wolffsprech{philosophisches} Wissen
aus Lehrbüchern zu erwerben, wenn
\authorcite{Wolff:Discursuspraeliminarisdephilosophiaingenere1996}s Darstellung
kompatibel sein soll mit der epistemischen Situation, in der wir uns notwendiger
Weise befinden. Denn dazu gehört, dass wissenschaftliche Ausbildung über
Lehrbücher und Vorlesungen vermittelt wird und wir Vieles von Anderen lernen.
Sähen wir dies stets als bloß historische Kenntnis, könnten wir niemanden als
\wolffsprech{Philosophen} in irgendeiner Disziplin bezeichnen. Wir wären
epistemische Individualisten wie \authorcite{Descartes:OeuvresdeDescartes1983}
und diese Position ist -- so war in Kapitel \ref{section:autonomieunddaszeugnisanderer} zu sehen -- in
keiner Weise befriedigend. Wir müssen davon ausgehen, dass es prinzipiell
möglich ist, philosophisches Wissen durch das Lesen von Büchern oder Auskünfte
von anderen zu erwerben, wenn uns die Unterscheidung von philosophischer und
historischer Erkenntnis in unserem Bestreben, einen Begriff mündigen
Wissenserwerbs zu erarbeiten, irgendwie weiterhelfen soll.

Es ist zunächst instruktiv zu sehen, dass auch aus Erfahrung nichts als
historische Erkenntnis entspringen
kann.\footnote{\enquote{Quae per experientiam stabiliuntur, eorum nonnisi
historica est cognitio} \parencite[][\S~10]{Wolff:Discursuspraeliminarisdephilosophiaingenere1996}.}
Aus Erfahrung wissen wir also, \emph{was} ist und geschieht, jedoch nicht,
\emph{warum} es ist oder geschieht. Doch auch dies ist zunächst
schwer zu verstehen, denn Naturwissenschaften erkennen doch, warum sich die Dinge so
verhalten, wie sie sich nun einmal verhalten, indem sie mit Experimenten und
Beobachtungen -- also \emph{empirisch} -- herausfinden.


Eine wohlfeil zu habende Antwort läge darin zu vermuten, philosophische
Erkenntnis ginge über historische Erkenntnis nach
\authorcite{Wolff:Discursuspraeliminarisdephilosophiaingenere1996} hinaus, indem
sie das, was wir in der historischen Erkenntnis aus Erfahrung wissen, aus
\singlequote{reiner Vernunft} erweist. Wir interpretierten ihn dann als
denjenigen Rationalisten, der die Erfahrung nicht als gültige Grundlage echter
Wissenschaft zu akzeptieren bereit wäre. Jedoch ist
\authorcite{Wolff:Discursuspraeliminarisdephilosophiaingenere1996} kein
Rationalist in dem genannten Sinne.\footnote{Siehe dazu ausführlich
\cite{Kreimendahl:EmpiristischeElementeimDenkenChristianWolffs2007}.
\authorfullcite{Ecole:Enquelssenspeut-ondirequeWolffestrationaliste?1979}
liefert eine sehr differenzierte Darstellung, die rationalistische Elemente im
Denken \authorcite{Wolff:Discursuspraeliminarisdephilosophiaingenere1996}s
abseits des naiven Rationalismusverständnisses beschreibt
\parencite[vgl.][]{Ecole:Enquelssenspeut-ondirequeWolffestrationaliste?1979}.}
Besonders deutlich wird dies dort, wo er die \enquote{\emph{psychologia
empirica}}, die nach ihm wie nach
\authorcite{Baumgarten:Metaphysica---Metaphysik2011} zur Metaphysik
gehört,\footnote{\cite[Vgl.][\S~99]{Wolff:Discursuspraeliminarisdephilosophiaingenere1996},
wo die Psychologie schlechthin (und \emph{a fortiori} auch die empirische
Psychologie) zur Metaphysik gezählt wird;
\cite[][\S\S~501--503]{Baumgarten:Metaphysica---Metaphysik2011}.} sowie die
\enquote{\emph{psychologia rationalis}} einführt und ihr Verhältnis zueinander
bestimmt. Denn von der \emph{rationalen} Psychologie sagt er:
\begin{quote}
  Da dies ein neues und der vorgefaßten Meinung entgegengesetztes
  Unternehmen ist, neue Dinge aber anfangs von den meisten ungern zugestanden
  werden, war der wichtigste Grund, warum ich die rationale Psychologie von
  der empirischen trennte, dass sie nicht ohne Unterschied zurückgewiesen
  würden. {\punkt} Die praktische Philosophie ist von größter Wichtigkeit:
  welche Dinge aber von größter Bedeutung sind, wollten wir nicht auf solche
  Grundsätze aufbauen, die zu Gegenständen einer Streitfrage werden können. Aus
  diesem Grund bauen wir die Wahrheiten der praktischen Philosophie nur auf
  Grundsätzen auf, die durch die Erfahrung in der empirischen Psychologie evident gesichert
  werden.\footnote{\cite[][\S~112]{Wolff:Discursuspraeliminarisdephilosophiaingenere1996}:
  \enquote{Novus cum sit ausus {\&} praejudicatae opinioni adversus, nova vero
  ab initio a plerisque aegre admittantur; praegnans maxime ratio fuit, cur
  Psychologiam rationalem ab empiricam discernerem, ne psychologica promiscue
  rejicerentur. {\punkt} Philosophia practica est maximi momenti: quae igitur
  maximi sunt momenti, istiusmodi principiis superstruere noluimus, quae in
  disceptationem vocantur. Ea de causa veritates philosophiae practicae non
  superstruimus nisi principiis, quae per experientiam in Psychologia empirica
  evidenter stabiliuntur.}}
\end{quote}
Die historische Erkenntnis liefert dadurch, dass sie der Erfahrung
entspringt, \enquote{eine feste und unerschütterliche
Grundlage.}\footnote{\cite[][\S~11]{Wolff:Discursuspraeliminarisdephilosophiaingenere1996}:
\enquote{\dots\unkern firmo ac inconcusso nititur fundamento.}} Gerade deswegen
sollen wir nach \authorcite{Wolff:Psychologiaempirica1968} die \singlequote{Ehe}
zwischen historischer und philosophischer Erkenntnis
heiligen.\footnote{\cite[Vgl.][\S~12]{Wolff:Discursuspraeliminarisdephilosophiaingenere1996}:
\enquote{\dots\unkern{}nobis per omnem philosophiam sanctum est utriusque
connubium.}} Denn beide stehen nicht unverbunden nebeneinander, sondern
philosophische Erkenntnis basiert auf historischer Erkenntnis.
\enquote{\ori{Wenn durch Erfahrung dasjenige festgestellt wird, woraus sich
anderes, was ist und geschieht oder geschehen kann, begründen lässt, liefert die
historische Erkenntnis die Grundlage der
philosophischen.}}\footnote{\Cite[][\S~10]{Wolff:Discursuspraeliminarisdephilosophiaingenere1996}:
\enquote{\ori{Si per experientiam stabiliuntur ea, ex quibus aliorum, quae sunt
atque fiunt, vel fieri possunt, ratio reddi potest, cognitio historica
philosophicae fundamentum praebet.}}}  Generell könne uns zwar Erfahrung nur
historische, aber keine philosophische Erkenntnis liefern. Doch diese
historische Erkenntnis \emph{fundiere} die philosophische
Erkenntnis.\footnote{\enquote{Quae per experientiam stabiliuntur, eorum nonnisi historica est
cognitio (\S~3). Quodsi ergo ex iis rationem reddis aliorum, quae sunt atque
fiunt, cognitionem philosophicam iisdem superstruis (\S~6)}
\parencite[][\S~10]{Wolff:Discursuspraeliminarisdephilosophiaingenere1996}.}


Was zur historischen Erkenntnis hinzukommen muss, damit philosophische
Erkenntnis vorliegt, ist die Einsicht in den \emph{Zusammenhang} von
\emph{explanans} und \emph{explanandum}. Und dieser Zusammenhang lässt sich
durch die genannten Sätze nicht mitteilen; um um ihn zu wissen -- um über
philosophische Erkenntnis zu verfügen --, müssen wir die \emph{Kompetenz}
besitzen, diesen Zusammenhang -- wie
\authorcite{Wolff:Psychologiaempirica1968} sagt -- zu \singlequote{beweisen}
oder -- etwas abseits von \authorcite{Wolff:Psychologiaempirica1968}s Sichtweise
-- zu erläutern. Max und Ingrid können beide sagen, dass Holz wegen seiner
geringen Dichte schwimmt (und dass alle Körper mit einer entsprechend geringen Dichte
schwimmen). Aber wenn wir Max fragen, warum Körper mit einer so geringen Dichte
schwimmen, wird er dies vielleicht nicht beantworten, sondern nur auf Ingrid
verweisen können. Ingrid wiederum wird in der Lage sein zu erläutern, wie es
kommt, dass Körper mit einer geringen Dichte schwimmen. Denn wir nahmen an, dass Ingrid philosophisches
Wissen besitzt. Insofern Max nicht in der Lage ist,
den Zusammenhang zu erläutern, besitzt er keine eigene philosophische
Erkenntnis, sondern historische Erkenntnis der philosophischen Erkenntnis
(\emph{cognitio cognitionis philosophicae historica})
Ingrids\footcite[Vgl.][\S~8]{Wolff:Discursuspraeliminarisdephilosophiaingenere1996}
oder historische Erkenntnis der
Philosophie\footcite[Vgl.][\S~50]{Wolff:Discursuspraeliminarisdephilosophiaingenere1996}.
Beide -- Max und Ingrid -- haben ihr Wissen, warum Holz schwimmt, aus zweiter
Hand. Aber während Ingrid philosophisches Wissen erwarb, bleibt das Wissen bei
Max bloß historisch. Der Unterschied besteht schlicht darin, dass Ingrid den
Zusammenhang von \emph{explanans} und \emph{explanandum} verstanden hat und
erläutern kann, während Max nur die Angabe des Grundes zu wiederholen vermag.



\authorcite{Wolff:Discursuspraeliminarisdephilosophiaingenere1996} beabsichtigt
nicht die Abwertung historischer Erkenntnis der Philosophie, sondern die
begriffliche Differenzierung zwischen historischer und eigener philosophischer
Erkenntnis. Dabei fehlt die eigene philosophische Erkenntnis, solange jemand
nicht zeigen kann, dass der angeführte Grund wirklich die geforderte
Erklärungsleistung erbringen
kann.\footnote{\cite[Vgl.][\S~9]{Wolff:Discursuspraeliminarisdephilosophiaingenere1996}:
\enquote{Si quis demonstrare non noverit, rationem facti ab altero allegatam
esse ejus rationem, is cognitione philosophica ejusdem facti destituitur}
(\ohio ).} Wem diese Erkenntnis fehlt, der hat eine \emph{bloß} historische
Erkenntnis philosophischer Erkenntnis. Nach
\authorcite{Wolff:Psychologiaempirica1968} erkennt man den Mangeln philosophischer Erkenntnis an
der Unfähigkeit zu zeigen, dass der angeführte Grund zureichend
ist.\footnote{\cite[Vgl.][\S~9]{Wolff:Discursuspraeliminarisdephilosophiaingenere1996}:
\enquote{Si quis demonstrare non noverit, rationem facti ab altero allegatam
esse ejus rationem, is cognitione philosophica ejusdem facti destituitur.} Das
muss nicht heißen, dass er -- etwa bei physikalischem Wissen -- die nötigen
Ausgangsdaten selbst ermittelt, etwa selbst Experimente durchführt oder Studien
erstellt. Diese Ausgangsdaten sind historische Kenntnisse und können tradiert
werden. Ansonsten könnte niemand je ein Naturgesetz verstehen, ohne selbst
entsprechende Experimente durchzuführen, was mitunter zu absurden Konsequenzen
führte. (Man denke an Statistiken in den Sozialwissenschaften, Beobachtungen in
der physikalischen Kosmologie mit ihrem notorisch hohen technischen Aufwand oder
Fossilienfunde.) Dass die angeführten Gründe selbst wieder nur historisch
gewusst werde, ist kein Mangel, denn es handelt sich um Tatsachen. Aber ihr
Nexus mit dem, was sie erklären, ist Gegenstand philosophischen Wissens
\parencite[vgl.][\S~10]{Wolff:Discursuspraeliminarisdephilosophiaingenere1996}.}
Man muss den Zusammenhang erläutern können. Wer etwas versteht, der besitzt eine Reihe
von Fähigkeiten, zu denen auch, aber nicht nur die Artikulation des
Verstandenen gehört. Er hat vielmehr selbst Kompetenzen eines Naturforschers.
Und darum ist nicht derjenige aufgeklärt, der viel gelernt und behalten hat, was
er verbal reproduzieren kann, sondern derjenige, der viel verstanden und
entsprechende Kompetenzen erworben hat. Aufklärung in der
wolffschen Ausprägung geht daher darauf, die Welt um uns herum
nicht nur zu kennen, sondern zu verstehen, und stellt daher den Naturforscher
mit seinen Kompetenzen in den Mittelpunkt der Aufklärungsbemühungen.



\authorcite{Wolff:Discursuspraeliminarisdephilosophiaingenere1996} ist bewusst,
dass jeder Einzelne mit der Aufgabe, in allen Bereichen philosophisches Wissen zu haben, überfordert
wäre.\footnote{\cite[Vgl.][\S~48]{Wolff:Discursuspraeliminarisdephilosophiaingenere1996}:
\enquote{\ori{Nemo in omnibus philosophus}}.} Deswegen ist es nicht ehrenrührig,
in vielen Dingen bloß historische Erkenntnis der Weltweisheit zu haben. Wir sind
damit noch immer gebildeter als ohne jedes Wissen (wenngleich wir uns
eingestehen müssen, dass uns in vielen Fällen selbst das historische Wissen um
die Weltweisheit fehlt.)
Wer historische Kenntnis der Philosophie hat, hat in der Anwendung oft dieselben
Vorteile wie jemand, der tatsächlich philosophisches Wissen hat. Nachdem Ingrid
ihm sagte, warum Holz schwimmt und Steine untergehen, kann Max seine
historischen Kenntnisse genauso anwenden wie Ingrid ihr philosophisches Wissen:
Auch er wird sein Floß nicht aus Azob{\'e} bauen und seine Sachen am Strand
nicht mit Bimsstein beschweren. Und er weiß, welche Gegenstände denselben Dienst
verrichten können, wenn kein Holz oder keine Steine zur Hand sind.

Was Ingrid im Gegensatz zu Max vermag ist, philosophische Behauptungen zu
beurteilen.\footnote{\cite[Vgl.][\S~52]{Wolff:Discursuspraeliminarisdephilosophiaingenere1996}.}
Max kann Ingrids Behauptung über den Grund, warum Holz schwimmt, zwar benennen
und auch anderen mitteilen, aber er vermag auf Einwände gegen die Erklärung
nicht zu reagieren. Und wenn Ingrid ihm eine falsche Erklärung genannt hätte,
könnte er dies nicht bemerken. Ingrid wiederum kann beurteilen, welche
Eigenschaften des Holzes die Tatsache erklären, dass es schwimmt, und welche
dies nicht tun. Mündigkeit und Selbstdenken sind Forderungen an denjenigen, der
philosophisches Wissen zu haben beansprucht. Bei historischen Erkenntnissen
wären sie völlig fehl am Platze. Über philosophisches Wissen \emph{kann} aber
nur derjenige verfügen, der methodisch ausgebildet ist. Daraus ergibt sich
\authorcite{Wolff:Discursuspraeliminarisdephilosophiaingenere1996}s Behauptung,
dass Selbstdenken keine Frage des Entschlusses, sondern der Kompetenz ist, die
sich nur auf der Grundlage einer methodischen Schulung entwickeln lasse (siehe
Kapitel \ref{Abschnitt:WolffunddieWissenschaftlichkeitderPhilosophiemoregeometrico}).

Wer sein Wissen lediglich rezipierend aus Büchern aufnimmt, erlangt keine
Wissenschaft, sondern lediglich historische Kenntnis von Wissenschaften. Dies
liegt nicht daran, dass er die Wahrheit der Aussagen nicht selbst kontrolliert hat und diese
daher fraglich wäre -- es gibt kaum bessere Möglichkeiten, die Wahrheit von
Aussagen zu überprüfen, als in einschlägigen wissenschaftlichen Publikationen
nachzuschlagen --, sondern weil er damit noch nicht die entsprechenden
Kompetenzen erworben hat, den Zusammenhang der Wahrheiten zu beurteilen. Das
\singlequote{gemeine Volk} weiß um Ergebnisse der Wissenschaften und es hat --
auch in einem strengen Sinne -- \emph{Wissen}, aber es verfügt nicht über
Wissenschaft, weil es seine Erkenntnisse nicht entsprechend systematisiert. Und
so ist die Forderung nach einem \emph{System} der Wissenschaften zentral für das
entsprechende Aufklärungsverständnis. Denn das, was ist oder geschieht, in
seinem systematischen Zusammenhang darzustellen, heißt nichts anderes als zu
verstehen, warum es ist oder geschieht.

\subsection{Exkurs: Historische Erkenntnis allgemeiner
Tatsachen}\label{subsubsection:HistorischeErkenntnisallgemeinerTatsachen}
Bei den bisherigen Überlegungen bin ich davon ausgegangen, dass
historische Erkenntnis sich auch in allgemeinen Urteilen wie \enquote{\emph{Alle} Steine
sind schwer} äußern kann. Immerhin führt
\authorcite{Wolff:Psychologiaempirica1968} gerade solche Urteile als Beispiele
historischer Erkenntnisse an. Das Vermögen der philosophischen Erkenntnis, die
nicht einzelne Wahrheiten, sondern den Zusammenhang von Wahrheiten thematisiert, ist die Vernunft. Die
\emph{ratio} -- sagt \authorcite{Wolff:Psychologiaempirica1968} entsprechend -- ist das Vermögen, den
\emph{Zusammenhang} allgemeiner Wahrheiten (\emph{nexus veritatum universalium})
zu
erkennen.\footnote{\phantomsection\label{Anmerkung:Wolff:VernunftalsEinsichtinZusammenhangallgemeinerWahrheiten}\cite[Vgl.][\S~483]{Wolff:Psychologiaempirica1968}:
\enquote{\ori{Ratio} est facultas nexus veritatum universalium intuendi seu
perspiciendi.} Siehe auch
\cite[][\S~368]{Wolff:VernuenftigeGedanckenvonGottderWeltundderSeeledesMenschenauchallenDingenueberhauptDeutscheMetaphysik1983}:
\enquote{Die Einsicht, so wir in den Zusammenhang der Wahrheiten haben, oder
das Vermögen den Zusammenhang der Wahrheiten einzusehen, heisset
\ori{Vernunft}.}} Die allgemeinen
Wahrheiten selbst sind Gegenstand historischer Erkenntnis und können durch
eigene Beobachtung und Erfahrung oder durch Mitteilung von anderen erworben
werden.

Die Vernunft, der wir die philosophische Erkenntnis zuordnen können, ist auf der
anderen Seite bei \authorcite{Wolff:Psychologiaempirica1968} auch nicht einfach das Vermögen des Schließens oder des
Verbindens von Wahrheiten schlechthin, sondern des Verbindens
\emph{allgemeiner} Wahrheiten\footnote{Siehe oben, Anmerkung
\ref{Anmerkung:Wolff:VernunftalsEinsichtinZusammenhangallgemeinerWahrheiten},
Seite
\pageref{Anmerkung:Wolff:VernunftalsEinsichtinZusammenhangallgemeinerWahrheiten}.},
die als solche der Erfahrung entstammen. Nach
\authorcite{Meier:Vernunftlehre1752}, der hierin \authorcite{Wolff:Psychologiaempirica1968} folgt,
ist die Erkenntnis, dass \emph{alle} Menschen irren können, für sich noch keine
rationale, sondern eine historische Erkenntnis. Sie wird zur cognitio
rationalis, wenn sie \enquote{auf eine deutliche Art aus Gründen} erkannt
wird.\footnote{\Cite[][\S~17]{Meier:AuszugausderVernunftlehre1752},
\cite[][XVI: 93.17--18]{Kant:GesammelteWerke1900ff.}.} Die Verständlichkeit der
Welt ist nicht schon darin gegeben, dass alles, was geschieht, nach Regeln (allgemeinen Naturgesetzen) geschieht, sondern darin, dass diese allgemeinen
Gesetze selbst wiederum untereinander verbunden und auseinander erkennbar
sind.\footnote{\cite[Vgl.][\S~482]{Wolff:Psychologiaempirica1968}:
\enquote{\ori{Veritates universales inter se connectuntur.}} Und in der Anmerkung
dazu heißt es: \enquote{Veritatum universalium nexus fundatur in nexu rerum, quem in
Cosmologia stabilivimus \punkt\ quemadmodum veritas logica in veritate rerum
transcendentali}.} Hieran knüpft dann auch \name[Immanuel]{Kant} an, wenn er in der \titel{Kritik der
Urteilskraft} das transzendentale Prinzip der reflektierenden Urteilskraft als
eines beschreibt, das \enquote{die Einheit aller empirischen Prinzipien unter
gleichfalls empirischen, aber höheren Prinzipien und also die Möglichkeit der
systematischen Unterordnung derselben untereinander begründen soll.} Erst durch
eine solche Kombination mehrerer \emph{allgemeiner} Erkenntnisse entsteht
\enquote{ein System der Erfahrung nach besonderen
Naturgesetzen}\footnote{\cite[][B xxvii]{Kant:KritikderUrteilskraft2009},
\cite[][V: 180.8--11, 24--25]{Kant:GesammelteWerke1900ff.}.}. Und dies
entspricht dann wiederum dem regulativen Prinzip der systematischen Einheit, wie
es im Anhang zur transzendentalen Dialektik beschrieben und noch der Vernunft,
nicht der Urteilskraft zugeschrieben wird.\footnote{\cite[Vgl.][B
708]{Kant:KritikderreinenVernunft2003}, \cite[][III:
448.22--30]{Kant:GesammelteWerke1900ff.}: \enquote{Die reine Vernunft ist in der
Tat mit nichts als sich selbst beschäftigt, und kann auch kein anderes Geschäfte haben, weil ihr nicht
die Gegenstände zur Einheit des Erfahrungsbegriffs, sondern die
Verstandeserkenntnisse zur Einheit des Vernunftbegriffs, d.\,i.\ des
Zusammenhanges in einem Prinzip gegeben werden. Die Vernunfteinheit ist die
Einheit des Systems, und diese systematische Einheit dient der Vernunft nicht
objektiv zu einem Grundsatze, um sie über die Gegenstände, sondern subjektiv als
Maxime, um sie über alles mögliche empirische Erkenntnis der Gegenstände zu
verbreiten.}} Nicht allgemeine Gesetze schlechthin sind Ausdruck der Vernunft (im engeren
Sinne), sondern die Rückführung verschiedener Naturgesetze auf höhere und ihre
Verbindung in einem einheitlichen System. Es ist genau dies, was -- wie
\authorcite{Wolff:Psychologiaempirica1968} behauptet -- das \distanz{gemeine Volk} (\enquote{vulgus}) nicht
besitzt, welches zwar wisse, dass das Wasser auf dem Feuer siedet, aber nicht,
warum dies so
ist.\footnote{\cite[Vgl.][\S~23]{Wolff:Discursuspraeliminarisdephilosophiaingenere1996}.}
Behauptete er, dass gemeine Volk verfüge nicht über allgemeine Wahrheiten, weil
ihm nur historische Erkenntnisse zugänglich sind, könnten wir dies kaum ernst
nehmen.



Doch gegen diese Interpretation argumentieren unabhängig voneinander
\authorfullcite{Kambartel:ErfahrungundStruktur1968} und
\authorfullcite{Holzhey:KantsErfahrungsbegriff1970} jeweils unter Rückgriff auf
eine Stelle aus der Deutschen
Logik.\footnote{\cite[Vgl.][\pno~57\,f.,]{Kambartel:ErfahrungundStruktur1968}
und \cite[][91]{Holzhey:KantsErfahrungsbegriff1970}.} Diese handelt zwar \emph{prima
facie} nicht von historischer, sondern von Erfahrungserkenntnis; aber da
nach \authorcite{Wolff:Psychologiaempirica1968} der Begriff \enquote{historische
Erkenntnis} mit dem Begriff \enquote{Erkenntnis aus Erfahrung} zumindest
extensionsgleich ist, ist sie für den Begriff historischer Erkenntnis durchaus
einschlägig. Und sie ist hier von einigem Interesse, weil sich die entsprechende
Deutung auf das Verständnis der Konzeption \name[Immanuel]{Kant}s auswirkt
(auch \authorcite{Kambartel:ErfahrungundStruktur1968} und
\authorcite{Holzhey:KantsErfahrungsbegriff1970} geht es letztlich um eine
Interpretation \name[Immanuel]{Kant}s, nicht
\authorcite{Wolff:Psychologiaempirica1968}s).
\authorcite{Wolff:Psychologiaempirica1968} schreibt in der
\singlequote{Deutschen Logik} im Kapitel \titel{Von der Erfahrung, und wie
dadurch die Sätze gefunden werden}:
\begin{quote}\phantomsection\label{Zitat:Wolff:ErfahrungnurvoneinzelnenDingen}
 Da wir nun nichts als eintzele Dinge empfinden; so sind auch die Erfahrungen
nichts als Sätze von eintzelen Dingen. Derowegen wer sich auf die Erfahrung
geruffet, ist billig gehalten, einen besonderen Fall anzuführen, es sey denn,
daß die Erfahrung so beschaffen ist, daß sie entweder ein jeder gleich haben
kan, wenn er sie verlanget, oder wenigstens sich bald darauf zu besinnen weiß,
weil er sie vorhin selbst mehr als einmal gehabt.\footnote{\cite[][Capitel 5,
\S~2]{Wolff:VernuenftigeGedankenvondenKraeftendesmenschlichenVerstandesundihremrichtigenGebraucheinErkenntnisderWahrheit1978}.
Parallel heißt es in \authorcite{Wolff:Psychologiaempirica1968}s lateinischer
Übersetzung: \enquote{Quoniam nonnisi singularia percipimus; experientia
singularium est seu propositionibus singularibus constat, quarum subjectum est
individuum quoddam} \parencite[][\S~119]{Wolff:Cogitationesrationalesdeviribusintellectushumani1983}.
Siehe dazu auch
\cite[][\S\S~664--667]{Wolff:PhilosophiarationalissiveLogica1740}.
\enquote{Quoniam nonnisi singularia percipimus, \ori{experientia singularium
est}} \parencite[][\S~665]{Wolff:PhilosophiarationalissiveLogica1740}.}
\end{quote}
\begin{comment}
Empfindungen werden von \authorcite{Wolff:Psychologiaempirica1968} nicht mit Erfahrungen identifiziert,
weswegen der Zusammenhang nicht ganz so klar ist, wie es zunächst scheint.
\enquote{Wir erfahren alles dasjenige, was wir erkennen, wenn wir auf unsere
Empfindungen acht haben.}\footnote{\cite[][5. Capitel,
\S~1]{Wolff:VernuenftigeGedankenvondenKraeftendesmenschlichenVerstandesundihremrichtigenGebraucheinErkenntnisderWahrheit1978}.
Siehe auch die Parallelstelle in der Lateinischen Logik, wo der Begriff der
Erfahrung (\enquote{experientia}) folgendermaßen eingeführt wird:
\enquote{\ori{Experiri} dicimur, quicquid ad perceptiones nostras attenti
cognoscimus. Ipsa vero horum cognitio, qu\ae{} sola attentione ad perceptiones
nostras patent,
\ori{experientia} vocatur}
\parencite[][\S~664]{Wolff:PhilosophiarationalissiveLogica1740}.} Natürlich
empfinden wir nicht, dass die Flamme das Papier anzündet, aber wir erfahren es,
wenn wir auf die entsprechenden Empfindungen achten (und nicht etwa auf den
Begriff der Flamme oder des Papiers), d.\,h.\ in kantischer Terminologie wenn
wir unser rezeptives (sinnliches) Erkenntnisvermögen gebrauchen. Erfahren und
Empfindungen haben ist nicht dasselbe, schon weil wir Dinge
empfinden\footnote{\cite[Vgl.][1. Capitel,
\S~1]{Wolff:VernuenftigeGedankenvondenKraeftendesmenschlichenVerstandesundihremrichtigenGebraucheinErkenntnisderWahrheit1978}:
\enquote{Ein jeder nimmet bey sich selbst wahr, daß er viele Dinge empfindet.
Ich sage aber, daß wir etwas empfinden, wenn wir uns desselben als uns
gegenwärtig bewust sind. So empfinden wir den Schmertz, den Schall, das Licht
und unsere eigene Gedancken.} Durch den zweiten Satz wird die Angabe, dass wir
Dinge empfinden, nicht widerrufen, sondern um weitere Objekte unseres
Empfindens ergänzt, wie aus \S~3 hervorgeht.}, aber Tatsachen erfahren. Was wir
empfinden, sind einzelne Gegenstände wie Gläser, Wasser und Tische bzw.\ die
Wirkungen auf unsere Sinnesorgane wie Licht oder Nässe. Was wir erfahren oder
wahrnehmen ist, dass das Wasser den Tisch nass macht.
\end{comment}
Da historische Erkenntnisse genau diejenigen Erkenntnisse sind, die wir aus
Erfahrung haben, können wir schließen, dass historische Erkenntnisse
Erkenntnisse von einzelnen Dingen (\emph{singularia}) sind. Historische
Erkenntnisse gibt es \emph{a fortiori} -- so scheint es -- nur in der Form
singulärer Urteile wie \enquote{Peter ist
wütend} oder  \enquote{Dieser Stein ist schwer}, nicht aber als
allgemeine Urteile wie \enquote{Alle Menschen sind sterblich} oder
\enquote{(Alle) Steine sind schwer}. Auch schon
\authorcite{Hobbes:Leviathan1962}' Darstellung legt in diesem Sinne nahe,
dass absolutes Wissen, welches wir legitimer Weise als testimoniales Wissen
akzeptieren, ausschließlich singuläre oder besondere Urteile betrifft, also
solche Urteile, die von einem einzelnen Gegenständen aussagen, dass sie die und
die Eigenschaften haben oder ihnen dies und jenes zugestoßen ist. Allgemeine
Urteile, die eine Eigenschaft von \emph{allen} Gegenständen (einer bestimmten
Art) aussagen, scheinen eher auf die Seite konditionalen Wissens zu gehören. Sie sagen, immer
dann, \emph{wenn} etwas ein Gegenstand der Art $\Phi $ ist, \emph{dann} hat es
auch die Eigenschaft $\Psi $. \emph{Wenn} beispielsweise die Sonne auf einen
Stein scheint, \emph{dann} wird dieser erwärmt. Insbesondere allgemeine Naturgesetze gehören
zumindest nach \authorcite{Hobbes:Leviathan1962} zu dem konditionalen, nicht zu dem absoluten
Wissen.

Hinzu kommt eine terminologische Schwierigkeit: Der Ausdruck
\enquote{historisch} (\emph{historicus}) hat bei \authorcite{Wolff:Psychologiaempirica1968} je nach Kontext unterschiedliche
Gegenbegriffe. In der Unterteilung dreier Erkenntnisarten sind der historischen
Erkenntnis (\emph{cognitio historica}) die \enquote{mathematische}
(\emph{cognitio mathematica}) und die \enquote{philosophische} (\emph{cognitio
philosophica}) entgegengesetzt. Bei der Einteilung von Schriften ist
\enquote{historisch} (\emph{liber historicus}) der Gegenbegriff zu
\enquote{dogmatisch} (\emph{liber dogmaticus}).
\emph{Libri historici} berichten Tatsachen der Natur
oder der Menschen, während die \emph{libri dogmatici}
allgemeine Wahrheiten enthalten.\footnote{\enquote{\ori{In libris vel
recensentur facta sive naturae , sive hominum ; vel proponuntur dogmata , seu
veritates universales}}
\parencite[][\S~743]{Wolff:PhilosophiarationalissiveLogica1740}.
  \enquote{\ori{Libri} facta recensentes dicuntur \ori{historici}. Et
  propositiones singulares , si fuerint verae , \ori{veritatum historicarum}
  nomine veniunt}
  \parencite[][\S~744]{Wolff:PhilosophiarationalissiveLogica1740}.
  \enquote{\ori{Liber dogmaticus} a nobis dicitur , qui dogmata , seu veritates
  universales proponit}
  \parencite[][\S~750]{Wolff:PhilosophiarationalissiveLogica1740}.} Und dann
  kann \enquote{historisch} auch als Gegenbegriff zu \enquote{wissenschaftlich}
  oder \enquote{szientifisch} (\emph{scientificus}) verwendet werden: Die
  \emph{libri dogmatici} unterteilt
\authorcite{Wolff:Psychologiaempirica1968} wiederum in \emph{libri dogmatici
historici} und \emph{libri dogmatici scientifici}, je nachdem, ob sie selbst der
Forderung nach wissenschaftlicher Darstellung der Lehren genügen, oder bloß
von wissenschaftlichen Erkenntnissen berichten, ohne selbst wissenschaftliche Ansprüche zu
erfüllen.\footnote{\cite[Vgl.][\S~751]{Wolff:PhilosophiarationalissiveLogica1740}.}




Aus \authorcite{Wolff:Psychologiaempirica1968}s Sicht sind gerade allgemeine Naturgesetze
paradigmatische Fälle historischer Erkenntnisse. Als Beispiel einer historischen
Wahrheit nennt \authorcite{Wolff:Psychologiaempirica1968} den Fall, in dem jemand aus Erfahrung
weiß, dass ein Regenbogen entsteht, wenn Regen und Licht der Sonne
zusammentreffen.\footnote{\cite[Vgl.][\S~744]{Wolff:PhilosophiarationalissiveLogica1740}:
\enquote{Si quis observet, \ori{coelo pluvio ac lucente Sole apparuisse
iridem,} {\&} factum idem recenseat , is veritatem historicam proponit. Et
historicus erit liber , in quo plura istiusmodi facta recensentur.}}
\enquote{Historisch} nennt er jede Erkenntnis dessen, was ist oder geschieht
(\enquote{\ori{Cognitio} eorum, quae sunt atque
fiunt}\footnote{\cite[][\S~3]{Wolff:Discursuspraeliminarisdephilosophiaingenere1996}.}),
und er exemplifiziert dies durch die Erfahrungserkenntnisse, dass morgens die
Sonne aufgeht und sich Tiere durch Zeugung
fortpflanzen.\footnote{\enquote{E.\,gr.\ historica ejus est cognitio, qui
expertus novit, Solem mane oriri, vespere autem occidere; initio veris gemmas
effrondescere arborum; animalia propagari per generationem; nos nil appetere
nisi sub ratione boni}
\mkbibparens{\cite[][\S~3]{Wolff:Discursuspraeliminarisdephilosophiaingenere1996},
\cite[vgl.][\pno~165\,f.]{Seifert:Cognitiohistorica1976}}.} Aber wie lassen sich
\authorcite{Wolff:Psychologiaempirica1968}s Äußerungen zur Erfahrung als
Erkenntnis einzelner Dinge in der \titel{Deutschen Logik} dann mit seiner
Schilderung der historischen Erkenntnis im \titel{Discursus praeliminaris} in
Einklang bringen?

\authorcite{Wolff:Psychologiaempirica1968} vertritt im \titel{Discursus praeliminaris} die Ansicht,
dass wir allgemeine Erkenntnisse über kausale Zusammenhänge aus der Erfahrung haben. Wir erfahren, dass verschüttetes
Wasser den Tisch nass macht; nicht nur, dass Peter gestern ein Wasserglas
verschüttete und anschließend der Tisch nass war und dass Wolfgang am Tag zuvor
ein Glas Wasser verschüttete und ebenfalls der Tisch nass wurde und so weiter,
sondern dass verschüttetes Wasser \emph{immer} (oder simpliciter) den Tisch nass
macht. Aber wir nehmen dies nicht abstrakt wahr, sondern durch die Wahrnehmung
konkreter Beispiele. Es gibt -- so ein erster Deutungsversuch -- keine Erfahrung
allgemeiner Erkenntnisse ohne Wahrnehmung besonderer Fälle. Und von diesen konkreten Beispielen sagt
\authorcite{Wolff:Psychologiaempirica1968}, dass sie \emph{in bestimmten Fällen} zusätzlich zu der allgemeinen
Erfahrungserkenntnis anzuführen seien. Sie müssen angeführt werden, wenn nicht
jeder in der Lage ist, selbst die Erfahrung zu machen, oder nicht jeder die
Erfahrung selbst schon hätte gemacht haben können (wenn er nur auf seine
Wahrnehmungen geachtet hätte). Im Falle des Wassers, das den Tisch nass macht, ist kein
konkreter Fall anzuführen, da jeder in der Lage ist, eine solche Wahrnehmung
selbst zu haben. Hier liegt die Erfahrungserkenntnis einfach in dem allgemeinen
Urteil, dass Wasser nass macht. Nun gibt es Wahrnehmungen, die nicht jeder nach
Belieben erzeugen kann, aber jeder schon einmal gemacht hat. Dass Donner auf
Blitz folgt, ist eine solche Erfahrungserkenntnis, zu der nach \authorcite{Wolff:Psychologiaempirica1968}
ebenfalls keine Nennung konkreter Beispiele nötig ist. Aber es gibt auch Fälle,
bei denen nicht jeder die Erfahrung hat machen können und die auch nicht
reproduzierbar sind, ohne dass dies gegen ihre Zuverlässigkeit spräche. Man
denke an den Geologen, der sich auf seismologische Messungen verschiedener
Erdbeben beruft.
In solchen Fällen ist die konkrete Erfahrungssituation zu beschreiben, damit der
Zuhörer beurteilen kann, ob die Erfahrung richtig gedeutet wurde. Schließlich
täusche man sich leicht und oft darin, welche Erkenntnis in einer bestimmten
Situation wirklich erlangt werden kann.\footnote{\cite[Vgl.][Capitel 5,
\S~2]{Wolff:VernuenftigeGedankenvondenKraeftendesmenschlichenVerstandesundihremrichtigenGebraucheinErkenntnisderWahrheit1978}:
\enquote{Man erfordert aber dieses aus zweyerley Ursachen: einmahl, damit man
sehe, was einer vor Empfindungen gehabt, da er zu seiner Erfahrung gelanget:
darnach, damit man sehen kan, wie einer nach Anleitung seiner Empfindungen
seinen Satz formire. Und dieses ist höchstnöthig, denn wir finden, daß gar ofte
Leute einander widersprechen, und doch beyde sich auf die Erfahrung beruffen.}}

Entgegen der Deutung \authorcite{Kambartel:ErfahrungundStruktur1968}s geht es
hier also nicht um allgemeine im Unterschied zu singulären Urteilen, sondern um
inferentielles im Unterschied zu nicht-inferentiellem Wissen.\footnote{Siehe
dazu auch oben, Kap. \ref{subsubsection:BegriffderDiskursivitaet} ab S.
\pageref{Abschnitt:judiciumintuitivumdiscursivum}.} Im \titel{Discursus
praeliminaris} finden wir für diese Thematik den Ausdruck \emph{cognitio
(historica) arcana}, der in der \titel{Deutschen Logik} noch fehlt. Bei
\emph{beiden} Wissensarten handelt es sich um historische Erkenntnis und bei
\emph{beiden} kann es sich um singuläre wie auch um allgemeine Erkenntnisse
handeln. Dabei geht es nicht nur um eine gründliche Methodik empirischer
Forschung, sondern auch um die Möglichkeit kritischer Kontrolle vermeintlich
objektiver Erkenntnisse. Denn
\begin{quote}
 daß einer solche oder andere Empfindungen gehabt, kan er nicht erweisen, als
durch Zeugen, wenn einige mit dabey gewesen, und also fordert er mit Recht, man
solle ihm dies glauben. Allein er kan nicht fordern, daß ich glaube, er habe
richtig geschlossen: denn dieses kan nach den Regeln, vernünftig zu schlüssen,
beurtheilet werden, weil nichts gewöhnlicher, als daß man durch die Erfahrung
Sätze zu erschleichen suchet, wovon ich in der Methaphysick \punkt\ ein Exempel
gegeben.\footnote{\Cite[][Capitel 5, \S~4]{Wolff:VernuenftigeGedankenvondenKraeftendesmenschlichenVerstandesundihremrichtigenGebraucheinErkenntnisderWahrheit1978}.}
\end{quote}
Nicht alles, was wir aus der Erfahrung zu haben beanspruchen, entstammt
tatsächlich direkt unserer nicht-inferentiellen Wahrnehmung. Vieles muss erst
erschlossen werden. In der \titel{Deutschen Metaphysik} verweist
\authorcite{Wolff:Psychologiaempirica1968} auf die Behauptung, man könne aus der
Erfahrung auf die Einwirkung des Körpers auf die Seele schließen, weil man einen
steten Zusammenhang zwischen bestimmten körperlichen und seelischen Ereignissen
erfahre. Ein Schluss von der zeitlichen Koexistenz solcher Ereignisse auf eine
Einwirkung sei aber nicht statthaft, weil wir über keinen Begriff einer solchen
Einwirkung
verfügten.\footnote{\cite[Vgl.][529]{Wolff:VernuenftigeGedankenvondenKraeftendesmenschlichenVerstandesundihremrichtigenGebraucheinErkenntnisderWahrheit1978}.
In der Lateinischen Logik nennt
\authorcite{Wolff:PhilosophiarationalissiveLogica1740} das umgekehrte
Beispiel eines \enquote{\ori{[i]nfluxus physici anim\ae{} in corpus}} sowie die
magnetische Anziehung, von der wir nicht sagen könnten, wir nähmen sie wahr
\parencite[vgl.][\S~667]{Wolff:PhilosophiarationalissiveLogica1740}.}

Die kritische Funktion der Unterscheidung von inferentiellem und
nicht-inferentiellem Erfahrungswissen -- diskursive und intuitive Urteile,
verborgene und gemeine Erkenntnis -- liegt darin, dass nachprüfbar sein muss, ob
jemand tatsächlich erfahrungsbasiertes Wissen referiert. Die zu bannende Gefahr
liegt dabei nicht in einer möglichen Unaufrichtigkeit des Informanten, sondern
in der Fehldeutung und Überinterpretation des Berichteten. Es ist eine Maxime
des wissenschaftlichen Arbeitens nach
\authorcite{Wolff:Discursuspraeliminarisdephilosophiaingenere1996}, dass den
Rezipienten von Forschungsergebnissen transparent gemacht werden soll, auf
welcher Erfahrungsgrundlage die Ergebnisse erzielt wurden. Dabei handelt es sich
noch immer um historische Erkenntnisse, wenn mitgeteilt wird, dass etwas der
Fall ist (unabhängig davon, ob es sich um singuläre oder allgemeine Urteile
handelt). Es ist dies ein Grundsatz, der noch die Übernahme von historischen
Erkenntnisse aus zweiter Hand betrifft. Das Hauptaugenmerk legt
\authorcite{Wolff:Discursuspraeliminarisdephilosophiaingenere1996} aber
weiterhin auf die Differenz von historische und philosophischer Erkenntnis, die
durch die von \authorcite{Kambartel:ErfahrungundStruktur1968} und
\authorcite{Holzhey:KantsErfahrungsbegriff1970} angeführten Stellen gar nicht
berührt wird.

\section{Zusammenfassung und Ausblick}
In Kapitel \ref{section:autonomieunddaszeugnisanderer} hatte ich auf
\authorfullcite{Crusius:WegzurGewissheitundZuverlaessigkeitdermenschlichenErkenntniss1965}
als einen Vertreter der deutschen Aufklärung verwiesen, der eine
\singlequote{defaultistische} Position bezüglich der Möglichkeit testimonialen
Wissens vertritt: Wir präsupponieren die Zuverlässigkeit von Mitteilungen,
solange nicht gegen sie spricht. Auch \name[Immanuel]{Kant} scheint diese
Theorie zu übernehmen, wie aus einer Anmerkung im handschriftlichen Nachlass
hervorgeht.\footnote{Siehe Anm.
\ref{Anmerkung:KantundCrusiusPraesuppositionsTheorie} auf S.
\pageref{Anmerkung:KantundCrusiusPraesuppositionsTheorie} dieser Arbeit.} Die
Frage war nun, wie eine solche Position als aufgeklärt oder \singlequote{kritisch} verstanden
werden kann. Hierbei fanden sich zwei Herangehensweise:
\begin{nummerierung}
\item Die eine Möglichkeit, welche sich bei Autoren wie
\authorcite{Crusius:WegzurGewissheitundZuverlaessigkeitdermenschlichenErkenntniss1965},
\authorcite{Meier:AuszugausderVernunftlehre1752} und
\authorcite{Reimarus:DieVernunftlehrealseineAnweisungzumrichtigenGebrauchderVernunftinderErkenntnisderWahrheit1756}
ausgearbeitet findet, verweist auf die Betrachtung der Zuverlässigkeit der
Informationsquellen. Der Erwerb testimonialen Wissens ist also kritisch genau
dann, wenn wir keine Informationen übernehmen, deren Überbringer
Verdachtsmomente liefern oder deren Ursprung fraglich ist. Wir bewerten also die
Bonität eines Informanten oder einer Information.
\item Die zweite Möglichkeit besteht darin, bestimmte Arten von Erkenntnissen
von der Möglichkeit testimonialen Wissens auszunehmen beziehungsweise zu
konkretisieren, welche Gefahren bei der Übernahme solcher Erkenntnisse auf
Mitteilung hin zu erwarten sind.
\authorfullcite{Wolff:Discursuspraeliminarisdephilosophiaingenere1996}
Unterscheidung der drei Erkenntnisarten zu Beginn des \titel{Discursus
praeliminaris de philosophia in genere} führt dies exemplarisch vor und schließt
damit an \authorcite{Descartes:OeuvresdeDescartes1983}' Warnung vor einer bloß
historischen Kenntnis an, die aus der Büchergelehrsamkeit hervorgehe. Er macht
einen Anfang in dem Bemühen, den Begriff der \singlequote{historischen Kenntnis}
zu spezifizieren.
\end{nummerierung}
Die Warnung vor einer bloßen \emph{cognitionis philosophicae cognitio historica}
besagt freilich nicht, dass eine Übernahme philosophischen Wissens von anderen
nicht möglich oder nicht nicht wünschenswert sei. Sie konkretisiert, welche
Defizite möglich sind, wenn wir solche Erkenntnisse lediglich passiv rezipieren,
ohne zu beachten, dass der Erwerb wissenschaftlichen Wissens ganz wesentlich ein
Erwerb von wissenschaftlichen \emph{Kompetenzen} ist.
\authorcite{Descartes:OeuvresdeDescartes1983}' Warnung ist korrekt, wenn sie
verstanden wird als Warnung vor einem bloßen Nachsprechen wissenschaftlicher
Erkenntnisse ohne wissenschaftliche Ausbildung.

Bei \name[Immanuel]{Kant} finden sich -- wie bei vielen Autoren der damaligen
Zeit -- \emph{beide} Herangehensweisen; doch zeigte sich, dass seine Anknüpfung
an die Überlegungen zur Bonität des Informanten oder der Information nicht
ausreichen, einen Begriff aufgeklärten und mündigen Erwerbs testimonialen
Wissens zu erläutern. Das folgende \ref{Chapter:KantsSocialEpistemology}.
Kapitel soll zeigen, dass \name[Immanuel]{Kant} vielmehr an
\authorcite{Wolff:Discursuspraeliminarisdephilosophiaingenere1996} anschließt,
wenn es um den Umgang mit testimonialem Wissen geht. Seine Sorge gilt weniger
der Verlässlichkeit der Information als dem Verständnis derselben auf der Seite
des Rezipienten.
