Wenn jemand absichtlich die Unwahrheit sagt, dann lügt er und verdient Tadel.
Ebenso handelt unmoralisch, wer ein falsches Versprechen abgibt oder wer andere
Menschen durch rassistische Äußerungen zum Hass aufwiegelt. Aber wie verhält es
sich mit den Ansichten, die jemand hegt, ohne sie anderen mitzuteilen? Falsche
Versprechen und Lügen sind Verstöße gegen ethische Normen, die Menschen durch
sprachliche Handlungen gegenüber anderen begehen. Wenn wir nun nicht nur die äußeren
Handlungen eines Menschen ethisch bewerten, sondern bereits seine
Überzeugungen selbst sowie die Art und Weise, wie er Überzeugungen bildet; wenn
wir Menschen tadeln, weil sie an Hexen und Magie, an Horoskope und Homöopathie
glauben, weil sie rassistische, sexistische oder andere Vorurteile hegen oder
weil sie leichtgläubig sind und anderen Menschen alles Mögliche glauben, dann
vertreten wir eine \emph{ethics of belief}. Wir kritisieren Menschen für
ihre Überzeugungen oder die Weisen ihrer Überzeugungsbildung unabhängig davon,
was sie mit diesen Überzeugungen anstellen.

\authorfullcite{Clifford:TheEthicsofBelief1877} argumentiert, dass all unsere
Überzeugungen Gegenstand ethischer Bewertung sein sollten, weil jede
Überzeugung, die wir haben, sich auf irgendeinem Weg in Handlungen äußern wird,
die andere Menschen
affektieren.\footnote{\cite[Vgl.][289--295]{Clifford:TheEthicsofBelief1877}.}
Auch \name[Immanuel]{Kant} hält manche Überzeugungen oder vielmehr
die Art und Weise, wie sie gebildet werden, für tadelns- oder
lobenswert.\footnote{\enquote{Daher können wir freylich jemand Tadeln, welcher
einer falschen Erkenntniß beyfall gegeben hat; wenn nemlich die Schuld
wircklich an ihm selbst lieget, daß er nemlich die jenige gründe, welche ihm
von dem Gegenstand der Erkenntniß, die er hat, hätten überzeugen, und also von
seinem Irrthum befreyen können, abweist}
\mkbibparens{\cite{Kant:LogikBlomberg1966}, \cite[][XXIV:
160.12--17]{Kant:GesammelteWerke1900ff.}}.
Diese Stelle zitiert \textcite[vgl.][317]{Cohen:KantontheEthicsofBelief2014}.} Sie sind als
Ausdruck von Mündigkeit oder eben Unmündigkeit bewertbar, auch ohne auf die
Konsequenzen der Überzeugungen zu achten. Systematischen Niederschlag findet
dies dort, wo er auch die Wahrhaftigkeit \emph{sich selbst gegenüber} zu den
moralischen Pflichten
zählt.\footnote{\cite[Vgl.][\S~9]{Kant:DieMetaphysikderSitten1977Tugendlehre},
\cite[][VI: 430.9--26]{Kant:GesammelteWerke1900ff.}.} Aber es scheint sich auch
direkt aus dem Programm der Aufklärung und der Aufforderung zum Ausgang aus
selbstverschuldeter Mündigkeit zu ergeben, denn schon die Anwendung des
Prädikats \enquote{selbstverschuldet} setzt die Möglichkeit der ethischen Bewertung voraus.

Seit geraumer Zeit wird über eine \emph{ethics of belief} bei
\name[Immanuel]{Kant} gesprochen, die sich im vorletzten Abschnitt der
\titel{Kritik der reinen Vernunft} ausmachen lasse. Allerdings sind seine
Bemerkungen im \titel{Kanon der reinen Vernunft} durchgehend kurz und
skizzenhaft,\footcite[Vgl.][323]{Chignell:BeliefinKant2007} was ihre
Interpretation merklich erschwert. Seit einigen Jahren steigt das Interesse an
diesem Thema,\footnote{Siehe
\cite{Stevenson:OpinionBelieforFaithandKnowledge2003}, \cite{Chignell:BeliefinKant2007},
\cite{Chignell:KantsConceptsofJustification2007},
\cite{Cohen:KantontheEthicsofBelief2014}.} was nicht zuletzt an dem Begriff des
Glaubens liegt, mit dem
\name[Immanuel]{Kant} in den Augen mancher Interpreten einen interessanten
Ansatz zu einem \enquote{nicht-epistemischen Rechtfertigungsbegriff}
entwickle\footnote{\cite[Vgl.][\pno~33\,f.]{Chignell:KantsConceptsofJustification2007}.}
und durch den seine \emph{ethics of belief} in Richtung des späteren
Pragmatismus tendiere\footnote{\cite[Vgl.][335]{Chignell:BeliefinKant2007}.}.

Nun liegt es nahe, gerade hier eine systematische Verbindung von
Aufklärungsprogramm und Vernunftkritik und damit der Endlichkeit des Menschen,
auf deren Konzeption die Vernunftkritik basiert, zu vermuten.
Diese Vermutung muss sich freilich mit zwei Vorbehalten auseinandersetzen:
\name[Immanuel]{Kant} spricht dort erstens nicht explizit von Aufklärung,
Mündigkeit und Selbstdenken, aber es lässt sich leicht zeigen, dass genau diese
Programmatik den Ausführungen zugrunde liegt. Und zweitens ließe sich fragen, in
welchem Zusammenhang diese kurzen und skizzenhaften Abschnitte, die in Arbeiten
zur Vernunftkritik kaum Erwähnung
finden,\footcite[Vgl.][\pno~323\,f.]{Chignell:BeliefinKant2007} mit den
Kernthemen der Vernunftkritik -- Deduktion der Kategorien, Grundsätze des reinen
Verstandes, transzendentale Dialektik -- stehen \emph{respective} ob es
überhaupt einen näheren Zusammenhang gibt.

Ich werde argumentieren, dass die Überlegungen im Abschnitt \titel{Vom Meinen,
Wissen und Glauben} eine \emph{ethics of belief} auf Grundlage des
\enquote{sapere aude!} enthalten und dass es sich in der Tat um einen
interessanten und brauchbaren Ansatz handelt. Vielleicht sollte ich genauer sagen, dass sich
auf dieser Grundlage eine solche entwickeln lässt, denn die Skizzenhaftigkeit
seiner Ausführungen überlässt doch noch sehr viel dem Leser. Es sind dabei vor
allem die Begriffe der \emph{Überzeugung} und der \emph{Überredung}, die seinen
Ansatz auszeichnen (Abschnitt \ref{subsubsection:UEberredungundÜberzeugung}). Die Frage, warum es im Umgang
mit rationalen Erkenntnissen zu beachtende Unterschiede zwischen Metaphysik und
Mathematik gebe, führt auf einige abschließende Beobachtungen dazu, wie
philosophische Inhalte mündig erkannt werden (Abschnitt
\ref{subsubsection:EndlichesundUnendlichesErkennen}).

\Revision[Heidemann]{In diesen beiden Abschnitten werden die relevanten Punkte
einer auf Mündigkeit hin ausgerichteten \emph{ethics of belief} bereits enthalten sein. Nun ist es jedoch
so, dass \name[Immanuel]{Kant}s Überlegungen sich besonders durch seinen
Begriff des \emph{Vernunftglaubens} auszeichnen, der in der Darstellung
sicherlich nicht stillschweigend übergangen werden darf.}
Die Besonderheit der Position \name[Immanuel]{Kant}s liegt in der
Behauptung, es gebe auch Gründe für die Vernünftigkeit einer Überzeugung, die sich nicht auf
die Wahrheit der Überzeugung, sondern ein Bedürfnis des Subjekts richten. Die
Motivation dieser Position liegt darin, dass \name[Immanuel]{Kant}s
Aufklärungsprogramm einerseits religiöse Überzeugungen fokussiert, andererseits
aber nur vernünftig begründete Überzeugungen als zulässig ansieht. Da sich aus
unserer Endlichkeit ergibt, dass wir bezüglich zentraler religiöser Aussagen
keine Gründe finden, welche die Wahrheit der Überzeugungen belegen können, sieht
sich \name[Immanuel]{Kant} gezwungen, einen Begriff subjektiv zureichenden
Fürwahrhaltens zu entwickeln (Abschnitt
\ref{section:HandelnAufEpistemischDuennerGrundlage}).

\section{Überreden und
Überzeugen}\label{subsubsection:UEberredungundÜberzeugung}
\Revision[Pelletier]{Dem Titel nach befassen sich
\name[Immanuel]{Kant}s Überlegungen im Abschnitt \KapitelTitel{Vom
Meinen, Wissen und Glauben} im \KapitelTitel{Kanon der reinen Vernunft} mit
einer Begriffstrias, die sich beispielsweise im siebenten Kapitel der
\titel{Deutschen Logik} von
\cite{Wolff:VernuenftigeGedankenvondenKraeftendesmenschlichenVerstandesundihremrichtigenGebraucheinErkenntnisderWahrheit1978}
findet.\footcite[Vgl.][200--205]{Wolff:VernuenftigeGedankenvondenKraeftendesmenschlichenVerstandesundihremrichtigenGebraucheinErkenntnisderWahrheit1978}
Er beginnt seine Darstellung jedoch mit einem anderen Begriffspaar, nämlich} mit
der Unterscheidung von Überredung und Überzeugung. \Revision[Pelletier]{Auch
dieses Begriffspaar findet sich freilich bereits bei früheren Autoren.
\authorcite{Baumgarten:Metaphysica---Metaphysik2011} etwa bezeichnet Überredung
als sinnliche, Überzeugung hingegen als intellektuelle
Gewissheit.\footnote{\Revision[Pelletier]{\cite[Vgl.][531]{Baumgarten:Metaphysica---Metaphysik2011}.}}
Und \authorcite{Meier:AuszugausderVernunftlehre1752} sieht in der Überredung
einen deffizienten Zustand, in dem wir einen Irrtum fälschlich für gegründetes
Wissen halten.\footnote{\cite[][\S~184]{Meier:AuszugausderVernunftlehre1752}.}
\name[Immanuel]{Kant}s Bestimmung weicht von beiden ab}:
\begin{quote}
Das Fürwahrhalten ist eine Begebenheit in unserem Verstande, die auf objektiven
Gründen beruhen mag, aber auch subjektive Ursachen im Gemüte dessen, der da
urteilt, erfordert. Wenn es für jedermann gültig ist, so fern er nur Vernunft
hat, so ist der Grund desselben objektiv hinreichend, und das Fürwahrhalten
heißt alsdenn \ori{Überzeugung}. Hat es nur in der besonderen Beschaffenheit des
Subjekts seinen Grund, so wird es \ori{Überredung} genannt.\\
Überredung ist ein bloßer Schein, weil der Grund des Urteils, welcher lediglich
im Subjekte liegt, für objektiv gehalten
wird.\footnote{\label{Anmerkung:KU90UeberredenUeberzeugen}\cite[][B
848]{Kant:KritikderreinenVernunft2003}, \cite[][III: 531.27--532.4]{Kant:GesammelteWerke1900ff.}. Eine Parallelstelle
findet sich in der \titel{Kritik der Urteilskraft}: \enquote{Zuerst wir zu
jedem Beweise, er mag (wie bei dem Beweise durch Beobachtung des Gegenstandes
oder Experiment) durch unmittelbare empirische Darstellung dessen, was bewiesen
werden soll, oder durch Vernunft a priori aus Prinzipien geführt werden,
erfordert: daß er nicht \ori{Überrede}, sondern \ori{Überzeuge}, oder wenigstens
auf Überzeugung wirke, d.\,i. daß der Beweisgrund oder der Schluß nicht ein bloß
subjektiver (ästhetischer) Bestimmungsgrund des Beifalls (bloßer Schein),
sondern objektiv-gültig und ein logischer Grund der Erkenntnis sei; denn sonst
wird der Verstand berückt, aber nicht überführt}
\mkbibparens{\cite[][\S~90]{Kant:KritikderUrteilskraft2009}, \cite[][V:
461.14--22]{Kant:GesammelteWerke1900ff.}}.}
\end{quote}
Jedes Urteil hat subjektive Ursachen, aber manche Urteile haben darüber hinaus
objektive Gründe. Im Idealfall ist es wohl so, dass beides zusammenfällt und
der objektive Grund eines Urteils zugleich seine subjektive Ursache ist.
Wenn Ingrid sieht, dass Max den Kaffee verschüttet, dann hat ihre Überzeugung,
dass Max seinen Kaffee verschüttet hat, eine subjektive Ursache, die zugleich ein
objektiver Grund ist: ihre Wahrnehmung von Max' Ungeschicklichkeit. Es kommen
dabei nur solche Gründe als objektive Gründe in Betracht, die dem Urteilenden
\emph{de facto} zugänglich sind. Wenn Ingrid unaufmerksam war und nicht sah, wie
Max den Kaffee verschüttet, dann hat sie keinen objektiven Grund für diese
Überzeugung, auch wenn sie einen solchen haben \emph{könnte}, wäre sie nicht in
Gedanken gewesen. Und ebenso hatte Pierre de \name[Pierre de]{Fermat} keine
objektiven Gründe für die Behauptung, dass die Gleichung $a^n + b^n = c^n$ für
$a,b,c,n \in \mathbb{N}^{+}$ und $n > 2$ keine Lösungen hat. Diese Behauptung
war daher zunächst nicht als \emph{Theorem} aufzufassen, sondern als
\name[Pierre de]{Fermat}s \emph{Vermutung}, auch wenn ein Beweis dieser Aussage
natürlich schon damals \singlequote{existierte}\footnote{Zumindest gilt dies in einer naheliegenden
Bedeutung des Ausdrucks \singlequote{exisitieren}, nach der mathematische
Beweise auch dann existieren, wenn sie noch nicht entdeckt wurden.}, wie sich
später zeigen ließ.
Als objektive Gründe zählen also nur solche Gründe, die ein Urteilender als
Gründe eines Urteils nennen kann und von denen er weiß, dass sie objektive
Gründe sind.

\subsection{Vernünftige Gründe}\label{subsection:VernuenftigeGruende}
Paradigmatische Fälle von Überredung finden wir bei Wunschdenken und
Selbsttäuschung.\footnote{\cite[Vgl.][331]{Chignell:BeliefinKant2007}, sowie
\cite[][323]{Cohen:KantontheEthicsofBelief2014}.} In der \titel{Logik Blomberg}
findet sich als Beispiel der Gedanke an ein Leben nach dem Tod, den wir
annehmen, weil uns dieser Gedanke gefällt \emph{respective} weil der Gedanke,
dass es kein solches Leben gibt, Unbehagen bereitet.\footnote{Vgl.
\cite{Kant:LogikBlomberg1966}, \cite[][XXIV:
198.25--30]{Kant:GesammelteWerke1900ff.}. Weiter heißt es dort: \enquote{Das
menschliche Gemüth ist auf diese Art wircklich sehr vielen Illusionen, und
Blendwercken unterworfen, blos darum öfters, weil uns etwas gefällt, so halten
wir es vor gewiß, und blos darum, weil uns etwas mißfällt, oder verdreußt,
halten wir es vor ungewiß. Diese Gewisheit [sic], oder Ungewißheit aber ist
nicht objectiv, sondern vielmehr Subjectiv [sic]}
\mkbibparens{\cite{Kant:LogikBlomberg1966}, \cite[][XXIV:
198.31--36]{Kant:GesammelteWerke1900ff.}}.} Wäre der doxastische Voluntarismus
wahr, dann \enquote{würden wir uns beständig Chimären von einem glücklichen
Zustande machen und sie sodann auch immer für wahr halten.}\footnote{\cite[][A
113]{Kant:ImmanuelKantsLogik1977}, \cite[][IX:
74.4--5]{Kant:GesammelteWerke1900ff.}.} Aber auch wenn wir nicht
einfach für wahr halten können, was wir möchten, tendieren wir doch gerade dort,
wo wir keine objektiven Gründe kennen, dazu, zu glauben, was uns gefällt. Dass
der doxastische Voluntarismus falsch ist,\footnote{Siehe
dazu auch Anm. \ref{Anmerkung:KantundderDoxastischeVoluntarimus} auf S.
\pageref{Anmerkung:KantundderDoxastischeVoluntarimus} und die dort zitierte
Literatur.} heißt zumindest nach \name[Immanuel]{Kant} nicht, dass der Wille und
dass unsere Wünsche und Neigungen \emph{keinen} Einfluss auf unser Fürwahrhalten
haben. Es heißt lediglich, dass sie gegen überzeugende Beweise von Wahrheiten
nichts ausrichten können und dass es uns nicht gelingt, bewusst unsere
Überzeugungen zu wählen. Es heißt nicht, dass sie nicht unbewusst unsere Urteile
beeinflussen.

Ein Beispiel für Selbsttäuschung lässt sich im Anschluss an
\name[Immanuel]{Kant}s Ausführungen zur \emph{natürlichen Dialektik} der
\titel{Grundlegung} finden.
Wir erwerben moralische \emph{Überzeugungen}, wenn wir erkennen, dass etwas eine
moralische Forderung darstellt (oder dass es erlaubt ist), weil wir die
objektiven Gründe dieses Urteils einsehen. Aber allzu oft reden wir uns und anderen ein, etwas sei
(generell oder in einem speziellen Fall) erlaubt, weil wir ein subjektives
Bedürfnis haben. Wir bestreiten dann die Allgemeingültigkeit eines moralischen
Gebotes oder behaupten, dass in einem gegebenen Fall eine Ausnahme legitim sei.
Die Gründe, die wir dafür vorbringen, sind allesamt unzureichend, es ist
lediglich eine subjektive Neigung, die uns dazu antreibt, ein bestimmtes Urteil
zu fällen. Dennoch -- das ist der Witz der natürlichen Dialektik -- überreden
wir nicht nur andere, sondern auch uns selbst, unserer Argumentation zu
vertrauen.

Wie entscheiden wir, ob ein Urteil gut begründet ist oder ob es sich um
Wunschdenken und Selbsttäuschung handelt? Offenbar gibt es zwei Punkte, an denen
wir ansetzen können: Wir können die irrationalen subjektiven Ursachen zu eliminieren
versuchen und wir können untersuchen, ob unsere Gründe tatsächlich objektiv sind.
Dazu müssen wir zunächst eruieren, was der \singlequote{wahre} Grund für eine
Überzeugung ist. Max muss ausschließen, dass der Grund für seine Urteile über
Peter in seinem Neid liegt. Wenn er erkennt, dass dies sein Grund ist, stellt er
fest, dass er keinen objektiven Grund hat und sein Urteil unbegründet ist. Wie
erkennt Max aber, was der Grund für sein Urteil ist und ob es sich um einen
objektiven Grund oder eine bloß subjektive Ursache handelt?

\authorfullcite{Stevenson:OpinionBelieforFaithandKnowledge2003} hält es
für naheliegend, die Frage, ob ein objektiver Grund vorliegt, als eine Frage der
\singlequote{Introspektion} aufzufassen. Wenn Max urteilt, dass Peter unsympathisch ist, und nun wissen
möchte, ob dieses Urteil auf einem objektiv zureichenden Grund beruht, dann
solle er sich fragen, welcher Grund ausschlaggebend für sein Urteil ist. Es
könnte ja sein, dass er entdeckt, neidisch auf Peter zu sein und aus diesem
Neid heraus zu urteilen. Aber das ist offensichtlich
unbefriedigend, denn dass Max eine mögliche irrationale Ursache in sich
entdeckt, mag ihn zur Vorsicht mahnen, aber es beweist noch lange nicht, dass
sein Urteil unbegründet ist. Es ist hilfreicher, wenn er die Gründe untersucht,
die für sein Urteil sprechen (und vielleicht auch die, die gegen sein Urteil
sprechen), und diese Gründe anderen zur Prüfung vorlegt. Und so sagt auch
\name[Immanuel]{Kant}, dass die Überprüfung des eigenen Urteils an dem Urteil
anderer die \emph{einzige} Möglichkeit sei, Überredung und Überzeugung zu
unterscheiden.\footnote{\enquote{Überredung demnach kann von der Überzeugung
subjektiv zwar nicht unterschieden werden, wenn das Subjekt das Fürwahrhalten,
bloß als Erscheinung seines eigenen Gemüts, vor Augen hat; der Versuch aber,
den man mit den Gründen desselben, die für uns gültig sind, an anderer
Verstand macht, ob sie auf fremde Vernunft eben dieselbe Wirkung tun, als auf
die unsrige, ist doch ein, obzwar nur subjektives, Mittel, zwar nicht
Überzeugung zu bewirken, aber doch die Privatgültigkeit des Urteils, d.\,i.
etwas in ihm, was bloße Überredung ist, zu entdecken} \mkbibparens{\cite[][B
849]{Kant:KritikderreinenVernunft2003}, \cite[][III:
532.17--24]{Kant:GesammelteWerke1900ff.}}. Darauf verweist schließlich auch
\textcite[vgl.][79]{Stevenson:OpinionBelieforFaithandKnowledge2003}.}

Erschwerend kommt hinzu, dass \name[Immanuel]{Kant} neben objektiven Gründen und
subjektiven Ursachen auch objektiv zureichende und subjektiv zureichende Gründe
erwähnt. Nun ließe sich \emph{prima facie} vermuten, dass die objektiven Gründe
einfach mit den objektiv zureichenden Gründen zu identifizieren seien und
\emph{a fortiori} die subjektiv zureichenden Gründe den subjektiven Ursachen
entsprächen. Ich halte dies für unbefriedigend, weil es das, was
\name[Immanuel]{Kant} \emph{Glauben} nennt, zu einer Wirkung kontingenter
Ursachen degradiert. Ein Grund ist nicht allein dadurch bereits subjektiv
\emph{zureichend}, dass er in uns ein Fürwahrhalten (egal ob vernünftiger oder
unvernünftiger Weise) \emph{verursacht}.
Die Überlegungen zu subjektiv zureichenden Gründen können nur dann
gewinnbringend sein, wenn ein subjektiv zureichender Grund Gewähr dafür leistet,
dass unser Fürwahrhalten \emph{vernünftig} ist.\footnote{Dagegen urteilt
\authorfullcite{Pasternack:KantonOpinion2014}: \enquote{It refers to the
psychological state of firmly holding a proposition to be true}
\parencite[][43]{Pasternack:KantonOpinion2014}. Allerdings schreibt er weiter:
\enquote{Conviction has the same extension as subjective sufficiency, but it
explicitly brings out a normative element}
\parencite[][46]{Pasternack:KantonOpinion2014}, womit doch eine Wertung aus der
Perspektive der Vernunft Einzug hält.
Tatsächlich scheint gerade der \singlequote{doktrinale Glaube} lediglich einen
psychologischen Zustand zu beschreiben (siehe Kapitel \ref{section:HandelnAufEpistemischDuennerGrundlage}); aber
wenn die subjektive Zulänglichkeit nichts anderes bezeichnete, ginge der Witz der Überlegungen
verloren.} Einen Grund als
subjektiv zureichend zu bezeichnen, heißt nicht, ihm eine kausale Funktion zuzuschreiben, sondern einen epistemischen Status. Mir scheint \name[Immanuel]{Kant} bei der thematischen Hinführung im Abschnitt \KapitelTitel{Vom Meinen, Wissen und Glauben} das subjektiv zureichende Fürwahrhalten noch nicht zu berücksichtigen.

Ich schlage zunächst vor, in einem ersten Schritt zu erläutern, wann ein Grund
zureichend ist und erst dann den Unterschied zwischen subjektiv und objektiv
zureichenden Gründen zu klären. Um zu verstehen, wann ein Grund (subjektiv oder
objektiv) zureichend ist, sollten wir davon ausgehen, dass ein Grund nur dann
ein zureichender Grund sein kann, wenn die Bildung eines Fürwahrhaltens auf
seiner Grundlage \emph{vernünftig} ist. Um diesem Gedanken gerecht zu werden,
schlage ich vor, den Begriff des \emph{zureichenden Grundes} unter Rückgriff auf
doxastische Grundsätze zu erläutern. Unter einem \emph{doxastischen Grundsatz}
verstehe ich eine Regel, die (bewusst oder unbewusst) unser Fürwahrhalten
leitet, indem sie als Grundsatz bestimmt, was wir als Grund für ein Urteil
akzeptieren.\footnote{Ich verwende den Ausdruck \enquote{doxastisch} in einem
allgemeinen Sinne, so dass er auch \enquote{epistemisch} mit umgreift und auf
alle Modi des Fürwahrhaltens -- Meinen, Glauben und Wissen -- bezogen werden
kann. Es ließe sich stattdessen auch der Ausdruck \enquote{epistemisch}
verwenden; jedoch wird dieser öfter in einem eingeschränkten Sinne verwendet
und lediglich auf Wissen bezogen, was hier die Folge hätte, dass der Ausdruck
\enquote{epistemischer Grundsatz} nur auf solche Grundsätze passte, die
objektive Gründe als solche ausweisen und Wissen konstituieren.}  Ein
Grund ist \emph{zureichend} genau dann, wenn der doxastische Grundsatz, der ihn als
Grund für ein Urteil ausweist, ein vernünftiger Grundsatz gemäß der positiven Bestimmung
des Selbstdenkens und \emph{a fortiori} kein Vorurteil ist. Der Grundsatz,
Informationen von anderen als Grundlage unseres Wissens zu akzeptieren, solange
nichts gegen deren Vertrauenswürdigkeit spricht, ist ein solcher Grundsatz
(siehe Kapitel \ref{section:autonomieunddaszeugnisanderer}).
Wenn Ingrid sich auf diesen Grundsatz berufen kann, dann liegt eine Überzeugung
und keine Überredung vor, auch wenn Peter lügt. Überzeugungen sind also die Folgen
von vernünftigen doxastischen Grundsätzen, Überredung ist die Folge eines
Vorurteils.

Die zentrale Forderung \enquote{sapere aude!} tritt in der \titel{Kritik der
reinen Vernunft} in Form der Unterscheidung von \emph{Überzeugung} und
\emph{Überredung} hervor. Selbstdenken heißt nach \name[Immanuel]{Kant}, sich im
Denken am Maßstab des eigenen Vernunftgebrauchs auszurichten, also nur das für
wahr zu halten, was wir selbst als nach vernünftigen doxastischen Grundsätzen
fundiert ansehen können. Ein Vorurteil wiederum ist ein unvernünftiger Grundsatz
des Denkens, also ein Urteil, dass selbst nicht hinreichend fundiert ist, uns
aber als Prinzip dient, auf dessen Grundlage wir weitere Urteile generieren.
Wenn Max glaubt, dass Peter unsympathisch ist, und Max dies glaubt, weil Peter
aus A-Dorf kommt, dann besteht sein Vorurteil darin, dass er ein unbegründetes
Inferenzschema akzeptiert: \emph{Wenn} Person $A$ aus A-Dorf kommt, \emph{dann}
ist $A$ unsympathisch. Sein Urteil, dass Peter unsympathisch ist, ist selbst
kein Vorurteil, sondern eine \emph{Folge} seines Vorurteils. Er könnte dies
entdecken, indem er seine Urteile an den Urteilen anderer Menschen überprüft;
diese werden Urteile, die nach dem genannten Inferenzschema begründet werden,
nicht akzeptieren, woraufhin Max erkennen kann, dass er manche seiner Urteile
vermutlich auf der Grundlage eines Vorurteils bildet. Er würde dann zunächst
versuchen, sein Urteil mit weiteren Gründen auszustatten und eine explizite
Grundlage suchen, die sein Vorurteil offensichtlich nicht bieten kann.

\name[Immanuel]{Kant}s Darstellung setzt voraus, dass unser Fürwahrhalten
überhaupt Regeln folgt, die wir als vernünftig oder unvernünftig bewerten können.
Wenn unser Fürwahrhalten aber nicht auf einem objektiven Grund, sondern
lediglich auf einer subjektiven Ursache beruht, dann ist denkbar, dass wir überhaupt keiner
Regel folgen. Für gewöhnlich sind es jedoch wiederkehrende Beeinflussungen wie
im Falle des Wunschdenkens oder der Sympathie und Antipathie, die uns dazu bringen, etwas
aufgrund subjektiver Ursachen statt objektiver Gründe für wahr zu halten. Und
wenn wir dem auch nur in Einzelfällen nachgeben, laufen wir Gefahr, die
entsprechende Urteilsweise zu habitualisieren\footnote{Dies ist ein
entscheidendes Argument bei
\textcite[vgl.][294]{Clifford:TheEthicsofBelief1877}.} und zu einer Regel zu
machen, der wir (unbewusst) folgen. Dass wir eine entsprechende Regel nicht als
solche unseres Denkens benennen können, heißt nicht, dass sie nicht als
Grundsatz unseres Denkens fungiert.

Wie \name[Immanuel]{Kant} in \titel{Was heißt: sich im Denken orientieren?}
ausführt, ist es eine Illusion, wenn jemand glaubt, er könne sein Denken
dauerhaft \emph{ohne} verbindliche Regeln gestalten. Er wird sein Denken
letztlich an den Regeln orientieren, die er \emph{nicht} frei
wählt.\footnote{\cite[Vgl.][A
327\,f.,]{Kant:Washeisst:SichimDenkenorientieren?1977}
\cite[][VIII: 145.6--35]{Kant:GesammelteWerke1900ff.}.} Urteile, die nicht
Ausdruck des Selbstdenkens sind, bilden wir nicht \emph{ohne} Regeln. Wir bilden
sie auf der Grundlage anderer Regeln, die wir nicht als vernünftig einsehen
können und die uns zumeist auch gar nicht bewusst sind.
Statt vernünftiger epistemischer Grundsätze regieren biografische Zufälligkeiten
wie Gewohnheiten und Neigungen unser Denken. Max' Urteil, dass Peter
unsympathisch ist, steht in Zusammenhang zu anderen Urteilen -- hier mit dem
Urteil, dass Peter aus A-Dorf kommt.
Und dieser Zusammenhang wird selbst Ursachen haben; vielleicht ist Max durch ein
traumatisierendes Erlebnis im Zusammenhang mit A-Dorf geprägt. Er folgt dann
immer noch einem Grundsatz in seinem Urteilen, aber einem Grundsatz, der nicht
Ausdruck seiner Vernunft, sondern Resultat biographischer Zufälligkeiten ist.
Wir können nur nach Regeln urteilen, doch können die entsprechenden Regeln
vernünftig sein oder unvernünftig. Vernünftig ist eine Regel, die wir ganz
bewusst zum allgemeinen Grundsatz unseres Denkens machen (und im Falle einer
Diskussion als zustimmungsfähigen Grundsatz benennen) können. Ein Vorurteil ist
eine Regel, der wir im Denken folgen, obwohl wir sie nicht als vernünftigen
Leitfaden ansehen können.

Wenn meine Analyse richtig ist, dann müssen wir bei der Beurteilung, ob ein
Grund zureichend ist, zunächst einen doxastischen Grundsatz identifizieren,
den es von den Bedingungen seiner Anwendung zu unterscheiden gilt. Mir scheint,
dass \name[Immanuel]{Kant} in den Abschnitten der transzendentalen Methodenlehre
lediglich auf diese doxastischen Grundsätze achtet und die Frage nach
Anwendungsbedingung nicht berücksichtigt. Die Frage, ob wir vernünftigen
doxastischen Grundsätzen folgen und dabei Erfolg haben,  oder ob wir mit diesen
vernünftigen Grundsätzen scheitern, ist zwar für eine Analyse des Begriffs
\enquote{Wissen} von zentraler Bedeutung, aber aus Sicht der \emph{ethics of
belief} nicht relevant. Ob in einem konkreten Fall Wissen vorliegt, mag von
Zufällen abhängen, die wir selbst nicht kontrollieren können. Aber ob wir mündig
und verantwortlich urteilen, das kann aus der Sicht \name[Immanuel]{Kant}s
wegen des ethischen Charakters der Fragestellung nicht davon abhängen, ob wir
\singlequote{Glück} haben -- man könnte hier von \emph{epistemic luck}
sprechen.\footnote{Der Ausdruck ist in Anlehnung an
\textcite[vgl.][]{Williams:MoralLuck1981} gebildet und findet sich etwa bei
\textcite[vgl.][]{Engel:IsEpistemicLuckCompatiblewithKnowledge1992} und
\textcite[vgl.][\pno~104\,f.]{Hills:MoralTestimonyandMoralEpistemology2009}.

Gegen diese Interpretation spricht \emph{prima facie}, dass
\name[Immanuel]{Kant} Urteile, die auf objektiv zureichenden Gründen beruhen, als \enquote{Wissen} bezeichnet.
\enquote{Wissen} wiederum ist ein eindeutig veridischer Begriff: Wir können eine
Überzeugung ausschließlich dann als Wissen bezeichnen, wenn sie tatsächlich wahr
ist. Wir stehen also vor der Wahl, entweder den Ausdruck \enquote{objektiv
zureichend} mit einer eigenen Wahrheitsbedingung zu versehen, die
Wahrheitsbedingung zusätzlich zum Vorliegen objektiv zureichender Gründe in den
Begriff des Wissens zu integrieren oder als zusätzliche epistemische Kategorie
neben \enquote{Wissen} auch \enquote{wahres Wissen}
zuzulassen \parencite[vgl.][330]{Chignell:BeliefinKant2007}.

Um das Sprechen von falschen Überzeugungen zu vermeiden, könnten wir auch sagen,
dass weder Überzeugung noch Überredung, sondern einfach eine Täuschung vorliegt.
Aber letztlich scheint mir \emph{Überzeugung} ein epistemischer Begriff zu sein,
der nicht von der Wahrheit des Urteils, sondern lediglich von den Gründen
abhängt, die den Urteilenden verfügbar sind. Als Beleg dafür, dass dies mit
\name[Immanuel]{Kant}s Wortgebrauch vereinbar ist, ließe sich folgende Bemerkung
anführen: \enquote{Die subjektive Zulänglichkeit heißt \ori{Überzeugung} (für
mich selbst), die objektive, \ori{Gewißheit} (für
jedermann)} \mkbibparens{\cite[][B 850]{Kant:KritikderreinenVernunft2003},
\cite[][III: 533.5--7]{Kant:GesammelteWerke1900ff.}}. Dass Überzeugung sich hier
nicht auf das nur subjektiv zureichende Fürwahrhalten bezieht, erhellt daraus,
dass Wissen, Glauben und Meinen alle als Stufen der \enquote{subjektive[n]
Gültigkeit des Urteils, in Beziehung auf die Überzeugung (welche zugleich
objektiv gilt)} \mkbibparens{\cite[][B 850]{Kant:KritikderreinenVernunft2003},
\cite[][III: 532.36--37]{Kant:GesammelteWerke1900ff.}} angesprochen
werden. Leider macht diese Bemerkung das Verständnis letztlich
schwieriger, da \name[Immanuel]{Kant} die Zusammenhänge nicht weiter
erläutert, sondern lapidar anmerkt, er werde sich \enquote{bei der Erläuterung
so faßlicher Begriffe nicht aufhalten} \mkbibparens{\cite[][B
850]{Kant:KritikderreinenVernunft2003}, \cite[][III:
533.7--8]{Kant:GesammelteWerke1900ff.}}.}

Meine Analyse kongruiert mit dem, was wir von einer \emph{ethics of belief}
vernünftigerweise erwarten sollten. Es geht um die Unterscheidung epistemisch
verantwortungsvoller (lobenswerter) Haltungen auf der einen und epistemisch
fahrlässiger (tadelnswerter) Ansichten auf der anderen Seite. Wir wollen aber
nicht denjenigen tadeln, der einer Täuschung anheim fiel, die er selbst nicht
entdecken konnte. Wenn Ingrid und Max sich in einer fremden Stadt nach dem Weg
erkundigen, aber das Pech haben, dass ein Witzbold ihnen die falsche Richtung
weist, dann ist dessen Lüge zu tadeln, nicht aber Ingrids und Max'
Gutgläubigkeit. Ingrid und Max erwerben eine Überzeugung, die nicht auf
subjektiven Ursachen beruht, aber auch nicht unbegründet ist. Und dies liegt daran, dass sie
vernünftigen epistemischen Grundsätzen folgen, wenngleich ihre Ausführung
unvorhergesehen (und unvorhersehbar) scheitert.

Anders verhält es sich bei falschen Grundsätzen. Wir tadeln ja gerade
diejenigen, die aus (beispielsweise) rassistischen Vorurteilen heraus urteilen
(und zwar ganz unabhängig davon, ob aus diesen Vorurteilen in Einzelfällen wahre
Urteile resultieren). Und wir tadeln denjenigen, der dem Grundsatz folgt, seine
politischen Urteile generell an denen seiner Eltern auszurichten. Wir sagen,
dass er unmündig sei, weil er hierin einem unvernünftigen epistemischen
Grundsatz folgt, der keine objektiven Gründe generiert, sondern geradezu dazu
aufruft, objektiv rationale Erkenntnisse subjektiv historisch zu erwerben. Der
Grundsatz ruft zum Verzicht auf Selbstdenken auf und ist als solcher Aufruf
tadelnswert, mögen aus diesem Verzicht auch wahre Urteile resultieren.

Ein zweiter Aspekt der Unterscheidung von Überzeugung und Überredung ist, dass
Überredung unbewusst stattfindet: Sie liegt vor, wenn jemand aus einer
subjektiven Zufälligkeit heraus urteilt, aber denkt, vernünftigen epistemischen
Grundsätzen zu folgen.\footnote{Allerdings suggeriert eine Bemerkung
\name[Immanuel]{Kant}s im selben Abschnitt, dass uns auch Überredungen manchmal
bewusst sein können: \enquote{Überredung kann ich für mich behalten, wenn ich
mich dabei wohl befinde}
\mkbibparens{\cite[][B 850]{Kant:KritikderreinenVernunft2003},
\cite[][III: 532.33--34]{Kant:GesammelteWerke1900ff.}}.} Max denkt in unserem
Beispiel vielleicht, er fälle ein erfahrungsbasiertes Urteil über Peter, doch in
Wahrheit folgt er einer zufällig erworbenen Abneigung gegen A-Dorf oder seinem
Neid gegenüber Peter. Sein Urteil ist Ausdruck eines Vorurteils, wenngleich er
selbst denkt, vernünftigen Grundsätzen zu folgen. Wir können in aller Regel wohl
davon ausgehen, dass der Grund darin liegt, dass derjenige, der einem Vorurteil
aufsitzt, selbst nicht weiß, welchem Grundsatz er folgt (sei es, dass er gar
keinen Grundsatz nennen könnte, sei es, dass er glaubt, einem Grundsatz zu
folgen, der in seinem Denken nicht maßgebend war und den er vielleicht sogar
verletzt).

In Anlehnung an \name[Immanuel]{Kant} bezeichne ich ein Urteil, bei dem der
Urteilende sich die Gründe seines Urteilens transparent gemacht hat und weiß,
welchen epistemischen Grundsätzen er folgt, als
\enquote{überlegt}.\footnote{\cite[Vgl.][B
316\,f.,]{Kant:KritikderreinenVernunft2003} \cite[][III:
214.33--215.31]{Kant:GesammelteWerke1900ff.}.} Aufklärung fordert von uns also,
überlegt zu urteilen.
Und das bedeutet, dass wir uns bewusst machen sollen, warum wir etwas für
wahr halten und ob es sich dabei um einen zureichenden Grund handelt.
Ein brauchbares Kriterium, dies zu erkennen, ist der in Kapitel
\ref{section:sensuscommunis} beschriebene \singlequote{Pluralismus} oder die
\singlequote{erweiterte Denkungsart}. Auf diese Weise können wir auch ohne
explizite Überlegung zu mündigen und vorurteilsfreien Urteilen gelangen. Denn
unsere bloß kontingenten Grundsätze verraten sich als
unvernünftig, wenn wir bei anderen keine Zustimmung finden. Freilich ist dieses
Kriterium dabei weder hinreichend -- die Menschen in unserer Umgebung mögen
dieselben Vorurteile hegen -- noch ist es stets zuverlässig -- mitunter können
wir mit einem vernünftigen Urteil auf vorurteilsbeladene Gesprächspartner
treffen.


Nun kann ein Grundsatz, der zureichende Gründe liefert, einerseits auf
die empirische Darstellung des Gegenstandes, andererseits auf Prinzipien der
Vernunft gestützt sein.\footnote{Siehe dazu das Zitat in Anmerkung
\ref{Anmerkung:KU90UeberredenUeberzeugen} auf S. \pageref{Anmerkung:KU90UeberredenUeberzeugen}.}
Wenn wir überlegen, auf welchen doxastischen Grundsätzen unser Urteil beruht,
dann ist nach \name[Immanuel]{Kant} genau die Frage relevant, die -- wie Kapitel
\ref{chapter:MuendigerErwerbTestimonialenWissens} zeigte -- seinen Umgang mit
testimonialem Wissen bestimmt:
\begin{quote}
Die erste Frage vor aller weitern Behandlung unserer Vorstellungen ist die: in
welchem Erkenntnisvermögen gehören sie zusammen? Ist es der Verstand, oder sind
es die Sinne, vor denen sie verknüpft, oder verglichen werden? Manches Urteil
wird aus Gewohnheit angenommen, oder durch Neigung geknüpft; weil aber keine
Überlegung vorhergeht, oder wenigstens kritisch darauf folgt, so gilt es für ein
solches, das im Verstande seinen Ursprung erhalten
hat.\footnote{\cite[][B 316]{Kant:KritikderreinenVernunft2003},
\cite[][III: 215.6--12]{Kant:GesammelteWerke1900ff.}.}
\end{quote}
Was hat die Frage nach dem Ursprung einer Erkenntnis mit der Frage zu tun, ob
sie der Vernunft oder bloßer Gewohnheit und Neigung entstammt? Mir scheint die
Antwort in folgendem zu liegen. Wir erwerben empirische Erkenntnis, indem wir
bestimmte Ereignisse stets verknüpft vorfinden. Eine solche Art des
Erkenntniserwerbs erinnert daher nicht grundlos an den Erwerb einer Gewohnheit,
weswegen es nicht absurd scheint, wenn
\authorfullcite{Hume:AnEnquiryConcerningHumanUnderstanding1964} die Grundlage
all unserer Schlüsse aus Erfahrung in der Gewohnheit (\emph{custom},
\emph{habit})
vermutet.\footnote{\cite[Vgl.][37]{Hume:AnEnquiryConcerningHumanUnderstanding1964}.}
Aber rationale Erkenntnisse dürfen wir nicht auf solche Art annehmen; hier ist
es gerade kein legitimer Maßstab, ob wir etwas regelmäßig in unserer Wahrnehmung
verknüpft finden.

Wenn wir nun aber nicht wissen, ob eine Erkenntnis eine
empirische oder eine rationale Erkenntnis ist (und sie vielleicht fälschlich
für empirisch halten), dann kann es uns passieren, dass wir sie als Erkenntnis
akzeptieren, weil wir sie in der Erfahrung bestätigt finden. Wir sehen die
Grundsätze des reinen Verstandes etwa in der Erfahrung bestätigt und akzeptieren
sie als vermeintlich empirisch fundiertes Wissen. Oder wir akzeptieren ein
moralisches Urteil, weil alle Menschen in unserer Umgebung beständig
entsprechend urteilen. In solchen Szenarien fällen wir Urteile aus bloßer
Gewohnheit, also aus Vorurteil, weil die Grundsätze, denen wir bei dem Fällen
unserer Urteile folgen, gar nicht die relevanten Grundsätze sind, die bei den
entsprechenden Urteilen anzulegen sind.

Um mündig und vorurteilsfrei zu denken, müssen wir nicht bei jedem Urteil
die Gründe und Gegengründe abwägen; aber wir müssen darüber Gewissheit erlangen,
auf welcher Grundlage wir urteilen, und speziell darüber, ob es sich um eine
(objektiv) rationale oder empirische Erkenntnis handelt und entsprechend
(subjektiv) rationale Erkenntnis gefordert ist oder historische Erkenntnis
legitim und ausreichend ist. \name[Immanuel]{Kant} sagt dazu, wir bedürften
nicht jederzeit einer \singlequote{Untersuchung}, wohl aber stets der
\singlequote{Überlegung} (\emph{reflexio}).\footnote{\enquote{Nicht alle Urteile
bedürfen einer \ori{Untersuchung}, d.\,i. einer Aufmerksamkeit auf die Gründe
der Wahrheit {\punkt}. Aber alle Urteile {\punkt} bedürfen einer
\ori{Überlegung}, d.\,i. einer Unterscheidung der Erkenntniskraft, wozu die
gegebenen Begriffe gehören}
\mkbibparens{\cite[][B 316\,f.,]{Kant:KritikderreinenVernunft2003}
\cite[][III: 215.12--18]{Kant:GesammelteWerke1900ff.}}.} Aufklärung und die
Forderung nach Selbstdenken erfordern keine tiefgreifenden Untersuchungen. Wenn
wir sehen, dass etwas der Fall ist, oder wenn uns jemand mitteilt, dass es der
Fall ist, dann brauchen wir nicht weiter zu untersuchen, ob dieser Grund denn
auch zuverlässig ist. Wir müssen etwa keine weiteren Informationen über die
Glaubwürdigkeit unseres Informanten einholen; der epistemische Grundsatz,
Mitteilungen Glauben zu schenken, ist völlig ausreichend.
Aber wir sollen doch in dem angesprochenen Sinne \emph{überlegt} urteilen, also
wissen, ob es sich um eine Erkenntnis handelt, bei welcher der entsprechende epistemische
Grundsatz Anwendung finden kann.

\subsection{Objektive Gründe}\label{subsection:ObjektiveGruende}
Im weiteren Verlauf diskutiert \name[Immanuel]{Kant} ausschließlich das
überlegte, aufgeklärte und vernünftige Fürwahrhalten. Dieses hat drei Stufen:
Meinen, Wissen und Glauben. Sie unterscheiden sich in ihrem Status, insofern
Wissen allein objektiv und subjektiv zureichend ist, Glauben immerhin noch
subjektiv zureichend, objektiv aber unzureichend, Meinen hingegen objektiv und
subjektiv unzureichend ist. Wissen liegt dort vor, wo wir über objektive Gründe
verfügen. Glaube ist ein Status, der zwar keine objektiv zureichenden Gründe zu
haben beanspruchen kann, aber subjektiv zureichende Gründe geltend macht.
Da es bei allen drei Arten des Fürwahrhaltens um überlegte Überzeugungen geht, täuscht
sich der Urteilende in allen Fällen nicht über den Status seiner Überzeugung.

\Revision[Pelletier]{Trotz der oberflächlichen Ähnlichkeit zu verbreiteten
Darstellungen beispielsweise bei
\authorcite{Wolff:Discursuspraeliminarisdephilosophiaingenere1996}\footnote{Siehe
\cite[][200--205]{Wolff:VernuenftigeGedankenvondenKraeftendesmenschlichenVerstandesundihremrichtigenGebraucheinErkenntnisderWahrheit1978},
sowie \cite[][\S\S~594--661]{Wolff:PhilosophiarationalissiveLogica1740}.} ist diese Systematik durchaus
originell. Autoren wie \authorcite{Wolff:Discursuspraeliminarisdephilosophiaingenere1996} und
\authorcite{Meier:AuszugausderVernunftlehre1752} verstehen den Glauben schlicht
als \singlequote{historischen Glauben}, also als testimoniales
Wissen,\footnote{\Revision[Pelletier]{\cite{Wolff:PhilosophiarationalissiveLogica1740}:
\enquote{\ori{Fides} dicitur assenus, quem praebemus propositioni propter
autoritatem dicentis, ipsamque propositionem \ori{credere} dicimur.}
\authorcite{Meier:AuszugausderVernunftlehre1752} schreibt: \enquote{Aus
anderer Leute Erfahrung werden wir, vermittelst des Glaubens, gewiss}
\mkbibparens{\cite[][\S~206]{Meier:AuszugausderVernunftlehre1752}, \cite[][XVI:
496.28--29]{Kant:GesammelteWerke1900ff.}}.}} welches \name[Immanuel]{Kant} hier
wiederum unter den Begriff des Wissens subsumiert, ohne ihm eine eigene
Kategorie zuzuweisen. Meinungen werden in aller Regel als unzureichend bewiesene
Behauptungen verstanden; ein Rekurs auf das Wissen um den Status des je eigenen
Fürwahrhaltens findet sich zumindest bei
\authorcite{Meier:AuszugausderVernunftlehre1752}, der zugleich zwischen gemeiner
(\singlequote{\emph{opinio vulgaris}}) und gelehrter oder philosophischer
Meinung (\singlequote{\emph{hypothesis
philosophica, erudita}})
unterscheidet.\footnote{\Revision[Pelletier]{\cite[Vgl.][\S~181]{Meier:AuszugausderVernunftlehre1752},
\cite[][XVI: 461.22--24]{Kant:GesammelteWerke1900ff.}: \enquote{\ori{Eine Meinung} (opinio) ist eine jede ungewisse Erkenntniss, in so ferne wir sie
annehmen, und zugleich erkennen, dass sie nicht gewiss sei.} Dagegen definiert
\authorcite{Wolff:Discursuspraeliminarisdephilosophiaingenere1996} schlichter:
\enquote{Propositio insufficienter probata dicitur \ori{Opinio}}
\parencite[][\S~602]{Wolff:PhilosophiarationalissiveLogica1740}.}} Was sich
meines Wissens nirgends vor \name[Immanuel]{Kant} findet, das ist die
Unterscheidung zwischen subjektiv und objektiv zureichendem Fürwahrhalten.}

Nun habe ich bisher nur gesagt, wann ein Grund zureichend ist, ohne auf den
Unterschied zwischen objektiv und subjektiv zureichenden Gründen einzugehen.
Ein Grund ist ein zureichender Grund, wenn es nach Maßgabe des
positiven Begriffs des Selbstdenkens vernünftig ist, auf seiner Grundlage einem
Urteil seine Zustimmung zu geben. Einen Zusammenhang zwischen beiden können wir jedoch bereits festhalten. Denn
wie aus der Einteilung in Wissen, Glauben und Meinen unmittelbar hervorgeht,
folgt aus der objektiven Zulänglichkeit die subjektive. Wenn ich einsehe, dass
ich über einen objektiven Grund $G$ verfüge zu urteilen, dass $p$, dann ist $G$
ein Grund, der auch subjektiv zureichend ist.\footnote{\name[Immanuel]{Kant} kennt kein Fürwahrhalten, das
objektiv zureichend, subjektiv aber unzureichend ist. Wenn Ingrid einen objektiv
zureichenden Grund hat, dann hat sie \emph{eo ipso} einen subjektiv zureichenden
Grund. \authorfullcite{Stevenson:OpinionBelieforFaithandKnowledge2003} verweist
darauf, dass objektive Zulänglichkeit subjektive Zulänglichkeit impliziert und sich so
die Dreiteilung in Meinen, Glauben und Wissen ergibt
\parencite[vgl.][78]{Stevenson:OpinionBelieforFaithandKnowledge2003}.
\authorfullcite{Chignell:BeliefinKant2007} hält die Annahme dieser Implikation
für unbegründet und spricht im Falle von objektiv zureichenden Gründen, die dem
Urteilenden nicht zugänglich sind, von \enquote{mere convictions}
\parencite[vgl.][\pno~331\,f.]{Chignell:BeliefinKant2007}. Es ist aber
eindeutig, dass \name[Immanuel]{Kant} eine entsprechende Kategorie nicht für
diskussionswürdig erachtet.}
Wann ist ein Grund nun aber auch \emph{objektiv}
und wann ist er nur \emph{subjektiv} zureichend? \name[Immanuel]{Kant}s einzige
Auskunft scheint zu sein: \enquote{Die subjektive Zulänglichkeit heißt
\ori{Überzeugung} (für mich selbst), die objektive, \ori{Gewißheit} (für jedermann).}\footnote{\cite[][B
850]{Kant:KritikderreinenVernunft2003}, \cite[][III:
533.5--7]{Kant:GesammelteWerke1900ff.}. \name[Immanuel]{Kant} fügt an, er werde
sich \enquote{bei der Erläuterung so faßlicher Begriffe nicht aufhalten}
\mkbibparens{\cite[][B 850]{Kant:KritikderreinenVernunft2003}, \cite[][III:
533.7--8]{Kant:GesammelteWerke1900ff.}}. Siehe auch \cite[][A
98\,f.,]{Kant:ImmanuelKantsLogik1977} \cite[][IX:
66.4--7]{Kant:GesammelteWerke1900ff.}: \enquote{Das gewisse Fürwahrhalten oder
die \ori{Gewißheit} ist mit dem Bewußtsein der Notwendigkeit verbunden; das
ungewisse dagegen oder die \ori{Ungewißheit} mit dem Bewußtsein der Zufälligkeit
oder der Möglichkeit des Gegenteils.}} Doch was heißt dies? Die Schwierigkeit in
der Interpretation liegt offensichtlich darin, dass auch die subjektive
Zulänglichkeit vernünftig sein soll und insofern scheinbar nicht nur für mich
gelten kann. Die Bedeutung dieser Unterscheidung ist nur durch die Darstellung
des Begriffs des Vernunftglaubens zu erhellen. Schließlich sind fast alle
Gründe, die wir kennen, entweder objektiv zureichende oder schlicht
unzureichende Gründe einer Überzeugungsbildung. Der Bereich möglicher subjektiv
zureichender und objektiv unzureichender Gründe ist auch nach
\name[Immanuel]{Kant} sehr begrenzt und es fragt sich (aus der Perspektive des
Philosophen, nicht des bloßen Interpreten), ob es solche Gründe tatsächlich
gibt. Sowohl die bisher erwähnten empirischen Erkenntnisse \emph{ex datis} als
auch die Vernunfterkenntnisse \emph{ex principiis} zählen zu den
Überzeugungen mit objektiv zureichender Grundlage.

\authorfullcite{Chignell:KantsConceptsofJustification2007} sagt, ein Grund sei
objektiv zureichend, wenn er die Wahrheit des Urteils hinreichend
\emph{wahrscheinlich} macht.\footnote{\enquote{A \ori{sufficient objective ground} is
one that renders the proposition in question moderately-to-highly likely}
\parencite[][42]{Chignell:KantsConceptsofJustification2007}. \enquote{A
\ori{sufficient objective ground} for assent to a proposition is one that
indicates to a moderate-to-high degree---though not always infallibly---that
the proposition is true} \parencite[][326]{Chignell:BeliefinKant2007}.} Dies
könne der Urteilende sicherlich nicht mit letzter Sicherheit ausmachen und darum
handle es sich um eine \emph{externalistische}
Position.\footnote{\cite[Vgl.][49]{Chignell:KantsConceptsofJustification2007}.}
Nach \authorcite{Chignell:KantsConceptsofJustification2007} können wir zwar
wissen, auf \emph{welchen} Gründen unser Urteil beruht, aber wir können nur
vermuten, ob sie objektiv zureichend sind.\footnote{\enquote{it is also worth
noting that despite the at bottom externalist character of the account, there is
an emphatic nod to internalist intuitions {\punkt}. Knowledge cannot merely be
based on sufficient objective grounds; rather, the subject must also be in a
position, on reflection, to cite those grounds, although she need not be
(and usually is not) able to determine that they are objectively
sufficient}
\parencite[][\pno~49\,f.]{Chignell:KantsConceptsofJustification2007}.} Ob ein
zureichender Grund objektiv zureichend ist, hänge also im Gegensatz zur Frage
nach subjektiv zureichenden Gründen doch von unserem \emph{epistemic luck} ab.

Dies ist aus mehreren Gründen unbefriedigend. Zum einen ist es unbefriedigend,
von einer externalistischen Position auszugehen, denn wenn wir eine Person für
ihre Überzeugungen und die Art ihrer Überzeugungsbildung verantwortlichen
machen, dann möchten wir auf Aspekte rekurrieren, die diese Person auch
überschauen und beeinflussen kann. Zum anderen konstituiert Wahrscheinlichkeit
kein Wissen. \enquote{Unter Wahrscheinlichkeit ist
ein Fürwahrhalten aus unzureichenden Gründen zu verstehen, die aber zu den
zureichenden ein größeres Verhältnis haben, als die Gründe des
Gegenteils.}\footnote{\cite[][A 126]{Kant:ImmanuelKantsLogik1977},
\cite[][IX: 81.23--26]{Kant:GesammelteWerke1900ff.}. Siehe dazu
\cite[][\nopp 2583]{Kant:Reflexionen1900ff.},
\cite[][XVI: 9--11]{Kant:GesammelteWerke1900ff.}.} Somit begründet
Wahrscheinlichkeit bloß Meinung, aber kein Wissen.

Wie im Falle der Überredung liegt bei der Meinung kein objektiv zureichender
Grund vor, aber der \singlequote{Meinende} lässt sich dadurch nicht täuschen,
sondern \emph{weiß}, dass er (noch) nicht über einen objektiven Grund verfügt,
und hält ihn daher auch subjektiv nicht für zureichend. Der Unterschied zwischen
legitimen und illegitimen vorläufigen Urteilen besteht primär darin, dass der
Urteilende bei legitimen vorläufigen Urteilen weiß, dass sie einer objektiven
Grundlage entbehren, und entsprechend mit ihnen verfährt.
Er wird keine Schlüsse aus ihnen ziehen beziehungsweise berücksichtigen, dass
alles, was er aus ihnen schließt, ebenfalls nicht objektiv begründet ist. Und er
wird sie niemandem als Wissen mitteilen, sondern sie als seine Vermutungen für
sich behalten oder sie zumindest explizit als Vermutungen kennzeichnen.

Allerdings bedarf es zumindest vereinzelter Anhaltspunkte, die für die Wahrheit
des Urteils sprechen; und dass diese Anhaltspunkte vorliegen, das muss wiederum
gewusst werden.\footnote{\enquote{Ich darf mich niemals unterwinden, zu \ori{meinen},
ohne wenigstens etwas zu \ori{wissen}, vermittelst dessen das an sich bloß
problematische Urteil eine Verknüpfung mit Wahrheit bekommt, die, ob sie
gleich nicht vollständig, doch mehr als willkürliche Erdichtung ist}
\mkbibparens{\cite[][B 850]{Kant:KritikderreinenVernunft2003},
\cite[][III: 533.9--12]{Kant:GesammelteWerke1900ff.}}.} Es mag sein, dass ich
nicht weiß, ob noch Schwarzbier im Kühlschrank ist. Ich erinnere mich vielleicht
nicht mehr genau, ob ich ein oder zwei Flaschen gekauft habe und wie viele ich
am Wochenende trank, und kann nicht ausschließen, dass sich mein Mitbewohner eine
nahm. Damit ich aber doch vermuten kann, dass noch Schwarzbier im Kühlschrank
ist, muss ich doch \emph{wissen}, dass ich Schwarzbier kaufte, dass ich die
Flaschen in den Kühlschrank lege und einiges mehr. Selbst wenn ich der Hoffnung,
es sei noch Schwarzbier im Kühlschrank, eine sehr geringe Wahrscheinlichkeit
beimesse, muss es doch Anhaltspunkte geben, die mehr als nur wahrscheinlich
sind, sonst ließe sich \emph{gar nicht} von Wahrscheinlichkeit sprechen.

\Revision[Pelletier]{Meinungen sind nicht generell abzulehnen. Zu ihnen gehören
nach traditioneller Auffassung auch
\emph{Hypothesen},\footnote{\Revision[Pelletier]{\cite[Vgl.][\S~606]{Wolff:PhilosophiarationalissiveLogica1740}.}}
worunter Meinungen verstanden werden, aus denen sich beobachtbare Geschehnisse
in der Welt erklären
lassen.\footnote{\Revision[Pelletier]{\cite[Vgl.][\S~181]{Meier:AuszugausderVernunftlehre1752},
\cite[][XVI: 461.24--29]{Kant:GesammelteWerke1900ff.}, sowie
\cite[][\S~126]{Wolff:Discursuspraeliminarisdephilosophiaingenere1996}.}}} Schon
\authorcite{Wolff:Discursuspraeliminarisdephilosophiaingenere1996} betont die
Bedeutung von vorläufigen Urteilen, legt aber zugleich Wert darauf, dass der
Status von \singlequote{Hypothesen} jederzeit als solcher berücksichtigt werden
muss.\footnote{\Revision[Pelletier]{\cite[Vgl.][\S\S~127--129]{Wolff:Discursuspraeliminarisdephilosophiaingenere1996},
sowie \cite[][\S\S~606--610]{Wolff:PhilosophiarationalissiveLogica1740}.}}
Sobald bloß vorläufige Urteile unter unseres Überzeugungen nicht mehr als solche
kenntlich sind, laufen wir Gefahr, sie zu Prämissen unserer Schlüsse zu machen
und damit die Zuverlässigkeit unseres gesamten Wissensgebäudes zu unterminieren.

Nach \name[Immanuel]{Kant} wiederum ist ein vorläufiges Urteil -- eine Meinung
-- ein Vorurteil, wenn es als Grundsatz unseres Schließens
gebraucht wird. Gerade daraus erhellt, warum Meinungen nicht Gegenstand von
Behauptungen sein dürfen: Wir verbreiten Vorurteile, wenn wir Meinungen im Modus
des Behauptens aussprechen und bezüglich einer Meinung \enquote{$p$} nicht
sagen, wir hielten es für möglich oder wahrscheinlich, dass $p$, sondern
schlicht behaupten, dass $p$. \Revision[Heidemann]{Alles, was Grundlage für
einen weiteren Erkenntniserwerb sein soll, muss gewusst werden, und zwar rationale Erkenntnisse
in rationaler und empirische Erkenntnisse in historischer Form. So lautet die
oberste Anweisung mündigen Denkens und Urteilens; wer dieser Anweisung nicht
nachkommt, der verfängt sich in Vorurteilen.}


\section{Von anderen
lernen}\label{subsubsection:EndlichesundUnendlichesErkennen}
Der wichtigste Unterschied zwischen Wissen und Meinen dürfte
unumstritten sein: Wissen kann und darf ich mitteilen, Meinungen soll ich für
mich behalten. Man könnte genauer sagen: Dann und nur dann wenn Person $A$
\emph{weiß}, dass $p$ (also \emph{objektiv gültige} Gründe für $p$ hat), dann
ist sie auch berechtigt, anderen mitzuteilen, dass $p$. Handelt es sich nicht um
Wissen, besitzt Person $A$ also keine objektiv zureichenden Gründe für $p$, dann
kann und darf sie natürlich auf ihre eigene Überzeugung verweisen. $A$ darf
sagen, dass sie vermutet, dass $p$, oder auch, sie sei sich sehr sicher, dass
$p$ (je nachdem, wie viele Anhaltspunkte sie dafür hat, dass $p$). Aber es
liegen Welten zwischen der einfachen Aussage, dass $p$, und der Aussage, man sei
sich sicher, dass $p$. \authorfullcite{Austin:OtherMinds1979} hat dies so
erläutert, dass er sagt, Behaupten sei wie das Geben von Versprechen; wenn ich
sage \enquote{Ich weiß, dass $p$}, dann \emph{verspreche} ich gleichsam die
Wahrheit von \enquote{$p$} und ein Anderer kann {$p$} auf \emph{meine}
Verantwortung hin glauben und weitersagen. Möchte ich diese Verantwortung nicht übernehmen, kann ich sagen,
dass ich \emph{mir sicher} bin (\name[Immanuel]{Kant} sagt -- im Kontext des
Glaubens -- \emph{ich} sei gewiss, statt \emph{es} sei
gewiss\footnote{\cite[Vgl.][B 857]{Kant:KritikderreinenVernunft2003},
\cite[][III: 537.1--2]{Kant:GesammelteWerke1900ff.}.}); der Andere mag es
immer noch auf meine Aussage hin glauben, aber er glaubt es dann auf
\emph{seine eigene}
Verantwortung.\footnote{\cite[Vgl.][98--103]{Austin:OtherMinds1979}.
\enquote{When I have said only that I am sure, and prove to have been mistaken,
I am not liable to be rounded on by others in the same way as when I have said
\enquote{I know}. I am sure \ori{for my part}, you can take it or leave it:
accept it if you think I'm an acute and careful person, that's your
responsibility} (\cite[][100]{Austin:OtherMinds1979}).}

Wer etwas behauptet und sagt, er wisse etwas, teilt nicht einfach mit, dass er
selbst die entsprechende wahre und adäquat begründete Überzeugung hat, sondern
übernimmt  Verantwortung für diese Überzeugung. Er \textit{verspricht}, eine
passende Begründung zu haben, \emph{verpflichtet} sich zur Angabe von Gründen
und gibt seinen Zuhörern die \textit{Berechtigung}, es \emph{auf seine
Verantwortung} hin zu wiederholen. \Revision[Heidemann]{Die Verantwortlichkeit
auf Seiten des Mitteilenden ist damit klar. Wie aber muss sich der Empfänger von
Informationen verhalten, um von dem Wissen anderer zu lernen und gleichzeitig
seiner eigenen Verantwortung für das \emph{eigene} Überzeugungssystem -- also
der \emph{ethics of belief} -- gerecht zu werden?}

\subsection{Gründe mitteilen}
Wir lernen von anderen, wenn sie uns ihr Wissen mitteilen. Das ist zumindest bei
empirischen Erkenntnissen für \name[Immanuel]{Kant} ganz unproblematisch. Wenn
Ingrid gesehen hat, dass etwas der Fall ist, und daraufhin Max mitteilt, dass es
der Fall ist, dann weiß Max hinterher genau so gut wie Ingrid, dass es der Fall
ist. Bei empirischen Erkenntnissen zählen eigene Wahrnehmungen und Mitteilungen
als objektive Gründe. Anders verhält es sich bei rationalen Erkenntnissen. Wenn
Ingrid aufgrund eigener Überlegungen \emph{weiß}, dass der Kompaktheitssatz aus
dem Endlichkeitssatz der Folgebeziehung folgt oder dass
\authorcite{Wolff:Discursuspraeliminarisdephilosophiaingenere1996}s Beweis der
Satzes vom zureichenden Grund aus dem Prinzip des ausgeschlossenen Widerspruchs
ungültig ist, dann hat sie objektiv zureichende Gründe für ihre Überzeugungen
und deswegen mathematisches beziehungsweise philosophisches Wissen. Wenn sie nun
Max mitteilt, dass der Kompaktheitssatz aus dem Endlichkeitssatz der
Folgebeziehung folgt und
\authorcite{Wolff:Discursuspraeliminarisdephilosophiaingenere1996}s Beweis
nicht gültig ist, dann zählt dies jedoch nicht als objektiv zureichend und Max
erwirbt kein Wissen (zumindest kein mathematisches und philosophisches,
sondern lediglich historisches). Kann Max nun dennoch von Ingrid lernen?

Wie schon bei \authorcite{Wolff:Psychologiaempirica1968} ist es auch im
Kontext der Systematik \name[Immanuel]{Kant}s nicht \emph{per se} verwerflich, historische
Erkenntnis der philosophischen Erkenntnis anderer zu
haben. Wenn wir uns mit der Geschichte der Philosophie befassen, dann sammeln
und systematisieren wir gerade solche Erkenntnisse. Damit verfügen wir natürlich
über philosophiehistorisches und nicht über philosophisches Wissen. In den
\titel{Prolegomena} schreibt \name[Immanuel]{Kant}: \enquote{Es gibt Gelehrte,
denen die Geschichte der Philosophie (der alten sowohl, als neuen) selbst ihre
Philosophie ist}\footnote{\cite[][A
3]{Kant:ProlegomenazueinerjedenkuenftigenMetaphysikdiealsWissenschaftwirdauftretenkoennen1977},
\cite[][IV: 255.5--6]{Kant:GesammelteWerke1900ff.}.}. Eine solche Haltung bleibt
hinter der Forderung der Aufklärung zurück. Zwar spricht nichts gegen den Erwerb
historischer Erkenntnisse, sie dürfen die rationalen Erkenntnisse jedoch nicht
als solche ersetzen; die Geschichte der Philosophie soll nicht mit der
Philosophie selbst verwechselt werden.

Unproblematisch ist natürlich das Lernen von anderen durch Kontrolle der
Vernünftigkeit eigener Gedanken. Angenommen, Max versucht, den Beweis des
Kompaktheitssatzes selbst zu führen und
\authorcite{Wolff:Discursuspraeliminarisdephilosophiaingenere1996}s Beweis
selbst zu widerlegen. Wenn er sich nun fragt, ob seine Gründe objektiv
zureichend sind, wendet er sich an Ingrid, die die Kompetenz besitzt, dies zu
entscheiden. Aber auf diese Weise würde er nie in größerem Umfang mathematisches
und philosophisches Wissen erwerben. Die Entdeckung solcher Beweise wie des
Kompektheitssatzes sind großartige Leistungen, die den wenigsten gelingen
können und die ein großes Maß an Originalität und Kreativität verlangen. Es geht
aber gerade nicht um Originalität; rationales Wissen kann ebenso derjenige
erlangen, der die Beweise von anderen lernt. Welchen Status mathematisches
Wissen hat, ob es sich um bloß historische Kenntnisse oder um Kompetenz und
Einsicht handelt, ist unabhängig davon, wie dieses Wissen ursprünglich erworben
wurde. Der Unterschied besteht in der aktuellen Kompetenz, nicht im vergangenen
Erwerb. Und auch bei philosophischem Wissen ist nicht der Ersterwerb
entscheidend, sondern die ausgebildete Kompetenz.

Max kann also Wissen erwerben, wenn Ingrid ihm nicht nur die Ergebnisse ihres
Nachdenkens mitteilt, sondern zugleich die (objektiv zureichenden) Gründe nennt,
die zeigen, dass ihre Überzeugungen wahr sind. Wenn sie Max die Beweise liefert
(und Max die Beweise versteht und nachvollzieht), dann erwirbt auch Max
entsprechendes Wissen. Und er mag den kategorischen Imperativ zunächst in
all seinen Formulierungen auswendig lernen, ohne ihn anwenden oder auch kritisch beurteilen
zu können. Später aber erwirbt er tieferes Verständnis und stellt fest, dass er
die Überzeugung, die er zunächst nur verbal reproduzieren konnte, weiterhin
teilt. Er ist dann selbst Philosoph geworden, wenngleich er die Überzeugungen,
die er hat, nicht selbst entdeckte. Somit kann eine Erkenntnis, die wir als
historische Erkenntnis erwerben, auch bei uns (subjektiv) zu einer eigenständig
philosophischen Erkenntnis werden.

Umgekehrt kann eine Überzeugung, die wir als eigenständige philosophische oder
mathematische Erkenntnis erworben haben, diesen Status auch wieder verlieren.
Denn mitunter verfügen wir nach einiger Zeit nicht mehr über die Kompetenz, die
Begründungen zu rekonstruieren. Auch wenn wir einst mathematische Theoreme
mitsamt ihrer Beweise gelernt haben sollten, muss uns der Beweis heute nicht
mehr verfügbar sein. Ingrid mag den Kompaktheitssatz selbst bewiesen haben. Aber
wenn sie sich nun nicht mehr an den
Beweis erinnert, hat sie dann ebenso lediglich historische Kenntnis wie Max?
\authorcite{Burge:ContentPreservation1993} nimmt an, dass eine objektiv
rationale Erkenntnis auch subjektiv rational ist, wenn jemandem die Gründe
nicht mehr einfallen.\footnote{\enquote{Although nothing in
\singlequote{Content Preservation} commits me on this matter, I believe that a
person clearly \ori{can be} entitled to believe a theorem she believes because
of preservative memory even if she cannot remember the proof she gave long ago,
and even if she cannot remember that she gave a proof. Most of what one is
entitled to believe from past reading, past interlocution, past reasoning, or
past empirical learning, derives from sources and warrants that one has
forgotten} \parencite[][38]{Burge:InterlocutionPerceptionandMemory1997}.}
\name[Immanuel]{Kant} scheint auf die gegenteilige Position festgelegt zu sein
und sagen zu müssen, dass nur subjektiv rationale Erkenntnis hat, wer die Gründe
dieser Erkenntnis zu nennen und zu explizieren vermag.

\subsection{Gründe und Kompetenzen}
Interessant ist hier  insbesondere folgende Überlegung zum Status der
Mathematik als Wissenschaft gegenüber der Philosophie: Beide sind keine
empirischen Erkenntnisse, also kein tradierbares Wissen, von dem es legitim ist,
bloß historisches Wissen zu haben. Einen beachtlichen Vorteil gegenüber der
Philosophie habe die Mathematik aber, insofern sie leichter \emph{lehr-} und
\emph{lernbar} sei und es bei ihr nicht -- oder nicht so oft -- den bedenklichen
Fall einer subjektiv bloß historischen Erkenntnis objektiv rationaler -- hier:
mathematischer -- Wahrheiten gebe. \Revision[Heidemann]{Es ist also im Bereich
der Mathematik leichter, Mündigkeit zu erwerben; mündiges mathematisches Urteilen wird durch
den Unterricht der Mathematik begünstigt.}

Nun sei es bei mathematischem Wissen viel leichter als bei philosophischen
Erkenntnissen, sie so mitzuteilen, dass der Rezipient tatsächlich rationales
Wissen erwirbt und nicht lediglich historische Kenntnis:
\begin{quote}
 Es ist aber doch sonderbar, daß das mathematische Erkenntnis, so wie man es
erlernet hat, doch auch subjektiv für Vernunfterkenntnis gelten kann {\punkt}.
Die Ursache ist, weil die Erkenntnisquellen, aus denen der Lehrer allein schöpfen
kann, nirgend anders als in den wesentlichen und echten Prinzipien der Vernunft
liegen, und mithin von dem Lehrlinge nirgend anders hergenommen, noch etwa
gestritten werden können, und dieses zwar darum, weil der Gebrauch der Vernunft
hier nur in concreto, obzwar dennoch a priori, nämlich an der reinen, und eben
deswegen fehlerfreien, Anschauung geschieht, und alle Täuschung und Irrtum
ausschließt.\footnote{\cite[][B~865]{Kant:KritikderreinenVernunft2003}, \cite[][III:
541.24--34]{Kant:GesammelteWerke1900ff.}.}
\end{quote}
Es gebe in der Mathematik keine anderen Erkenntnisquellen, auf die wir
verfallen können, als die Vernunft. Jedes Argument in der Mathematik sei
entweder objektiv gültig oder offensichtlich unzureichend.
Unmöglich hingegen sei es, dass andere Quellen sich unbemerkt einschleichen, die
Mathematik ist also nicht anfällig für Vorurteile. Aber warum ist sie nicht
anfällig für Vorurteile? Dies liegt -- so sagt \name[Immanuel]{Kant} -- daran,
dass die Vernunft zwar \emph{a priori} (wie die Philosophie), aber \emph{in
concreto} gebraucht werde. Und dies wiederum schließe Täuschung und Irrtum aus.
Wir müssen also auf den \emph{formalen} Unterschied zwischen Mathematik und
Philosophie achten, d.\,i. auf den Unterschied in der Art und Weise, wie
entsprechende Erkenntnisse generiert werden. Gerade dies nennt
\name[Immanuel]{Kant} ja die \emph{Form} einer Erkenntnis (siehe Kapitel
\ref{section:MuendigkeitundPhilosophie}).



Der Mathematik kommt in \name[Immanuel]{Kant}s Darstellung gegenüber der
Philosophie insofern ein besonderer Status zu, als er sie im expliziten Kontrast
zur Philosophie als Wissenschaft bezeichnet, obwohl sie keine Erfahrungswissenschaft ist.
Sie sei \enquote{von den frühesten Zeiten her {\punkt} den sichern Weg einer
Wissenschaft gegangen}\footnote{\cite[][B x]{Kant:KritikderreinenVernunft2003},
\cite[][III: 9.7--9]{Kant:GesammelteWerke1900ff.}.} und wegen dieser
Erfolgsgeschichte einer reinen Vernunftwissenschaft
mögliches Vorbild auf dem Weg zur Wissenschaftlichkeit auch der
Philosophie.\footnote{\cite[Vgl.][\S\S~5\,f.,]{Kant:ProlegomenazueinerjedenkuenftigenMetaphysikdiealsWissenschaftwirdauftretenkoennen1977}
\cite[][IV: 279.15--280.32]{Kant:GesammelteWerke1900ff.}, sowie
\cite[][B 14--17]{Kant:KritikderreinenVernunft2003},
\cite[][III: 36.14--38.24]{Kant:GesammelteWerke1900ff.}.} Damit referiert er
freilich die Hoffnung vieler Autoren der Neuzeit, die Wissenschaftlichkeit der
Philosophie durch Orientierung an der methodischen Strenge der Mathematik zu
gewährleisten.
\authorcite{Wolff:Discursuspraeliminarisdephilosophiaingenere1996} ist dabei
derjenige, den \name[Immanuel]{Kant} stets vor Augen hatte (siehe Kapitel
\ref{subsection:SelbstdenkenbeiKant}). Doch verwirft er gerade diese Hoffnung
und ersetzt sie durch die Forderung, die Vernunft gerade in Fragen der
Philosophie durch Intersubjektivität zu fundieren (Kapitel
\ref{section:sensuscommunis}).

Der primäre Grund für die Lehrbarkeit der Mathematik liegt nicht darin,
dass man nichts Falsches glaubt, weil die Mathematik eine mit besonderer
Gewissheit vorgehende Disziplin ist, sondern darin, dass die Grundlagen in den
\enquote{echten Prinzipien der Vernunft} liegen, die dann auch die je eigenen
sind. Das bloße Lernen einer Formel zur Berechnung ohne die entsprechenden
Herleitungen führt jedenfalls nach \name[Immanuel]{Kant} nicht zum adäquaten
Erlernen der Mathematik -- trotz ihrer Sicherheit, die dann möglicherweise
größer sein mag, als wenn ich selbst die Herleitung vorgenommen
habe.\footnote{\cite[Vgl.][A 105]{Kant:ImmanuelKantsLogik1977}, \cite[][IX:
69.8--10]{Kant:GesammelteWerke1900ff.}: \enquote{Mathematische
Vernunftwahrheiten kann man auf Zeugnisse zwar glauben, weil Irrtum hier teils
nicht leicht möglich ist, teils auch leicht entdeckt werden kann; aber man kann
sie auf diese Art doch nicht wissen.} Siehe auch
\cite[][\nopp 1631]{Kant:Reflexionen1900ff.},
\cite[][XVI: 52.2--3]{Kant:GesammelteWerke1900ff.}, sowie
\cite[][\nopp 2471]{Kant:Reflexionen1900ff.},
\cite[][XVI: 384.8--10]{Kant:GesammelteWerke1900ff.}.}


Dass Mathematik leichter lehrbar ist, liegt darin begründet, dass ein Unterricht
der Mathematik so eingerichtet sein könne und müsse, dass die lernende Teilnahme
darin gerade das selbständige Mitdenken der rationalen Inhalte sei. Wer
Mathematik lernt, memoriert nicht vorgefertigte Lösungswege, um sie im richtigen
Moment zu reproduzieren, sondern übt mit fortlaufend anderen, aber
ähnlichen Aufgaben, Lösungen selbst zu finden. Es ist natürlich nicht
garantiert, dass ein Unterricht der Mathematik ihrer Natur als rationaler
Wissenschaft adäquat ist, aber es sei doch leichter möglich, als in der
Philosophie. Ein Philosophieunterricht mag -- unter ungünstigen, aber häufigen
Umständen -- so eingerichtet sein, dass der Schüler bedeutende Argumente aus der
Philosophiegeschichte lernt, aber keine Methodenkompetenz erwirbt, selbständig
zu philosophieren. Aber ein Mathematikunterricht nimmt nicht die Form an, dass
ein Schüler nur Beweise auswendig lernt. Tut er dies doch, so erwirbt der
Schüler nur ein historisches Wissen von einem mathematischen Tatbestand.

Die Methodenorientierung ist freilich kein Alleinstellungsmerkmal der
Mathematik, sondern wünschenswert in \emph{jeder} wissenschaftlichen Disziplin,
auch den empirischen Wissenschaften. Der Physiker
soll die Gesetze der Physik nicht \emph{auswendig} kennen, sondern sie anwenden,
ihre Begründung nennen und sie bei widerstreitender Erfahrung entsprechend
modifizieren können. Er wird zum Forscher und Anwender ausgebildet, nicht zur
Formelsammlung.\footnote{Dasselbe gilt auch noch mit gewissen Einschränkungen
für die Vertreter der oberen Fakultäten, also Juristen, Mediziner und Theologen.
Auch sie sollen zur Anwendung ihrer Erkenntnisse ausgebildet werden, wenngleich
hier der Forschungsaspekt deutlich zurücktritt. Generell steht bei den oberen
Fakultäten der Anwendungs- und bei den Fächern der unteren Fakultät
(Naturwissenschaft, Mathematik, Philosophie) der Forschungsaspekt im
Vordergrund. \kantcite{Kant:DerStreitderFakultaeten1977}{Vgl.}{A 6--10}{VII:
18.30--20.11}.} Im Unterschied zur Mathematik kann es dabei in
empirischen Wissenschaften auch vorkommen, dass Inhalte gelehrt
werden, die schlicht als testimoniales Wissen gelernt werden,
weil sie auf Erfahrungen beruhen, die nicht jeder selbst machen kann oder muss.
In der Mathematik kommt solches empirisches Wissen wie in der (reinen)
Philosophie kaum\footnote{In beiden Fällen beinhaltet die Ausbildung notwendiger
Weise auch historisches Wissen um die Geschichte des eigenen Faches. Dieses mag
umfangreicher (Philosophie) oder geringer (Mathematik) ausfallen, aber
letztlich muss doch selbst der Mathematiker wenigstens die jüngere
Vergangenheit seines Faches zur Kenntnis nehmen, um auch nur relevante Fragen
der Forschung ausmachen zu können.} vor, weswegen der Unterricht in Mathematik
wie in Philosophie idealiter kein testimoniales Wissen generiert. Nur generiert
die Mathematik tatsächlich auch im Unterricht subjektiv rationales Wissen,
während die Philosophie dazu neigt, sich in historischer Erkenntnis zu
verfangen.

Wieso gelingt der Philosophie nicht, was in der Mathematik so
selbstverständlich ist? Der Grund dafür ist, dass
die Mathematik sich nach \name[Immanuel]{Kant} leichter auf eine gemeinsame
Grundlage beziehen kann, da in ihr Definitionen und Axiome und damit Beweise
möglich sind, die jeder entsprechend ausgebildete Zuhörer selbst nachvollziehen und kontrollieren
kann.\footnote{\cite[][B 754--766, insb. B
754\,f.,]{Kant:KritikderreinenVernunft2003} \cite[][III: 477.5--483.32, insb.\
477.5--8]{Kant:GesammelteWerke1900ff.}: \enquote{Die Gründlichkeit der Mathematik beruht auf Definitionen, Axiomen, Demonstrationen. Ich werde mcih
damit begnügen, zu zeigen: daß keines dieser Stücke in dem Sinne, darin sie der
Mathematiker nimmt, von der Philosophie könne geleistet, noch nachgeahmet
werden.}} Entscheidend ist, dass es seiner Darstellung nach nicht auf eine
vermeintliche Korrekturresistenz ankommt, sondern darauf, dass jeder ihre
Wahrheit unmittelbar \emph{selbst} einsehen könne. Und dies liege wiederum -- im
Falle der Axiome und Postulate -- an der Möglichkeit einer Darstellung in der
reinen Anschauung. Dass die Lehrbarkeit der Philosophie nicht in derselben Weise
gegeben ist, liegt also an der Unverfügbarkeit dessen, was \name[Immanuel]{Kant}
intellektuelle Anschauung nennt.

Philosophische Erkenntnis haben wir nur als \emph{diskursive} rationale
Erkenntnis; und in dieser Diskursivität unseres Erkennens besteht gerade die
Eigenart des endlichen Denkens.
Verfügten wir in der Philosophie über die Möglichkeit eines
\emph{intuitiven} rationalen Erkennens analog der reinen Anschauung in der
Mathematik, so wäre unser Denken
nicht endlich. Nun basiert \name[Immanuel]{Kant}s Definition mathematischer
Erkenntnis damit natürlich auf der in der transzendentalen Ästhetik der \titel{Kritik der reinen Vernunft}
entfalteten Theorie mathematischen Wissens, die heute kaum jemand ohne weiteres
zu akzeptieren bereit ist. In dem hier zu verhandelnden Zusammenhang
interessiert die mathematische Erkenntnis jedoch nur als Kontrastfolie, die die
Eigenart philosophischer Erkenntnis zu verdeutlichen helfen soll.\footnote{Wir
können dabei die Frage, ob \name[Immanuel]{Kant} eine überzeugende Theorie
mathematischen Erkennens anbietet, daher außer Acht lassen. Ob es so etwas wie
eine reine Anschauung (von Raum und Zeit) gibt und was unter einer solchen
Anschauung zu verstehen ist, das braucht uns nicht zu beschäftigen. Denn
philosophisches Wissen kann auf eine solche Anschauung nach
\name[Immanuel]{Kant} nicht zurückgreifen. Und nur dies interessiert; die reine
Anschauung der Mathematik könnten wir ebenso als bloß gedachten Kontrast zum
selben Zweck verwenden.}

Die Philosophie sei in dem Zustand, den \name[Immanuel]{Kant} vorzufinden
behauptet, keine Wissenschaft, weil ihr die gemeinsame Basis fehlt, so dass
philosophische Argumente auf Prämissen beruhen, die von der Anerkennung durch
uns und unsere Gesprächspartner abhängen. Sie verfährt daher viel stärker \emph{ad
hominem}, denn die Überlegungen, die auf unsicheren Axiomen aufbauen, sind nur
für denjenigen überzeugend, der die entsprechenden Grundüberzeugungen teilt. Der
Streit philosophischer Schulen und schließlich die Gegenreaktion des
Skeptizismus sind \name[Immanuel]{Kant}s Belege für diese Diagnose.

Wenn wir die Grundlagen philosophischen Vorgehens jedoch nicht nachvollziehen
können, erscheint uns jede Prämisse, die nicht weiter begründet wird, als
\singlequote{innere Eingebung}. Wollte Max von Ingrid lernen, und
verwendete Ingrid in ihren Argumentationen Prämissen, deren Gültigkeit Max nicht
selbst kontrollieren kann, so bliebe ihm nichts anderes übrig, als Ingrids Wort
als oberstes Kriterium der Wahrheit anzusehen. Die Tatsache, dass er nicht
selbst die Gültigkeit der Behauptungen kontrollieren kann, bedingt damit seine
Abhängigkeit von Ingrids Autorität und damit seine eigene Unmündigkeit. Das war
\authorcite{Wolff:Discursuspraeliminarisdephilosophiaingenere1996}s wichtige
Einsicht.\footnote{Siehe Kapitel
\ref{Abschnitt:WolffunddieWissenschaftlichkeitderPhilosophiemoregeometrico}.}
Daraufhin müssten wiederum -- sollte Ingrid schulbildend wirken --
\enquote{aus inneren Eingebungen durch Zeugnisse äußere bewährte Facta, aus
Traditionen, die anfänglich selbst gewählt waren, mit der Zeit aufgedrungene
Urkunden, mit einem Worte die gänzliche Unterwerfung der Vernunft unter Facta,
d.\,i. der \ori{Aberglaube} entspringen}\footnote{\cite[][A 327]{Kant:Washeisst:SichimDenkenorientieren?1977}, \cite[][VIII:
145.30--34]{Kant:GesammelteWerke1900ff.}.}. Das Problem der unzureichenden
Fundierung für die Lehrbarkeit der Philosophie besteht also nicht in der Gefahr,
falsche Sätze für wahr zu halten, sondern darin, dass es letzte Sätze gibt, die
nur durch die Autorität des Lehrers bezeugt sind. Es ist also die Unfähigkeit
des Lehrers, Gründe für alle seine Behauptungen anzugeben, die die Lehr- und
Lernbarkeit der Philosophie unterminieren.

Gerade darin liegt aber die Schwierigkeit in dem Bestreben, Philosophie als
auch subjektiv rationales Wissen zu erwerben. Weil sich die echten Quellen der
Vernunft nicht so leicht identifizieren lassen, verfallen wir schnell der
Versuchung, grundlegende Urteile (Prinzipien oder \singlequote{Grundsätze}) von
Autoritäten zu übernehmen.
Denn -- so lehrte uns schon \authorcite{Wolff:Psychologiaempirica1968}\footnote{Siehe oben auf
S.~\pageref{Stellenverweis:Wolff:SelbstaendigkeitnurdurchKompetenz}.} --  nur dadurch, dass wir
selbst zu fundierten, vernünftigen Einsichten gelangen, können wir vermeiden,
unser Fürwahrhalten an zufälligen Einflüssen statt an der Vernunft auszurichten.
Der bloße Anspruch auf Selbständigkeit führt nur zu einem Schein des
Selbstdenkens, \Revision[Heidemann]{zur Verwendung historischer Kenntnisse
philosophischer Urteile als Grundsätze des weiteren Erkenntniserwerbs. Die rationalen Grundlagen unseres
Denkens selbst zu verantworten macht aber gerade Aufklärung und Mündigkeit aus.
Solche Grundlagen selbst verantworten zu können heißt wiederum, über
entsprechende methodische Kompetenzen in der Philosophie zu verfügen.
Intellektuelle Tugendhaftigkeit äußert sich im Bemühen um den Erwerb
philosophischer Kompetenzen. Der bequemere Weg bestünde freilich darin, einfach
die Grundsätze anderer unkontrolliert zu übernehmen. So mag beispielsweise auch
derjenige, der von
\authorcite{Wolff:Discursuspraeliminarisdephilosophiaingenere1996},
\authorcite{Meier:AuszugausderVernunftlehre1752} und
\authorcite{Reimarus:DieVernunftlehrealseineAnweisungzumrichtigenGebrauchderVernunftinderErkenntnisderWahrheit1756}
die richtigen Grundsätze des Umgangs mit testimonialem Wissen erlernt hat, in
der Folge korrekt und kritisch urteilen. Aber aufgeklärt ist er erst, wenn er
über diese Grundsätze selbst kompetent urteilen und diskutieren kann, statt
sich nur auf die genannten Autoren oder eine \singlequote{\emph{best practice}}
zu berufen.}

Beiden -- Philosophie und Mathematik -- geht es letztlich nicht um den Erwerb
von Tatsachenwissen, sondern von philosophischen und mathematischen Kompetenzen.
Derjenige, der beispielsweise die Axiome der Klassischen Aussagenlogik bloß
benennen, aber nicht anwenden, geschweige denn ihre Korrektheit beweisen kann,
hat gar kein mathematisches Wissen. Und ebenso hat derjenige kein
philosophisches Wissen, der die verschiedenen Formulierungen des kategorischen
Imperativs zwar auswendig aufsagen, sie aber weder anwenden, noch vernünftig
darüber diskutieren kann, ob es sich um einen guten Vorschlag zur Grundlegung
der Ethik handelt. \Revision[Heidemann]{Und das ist eben der Kern der Auflösung
des vermeintlichen Widerspruchs zwischen der Forderung nach Selbstdenken und
unserer Abhängigkeit von anderen. \name[Immanuel]{Kant}s \emph{ethics of
belief} fordert nicht, von Anderen epistemisch und kognitiv unabhängig zu
werden. Sie fordert aber im Bereich rationaler Erkenntnisse die eigenständige
vernünftige Kontrolle über die Korrektheit unserer Überzeugungen zu erwerben.
Der Mathematik gelinge dies besser als der Philosophie, weil sie über eine
feste Grundlage verfüge und sich auf dieser Grundlage mathematische Kompetenzen
leichter vermitteln ließen, zumal diese in der Fähigkeit zur Befolgung einer
bestimmten Methode bestünden. Aufgrund des unterschiedlichen Charakters der
jeweiligen epistemischen Grundlagen könne die Mathematik aber nur scheinbar als
Vorbild für philosophisches Vorgehen dienen.}



\begin{comment}
\subsection{Vernunft und Gedächtnis}

\authorfullcite{Kripke:NameundNotwendigkeit1981} hält \name[Immanuel]{Kant}s
Unterscheidung von rationalen und empirischen Erkenntnissen für problematisch,
weil es doch sein könnte, dass wir von rationalen Erkenntnissen empirisches Wissen
erwerben. Wir nutzen beispielsweise einen Taschenrechner, um eine Rechnung
auszuführen, oder einen Computer, um zu ermittelt, ob eine bestimmte Zahl $k$
eine Primzahl ist. Damit beruht unser Wissen, dass $k$ eine Primzahl ist, jedoch
nicht auf rein rationalen Überlegungen, sondern auf empirischen Erkenntnissen
über das Funktionieren des
Computers.\footnote{\authorcite{Kripke:NameundNotwendigkeit1981} behauptet
explizit: \enquote{Wenn wir glauben, daß die Zahl eine Primzahl ist, glauben
wir es also aufgrund unseres Wissens von den Gesetzen der Physik, der
Konstruktion des Computers usw.}
\parencite[][45]{Kripke:NameundNotwendigkeit1981}.
\authorfullcite{Tymoczko:TheFour-ColorProblemanditsPhilosophicalSignificance1979}
schreibt: \enquote{The most natural interpretation {\punkt} is that
computer-assisted proofs introduce experimental methods into pure mathematics}
\parencite[][58]{Tymoczko:TheFour-ColorProblemanditsPhilosophicalSignificance1979}.
\authorfullcite{Burge:ContentPreservation1993} wiederum widerspricht und
behauptet, dass die Benutzung eines Computers nicht als Einsatz eines
empirischen Verfahrens zu verstehen sei, sondern als Erweiterung der eigenen rationalen Fähigkeiten. Die
Arbeitsweise eines solchen Instruments entspringe den mathematischen Kompetenzen
des Entwicklers und sei daher als Ausdruck vernünftiger Tätigkeit anzusehen
\parencite[vgl.][31]{Burge:ComputerProofAprioriKnowledgeandOtherMinds:TheSixthPhilosophicalPerspectivesLecture1998}.}
Wir sollten daher vielleicht gar nicht davon sprechen, ob eine Erkenntnis
\emph{a priori} ist, sondern davon, ob eine bestimmte Person etwas
\emph{a priori} oder \emph{a posteriori}
erkennt.\footcite[Vgl.][\pno~44\,f.]{Kripke:NameundNotwendigkeit1981}

Wir haben gesehen, dass \name[Immanuel]{Kant} beide Verwendungen kennt und
unterscheidet.

Nun wurde der Erwerb mathematischen Wissens durch Computer in den letzten
Jahrzehnten zu einem viel diskutierten Problem der Philosophie der Mathematik,
von dem \name[Immanuel]{Kant} freilich noch nichts wissen konnte. Die
Möglichkeit, mathematische Beweise von Computern führen zu lassen, bringt jedoch
eine Herausforderung an die hier dargestellt Position mit sich, deren Behandlung
durchaus erhellend ist.

Mathematische Beweise gehen in aller Regel schrittweise vor. Mitunter beweisen
wir erst einige Hilfssätze, bevor wir uns dem Beweis des eigentlichen Theorems
zuwenden. Und wenn das Theorem bewiesen ist, ist es häufig schwer, den gesamten
Beweis zu überblicken. Wir erinnern uns aber an die Ergebnisse der letzten
Beweisschritte und knüpfen an diese an.
\authorcite{Locke:TheWorksofJohnLocke1963} sagt daher, dass ein
Beweis, der in mehreren Schritten vorgeht, zweifelhafter und irrtumsanfälliger
sei, als eine \singlequote{intuitive} mathematische Gewissheit, also ein
einzelner Beweisschritt.

Nun argumentiert \authorfullcite{Chisholm:Erkenntnistheorie1979}, dass wir uns
bei der Rechtfertigung einer Behauptung mittels eines Beweises, der über mehrere
Beweisschritte geht, auch auf empirische Annahmen über unser Gedächtnis stützen
müssen und deswegen nicht sagen sollten, dass wir die Behauptung \emph{a priori}
gerechtfertigt
hätten.\footcite[Vgl.][\pno~72\,f.]{Chisholm:Erkenntnistheorie1979}
Nach \authorcite{Chisholm:Erkenntnistheorie1979} ist das Wissen um eine Tatsache
mittels der Erinnerung offenbar ein Fall inferenziellen Wissens: Wenn ich weiß,
dass ich den Herd ausgeschaltet habe, dann weiß ich zunächst, dass meine
Erinnerung mir sagt, ich hätte ihn ausgeschaltet. Mittels der Prämisse, dass
meine Erinnerung mir meistens Dinge sagt, die tatsächlich getan oder
wahrgenommen habe, \emph{schließe} ich dann darauf, dass ich den Herd
ausschaltete. Die Prämissen sind offenbar empirisch. Und wenn ich mich erinnere,
den Kompaktheitssatz bewiesen zu haben, dann verfahre ich analog und weiß
\emph{nun empirisch}, dass eine Aussagenmenge erfüllbar ist, wenn jede ihrer
endlichen Teilmengen erfüllbar ist. Vorhin -- als ich ihn bewies -- mag ich dies
\emph{a priori} gewusst haben. Nun weiß ich ihn (in \name[Immanuel]{Kant}s
Terminologie) bloß noch historisch.

Diese Konsequenz ist freilich nicht tragbar -- schon gar nicht aus
\name[Immanuel]{Kant}ischer Perspektive --, denn sie vereitelte jede
Möglichkeit, in nennenswertem Umfang rationales Wissen zu erlangen. Für
\authorfullcite{Hume:AnEnquiryConcerningHumanUnderstanding1964} ist die
Erinnerung neben der momentanen Wahrnehmung eine völlig unproblematische
Wissensquelle.\footnote{Die Leitfrage des \titel{Enquiry Concerning Human
Understanding} lautet entsprechend, wie gelangen wir zu fundierter Erkenntnis,
die über das momentane Zeugnis der Sinne und unser Gedächtnis hinausgeht:
\enquote{It may,therefore, be a subject worthy of curiosity, to enquire what is
the nature of that evidence, which assures us of any real existence and matter
of fact, beyond the present testimony of our senses, or the records of our
memory} \parencite[][23]{Hume:AnEnquiryConcerningHumanUnderstanding1964}.} Es
erscheint sogar geboten, sie gar nicht als genuine Wissens\emph{quelle} zu
bezeichnen, denn sie generiert kein Wissen, sondern bewahrt lediglich dasjenige
Wissen, welches wir zuvor erlangt hatten. \authorcite{Burge:ContentPreservation1993} bezeichnet die
Erinnerung als \enquote{content preserving}.
\footcite[Vgl.][]{Burge:ContentPreservation1993} Das Gedächtnis sei wie
ein logischer Schluss, insofern es das Fortschreiten der Vernunft ermöglicht,
ohne propositionalen Gehalt beizutragen. Es unterscheide sich von einem Schluss
darin, dass es kein Übergang und keine Bewegung ist, also auch kein Element
einer Begründung darstellt.

\authorfullcite{Burge:ContentPreservation1993} schlägt nun vor, die Weitergabe
einer Erkenntnis von einer Person zu einer anderen nicht in Analogie zu eigener
Wahrnehmung, sondern in Analogie zu der eigenen Erinnerung zu verstehen. Wenn
Ingrid gestern gesehen hat, dass Paul in der Stadt war, und sich heute daran
erinnert, dann bewahrt diese Erinnerung sowohl den Gehalt ihrer Überzeugung
(\enquote{Paul war in der Stadt}) als auch ihren Status als empirisches Wissen.
Wenn sie nun Max ihre Überzeugung mitteilt, dann geschehe dasselbe: Es werde
sowohl der Gehalt als auch der Status bewahrt. Die Mitteilung funktioniert also
wie ein Gedächtnis, das sich nicht auf eine Person beschränkt (und wir
sprechen ja auch von Büchern als dem Gedächtnis der Menschheit).

Und in der Tat unterscheiden wir nicht die Fälle, in denen Ingrid sich einen
Schritt in einem Beweis nur in ihr Gedächtnis einprägt, von solchen, in denen
sie sich die Zwischenergebnisse notiert. Es scheint absurd zu sein, darin einen
relevanten Unterschied zu sehen. Und warum sollte es dann relevant sein, wenn
nun nicht Ingrid, sondern Max sich an ihre Aufzeichnungen setzt und den Beweis
fortführt?

Wenn Ingrid gestern den Kompaktheitssatz bewiesen hat und sich heute daran
erinnert, dann verfügt sie über mathematisches Wissen. Es wäre absurd zu
erwarten, dass sie den Beweis in Gedanken noch einmal durchgehen müsste, um auch
heute mathematisches Wissen zu haben. In jedem längeren Beweis müssen wir uns
darauf verlassen, dass wir uns an frühere Beweisschritte korrekt erinnern. Wenn
das Ergebnis, das Ingrid sich vorhin notierte, jetzt nur noch als historische
Kenntnis zählt, dann wird niemand mathematisches Wissen erlangen. Doch wenn
Ingrids Wissen, das sich auf ihre Erinnerungen oder ihre gestrigen
Aufzeichnungen stützt, auch heute als rationales Wissen zählt, dann ist nicht
einzusehen, warum Max' Wissen, das dieser ebenso aus diesen Aufzeichnungen haben
kann, nicht als rational, sondern als historisch zählt.

Wir stehen damit vor einem Dilemma: Entweder wir sagen, das Aufschreiben von
Wahrheiten sei wie das Gedächtnis \singlequote{\emph{content preserving}}, dann
sind mitgeteilte mathematische Erkenntnis selbst subjektiv mathematisch (und
nicht historisch); oder wir bestehen darauf, dass eine mathematische Erkenntnis,
die ich gelesen habe, bloß historisch ist, dann wissen wir alle Ergebnisse
längerer Beweise bloß historisch.

\authorfullcite{Burge:ContentPreservation1993} entscheidet sich für die erste
Option und sagt, wenn der Rezipient eine Überzeugung auf eine Rechtfertigung
aufbaut, in die empirische Elemente als Prämissen eingehen, so könne seine Überzeugung nicht \emph{a priori} begründet sein.
Wenn jedoch ausschließlich \emph{a priori} begründete Prämissen eingehen, so
habe der Rezipient auch dann Wissen \emph{a priori}, wenn er von der Mitteilung eines anderen
abhängt.\footcite[Vgl.][\pno~486\,f.]{Burge:ContentPreservation1993}
\name[Immanuel]{Kant} kann sich nicht für diese Option entscheiden, ohne seine
gesamte Konzeption von historischen, empirischen und rationalen Erkenntnissen zu
gefährden. Er kann aber auch nicht die andere Option wählen, denn dann wäre die
Mathematik keine rationale Wissenschaft mehr.

Eine Lösung im Sinne \name[Immanuel]{Kant}s scheint mir jedoch naheliegend zu
sein: Ob eine Erkenntnis subjektiv historisch oder rational ist, hängt nicht
davon ab, wie sie ursprünglich erworben wurde, sondern davon, ob die Person über die Fähigkeit
verfügt, die Gründe der Erkenntnis zu benennen, zu verstehen und zu bewerten.
Wenn Max durch Ingrids Auskunft in die Lage versetzt wird, den Beweis des
Kompaktheitssatzes zu führen und zu erläutern, dann verfügt er über
mathematisches Wissen. Kann er ihn nur zitieren, dann ist sein Wissen bloß
historisch. Und dasselbe betrifft auch Ingrids Erinnerung: Nur weil sie
weiterhin ihre mathematischen Fähigkeiten besitzt, zählt ihr Wissen als
mathematisch. Kämen ihr plötzlich die Kompetenzen abhanden und verfügte sie nur
noch über die Erinnerung, dann wäre auch ihr Wissen bloß noch historisch. Dies
kongruiert auch mit \name[Immanuel]{Kant}s Darstellung des unmündigen
\authorcite{Wolff:Discursuspraeliminarisdephilosophiaingenere1996}ianers in der
\KapitelTitel{Architektonik der reinen Vernunft}. Dessen Unselbständigkeit zeigt
sich nicht darin, woher er seine Definitionen ursprünglich hatte, sondern darin,
dass er nicht die Kompetenz besitzt, selbst Definitionen
aufzustellen.\footnote{\cite[Vgl.][B 864]{Kant:KritikderreinenVernunft2003},
\cite[][III: 541.6--7]{Kant:GesammelteWerke1900ff.}.}
Auch damit knüpft er an
\authorcite{Wolff:Discursuspraeliminarisdephilosophiaingenere1996}s Überlegungen
zum Unterschied von \emph{cognitio philosophica} und \emph{cognitionis
philosophicae cognitio historica} an. Der Unterschied zwischen beiden liegt
nicht im ursprünglichen Erwerb, sondern in der aktuellen Kompetenz.

Es ist dann letztlich das Kriterium der Relevanz, für das uns die Überlegungen
aus Kapitel \ref{chapter:AufklaerungundWissenschaft} als Leitfaden dienen
können, welches anzeigt, in welchen Themenbereichen es nicht weiter schädlich
ist, historische Kenntnis rationaler Erkenntnisse zu haben.\footnote{\enquote{Bei einigen rationalen Erkenntnissen ist es schädlich, sie bloß historisch zu
wissen, bei andern hingegen ist dieses gleichgültig. So weiß z.\,B. der Schiffer
die Regeln der Schifffahrt historisch aus seinen Tabellen; und das ist für ihn
genug. Wenn aber der Rechtsgelehrte die Rechtsgelehrsamkeit bloß historisch
weiß: so ist er zum echten Richter und noch mehr zum Gesetzgeber völlig
verdorben} \mkbibparens{\cite[][A 21]{Kant:ImmanuelKantsLogik1977},
\cite[][IX: 22.21--26]{Kant:GesammelteWerke1900ff.}}.}
In der Mathematik wird es für die meisten Menschen unproblematisch sein. Wenn wir auf den Beweis
des Satzes des \singlename{Pythagoras} angesprochen werden, mögen wir sagen:
\enquote{Ich konnte ihn mal, aber ich habe vergessen, wie er ging.} Unser
ursprünglich rationales Wissen ist zur bloßen historischen Kenntnis geworden.
Dies ist aber auch nicht schlimm. Die Zuverlässigkeit mathematischer Erkenntnis
ist groß genug und in der Anwendung schadet es in der Regel nicht, die Beweise
nicht zu kennen. Anders verhält es sich in Fragen der Moral und Ethik unter
Einschluss politischer Fragen. Dort müssen uns die Gründe weiterhin verfügbar
sein, wenn wir mündig sein wollen.
\end{comment}


\section{Handeln auf epistemisch unzureichender
Grundlage}\label{section:HandelnAufEpistemischDuennerGrundlage}
Bis hierher fügt sich \name[Immanuel]{Kant}s \emph{ethics of belief} in seine
Aufklärungskonzeption, zumal sie die Überlegungen zu testimonialem Wissen aus
den Kapitel \ref{section:autonomieunddaszeugnisanderer} und
\ref{subsection:BewertungvonInformationennachihrerART} zu integrieren erlaubt
und eine eindeutig evidentialistische Position beschreibt.
Die klassische evidentialistische Position in den \emph{ethics of belief} -- wie
sie etwa von \authorfullcite{Clifford:TheEthicsofBelief1877} vertreten wird -- zeichnet sich
durch die Überzeugung aus, dass es \emph{immer} schlecht und tadelnswert ist,
eine unzureichend begründete Überzeugung zu
haben.\footnote{\enquote{To sum up: it is wrong always, everywhere, and for any
one, to believe anything upon insufficient evidence}
\parencite[][195]{Clifford:TheEthicsofBelief1877}.} \name[Immanuel]{Kant}s
Begriffe des Wissens und des Meinens bleiben dabei beide im Rahmen einer
evidentialistischen Position, denn auch im Falle des Meinens halten wir eine
Behauptung nicht schlechthin für wahr, sondern lediglich für möglich oder auch
wahrscheinlich. Nun gibt es auch Autoren, die behaupten, es sei doch zumindest
in manchen Situationen besser und sogar vernünftig, eine unzureichend begründete Überzeugung zu
haben.\footnote{Die klassische Reaktion findet sich bei William
\textcite[vgl.][]{James:TheWilltoBelieve1919}.} Eine Überzeugung zu haben sei
etwa dann vernünftig, wenn wir zwar keine hinreichenden Anhaltspunkte für ihre
Wahrheit haben, das Haben dieser Überzeugung jedoch positive Effekte zeitige.
Klassische Beispiele sind zum einen die \name[Blaise]{Pascal}'sche
Wette und zum anderen die Annahme, dass sich eine optimistische Einstellung bei
bestimmten Krankheiten positiv auf den Heilungserfolg
auswirke.\footcite[Vgl.][]{Chignell:TheEthicsofBelief2013}

\name[Immanuel]{Kant} scheint sich auf eine evidentialistische Position
festzulegen, denn er beschreibt die \enquote{innere} Lüge, also eine
Unaufrichtigkeit im Fürwahrhalten sich selbst gegenüber, als Verstoß gegen eine
unbedingte Pflicht, der auf einer Stufe mit der \singlequote{äußeren} Lüge
stehe.\footnote{\cite[Vgl.][A
85\,f.,]{Kant:DieMetaphysikderSitten1977Tugendlehre} \cite[][VI:
430.9--431.3]{Kant:GesammelteWerke1900ff.}.} Und die unbedingte Geltung des
Lügenverbots ohne Ansehen der Folgen steht spätestens durch
\name[Immanuel]{Kant}s Auseinandersetzung mit \name[Benjamin]{Constant} außer
Frage.\footnote{\cite[Vgl.][]{Kant:UebereinvermeintesRechtausMenschenliebezuluegen1977},
\cite[][VIII: 423--430]{Kant:GesammelteWerke1900ff.}.} Gerade weil er die
unbedingte Geltung des Gebots der Aufrichtigkeit nicht wie
\authorfullcite{Clifford:TheEthicsofBelief1877} konsequentialistisch über die
schlechten Folgen der Unaufrichtigkeit
begründet,\footnote{\authorcite{Clifford:TheEthicsofBelief1877} schreibt:
\enquote{No real belief, however trifling and fragmentary it may seem, is ever
truly insignificant; it prepares us to receive more of its like, confirms those
which resemble it before, and weakens others; and so gradually it lays a
stealthy train in our inmost thoughts, which may some day explode into overt
action, and leave its stamp upon our character for ever}
\parencite[][292]{Clifford:TheEthicsofBelief1877}. Er argumentiert dabei, dass
einzelne Urteilsakte, die zunächst harmlos erscheinen mögen, doch zur
Habitualisierung einer unkritischen Urteilsweise beitragen:
\enquote{Every time we let ouselves believe for unworthy reasons, we weaken our
powers of self-control, of doubting, of judicially and fairly weighing
evidence} \parencite[][294]{Clifford:TheEthicsofBelief1877}. Ein ähnliches
Argument findet sich bzgl. der Unaufrichtigkeit sich selbst gegenüber auch bei
\name[Immanuel]{Kant}
\mkbibparens{\cite[vgl.][A 86]{Kant:DieMetaphysikderSitten1977Tugendlehre},
\cite[][VI: 430.35--431.3]{Kant:GesammelteWerke1900ff.}}.} scheint es
\emph{prima facie} kaum denkbar, Gründe einzubeziehen, die nicht auf die
Wahrheit der Überzeugung, sondern ihre positiven oder negativen Folgen
rekurrieren. Denn er betont, dass es für die Bewertung einer Überzeugung nicht
relevant ist, ob sich aus ihr positive oder negative Folgen ergeben:
\begin{quote}
Es kann auch bloß Leichtsinn, oder gar Gutmütigkeit, die Ursache davon sein, ja
selbst ein wirklich guter Zweck dadurch beabsichtigt werden, so ist doch die
Art, ihm nachzugehen, durch die bloße Form ein Verbrechen des Menschen an seiner
eigenen Person, und eine Nichtswürdigkeit, die den Menschen in seinen eigenen
Augen verächtlich machen
muß.\footnote{\cite[][A 85]{Kant:DieMetaphysikderSitten1977Tugendlehre},
\cite[][VI: 430.4--8]{Kant:GesammelteWerke1900ff.}.}
\end{quote}
Gerade die \name[Blaise]{Pascal}'sche Wette kritisiert er, weil es unwürdig sei,
sich eine Überzeugung aus Hoffnung auf Belohnung oder Furcht vor Strafe selbst
einzureden.\footnote{\cite[Vgl.][A 85\,f.,]{Kant:DieMetaphysikderSitten1977Tugendlehre}
\cite[][VI: 430.19--23]{Kant:GesammelteWerke1900ff.}.}

Nun wird behauptet, \name[Immanuel]{Kant}s \emph{ethics of belief} seien gerade
deswegen von aktuellem Interesse, weil er mit dem Begriff des Glaubens die
evidentialistische Linie verlasse und sich der Position des späteren
Pragmatismus
annähere.\footnote{\cite[Vgl.][335]{Chignell:BeliefinKant2007}.}
Es gebe also andere Gründe für die Vernünftigkeit einer Überzeugung als solche,
die auf ihre Wahrheit hinzielen. Wir können Rechtfertigungen von
Überzeugungen, die die Wahrheit einer Überzeugung belegen, \enquote{epistemische
Rechtfertigungen} nennen und von \enquote{nicht-epistemischen Rechtfertigungen}
abgrenzen.\footcite[Vgl.][34]{Chignell:KantsConceptsofJustification2007}
\name[Immanuel]{Kant} spricht hier von \enquote{subjektiv zureichenden} Gründen,
die auch aus Sicht des Urteilenden nicht objektiv zureichend sind und die
Überzeugung ganz bewusst an ein eigenes \emph{Bedürfnis} rückbinden (es handelt
sich auch hier nicht um Überredung, sondern um Überzeugung oder ein
\emph{überlegtes} Urteil). Es handelt sich allem Anschein nach also bei objektiv
zureichenden Gründen um epistemische, bei bloß subjektiv zureichenden Gründen um
nicht-epistemische Gründe.

Ich stimme freilich der Aussage zu, dass \name[Immanuel]{Kant}
nicht-epistemische Rechtfertigungsgründe für Überzeugungen benennt, die eine
Überzeugung auch dann als vernünftig erscheinen lassen, wenn diese objektiv
nicht zureichend begründet werden kann und es sich auch nicht um die bescheidene
Position eines Meinens handelt. Allerdings denke ich nicht, dass es sich bei
\name[Immanuel]{Kant}s Glaubensbegriff um ein tragfähiges Konstrukt handelt.
Der Begriff des Glaubens ist schwierig, weil \name[Immanuel]{Kant}
darin die Legitimität bloß subjektiv zureichender Erkenntnisgründe behauptet.
Bis hierher schien es ja so zu sein, dass mündig und aufgeklärt derjenige
urteilt, der seine Zustimmung daran bemisst, ob etwas einen \emph{objektiv}
zureichenden Grund hat. Nun sagt \name[Immanuel]{Kant}, auch subjektiv
zulängliche Erkenntnisgründe können ein aufgeklärtes Fürwahrhalten
konstituieren, wenn der Urteilende nur \emph{weiß}, dass sie lediglich
subjektiv, nicht aber objektiv zureichend sind -- wenn also \emph{Überzeugung}
und keine \emph{Überredung} stattfindet.

Um in solchen Fragen nicht in Beliebigkeit zu verfallen, erwächst aus dieser
Konstellation -- die Fragen sind einerseits unvermeidlich, transzendieren aber
andererseits den Bereich unseres Wissens -- das Bedürfnis der Vernunft nach
Orientierung.\footnote{Siehe dazu
\cite[][A 309--311]{Kant:Washeisst:SichimDenkenorientieren?1977}, \cite[][VIII:
136.1--137.3]{Kant:GesammelteWerke1900ff.}.} Wie in Kapitel
\ref{section:KantalsliberalerAufklaerer} und zu Beginn von Kapitel
\ref{section:autonomieunddaszeugnisanderer} bereits aufgezeigt, ist es nach
\name[Immanuel]{Kant} kein gangbarer Weg, Fragen nach Übersinnlichem -- also
etwa die Frage nach der Existenz Gottes oder nach seinen Eigenschaften -- dem
privaten Belieben zu überantworten.
Sie können zwar nicht wissenschaftlich, ja nicht einmal im \emph{modus} des Wissens beantwortet
werden, aber sie sollen doch \emph{vernünftig} zu beantworten sein. Und dies
könne auch jeder, der sich nur von dem Vorurteil der eigenen Unfähigkeit und der
Überlegenheit anderer befreit. Hierzu wiederum leistet die Vernunftkritik einen
entscheidenden Beitrag:
\begin{quote}
Die Veränderung betrifft also bloß die arroganten Ansprüche der Schulen, die
sich gerne hierin (wie sonst mit Recht in vielen anderen Stücken) für die
alleinigen Kenner und Aufbewahrer solcher Wahrheiten möchten halten lassen, von
denen sie dem Publikum nur den Gebrauch mitteilen, den Schlüssel derselben aber
für sich behalten (quod mecum nescit, solus vult scire
videri).\footnote{\cite[][B xxxiii]{Kant:KritikderreinenVernunft2003},
\cite[][III: 20.28--33]{Kant:GesammelteWerke1900ff.}.}
\end{quote}
Der Vernunftkritik hat das Ziel, Aufklärung zu befördern, indem sie die
Ansprüche philosophischer Schulen beschränkt. Durch die Erkenntnis, dass es
keine wissenschaftliche Erkenntnis der \enquote{unvermeidlichen Aufgaben der
reinen Vernunft} gibt, braucht der Einzelne sich durch die Aussprüche und
Anmaßungen anderer nicht einschüchtern zu lassen, sondern kann den Mut finden,
seine eigene Vernunft zu gebrauchen. Auf der Grundlage der praktischen Vernunft
seien sie einfach zu beantworten. Dies ist zumindest dann nicht ganz unplausibel, wenn man
\name[Immanuel]{Kant}s moralisch-epistemischen Optimismus
voraussetzt.\footnote{Siehe oben, Kapitel
\ref{Abschnitt:moralischepistemischerOptimismus},
S.~\pageref{Abschnitt:moralischepistemischerOptimismus}--\pageref{Abschnitt:moralischepistemischerOptimismus-Ende}.}
Dieser ist die Grundlage dafür, dass sich jeder in Fragen der Metaphysik ganz
einfach selbst vernünftig orientieren könne, wenn er nur den Mut und die
Entschlusskraft dazu findet und sich von philosophischen Schulen nicht
einschüchtern lässt.

Jeder Glaube betrifft also etwas, das die Reichweite unseres Erkennens
übersteigt. Grundbedingung ist somit, dass wir nur dann sagen, wir
\emph{glauben}, dass $p$, wenn wir weder objektive Gründe {für} die
Behauptung, dass $p$, haben, noch solche dagegen. Der Glaube tritt nun wiederum
in dreierlei Gestalt auf, als \emph{doktrinaler Glaube}
(\ref{DoktrinalerGlaube}), als \emph{pragmatischer Glaube}
(\ref{PragmatischerGlaube}) und als \emph{moralischer Glaube}
(\ref{MoralischerGlaube}).
%
\begin{nummerierung}
\item\label{DoktrinalerGlaube} Der \emph{doktrinale Glaube} ist ein
theoretisches Fürwahrhalten in Fragen, die für uns keine direkte Relevanz haben.
Zu dieser Form des Glaubens zählt er insbesondere die Physikotheologie und gibt
als einzigen Unterschied zum Meinen die \singlequote{Festigkeit} an, die sich in
der Bereitschaft zeige, eine Wette einzugehen.\footnote{\cite[Vgl.][B
852--856]{Kant:KritikderreinenVernunft2003},
\cite[][III: 534.14--536.11]{Kant:GesammelteWerke1900ff.}.} Ein solches
Konstrukt ist freilich wenig überzeugend, spielt in \name[Immanuel]{Kant}s
Systematik aber auch keine weitere
Rolle. \authorfullcite{Chignell:BeliefinKant2007} behauptet, dass sich
auf Grundlage des doktrinalen Glaubens auch eine \singlequote{liberale
Metaphysik} errichten lasse, in der jeder nach seinen Überzeugungen über Dinge
an sich reden möge, solange diese Überzeugungen nicht mit objektivem Wissen
verwechselt werden. Dies sei möglich auf Grundlage der Einsicht in die
Unerkennbarkeit der Dinge an sich.\footnote{\enquote{Given this situation, we can and
should go ahead and build metaphysical arguments in all the usual ways, by
appealing to \enquote{intuitions} (of the Moorean rather tha the Kantian sort),
reflective equilibrium, inference to best explanation, simplicity, and so forth}
\parencite[][360]{Chignell:BeliefinKant2007}.} Eine Grundlage scheint mir diese
Interpretation in den Schriften \name[Immanuel]{Kant}s nicht zu haben (und
tatsächlich verweist \authorcite{Chignell:BeliefinKant2007} auch auf keine
textbasierte Evidenz). Auch scheint mir ein solches Unterfangen, dass dem
\enquote{frictionless spinning in the
void}\footcite[][11]{McDowell:MindandWorld1994} sehr nahe käme, nicht sehr
anziehend zu sein.
%
\item\label{PragmatischerGlaube} Der \emph{pragmatische Glaube} liegt vor, wenn
wir ein Ziel verfolgen und nur eine Bedingung wissen, unter der das Ziel erreicht werden kann. Nach
\name[Immanuel]{Kant} glauben wir dann vernünftiger Weise, dass die
entsprechende Bedingung erfüllt ist, da wir sonst unser Ziel nicht verfolgen
können. Sein eigenes Beispiel lautet: Ein Arzt versucht, einen Patienten zu
heilen, kann aber nicht aus objektiven Gründen beurteilen, welche Krankheit
dieser hat. Er vermutet, dass der Patient an der Schwindsucht leidet, und unter
dieser Annahme weiß er, was zu tun ist.

Nun scheint mir nicht einsichtig zu sein, warum \name[Immanuel]{Kant} nicht
einfach sagt, dass hier ein Fall von Meinen vorliegt, und es in manchen Fällen
nötig ist, bei unseren Entscheidungen Vermutungen über die zugrunde liegende
Situation (nicht über die moralischen Gebote!) einzubeziehen. Ein
verantwortungsvoller Arzt würde wohl urteilen, dass Schwindsucht die Krankheit
ist, die zu vermuten ist, und entsprechend handeln. Aber er wird weiter
berücksichtigen, dass es sich um eine Vermutung handelt; treten Erscheinungen
auf, die gegen diese Vermutung sprechen, wird er sie überdenken und
gegebenenfalls die Therapie ändern oder abbrechen. Vielleicht sollten wir
erstens sagen, dass es ein objektiver Grund ist, der dafür spricht, \emph{diese} Behauptung --
dass es die Schwindsucht und keine andere Krankheit ist -- für wahr zu halten,
wenngleich der Grund nicht zureichend ist und nur Wahrscheinlichkeit, aber kein
Wissen generiert. Und dann können wir zweitens sagen, dass ein subjektiver
Grund vorliegt, das Urteil nicht zu suspendieren, sondern trotz unzureichender
Grundlage \emph{sofort} zu urteilen; und dieser subjektive Grund ist der
eingetretene Handlungsdruck.

Ein recht eindrückliches Beispiel, welches \name[Immanuel]{Kant}s Position
entgegen kommt, liefert \authorfullcite{Chignell:BeliefinKant2007} mit folgendem
Szenario: Ein Bergsteiger befinde sich in einer Situation, in der er sein eigenes Leben nur
dadurch retten kann, dass er ungesichert über eine Kluft springt. Er hat keine
epistemische Grundlage, um objektiv entscheiden zu können, ob ihm dies gelingen
wird. Er \emph{weiß} also nicht, ob er über die Kluft springen kann. Nimmt man
nun aber an, dass die Chancen, eine entsprechende Leistung zu erzielen, von der
Festigkeit der Überzeugung abhängt, dass man es schaffen kann, so sei es
durchaus vernünftig, an die eigene Fähigkeit zu
glauben.\footcite[Vgl.][\pno~343\,f.]{Chignell:BeliefinKant2007}

Dieses Szenario verdeutlicht den Punkt, auf den es \name[Immanuel]{Kant}
anzukommen scheint: Wenn wir unser Handeln auf bloße Meinung gründen, dann
unterminiert dies unsere Motivation und Zuversicht. Es ändert zwar nichts daran,
was vernünftiger Weise zu tun ist, aber derjenige, der bewusst bloß vermutet,
wird verzagt handeln und damit den Handlungserfolg gefährden. Und dies mag eine
gewisse psychologische Plausibilität haben. Jeder, der sich sportlichen
Herausforderungen stellt, kennt die Bedeutung der eigenen Überzeugungen für die
eigene Leistungsfähigkeit.
%
\item\label{MoralischerGlaube} Bei dem \emph{moralischen Glauben} ist uns der
Zweck, den wir verfolgen, durch die Moral vorgegeben; des Weiteren wissen wir,
dass dieser Zweck nur erreicht werden kann, wenn eine bestimmte Bedingung
erfüllt ist. Der Zweck, der uns vorgegeben ist, besteht in dem höchsten Gut.
\begin{quote}
Glückseligkeit allein ist für unsere Vernunft bei weitem nicht das vollständige
Gut. Sie billigt solche nicht {\punkt}, wofern sie nicht mit der Würdigkeit,
glücklich zu sein, d.\,i. dem sittlichen Wohlverhalten, vereinigt ist.
Sittlichkeit allein, und, mit ihr, die bloße \ori{Würdigkeit}, glücklich zu
sein, ist aber auch noch lange nicht das vollständige Gut. Um dieses zu
vollenden, muß der, so sich als der Glückseligkeit nicht unwert verhalten hatte,
hoffen können, ihrer teilhaftig zu
werden.\footnote{\cite[][B 841]{Kant:KritikderreinenVernunft2003},
\cite[][III: 527.33--528.3]{Kant:GesammelteWerke1900ff.}.}
\end{quote}
Ob wir an die Existenz Gottes glauben oder nicht, das ändert nichts an dem
praktischen Wissen, das uns zur Verfügung steht. Insbesondere folgen keine
kategorischen oder moralischen Imperative, die ohne religiöse
Annahmen nicht zu begründen wären.
Und auch wenn Religion \enquote{das Erkenntnis aller unserer Pflichten als göttlicher
Gebote}\footnote{\cite[][B~229]{Kant:DieReligioninnerhalbderGrenzenderblossenVernunft1977}, \cite[][VI:
153.28--29]{Kant:GesammelteWerke1900ff.}.} sei, hängen doch unsere Pflichten
nicht von der Religion ab. Aber der Glaube stütze doch die Erfolgsaussichten
unseres Handelns gemäß den Geboten der Sittlichkeit. Denn er lasse uns
\emph{hoffen}, dass eine Lebensausrichtung, die Gebote der Sittlichkeit an
oberster und Ratschläge der Klugheit an zweiter Stelle ansiedelt, \emph{in
beidem} Erfolg haben könne, also sowohl zu Sittlichkeit, als auch zu
Glückseligkeit führe. Unglaube hingegen unterminiere die Motivation zu
moralischem Handeln, insofern er mit der Einsicht in die Möglichkeit (oder gar
Wahrscheinlichkeit) des Scheiterns verbunden sei. Dennoch -- das ist nicht
fraglich -- darf die Motivation zu moralischem Handeln nicht in der Erwartung
göttlicher Belohnung und Strafe liegen.
%

Bereits die \titel{Kritik der reinen Vernunft} artikuliert
die Hoffnung, \enquote{ob sich nicht in ihrer praktischen Erkenntnis Data
finden, jenen transzendenten Vernunftbegriff des Unbedingten zu
bestimmen}\footnote{\cite[][B xxi]{Kant:KritikderreinenVernunft2003},
\cite[][III: 14.16--18]{Kant:GesammelteWerke1900ff.}. Siehe auch
\cite[][B xxiv\,f.,]{Kant:KritikderreinenVernunft2003}
\cite[][III: 16.9--16]{Kant:GesammelteWerke1900ff.}.}.
Die \titel{Kritik der reinen Vernunft} beschneidet die Ansprüche der Schulen in
Fragen der Metaphysik, um \emph{auf eben diesem Gebiet} dem Einzelnen die
Freiheit des Selbstdenkens zu geben.
\begin{quote}
Nur solche Gegenstände sind Sachen des Glaubens, bei denen das Fürwahrhalten
notwendig frei, d.\,h. nicht durch objektive, von der Natur und dem Interesse
des Subjekts unabhängige, Gründe der Wahrheit bestimmt
ist.\footnote{\cite[][A 106]{Kant:ImmanuelKantsLogik1977},
\cite[][IX: 70.9--12]{Kant:GesammelteWerke1900ff.}.}
\end{quote}
Aber dies heißt gerade \emph{nicht}, Fragen der Religion, die nicht mit den
Mitteln der theoretischen Philosophie zu beantworten sind, dem privaten Belieben
und subjektiven Zufälligkeiten anheim zu stellen. Das Aufklärungsverständnis des
\enquote{sapere aude!} beruht wesentlich auf der Überzeugung, das diejenigen
Fragen, die für uns als Menschen von Bedeutung sind, einer \emph{vernünftigen}
und damit intersubjektiv kommunizierbaren und kritisierbaren Beantwortung
zugänglich sind. Der Indifferentismus, der verschiedene Antwortmöglichkeiten als
gleichrangige Optionen akzeptiert statt
sie dem Maßstab der Vernunft unterzuordnen, ist nach \name[Immanuel]{Kant} ja gerade
keine Option. Bester Beleg hierfür ist sein Begriff eines Vernunftglaubens.
\end{nummerierung}

Die Situation der Metaphysik, die \name[Immanuel]{Kant} so eindringlich in den
Vorreden der \titel{Kritik der reinen Vernunft} beschreibt, tritt zunächst
als Bedrohung der Aufklärung auf. Wenn Metaphysik ihrem Weltbegriff nach die
Fragen thematisiert, die wir beantworten müssen, um unserer Bestimmung gerecht
werden zu können, diese Fragen aber einer vernünftigen Antwort nicht zugänglich
zu sein scheinen, dann gefährdet dies die Möglichkeit, unser je eigenes Leben
vernünftig auszurichten. So erklärt sich die Emphase, mit der er das Schicksal
der menschlichen Vernunft beschreibt, durch unabweisbare und unbeantwortbare
Fragen belästigt zu
werden.\footnote{\cite[Vgl.][A vii]{Kant:KritikderreinenVernunft2003},
\cite[][IV: 7.2--6]{Kant:GesammelteWerke1900ff.}.}

Der anvisierte Ausweg lautet: Wir haben zwar keine theoretische
Ausgangsbasis für eine vernünftige Beantwortung der metaphysischen Fragen, aber eine
Ausgangsbasis in der praktischen Vernunft. Die praktische Vernunft sagt uns, was
wir tun sollen, und gibt uns einen notwendigen Zweck mit auf den Weg: das
höchste Gut in Gestalt einer \enquote{Glückseligkeit {\punkt} in dem genauen
Ebenmaße mit der Sittlichkeit der vernünftigen Wesen, dadurch sie derselben
würdig sind}\footnote{\cite[][B 842]{Kant:KritikderreinenVernunft2003},
\cite[][III: 528.13--14]{Kant:GesammelteWerke1900ff.}.} Da uns dieses höchste
Gut als notwendiger Zweck aufgegeben sei, sei es nur vernünftig, die Bedingungen
für gegeben zu halten, unter denen allein er realisierbar ist. Obwohl wir nicht
\emph{wissen} können, ob ein Gott existiert und ob es ein Leben nach dem Tod
gibt, können wir doch bestimmte Antworten als vernünftig ausmachen. Der Begriff
eines moralischen Glaubens und die Ethikotheologie sind Versuche, die
Aufklärungskonzeption vor den Resultaten der Vernunftkritik in Schutz zu nehmen.