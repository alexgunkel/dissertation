% \section{Endliche Vernunft}\label{section:EndlicheVernunft}
Es ist eine beiläufige Bemerkung aus dem Jahr 1763, in der \name[Immanuel]{Kant}
ein naheliegendes und oft angeführtes\footnote{So versteht bspw.
\authorfullcite{Kern:QuellendesWissens2006} die Endlichkeit unseres Verstandes:
\enquote{Wir sagen damit, daß es ein definierendes Merkmal desjenigen Wissens
ist, das wir zu verstehen versuchen, daß das Verhältnis zwischen dem Subjekt
dieses Wissens und dem Inhalt dieses Wissens auf eine Weise charakterisiert ist,
die Raum für die Möglichkeit des Irrtums hat. Dieses Wissen nennen wir endliches
Wissen im Kontrast zu unendlichem Wissen, welches genau diese Möglichkeit des
Irrtums nicht enthält} \parencite[][23]{Kern:QuellendesWissens2006}.} Merkmal
unserer Endlichkeit als solches erwähnt: unsere Fehlbarkeit. In der Schrift
\titel{Versuch, den Begriff der negativen Größen in die Weltweisheit
einzuführen} schreibt er:
\begin{quote}
 Der Mensch kann fehlen; der Grund dieser Fehlbarkeit liegt in der Endlichkeit
 seiner Natur, denn wenn ich den Begriff eines endlichen Geistes auflöse, so
 sehe ich, daß die Fehlbarkeit in demselben liege, das ist, einerlei sei mit
 demjenigen, was in dem Begriffe eines [endlichen; A.\,G.] Geistes enthalten
 ist.\footnote{\cite[][A 68]{Kant:VersuchdenBegriffdernegativenGroessenindieWeltweisheiteinzufuehren1977},
 \cite[][II: 202.23--27]{Kant:GesammelteWerke1900ff.}.}
\end{quote}
Die Fehlbarkeit endlicher Wesen können wir sowohl als epistemische Fehlbarkeit
verstehen -- wir können uns im Erkennen auch \emph{irren} --, als auch als
praktische Fehlbarkeit -- wir können unserer vernünftigen (insbesondere unserer
moralischen) Einsicht \emph{zuwider handeln}\footnote{Dies wird in Kapitel
\ref{chapter:AufklaerungundWissenschaft} als ein wesentlicher Gedanke aus der
Sicht der Forderung nach Mündigkeit herausgearbeitet werden (siehe
insbesondere das Ende von Kapitel \ref{section:MuendigeLebensfuehrung}). In
diesem (praktischen) Sinne ist unsere Endlichkeit ein Hindernis der Aufklärung oder -- vielleicht
treffender -- der Grund dafür, dass wir Aufklärung nötig haben und dass
Aufklärung an kein Ende gelangt.}.

Irrtumsanfälligkeit oder Fehlbarkeit ist dabei gewiss nicht die
einzige Möglichkeit, die Rede von unserer Endlichkeit in theoretischer
Hinsicht zu verstehen. Wir können die Endlichkeit des menschlichen
Verstandes als \textit{Begrenztheit} (in seinen praktischen und kognitiven
Fähigkeiten, räumlich, zeitlich etc.), als \textit{Abhängigkeit} (von Anderen,
von tradierten Wissensbeständen, von gesellschaftlichen Praxisformen, von der
Natur, der Welt, von Gott etc.) und auf manch andere Weise\footnote{Die
neuzeitliche Logik spricht beispielsweise von \emph{unendlichen Urteilen}
(\emph{iudicia infinita, indefinita, limitativa}), welche
oberflächlich betrachtet bejahende Urteile sind, deren Prädikate
jedoch Verneinungen enthalten und die daher
nicht angeben, was das Subjekt ist, sondern bloß kategorial bestimmen, was es nicht ist. Während dies
in der allgemeinen Logik vernachlässigt werden kann, interessiert es innerhalb
einer transzendentalen Logik; {\cite[vgl.][B
97\,f.,]{Kant:KritikderreinenVernunft2003} \cite[][III:
88.3--32]{Kant:GesammelteWerke1900ff.}; vgl. außerdem \cite[][A
160--162]{Kant:ImmanuelKantsLogik1977}, \cite[][IX:
103.23--104.24]{Kant:GesammelteWerke1900ff.}, und dessen Ursprung in
\cite[][\nopp 3065]{Kant:Reflexionen1900ff.}, \cite[][XVI:
639.2--5]{Kant:GesammelteWerke1900ff.}, als Anmerkung zu
\textcite[][\S~294]{Meier:AuszugausderVernunftlehre1752},
\cite[][XVI: 635.19--22, 636.14--22]{Kant:GesammelteWerke1900ff.}}.}
beschreiben. Ein Autor kann eine oder mehrere dieser Formen der Endlichkeit
beschreiben und er kann behaupten, dass sie untereinander in einem systematischen Zusammenhang
stehen oder nicht. \name[Immanuel]{Kant} behauptet -- wie ich zeigen werde --
einen solchen systematischen Zusammenhang.

\phantomsection\label{Absatz:DescarteszuEndlichkeitundVerhaeltnisvonWilleundVerstand}
Nun ist neben der Deutung der Endlichkeit unseres Verstandes als Fehlbarkeit
die Deutung als Begrenztheit vielleicht die naheliegendste.
\authorcite{Descartes:OeuvresdeDescartes1983} etwa interpretiert die
Endlichkeit unseres Verstandes in den \titel{Meditationes de prima philosophia}
dahingehend, dass wir nur einen Teil aller Wahrheiten kennen, während uns die
Einsicht in die Wahrheit (oder Falschheit) anderer Gedanken verborgen bleibe.
An Urteilen ist nach \authorcite{Descartes:OeuvresdeDescartes1983}
nun nicht nur der Verstand (\emph{intellectus}), sondern auch der Wille (\emph{voluntas})
beteiligt, insofern ein Urteil einen Akt der Zustimmung zu einer Behauptung
beinhaltet; und dieser Wille sei gerade nicht endlich. Der Wille sei unendlich, insofern es nichts gebe,
dem wir nicht unsere Zustimmung (oder Ablehnung) geben können. Auch Aussagen,
über deren Wahrheit unser Verstand wegen seiner Endlichkeit nicht entscheiden
könne, seien mögliche Gegenstände unserer Zustimmung. Doch wenn wir über Dinge,
die über die Reichweite unseres Verstandes hinausgehen, urteilen, dann unterliefen uns
möglicherweise Irrtümer.\footnote{\enquote{Unde ergo nascuntur mei errores?
Nempe ex hoc uno qu{\`o}d, c{\`u}m latius pateat voluntas qu{\`a}m intellectus, illam non intra
eosdem limites contineo, sed etiam ad illa qu{\ae} non intelligo extendo}
\parencite[][VII: 58.20--23]{Descartes:OeuvresdeDescartes1983}.} Unser
Verstand als solcher sei uns von Gott gegeben und könne als ein Vermögen,
welches uns von Gott, der kein Betrüger sein kann, gegeben wurde, auch nicht
täuschen. Nicht unsere Vermögen seien Ausgangspunkt von Irrtümern und
anderem Schlechten, sondern nur der Gebrauch, den wir endliche Wesen von
ihnen machen. Verstand und Wille seien jeweils ohne Einschränkung gut, denn sie
seien göttlichen Ursprungs; aber ein falscher, unvorsichtiger Gebrauch führe zu
Fehlern und Täuschungen -- wenn wir über Dinge urteilen, die wir nicht klar und
deutlich einsehen. Auf diese Weise kann
\authorcite{Descartes:OeuvresdeDescartes1983} erläutern, wie es zu Irrtümern
kommt, ohne behaupten zu müssen, eines unserer (von Gott verliehenen) Vermögen
sei fehleranfällig. Weder die \emph{voluntas} noch der \emph{intellectus} ist
für Fehler verantwortlich.\footnote{\enquote{Ex his autem percipio nec vim
volendi, quam a Deo habeo, per se spectatam, causam esse errorum meorum, est
enim amplissima, atque in suo genere perfecta; neque etiam vim intelligendi, nam
quidquid intelligo, c{\`u}m a Deo habeam ut intelligam, procul dubio recte
intelligo nec in eo fieri potest ut fallar} \parencite[][VII:
58.14--19]{Descartes:OeuvresdeDescartes1983}.}
Unser Verstand ist laut \authorcite{Descartes:OeuvresdeDescartes1983} also
endlich im Sinne von \emph{begrenzt}, denn er erkennt weniger Wahrheiten als ein unendlicher
Verstand; während unser Wille keinen solchen Begrenzungen unterliegt und somit
\enquote{unendlich} genannt werden kann. Unsere Irrtumsanfälligkeit ergibt sich
dann aber erst aus der Kombination eines endlichen Verstandes und eines
unendlichen Willens. Sie ist damit Folge, nicht Grund oder Wesen unserer
Endlichkeit.

Diese Konzeption hat \emph{prima facie} einige Ähnlichkeit mit
\name[Immanuel]{Kant}s Darstellung, denn auch dieser betont die Begrenztheit
unseres Verstandes und sagt, dass wir uns unvermeidlich in Irrtümer und
Scheinwahrheiten verlieren, wenn wir uns über diese Grenzen
hinauswagen.\footnote{Man vergleiche etwa das Bild von der Insel der Wahrheit
in \cite[][B 294\,f.,]{Kant:KritikderreinenVernunft2003}
\cite[][III: 202.12--203.3]{Kant:GesammelteWerke1900ff.}.}
Und die Unterscheidung zwischen einer Unbegrenzten \emph{voluntas} und einem
begrenzten \emph{intellectus} lässt sich leicht in Analogie bringen zur
kantianischen Unterscheidung von Denken und Erkennen. Doch
wie ich zeigen identifiziert \name[Immanuel]{Kant} die Endlichkeit des
Verstandes begrifflich gerade nicht mit
der Begrenztheit, sondern zunächst mit der Abhängigkeit des
Verstandes als eines Vermögens der Spontaneität von unserer Rezeptivität
(Sinnlichkeit). 
\phantomsection\label{Abschnitt:VertrauteBestimmungenergebensichausGrundbestimmungunsererEndlichkeit}
Die Begrenztheit hingegen ergibt sich bei ihm erst in Folge der Abhängigkeit.
Dasselbe gilt für die Diskursivität des Verstandes; auch sie folgt ebenso wie
seine Fehlbarkeit aus der Abhängigkeit, die sich als Abhängigkeit von der
Möglichkeit, von Gegenständen affiziert zu werden, erweist.

\section{Die vielen Bedeutungsfacetten des
Endlichkeitsbegriffs}\label{subsection:DievielenFacettendesEndlichkeitsbegriffs}
Bevor ich darauf eingehe, wie \name[Immanuel]{Kant} Diskursivität, Begrenztheit
und Fehlbarkeit unseres Verstandes als Resultat der
Abhängigkeit rekonstruiert, möchte ich auf die Notwendigkeit hinweisen, zwischen
verschiedenen Bedeutungen (oder Bedeutungsnuancen) des Unendlichkeits- wie des
Endlichkeitsbegriffs zu differenzieren. Diese Notwendigkeit
ist spätestens \authorcite{Spinoza:EthikingeometrischerOrdnungdargestellt2007}
bewusst gewesen und wird auch von \name[Immanuel]{Kant} beachtet. Aus ihr ergibt
sich eine Einschränkung der für das hier zu verhandelnde Thema einschlägigen
Literaturgrundlage: Es geht explizit nicht um Endlichkeit und Unendlichkeit
in einem quantitativen oder mathematischen Sinne (und darum sind auch
die Antithetik ebenso wie Überlegungen zur zeitlichen Begrenztheit
unseres Lebens hier nicht einschlägig).

\subsection{Quantitative und qualitative
Unendlichkeit}\label{subsection:QuantitativeundqualitativeUnendlichkeit}
Im zwölften Brief, in dem er ausführlich auf den Begriff des Unendlichen eingeht,
betont \authorcite{Spinoza:EthikingeometrischerOrdnungdargestellt2007}
drei wichtige Unterscheidung, deren Nichtbeachtung unweigerlich in
Konfusionen führe:
\begin{quote}
Die Frage nach dem Unendlichen erschien allen immer sehr schwierig, sogar
unlösbar, weil sie nicht unterschieden haben (i) zwischen (a\textsubscript{i})
dem, bei dem aus seiner Natur (\emph{sua natura}) oder kraft seiner Definition
folgt, dass es unendlich ist; und (b\textsubscript{i}) dem, was keine Grenzen
hat, aber nicht kraft seines Wesens (\emph{essentia}), sondern durch seine Ursachen (\emph{vi
suae causae}). (ii) Und auch weil sie nicht unterschieden haben zwischen
(a\textsubscript{ii}) dem, was unendlich heißt, weil es keine Grenzen hat; und
(b\textsubscript{ii}) dem, dessen Teile wir, auch wenn wir dessen Maximum und
Minimum haben, dennoch keiner Zahl angleichen (\emph{adaequare}) und durch sie
erläutern (\emph{explicare}) können. (iii) Schließlich weil sie nicht
unterschieden haben zwischen (a\textsubscript{iii}) dem, was wir ausschließlich
mit dem Verstand erkennen (\emph{intelligere}), uns aber nicht bildlich
vorstellen (\emph{imaginari}) können; und (b\textsubscript{iii}) dem, was wir
uns auch bildlich vorstellen können.\footnote{Brief an Ludwig \name[Ludwig]{Meyer} vom
20. April 1663: \enquote{Qu{\ae}stio de Infinito omnibus semper difficillima,
im{\`o} inextricabilis visa fuit, propterea qu{\`o}d non distinxerunt [(i)] inter
[(a\textsubscript{i})] id, quod su{\^a} natur{\^a}, sive vi su{\ae} definitionis
sequitur esse infinitum; {\&} [(b\textsubscript{i})] id, quod nullos fines
habet, non quidem vi su{\ae} essenti{\ae}; sed vi su{\ae} caus{\ae}. [(ii)] Ac etiam,
quia non distinxerunt inter [(a\textsubscript{ii})] id, quod infinitum dicitur,
quia nullos habet fines; {\&} [(b\textsubscript{ii})] id, cujus partes, quamvis
ejus maximum {\&} minimum habeamus, nullo tamen numero ad{\ae}quare, {\&}
explicare possumus. [(iii)] Denique quia non distinxerunt inter
[(a\textsubscript{iii})] id, quod solummod{\`o} intelligere, non ver{\`o}
imaginari; {\&} inter [(b\textsubscript{iii})] id, quod etiam imaginari
possumus} \parencite[][IV:
53.1--10]{Spinoza:SpinozaOpera1972}.}
\end{quote}
Hier beschreibt immer die durch \emph{a} gekennzeichnete Auffassung
das korrekte, hier relevante Verständnis von Unendlichkeit, während
die durch \emph{b} gekennzeichnete Auffassung ein Missverständnis
evoziert, wenn sie auf die Unendlichkeit Gottes angewendet oder auf
ihrer Grundlage die Endlichkeit des Menschen verstanden wird.
Die zweite Unterscheidung grenzt den Begriff der Unendlichkeit als
Unbegrenztheit (a\textsubscript{ii}) ab gegen die unendliche Mächtigkeit eines (nichtsdestotrotz
begrenzten) Kontinuums (b\textsubscript{ii}).\footnote{\authorcite{Spinoza:SpinozaOpera1972}
expliziert sie durch das Beispiel zweier Kreise unterschiedlicher Größe, die nicht
konzentrisch sind und von denen der kleinere zur Gänze innerhalb des größeren
liegt. Der Abstand zwischen beiden Kreisen hat ein Minimum und ein Maximum und
nimmt dazwischen unendlich viele verschiedene Werte an -- die Klasse der
Abstände bildet ein Kontinuum, das freilich durch Minimum und Maximum begrenzt
ist \mkbibparens{siehe \cite[][IV: 58.33--59.26]{Spinoza:SpinozaOpera1972}}.
\authorcite{Spinoza:SpinozaOpera1972} diskutiert dieses Beispiel, um auf
notwendige Unterscheidungen hinzuweisen, deren Nichtbeachtung in Antinomien
führt, wie sie \name[Immanuel]{Kant} als Antinomie der Teilung. Die
entsprechende Antinomie betrifft die \enquote{\ori{absolute Vollständigkeit} der
\ori{Teilung} eines gegebenen Ganzen in der Erscheinung} \mkbibparens{\cite[][B
443]{Kant:KritikderreinenVernunft2003}, \cite[][III:
287.7--10]{Kant:GesammelteWerke1900ff.}}. diskutiert. Wir dürfen die
Möglichkeit der beliebig feinen gedanklichen Unterteilung nicht mit der Existenz
entsprechender unendlich kleiner Dinge verwechseln, wenn wir uns bei der
geometrischen Beschreibung der Wirklichkeit nicht in Paradoxa wie dem von
Achilles und der Schildkröte verfangen wollen.}
Die Vorstellung eines Kontinuums hilft uns bei dem Verständnis unserer eigenen
Endlichkeit nicht weiter, weil dabei ein ganz anderer Unendlichkeitsbegriff
zugrunde gelegt wird. Es geht uns um das Unendliche, welches -- wie
\authorcite{Spinoza:SpinozaOpera1972} sagt -- keine äußeren Grenzen
hat.

Die erste Unterscheidung ist hier die wichtigste, weil sie im ersten
Teil (a\textsubscript{i}) die Definition des gesuchten Unendlichkeitsbegriffs enthält; sie reflektiert einen doppelten
Unendlichkeitsbegriff, den \authorcite{Spinoza:SpinozaOpera1972} im ersten Buch
der \titel{Ethica} als Unterscheidung von \enquote{absolut unendlich} (a\textsubscript{i}) und
\enquote{in seiner Art unendlich} (b\textsubscript{i}) entwickelt und der eine Unterscheidung
zwischen einer quantitativen und einer qualitativen Unendlichkeitsauffassung zur
Folge hat: Etwas sei in seiner Art endlich (\emph{res in suo genere finita}),
wenn es von etwas anderem von derselben Natur begrenzt werden
kann.\footnote{\enquote{Ea res dicitur in suo genere finita, quae alia ejusdem
naturae terminari potest} \parencite[][\nopp
1d2]{Spinoza:EthikingeometrischerOrdnungdargestellt2007}.}
Ein Körper sei beispielsweise endlich, weil wir uns einen größeren Körper denken
können. Und ein Gedanke werde von einem anderen Gedanken begrenzt,
weswegen er in seiner Art endlich sei.\footnote{Der Sinn der Behauptung, ein
Gedanke werde von einem anderem Gedanken \singlequote{begrenzt}, erschließt
sich sicherlich nicht unmittelbar. Denkbar ist, dass ein Gedanke endlich ist,
insofern er von anderen Gedanken unterschieden ist. Zu sagen, dass Katzen
Säugetiere sind, ist eben etwas anderen als zu sagen, dass Katzen Karnivore
sind. Zugleich ist ein Gedanke durch durch seine Negation begrenzt; dass Katzen
Säugetiere sind, heißt in diesem Sinne auch, dass sie keine Fische sind. Nur
durch solche Ausschlüsse sind Gedanken bestimmt gemäß dem Prinzip \emph{omnis
determinatio est negatio}, welches sich so bei
\authorcite{Spinoza:SpinozaOpera1972} jedoch nicht findet
\parencite[vgl.][\pno~50\,f.]{Schnepf:MetaphysikimerstenTeilderEthikSpinozas1996}.}
Ein Körper, der wegen seiner Größe von keinem anderen Körper begrenzt werden
kann, wäre in seiner Art unendlich. Und ebenso wäre ein Gedanke, den kein
anderer Gedanke begrenzen kann, \emph{in suo genere} unendlich.

Nun könne etwas auch in seiner Art unendlich sein, ohne doch schlechthin
Unendlichkeit beanspruchen zu können. Seine Unbegrenztheit dann liege nicht in seiner
Natur, sondern sei quasi von außen bedingt, sie sei Folge einer äußeren Ursache.
Hingegen sei Gott \emph{absolute infinitus}, insofern er aus unendlich vielen
Attributen bestehe, von denen jedes ein \singlequote{ewiges} und unendliches
Wesen (\emph{essentia})
ausdrückt.\footnote{\cite[Vgl.][\nopp
1d6]{Spinoza:EthikingeometrischerOrdnungdargestellt2007}.} Zu seinem
Wesen gehöre alles, was keine Verneinung in sich schließt, sondern ein
Wesen oder eine \singlequote{Essenz} ausdrückt (\emph{essentiam exprimit}). Die
absolute Unendlichkeit ist eine solche, bei der die Unendlichkeit aus dem Wesen
(und damit aus der korrekten Definition) der Sache selbst folgt. Gott ist etwa
unendlich, insofern -- gemäß \singlequote{ontologischem}
Gottesbeweis\footnote{Von einem \singlequote{ontologischen Gottesbeweis} wird
erst im Anschluss an \name[Immanuel]{Kant}s Bezeichnung
\mkbibparens{\cite[vgl.][B 619]{Kant:KritikderreinenVernunft2003},
\cite[][III: 396.29--31]{Kant:GesammelteWerke1900ff.}} gesprochen.
Ich gehe auf Fragen rund um diese Bezeichnung nicht ein, da es hier nicht weiter
relevant sein wird.} -- aus dem Begriff Gottes bereits dessen Existenz folgt.
Aus der absoluten Unendlichkeit folgt die Unendlichkeit \emph{in suo
  genere}: Wenn etwas absolut unendlich ist, dann kann es nicht in seiner
Art endlich sein, denn was über alle Realität verfügt, kann von nichts
begrenzt werden.\footnote{Außerdem folge, dass die Substanz, als das, was in
sich selbst ist und durch sich selbst verstanden
wird \mkbibparens{\enquote{\emph{quod in se est et per se concipitur}},
\cite[][\nopp 1d3]{Spinoza:EthikingeometrischerOrdnungdargestellt2007}}
und die bereits ihrer Natur nach existiert
\parencite[Vgl.][1p7]{Spinoza:EthikingeometrischerOrdnungdargestellt2007},
unendlich ist \parencite[vgl.][\nopp
1p8]{Spinoza:EthikingeometrischerOrdnungdargestellt2007}, während alle endlichen
Dinge in der einen unendlichen Substanz sind
\parencite[vgl.][1p15]{Spinoza:EthikingeometrischerOrdnungdargestellt2007}.}


Der Unterschied zwischen einer quantitativen und einer qualitativen
Unendlichkeitsauffassung -- und wie er aus dem Gesagten resultiert --
verdeutlicht sich an den Begriffen der Dauer und der Ewigkeit, die zugleich auf
\authorcite{Spinoza:SpinozaOpera1972}s dritte Unterscheidung verweisen. Es gebe
ein Unendliches, das wir uns vorstellen können (b\textsubscript{iii}); und es gebe ein Unendliches, das
wir zwar erkennen können, das aber nicht Gegenstand unserer Vorstellung sein
kann (a\textsubscript{iii}). Maß, Zeit und Zahl seien Modi des Denkens und auch des Vorstellens.
Eine Dauer, die keine Grenzen, kein Davor und Danach kennt,  sei denkbar und
auch vorstellbar, nicht aber die Ewigkeit, die wir zwar denken, uns aber nicht
vorstellen können. Und deshalb verfälschen wir jede Einsicht in die
Unendlichkeit Gottes, wenn wir ihn zum Gegenstand unseres Vorstellens zu machen
versuchen. Um die Substanz oder die Ewigkeit fassen zu können, müssen wir uns
auf den Verstand, nicht aber die Vorstellung berufen, denn etwas Unendliches vorzustellen heißt, es
nur als \singlequote{in seiner Art} unendlich zu verstehen, nicht aber als
\emph{absolute infinitus}.

Der zentrale Ausgangspunkt, der für ein Verständnis des absolut Unendlichen
unentbehrlich ist, ist also der, dass es sich hier nicht um einen mathematischen
Begriff von Endlichkeit respective Unendlichkeit handelt, sondern um Begriffe,
die wir \singlequote{\emph{metaphysisch}} nennen könnten. Etwas ist in diesem
Sinne nicht dann unendlich, wenn es eine Eigenschaft von entsprechender
Quantität hat, sondern genau dann, wenn seine Existenz bereits
aus seinem Wesen folgt. Freilich folgt die mathematische Unendlichkeit in
gewisser Hinsicht aus der metaphysischen Unendlichkeit: Was \emph{ewig} ist,
weil es als nicht-existierend gar nicht gedacht werden könnte, das existiert auch
über eine Dauer von unendlicher Ausdehnung. Aber darin geht der Begriff seiner
Unendlichkeit eben nicht auf.

Dies lässt sich auch auf den unendlichen Verstand übertragen:
Dieser kennt unendlich viele Wahrheiten, aber darin besteht nicht seine
Unendlichkeit. Denn die allgemeine metaphysische Position wirkt sich auch auf
den unendlichen Verstand (\emph{intellectus infinitus})\footnote{Dieser Ausdruck
findet sich erstmals in einem Folgesatz zur
\emph{propositio} 16 des ersten Teils der \titel{Ethica}
\parencite[siehe][1p16c1]{Spinoza:EthikingeometrischerOrdnungdargestellt2007}.}
aus, von dem der endliche menschliche Verstand ein Teil sei\footnote{\enquote{Hinc
sequitur mentem humanam partem esse infiniti intellectus Dei}
\parencite[][2p11c]{Spinoza:EthikingeometrischerOrdnungdargestellt2007}. Nun ist
der menschliche Geist freilich endlich, aber es gehört dennoch zu seiner Natur,
Dinge \emph{sub specie aeternitatis} zu erkennen
\mkbibparens{\cite[Vgl.][2p44c2]{Spinoza:EthikingeometrischerOrdnungdargestellt2007}}}.
Auch der menschliche Verstand steht nicht primär in einem quantitativen
Verhältnis zum unendlichen Verstand Gottes. Und entsprechend lässt sich die Unendlichkeit des
göttlichen Verstandes nicht mathematisch verstehen, wenngleich sie quantitative
Unendlichkeiten zur Folge hat. Der unendliche Verstand erkennt nicht mehr oder
mit größerer Gewissheit, sondern auf eine andere Art. (Er erkennt
ausschließlich auf die Art der \emph{scientia intuitiva} und nicht wie wir auch
in den Erkenntnisgattungen des Meinens und des logischen
Schlussfolgerns.\footnote{Siehe dazu Kapitel
\ref{subsection:IntuitiverVerstandunddasSynthetischAllgemeine}.}) Entsprechend
ist es Folge unserer Endlichkeit, dass unsere Existenz und unsere Kenntnis
wahrer Aussagen auch quantitativ begrenzt ist. Aber dies ist nicht Wesen unserer
Endlichkeit im metaphysischen Sinne, sondern eine ihrer Auswirkungen.

Es ist nicht nötig vorauszusetzen, dass \name[Immanuel]{Kant}
\authorcite{Spinoza:SpinozaOpera1972}s Überlegungen im Original
rezipiert hätte. Es handelt sich um Überlegungen, die sich auch an anderen
Stellen finden, mit denen \name[Immanuel]{Kant} ganz sicher vertraut war, etwa
in \authorcite{Baumgarten:Metaphysica---Metaphysik2011}s
\titel{Metaphysica}.\footnote{Nach \authorcite{Engelhard:DasEinfacheunddieMaterie2005}
greift \name[Immanuel]{Kant} die Unterscheidung von \emph{infinitum} und
\emph{indefinitum} vielfach auf
\parencite[vgl.][\pno~354\,f.]{Engelhard:DasEinfacheunddieMaterie2005}. In
vielen Punkten ist zu vermuten, dass \name[Immanuel]{Kant} die Position
\authorcite{Spinoza:SpinozaOpera1972}s nicht aus dessen Schriften selbst kannte,
aber doch mit Darstellungen und Anknüpfungen vertraut war.} Dieser bezeichnet
das mathematische Unendliche -- welches er auch \enquote{unendlichscheinend}
nennt -- als \emph{indefinitum} (oder \emph{infinitum imaginarium}) und
unterscheidet es von einem \emph{infinitum} (dem \enquote{uneingeschränkten}),
dessen \singlequote{Realitätsgrad} (\enquote{gradus realitatis}) keine Grenzen
habe.\footnote{Dem Kontext nach geht es \authorcite{Baumgarten:Metaphysica---Metaphysik2011}
ausschließlich um den Begriff des Unbeschränkten -- die Unterscheidung zwischen
\emph{infinitum} und \emph{indefinitum} scheint lediglich das Missverständnis
verhüten zu sollen, das aus einer Verwechslung resultierte, wenn die Bestimmung
des \emph{indefinitum} (dass wir die Grenzen nicht \emph{kennen} können oder
wollen) als ausreichend für den Begriff des \emph{infinitum} angesehen wird. Er
möchte sicherstellen, dass nur das als (wahrhaft) unendlich bezeichnet wird, was
tatsächlich keinerlei Beschränkungen hat. In einem uneigentlichen Sinne nämlich
können wir dasjenige als unendlich groß bezeichnet, von dem wir keine Größe
angeben können, obwohl es möglicherweise eine uns unbekannte Grenze
gibt.} Ein \emph{indefinitum} hingegen sei etwas, dessen Grenzen wir
nicht kennen können oder nicht kennen
wollen, unabhängig davon, ob es \emph{de facto} eine Grenze hat. Wir könnten es
auch als \enquote{unbestimmt} bezeichnen. Nur das Uneingeschränkte oder
\emph{infinitum} sei wirklich unendlich; und dieses Unendliche sei als
dasjenige, dessen \singlequote{Realitätsgrad} (\emph{gradus realitatis}) keine
Beschränkung habe, das notwendig Seiende (\emph{ens
necessarium}), welches keinen Veränderungen unterliege und solchen auch gar
nicht unterliegen
könne.\footnote{\cite[Vgl.][\S\S~253--258]{Baumgarten:Metaphysica---Metaphysik2011},
\cite[][XVII: 82.2--21]{Kant:GesammelteWerke1900ff.}.} Unendlichkeit im Sinne
des \emph{infinitum} sei nichts anderes als Realität,\footnote{\cite[Vgl.][\S~261]{Baumgarten:Metaphysica---Metaphysik2011}, \cite[][XVII: 82.34--35]{Kant:GesammelteWerke1900ff.}.} während das Endliche über Realitäten, aber auch über Negationen, also \emph{modi}
verfüge.\footnote{\cite[Vgl.][\S~263]{Baumgarten:Metaphysica---Metaphysik2011},
\cite[][XVII: 83.5--8]{Kant:GesammelteWerke1900ff.}} Gerade in dieser Anbindung
des \emph{infinitum} an die \emph{realitas} ähnelt
\authorcite{Baumgarten:Metaphysica---Metaphysik2011}s Differenzierung
derjenigen, die sich paradigmatisch und auf höchstem Reflexionsniveau bei \authorcite{Spinoza:SpinozaOpera1972}
findet.


Nach \name[Immanuel]{Kant}s Auskunft ist es nun zwar kein Fehler, von der
Unendlichkeit Gottes und unserer Endlichkeit zu sprechen, da man die Freiheit besitze, beide Ausdrücke
entsprechend auszulegen. Man müsse sich jedoch klar von einem mathematischen
Verständnis abgrenzen. Deshalb bevorzugt er eigenen Angaben aus dem Jahre 1763
zufolge den Ausdruck
\phantomsection\label{Allgenugsamkeit}\enquote{Allgenugsamkeit} gegenüber dem
der Unendlichkeit für die Charakterisierung göttlicher Vollkommenheit,
weil es ein Fehler sei, das Verhältnis des göttlichen zu unserem
menschlichen Denken als quantitativen Unterschied begreifen zu wollen.
Denn dies setzte voraus, dass beide ihrer Beschaffenheit nach grundsätzlich
gleich und nur von verschiedener Stärke seien.\footnote{\cite[Vgl.][A
186\,f.,]{Kant:DereinzigmoeglicheBeweisgrundvomDaseinGottes1977} \cite[][II:
154.4--19]{Kant:GesammelteWerke1900ff.}.}
Verstehen wir den Unterschied zwischen Gott und uns als lediglich graduell --
und sei die Differenz zahlenmäßig noch so groß -- dann befinden wir uns zwar auf
vertrautem Terrain, weil wir uns einen besseren Verstand leicht vorstellen
können (insofern hat diese Konzeption den Vorteil guter Verständlichkeit). Aber
wir bilden damit doch nur eine weitere anthropomorphe Gottesvorstellung, die dem
Begriff des wirklich Unendlichen nicht gerecht wird. Bei der Unterscheidung von
menschlichem und göttlichem Denken interessiert gerade ein qualitativ
Unendliches, nicht das mathematische oder quantitative Unendliche.\footnote{In
\titel{Was heißt: sich im Denken orientieren?} spricht \name[Immanuel]{Kant}
hingegen Gott \enquote{\ori{Unendlichkeit} der Größe nach
zur Unterscheidung von allem Geschöpfe} zu \mkbibparens{\cite[][A
322]{Kant:Washeisst:SichimDenkenorientieren?1977}, \cite[][VIII:
142.30]{Kant:GesammelteWerke1900ff.}}. Dort wird jedoch nicht der göttliche
Verstand beschrieben, sondern begründet, warum Gott in unserer Wahrnehmung nicht
als solcher identifizierbar sei.}

Die genannte Ablehnung eines mathematischen Verständnisses von Unendlichkeit bei
der Explikation göttlicher Eigenschaften stammt aus dem Jahr 1763; und
sicherlich ist denkbar, dass \name[Immanuel]{Kant} seine Ansicht später revidiert und doch
einen quantitativen Unterschied herausstellen möchte. Dies wird bezüglich des
Begriffs des intuitiven und diskursiven Verstandes in \S~77 der \titel{Kritik
der Urteilskraft} von
\authorfullcite{McLaughlin:KantsKritikderteleologischenUrteilskraft1989}
behauptet.\footnote{\enquote{Ein unendlicher Verstand könnte durch Kenntnis
sämtlicher empirischer Gesetze den Begriff eines jeden Besonderen durchgehend
bestimmen, so daß nichts Zufälliges übrig bleibt. Daß wir dies nicht können,
sondern auf die Urteilskraft angewiesen sind, hängt von der Endlichkeit
(Schranken) unseres Verstandes, nicht von seiner Qualität (Beschaffenheit) ab}
\parencite[][148]{McLaughlin:KantsKritikderteleologischenUrteilskraft1989}.
\name[Immanuel]{Kant}s Behauptung ist aber gerade, dass die empirischen Gesetze
als solche aus Sicht des Verstandes zufällig sind (sonst bräuchten wir keine
Erfahrung, um sie zu erkennen). Ein Gesetz ist eben empirisch, wenn wir es nur
historisch wissen \emph{können} (siehe Kapitel
\ref{section:MuendigkeitundPhilosophie}). Auch dann, wenn wir \emph{alle}
empirischen Gesetze kennten, wären diese in dieser Hinsicht zufällig und somit
die Natur zweckmäßig (siehe Kapitel \ref{subsection:MetaphysikundAutonomie}).
Siehe dazu auch weiter unten Kapitel
\ref{subsection:IntuitiverVerstandunddasSynthetischAllgemeine}.} Dennoch gilt
mindestens \emph{prima facie} die Interpretationsmaxime, dass die Endlichkeit
unseres Denkens bei \name[Immanuel]{Kant} nicht als quantitative Einschränkung
aufzufassen ist. Dafür sprechen die frühe Auskunft \name[Immanuel]{Kant}s von
1763 und die Tatsache, dass \name[Immanuel]{Kant} stets von der besonderen
\emph{Beschaffenheit} unseres Verstandes spricht und auch in der \titel{Kritik
der Urteilskraft} noch davon spricht, dass \enquote{keine menschliche Vernunft
(auch keine endliche, die der Qualität nach der unsrigen ähnlich wäre, sie aber
dem Grade nach noch so sehr überstiege) die Erzeugung auch nur eines Gräschens
aus bloß mechanischen Ursachen zu verstehen
hoffen}\footnote{\cite[][\S~77]{Kant:KritikderUrteilskraft2009}; \cite[][V:
409.33--37]{Kant:GesammelteWerke1900ff.}. Siehe auch
\cite[][\S~75]{Kant:KritikderUrteilskraft2009},
\cite[][V: 400.13--21]{Kant:GesammelteWerke1900ff.}.} könne.\footnote{Diese
Sichtweise teilt auch \authorfullcite{Allison:KantsTranscendentalIdealism2004}:
\enquote{[A]s discursive, human knowledge differs in kind, not merely in degree,
from that which might be had by a putative pure understanding}
\parencite[][17]{Allison:KantsTranscendentalIdealism2004}.} Ich gehe also davon
aus, dass \name[Immanuel]{Kant} in diesem Zusammenhang einen qualitativen
Unendlichkeitsbegriff vor Augen hat. Entsprechend hilft es auch nicht, die
Antithetik der \titel{Kritik der reinen Vernunft} zu konsultieren, denn die dort
thematisierte Totalität der \enquote{Reihe in der Synthesis des
Mannigfaltigen}\footnote{\cite[][B 442]{Kant:KritikderreinenVernunft2003},
\cite[][III: 286.37]{Kant:GesammelteWerke1900ff.}.} diskutiert nur mathematische
Unendlichkeitsvorstellungen, die vielleicht in aktual und potentiell Unendliches
differenziert werden können. Doch beides entspricht nicht dem Begriff des
Unendlichen, der bei der Explikation des Unterschiedes zwischen unserem
endlichen und einem göttlichen unendlichen Verstand hilfreich sein
könnte. Und der Begriff \enquote{Allgenugsamkeit} lässt vermuten, dass
\name[Immanuel]{Kant} bereits 1763 an die Unabhängigkeit als entscheidendes
Merkmal eines unendlichen Verstandes gedacht haben mag, ebenso wie er später die
Abhängigkeit unseres Verstandes als Charakteristikum unserer Endlichkeit
herausstellt.\footnote{Siehe dazu \cite[][B
72]{Kant:KritikderreinenVernunft2003}, \cite[][III:
72.29--73.4]{Kant:GesammelteWerke1900ff.}. Siehe auch
\cite[][\S~10]{Kant:Demundisensibilisatqueintelligibilisformaetprincipiis1968},
\cite[][II: 396.19--397.4]{Kant:GesammelteWerke1900ff.}.}

\subsection{Der Ursprung unserer Endlichkeitsvorstellung}
Es stellt sich nun aus methodischen Gründen die Frage, welcher Begriff der
ursprüngliche ist: der des endlichen oder der des unendlichen Verstandes. Mit
diesem Begriff müsste dann die Analyse begonnen werden. Autoren wie
\authorcite{Descartes:OeuvresdeDescartes1983}, \authorcite{Hobbes:Leviathan1962}
und \authorcite{Locke:TheWorksofJohnLocke1963} wenden sich dieser Frage
explizit zu und auch bei \name[Immanuel]{Kant} findet sich eine Antwort, die es
zu berücksichtigen gilt. Woher also haben wir unsere Begriffe von endlichem und unendlichem Verstand?
Können wir einen Begriff eines göttlichen Verstandes ausgehend von einem Begriff unseres endlichen
Denkens bilden? Oder liegt er dem Begriff unseres eigenen Verstandes sogar
zugrunde?


\authorcite{Descartes:OeuvresdeDescartes1983} sagt, der menschliche endliche Verstand sei
lediglich ein schwächeres Abbild des unendlichen göttlichen Verstandes und wer versteht,
was der Ausdruck \enquote{Verstand} bedeutet, wer also eine Vorstellung von
unserem endlichen Verstand hat, der müsse bereits eine Vorstellung von einem
nicht-endlichen, göttlichen Verstand haben. Denn die Vorstellung eines endlichen
Verstandes sei lediglich die Idee eines perfekten Verstandes verbunden mit einer
Privation: So wie wir die Idee einer defekten Maschine nur bilden können, wenn
wir zunächst eine Vorstellung von einer nicht-defekten Maschine besitzen und
diese mit der Vorstellung eines Mangels verbinden, ebenso könnten wir nur
dadurch eine Vorstellung von unserem Denken erwerben, dass wir die Idee des
Verstandes mit der Vorstellung eines Mangels verbinden. So lautet das Argument
der dritten Meditation, mit dem nachgewiesen werden soll, dass wir über eine
Idee des göttlichen Verstandes verfügen.\footnote{\enquote{Nec putare debeo me
non percipere infinitum per veram ideam, sed tant{\`u}m per negationem finiti,
ut percipio quietem {\&} tenebras per negationem mot{\^u}s {\&} lucis; nam
contr{\`a} manifeste intelligo plus realitatis esse in substanti{\^a}
infinit{\^a} qu{\`a}m in finit{\^a}, ac proinde priorem quodammodo in me esse
perceptionem infiniti qu{\`a}m finiti, hoc est Dei qu{\`a}m me{\^i} ipsius}
\parencite[][VII: 45.23--29]{Descartes:OeuvresdeDescartes1983}. Diese
Vorstellung eines göttlichen Verstandes können wir nicht aufbauend auf anderen
Ideen erworben haben, sie sei uns somit angeboren \parencite[vgl.][VII:
51.6--14]{Descartes:OeuvresdeDescartes1983}.}

Bei \authorfullcite{Locke:TheWorksofJohnLocke1963} findet sich die klassische
empiristische Gegenposition zu \authorcite{Descartes:OeuvresdeDescartes1983}'
Annahme einer ursprünglichen Gottesvorstellung.\footnote{Schon
\authorcite{Hobbes:Leviathan1962} behauptet gegen
\authorcite{Descartes:OeuvresdeDescartes1983}, dass dessen Behauptung, wir
verfügten über eine solche angeborene Idee Gottes, die als Ursprung unserer Idee
von unserem endlichen Verstand anzusehen ist, ungereimt sei. Vielmehr besäßen
wir \emph{keine} Vorstellung von einem göttlichen Verstand, da dieser als
unendlicher Verstand gar nicht geeignet sei, Gegenstand eines mentalen
\singlequote{Bildes} zu sein \parencite[vgl.][VII:
183.4--19]{Descartes:OeuvresdeDescartes1983}.
\authorcite{Descartes:OeuvresdeDescartes1983} erwidert wiederum, dass
\authorcite{Hobbes:Leviathan1962} über eine falsche Vorstellung darüber verfüge,
was eine Vorstellung (\enquote{\emph{idea}}) ist, und deshalb irrtümlich
annehme, es könne keine solche Vorstellung von einem göttlichen Verstand
geben \parencite[Vgl.][VII: 183.22--25]{Descartes:OeuvresdeDescartes1983}.}
Grundlage ist die Annahme, unser Geist gleiche einer \emph{tabula rasa}, der erst durch die sinnliche Wahrnehmung
-- äußere \emph{sensation} und innere \emph{reflection} -- Inhalte gegeben
werden. Unsere Vorstellungen erhalten wir dann entweder unmittelbar durch
entsprechende Wahrnehmungen oder mittelbar durch Modifikation solcher
Wahrnehmungen. Da wir keine unmittelbare Bekanntschaft mit einem unendlichen
Denken machen können, können wir unseren Begriff eines solchen Denkens nur
mittelbar bilden, indem wir von einem Denken ausgehen, das uns bekannt ist --
dem endlichen --, und diese Idee mit einer anderen Idee kombinieren. Nach
\authorcite{Locke:TheWorksofJohnLocke1963} bilden wir die Vorstellung eines
unendlichen Verstandes ausgehend von der durch Reflexion gewonnenen Vorstellung
von den Eigenschaften und Tätigkeiten unseres eigenen endlichen Verstandes durch
Steigerung mittels der Idee des Unendlichen.\footnote{\cite[Vgl.][II:
31]{Locke:TheWorksofJohnLocke1963}: \enquote{[H]aving, from what we experiment
in ourselves, got the ideas of existence and duration; of knowledge and power;
of pleasure and happiness; and of serveral other qualities and powers, which it
is better to have than to be without: when we would frame an idea the most
suitable we can to the Supreme Being, we enlarge every one of these with our
idea of infinity; and so putting them together, make our complex idea of God.}}


Verfügen wir also zunächst über eine Vorstellung von einem endlichen Verstand
und bilden auf dieser Grundlage die Vorstellung von einem unendlichen
Erkenntnisvermögen, wie dies die Empiristen
\authorcite{Hobbes:Leviathan1962} und \authorcite{Locke:TheWorksofJohnLocke1963}
behaupten? Oder ist der Begriff des unendlichen Verstandes ursprünglich gegeben
und der des endlichen Verstandes davon abgeleitet, wie
\authorcite{Descartes:OeuvresdeDescartes1983} nachweisen zu können glaubt?
Welche Annahme muss einer Interpretation \name[Immanuel]{Kant}s zugrunde gelegt
werden? \name[Immanuel]{Kant} kommentiert diese Auseinandersetzung nicht
explizit. Um seine Vorgehensweise zu verstehen, sollten wir jedoch
eine dritte Möglichkeit erwägen, die sich im Laufe der Explikation als
korrekt erweisen wird: So verfügen wir zunächst
über den Begriff des Verstandes, ohne zwischen einem endlichen und
einem unendlichen Verstand zu differenzieren. Ein Verstand wiederum ist entweder
diskursiv oder intuitiv, je nachdem, ob er lediglich denken kann oder anschaut.
Es handelt sich also weder bei der Vorstellung eines
intuitiven Verstandes, noch bei der Vorstellung einer intellektuellen Anschauung um Derivate ausgehend
von unseren Vorstellungen von einem diskursiven Verstand oder einer sinnlichen
Anschauung. Und wir benötigen auch keinen Begriff nicht-endlicher
Erkenntnisvermögen als Ursprung der Vorstellung unserer endlichen
Vermögen.\footnote{Es ist sicherlich korrekt, dass ein Verständnis einer
Fähigkeit und einer Tätigkeit dem Verständnis der fehlerhaften Ausübung vorausgeht.
Der Begriff \enquote{sich verrechnen} ist derivativ zu dem Begriff
\enquote{rechnen}: Nur wer (richtig) rechnen kann, kann
sich auch verrechnen; und nur wer weiß, was \enquote{rechnen} heißt,
kann auch wissen, was heißt, sich zu verrechnen. Wir können den Ausdruck \enquote{sich verrechnen} nur
verstehen als Beschreibung der fehlerhaften Ausführung der Operation des
Rechnens. Um es mit \authorfullcite{Ryle:TheConceptofMind2002} zu sagen: Dass jemand rechnen kann, beschreibt eine Kompetenz; dass er
sich auch verrechnen kann, beschreibt eine Anfälligkeit.
Kompetenzen gehen Anfälligkeiten logisch voraus: Um eine Anfälligkeit zu haben,
muss man zunächst über eine entsprechende Kompetenz verfügen. Wer nie rechnen
lernte, kann sich auch nicht verrechnen \parencite[vgl. hierzu][\pno~60, 130\,f.]{Ryle:TheConceptofMind2002}.
Aber daraus folgt gerade nicht, dass wir einen Begriff von Unfehlbarkeit
bräuchten, um unsere fehlbaren Vermögen zu verstehen.
\authorfullcite{Ryle:TheConceptofMind2002} behauptet, es gehöre zum
Verfügen über eine Fähigkeit dazu, auch für Fehler anfällig zu sein
\parencite[vgl.][130]{Ryle:TheConceptofMind2002}. Sollte das stimmen,
dann ist ein entsprechendes Vermögen überhaupt nur als endliches denkbar.}

Nun schreibt \name[Immanuel]{Kant} des öfteren, wir könnten die Möglichkeit
eines \singlequote{anderen} Verstandes oder einer \singlequote{anderen}
Anschauung nicht einsehen.\footnote{Vgl. \cite[][B
213]{Kant:KritikderreinenVernunft2003}, \cite[][III:
212.20--21]{Kant:GesammelteWerke1900ff.}} Ist es vielleicht nach
\name[Immanuel]{Kant} der Fall, dass wir
über gar keinen Begriff und keine Vorstellung eines unendlichen Verstandes
verfügen? Haben wir vielleicht nur eine Vorstellung von unserem eigenen,
endlichen Verstand. Es wäre merkwürdig, wenn \name[Immanuel]{Kant} sagen wollte,
dass wir über keinen Begriff eines nicht-endlichen Verstandes verfügen, denn
dann verstünden wir auch das nicht, was er mit Hilfe dieses Begriffs zu
erläutern versucht.  Denn nur durch die Konzeption eines
\singlequote{anderen} Verstandes und einer \singlequote{anderen} Anschauung sei
es möglich zu erkennen, dass wir die Dinge lediglich so erkennen, wie sie uns
erscheinen, nicht so, wie sie an sich sind.\footnote{\authorfullcite{Allison:KantsTranscendentalIdealism2004} betont,
dass der transzendentale Idealismus als Resultat der Analyse unserer
Endlichkeit -- \authorcite{Allison:KantsTranscendentalIdealism2004} schreibt
\enquote{discursivity} -- zu interpretieren sei
\parencite[vgl.][1--73]{Allison:KantsTranscendentalIdealism2004}.} Was wir nicht einsehen, ist vielmehr dieses: Wir
können die Wirkungsweise eines nicht-endlichen Verstandes
nicht näher erläutern. Die Wirkungsweise unseres eigenen Verstandes können wir analysieren,
die Wirkungsweise eines gänzlich anderen Verstandes ist eine \emph{terra
incognita}, über die wir nichts weiter sagen können, als diesen allgemeinen Begriff anzugeben,
dass er nicht denkt, sondern anschaut. Dieser ist konsistent, stellt
aber keine reale Möglichkeit dar, weil wir nicht wissen, wie er realisierbar
sein könnte.\footnote{Ganz ähnlich sagt
\authorcite{Descartes:OeuvresdeDescartes1983}, ein unendlicher
Verstand könne von uns nicht begriffen werden, weil wir selbst endlich sind.
\enquote{[E]st {\punkt} de ratione infiniti, ut a me, qui sum finitus, non
comprehendatur} \mkbibparens{\cite[][VII:
46.21--23]{Descartes:OeuvresdeDescartes1983}}.
Dennoch verfügten wir über eine Idee eines solchen Verstandes.
Zu beachten ist, dass \authorcite{Descartes:OeuvresdeDescartes1983} für das, was wir
\emph{nicht} können, den Ausdruck \enquote{comprehendere}
(\singlequote{begreifen}) verwendet, für das, was wir können, hingegen den
Ausdruck \enquote{intendere} (verstehen) bzw. \enquote{percipere}. Siehe dazu
auch die Stufen des Erkennens nach \name[Immanuel]{Kant} in \cite[][A
96\,f.,]{Kant:ImmanuelKantsLogik1977} \cite[][IX:
64.33--65.24]{Kant:GesammelteWerke1900ff.}. Siehe zur logischen im Gegensatz
zur realen Möglichkeit auch folgende Anmerkung zur Vorrede der zweiten Auflage
der \titel{Kritik der reinen Vernunft}: \cite[][B
xxvi]{Kant:KritikderreinenVernunft2003},
\cite[][III: 17.29--38]{Kant:GesammelteWerke1900ff.}.}

Ich gehe also davon aus, dass wir (mindestens) einen Begriff von einem
\singlequote{anderen} Verstand haben und dass dieser sich von unserem Verstand
nicht nur graduell, sondern qualitativ unterscheidet. Der Begriff dieses
Verstandes wird wie der Begriff unseres eigenen Verstandes im Ausgang des
Begriffs eines Verstandes überhaupt gebildet, unter den unserer ebenso
wie der \singlequote{andere} Verstand fallen, indem ihm bestimmte
Eigenschaft jeweils zu- oder abgesprochen werden. (Dasselbe wird auch
für den Begriff einer
sinnlichen \emph{respective} einer intellektuellen Anschauung gelten.) Im
weitere Verlauf wird es darum gehen, zunächst den Begriff des Verstandes unabhängig
davon, ob es sich um einen endlichen oder unendlichen Verstand handelt, zu
explizieren und anschließend herauszuarbeiten, was es ist, das dem
endlichen und unendlichen Verstand jeweils zu- oder abgesprochen wird.


\subsection{Eine mögliche
Vieldeutigkeit}\label{subsection:EineMoeglicheVieldeutigkeit}
Während die Wörter \enquote{endlich} und \enquote{unendlich} bei
\name[Immanuel]{Kant} eher selten zur Charakterisierung von Erkenntnisvermögen
herangezogen werden, finden sich sehr unterschiedliche Bezeichnungen, deren
systematische Zusammenhänge sich nicht von selbst verstehen. An vielen Stellen
der Schriften \name[Immanuel]{Kant}s kommt die Beschreibung eines Verstandes
vor, der nicht wie der unsrige nur denken kann und zur Gewährleistung des
Gehalts seines Denkens (zur Gewinnung von Anschauungen) auf die Sinne
rekurrieren muss, sondern der selbst anschaut.\footnote{\enquote{Ein Verstand,
in welchem durch das Selbstbewußtsein zugleich alles Mannigfaltige gegeben
würde, würde \ori{anschauen}; der unsere kann nur \ori{denken} und muß in den
Sinnen die Anschauung suchen} \mkbibparens{\cite[][B
135]{Kant:KritikderreinenVernunft2003}, \cite[][III:
110.26--29]{Kant:GesammelteWerke1900ff.}}. Ohne das Mannigfaltige der Anschauung
wäre unser Denken, wie \name[Immanuel]{Kant} durchgängig betont, leer.
\cite[Siehe z.\,B. den bekannten Ausspruch in][B
75]{Kant:KritikderreinenVernunft2003}, \cite[][III: 75.14--15]{Kant:GesammelteWerke1900ff.}:
\enquote{Gedanken ohne Inhalt sind leer, Anschauungen ohne Begriffe sind
blind.} An anderer Stelle schreibt er: \enquote{\ori{Verstand} und
\ori{Sinnlichkeit} können bei uns \ori{nur in Verbindung} Gegenstände
bestimmen. Wenn wir sie trennen, so haben wir Anschauungen ohne Begriffe, oder
Begriffe ohne Anschauungen, in beiden Fällen aber Vorstellungen, die wir auf
keinen bestimmten Gegenstand beziehen können}
\mkbibparens{\cite[][B 314]{Kant:KritikderreinenVernunft2003},
\cite[][III: 213.32--36]{Kant:GesammelteWerke1900ff.}}.} Er beschreibt einen
solchen Verstand auch als einen Verstand, der durch eine nicht-sinnliche
Anschauung seinen Gegenstand \singlequote{intuitiv} erkennt.\footnote{\cite[][B
311\,f.,]{Kant:KritikderreinenVernunft2003} \cite[][III:
212.16--21]{Kant:GesammelteWerke1900ff.}.} Ebenfalls in der \titel{Kritik der
reinen Vernunft} spricht \name[Immanuel]{Kant} von einer
\enquote{gesetzgebende[n] Vernunft (intellectus archetypus) {\punkt}, von der
alle systematische Einheit der Natur, als dem Gegenstande unserer Vernunft,
abzuleiten sei.}\footnote{\cite[][B 723]{Kant:KritikderreinenVernunft2003},
\cite[][III: 456.37--457.2]{Kant:GesammelteWerke1900ff.}.} Der Ausdruck
\enquote{\emph{intellectus archetypus}} findet sich auch in der \titel{Kritik
der Urteilskraft} zur Bezeichnung des \emph{intuitiven Verstandes} als
Gegenbegriff zu unserem diskursiven Verstand, den \name[Immanuel]{Kant}
\enquote{intellectus ectypus}
nennt.\footnote{\cite[Vgl.][\S~77]{Kant:KritikderUrteilskraft2009},
\cite[][V: 408.18--23]{Kant:GesammelteWerke1900ff.}.}


In einer \enquote{nur episodisch, zur
Erläuterung}\footnote{\cite[][\S~76]{Kant:KritikderUrteilskraft2009}, \cite[][V:
401.6--7]{Kant:GesammelteWerke1900ff.}.} angeführten Anmerkung in der
\titel{Kritik der Urteilskraft} finden wir die systematischste Ausarbeitung von
\name[Immanuel]{Kant}s Überlegungen zu einem menschlichen Denken:
\begin{nummerierung}
 \item Unser menschlicher Verstand müsse Möglichkeit und Wirklichkeit der Dinge
 unterscheiden, während ein anschauender Verstand nur das Wirkliche zum
 Gegenstand hätte.\footnote{\cite[Vgl.][\S~76]{Kant:KritikderUrteilskraft2009};
 \cite[][V: 401.31--403.19]{Kant:GesammelteWerke1900ff.}.} Unser
 Verstand sei ein Vermögen der Begriffe, das nur denken, nicht aber anschauen
 kann und daher auf Anschauungen angewiesen ist, die ihm von den Sinnen
 gegeben werden.\footnote{\cite[Vgl.][B
 92\,f.,]{Kant:KritikderreinenVernunft2003} \cite[][III:
 85.10--16]{Kant:GesammelteWerke1900ff.}. Dort sagt \name[Immanuel]{Kant}, dass
 die \emph{Erkenntnis} des menschlichen Verstandes diskursiv und nicht intuitiv
 sei \mkbibparens{ebenso \cite[][B 311\,f.,]{Kant:KritikderreinenVernunft2003}
 \cite[][III: 166.37--167.5]{Kant:GesammelteWerke1900ff.}}.
 Siehe dazu auch \cite[][B 135]{Kant:KritikderreinenVernunft2003}, \cite[][III:
 110.26--29]{Kant:GesammelteWerke1900ff.}, wo unser endlicher Verstand als
 Vermögen zu \emph{denken} beschrieben wird.}
 
 \item Unsere praktische \emph{Vernunft} ist endlich, insofern ihre Gesetze uns
 als Imperative entgegentreten, weil bei endlichen Wesen wie uns erstens die
 Einsicht in Gesetze, Regeln und Ratschläge nicht automatisch die
 Handlungsausführung herbeiführt, wir eben auch anders handeln können, und
 wir uns zweitens als endliche Wesen mit antagonistischen Neigungen
 konfrontiert sehen, die uns  gerade zum Zuwiderhandeln \emph{gegen}
 die Imperative der Vernunft verleiten.

 \item\label{IntuitiverVerstandvomAllgemeinenzumBesonderen} Nach Auskunft von
 \S~77 der \titel{Kritik der Urteilskraft} \enquote{können wir uns {\punkt}
 einen Verstand denken, der, weil er nicht wie der unsrige diskursiv, sondern
 intuitiv ist, vom Synthetisch-Allgemeinen (der Anschauung eines Ganzen als
 eines solchen) zum Besonderen geht, d.\,i. vom Ganzen zu den
 Teilen}\footnote{\cite[][\S~77]{Kant:KritikderUrteilskraft2009}, \cite[][V:
 407.19--23]{Kant:GesammelteWerke1900ff.}.}. Eben die entgegengesetzte
 Eigentümlichkeit unseres Verstandes, vom Analytisch-Allgemeinen (von Begriffen)
 zum Besonderen gehen zu müssen, sei verantwortlich dafür, dass die mechanische und
 die teleologische Erklärungsart von Naturprodukten \emph{aus unserer Sicht} als
 unterschiedliche Arten der Erklärung nebeneinander bestünden. Einen intuitiven
 Verstand nennt \name[Immanuel]{Kant} hier auch \enquote{\emph{intellectus
 archetypus}} im Gegensatz zu unserem \enquote{\emph{intellectus
 ectypus}}\footnote{\cite[Siehe][\S~77]{Kant:KritikderUrteilskraft2009},
 \cite[][V: 408.18--23]{Kant:GesammelteWerke1900ff.}.}, wobei offen bleibt, ob
 dieser Begriff identisch ist mit dem eines \emph{intellectus archetypus} als
 \singlequote{gesetzgebender Vernunft} in der \titel{Kritik der reinen
 Vernunft}\footnote{\cite[Siehe][B 723]{Kant:KritikderreinenVernunft2003},
 \cite[][III: 456.37]{Kant:GesammelteWerke1900ff.}.}.
\end{nummerierung}


Es liegt \emph{prima facie} nahe davon auszugehen, dass \name[Immanuel]{Kant} in
einem einheitlichen Sinn von unserem menschlichen, endlichen Verstand, einem
diskursiven Verstand oder einem \emph{intellectus ectypus} und entsprechend einheitlich von
einem nicht-endlichen oder intuitiven Verstand oder einem \emph{intellectus
archetypus} spricht. Nicht ganz so nah liegt es, den intuitiven Verstand zugleich
mit der intellektuellen Anschauung zu identifizieren. Andererseits drängt sich
folgender Gedankengang auf: Wenn ein Verstand diskursiv ist, der denkt und nicht
anschaut, ein intuitiver Verstand aber anschaut, dann muss man doch vermuten,
dass die Anschauungen eines solchen Verstandes eben intellektuelle im Gegensatz
zu sinnlichen Anschauungen sind; schließlich entstammen sie dem Verstand
(\emph{intellectus}) und nicht den Sinnen. Es ist nichts natürlicher, als die
intellektuellen Anschauungen eben als die Produkte eines anschauenden Verstandes
anzusehen.


Nun kann man dennoch mit
\authorfullcite{Foerster:DieBedeutungvonSS7677deremphKritikderUrteilskraftfuerdieEntwicklungdernachkantischenPhilosophieTeil12002}
die Auffassung vertreten, dass solche Annahmen zumindest
begründungsbedürftig sind. Es handelt sich schließlich um gehaltvolle
philosophiehistorische Behauptungen mit erheblicher Tragweite für unser
Verständnis der Philosophie \name[Immanuel]{Kant}s, wie auch der
nach-kantischen Philosophie von \name[Salomon]{Maimon} über
\authorcite{Fichte:DieBestimmungdesMenschen1800} und
\authorcite{Schelling:Historisch-kritischeAusgabe1976-} bis
\authorcite{Hegel:GesammelteWerke}.\footnote{Insbesondere
\authorcite{Hegel:GesammelteWerke} und
\authorcite{Schelling:Historisch-kritischeAusgabe1976-} wird vorgeworfen
vorauszusetzen, dass die intellektuelle Anschauung der
intuitive Verstand sei. \enquote{Dazu ist festzuhalten, dass Hegel sich bis 1803
hinsichtlich des Absoluten genau wie Schelling durchgängig am Urgrundgedanken
des \S~76 der Kritik der Urteilskraft orientiert und, genau wie dieser,
intellektuelle Anschauung und intuitiven Verstand identifiziert bzw.
nicht zwischen ihnen unterscheidet. Der wesentliche Unterschied zwischen den
\S\S~76 und 77, also zwischen dem, was nach Kant zwei je verschiedene Grenzen
des menschlichen Erkenntnisvermögens bezeichnet, wird von beiden gleichermaßen
übersehen}
\parencite[][325]{Foerster:DieBedeutungvonSS7677deremphKritikderUrteilskraftfuerdieEntwicklungdernachkantischenPhilosophieTeil22002}.
Dabei identifiziert \authorcite{Hegel:GesammelteWerke}
den intuitiven Verstand und damit zugleich die intellektuelle Anschauung
mit der transzendentalen Einbildungskraft, um zu zeigen, dass
\name[Immanuel]{Kant} inkonsequent verfahre, wenn er die Möglichkeit eines
intuitiven Verstandes leugnet: \enquote{Von dieser Idee erkennt Kant zugleich,
daß wir nothwendig auf sie getrieben werden, und die \ori{Idee} dieses
urbildlichen, \ori{intuitiven Verstandes} ist im Grunde
durchaus nichts anders als \ori{dieselbe Idee der transcendentalen
Einbildungskraft}, die wir oben betrachteten, denn sie ist
anschauende Thätigkeit, und zugleich ist ihre innere Einheit gar keine andere,
als die Einheit des Verstandes selbst, die Kategorie in die Ausdehnung versenkt, die
erst Verstand und Kategorie wird, insofern sie sich von der Ausdehnung
absondert; die transcendentale Einbildungskraft ist also selbst anschauender
Verstand}
\mkbibparens{\cite[][IV: 341.1--8]{Hegel:GesammelteWerke}}. Während es mir korrekt zu sein
scheint, intellektuelle Anschauung und intuitiven Verstand zwar nicht zu
identifizieren, aber doch zu sagen, dass intellektuelle Anschauungen eben die
Erkenntnisse sind, die ein Verstand generiert, der nicht denkt, sondern
anschaut, also ein intuitiver Verstand, ist die Identifizierung von intuitivem
Verstand und transzendentaler Einbildungskraft freilich zurückzuweisen.}
In Opposition zu leichtfertigen Identifizierungen ließe sich eben auch denken,
dass nur die \emph{Funktion} der beschriebenen Verstandesarten gleich
sei.\footnote{So behauptet auch
\authorfullcite{McLaughlin:KantsKritikderteleologischenUrteilskraft1989}, es
gebe bei \name[Immanuel]{Kant} eine ganze Reihe \singlequote{intuitiver}
Verstandesarten, die keine gemeinsamen Eigenschaften haben, aber alle eine
gemeinsame Funktion als Vergleichsverstand
\parencite[vgl.][\pno~153\,f.]{McLaughlin:KantsKritikderteleologischenUrteilskraft1989}.}
Sie werden konstruiert, um Eigentümlichkeiten unseres endlichen Denkens vor Augen zu führen.
Nichts garantiert dabei, dass es sich stets um dieselbe Eigentümlichkeit handelt. 
Und so differenziert \authorcite{Foerster:Die25JahrederPhilosophie2011} streng
zwischen den drei Charakterisierungen unserer Endlichkeit:
\begin{quote}
\ori{Weil} wir in Verstand und Sinnlichkeit zwei voneinander unabhängige Stämme der
Erkenntnis haben, müssen wir zwischen Möglichkeit und Wirklichkeit unterscheiden
(anders: eine intellektuelle Anschauung); \ori{weil} wir sowohl Sinnen- als auch
Vernunftwesen sind, erscheint uns das Sittengesetz als ein Sollen, nicht als ein
Sein oder Wollen (anders: ein heiliger Wille); \ori{weil} unser Verstand diskursiv
ist, beurteilen wir Organismen unweigerlich als Naturzwecke (anders: ein
intuitiver Verstand).\footcite[][153]{Foerster:Die25JahrederPhilosophie2011}
\end{quote}
In der \titel{Kritik der reinen Vernunft} gehe es um die Besonderheit unserer
\emph{Anschauung} als sinnlicher und intellektueller; die \titel{Grundlegung
zur Metaphysik der Sitten} und die \titel{Kritik der praktischen Vernunft}
thematisierten hingegen die Besonderheit unseres endlichen im Vergleich zu einem
heiligen \emph{Willen} und schließlich die \titel{Kritik der Urteilskraft} die
Besonderheit unseres diskursiven im Kontrast zu einem intuitiven
\emph{Verstand}.\footnote{\authorcite{Foerster:DieBedeutungvonSS7677deremphKritikderUrteilskraftfuerdieEntwicklungdernachkantischenPhilosophieTeil12002}
geht es dabei vor allem um den Unterschied zwischen der in der \titel{Kritik der
reinen Vernunft} thematisierten Besonderheit unseres Erkenntnisvermögens und der
in der \titel{Kritik der Urteilskraft} diskutierten Diskursivität unseres
Verstandes. Die Einordnung der Endlichkeit unseres Willens wird nicht eigens
thematisiert.}
\begin{comment}
\begin{figure}[htb]
\begin{minipage}[t]{\textwidth}
\centering
\begin{tikzpicture}[edge from parent fork down,
level 1/.style={sibling distance=6cm, level distance=0cm},
level 2/.style={sibling distance=4cm, level distance=0cm},
level 3/.style={sibling distance=3cm, level distance=1.5cm},
level 4/.style={sibling distance=3cm, level distance=1.5cm},
level 5/.style={sibling distance=3cm, level distance=1.5cm},
every node/.style={rectangle,draw=black,fill=gray!25, thin, inner sep=0.5em, minimum size=0.5em, align=center},
edge from parent/.style={draw=none},
mylabel/.style={draw=none, fill=none, text width=5cm,text centered, inner sep=0.5em, anchor=base} ]
\node[draw=none,fill=none] {}
child {node[draw=none,fill=none] {}
	child {node[draw=none,fill=none] {}
		child {node[text width=3cm,rounded corners,thick] (KrV) {\emph{KrV}}
		child {node[text width=3cm,rounded corners,thick] (KpV) {\emph{KpV}, \emph{GMS}}
		child {node[text width=3cm,rounded corners,thick] (KU) {\emph{KU}}}}}}}
child {node[draw=none,fill=none] {}
	child {node[text width=3cm,rounded corners,thick] (endlichkeit) {Unser
	endliches Vermögen} child {node[fill=none,draw=none,text width=3cm]
	(Anschauung) {sinnliche Anschauung} child {node[fill=none,draw=none,text width=3cm] (Wille) {endlicher Wille}
		child {node[fill=none,draw=none,text width=3cm] (Verstand) {diskursiver
		Verstand}}}}} child {node[text width=3cm,rounded corners,thick] (gegenbegriff)
		{\singlequote{Anderes} Vermögen} child {node[fill=none,text
		width=3cm,draw=none] (iA) {intellektuelle Anschauung}
		child {node[fill=none,text width=3cm,draw=none] (hW) {heiliger
		Wille}
		child {node[fill=none,text width=3cm,draw=none] (iV) {intuitiver
		Verstand}}}}}}; \draw[<->] (Anschauung.east) to (iA.west);
\draw[<->] (Wille.east) to (hW.west);
\draw[<->] (Verstand.east) to (iV.west);
% \draw[->] (KrV.east) to (Anschauung.west);
% \draw[->] (KpV.east) to (Wille.west);
% \draw[->] (KU.east) to (Verstand.west);
\end{tikzpicture}
  \caption{Gegenüberstellungen nach
  \authorcite{Foerster:Die25JahrederPhilosophie2011}}\label{abbildung:Foerster:Gegenueberstellungen}
\end{minipage}
\end{figure}
\end{comment}


\phantomsection\label{Abschnitt:FoerstersDifferenzierungDerArtenIntuivenVerstandes}

\authorcite{Foerster:Die25JahrederPhilosophie2011} warnt vor allem davor, die
intellektuelle Anschauung der ersten Kritik mit dem intuitiven Verstand der
dritten Kritik zu identifizieren. Zwar habe auch \name[Immanuel]{Kant} zunächst
nicht klar zwischen beiden differenziert, doch spätestens in den Ausführungen der \titel{Kritik der
Urteilskraft} seien beide systematisch unterschieden. Während die intellektuelle
Anschauung, die \name[Immanuel]{Kant} in \S~76 der \titel{Kritik der
Urteilskraft} thematisiert, Überlegungen zum Gegensatz von Rezeptivität und
Spontaneität zu explizieren helfe, betreffe die Konzeption
eines intuitiven Verstandes in \S~77 den Unterschied \emph{diskursiver} und
\emph{intuitiver} Erkenntnisse.\footnote{\cite[Vgl.][177]{Foerster:DieBedeutungvonSS7677deremphKritikderUrteilskraftfuerdieEntwicklungdernachkantischenPhilosophieTeil12002}.}

Nun spricht auch der \S~77 der \titel{Kritik der Urteilskraft} von einer
intellektuellen Anschauung, wie
\authorcite{Foerster:DieBedeutungvonSS7677deremphKritikderUrteilskraftfuerdieEntwicklungdernachkantischenPhilosophieTeil12002}
einräumen
muss.\footcite[Vgl.][178]{Foerster:DieBedeutungvonSS7677deremphKritikderUrteilskraftfuerdieEntwicklungdernachkantischenPhilosophieTeil12002}
Somit ist die ursprüngliche Einteilung in dieser Einfachheit letztlich nicht
haltbar; stattdessen sieht sich \authorcite{Foerster:Die25JahrederPhilosophie2011} gezwungen, eine
Mehrdeutigkeit auch des Begriff \enquote{intellektuelle Anschauung} und mehrere
Konzeptionen eines \singlequote{anderen} Verstandes anzunehmen.
Somit seien verschiedene Konzeptionen von intellektueller Anschauung und intuitivem Verstand
zu unterscheiden.\footnote{Siehe zu diesen Unterscheidungen bei
\authorcite{Foerster:DieBedeutungvonSS7677deremphKritikderUrteilskraftfuerdieEntwicklungdernachkantischenPhilosophieTeil12002}
auch \cite[][151]{Quarfood:DiscursivityandTranscendentalIdealism2012}.} Manche
dieser Konzeptionen seien als unendlich oder \singlequote{göttlich} anzusehen.
Andere wiederum seien nicht als unendlich konzipiert, sondern stellten andere
endliche Formen des Verstandes und des Anschauens dar. Zunächst sind nach
\authorcite{Foerster:DieBedeutungvonSS7677deremphKritikderUrteilskraftfuerdieEntwicklungdernachkantischenPhilosophieTeil12002}
folgende zwei Verständnisse von \enquote{intellektuelle Anschauung} zu
unterscheiden:
\begin{nummerierung}
  \item Die intellektuelle Anschauung, die in der Deduktion der reinen
  Verstandesbegriffe in der \titel{Kritik der reinen Vernunft} zum Vergleich
  herangezogen wird, sei eine \emph{produktive Anschauung}, die ihre
  Gegenstände selbst hervorbringe. Sie sei \enquote{produktive Einheit von
  Möglichkeit (Denken) und Wirklichkeit
  (Sein)}\footcite[][179]{Foerster:DieBedeutungvonSS7677deremphKritikderUrteilskraftfuerdieEntwicklungdernachkantischenPhilosophieTeil12002}.
  Das \emph{könne} heißen, dass die produktive Anschauung das Ganze der Welt
  hervorbringt, aber die Konzeption einer produktiven Anschauung verpflichte
  doch nicht darauf und sei daher nicht \emph{per se} überschwenglich.
  \authorcite{Fichte:DieBestimmungdesMenschen1800} und
  \name[Friedrich Wilhelm Joseph]{Schelling} hätten die
  Realisierbarkeit einer solchen Anschauung behauptet und für erfahrbar
  gehalten.
  \item Eine andere Form der intellektuellen Anschauung bringe ihre
  Gegenstände nicht selbst hervor, erkenne aber die Dinge an sich, da sie keine
  sinnliche Anschauung und daher nicht den Bedingungen unserer Sinnlichkeit
  unterworfen sei. (Die Abhängigkeit unserer Wahrnehmung von der Form unserer
  Sinnlichkeit macht \name[Immanuel]{Kant} dafür verantwortlich, dass wir die
  Dinge nur so wahrnehmen, wie sie uns \emph{erscheinen}, nicht so, wie sie
  \emph{an sich} sind.\footnote{\cite[Vgl.][B 59]{Kant:KritikderreinenVernunft2003},
  \cite[][III: 65.9--22]{Kant:GesammelteWerke1900ff.}.}) Als Beispiele für
  Konzeptionen nicht-sinnlicher Anschauungen verweist
  \authorcite{Foerster:Die25JahrederPhilosophie2011} auf antike
  \singlequote{Sehstrahltheorien}; in einer solchen Theorie werde unsere
  Anschauung beschrieben als nicht rezeptiv und daher fähig, die Dinge als Dinge
  an sich zu erkennen und nicht bloß, wie sie uns erscheinen.\footnote{Vgl.
  \cite[][178]{Foerster:DieBedeutungvonSS7677deremphKritikderUrteilskraftfuerdieEntwicklungdernachkantischenPhilosophieTeil12002}.
  Warum solche Anschauungen dies leisten können sollten, erläutert
  \authorcite{Foerster:Die25JahrederPhilosophie2011} jedoch nicht.} Es sei diese
  Form der intellektuellen Anschauung, die in \S~77 der \titel{Kritik der
  Urteilskraft} bemüht werde und die sich zuvor im
  \titel{Noumena und Phaenomena}-Kapitel der \titel{Kritik der reinen Vernunft}
  finde. Unter den Philosophen, die im Anschluss an \name[Immanuel]{Kant}
  philosophierten, habe keiner eine solche übersinnliche Anschauung für möglich
  gehalten.
\end{nummerierung}
\authorfullcite{Prien:KantsLogikderBegriffe2006} betont, dass nur die
produktive intellektuelle Anschauung die Anschauung eines
göttlichen Verstandes sei, während derselbe Ausdruck oft auch lediglich auf eine
\singlequote{übersinnliche Anschauung} verweise, die unserem Verstand ebenso ein Mannigfaltiges darbiete, das
zwar nicht von den Bedingungen unserer Sinnlichkeit abhänge, aber dennoch die
Gegenstände der Erkenntnis nicht hervorbringe. Dabei denkt
\authorcite{Prien:KantsLogikderBegriffe2006} nicht wie
\authorcite{Foerster:Die25JahrederPhilosophie2011} an antike Sehstrahltheorien,
sondern an eine \singlequote{Ideenschau} im Sinne der Mathematikauffassung
\singlename{Platon}s.\footnote{\cite[Vgl.][96]{Prien:KantsLogikderBegriffe2006}.
Auch \authorfullcite{Duesing:NaturteleologieundMetaphysikbeiKantundHegel1990} sieht
\name[Immanuel]{Kant}s intellektuelle Anschauung in der
platonisch-neuplatonischen Tradition der Ideenschau; die Ideen seien mit den
intellektuellen Anschauungen sogar zu identifizieren
\mkbibparens{\cite[vgl.][\pno~144\,f.,]{Duesing:NaturteleologieundMetaphysikbeiKantundHegel1990},
siehe außerdem
\cite[][\pno~72\,f.]{Duesing:DieTeleologieinKantsWeltbegriff1968}}.} Durch eine
solche übersinnliche Anschauung wäre uns -- wenn wir über sie verfügten --
ebenso wie durch die sinnliche Anschauung ein Mannigfaltiges gegeben, welches
der Synthesis durch unseren Verstand bedürfte. Eine übersinnliche Anschauung
vertrage sich daher mit der Endlichkeit unseres
Verstandes.\footnote{\cite[Vgl.][97]{Prien:KantsLogikderBegriffe2006}.}

Nun ist es hier nicht von weiterer Relevanz, ob es sich bei der nicht-sinnlichen
Anschauung, die jedoch nicht produktiv sein soll, eher um eine Art Sehstrahl
oder um eine Art Ideenschau handelt. Ich werde sie der Kürze halber im folgenden
schlicht als \emph{übersinnliche} Anschauung bezeichnen. Es ist somit nach
\authorcite{Prien:KantsLogikderBegriffe2006} wie nach \authorcite{Foerster:Die25JahrederPhilosophie2011}
zwischen zwei Arten einer nicht-sinnlichen Anschauung zu unterscheiden:
der übersinnlichen und der produktiven Anschauung. Beide Konzeptionen
wären je nach konkreter Ausgestaltung auch mit einem endlichen
Verstand kombinierbar.

Eine ähnliche Unterscheidung wie diejenige zwischen produktiver und
übersinnlicher Anschauung lässt sich nach
\authorcite{Foerster:DieBedeutungvonSS7677deremphKritikderUrteilskraftfuerdieEntwicklungdernachkantischenPhilosophieTeil12002}
zwischen zwei Arten, den Terminus \enquote{intuitiver Verstand} zu explizieren,
rekonstruieren.\footcite[Vgl.][\pno~178\,f.]{Foerster:DieBedeutungvonSS7677deremphKritikderUrteilskraftfuerdieEntwicklungdernachkantischenPhilosophieTeil12002}
Beide Verwendungsweisen finden sich in \S~77 der
\titel{Kritik der Urteilskraft} und entsprechen einer bekannten
\singlequote{Doppelfunktion} des unserem endlichen kontrastierten Verstandes\footnote{\phantomsection\label{FussnoteDoppelfunktion}Dieser dient einerseits als
gedanklicher Kontrast zu Beschreibung der Besonderheiten unseres Verstandes, andererseits aber auch als
übersinnlicher Grund der Natureinheit \mkbibparens{vgl.
\cite[][68]{Duesing:DieTeleologieinKantsWeltbegriff1968}}.
\authorfullcite{Duesing:DieTeleologieinKantsWeltbegriff1968} geht jedoch
offensichtlich davon aus, dass es sich stets um die Konzeption desselben
(göttlichen) Verstandes handelt. Siehe dazu
\cite[][66--74]{Duesing:DieTeleologieinKantsWeltbegriff1968}.} in
der Argumentation dieses Paragraphen:
\begin{nummerierung}
  \item Einerseits werde der intuitive Verstand am Ende von \S~77 als
  \emph{ursprünglicher Verstand} beschrieben, der als \emph{Weltursache} zu
  denken wäre.\footnote{Siehe \cite[][\S~77]{Kant:KritikderUrteilskraft2009},
  \cite[][V: 410.11]{Kant:GesammelteWerke1900ff.}. Auch wenn
  \authorcite{Foerster:Die25JahrederPhilosophie2011} dies an dieser Stelle
  nicht ausführt, könnte man in diesem Kontext auch den \emph{intellectus
  archetypus} der \titel{Kritik der reinen Vernunft} einordnen als
  \enquote{gesetzgebende Vernunft (intellectus archetypus), {\punkt}, von der
  alle systematische Einheit der Natur, als dem Gegenstande unserer Vernunft,
  abzuleiten sei} \mkbibparens{\cite[][B 723]{Kant:KritikderreinenVernunft2003},
  \cite[][III: 456.37--457.2]{Kant:GesammelteWerke1900ff.}}.} Ein solcher
  Verstand erkennte nicht nur anders als wir, er brächte sogar den Gegenstand
  seiner Erkenntnis hervor. \enquote{Hier besteht zweifellos die größte Nähe
  zwischen intuitivem Verstand und produktiver, intellektueller Anschauung,
  obwohl letztere natürlich nicht als Ursache gleich des Ganzen der Welt gedacht
  werden muß, sondern diese Möglichkeit nur
  zulässt.}\footnote{\cite[][179]{Foerster:DieBedeutungvonSS7677deremphKritikderUrteilskraftfuerdieEntwicklungdernachkantischenPhilosophieTeil12002}.
  Die Realisierbarkeit eines solchen ursprünglichen Verstandes sei im Anschluss
  an \name[Immanuel]{Kant} von niemandem behauptet
  worden \parencite[vgl.][179]{Foerster:DieBedeutungvonSS7677deremphKritikderUrteilskraftfuerdieEntwicklungdernachkantischenPhilosophieTeil12002}.}
    \item Andererseits charakterisiere Kant den Verstand als
  \emph{synthetisch-allgemeinen Verstand}, also als einen solchen, der nicht wie
  der unsrige vom Analytisch-All\-ge\-mei\-nen, sondern vom
  \enquote{\ori{Synthetisch-Allgemeinen} (der Anschauung eines Ganzen als
  eines solchen)}\footnote{\cite[][\S~77]{Kant:KritikderUrteilskraft2009},
  \cite[][V: 407.21--22]{Kant:GesammelteWerke1900ff.}.} aus zum Besonderen gehe.
  Er könne auch bei einzelnen Naturprodukten das Besondere aus dem
  Synthetisch-All\-ge\-mei\-nen bestimmen. Damit beschreibe
  \name[Immanuel]{Kant} einen Verstand, der der \emph{scientia intuitiva} bei
  \authorcite{Spinoza:EthikingeometrischerOrdnungdargestellt2007}
  entspreche.\footnote{\cite[Vgl.][189]{Foerster:DieBedeutungvonSS7677deremphKritikderUrteilskraftfuerdieEntwicklungdernachkantischenPhilosophieTeil12002}.
  Es sei \name[Johann Wolfgang von]{Goethe} gewesen, der die Möglichkeit eines
  solchen Verstandes als erster in Erwägung gezogen und damit
  \authorcite{Hegel:GesammelteWerke}s Bruch mit
  \authorcite{Schelling:Historisch-kritischeAusgabe1976-} eingeleitet
  habe \parencite[vgl.][180--190]{Foerster:DieBedeutungvonSS7677deremphKritikderUrteilskraftfuerdieEntwicklungdernachkantischenPhilosophieTeil12002}.}
\end{nummerierung}
\phantomsection\label{Absatz:IntuitiverVerstandIntellektuelleAnschauungEndlicherWesen}
Ein synthetisch-allgemeiner Verstand müsse nun als solcher noch kein
ursprünglicher Verstand sein, der als Ursache der Welt zu denken wäre -- genau
so wenig, wie eine übersinnliche Anschauung als produktiv oder eine produktive
Anschauung als Anschauung von Dingen an sich selbst zu denken wäre.
Synthetisch-allgemeiner Verstand und übersinnliche wie produktive Anschauung
mögen zwar uns Menschen möglicherweise verschlossen sein, aber es sei doch denkbar, dass sie
Wesen zukommen, die zwar anders als wir, aber dennoch \emph{endliche} Wesen
sind. Und so konnten \authorcite{Fichte:DieBestimmungdesMenschen1800} und
\name[Friedrich Wilhelm Joseph]{Schelling} die Möglichkeit einer
produktiven Anschauung und \name[Johann Wolfgang von]{Goethe} die
Realisierbarkeit eines synthetisch-allgemeinen Verstandes erwägen, ohne damit Ungeheuerliches zu
behaupten. Nur wenn die verschiedenen zum Vergleich konzipierten
Erkenntnisvermögen nicht auseinandergehalten werden,
ergebe sich der Anschein, die genannten Autoren müssten die Realisierbarkeit
von Fähigkeiten behaupten, die lediglich Göttern
zukommen.\footnote{\cite[Vgl.][175--180]{Foerster:DieBedeutungvonSS7677deremphKritikderUrteilskraftfuerdieEntwicklungdernachkantischenPhilosophieTeil12002}.
Eine ähnliche Differenzierung lässt sich freilich auch vornehmen, ohne auf der
strikten Trennung von intellektueller Anschauung und intuitivem Verstand zu
bestehen. Nach \authorfullcite{Gram:IntellectualIntuition1981} finden sich bei
\name[Immanuel]{Kant} drei verschiedene Konzeptionen einer intellektuellen
Anschauung, die jeweils unterschiedliche Begriffe eines korrelierten Verstandes
voraussetzen, über den wir gerade nicht verfügen.
Auch wenn er dies nicht explizit schreibt, setzt \authorcite{Gram:IntellectualIntuition1981}
offensichtlich voraus, dass intellektuelle Anschauung gerade die Erkenntnis ist, die der intuitive Verstand
hervorbringt; er trennt also nicht wie
\authorcite{Foerster:DieBedeutungvonSS7677deremphKritikderUrteilskraftfuerdieEntwicklungdernachkantischenPhilosophieTeil12002}
zwischen der intellektuellen Anschauung und dem intuitiven Verstand als
unterschiedlichen Themenfeldern, sondern sieht mit beiden Ausdrücken jeweils
dieselben Sachverhalte angesprochen. Nun beschreibe
\name[Immanuel]{Kant}
\begin{nummerierung}
  \item einen Verstand, der seinen eigenen Gegenstand
  hervorbringt -- dies  entspricht dem Konzept einer produktiven Anschauung --,
  \item einen Verstand, der Dinge an sich unabhängig von Bedingungen
  unserer Sinnlichkeit erkennt, -- was dem Konzept einer übersinnlichen
  Anschauung entspricht --, sowie
  \item  einen Verstand, der das Ganze aller \emph{Phänomena} anzuschauen
  vermag -- dies entspricht dem vom Synthetisch-Allgemeinen ausgehenden
  intuitiven Verstand, insofern es sich um einen Verstand handelt, der das Ganze
  (als Ganzes) anzuschauen
  vermag \parencite[vgl.][288]{Gram:IntellectualIntuition1981}.
\end{nummerierung}
\authorfullcite{Leech:MakingModalDistinctions2014} erwähnt außerdem die
Idealisten nach \name[Immanuel]{Kant} mit Theorien eines \singlequote{sich
selbst setzenden Ich} \parencite[vgl.][\pno~348,
353\,f.]{Leech:MakingModalDistinctions2014}. Diese Überlegungen bleiben hier
aus Gründen, die in der Einleitung ausgeführt  wurden (siehe
S.~\pageref{Einleitung:AbschnittIdealistennachKant}), unberücksichtigt.}



Das Argument, welches Autoren wie \authorcite{Foerster:Die25JahrederPhilosophie2011} vorbringen, die zwischen
den verschiedenen Beschreibungen eines nicht-endlichen Denkens lediglich einen funktionalen
Zusammenhang sehen, lautet: Mag auch der Name \enquote{intuitiver Verstand}
oder \enquote{\emph{intellectus archetypus}} gleich oder die Darstellung ähnlich
sein, so wechseln doch die Beschreibungen und Charakterisierungen. Diese seien aber logisch voneinander
unabhängig. Ob ein Verstand anschaut oder nur denkt, ob er seinen Gegenstand
selbst hervorbringt oder auf sinnlich Gegebenes angewiesen bleibt, das seien
ganz andere Fragen als die, ob er von den Teilen zum Ganzen oder
vom Ganzen zu den Teilen geht. Und letztlich gehe es \name[Immanuel]{Kant} auch
nicht darum, eine kohärente Konzeption eines nicht-endlichen Erkenntnisvermögens
zu beschreiben, sondern darum, verschiedene Besonderheiten unseres Verstandes
herauszustellen. Insofern sei zwar die \emph{Funktion} der Beschreibung eines
\singlequote{anderen} Verstandes stets dieselbe. Aber insofern an verschiedenen
Stellen seines Werks verschiedene Eigenschaften unseres Verstandes zu
beschreiben seien, variiere eben auch die Beschreibung der
Eigenschaften eines nicht-endlichen Verstandes.

Wenn das stimmen sollte, dann kann auch von einem einheitlichen Begriff
menschlicher Endlichkeit bei \name[Immanuel]{Kant} nicht gesprochen werden. Es
gäbe möglicherweise verschiedene Aspekte unserer Endlichkeit, die durch kein
systematisches Band zusammengehalten werden. Und dann wäre es auch falsch zu
fragen, was denn die \emph{grundlegende} Bestimmung unserer Endlichkeit ist. Es
wäre bei den Charekterisierungen jeweils fraglich, ob überhaupt eine Konzeption von \emph{Endlichkeit}
vorliegt. Schließlich wären zwar (möglicher- aber nicht notwendigerweise)
\emph{einige} Beschreibungen eines \singlequote{anderen} Verstandes
Beschreibungen eines unendlichen, andere aber wiederum bloß Beschreibungen eines
zwar anderen, aber nichtsdestotrotz ebenso endlichen Verstandes. Ob bei
\name[Immanuel]{Kant} überhaupt systematisch von Endlichkeit gesprochen wird,
wäre somit eine offene Frage.

Vorausgesetzt werden muss dabei freilich, dass die Bestimmungen tatsächlich
voneinander unabhängig sind, dass also beispielsweise ein Verstand, der seinen
Gegenstand selbst hervorbringt, ebenso von den Teilen zum Ganzen gehen kann, und
dass ein Verstand, der selbst anschaut, seinen Gegenstand damit nicht selbst
hervorbringt. Dass sich bei \name[Immanuel]{Kant} unterschiedliche
Beschreibungen eines \singlequote{anderen} Verstandes und einer \singlequote{anderen} Anschauung
finden lassen, ist nicht zu bestreiten. Eine Anschauung als produktiv zu
beschreiben ist etwas anderes als zu sagen, sie sei eine Anschauung von Dingen
an sich selbst. Und einen Verstand als anschauend zu bezeichnen scheint etwas
ganz anderes zu sein als zu sagen, er gehe vom Allgemeinen zum Besonderen.
Und dies alles unterscheidet sich weiterhin von der Aussage, eine Anschauung sei
nicht unseren Formen der Sinnlichkeit unterworfen. Insofern ist den genannten
Autoren dem ersten Anschein nach Recht zu geben. Aber aus der
Unterschiedlichkeit der Beschreibungen lässt sich nicht darauf schließen, dass es möglich ist, eine
Anschauung als nicht sinnlich und dennoch als nicht-produktiv zu konzipieren
oder einen Verstand, der zwar denkt und nicht anschaut, dabei aber nicht vom
Synthetisch-Allgemeinen, sondern vom Analytisch-Allgemeinen zum Besonderen
geht.\footnote{Lediglich \authorfullcite{Gram:IntellectualIntuition1981}
argumentiert ausführlich für die These, die Beschreibungen des intuitiven
Verstandes seien unterschiedlich, sondern auch wechselseitig logisch
inkompatibel \parencite[vgl.][287--296]{Gram:IntellectualIntuition1981}.}
Möglicherweise lassen sich zwischen den verschiedenen Charakterisierungen logische
Zusammenhänge entwickeln, die zeigen, dass \name[Immanuel]{Kant}s
unterschiedliche Beschreibungen eines Verstandes und einer Anschauung, die nicht
den unseren entsprechen, letztlich auf denselben Begriff eines intuitiven
Verstandes und seiner intellektuellen Anschauung hinauslaufen.\footnote{So
argumentiert auch Jessica
\textcite[vgl.][348--356]{Leech:MakingModalDistinctions2014}.}

\section{Abhängigkeit als Grundbestimmung unserer
Endlichkeit}\label{subsection:DiskursiverVerstandundsinnlicheAnschauung}
\name[Immanuel]{Kant} sagt, dass unser Verstand nur zusammen mit unserer
Sinnlichkeit Gegenstände bestimmen kann, weil er nur
\emph{denken}, nicht aber anschauen kann.\footnote{\enquote{Ein Verstand, in
welchem durch das Selbstbewußtsein zugleich alles Mannigfaltige gegeben würde, würde \ori{anschauen}; der unsere kann nur
\ori{denken} und muß in den Sinnen die Anschauung
suchen.} \mkbibparens{\cite[][\S~16]{Kant:KritikderreinenVernunft2003},
\cite[][III: 110.26--29]{Kant:GesammelteWerke1900ff.}}.
\cite[Vgl.\ außerdem][\S\S~17, 21, B~311f.]{Kant:KritikderreinenVernunft2003},
\cite[][III: 112.20--33, 116.13--18, 212.16--21]{Kant:GesammelteWerke1900ff.}.
\name[Immanuel]{Kant} vermeidet das Prädikat \enquote{endlich} und spricht eher
von \emph{unserem} Verstand, den er auch
\enquote{intellectus ectypus} im Unterschied zu einem \enquote{intellectus archetypus}
nennt, oder einer nicht-sinnlichen Anschauung und einem Verstand, der die
Objekte selbst hervorbringt.} Doch der Versuch zu
explizieren, was es heißt, dass unser Verstand nicht
anschaut, sondern denkt, erweist sich als schwieriger als zunächst angenommen.
\name[Immanuel]{Kant} spricht an entsprechenden Stellen davon, dass die
Erkenntnis unseres Verstandes nicht intuitiv, sondern diskursiv sei; doch macht
dies die Interpretation nicht leichter, denn den Begriff der Diskursivität
finden wir nirgends explizit erläutert. Hinzu kommt eben, dass dies nicht die
einzige Beschreibung der Besonderheit unseres endlichen Denkens und Erkennens
ist.

Ich möchte im folgenden zeigen, dass es eine
grundlegende Bestimmung unserer Endlichkeit gibt und dass diese in der
Abhängigkeit unseres Verstandes von der Sinnlichkeit liegt und er \emph{a
fortiori} diskursiv oder ein Vermögen der Begriffe ist. Dafür werde ich
argumentieren, dass \name[Immanuel]{Kant} über genau \emph{eine} Konzeption
eines intuitiven Verstandes als eines alternativ zu unserem konzipierten
Erkenntnisvermögen und einer intellektuellen Anschauung als der
korrespondierenden Erkenntnisart verfügt. Zunächst gilt es jedoch, die
beteiligten Begriffe genau zu differenzieren. Ohne Klarheit darüber, welche
Merkmale grundlegend für Begriffe wie \enquote{Verstand}, \enquote{Anschauung}
oder \enquote{diskursiv} sind, sind seriöse Antworten nicht zu erwarten. Daher
soll die nötige Begriffsklärung im folgenden Teil geleistet werden, wobei sich
insbesondere der Begriff der Diskursivität als klärungsbedürftig erweist; Kapitel
\ref{subsubsection:BegriffderDiskursivitaet} widmet sich entsprechend dem
Begriffspaar \enquote{diskursiv} und \enquote{intuitiv} und wird damit die
Begriffe \enquote{Begriff} und \enquote{Anschauung} zu differenzieren haben.
Erst im Anschluss an diese Klärungen ist es sinnvoll, sich über \name[Immanuel]{Kant}s Angaben zu den
Begriffen einer intellektuellen, sinnlichen oder nicht-sinnlichen Anschauung und
eines intuitiven, anschauenden oder diskursiven Verstandes zu verständigen.
Kapitel \ref{subsection:VerstandundRezeptivitaet} wird
sich dieser Aufgabe widmen, die Begriffe \enquote{Sinnlichkeit} und \enquote{Verstand} und
darin inbegriffen die Begriffe \enquote{intellektuell}, \enquote{sinnlich},
\enquote{spontan} und \enquote{rezeptiv} zu bestimmen.

\subsection[Begriffe und Anschauungen]{Begriffe und Anschauungen:
\enquote{diskursiv} und
\enquote{intuitiv}}\label{subsubsection:BegriffderDiskursivitaet}
Zunächst gilt es, die Adjektive \enquote{intuitiv} und \enquote{diskursiv} zu
analysieren; \name[Immanuel]{Kant} gibt dafür keine eigenständige Definition,
verknüpft sie aber mit den Begriffen \enquote{Anschauung} und \enquote{Begriff}.
Bevor jedoch auf genauere Zusammenhänge eingegangen werden kann, ist zu fragen,
worauf \name[Immanuel]{Kant} diese Adjektive überhaupt anwendet; denn nicht nur
den Verstand nennt er diskursiv oder intuitiv.
Es finden sich hier mindestens drei Ansätze, wovon es sinnvoll ist zu behaupten,
es sei diskursiv oder intuitiv:
\begin{nummerierung}
\item Fast ausschließlich in \S~77 der \titel{Kritik der Urteilskraft} ist es
der \emph{Verstand}, der als diskursiv oder intuitiv beschrieben
wird.\footnote{\cite[Vgl.][\S~77]{Kant:KritikderUrteilskraft2009}, \cite[][V:
407.19--21]{Kant:GesammelteWerke1900ff.}.} In der \titel{Kritik der reinen
Vernunft} werden die Adjektive \enquote{diskursiv} und \enquote{intuitiv} nicht ein
einziges Mal auf den Verstand selbst bezogen, sondern ausschließlich auf
Erkenntnisse des Verstandes.\footnote{In der \titel{Kritik der reinen Vernunft} nennt er die
\enquote{Erkenntnis eines jeden, jedenfalls des menschlichen, Verstandes}
diskursiv \mkbibparens{\cite[][B 93]{Kant:KritikderreinenVernunft2003},
\cite[][III: 85.14--15]{Kant:GesammelteWerke1900ff.}}. Eine
Erkenntnis durch diskursive Vorstellungen, also Begriffe, heißt nach
der \titel{Logik} eine \enquote{cognitio
discursiva} \mkbibparens{\cite[][\S~1]{Kant:ImmanuelKantsLogik1977}, \cite[][IX:
91.11]{Kant:GesammelteWerke1900ff.}}, wobei unausgemacht bleibt, ob es sich
hier um die Tätigkeit des Verstandes handelt oder um deren Resultate.} In den
\titel{Prolegomena} hingegen findet sich die Attribuierung der Diskursivität von
unserem Verstand in \S~57 \titel{Von der Grenzbestimmung der reinen
Vernunft}.\footnote{\cite[Siehe][A 163\,f.,
172]{Kant:ProlegomenazueinerjedenkuenftigenMetaphysikdiealsWissenschaftwirdauftretenkoennen1977},
\cite[][IV: 351.2--3, 355.7]{Kant:GesammelteWerke1900ff.}. Eine eher beiläufige
Bezeichnung des Verstandes als diskursiv findet sich außerdem in der
\titel{Kritik der praktischen Vernunft}
\mkbibparens{\cite[siehe][A 247]{Kant:KritikderpraktischenVernunft1974}, \cite[][V:
137.12]{Kant:GesammelteWerke1900ff.}}.} Wenn
\name[Immanuel]{Kant} nun die Erkenntnis des Verstandes diskursiv (oder
intuitiv) nennt, dann kann damit wegen der Vieldeutigkeit von
\enquote{Erkenntnis} immer noch dreierlei gemeint
sein: Zum einen die Tätigkeit des Verstandes und zum anderen die Erkenntnisse,
die wir mittels des Verstandes erwerben oder besitzen, also -- wiederum doppelt
-- Erkenntnisse im Sinne objektiv gültiger Urteile (Erkenntnisse im eigentlichen
Sinne) und Erkenntnisse im Sinne objektbezogener Vorstellungen (Erkenntnisse im
uneigentlichen
Sinne).\footnote{\phantomsection\label{Anmerkung:ErkenntnisInZweierleiSinn}Es
ist zu beachten, dass \name[Immanuel]{Kant} verschiedene Erkenntnisbegriffe hat;
in einem Sinne nennt er ausschließlich Urteile \enquote{Erkenntnisse}, in einem
anderen Sinne ist \enquote{Erkenntnis} der Oberbegriff zu Anschauungen und
Begriffen. \authorfullcite{Gruene:BlindeAnschauung2009} nennt die Erkenntnisse,
die sich in Urteilen äußern, deren Begriffe objektive Realität besitzen, \enquote{Erkenntnis im engen Sinn} und unterscheidet sie von
solchen im weiten Sinne, als bewusster Vorstellungen, die sich auf Gegenstände
beziehen und in Anschauungen und Begriffe unterteilt werden
\parencite[vgl.][29]{Gruene:BlindeAnschauung2009}.
\authorfullcite{Prien:KantsLogikderBegriffe2006} spricht von Erkenntnissen im
eigentlichen und im uneigentlichen Sinn. \enquote{Erkenntnis im eigentlichen
Sinne ist immer ein Urteil, denn nur in Urteilen kann man Gegenstände erkennen,
da man nur in Urteilen etwas von ihnen aussagt}
\parencite[][7]{Prien:KantsLogikderBegriffe2006}.} Die Tätigkeit des Erkennens
durch den Verstand oder dessen \singlequote{Form}\footnote{\cite[][B
283]{Kant:KritikderreinenVernunft2003}, \cite[][III:
195.37--196.1]{Kant:GesammelteWerke1900ff.}.} als diskursiv zu bezeichnen, kann
noch damit identifiziert werden, den Verstand selbst als diskursiv zu
bezeichnen; die anderen Möglichkeiten hingegen sind zweifellos gesondert zu betrachten:

\item In einem wichtigen Sinn sind objektive Vorstellungen diskursiv oder
intuitiv: Diskursivität ist in diesem Sinne eine Eigenschaft von Begriffen im
Gegensatz zu Anschauungen, die nicht diskursiv sind, sondern intuitiv. Hier bestimmt das
Adjektiv \enquote{diskursiv} \emph{Erkenntnisse im uneigentlichen
Sinne}, also Vorstellungen, die
bewusst und objektiv sind, und unterteilt sie in Begriffe (diskursive objektive
Vorstellungen, \enquote{representatio[nes]
discursiva[e]}\footnote{\cite[][\S~1]{Kant:ImmanuelKantsLogik1977}, \cite[][IX:
91.10]{Kant:GesammelteWerke1900ff.}.}) und Anschauungen (intuitive objektive
Vorstellungen).

\item\label{Aufzaehlung:Vernunftgebrauchintuitivoderdiskursiv} Schließlich
findet sich aber auch die Redeweise von diskursiven oder intuitiven
Erkenntnissen im eigentlichen Sinne, also objektiv gültigen Urteilen. In der
\titel{Kritik der reinen Vernunft} heißt es etwa: \enquote{Also ist ein
transzendentaler Satz ein synthetisches Vernunfterkenntnis nach bloßen
Begriffen, und mithin diskursiv}\footnote{\cite[][B
750]{Kant:KritikderreinenVernunft2003}, \cite[][III:
474.21--23]{Kant:GesammelteWerke1900ff.}.}. So unterscheidet Kant zumindest bei
rationalen Erkenntnissen -- also solchen \emph{a priori} oder \emph{ex principiis}\footnote{Siehe
hierzu Kapitel \ref{section:MuendigkeitundPhilosophie} dieser Arbeit.} --
zwischen intuitiven rationalen Erkenntnissen (aus der Konstruktion von Begriffen
in der reinen Anschauung) in der Mathematik und diskursiven rationalen
Erkenntnissen (aus Begriffen) in der Philosophie. Einen Beweis nennt
\name[Immanuel]{Kant} in der Methodenlehre der \titel{Kritik der reinen
Vernunft} diskursiv oder intuitiv je nach Art der Beweisführung (ob sie
mathematisch oder philosophisch ist).\footnote{\cite[][B
762\,f.,]{Kant:KritikderreinenVernunft2003} \cite[][III:
481.15--482.2]{Kant:GesammelteWerke1900ff.}.} Ganz analog dazu unterteilt die
\titel{Logik} die Grundsätze von Beweisen in diskursive und
intuitive\footnote{\cite[Vgl.][\S~35]{Kant:ImmanuelKantsLogik1977}, \cite[][IX:
110.24--28]{Kant:GesammelteWerke1900ff.}.} und ebenso die resultierende
\singlequote{Gewissheit} in intuitive Gewissheit (Evidenz in der Mathematik) und
diskursive Gewissheit (in der Philosophie)\footnote{\cite[Vgl.][A
107]{Kant:ImmanuelKantsLogik1977}, \cite[][IX:
70.34--37]{Kant:GesammelteWerke1900ff.}.}. In der Kritik der reinen Vernunft
wird in diesem Zusammenhang der \emph{Vernunftgebrauch} als diskursiv oder
intuitiv beschrieben.\footnote{\cite[Vgl.][B
747]{Kant:KritikderreinenVernunft2003}, \cite[][III:
472.25--28]{Kant:GesammelteWerke1900ff.}.}
Diese Verwendungsweise von \enquote{diskursiv} und \enquote{intuitiv} ist in der Philosophie
vor \name[Immanuel]{Kant} die gebräuchliche. So unterscheiden
\authorcite{Wolff:Discursuspraeliminarisdephilosophiaingenere1996},
\authorcite{Baumgarten:Metaphysica---Metaphysik2011} und
\authorcite{Meier:AuszugausderVernunftlehre1752} zwischen \emph{iudicia
intuitiva} und \emph{iudicia discursiva} (beziehungsweise
zwischen \emph{propositiones intuitivae} und \emph{discursivae}).\footnote{Siehe
\cite[][\S~51]{Wolff:PhilosophiarationalissiveLogica1740},
\cite[][\S~166]{Baumgarten:AcroasislogicainChristianumL.B.deWolff1983}, sowie
\cite[][\S~319]{Meier:AuszugausderVernunftlehre1752}
\parencite[][XVI: 674.24--28]{Kant:GesammelteWerke1900ff.}. Bzgl.
\authorcite{Wolff:Discursuspraeliminarisdephilosophiaingenere1996} siehe
auch
\cite{Ecole:Duroledelentendementintuitifdanslaconceptionwolffiennedelaconnaissance1986}.}
\end{nummerierung}

Daneben gibt es einige Zuschreibungen von Diskursivität, die selten
vorkommen und eindeutig als derivativ aufgefasst werden können: In der
Anthropologie in pragmatischer Hinsicht findet sich die Zuschreibung der
Adjektive \enquote{diskursiv} und \enquote{intuitiv} bezüglich des
Ausdrucks \enquote{Bewusstsein}.\footnote{\enquote{Weil Erfahrung empirisches
Erkenntnis ist, zum Erkenntnis aber (da es auf Urteilen beruht) Überlegung
(reflexio), mithin Bewußtsein, d.\,i. Tätigkeit in Zusammenstellung des
Mannigfaltigen der Vorstellung nach einer Regel der Einheit desselben, d.\,i.
\ori{Begriff} und (vom Anschauen unterschiedenes) Denken überhaupt erfordert
wird: so wird das Bewußtsein in das \ori{diskursive} (welches, als logisch, weil
es die Regel gibt, vorangehen muß), und das \ori{intuitive} Bewußtsein
eingeteilt werden} \mkbibparens{\cite[][BA
27]{Kant:AnthropologieinpragmatischerHinsicht1977}, \cite[][VII:
141.21--27]{Kant:GesammelteWerke1900ff.}}.} Dabei identifiziert
\name[Immanuel]{Kant} das diskursive Bewusstsein mit der reinen Apperzeption,
das intuitive Bewusstsein hingegen mit dem inneren Sinn. Der Ausdruck \enquote{intuitiv}
meint an dieser Stelle \enquote{empirisch} und verweist auf den inneren Sinn der
\titel{Kritik der reinen Vernunft}, während \enquote{diskursiv} hier mit
\enquote{rein} zu übersetzen ist und auf das \enquote{Ich denke} verweist, das
alle meine Vorstellungen muss begleiten können, welches aber nur die Einheit
enthält, durch die noch nichts Mannigfaltiges gegeben
ist.\footnote{Siehe \cite[][\S~16]{Kant:KritikderreinenVernunft2003},
\cite[][III: 108.16--110.35]{Kant:GesammelteWerke1900ff.}.} Ebenfalls in der
\titel{Anthropologie} spricht \name[Immanuel]{Kant} von der \enquote{diskursive[n] Vorstellungsart
durch laute Sprache oder durch Schrift}\footnote{\cite[][BA
192]{Kant:AnthropologieinpragmatischerHinsicht1977}, \cite[][VII:
244.36--245.1]{Kant:GesammelteWerke1900ff.}.} in Beredsamkeit und Dichtkunst im
Unterschied zur intuitiven Vorstellungsart in Musik und bildender Kunst.
Offensichtlich ist dies darin fundiert, dass Dicht- und Redekunst
(diskursive) Begriffe verwenden, während Musik und bildende Kunst die Anschauung
ansprechen (die -- wie hier deutlich wird -- nicht ausschließlich visuell
verstanden werden darf).

Oft wird angenommen, dass Diskursivität eine Eigenschaft unseres
\emph{Verstandes} ist und als solche expliziert zu werden
verlangt. \authorfullcite{Quarfood:DiscursivityandTranscendentalIdealism2012} sagt, Kant
verwende den Ausdruck \enquote{diskursiv}, um auf die Tatsache zu verweisen,
dass unser Erkenntnisvermögen über zwei Stämme verfügt -- Sinnlichkeit  und
Verstand --, die nur zusammen Erkenntnisse zu generieren vermögen. Der Verstand
sei diskursiv, insofern unser kognitives Vermögen seine eigenen Gegenstände
nicht durch Denken bereitzustellen vermag, sondern auf die Sinnlichkeit angewiesen
bleibt.\footnote{\enquote{Kant uses the term \enquote{discursivity} to refer to
this fundamental fact about our cognitive capacity, the fact that it doesn't
provide its own objects by thinking but must rely on sensibility's reception of
objects} \parencite[][143]{Quarfood:DiscursivityandTranscendentalIdealism2012}.}
Nach \authorcite{Allison:KantsTranscendentalIdealism2004} ist die Annahme der
Diskursivität menschlichen Erkennens (\emph{cognition}) die entscheidende
Voraussetzung der Vernunftkritik.\footnote{Kant’s \enquote{idealism is more
properly seen as epistemological or perhaps \enquote{metaepistemological} than
as metaphysical in nature, since it is grounded in an analysis of the discursive
nature of human cognition}
\parencite[][4]{Allison:KantsTranscendentalIdealism2004}.} Dass menschliches
Erkennen diskursiv ist, bedeute, dass es sowohl Begriffe als auch sinnliche
Anschauungen erfordere -- \enquote{cognition} meint hier die
\emph{Tätigkeit} des Verstandes. Und die Annahme, dass der transzendentale
Idealismus vom Verständnis menschlichen Erkennens als diskursiv abhänge, nennt
\authorcite{Allison:KantsTranscendentalIdealism2004} die
\enquote{\emph{discursivity
thesis}}.\footnote{\cite[Siehe][12]{Allison:KantsTranscendentalIdealism2004}.
Er schreibt weiter: \enquote{[W]e understand discursivity in the
\name[Immanuel]{Kant}ian sense, as requiring the joint contribution of
sensibility and understanding}
\parencite[][13]{Allison:KantsTranscendentalIdealism2004}. Dabei stellt er sich
-- soweit ich sehe -- nirgends explizit die Frage, wovon Diskursivität
eigentlich primär ausgesagt wird. Auch eine explizite Klärung dieses Begriffs
sucht man vergeblich.} Nach
\authorfullcite{Duesing:SpontanediskursiveSynthesis2004} wiederum bedeutet die
Lehre von der Diskursivität, \enquote{daß im schrittweisen Durchgehen durch
Mannigfaltiges Begriffe als Allgemeinheitsvorstellungen sowie
Begriffsverhältnisse gebildet werden, für die die Reziprozität von Umfang und
Inhalt gilt}\footnote{\cite[][103]{Duesing:SpontanediskursiveSynthesis2004}.}.

Doch in der \titel{Kritik der reinen Vernunft} wird Diskursivität
von Begriffen und Urteilen und auch von der Tätigkeit des Erkennens (durch
Begriffe), nicht aber von unserem Verstand ausgesagt. Dieser wird stattdessen
als Vermögen zu denken beschrieben, das nicht anzuschauen vermag. Lediglich
seine \emph{Tätigkeit} heißt in einem übertragenen Sinne diskursiv, insofern sie
durch Begriffe geschieht.\footnote{\enquote{Es gibt aber, außer der Anschauung,
keine andere Art, zu erkennen, als durch Begriffe. Also ist die Erkenntnis eines
jeden, wenigstens des menschlichen Verstandes, eine Erkenntnis durch Begriffe,
nicht intuitiv, sondern diskursiv}
\mkbibparens{\cite[][B 92\,f.,]{Kant:KritikderreinenVernunft2003}
\cite[][III: 85.13--16]{Kant:GesammelteWerke1900ff.}}.} Insofern der Verstand als diskursiv beschrieben wird, stellt
\name[Immanuel]{Kant} heraus, dass es sich um ein Vermögen der Begriffe
handelt.\footnote{So schreibt \name[Immanuel]{Kant} in der \titel{Kritik der
Urteilskraft}: \enquote{Unser Verstand ist ein Vermögen der Begriffe,
\myemph{d.\,i.} ein diskursiver Verstand}
\mkbibparens{\cite[][\S~77]{Kant:KritikderUrteilskraft2009}, \cite[][V:
406.16--17]{Kant:GesammelteWerke1900ff.}}} Und auch bei diskursiven
Erkenntnissen im engeren Sinne oder dem diskursiven Vernunftgebrauch besteht die Grundlage der
Zuschreibung von Diskursivität darin, dass ein Urteil auf der Grundlage
\emph{von Begriffen} gefällt wird. So heißen die
rationalen Erkenntnisse diskursiv, wenn es sich um rationale Erkenntnisse
aus Begriffen (statt aus der Konstruktion von Begriffe in reiner
Anschauung) handelt.\footnote{\enquote{Also ist ein
transzendentaler Satz ein synthetisches Vernunfterkenntnis nach bloßen
Begriffen, und mithin diskursiv} \mkbibparens{\cite[][B
750]{Kant:KritikderreinenVernunft2003}, \cite[][III:
474.21--23]{Kant:GesammelteWerke1900ff.}}. Siehe auch \cite[][B
747]{Kant:KritikderreinenVernunft2003}, \cite[][III:
472.25--28]{Kant:GesammelteWerke1900ff.}. Zu den unterschiedlichen Arten von
Erkenntnissen siehe Kapitel \ref{section:MuendigkeitundPhilosophie} dieser
Arbeit.} Es sind also zunächst Begriffe, von denen
wir Diskursivität im Unterschied zur Intuitivität von Anschauungen aussagen.

Eine Explikation der Unterscheidung von Anschauungen und Begriffen, die sich für
die Frage nach dem Begriff der Diskursivität als Ausgangspunkt anbietet, findet
sich in der {\jaeschelogik}. Dort heißt es:
\begin{quote}
Alle Erkenntnisse, das heißt: alle mit Bewußtsein auf ein Objekt bezogene
Vorstellungen sind entweder \ori{Anschauungen} oder \ori{Begriffe}. -- Die
Anschauung ist eine \ori{einzelne} Vorstellung (repraesentat. singularis), der
Begriff eine \ori{allgemeine} (repraesentat. per notas communes) oder
\ori{reflektierte} Vorstellung (repraesentat. discursiva).\\ Die Erkenntnis
durch Begriffe heißt das \ori{Denken} (cognitio
discursiva).\footnote{\cite[][\S~1]{Kant:ImmanuelKantsLogik1977}, \cite[][IX:
91.6--11]{Kant:GesammelteWerke1900ff.}. \name[Gottlob Benjamin]{Jäsche} hat
folgende Vorlagen aus \name[Immanuel]{Kant}s Handexemplar des Logikkompendiums
\authorcite{Meier:AuszugausderVernunftlehre1752}s nutzen können:
\enquote{\ori{cognitio est vel intuitus vel conceptus} (\ori{repraesentatio
discursiva}), Beym ersteren bin ich leidend (receptivitaet), beym zweyten
handelnd (spontaneitaet). \ori{intuitus} ist einzeln, \ori{conceptus} ist
\ori{repraesentatio per notam communem}}
\mkbibparens{\cite[][\nopp 2836]{Kant:Reflexionen1900ff.}, \cite[][XVI:
538.22--25]{Kant:GesammelteWerke1900ff.}}. \enquote{Denken ist
\ori{repraesentare per conceptus}: \ori{cognitio discursiva}}
\mkbibparens{\cite[][\nopp 2841]{Kant:Reflexionen1900ff.},
\cite[][XVI: 541.5]{Kant:GesammelteWerke1900ff.}}.}
\end{quote}
Die Bestimmung der {\jaeschelogik} ist somit sehr einfach:
Erkenntnisse unterteilen sich in Anschauungen und Begriffe;
Anschauungen sind einzelne, Begriffe allgemeine
Vorstellungen. Lediglich die lateinischen Verweise signalisieren
weitere Zusammenhänge: Begriffe sind reflektierte Vorstellungen oder
\emph{repraesentationes discursivae} und das Denken heißt als Erkennen durch Begriffe
\singlequote{diskursiv} (\emph{cognitio discursiva}). Danach scheint
\enquote{diskursiv} ein Synonym zu \enquote{reflektiert} zu sein und den Begriff
gegenüber der Anschauung durch die Eigenschaft auszuzeichnen, dass der Begriff
durch allgemeine Merkmale (\emph{per notas communes}) repräsentiert.
Hierin liegt -- wie ich zeigen werde -- die korrekte Deutung der Diskursivität:
Ein Begriff ist diskursiv, insofern er sich als reflektierte Vorstellung durch
allgemeine Merkmale auf Gegenstände bezieht.

In der \titel{Kritik der reinen Vernunft} ist die Unterscheidung zwischen
Anschauung und Begriff Teil der \singlequote{\emph{Stufenleiter}}, die freilich
primär der Frage nach der Bedeutung von \enquote{Idee} nachgeht und die
Ausdrücke \enquote{Anschauung} und \enquote{Begriff} nur in Vorbereitung mit
expliziert.\footnote{Auf diesen Punkt machte mich
\authorfullcite{Heidemann:AnschauungundBegriff2002} aufmerksam.} Dennoch handelt
es sich um die beste Grundlage, die sich in \name[Immanuel]{Kant}s selbst
publizierten Schriften findet. Sie lautet ähnlich wie die Unterscheidung in der
{\jaeschelogik}, verzichtet aber leider auf die Ausdrücke \enquote{diskursiv} und
\enquote{reflektiert}. Dafür geht sie auf die Rolle der allgemeinen Merkmale
(\emph{notae communes}) stärker ein:
\begin{quote}
Die Gattung ist \ori{Vorstellung} überhaupt (repraesentatio). Unter ihr steht
die Vorstellung mit Bewußtsein (perceptio). Eine \ori{Perzeption}, die sich
lediglich auf das Subjekt, als die Modifikation seines Zustandes bezieht, ist
\ori{Empfindung} (sensatio), eine objektive Perzeption ist \ori{Erkenntnis}
(cognitio). Diese ist entweder \ori{Anschauung} oder \ori{Begriff} (intuitus
vel conceptus). Jene bezieht sich unmittelbar auf den Gegenstand und ist
einzeln; dieser mittelbar, vermittelst eines Merkmals, was mehreren Dingen
gemein sein kann.\footnote{\cite[][B
376\,f.,]{Kant:KritikderreinenVernunft2003} \cite[][III:
249.37--250.7]{Kant:GesammelteWerke1900ff.}. Ich sehe hier ab von einer
Diskussion der Frage, ob die gesamte Einteilung schlüssig ist.
Insbesondere die Einteilung der Perzeptionen in Empfindungen und Erkenntnisse
erregt Kritik, insofern es sich doch bei der Empfindung als Wahrnehmung des
inneren Sinnes ebenfalls um eine Erkenntnis zu handeln scheint
\parencite[vgl.][79--81]{Heidemann:AnschauungundBegriff2002}.}
\end{quote}
Die Definitionen der Begriffe \enquote{Anschauung} und \enquote{Begriff} lauten
nach diesem Zitat: (a) Eine Anschauung ist eine Erkenntnis (im
\singlequote{uneigentlichen} Sinne), die (a\textsubscript{i}) \emph{einzeln} ist
und sich (a\textsubscript{ii}) \emph{unmittelbar auf einen Gegenstand bezieht}.
(b) Ein Begriff ist eine \singlequote{uneigentliche} Erkenntnis, die
(b\textsubscript{i}) \emph{allgemein} ist und sich (b\textsubscript{ii})
\emph{über allgemeine Merkmale vermittelt} auf Gegenstände bezieht. An dieser
Definition lassen sich zunächst \emph{genus} und \emph{differentia specifica}
herausstellen.

Das \emph{genus} zu Anschauungen und Begriffen lautet \enquote{Erkenntnis}: Eine
Erkenntnis ist eine bewusste Vorstellung, die sich auf einen oder mehrere Gegenstände bezieht; sie ist
objektiv, insofern sie nicht bloß eine Modifikation des Zustandes eines Subjekts
beschreibt, sondern geeignet ist, von Gegenständen etwas auszusagen.
Anschauungen und Begriffe, nicht aber Urteile sind die beiden Arten von
Erkenntnissen in diesem Sinne, so dass der Anblick von Peter, der gerade zur Tür
hereinkommt, ebenso wie die Begriffe \enquote{Pferd} oder \enquote{Säugetier}
Beispiele für Erkenntnisse in diesem Sinne sind.

Schwieriger gestaltet sich die Erläuterung der \emph{differentia specifica},
denn hier werden gleich zwei Unterschiede genannt. Das Merkmal von Anschauungen, sich unmittelbar
auf Gegenstände zu beziehen (a\textsubscript{ii}), ist von
\authorfullcite{Parsons:KantsPhilosophyofArithmetic1992} als \enquote{\emph{immediacy condition}} bezeichnet worden, die
Bedingung, sich auf einzelne Gegenstände zu beziehen (a\textsubscript{i}), als
\enquote{\emph{singularity condition}}.\footnote{\cite[Vgl.][\pno~43\,f.]{Parsons:KantsPhilosophyofArithmetic1992}.}
Analog ließen sich die Merkmale des Begriffs \enquote{Begriff} als
\enquote{\emph{universality condition}} (b\textsubscript{i}) und
\enquote{\emph{mediacy condition}} (b\textsubscript{ii}) bezeichnen.
\name[Immanuel]{Kant} erwähnt also gleich zwei Charakteristika von
Anschauungen gegenüber Begriffen (und \emph{vice versa}).\footnote{\authorfullcite{Hanna:KantandtheFoundationsofAnalyticPhilosophy2001}
zählt insgesamt sogar fünf verschiedene Merkmale auf, die \name[Immanuel]{Kant}
an verschiedenen Stellen zur Bestimmung des Begriffs \enquote{Anschauung}
anführe. Neben den Merkmalen der Unmittelbarkeit und der Singularität verweise
\name[Immanuel]{Kant} darauf, dass Anschauungen stets sinnlich seien, dass sie
vor allem Denken gegeben werden können, und dass sie abhängig sind von der
Gegenwart des Gegenstandes
\parencite[Vgl.][195]{Hanna:KantandtheFoundationsofAnalyticPhilosophy2001}.
Damit nennt er freilich auch Eigenschaften, die ausschließlich auf
\emph{sinnliche} Anschauungen zutreffen, und nicht nur solche, die den Begriff
\enquote{Anschauung} zu explizieren helfen.}

Die Frage, wie sich die verschiedenen Bestimmungen der Begriffe
\enquote{Anschauung} und \enquote{Begriff} zueinander verhalten, wird in der
\name[Immanuel]{Kant}forschung breit diskutiert.\footnote{Siehe etwa
\cite{Hintikka:OnKantsNotionofIntuition1969},
\cite{Hintikka:KantianIntuitions1972},
\cite{Thompson:SingularTermsandIntuitioninKantsEpistemology1972},
\cite{Howell:IntuitionSynthesisandIndividuationintheCritiqueofPureReason1973},
\cite[][194--211]{Hanna:KantandtheFoundationsofAnalyticPhilosophy2001}, sowie
zuletzt \cite[][35--53]{Gruene:BlindeAnschauung2009}.} Da die diesbezügliche
Forschung sich vor allem auf \name[Immanuel]{Kant}s Philosophie der Mathematik
und die Frage, wie Erkenntnis durch reine Anschauung möglich ist, bezieht, steht
dabei der Begriff der Anschauung mit seinen beiden Merkmalen im Zentrum der
Aufmerksamkeit.\footnote{Eine Ausnahme bezüglich der Fokussierung der
Mathematik stellt \textcite[vgl.][35--53]{Gruene:BlindeAnschauung2009} dar, die
jedoch ohnehin den Begriff der Anschauung selbst thematisiert und den des
Begriffs daher nur nebenbei mit thematisiert. Des weiteren bespricht
\authorfullcite{Thompson:SingularTermsandIntuitioninKantsEpistemology1972} den
Begriff der Anschauung mit Blick auf \emph{empirische} Anschauungen
\parencite[vgl.][314]{Thompson:SingularTermsandIntuitioninKantsEpistemology1972}.}
Ist eine Anschauung einzeln, weil sie sich unmittelbar auf den Gegenstand
bezieht? Oder sind es zwei verschiedene Bestimmungen, so dass es einzelne
Vorstellungen gibt, die sich mittelbar auf den Gegenstand beziehen und \emph{a
fortiori} keine Anschauungen sind? Analog lässt sich freilich fragen: Ist der
Begriff eine allgemeine Vorstellung, weil er sich vermittelt über allgemeine
Merkmale auf Gegenstände bezieht? Oder bezieht er sich über allgemeine Merkmale
auf Gegenstände, weil er eine allgemeine Vorstellung ist? Kann eine allgemeine
Vorstellung sich auch unmittelbar auf Gegenstände beziehen und somit kein Begriff sein? Können sich
Erkenntnisse vermittelt über Merkmale auf Gegenstände beziehen, ohne allgemein
zu sein?\footnote{Weiter ließe sich fragen, welchen Status die Formulierung hat, dass sich
Begriffe vermittelt über \emph{allgemeine} Merkmale auf Gegenstände beziehen? Beziehen sich nur
Begriffe durch Merkmale auf Gegenstände? Oder haben nur Begriffe Merkmale, die
\enquote{mehreren Dingen gemein} sind, Anschauungen aber andere Merkmale?
\authorfullcite{Gruene:BlindeAnschauung2009} legt dar, dass auch Anschauungen
Merkmale haben, durch die sie sich auf Gegenstände beziehen. Wenn Ingrid Max am
Gang erkennt, dann ist Max' Gang ein Merkmal, das als Erkenntnisgrund fungiert.
Aber nur Begriffe verfügen ausschließlich über Merkmale, die mehreren Dingen
zukommen, während Anschauung sich durch Merkmale auf Gegenstände beziehen, die
diesen exklusiv zukommen. Max' Gang ist eben deshalb ein Merkmal von Max, weil
er selbst singulär ist. Es ist kein Begriff -- etwa des Gehens -- anhand dessen
Ingrid Max erkennt. \name[Immanuel]{Kant} spreche hier von \emph{intuitiven}
Merkmalen als Inhalten von Anschauungen, während Begriffe \emph{diskursive}
Merkmale enthielten \parencite[vgl.][65--71]{Gruene:BlindeAnschauung2009}. Dabei
fällt jedoch auf, dass eine solche Unterscheidung nicht in Schriften vorkommt,
die \name[Immanuel]{Kant} zur Publikation freigegeben hat
\parencite[vgl.][62]{Prien:KantsLogikderBegriffe2006}. Da es mir hier nicht
primär um den Begriff der Anschauung, sondern um den des Begriffs -- resp. der
Diskursivität -- geht, kann ich auf eine abschließende Beantwortung solcher
Fragen an dieser Stelle verzichten.}


Dem ersten Anschein nach handelt es sich nun bei der \emph{immediacy condition}
um eine von der \emph{singularity condition} unabhängige Bedingung. Das wäre
natürlich problematisch, denn dann definierte \name[Immanuel]{Kant} den
Unterschied zwischen Begriffen und Anschauungen durch zwei voneinander
unabhängige \emph{definientia}. Nun gilt als unstrittig, dass aus der
Unmittelbarkeit einer Vorstellung folgt, dass sie sich auf Einzelnes bezieht;
denn zumindest die menschliche Anschauung ist stets eine Anschauung einzelner
Dinge, wir schauen keine \emph{universalia}
an.\footnote{Siehe etwa \cite[][45]{Parsons:KantsPhilosophyofArithmetic1992}.
Dies konstatiert auch
\authorcite{Wolff:Discursuspraeliminarisdephilosophiaingenere1996} (siehe dazu
das Zitat in Kapitel \ref{Zitat:Wolff:ErfahrungnurvoneinzelnenDingen} auf
S.~\pageref{Zitat:Wolff:ErfahrungnurvoneinzelnenDingen} dieser Arbeit).} Also
bleiben nur zwei Optionen: Entweder beide Bestimmungen sind identisch -- die \emph{singularity
condition} ist genau dann erfüllt, wenn die \emph{immediacy condition} erfüllt ist --, oder
die \emph{immediacy condition} stellt eine Spezifizierung der \emph{singularity
condition} dar, weil \name[Immanuel]{Kant} denkt, dass auch Begriffe sich
mitunter auf einzelne Gegenstände beziehen können, jedoch nicht unmittelbar. Die
erste Position wird von
\authorfullcite{Hintikka:KantontheMathematicalMethod1992}\footnote{Die
einschlägigen Publikationen sind: \cite{Hintikka:OnKantsNotionofIntuition1969},
\cite{Hintikka:KantianIntuitions1972}, sowie \cite{Hintikka:KantontheMathematicalMethod1992}.} und
\authorfullcite{Thompson:SingularTermsandIntuitioninKantsEpistemology1972}\footnote{Siehe
\cite{Thompson:SingularTermsandIntuitioninKantsEpistemology1972}.} vertreten,
die zweite von
\authorfullcite{Parsons:KantsPhilosophyofArithmetic1992}\footnote{Vgl. 
\cite{Parsons:KantsPhilosophyofArithmetic1992}. Siehe hierzu insgesamt
\cite[Vgl.][237]{VanCleve:ProblemsfromKant1999}.}.

Das Problem der Lesart, wonach \emph{singularity condition} und \emph{immediacy
condition} auf dasselbe hinauslaufen, liegt darin, dass damit jede singuläre
Repräsentation von Individuen als Anschauung zählte. Die Bezeichnung
\enquote{höchster Berg der Alpen} bezieht sich eindeutig auf einen einzelnen
Gegenstand (den Mont Blanc), aber ebenso eindeutig handelt es sich nicht um eine
Anschauung, denn definite Kennzeichnungen sind keine Anschauungen. Dies wiederum
wird auch von den meisten Interpreten so gesehen, die die These von der
Identität beider Bestimmungen  zurückweisen.\footnote{Siehe etwa
\cite[][\pno~207\,f.,]{Howell:IntuitionSynthesisandIndividuationintheCritiqueofPureReason1973}
sowie \cite[][70]{Parsons:KantsPhilosophyofArithmetic1992}.} Lediglich
\authorfullcite{Hintikka:KantianIntuitions1972} akzeptiert die
Schlussfolgerung, dass nach \name[Immanuel]{Kant} jede singuläre Bezugnahme als Anschauung zähle.
Aber das ist ein hoher Preis, der außerdem mit \name[Immanuel]{Kant}s Angabe
konfligiert, Begriffe könnten auch einzeln \singlequote{gebraucht}
werden.\footnote{\enquote{Es ist eine bloße Tautologie, von
allgemeinen oder gemeinsamen Begriffen zu reden; -- ein Fehler, der sich auf
eine unrichtige Einteilung der Begriffe in \ori{allgemeine}, \ori{besonderen}
und \ori{einzelne} gründet. Nicht die Begriffe selbst -- nur \ori{ihr Gebrauch}
kann so eingeteilt werden}
\mkbibparens{\cite[][\S~1]{Kant:ImmanuelKantsLogik1977}, \cite[][IX:
91.17--20]{Kant:GesammelteWerke1900ff.}}; siehe dazu auch
\cite[][B 96\,f.,]{Kant:KritikderreinenVernunft2003}
\cite[][III: 87.20--88.2]{Kant:GesammelteWerke1900ff.}.} Für den
einzelnen Gebrauch von (allgemeinen) Begriffen scheint gerade die
Bezugnahme auf einzelne Gegenstände in (gelingenden\footnote{Eine definite Kennzeichnung
  \singlequote{gelingt}, wenn die Beschreibung tatsächlich auf genau
  einen Gegenstand zutrifft und nicht wie \enquote{der gegenwärtige
    König von Frankreich} gar keinen Gegenstand herausgreift oder wie
  \enquote{Donald Ducks Neffe} auf zu viele Gegenstände zutrifft.})
definiten Kennzeichnungen paradigmatisch zu sein.

Die zweite Lesart, wonach die \emph{immediacy condition} als Spezifizierung der
\emph{singularity condition} aufzufassen ist, führt zu einer anderen
Überlegung: Wir müssten dann sagen, dass nur die Konjunktion aus beiden
Bedingungen den Begriff der Anschauung definiert. Aber wie verhält es sich dann
mit dem Begriff des Begriffs und seinen beiden Merkmalen? \emph{Prima facie}
müssten \emph{universality condition} und \emph{mediacy condition} ihn
disjunktiv definieren. Aber wenn aus der Unmittelbarkeit einer Bezugnahme folgen
soll, dass sie auf einen einzelnen Gegenstand geht, aus dem Erfülltsein der
\emph{immediacy condition} also folgt, dass auch die \emph{singularity
condition} erfüllt ist, dann gilt umgekehrt, dass aus der \emph{universality condition} auf
die \emph{mediacy condition} geschlossen werden kann.  Es reicht dann aber
völlig aus, Begriffe mittels der
\emph{mediacy condition} und Anschauung mittels der \emph{immediacy condition}
zu bestimmen. \emph{Singularity condition} und \emph{universality condition}
wären verzichtbar und nicht als Bestandteil der Definition, sondern als Folge
derselben aufzufassen. Dass Anschauungen singuläre Vorstellungen sind, Begriffe hingegen
allgemeines repräsentieren, folgt aus dieser Explikation (wenn auch nur mit
Einschränkungen).

Dass es die Bedingungen der Unmittelbarkeit \emph{respective} der Mittelbarkeit
sind, die objektive Vorstellungen zu Anschauungen \emph{respective} Begriffen
machen, bestätigt auch die Vermutung, dass der Ausdruck \enquote{diskursiv} die
Eigenschaft von Begriffen beschreibt, sich mittels allgemeiner Merkmale auf
Gegenstände zu beziehen. Ein Textbeleg findet sich in der Angabe, dass es die
Erkenntnis durch Merkmale ist, die unser Erkennen zu einem diskursiven mache:
\begin{quote}
Das menschliche Erkenntnis ist von Seiten des Verstandes \ori{diskursiv}; d.\,h.
es geschieht durch Vorstellungen, die das, was mehreren Dingen gemein ist, zum
Erkenntnisgrunde machen, mithin durch \ori{Merkmale}, als
solche.\footnote{\cite[][A 84\,f.,]{Kant:ImmanuelKantsLogik1977}
\cite[][IX: 58.9--12]{Kant:GesammelteWerke1900ff.}. Dabei handelt es sich um
die wörtliche Übernahme einer Anmerkung \name[Immanuel]{Kant}s im
\titel{Auszug aus der Vernunftlehre}
\mkbibparens{\cite[siehe][\nopp 2288]{Kant:Reflexionen1900ff.},
\cite[][XVI: 300.8--10]{Kant:GesammelteWerke1900ff.}}. Siehe auch
\cite[][\nopp 2281]{Kant:Reflexionen1900ff.}, \cite[][XVI:
298.7--10]{Kant:GesammelteWerke1900ff.}:
\enquote{Wir erkennen Dinge nur durch Merkmale; das heißt eben erkennen, welches
von kennen herkommt. Denn der Verstand ist ein Vermögen zu denken, d.\,i.
discursiv durch Begriffe zu erkennen; Begriffe aber sind Merkmale von
allgemeinem Gebrauche.}}
\end{quote}
Der Begriff der Intuitivität verweist entsprechend auf die Unmittelbarkeit, der
der Diskursivität auf die Mittelbarkeit, die sich auf allgemeine Merkmale
stützt. Dass unser Verstand diskursiv und nicht intuitiv ist (dass er ein
Vermögen zu denken oder der Begriffe, nicht aber ein Vermögen anzuschauen ist),
heißt \emph{a fortiori}: Unser Verstand kann sich nicht unmittelbar, sondern nur
vermittelt auf Gegenstände beziehen. Die Vermittlung geschieht dabei über
allgemeine Merkmale. Damit sich der Verstand jedoch überhaupt auf Gegenstände
beziehen kann, muss es Vorstellungen geben, die sich unmittelbar auf Gegenstände
beziehen. Sonst gerieten wir in einen Regress, der uns nur zu immer weiteren
Vorstellungen, niemals aber zu Gegenständen führte. Daher kann es kein Denken
ohne Anschauungen geben. Wir hätten nur Begriffe ohne Inhalt, das hieße ohne
Gegenstandsbezug.\footnote{Eine Schwierigkeit der Darstellung
  \name[Immanuel]{Kant}s besteht freilich darin, dass sich Begriffe
  zunächst auf \emph{allgemeine} Merkmale und damit wiederum auf
  andere Begriffe beziehen, die wiederum derselben Bedingung
  unterliegen. Somit ließe sich der infinite Regress allem Anschein
  nach nicht vermeiden. \name[Immanuel]{Kant} müsste m.\,E. einräumen,
  dass sich Begriffe auch vermittelt über Anschauungen auf Gegenstände
  beziehen können. Siehe hierzu auch weiter unten
  Anm. \ref{Fussnote:InfiniterRegressBegriffeueberMerkmale} auf Seite
  \pageref{Fussnote:InfiniterRegressBegriffeueberMerkmale}.}

Statt der \emph{singularity} und \emph{universality
condition} müssen wir also die Mittelbarkeit von Begriffen und Unmittelbarkeit
von Anschauungen verstehen, um Einsicht in das Wesen von Anschauungen und
Begriffen zu erhalten.\footnote{Die \emph{singularity condition} ist in der Tat
wenig problematisch. Eine Anschauung ist stets die Anschauung von diesem oder jenem
Gegenstand, eine Anschauung von Max bezieht sich eben auf Max und auf niemanden
sonst, während Begriffe wie \enquote{Junggeselle} sich nicht auf diesen oder
jenen Junggesellen -- etwa auf Max -- beziehen, sondern allgemein auf
Junggesellen. Das scheinbare Problem, dass eine Anschauung ja auch eine
Anschauung von mehrere Gegenständen sein kann (z.\,B. eine Anschauung von Ingrid und Max) lässt sich
einfach dadurch lösen, dass wir sagen, sie beziehe sich in solchen Fällen auf
die mereologische Summe dieser Gegenstände
\parencite[vgl.][47]{Gruene:BlindeAnschauung2009}. Schwieriger als
\emph{singularity condition} und \emph{universality condition} sind die
\emph{immediacy condition} und die \emph{mediacy condition} zu verstehen, da
hier notwendiger Weise zu klären ist, \emph{durch was} Begriffe als solche
vermittelt sind.}
Was also besagen \emph{mediacy} und \emph{immediacy condition}?
\authorfullcite{Hanna:KantandtheFoundationsofAnalyticPhilosophy2001} sagt,
Anschauungen seien nach \name[Immanuel]{Kant} di\-rekt-re\-feren\-tiell, während
sich Begriffe vermittelt über \emph{Beschreibungen} auf ihre Gegenstände
beziehen.\footnote{\enquote{So the Kantian distinction between conceptual
(mediate) reference and intuitive (immediate) reference is most accurately
construed as the difference between, on the one hand, indirect or
description-determined reference to an object, and, on the other, direct or
non-description-determined reference to an object. More plainly put, intuitional
reference is \ori{direct reference}}
\parencite[][197]{Hanna:KantandtheFoundationsofAnalyticPhilosophy2001}.}
Wenn wir von dem höchsten Berg der Alpen sprechen, dann
bezieht sich dieser Ausdruck mittels einer Beschreibung auf den Mont Blanc. Die
Bezugnahme ist vermittelt über die Merkmale, dass es sich um einen Berg handelt,
der zu den Alpen gehört und höher ist, als alle anderen Berge der Alpen. Eine
Anschauung des Mont Blanc, die wir haben, während wir direkt davor stehen,
bezieht sich hingegen ohne solche Merkmale direkt auf den Berg. Es ist dann
naheliegend, den Begriff der Anschauung über die bei endlichen Wesen stets involvierte
Sinnlichkeit bestimmen zu wollen. Ein solches Vorgehen findet sich bei
\authorfullcite{Willaschek:DertranszendentaleIdealismusunddieIdealitaetvonRaumundZeit1997}:
\enquote{Anschauungen beziehen sich nicht durch Merkmale auf ihren Gegenstand,
sondern kausal und insofern
unmittelbar.}\footnote{\cite[][548]{Willaschek:DertranszendentaleIdealismusunddieIdealitaetvonRaumundZeit1997}.
\enquote{Anschauungen, so kann man zusammenfassen, sind Vorstellungen, die auf eine
Affektion der Sinnlichkeit durch einen Gegenstand zurückgehen, die über einen
qualitativen Gehalt verfügen und die sich unmittelbar auf einen einzelnen
Gegenstand beziehen}
\parencite[][\pno~545\,f.]{Willaschek:DertranszendentaleIdealismusunddieIdealitaetvonRaumundZeit1997}.}

Nun fällt auf, dass \name[Immanuel]{Kant} in der Stufenleiter der \titel{Kritik
der reinen Vernunft} wie auch in der von \name[Gottlob Benjamin]{Jäsche} übernommenen
Passage darauf verzichtet, die Sinnlichkeit für die Definition
heranzuziehen.\footnote{Dies fiel schon \authorcite{Frege:DieGrundlagenderArithmetik1988} auf, der jedoch
anzunehmen scheint, es fänden sich in {\jaeschelogik} und
\titel{Kritik der reinen Vernunft} unterschiedliche Anschauungsbegriffe
\parencite[vgl.][27]{Frege:DieGrundlagenderArithmetik1988}.}
Dies ist an sich schon Grund genug, einer definitorischen Verbindung von
Anschauungen und Sinnlichkeit skeptisch gegenüber zu stehen. Aber es gibt
weitere Gründe, die eine solche Verbindung explizit ausschließen: Wären
Anschauungen \emph{per definitionem} sinnlich, dann könnte es weder reine
Anschauungen noch intellektuelle Anschauungen geben -- sie wäre in sich
widersprüchlich und schon begrifflich
ausgeschlossen.\footnote{\authorfullcite{Willaschek:DertranszendentaleIdealismusunddieIdealitaetvonRaumundZeit1997}
beschränkt seine Überlegungen bewusst auf das endliche Denken
\parencite[vgl.][\pno~547,
Anm.]{Willaschek:DertranszendentaleIdealismusunddieIdealitaetvonRaumundZeit1997},
mir scheint damit aber auch die begriffliche Durchdringung des endlichen Denkens
behindert zu werden, indem das, was zum Begriff der Anschauung gehört, und das,
was die weitere Bestimmung derselben als sinnlicher ausmacht, nicht eigens
herausgearbeitet wird.}

Das Wort \enquote{Anschauung} suggeriert eine Verbindung mit dem Begriff der
Sinnlichkeit, insofern \enquote{schauen} die sinnliche Wahrnehmung mittels des
Gesichtssinns
bezeichnet.\footnote{\cite[Vgl.][38]{Hintikka:OnKantsNotionofIntuition1969}.}
Aber zumindest in \name[Immanuel]{Kant}s philosophischer Terminologie ist die
Verbindung nicht so eindeutig.
\enquote{Anschauung} und \enquote{intuitiv} haben stattdessen die
Unmittelbarkeit zu ihrem zentralen Merkmal, was an die noch heute erhaltene
Bedeutung von \enquote{intuitiv} verweist, die sich etwa in der
Gegenüberstellung von Intuition und Demonstration bei
\authorcite{Descartes:OeuvresdeDescartes1983} und anderen Autoren der Neuzeit
zeigt.\footnote{Auch \authorfullcite{Hintikka:OnKantsNotionofIntuition1969}
bemerkt, dass im relevanten Zeitraum der Begriff der Anschauung
(\enquote{\emph{intuitus}}) nicht zwingend mit Sinnlichkeit, sondern eher mit
Unmittelbarkeit verbunden wurde. Die Verbindung zum Begriff der Sinnlichkeit sei
über das Merkmal der Individualität zustande gekommen
\parencite[vgl.][40--44]{Hintikka:OnKantsNotionofIntuition1969}.}
Bei \emph{uns} beziehen sich Anschauungen auf ihre Gegenstände mittels einer
kausalen Verknüpfung. Wir werden von Gegenständen nur dann affiziert, wenn diese
in bestimmter Hinsicht kausal auf uns einwirken. Es ist naheliegend davon
auszugehen, dass jeder Form des Affiziertwerdens eine solche Kausalrelation
zugrunde liegt. Aber nicht jeder denkbaren Anschauung liegt eine solche
Relation zugrunde; sonst wäre eine intellektuelle Anschauung nicht
denkbar.\footnote{Denkbar wäre es natürlich, einer intellektuellen Anschauung
läge ebenfalls eine Kausalrelation zugrunde, nur eben in die andere Richtung. Eine Anschauung
bezöge sich dann auf Gegenstände, weil sie die Gegenstände hervorbringt; sie
hätte einen unmittelbaren Gegenstandsbezug als produktive Anschauung.
Dies passt insbesondere auch zu den Überlegungen in \S~14 der \titel{Kritik der
reinen Vernunft} \mkbibparens{\cite[vgl.][B
124\,f.,]{Kant:KritikderreinenVernunft2003} \cite[][III:
104.6--17]{Kant:GesammelteWerke1900ff.}} sowie mit explizitem Bezug auf eine
vorgestellte göttliche Erkenntnis im Brief an \name[Marcus]{Herz} vom 21.
Februar 1772 \mkbibparens{\cite[vgl.][X:
130.6--21]{Kant:GesammelteWerke1900ff.}}.}

Es muss keine Gemeinsamkeit aller denkbaren unmittelbaren Bezugnahmen
geben. Die einzige garantierte Gemeinsamkeit von Anschauungen besteht darin,
dass ihr Bezug auf Gegenstände nicht über allgemeine Merkmale vermittelt ist. Hilfreicher als der Versuch, die
Unmittelbarkeit von Anschauungen über kausale Verbindungen zu erläutern, ist
daher die Untersuchung der Mittelbarkeit des Gegenstandsbezugs von Begriffen. Die Unmittelbarkeit oder Mittelbarkeit von etwas auszusagen,
bedarf immer der Qualifizierung, denn nichts ist \emph{schlechthin} unmittelbar
oder vermittelt, sondern immer nur in dieser oder jener Hinsicht.\footnote{Siehe
dazu \cite{Sellars:EmpiricismandthePhilosophyofMind1997}.} Anschauungen beziehen sich
beispielsweise vermittelst der Sinne auf Gegenstände. Hier aber geht es darum,
ob eine Vorstellung sich nur \enquote{vermittelst eines Merkmals, was mehreren
Dingen gemein sein kann,} auf Gegenstände bezieht.
\authorfullcite{Allison:KantsTranscendentalIdealism2004} sagt unter Rekurs auf
eine Beobachtung von \authorfullcite{Longuenesse:KantandtheCapacitytoJudge1998}, dass Begriffe als diskursive
Regeln gelten, weil sie begriffliche Zusammenhänge
behaupten.\footnote{\cite[Vgl.][79]{Allison:KantsTranscendentalIdealism2004}:
\enquote{On the other hand, concepts also serve as discursive rules affirming
conceptual connections.} \enquote{[I]t is the function of concepts as discursive rules
that accounts for their role in judgment. To form the concept of body as a
discursive representation is to think together the features of extension,
impenetrability, figure, and so forth, as marks or components of the concept
that are in some sense \punkt{} necessarily connected to it}
\parencite[][79]{Allison:KantsTranscendentalIdealism2004}.}
Nach \authorfullcite{Longuenesse:KantandtheCapacitytoJudge1998} gibt
es zwei Arten, \name[Immanuel]{Kant}s Redeweise von Begriffen als
Regeln zu verstehen: Einerseits verstehe \name[Immanuel]{Kant}
darunter Regeln der Synthesis des Mannigfaltigen in der Anschauung,
andererseits aber auch \emph{diskursive Regeln}, die die Merkmale des
Begriffs angeben und sagen, unter welchen Begriffen ein Gegenstand des weiteren
fällt.\footnote{\cite[Vgl.][48--50]{Longuenesse:KantandtheCapacitytoJudge1998}.
\enquote{The concept is a rule insofar as it is the consciousness of
  the unity of an act of sensible synthesis or the consciousness of
  the procedure for generating a sensible intuition. This first sense
  of rule anticipates what Kant, in the Schematism of the Pure
  Concepts of the Understanding, calls a schema. But the concept is a
  rule also in another, discursive sense. It is a rule in that
  thinking an object under a concept provides a reason to predicate of
  this object the marks that define the
  concept} \parencite[][50]{Longuenesse:KantandtheCapacitytoJudge1998}.}


Nach \S~1 der {\jaeschelogik} ist eine Erkenntnis diskursiv, insofern
sie \emph{reflektiert} ist. Denn diskursiv ist eine Erkenntnis durch Begriffe
und diese wiederum sind zunächst als reflektierte Vorstellungen zu verstehen.
Belege dafür finden sich auch im Handexemplar des
\authorcite{Meier:AuszugausderVernunftlehre1752}schen
Logiklehrbuchs: Für seine Logikvorlesungen notiert er: \enquote{Ein
  Begrif ist eine reflectierte 
Vorstellung.}\footnote{\cite[][\nopp 2834]{Kant:Reflexionen1900ff.},
\cite[][XVI: 536.2]{Kant:GesammelteWerke1900ff.}. Danach besteht die logische
Form eines Begriffs in der Reflexion, wodurch er eine allgemeine Vorstellung
werde \mkbibparens{\cite[vgl.][\nopp 2851]{Kant:Reflexionen1900ff.},
\cite[][XVI: 546.14--16]{Kant:GesammelteWerke1900ff.}}.} Und gegen
\authorcite{Meier:AuszugausderVernunftlehre1752}, der (allgemeine) Begriffe als
Ergebnis der Abstraktion ansieht,\footnote{\enquote{Alle Begriffe, welche
durch die logische Absonderung gemacht werden, sind \ori{abgesonderte} oder
\ori{abstracte Begriffe} (conceptus abstractus, notio). Begriffe, die nicht
abgesondert sind, heissen \ori{einzelne Begriffe} (conceptus singularis, idea)}
\mkbibparens{\cite[][\S~260]{Meier:AuszugausderVernunftlehre1752},
\cite[][XVI: 551.25--28]{Kant:GesammelteWerke1900ff.}}.} wendet er ein:
\enquote{Durch abstraction werden keine Begriffe, sondern durch reflexion:
entweder, wenn der Begrif gegeben ist, nur die Form und heißt reflectirter, oder
selbst der Begrif: reflectirender.}\footnote{\cite[][\nopp 2865]{Kant:Reflexionen1900ff.},
\cite[][XVI: 552.9--11]{Kant:GesammelteWerke1900ff.}.} Wir finden den Begriff
der Reflexion in der Redeweise von der \emph{reflektierenden} (im Gegensatz zur
bestimmenden) Urteilskraft, die nicht gegebene Begriffe anwendet, sondern
Begriffe bildet.\footnote{\enquote{Urteilskraft überhaupt ist das Vermögen, das Besondere als enthalten unter dem
Allgemeinen zu denken. Ist das Allgemeine (die Regel, das Prinzip, das Gesetz)
gegeben, so ist die Urteilskraft, welche das Besondere darunter subsumiert,
{\punkt} \ori{bestimmend}. Ist aber nur das Besondere gegeben, wozu sie das
Allgemeine finden soll, so ist die Urteilskraft bloß
\ori{reflektierend}} \mkbibparens{\cite[][B
xxv\,f.,]{Kant:KritikderUrteilskraft2009}
\cite[][V: 179.19--26]{Kant:GesammelteWerke1900ff.}}. In der \titel{Ersten
Einleitung} lesen wir: \enquote{\ori{Reflectiren} (Überlegen) aber ist:
gegebene Vorstellungen entweder mit andern, oder mit seinem
Erkenntnißvermögen, in Beziehung auf einen dadurch möglichen Begrif, zu
vergleichen und zusammen zu halten}
\mkbibparens{\cite[][16]{Kant:ErsteEinleitungindieenquoteKritikderUrteilskraft2009},
\cite[][XX: 211.14--16]{Kant:GesammelteWerke1900ff.}}. In der \titel{Kritik der
reinen Vernunft} spricht \name[Immanuel]{Kant} des weiteren von \enquote{transzendentaler} und
\enquote{logischer Reflexion}. Reflexion wird hier zunächst allgemein mit
\enquote{Überlegung} übersetzt, um dann zu sagen, dass die transzendentale
Überlegung darin besteht zu unterscheiden, ob die Vorstellungen
(\emph{respective} ihre \singlequote{Vergleichung}) zum reinen Verstand oder zur
sinnlichen Anschauung gehört. Dagegen betrachte die logische Reflexion lediglich
die Vorstellungen selbst, ohne darauf zu achten, ob sie zum reinen Verstand oder
zur sinnlichen Anschauung gehören \mkbibparens{\cite[vgl.][B
316--319]{Kant:KritikderreinenVernunft2003}, \cite[][III:
214.33--216.28]{Kant:GesammelteWerke1900ff.}}. Im Falle der Begriffsbildung
haben wir es -- wenn überhaupt mit einer Reflexion im Sinne der \titel{Kritik
der reinen Vernunft} -- mit einer logischen Überlegung zu tun.}
In der {\jaeschelogik} sowie in \singlequote{Reflexionen} und
Vorlesungsnachschriften, nicht aber in eigenständigen Publikationen finden sich
konkretere Ausführungen zu einer Theorie der Begriffsbildung, die auch den
Begriff der Reflexion weiter explizieren. Danach gehören drei Operationen des
Verstandes dazu, einen (empirischen) Begriff zu bilden:
\begin{quote}
Die logischen Verstandes-Actus, wodurch Begriffe ihrer Form nach erzeugt werden,
sind:
\begin{nummerierung}
\item die \ori{Komparation}, d.\,i. die Vergleichung der Vorstellungen unter
einander im Verhältnisse zur Einheit des Bewußtseins;
\item die \ori{Reflexion}, d.\,i. die Überlegung, wie verschiedene Vorstellungen
in Einem Bewußtsein begriffen sein können; und endlich
\item die \ori{Abstraktion} oder die Absonderung alles übrigen, worin die
gegebenen Vorstellungen sich
unterscheiden.\footnote{\cite[][\S~6]{Kant:ImmanuelKantsLogik1977},
\cite[][IX: 94.20--27]{Kant:GesammelteWerke1900ff.}. Siehe auch
\cite[][\nopp 2854]{Kant:Reflexionen1900ff.}, \cite[][XVI:
547.8--13]{Kant:GesammelteWerke1900ff.}. Es ergäbe eine krude Vorstellung,
verstünden wir solche Überlegungen als (quasi psychologische) Beschreibung unseres tatsächlichen
Begriffserwerbs. Doch das ist nicht nötig; \name[Immanuel]{Kant} beschreibt
hier nicht den Vorgang des Erwerbs eines Begriffs, sondern gedankliche
Operationen, derer ein Subjekt, das über Begriffe verfügt, \emph{fähig} sein
muss. \cite[Vgl.][\pno~82\,f.:]{Stuhlmann-Laeisz:KantsLogik1976} \enquote{Diese
Erklärung des Ursprungs von Begriffen will \name[Immanuel]{Kant} nicht verstanden wissen als eine
Beschreibung des wirklichen Vorgangs bei der Erwerbung eines Begriffs durch ein
denkendes Subjekt. Eine solche Beschreibung würde ja in die empirische
(Denk-)Psychologie, nicht aber in die formale Logik gehören. Der Anspruch dieser
Theorie ist vielmehr, die Bedingungen der Möglichkeit, Begriffe zu besitzen,
aufgezeigt zu haben: Ein Verstand kann genau dann Begriffe besitzen, wenn er der
drei betreffenden logischen Akte fähig ist. Nicht behauptet wird, daß wir de
facto auf die beschriebene Weise in den Besitz jedes unserer Begriffe
gelangen.}}
\end{nummerierung}
\end{quote}
Das Beispiel, das zur Erläuterung dient, lautet folgendermaßen: Wir sehen drei
verschiedene Bäume, beispielsweise eine Linde, eine Weide und eine Fichte. Wenn
wir diese vergleichen (komparieren), dann finden wir Gemeinsamkeiten und
Unterschiede. Um einen allgemeinen Begriff zu bilden, betrachten wir die
Gemeinsamkeiten: Sie alle verfügen über einen Stamm, Äste und Blätter. Diesen
Vorgang nennt \name[Immanuel]{Kant} Reflexion. Außerdem abstrahieren wir von den
Unterschieden in Größe und Gestalt. Was wir erhalten ist der Begriff des Baumes
mit seinen Merkmalen des Stammes, der Äste und der Blätter.

Dabei richten sich solche Ausführungen primär gegen die Auffassung,
(allgemeine) Begriffe seien \singlequote{abstrakte} Vorstellungen, die aus einer
Operation des Abstrahierens hervorgehen.
\authorfullcite{Meier:AuszugausderVernunftlehre1752} artikuliert eine solche
Auffassung, die schon daran scheitern muss, dass sie voraussetzt, Anschauungen
seien einzelne Begriffe. Wir müssen zwar auch in der Lage sein, von
Unterschieden zu abstrahieren. Aber diese Operation liefert uns noch keine
allgemeinen Merkmale. Dadurch, dass ich bei der Anschauung einer Fichte von der
Form der Äste und vielem anderen absehe, erhalte ich noch lange nicht den
Begriff eines Baumes. Dazu muss ich aktiv nach Gemeinsamkeiten Ausschau halten,
also reflektieren: überlegen, wie die Vorstellungen von Fichten, Linden und
Weiden in einem einzigen Bewusstsein vereinigt sein können.

Doch warum heißen Begriffe \enquote{\emph{diskursive}} Vorstellungen, wenn ihr
Wesen doch darin besteht, sich als \emph{reflektierte} Vorstellungen mittels
allgemeiner Merkmale auf ihre Gegenstände zu beziehen?
Nach der {\jaeschelogik} ist der Begriff als \emph{reflektierte} Vorstellung
eine \emph{repraesentatio
discursiva},\footnote{\cite[Vgl.][\S~1]{Kant:ImmanuelKantsLogik1977},
\cite[][IX: 91.10]{Kant:GesammelteWerke1900ff.}. Allerdings findet sich hierfür
keine direkte Vorlage in \name[Immanuel]{Kant}s Handexemplar des
\titel{Auszugs aus der Vernunftlehre}.} seine Diskursivität ist also daran
gebunden, dass er -- neben Komparation und Abstraktion -- auf der geistigen
Operation der \emph{Reflexion} beruht. Doch warum soll \enquote{reflektierte
Vorstellung} auf Latein als \enquote{\emph{representatio discursiva}} übersetzt
werden?

Ein Begriff ist reflektiert, weil er sich über allgemeine
Merkmale auf Gegenstände bezieht. Solche Merkmale sind Eigenschaften, die allen
Gegenständen zukommen, die unter den Begriff fallen. Sie entstammen der
Operation der Reflexion. Nun nennt \name[Immanuel]{Kant} Begriffe auch
\emph{Regeln}, worunter -- wie
\authorfullcite{Longuenesse:KantandtheCapacitytoJudge1998} feststellt --
zweierlei gemeint sein kann: Ein Begriff kann als Regel der Synthesis des
Mannigfaltigen der Anschauung oder als \singlequote{diskursive Regel} aufgefasst
werden.\footnote{\enquote{The concept is a rule insofar as it is the
consciousness of the unity of an act of sensible synthesis or the consciousness
of the procedure for generating a sensible intuition. This first sense of rule
anticipates what Kant, in the Schematism of the Pure Concepts of the
Understanding, calls a schema. But the concept is a rule also in another,
discursive sense. It is a rule in that thinking an object under a concept
provides a reason to predicate of this object the marks that define the concept}
\parencite[][50]{Longuenesse:KantandtheCapacitytoJudge1998}.} Dass ein Begriff
als diskursive Regel aufgefasst werden kann, heißt, dass er durch
Merkmale bestimmt ist, deren Angabe als Merkmale des Begriffs
Prinzipien an die Hand geben, die 
als Obersätze in Syllogismen fungieren (können).\footnote{\enquote{Every concept
is a rule insofar as its explication (e.\,g., a body is extended, limited in
space and impenetrable) can function as the major premise in a syllogism whose
conclusion would be the attribution of the marks belonging to this concept to an
object of sensible intuition}
\parencite[][50]{Longuenesse:KantandtheCapacitytoJudge1998}.}
Wenn ich vor mir eine bellende Dogge sehe und das Bellen ein
hinreichendes Merkmal dafür ist, dass etwas in Hund ist, dann kann ich
darauf schließen, dass die Dogge vor mir ein Hund ist. Ich erkenne die Dogge als Hund mittels des
allgemeinen Merkmals des Bellens. Um dies bewusst zu tun -- um das Bellen
\emph{als} Merkmal zu erkennen --, muss ich einen solchen Vernunftschluss
ausführen. Und einen solchen Vernunftschluss wiederum nennt die philosophische
Tradition des 18. Jahrhunderts einen \enquote{\emph{discursus}}. In
diesem Sinne ist es dann naheliegend, auch Begriffe \emph{diskursiv}
zu nennen, insofern sie sich vermittelt über allgemeine Merkmale auf
Gegenstände beziehen.

\phantomsection\label{Abschnitt:judiciumintuitivumdiscursivum}
Die historische Vorlage dieser Verwendungsweise der Begriffe \enquote{diskursiv}
und \enquote{intuitiv} findet sich in der Gegenüberstellung intuitiver und
diskursiver \emph{Urteile}, wie wir sie etwa bei
\authorfullcite{Wolff:Discursuspraeliminarisdephilosophiaingenere1996}
finden. \authorcite{Wolff:PhilosophiarationalissiveLogica1740}
unterscheidet mit den Attributen \enquote{intuitiv} und \enquote{diskursiv}
Urteile als je konkrete Akte des menschlichen Geistes. Nach
\authorcite{Wolff:Discursuspraeliminarisdephilosophiaingenere1996}
heißen diejenigen Urteile
diskursiv, die wir erst durch einen Syllogismus aus anderen Urteilen gewinnen,
während alle Urteile intuitiv genannt werden, deren Wahrheit wir ohne einen solchen
Vernunftschluss erkennen.\footnote{\enquote{\ori{Judicium} istud dicimus
\ori{intuitivum}, quo enti cuidam tribuimus, qu{\ae} in ipsius notione
comprehensa intuemur. Istud autem \ori{judicium discursivum} appellamus, quod
per ratiocinium {\punkt} elicitur.
Posset quoque dici \ori{diano{"e}ticum}}
\parencite[][\S~51]{Wolff:PhilosophiarationalissiveLogica1740}.} Der
\emph{discursus} ist die Tätigkeit des logischen
Schließens.\footnote{\cite[Vgl.][\S~52\,f.]{Wolff:PhilosophiarationalissiveLogica1740}.
Mitunter scheint er auf Wortgebräuche bei anderen verweisen zu wollen:
\enquote{Alii ratiocinationem \ori{Dian{\oe}am}, \ori{Discursum},
\ori{Argumentationem} vocant}
\parencite[][\S~50]{Wolff:PhilosophiarationalissiveLogica1740}. Siehe auch
\cite[][\S~204]{Baumgarten:AcroasislogicainChristianumL.B.deWolff1983}.}
Diskursive Urteile sind also -- in neuerer Sprache -- inferentiell, während
intuitive Urteile nicht-inferentielle Erkenntnisse sind.\footnote{Dies passt
zunächst natürlich zu \name[Immanuel]{Kant}s Unterscheidung diskursiver und
intuitiver rationaler Erkenntnisse im eigentlichen Sinne, also der
Unterscheidung des Vorgehens von Philosophie und Mathematik als
unterschiedlicher Vernunftwissenschaften. Die Mathematik verfügt über Axiome,
also unmittelbar einsichtige Vernunftwahrheiten, die nach
\authorcite{Wolff:Discursuspraeliminarisdephilosophiaingenere1996} als intuitive
Urteil gelten können (wobei
\authorcite{Wolff:Discursuspraeliminarisdephilosophiaingenere1996} primär an
sinnliche Anschauungen denkt).}
\authorcite{Baumgarten:AcroasislogicainChristianumL.B.deWolff1983} konkretisiert
dies dahingehend, dass die intuitiven Urteile (\emph{propositiones intuitivae})
diejenigen sind, derer wir aus unserer Erfahrung gewiss sind. Die
Gesamtmenge unserer Urteile unterteilt er also in
diskursive Urteile, die wir auf der Grundlage anderer Wahrheiten erschlossen
haben, auf der einen Seite und intuitive Erfahrungsurteile auf der
anderen Seite.\footnote{\enquote{\ori{Propositio} per experientiam
    nobis complete certa, est \ori{intuitiva}, ex aliis vero cognita,
    \ori{discursiva}} \parencite[][\S~166]{Baumgarten:AcroasislogicainChristianumL.B.deWolff1983}.}
Das intuitive Urteil nennt er auch einen \enquote{Erfahrungssatz}, das
diskursive Urteil eine
\enquote{Folgerung}.\footcite[Vgl.][\S~166]{Baumgarten:AcroasislogicainChristianumL.B.deWolff1983}
\authorcite{Meier:AuszugausderVernunftlehre1752} schließt sich
\authorcite{Baumgarten:Metaphysica---Metaphysik2011} an und nennt das
\emph{iudicium discursivum} ein \enquote{Nachurteil}, das \emph{iudicium
intuitivum} hingegen ein \enquote{anschauendes Urteil}, welches eine
unmittelbare Erfahrung darstelle (und \emph{a fortiori} singulär
sei).\footnote{\enquote{Die erweislichen Urtheile sind entweder bloss durch die
Erfahrung gewiss, oder nicht. Jene sind \ori{anschauende Urtheile} (iudicium
intuitivum), diese aber \ori{Nachurtheile} (iudicium discursivum). Das
\ori{anschauende Urtheil besteht aus lauter Erfahrungsbegriffen}, und ist eine
unmittelbare Erfahrung {\punkt}, und ein einzelnes Urtheil}
\mkbibparens{\cite[][\S~319]{Meier:AuszugausderVernunftlehre1752},
\cite[][XVI: 674.24--28]{Kant:GesammelteWerke1900ff.}}.}


\subsection[Verstand und Sinnlichkeit]{Verstand und Sinnlichkeit:
\enquote{intellektuell} und
\enquote{sinnlich}}\label{subsection:VerstandundRezeptivitaet}
Ein offensichtliches Problem bringt die Vielfalt der Charakterisierungen des
Verstandes mit sich, wenngleich \name[Immanuel]{Kant} behauptet, sie liefen
letztlich alle auf dasselbe hinaus\footnote{\cite[Vgl.][A
126]{Kant:KritikderreinenVernunft2003}, \cite[][IV:
92.25--29]{Kant:GesammelteWerke1900ff.}.}. Allein in der \titel{Kritik der
reinen Vernunft} finden sich folgende Charakterisierungen:
In der Einleitung sagt \name[Immanuel]{Kant}, \enquote{daß es zwei Stämme der
menschlichen Erkenntnis gebe, die vielleicht aus einer gemeinschaftlichen, aber
uns unbekannten Wurzel entspringen, nämlich Sinnlichkeit und Verstand, durch
deren ersteren uns Gegenstände \ori{gegeben}, durch den zweiten aber
\ori{gedacht} werden.}\footnote{\cite[][B 29]{Kant:KritikderreinenVernunft2003},
\cite[][III: 46.7--11]{Kant:GesammelteWerke1900ff.}.} Der Verstand ist danach also das
\emph{Vermögen, zu denken}, und in dieser Funktion ist er das Vermögen der
\enquote{Spontaneität der Begriffe}\footnote{\cite[][B
74]{Kant:KritikderreinenVernunft2003}, \cite[][III:
74.12]{Kant:GesammelteWerke1900ff.}.}.
Direkt im Anschluss wird der Verstand als \enquote{das Vermögen,
Vorstellungen selbst hervorzubringen, oder die \ori{Spontaneität} des
Erkenntnisses}\footnote{\cite[][B 75]{Kant:KritikderreinenVernunft2003},
\cite[][III: 75.7--8]{Kant:GesammelteWerke1900ff.}.} charakterisiert. Und später
wird der Verstand auch als das \emph{Vermögen der Begriffe}
beschrieben.\footnote{\cite[Vgl.][B 199]{Kant:KritikderreinenVernunft2003},
\cite[][III: 146.29--30]{Kant:GesammelteWerke1900ff.}.} An wiederum anderer
Stelle findet sich die Charakterisierung des Verstandes als des \emph{Vermögens
zu urteilen}\footnote{\cite[Vgl.][B 94]{Kant:KritikderreinenVernunft2003},
\cite[][III: 86.12]{Kant:GesammelteWerke1900ff.}.}. In einer
Anmerkung zu \S~16 der \titel{Kritik der reinen Vernunft} wiederum identifiziert
\name[Immanuel]{Kant} den Verstand mit dem \emph{Vermögen der synthetischen
Einheit der Apperzeption}: \enquote{Und so ist die
synthetische Einheit der Apperzeption der höchste Punkt, an dem man allen
Verstandesgebrauch, selbst die ganze Logik, und, nach ihr, die
Transzendental-Philosophie heften muß, ja dieses Vermögen ist der Verstand
selbst}\footnote{\cite[][\S~16]{Kant:KritikderreinenVernunft2003}, \cite[][III:
109.35--38]{Kant:GesammelteWerke1900ff.}}. Er ist dadurch das \emph{Vermögen,
\emph{a priori} zu
verbinden}\footnote{\cite[Vgl.][\S~16]{Kant:KritikderreinenVernunft2003},
\cite[][III: 110.15]{Kant:GesammelteWerke1900ff.}.}.
Wenige Seiten später heißt es, der Verstand sei schlicht das
\emph{Vermögen der Erkenntnisse}.\footnote{\enquote{\ori{Verstand} ist,
allgemein zu reden, das Vermögen der
\ori{Erkenntnisse}}
\mkbibparens{\cite[][\S~17]{Kant:KritikderreinenVernunft2003}, \cite[][III:
111.16]{Kant:GesammelteWerke1900ff.}}.}
In der \titel{Transzendentalen Dialektik} taucht außerdem die Charakterisierung
des Verstandes als \enquote{Vermögen der Einheit der Erscheinungen vermittelst
der Regeln}\footnote{\cite[][B 359]{Kant:KritikderreinenVernunft2003},
\cite[][III: 239.27--28]{Kant:GesammelteWerke1900ff.}.
Diese Definition liegt auch der Unterscheidung von Urteils\emph{vermögen}
(Verstand) und Urteils\emph{kraft} zugrunde:
\enquote{Wenn der Verstand überhaupt als das Vermögen der Regeln erklärt wird,
so ist Urteilskraft das Vermögen unter Regeln zu \ori{subsumieren}}
\mkbibparens{\cite[][B 171]{Kant:KritikderreinenVernunft2003}, \cite[][III:
131.13--14]{Kant:GesammelteWerke1900ff.}}. Siehe auch \cite[][B 197f., 356, A
126f.,]{Kant:KritikderreinenVernunft2003} \cite[][III:
146.6--12, 238.8--11, IV: 92.29--30]{Kant:GesammelteWerke1900ff.}.} auf.


Es hat zunächst den Anschein, als brächte \name[Immanuel]{Kant} etliche
verschiedene Definitionen des Verstandesbegriffs, ohne deren Zusammenhang zu
verdeutlichen. Es liegt nahe davon auszugehen, dass all diese
Bestimmungen letztlich übereinkommen, dass es verschiedene
Beschreibungen desselben Vermögens -- des Verstandes -- sind. 
Allerdings wird sich zeigen, dass dies nur gilt, solange es sich um einen
\emph{endlichen} Verstand handelt. Die Endlichkeit des Verstandes ist eine
Voraussetzung, auf deren Grundlage sich erst die Äquivalenz der
genannten Definitionen ergibt. Einige der Beschreibungen treffen auf
jeden denkbaren Verstand zu; sie bilden die Bestimmung des Begriffs
\enquote{Verstand}. Andere wiederum treffen nur auf einen endlichen
Verstand, nicht aber auf einen unendlichen oder
\singlequote{intuitiven} Verstand zu. Sie bestimmen den Begriff des
diskursiven Verstandes.

Eine Bestimmung des Verstandesbegriffs, die neutral ist gegenüber der Frage, ob
der Verstand diskursiv oder intuitiv ist und ob es sich um unseren endlichen
oder einen göttlichen (oder irgendwie \singlequote{anderen}) Verstand handelt,
findet sich in den Anfangsabschnitten der \titel{Transzendentalen Logik} der
\titel{Kritik der reinen Vernunft}:
\begin{quote}
Wollen wir die \ori{Rezeptivität} unseres Gemüts, Vorstellungen zu empfangen, so
fern es auf irgend eine Weise affiziert wird, \ori{Sinnlichkeit} nennen; so ist
dagegen das Vermögen, Vorstellungen selbst hervorzubringen, oder die
\ori{Spontaneität} des Erkenntnisses, der
\ori{Verstand}.\footnote{\cite[][B 75]{Kant:KritikderreinenVernunft2003},
\cite[][III: 75.5--8]{Kant:GesammelteWerke1900ff.}.}
\end{quote}
Diese Bestimmung baut unmittelbar auf der vorausgesetzten Dualität der
Erkenntnisstämme auf: \name[Immanuel]{Kant} spricht von Rezeptivität und
Spontaneität als den beiden \enquote{Grund\-quel\-len des
Gemüts}\footnote{\cite[][B 74]{Kant:KritikderreinenVernunft2003},
\cite[][III: 74.9]{Kant:GesammelteWerke1900ff.}.} oder den beiden
\enquote{Stämme[n] der menschlichen Erkenntnis}\footnote{\cite[][B
29]{Kant:KritikderreinenVernunft2003}, \cite[][III:
46.7]{Kant:GesammelteWerke1900ff.}.}. Er sagt damit, dass unser Gemüt einerseits
aus sich selbst heraus tätig und andererseits bloß leidend ist und insofern über
zwei Stämme oder Grundquellen verfügt: Spontaneität und Rezeptivität.

Die Dualität der Erkenntnisstämme oder Grundquellen des Gemüts -- Spontaneität
und Rezeptivität oder Verstand und Sinnlichkeit -- ist eine entscheidende
Voraussetzung der \titel{Kritik der reinen Vernunft}. In der Einleitung schreibt
\name[Immanuel]{Kant}:
\begin{quote}
Nur so viel scheint zur Einleitung, oder Vorerinnerung, nötig zu sein, daß es
zwei Stämme der menschlichen Erkenntnis gebe, die vielleicht aus einer
gemeinschaftlichen, aber uns unbekannten Wurzel entspringen, nämlich
Sinnlichkeit und Verstand, durch deren ersteren uns Gegenstände \ori{gegeben},
durch den zweiten aber \ori{gedacht}
werden.\footnote{\cite[][B 29]{Kant:KritikderreinenVernunft2003},
\cite[][III: 46.6--11]{Kant:GesammelteWerke1900ff.}.}
\end{quote}
Dieses Zitat versammelt die entscheidenden Voraussetzungen der Vernunftkritik, insofern es
die Endlichkeit unseres Verstandes als etwas ansieht, das keiner Begründung und
keines Nachweises, sondern lediglich einer \enquote{Vorerinnerung} fähig und
bedürftig sei. \authorfullcite{Gloy:DieKantischeDifferenzvonBegriffundAnschauungundihreBegruendung1984}
hat \name[Immanuel]{Kant}s Dualismus kritisiert, weil sie eine Begründung für
dieses Vorgehen
vermisst.\footnote{\cite[Vgl.][\pno~1\,f.]{Gloy:DieKantischeDifferenzvonBegriffundAnschauungundihreBegruendung1984}.
Zu weiteren Vertretern dieser Kritik an \name[Immanuel]{Kant} siehe die von
\textcite[vgl.][Anm. 3 auf S.~66\,f.]{Heidemann:AnschauungundBegriff2002}
angeführte Literatur.} Es ist ein alter Vorwurf an \name[Immanuel]{Kant}, dass seine
Philosophie auf unbegründeten Annahmen errichtet sei, zu denen insbesondere die
Endlichkeit des Menschen und seines Verstandes gehöre, über deren Momente er
sich und dem Leser keine Rechenschaft ablege.\footnote{So schreibt
\authorcite{Hegel:GesammelteWerke} in \titel{Glauben und Wissen}: \enquote{Kant
hat keinen andern Grund als schlechthin die Erfahrung und die empirische
Psychologie, daß das menschliche Erkenntnißvermögen seinem Wesen nach in dem
bestehe, wie es erscheint, nemlich in jenem Fortgehen vom Allgemeinen zum
Besondern oder rückwärts vom Besondern zum Allgemeinen; aber indem er selbst
einen intuitiven Verstand denkt, auf ihn als absolut nothwendige Idee geführt
wird, stellt er selbst die entgegengesetzte Erfahrung von dem Denken eines nicht
discursiven Verstandes auf, und erweist, daß sein Erkenntnißvermögen erkennt,
nicht nur die Erscheinung und die Trennung des Möglichen und Wirklichen in
derselben, sondern die Vernunft und das An-sich} \parencite[][IV:
341.21--29]{Hegel:GesammelteWerke}.} In der Tat gehen auch die meisten
Interpreten davon aus, dass sich eine Begründung dieser zentralen Annahme in der
\titel{Kritik der reinen Vernunft} wie in anderen Schriften
\name[Immanuel]{Kant}s nicht finden lässt.\footnote{Eine Ausnahme bildet
\authorfullcite{Heidemann:AnschauungundBegriff2002}, der behauptet, die
Unterscheidung zweier Stämme der Erkenntnis folge aus der Trennung von
Anschauung und Begriff, welche \name[Immanuel]{Kant} zunächst voraussetze, um
sie dann \enquote{im Nachhinein rekonstruktiv zu beweisen}
\parencite[][69]{Heidemann:AnschauungundBegriff2002}.}

\name[Immanuel]{Kant} sagt an der genannten Stelle aber nicht nur, dass es zwei
Erkenntnisstämme gebe und dass sie vielleicht einer uns unbekannten gemeinsamen
Wurzel entspringen. Er sagt außerdem, dass durch Sinnlichkeit Gegenstände
\emph{gegeben}, durch den Verstand dieselben jedoch \emph{gedacht} werden. Und
\emph{prima facie} könnte es scheinen, als definierte \name[Immanuel]{Kant} hier
erstmalig die Begriffe \enquote{Sinnlichkeit} und \enquote{Verstand}, die später
dann als Rezeptivität und Spontaneität beschrieben werden.
\name[Immanuel]{Kant} sagt also, (a) \emph{Sinnlichkeit} sei
(a\textsubscript{i}) die \emph{Rezeptivität} des Gemüts, durch die uns
(a\textsubscript{ii}) Gegenstände \emph{gegeben} werden; und weiter, (b) der
\emph{Verstand} sei (b\textsubscript{i}) ein Vermögen der \emph{Spontaneität},
durch das (b\textsubscript{ii}) Gegenstände \emph{gedacht} werden. Wie im Falle
von Begriffen und Anschauungen konfrontiert uns \name[Immanuel]{Kant} mit
verschiedenen Merkmalen, deren Zusammenhang erst zu explizieren ist. Die beiden
jeweils angegebenen Merkmale stimmen jedoch auch nach \name[Immanuel]{Kant}s
eigener Auffassung nicht \emph{per se} überein.
Daran ändert auch nicht, dass sich bei \name[Immanuel]{Kant} mitunter
Formulierungen finden, die den Anschein erwecken, als stelle der Verweis auf
Anschauungen und Begriffe eine ebenso legitime Möglichkeit dar, die Begriffe
\enquote{Sinnlichkeit} und \enquote{Verstand} zu definieren, wie der Verweis auf
Rezeptivität und Spontaneität. In der \jaeschelogik{} steht:
\begin{quote}
Reflektieren wir auf unsre Erkenntnisse in Ansehung der beiden wesentlich
verschiedenen Grundvermögen der Sinnlichkeit und des Verstandes, woraus sie
entspringen: so treffen wir auf den Unterschied zwischen Anschauungen und
Begriffen. Alle unsre Erkenntnisse nämlich sind, in dieser Rücksicht betrachtet,
entweder \ori{Anschauungen} oder \ori{Begriffe}. Die erstern haben ihre Quelle
in der \ori{Sinnlichkeit} -- dem Vermögen der Anschauungen; die letztern im
\ori{Verstande} -- dem Vermögen der Begriffe. Dieses ist der \ori{logische}
Unterschied zwischen Verstand und Sinnlichkeit, nach welchem diese nichts als
Anschauungen, jener hingegen nichts als Begriffe
liefert.\footnote{\cite[][A 45]{Kant:ImmanuelKantsLogik1977},
\cite[][IX: 35.33--36.8]{Kant:GesammelteWerke1900ff.}.}
\end{quote}
Neben diesem \singlequote{logischen Unterschied} gebe es jedoch auch einen
\emph{metaphysischen} Unterschied beider \singlequote{Grundvermögen}, wonach die Sinnlichkeit
ein \singlequote{Vermögen} der Rezeptivität, der Verstand eines der Spontaneität
sei.\footnote{\cite[Vgl.][A 45]{Kant:ImmanuelKantsLogik1977},
\cite[][IX: 36.8--12]{Kant:GesammelteWerke1900ff.}.
\authorfullcite{Heidemann:AnschauungundBegriff2002} spricht von einer
vermögenstheoretischen Unterscheidung im Gegensatz zur erkenntnistheoretischen
Unterscheidung zwischen dem Vermögen der Anschauungen und dem der Begriffe
\parencite[vgl.][84]{Heidemann:AnschauungundBegriff2002}.} Doch auch wenn diese
Stelle \name[Immanuel]{Kant}s Auffassung treffen sollte,\footnote{Diese Stelle
der {\jaeschelogik} scheint mir wenig verlässlich zu sein, insofern die Nennung
von logischem und metaphysischem Gesichtspunkt an keiner anderen Stelle im Werk
\name[Immanuel]{Kant}s vorzukommen scheint
\parencite[vgl.][83]{Heidemann:AnschauungundBegriff2002}. Wegen der in Anm.
\ref{Anmerkung:Einleitung:VorbehaltegegenueberderJaescheLogik} auf
S.~\pageref{Anmerkung:Einleitung:VorbehaltegegenueberderJaescheLogik}
angesprochenen Vorbehalte sollte diese Passage daher nur unter Vorbehalt
herangezogen werden.} gilt: Nur der metaphysische Unterschied markiert eine
grundlegende Definition von Verstand und Sinnlichkeit, die sich unabhängig von
weiteren Voraussetzungen anwenden lässt; der logische Unterschied ist nur bei
endlichen Wesen anwendbar. Das entscheidende Argument ist hier:
Wären beide gleich zulässig, könnten wir weder den Begriff einer intellektuellen
Anschauung noch den eines anschauenden Verstandes bilden.

Eine Darlegung, wie sich die Charakterisierung des Verstandes als des
Vermögens der Begriffe aus der grundlegenden Charakterisierung als Vermögen der
Spontaneität unter Zugrundelegung seiner Endlichkeit ergibt, bestätigt die hier
vorgelegte Interpretation.
\begin{quote}
Der Verstand wurde oben bloß negativ erklärt: durch ein nicht-sinnliches
Erkenntnisvermögen. Nun können wir, unabhängig von der Sinnlichkeit, keiner
Anschauung teilhaftig werden. Also ist der Verstand kein Vermögen der
Anschauung. Es gibt aber, außer der Anschauung, keine andere Art, zu erkennen,
als durch Begriffe. Also ist die Erkenntnis eines jeden, wenigstens des
menschlichen, Verstandes, eine Erkenntnis durch Begriffe, nicht intuitiv,
sondern diskursiv.\footnote{\cite[][B
92\,f.,]{Kant:KritikderreinenVernunft2003} \cite[][III:
85.10--16]{Kant:GesammelteWerke1900ff.}.}
\end{quote}
Wenn \name[Immanuel]{Kant} den Verstand hier zunächst nur negativ, als ein
nicht-sinnliches Erkenntnisvermögen bestimmt, dann ist damit seine grundlegende
Bestimmung als Verstand bereits vollständig angegeben. Die weitere Bestimmung
als ein Vermögen der Begriffe bestimmt ihn bereits als einen endlichen Verstand.
Denn \emph{wir} verfügen über keine Anschauung unabhängig von unserer
Sinnlichkeit. Anschauung ist also an Rezeptivität gebunden -- aber nur für
unseren endlichen Verstand. Da unser Verstand ein Erkenntnisvermögen ist,
welches nicht anzuschauen vermag, bleibt nur, dass er ein Vermögen der Begriffe
ist.

Dass der Verstand ein Vermögen der Spontaneität ist, dies ist die primäre
Bestimmung seines Begriffs, denn es trifft auch auf den intuitiven Verstand zu.
Daher kann \name[Immanuel]{Kant} in der \titel{Kritik der Urteilskraft}
schreiben, dass \enquote{ein Vermögen einer \ori{völligen Spontaneität der
Anschauung} {\punkt} Verstand in der allgemeinsten Bedeutung sein
würde}\footnote{\cite[][\S~77]{Kant:KritikderUrteilskraft2009},
\cite[][V: 406.21--24]{Kant:GesammelteWerke1900ff.}. Dieses Zitat missversteht
\textcite[vgl.][285]{Westphal:KantHegelandtheFateofenquotetheIntuitiveIntellect2000},
der nicht erkennt, dass \enquote{Verstand in der allgemeinsten Bedeutung} auf
die ursprüngliche Definition als Vermögen der Spontaneität verweist. Diese
Bedeutung ist allgemeiner als die Bestimmung des Verstandes als Vermögen der
Spontaneität der Begriffe, insofern sie auch den anschauenden Verstand mit
einschließt.
\authorcite{Westphal:KantHegelandtheFateofenquotetheIntuitiveIntellect2000}s
Schlussfolgerung, dass auch der anschauende Verstand ein Vermögen der Begriffe
sei, übersieht genau dies.}.
Ich behaupte also, dass für den Begriff der Sinnlichkeit das Merkmal der
Rezeptivität (a\textsubscript{i}) und für den Begriff des Verstandes das
Merkmal der Spontaneität (b\textsubscript{i}) grundlegend ist und sich andere
Charakterisierungen -- insbesondere (a\textsubscript{ii}) und
(b\textsubscript{ii}) -- erst auf seiner Grundlage ergeben. Für viele
Beschreibungen ist dabei die Endlichkeit des Verstandes bereits vorausgesetzt,
denn nur unter dieser Annahme lässt sich der Verstand auch als Vermögen der
Begriffe, des Urteilens, des Denkens oder der Regeln beschreiben. Es sind dies
keine Merkmale des Verstandes als eines solchen, sondern Merkmale des
\emph{endlichen} Verstandes.

\subsection{Endlichkeit des
Verstandes}\label{subsection:EndlichkeitdesVerstandes}
Begriffe sind diskursive Vorstellungen, weil sie sich nur \emph{vermittelt über
allgemeine Merkmale} auf Gegenstände beziehen. Anschauungen hingegen sind als
Vorstellungen mit unmittelbarem Gegenstandsbezug intuitiv. Dass Begriffe
allgemeine und Anschauungen einzelne Vorstellungen sind, ist eine nachrangige
Bestimmung, die aus Mittelbarkeit und Unmittelbarkeit folgt. Sinnlichkeit ist
nach \name[Immanuel]{Kant} definiert durch die Rezeptivität, der Verstand
hingegen durch Spontaneität. Andere Charakterisierungen setzen in der Regel
voraus, dass es sich um einen \emph{endlichen} Verstand handelt. Was ist nun die
Endlichkeit oder \singlequote{Diskursivität} des Verstandes, die als
zentrale Voraussetzung anzusehen ist? Wodurch wird sie bestimmt?

\authorfullcite{Foerster:Die25JahrederPhilosophie2011} behauptet,
\name[Immanuel]{Kant} betrachte im Themenbereich \enquote{intellektuelle
Anschauung} den Unterschied zwischen Rezeptivität und Spontaneität (der
Anschauung), im Themenbereich \enquote{intuitiver Verstand} aber den Unterschied
von Diskursivität und Intuitivität (des
Verstandes).\footcite[Vgl.][177]{Foerster:DieBedeutungvonSS7677deremphKritikderUrteilskraftfuerdieEntwicklungdernachkantischenPhilosophieTeil12002}
Er hat dabei sicherlich vollkommen Recht, wenn er darauf pocht, dass
\name[Immanuel]{Kant} einerseits zwischen \enquote{Rezeptivität} und
\enquote{Spontaneität}, andererseits aber zwischen \enquote{intuitiv} und
\enquote{diskursiv} unterscheidet. Aber daraus folgt gerade nicht, dass es sich
bei dem Vergleich unserer sinnlichen mit einer denkbaren intellektuellen
Anschauung und dem Vergleich unseres diskursiven Verstandes mit einem intuitiven
Verstand um völlig heterogene Ansätze handelt.

Wie so oft führt \name[Immanuel]{Kant} verschiedene Unterscheidungen ein, die
leicht konfundiert werden können, aber nicht verwechselt werden sollten, weil sie
jeweils ganz unterschiedliches unterscheiden: Das Begriffspaar
\enquote{Anschauung}/\enquote{Begriff} differenziert \emph{Erkenntnisse}, das
Begriffspaar \enquote{Verstand}/\enquote{Sinnlichkeit} differenziert
Erkenntnis\emph{vermögen}. Erkenntnisvermögen können spontan oder rezeptiv sein,
Erkenntnisse unmittelbar oder vermittelt, einzeln oder allgemein. Und die Frage nach einer intellektuellen
Anschauung geht ebenso wie die Frage nach der Möglichkeit eines anschauenden
oder intuitiven Verstandes darauf, welche Kombinationsmöglichkeit von
Erkenntnissen und Erkenntnisvermögen es gibt. Konkret geht es in beiden Fällen
darum, ob ein Vermögen der Spontaneität Anschauungen generieren kann, ohne auf
Rezeptivität als die andere Grundquelle unseres Gemüts zurückzugreifen.

\authorfullcite{Foerster:Die25JahrederPhilosophie2011} scheint mit dem Dualismus
von Rezeptivität und Spontaneität bereits unsere Endlichkeit -- zumindest
hinsichtlich der \titel{Kritik der reinen Vernunft} --
charakterisiert zu sehen, denn er kontrastiert dem
Stämmedualismus die intellektuelle Anschauung.\footnote{\enquote{Weil wir in
Verstand und Sinnlichkeit zwei voneinander unabhängige Stämme der Erkenntnis
haben, müssen wir zwischen Möglichkeit und Wirklichkeit unterscheiden (anders:
eine intellektuelle Anschauung)}
\parencite[][153]{Foerster:Die25JahrederPhilosophie2011}.}
\authorfullcite{Leech:MakingModalDistinctions2014} sagt, ein Verstand
sei unendlich, bei dem die beiden Vermögen zu einem Vermögen
verschmelzen. Ihre Darstellung beruht jedoch darauf, den Verstand mit dem
Vermögen zu denken zu identifizieren.\footnote{\enquote{My suggestion is that
the best way to understand these capacities is primarily as the product of collapsing our two
distinct capacities for thought and intuition into one. If understanding is a
capacity for thought, then \ori{intuitive understanding} is a capacity for
thought which can provide its own intuitions -- a capacity for thoughts
directly (immediately) about individuals. Likewise, if intuition is a capacity
for immediate, singular representation of individuals, then \ori{intellectual
intuition} is a capacity for such representations of individuals through an
intellectual capacity for thinking (not a capacity for sensing)}
\parencite[][345]{Leech:MakingModalDistinctions2014}.}
Ein unendlicher Verstand, bei dem beide Vermögen vereinigt sind, verfügte allem
Anschein nach über Rezeptivität wie Spontaneität und wäre ein Vermögen der
Anschauung ebenso wie ein Vermögen der Begriffe. Das jedoch scheint mir falsch
zu sein: Ein unendlicher Verstand verfügte nicht über Rezeptivität -- er wäre
ausschließlich aus sich selbst heraus tätig, also ganz Spontaneität -- und er
wäre auch kein Vermögen der Begriffe, sondern ausschließlich ein Vermögen der
Anschauung.

Dass ein unendlicher Verstand keine Rezeptivität beinhaltet, scheint mir
offensichtlich zu sein, denn Rezeptivität ist abhängig von äußeren
Einflüssen. Fraglich ist lediglich, ob ein unendlicher Verstand über
Begriffe verfügt. \authorfullcite{Westphal:KantHegelandtheFateofenquotetheIntuitiveIntellect2000}
bejaht dies: Es sei zwar üblich, den intuitiven Verstand als ein
nicht-begriffliches Vermögen zu beschreiben, doch bestimme \name[Immanuel]{Kant}
den Verstand als Vermögen der Begriffe und bezeichne auch den anschauenden
Verstand als Verstand in der allgemeinsten
Bedeutung.\footnote{\cite[Vgl.][\pno~284\,f.]{Westphal:KantHegelandtheFateofenquotetheIntuitiveIntellect2000}.}
\authorcite{Westphal:KantHegelandtheFateofenquotetheIntuitiveIntellect2000}s
Fehler liegt aber darin, nicht zu sehen, dass \name[Immanuel]{Kant} unseren
Verstand in zwei Stufen bestimmt: Als \emph{Verstand} ist er das Vermögen der
Spontaneität, als \emph{endlicher} Verstand ein Vermögen der Spontaneität der Begriffe.
\name[Immanuel]{Kant} selbst sagt explizit, dass ein unendlicher Verstand nicht
über Begriffe verfügte; denn eine Erkenntnis durch Begriffe heißt Denken, und
Denken wiederum beweise Schranken und könne daher dem göttlichen Verstand nicht
zugesprochen werden.\footnote{\cite[Vgl.][B
71]{Kant:KritikderreinenVernunft2003}, \cite[][III:
72.10--16]{Kant:GesammelteWerke1900ff.}. Siehe auch
\cite[][\nopp 6050]{Kant:Reflexionen1900ff.}, \cite[][XVIII:
434.22--24]{Kant:GesammelteWerke1900ff.}: \enquote{Die Ideen aber dieses
Ursprünglichen Verstandes können nicht Begriffe, sondern nur Anschauungen, aber
intellectuelle, seyn.}} Es spricht auch die Bestimmung von Begriffen als
mittelbaren Vorstellungen dafür, sie einem unendlichen Verstand nicht
zuzusprechen. Denn dieser erkennte allem Vermuten nach ausschließlich
unmittelbar, also anschaulich. Von Begriffen und von Denken kann daher bei einem
unendlichen Verstand gar nicht gesprochen werden. Stattdessen geht es darum, einem Vermögen der Spontaneität
Erkenntnisse zuzuschreiben, die sich unmittelbar auf ihre Gegenstände
beziehen.\footnote{Auch \authorfullcite{Leech:MakingModalDistinctions2014} tendiert an einer Stelle in
diese Richtung, bringt aber sofort das Vermögen des Denkens wieder ins Spiel,
weil sie den Verstand primär als Denkvermögen und erst \emph{a fortiori} als
Spontaneität bestimmt sieht
\parencite[vgl.][\pno~345\,f.]{Leech:MakingModalDistinctions2014}.
Dadurch muss sie wiederum annehmen, in der Tätigkeit des anschauenden Verstandes
fielen Denken und Anschauen in eins, was hieße, dass Mittelbarkeit (Begriffe)
und Unmittelbarkeit (Anschauungen) identifiziert würden. Dies
scheint mir jedoch keinen vernünftigen Sinn zu ergeben.}

Ebenso wäre freilich ein Wesen denkbar, dass zwar über beide Stämme verfügt,
dessen Verstand aber zumindest manchmal Dinge anschaut, ohne auf Sinnlichkeit
angewiesen zu sein; er könnte \emph{manche} Gegenstände sinnlich, andere rein
intellektuell anschauen. \authorcite{Fichte:DieBestimmungdesMenschen1800}s
intellektuelle Anschauung des Ich -- \name[Immanuel]{Kant} hat in den 1770er
Jahren möglicherweise eine ähnliche Konzeption
vertreten\footnote{\cite[Vgl.][78]{Duesing:SpontanediskursiveSynthesis2004}.} --
wäre ein Beispiel für eine solche Konzeption.
Anders verhält es sich bei unserem Verstand: Abstrahieren wir von der
Sinnlichkeit, so bleibt \enquote{nichts als die bloße Form des Denkens ohne
Anschauung übrig, wodurch allein ich nichts Bestimmtes, also keinen Gegenstand
erkennen kann. Ich müßte mir zu dem Ende einen andern Verstand denken, der die
Gegenstände anschauete, wovon ich aber nicht den mindesten Begriff habe, weil
der menschliche diskursiv ist, und nur durch allgemeine Begriffe erkennen
kann.}\footnote{\cite[][\S~57]{Kant:ProlegomenazueinerjedenkuenftigenMetaphysikdiealsWissenschaftwirdauftretenkoennen1977},
\cite[][IV: 355.33--356.1]{Kant:GesammelteWerke1900ff.}.}
Ein endlicher Verstand kann ohne Sinnlichkeit gar keine Erkenntnis
erwerben. Begriffe ohne Inhalt sind leer; und ihren Inhalt erhalten sie
durch den Bezug auf Gegenstände. Auf diese beziehen sie sich aber nur mittelbar
über allgemeine Merkmale, also über weitere Begriffe, die sich ebenso mittelbar auf
Gegenstände beziehen. Gäbe es nun keine Vorstellungen, die sich unmittelbar auf
ihre Gegenstände beziehen -- Anschauungen --, so käme gar kein Gegenstandsbezug
zustande.\footnote{\enquote{Auf welche Art und durch welche Mittel sich auch
immer eine Erkenntnis auf Gegenstände beziehen mag, so ist doch diejenige,
wodurch sie sich auf dieselbe[n] unmittelbar bezieht, und worauf alles Denken
als Mittel abzweckt, die \ori{Anschauung}}
\mkbibparens{\cite[][B 33]{Kant:KritikderreinenVernunft2003},
\cite[][III: 49.6--9]{Kant:GesammelteWerke1900ff.}}.} Der systematische
Zusammenhang der Begriffe untereinander und unser Denken in Begriffen verkämen
zu einem leeren \enquote{frictionless spinning in the void}, wie
\authorfullcite{McDowell:MindandWorld1994} schreibt.\footnote{\enquote{We need
to conceive this expansive spontaneity as subject to controll from outside our
thinking, on pain of representing the operations of spontaneity as a
frictionless spinning in the void} \parencite[][11]{McDowell:MindandWorld1994}.}
\name[Immanuel]{Kant} sagt, weil Gedanken ohne Inhalt
leer seien, deshalb müssten wir Begriffe sinnlich machen und ihnen den
Gegenstand in der Anschauung
beifügen.\footnote{\phantomsection\label{Fussnote:InfiniterRegressBegriffeueberMerkmale}\enquote{Gedanken
    ohne Inhalt sind leer, Anschauungen ohne Begriffe sind
    blind. Daher ist es eben so notwendig, seine Begriffe sinnlich zu
    machen, (d.\,i. ihnen den Gegenstand in der Anschauung
    beizufügen,) als seine Anschauungen sich verständlich zu machen
    (d.\,i. sie unter Begriffe zu bringen)} \mkbibparens{\cite[][B
    75]{Kant:KritikderreinenVernunft2003}, \cite[][III:
    75.14--18]{Kant:GesammelteWerke1900ff.}}. \name[Immanuel]{Kant}s
  Konzeption von Begriffen als Merkmalskomplexionen führt dabei auf
  zwei Schwierigkeiten: Zum einen scheint es, als könnten sich
  Begriffe immer nur auf weitere Begriffe, niemals aber auf
  Gegenstände beziehen. Denn wenn er sagt, dass sie sich mittels
  allgemeiner Merkmale auf Gegenstände beziehen, die selbst wiederum
  Begriffe sind, dann bezieht sich zunächst ein Begriff $ P_1 $ auf
  weitere Begriffe $ P_2^1 , P_2^2 \dots $, die sich wiederum auf
  Begriffe $P_3^1, P_3^2 \dots$ als ihre Merkmale beziehen. Der Bezug
  auf Gegenstände muss letztlich über Anschauungen geschehen, doch
  enthält die Begriffstheorie keinerlei Hinweise, welcher
  systematische Ort den Anschauungen hier zukommt. Zum anderen ist
  unklar, wie die Theorie von Begriffen als Merkmalskomplexionen sich auf
  \singlequote{höchste Gattungen} oder allgemeinste Begriffe anwenden
  lässt. Wenn $P_2$ Merkmal von $P_1$ ist, dann gilt dass alle unter
  $P_1$ fallenden Gegenstände auch unter $P_2$ fallen: $ \forall x
  (P_1(x) \supset P_2(x)) $. $P_2$ ist dann allgemeiner als $P_1$. Es
  muss aber eine oder mehrere allgemeinste Begriffe geben, auf die die
  Theorie von Begriffen als Merkmalskomplexionen \emph{a fortiori} gar
  nicht mehr anwendbar ist. Es ist hier nicht der Ort, diese
  Schwierigkeiten aufzulösen.}

Letztlich gilt dies auch für die reinen Begriffe, die -- obwohl gänzlich frei
von sinnlichem Gehalt -- ebenso nur dadurch gehaltvoll sind, dass sie sich auf
mögliche Erfahrungen beziehen und entsprechend schematisieren lassen. Die Frage
nach der realen Möglichkeit eines Begriffs wird von \name[Immanuel]{Kant} so
beantwortet, dass real möglich ist, was sich auf eine mögliche Erfahrung
bezieht.\footnote{\enquote{Ein Begriff, der eine Synthesis in sich faßt, ist
für leer zu halten, und bezieht sich auf keinen Gegenstand, wenn diese
Synthesis nicht zur Erfahrung gehört, entweder als von ihr erborgt, und dann
heißt er ein \ori{empirischer Begriff}, oder als eine solche, auf der, als
Bedingung a priori, Erfahrung überhaupt (die Form derselben) beruht, und denn
ist es ein \ori{reiner Begriff}, der dennoch zur Erfahrung gehört, weil sein
Objekt nur in dieser angetroffen werden kann}
\mkbibparens{\cite[][B 267]{Kant:KritikderreinenVernunft2003},
\cite[][III: 186.29--35]{Kant:GesammelteWerke1900ff.}}.}
Besonders deutlich wird dies unter Bezug auf die Mathematik, von der
\name[Immanuel]{Kant} in \S~22 der \titel{Kritik der reinen Vernunft} sagt, sie enthalte letztlich nur
deswegen Erkenntnisse, weil wir voraussetzen können, dass es Dinge gibt, von
denen wir \emph{empirische} Anschauungen haben, von denen die reine Anschauung
der Mathematik wiederum die Form angibt. Und dasselbe gilt dann natürlich von
den reinen
Verstandesbegriffen.\footnote{\enquote{Folglich verschaffen die reinen Verstandesbegriffe, selbst wenn sie auf
Anschauungen a priori (wie in der Mathematik) angewandt werden, nur so fern
Erkenntnis, als diese, mithin auch die Verstandesbegriffe vermittelst ihrer, auf
empirische Anschauung angewandt werden können}
\mkbibparens{\cite[][\S~22]{Kant:KritikderreinenVernunft2003},
\cite[][III: 117.22--26]{Kant:GesammelteWerke1900ff.}}.}
Der endliche Verstand kann ohne Sinnlichkeit keine objektiven Erkenntnisse
generieren, weder Begriffe noch Urteile. Auch reine Begriffe und transzendentale
Grundsätze sind nur dadurch Erkenntnisse, denen objektive Realität oder
objektive Gültigkeit zukommt, insofern sie sich auf mögliche sinnliche Erfahrungen beziehen.
Der unendliche Verstand, der selbst anschaute, unterläge einer solchen
Einschränkung nicht. Er wäre in seinem Erkennen nicht von Rezeptivität
abhängig, sondern gewährleistete aus reiner Selbsttätigkeit die
objektive Gültigkeit seiner Erkenntnis.

Der Verstand ist zunächst das Vermögen geistiger
Selbsttätigkeit.\footnote{\enquote{Selbsttätigkeit} ist bei
\name[Immanuel]{Kant} wie schon bei
\authorcite{Baumgarten:Metaphysica---Metaphysik2011} die Übersetzung des
lateinischen \enquote{spontaneitas}
\mkbibparens{\cite[vgl.][\S~15]{Kant:KritikderreinenVernunft2003}, \cite[][III:
107.7--25]{Kant:GesammelteWerke1900ff.};
\cite[][\S\,704]{Baumgarten:Metaphysica---Metaphysik2011}, \cite[][XVII:
131.26, 33]{Kant:GesammelteWerke1900ff.}}. Noch \authorcite{Wolff:Discursuspraeliminarisdephilosophiaingenere1996} übersetzt
\enquote{spontaneitas} hingegen als \enquote{Willkür}, wie er im ersten Register der
Deutschen Metaphysik vermerkt \parencite[vgl.][\pno~677
\mkbibparens{n.\,p.}]{Wolff:VernuenftigeGedanckenvonGottderWeltundderSeeledesMenschenauchallenDingenueberhauptDeutscheMetaphysik1983}.}
Als solches heißt er \enquote{oberes Erkenntnisvermögen}, welches dann auch mit
der Vernunft als dem Vermögen der Prinzipien identifiziert
wird.\footnote{\phantomsection\label{Anmerkung:ObereErkenntnisvermoegenSingularPlural}\name[Immanuel]{Kant}
kennt sowohl die Ausdrucksweise von dem einen oberen Erkenntnisvermögen, welches
Vernunft, Verstand und Urteilskraft umfasst \mkbibparens{\cite[siehe z.\,B.][BA
115]{Kant:AnthropologieinpragmatischerHinsicht1977},
\cite[][VII: 196.17--19]{Kant:GesammelteWerke1900ff.}}, als auch die
Ausdrucksweise von den oberen Erkenntnisvermögen, der zufolge jedes einzelne
dieser Vermögen als ein oberes Erkenntnisvermögen zählt
\mkbibparens{\cite[siehe z.\,B.][B 169]{Kant:KritikderreinenVernunft2003},
\cite[][III: 130.7--8]{Kant:GesammelteWerke1900ff.}}. Das obere
Erkenntnisvermögen als Ganzes nennt er mitunter \enquote{Verstand}
\mkbibparens{\cite[siehe z.\,B.][BA
115\,f.,]{Kant:AnthropologieinpragmatischerHinsicht1977} \cite[][VII:
196.17--197.3]{Kant:GesammelteWerke1900ff.}}
und manchmal \enquote{Vernunft} \mkbibparens{\cite[z.\,B.][B
863]{Kant:KritikderreinenVernunft2003}, \cite[][III:
540.28]{Kant:GesammelteWerke1900ff.}}. Hieraus erklärt sich
möglicherweise auch, warum in seiner Konzeption des Selbstdenkens -- etwa auch in dem Aufklärungsaufsatz -- von
Verstand und Vernunft teilweise synonym gesprochen zu werden scheint
\mkbibparens{\cite[siehe
etwa][13]{Bartuschat:KantueberPhilosophieundAufklaerung2009}, unter Verweis auf
\cite[][\pno~125\,ff.]{LaRocca:WasAufklaerungseinwird2004}}.} Unter
dem Titel \enquote{Verstand} wird das obere Erkenntnisvermögen als
Vermögen der Spontaneität oder Selbsttätigkeit beschrieben; unter der Überschrift
\enquote{Vernunft} als Vermögen der Erkenntnisse \emph{a priori} oder der
Autonomie.\footnote{Darauf werde ich in Kapitel
\ref{subsection:MetaphysikundAutonomie} dieser Arbeit näher eingehen.} Beides
muss nach der Systematik \name[Immanuel]{Kant}s koinzidieren, soll diese
terminologische Fixierung haltbar sein; es entspräche dann der Rede von einem
negativen und einem positiven Freiheitsbegriff oder -- noch näher -- einem
negativen und einem positiven Begriff des Selbstdenkens.\footnote{Zum negativen
und positiven Begriff des Selbstdenkens siehe Kapitel
\ref{subsection:DerBegriffdesSelbstdenkens}.} Insofern wir über ein solches
Vermögen verfügen, können wir selbständig denken und handeln.
Als endliche Wesen sind wir aber nicht nur selbständig, sondern in unserer
Selbständigkeit zugleich \emph{abhängig}. Diese Abhängigkeit betrifft die
Sinnlichkeit oder Rezeptivität, also die Tatsache, dass wir affiziert werden von
Gegenständen unserer Erkenntnis. Ein unendliches Wesen bräuchte keine äußeren
Einflüsse, denen gegenüber es sich leidend oder passiv verhielte. Es wäre sich
in seinem Denken selbst genug; daher entspricht dem Begriff eines nicht-endlichen
Verstandes der Begriff der \singlequote{Allgenugsamkeit}, den
\name[Immanuel]{Kant} als besseren Ausdruck für die göttliche Unendlichkeit
vorschlägt.\footnote{Siehe oben Seite \pageref{Allgenugsamkeit}.}

\section{Drei Vergleiche und ein Ursprung}\label{subsection:Unendlichkeiten}
Es gibt nach \name[Immanuel]{Kant} eine einheitliche Grundlage unserer
Endlichkeit: Die Abhängigkeit unseres Verstandes, der als Vermögen der
\singlequote{Spontaneität} oder \singlequote{Selbsttätigkeit} das obere
Erkenntnisvermögen ausmacht, von der Sinnlichkeit oder \emph{Rezeptivität} als
der Möglichkeit, von Dingen affiziert zu werden. Unser Verstand ist abhängig,
weil er die Gegenstände seines Erkennens nicht selbst hervorbringen kann,
sondern darauf angewiesen ist, dass sie ihm \emph{sinnlich gegeben} werden. Im
zurückliegenden Kapitel
\ref{subsection:DiskursiverVerstandundsinnlicheAnschauung} habe ich die
Konzeptionen eines in dieser Hinsicht endlichen und eines solcherart unendlichen
Verstandes herausgearbeitet. Im folgenden werde ich diese Darstellung in einigen
Punkten konkretisieren und dabei zeigen, dass dies tatsächlich die allgemeine
Grundlage der Vergleiche unseres endlichen mit einem \singlequote{anderen}
Erkenntnisvermögen ist. Dazu betrachte ich die drei Instanzen unserer
Endlichkeit, wie \name[Immanuel]{Kant} sie in der \titel{Kritik der
Urteilskraft} aufzählt, um zu zeigen, dass alle drei ihre Grundlage in dieser
allgemeinen Charakterisierung unserer Endlichkeit haben. Es gibt also eine
einheitliche Beschreibung dessen, was es heißt, ein endliches Vernunftwesen zu
sein: Es heißt, ein von Rezeptivität \emph{abhängiges} Wesen zu sein.

\begin{comment}
Bevor \name[Immanuel]{Kant} auf die jeweiligen Besonderheiten in Bezug zu den
drei oberen Erkenntnisvermögen \emph{en detail} eingeht, macht er eine
Bemerkung, die sich auf alle drei Betrachtungen erstreckt:
\begin{quote}
Man wird bald inne, daß, wo der Verstand nicht folgen kann, die Vernunft
überschwänglich wird, und in zwar gegründeten Ideen (als regulativen
Prinzipien), aber nicht objektiv gültigen Begriffen sich hervortut; der Verstand
aber, der mit ihr nicht Schritt halten kann, aber doch zur Gültigkeit für
Objekte nötig sein würde, die Gültigkeit jener Ideen der Vernunft nur auf das
Subjekt, aber doch allgemein für alle von dieser Gattung, d.\.i. auf die
Bedingung einschränke, daß nach der Natur unseres (menschlichen)
Erkenntnisvermögens, oder gar überhaupt nach dem Begriffe, \ori{den wir uns} von
dem Vermögen eines endlichen Vernünftigen Wesens überhaupt \ori{machen} können,
nicht anders als so könne und müsse gedacht werden: ohne doch zu behaupten, daß
der Grund eines solchen Urteils im Objekt
liegt.\footnote{\cite[][\S~76]{Kant:KritikderUrteilskraft2009},
\cite[][V: 401.14--26]{Kant:GesammelteWerke1900ff.}.}
\end{quote}
Es gibt Begriffe und Urteile, die ihren Ursprung nicht in ihren Gegenständen
haben, sondern in dem erkennenden Subjekt, und die dennoch nicht beliebig sind.
Um zu zeigen, dass der Grund für solche Begriffe und Urteile nicht in ihren
Objekten liegt, muss \name[Immanuel]{Kant} einen denkbaren Vergleichsverstand
konzipieren, der sich von jedem endlichen qualitativ Verstand unterscheidet. Der
Grund liegt aber in einer Besonderheit unseres Verstandes, die ihn als
\emph{endlichen} Verstand notwendigerweise auszeichnet; handelte es sich um eine
zufällige Besonderheit, die wir empirisch bei uns feststellten, dann wäre es gar
kein hinreichend begründetes Urteil. Wäre ein Urteil so beschaffen, dass es für
jeden denkbaren Verstand zwingend wäre, dann müssten wir eingestehen, dass es
eine objektive Eigenschaft der Gegenstände und nicht eine notwendige subjektive
Beschaffenheit unseres Erkenntnisvermögens betrifft.
\end{comment}

Meine Position lässt sich dabei unter Berücksichtigung der Tatsache, dass
\name[Immanuel]{Kant} einmal von dem Verstand (im weiteren Sinne) als \emph{dem}
oberen Erkenntnisvermögen, daneben aber auch von Verstand (im engeren Sinne),
Vernunft und Urteilskraft als \emph{den} oberen Erkenntnisvermögen
spricht,\footnote{Siehe oben Anm.
\ref{Anmerkung:ObereErkenntnisvermoegenSingularPlural} auf
S.~\pageref{Anmerkung:ObereErkenntnisvermoegenSingularPlural}.} folgendermaßen
artikulieren: Die Dreiteilung in \S~76 der \titel{Kritik der Urteilskraft} nennt
drei Besonderheiten unseres Verstandes als des gesamten oberen Erkenntnisvermögens
\emph{in Ansehung} jeweils eines der drei besonderen oberen Erkenntnisvermögen 
Verstand, Vernunft und Urteilskraft. Es bleibt dabei dieselbe grundlegende
Charakteristik eines Verstandes im weiteren Sinne als eines endlichen Vermögens
(als auf Rezeptivität angewiesene Spontaneität), die sich aber in Ansehung der
drei oberen Erkenntnisvermögen unterschiedlich äußert.

Ich werde im folgenden auch auf
\authorcite{Foerster:Die25JahrederPhilosophie2011}s Behauptung eingehen,
\name[Immanuel]{Kant} verfüge über jeweils mehrere Konzeptionen einer
intellektuellen Anschauung und eines intuitiven Verstandes.\footnote{Siehe
Kapitel \ref{subsection:EineMoeglicheVieldeutigkeit}.} Da
\authorcite{Foerster:Die25JahrederPhilosophie2011} die intellektuelle Anschauung
primär der Unterscheidung von Wirklichkeit und Möglichkeit zuordnet, werde ich
ihre vermeintliche Mehrdeutigkeit in Kapitel
\ref{subsubsection:UnterscheidungvonDenkenundErkennen} besprechen. Die
Mehrdeutigkeit des Ausdruck \enquote{intuitiver Verstand} ist Thema in Kapitel
\ref{subsection:IntuitiverVerstandunddasSynthetischAllgemeine}, da
\authorcite{Foerster:Die25JahrederPhilosophie2011} den intuitiven Verstand
primär den Überlegungen in \S~77 der \titel{Kritik der Urteilskraft} zuordnet.
Im letzten Teil komme ich auf die Endlichkeit des \emph{Willens} zu sprechen
(Kapitel \ref{subsubsection:DieEndlichkeitdesWillens}). Dass diese eine
Sonderform der Endlichkeit des Verstandes darstellt, mag \emph{prima facie}
überraschen; es wird sich aber aus \name[Immanuel]{Kant}s Verständnis beider
Vermögen leicht erhellen lassen.

\subsection{Die Unterscheidung von Denken und
Erkennen}\label{subsubsection:UnterscheidungvonDenkenundErkennen}

Die erste in \S~76 der \titel{Kritik der Urteilskraft} beschriebene Besonderheit
unseres Erkenntnisvermögens betrifft die Unterscheidung von Möglichkeit und
Wirklichkeit.\footnote{In der \titel{Kritik der reinen Vernunft} entspricht dem die
Unterscheidung von \emph{Denken} und \emph{Erkennen}:
\enquote{Sich einen Gegenstand \ori{denken}, und einen Gegenstand \ori{erkennen}, ist
also nicht einerlei. Zum Erkenntnisse gehören nämlich zwei Stücke: erstlich der
Begriff, dadurch überhaupt ein Gegenstand gedacht wird (die Kategorie), und
zweitens die Anschauung, dadurch er gegeben
wird} \mkbibparens{\cite[][\S~22]{Kant:KritikderreinenVernunft2003},
\cite[][III: 116.34--117.2]{Kant:GesammelteWerke1900ff.}}.
Diese Unterscheidung fundiert auch \name[Immanuel]{Kant}s Begriffe von Glauben
und Wissen.} Weil unser Erkenntnisvermögen über \enquote{Verstand für
Begriffe und sinnliche Anschauung für Objekte, die ihnen korrespondieren,} als
\enquote{ganz heterogene
Stücke}\footnote{\cite[][\S~76]{Kant:KritikderUrteilskraft2009}, \cite[][V:
401.34--36]{Kant:GesammelteWerke1900ff.}.} verfügt, die beide zur Ausübung
unserer Erkenntnisvermögen notwendig sind, können und müssen wir zwischen dem
bloß Möglichen und dem Wirklichen unterscheiden.\footnote{\enquote{Es
    ist dem menschlichen Verstande unumgänglich notwendig, Möglichkeit
    und Wirklichkeit der Dinge zu unterscheiden. Der Grund davon liegt
    im Subjekte und der Natur seiner Erkenntnisvermögen. Denn wären zu
    dieser ihrer Ausübung nicht zwei ganz heterogene Stücke, Verstand
    für Begriffe und sinnliche Anschauung für Objekte, die ihnen
    korrespondieren, erforderlich, so würde es keine solche
    Unterscheidung (zwischen dem Möglichen und dem Wirklichen) geben}
  \mkbibparens{\cite[][\S~76]{Kant:KritikderUrteilskraft2009},
    \cite[][V: 401.31--402.1]{Kant:GesammelteWerke1900ff.}}.}
\enquote{Wäre nämlich unser Verstand anschauend, so hätte er keine Gegenstände als das
Wirkliche.}\footnote{\cite[][\S~76]{Kant:KritikderUrteilskraft2009}, \cite[][V:
402.1--2]{Kant:GesammelteWerke1900ff.}.} Die Unterscheidung von
Möglichkeit und Wirklichkeit ist also dem Dualismus der
Grundkräfte unseres Gemüts und der Notwendigkeit der Rezeptivität für unser
Erkennen geschuldet.

Nun ist es zunächst der Besonderheit unseres Erkenntnisvermögens geschuldet,
dass wir überhaupt über Kategorien und damit über die Begriffe der
Möglichkeit und der Wirklichkeit, die ja selbst zu den Kategorien
gehören, verfügen. Aber das kann nicht  der Hintergrund sein, wenn
\name[Immanuel]{Kant} feststellt, dass die Unterscheidung \emph{von 
Wirklichkeit und Möglichkeit} Folge der Endlichkeit unseres Verstandes
ist.\footnote{Ein Problem  dieser Überlegungen liegt darin, dass
\name[Immanuel]{Kant} sagt, die Kategorien seien ohnehin nur Begriffe des
endlichen Verstandes; ein unendlicher Verstand -- ein Verstand, der selbst anschaut --
verfügte nicht über sie. Auch die übrigen Kategorien gelten nur aus der Sicht
eines endlichen Verstandes, wie \name[Immanuel]{Kant} in der \titel{Kritik der reinen Vernunft} erläutert:
\enquote{Sie sind nur Regeln für einen Verstand, dessen ganzes Vermögen im Denken
besteht, d.\,i. in der Handlung, die Synthesis des Mannigfaltigen, welches ihm
anderweitig in der Anschauung gegeben worden, zur Einheit der Apperzeption zu
bringen, der also für sich gar nichts erkennt, sondern nur den Stoff zum
Erkenntnis, die Anschauung, die ihm durchs Objekt gegeben werden muß, verbindet
und ordnet} \mkbibparens{\cite[][\S~21]{Kant:KritikderreinenVernunft2003},
\cite[][III: 116.18--23]{Kant:GesammelteWerke1900ff.}}. Dann aber scheint die
Bemerkung zu den Kategorien der Modalität gänzlich überflüssig zu sein. Siehe
hierzu auch \cite[][\pno~356\,f.]{Leech:MakingModalDistinctions2014}.}
Stattdessen informiert \name[Immanuel]{Kant} über eine Besonderheit der Kategorien der
Modalität, die diese von den anderen neun Kategorien unterscheidet. In der
\titel{Kritik der reinen Vernunft} lesen wir:
\begin{quote}
Die Kategorien der Modalität haben das Besondere an sich: daß sie den Begriff,
dem sie als Prädikate beigefüget werden, als Bestimmung des Objekts nicht im
mindesten vermehren, sondern nur das Verhältnis zum Erkenntnisvermögen
ausdrücken. {\punkt} Hiedurch werden keine Bestimmungen mehr im Objekte selbst
gedacht, sondern es frägt sich nur, wie es sich (samt allen seinen Bestimmungen)
zum Verstande und dessen empirischen Gebrauche, zur empirischen Urteilskraft,
und zur Vernunft (in ihrer Anwendung auf Erfahrung)
verhalte?\footnote{\cite[][B 266]{Kant:KritikderreinenVernunft2003},
\cite[][III: 186.4--14]{Kant:GesammelteWerke1900ff.}.}
\end{quote}
Die übrigen neun Kategorien beschreiben Eigenschaften des Objekts, sie drücken
beispielsweise dessen Substantialität (Kategorie der Substanz) aus oder
artikulieren eine objektive Verbindung zwischen Gegenständen oder Ereignissen
(Kategorien der Ursache und der Wirkung). Was sie gegenüber den Kategorien der
Modalität auszeichnet ist folgendes: Sie sind zwar reine Begriffe -- sie sind also
nicht der Erfahrung entlehnt, sondern entstammen der Selbsttätigkeit
(Spontaneität) des Verstandes --, aber sie drücken doch objektive Eigenschaften \emph{des
Gegenstandes} aus.\footnote{Aus diesem Grund behauptet
\authorcite{Hegel:GesammelteWerke}, dass es in \name[Immanuel]{Kant}s
Darstellung nur neun statt zwölf Kategorien gebe:
\enquote{Die Identität des Subjekts und Objekts schränkt sich auf zwölf oder
vielmehr nur auf neun reine Denkthätigkeiten ein -- denn die Modalität giebt
keine wahrhaft objektive Bestimmung, es besteht in ihr wesentlich die
Nichtidentität des Subjekts und Objekts} \mkbibparens{\cite[][IV: 6.
8--11]{Hegel:GesammelteWerke}}.} Wenn wir sagen, dass $A$ die Ursache von $B$
sei, dann sagt dies etwas über $A$ und $B$ aus, nicht jedoch über uns
als erkennende Subjekte oder über unser Verhältnis zu $A$ und $B$. Bei den Kategorien der Modalität
verhalte es sich nun anders, sie bestimmen laut \name[Immanuel]{Kant} nicht das
Objekt, sondern unser Verhältnis zu dem Objekt. Wenn wir also sagen, $A$ sei
möglich (oder wirklich), dann sagen wir gar nichts über $A$ aus, sondern
darüber, in welcher Beziehung $A$ zu uns als erkennenden Subjekten steht. Die
Erläuterung in der \titel{Analytik der Grundsätze} gibt nähere Auskunft, warum
dies so ist:
\begin{quote}
\begin{nummerierung}
\item Was mit den formalen Bedingungen der Erfahrung (der Anschauung und den
Begriffen nach) übereinkommt, ist \ori{möglich}.
\item Was mit den materialen Bedingungen der Erfahrung (der Empfindung)
zusammenhängt, ist \ori{wirklich}.
\item Dessen Zusammenhang mit dem Wirklichen nach allgemeinen Bedingungen der
Erfahrung bestimmt ist, ist (existiert)
\ori{notwendig}.\footnote{\cite[][B 265\,f.,]{Kant:KritikderreinenVernunft2003}
\cite[][III: 185.22--186.2]{Kant:GesammelteWerke1900ff.}.}
\end{nummerierung}
\end{quote}
Sowohl das Wirkliche als auch das Mögliche verträgt sich zunächst mit den formalen Bedingungen der
Erfahrung, also mit den metaphysischen Erkenntnissen, die wir von der Natur
haben.\footnote{Ich fokussiere hier die Kategorien Wirklichkeit und Möglichkeit unter
Auslassung der Kategorie der Notwendigkeit. Diese Entscheidung
gründet in der Tatsache, dass \name[Immanuel]{Kant} in der Rekapitulation in
\S~76 der \titel{Kritik der Urteilskraft} nicht auf die Notwendigkeit eingeht.
Sie lässt sich nicht durch Verweis auf eine Definierbar der Notwendigkeit durch
die anderen Modalbegriffe begründen, da \name[Immanuel]{Kant} alle drei für
jeweils ursprünglich hält. \cite[Vgl.][196]{Poser:DieStufenderModalitaet1981},
sowie
\cite[][42--45]{Kamlah:KantsAntwortaufHumeundeinelinguistischeAnalyseseinerModalbegriffe2009}.
\authorfullcite{Poser:DieStufenderModalitaet1981} verweist bei
\name[Immanuel]{Kant} insb. auf \cite[][B 111]{Kant:KritikderreinenVernunft2003},
\cite[][III: 96.8--17]{Kant:GesammelteWerke1900ff.}.} Schon
\authorcite{Leibniz:Meditationesdecognitioneveritateetideis1999} ringt
in den \titel{Meditationes de veritate, cognitione et ideis} mit der
Frage, wann etwas (\emph{realiter} und nicht nur logisch) möglich ist. Ihm folgen
\authorcite{Wolff:Discursuspraeliminarisdephilosophiaingenere1996} und andere
schließlich darin, die reale Möglichkeit letztlich in der logischen Möglichkeit
fundieren zu wollen; und gerade hierin weicht \name[Immanuel]{Kant} von der
Schule \authorcite{Wolff:Discursuspraeliminarisdephilosophiaingenere1996}s
ab,\footnote{\cite[Vgl.][195]{Poser:DieStufenderModalitaet1981}.} wenn er
schreibt:
\begin{quote}
Das Postulat der \ori{Möglichkeit} der Dinge fordert also, daß der Begriff
derselben mit den formalen Bedingungen einer Erfahrung überhaupt
zusammenstimme.\footnote{\cite[][B 267]{Kant:KritikderreinenVernunft2003},
\cite[][III: 186.25--27]{Kant:GesammelteWerke1900ff.}.}
\end{quote}
Die formalen Bedingungen der Erfahrung entstammen der Konstitution des Subjekts,
sie sind daher Thema der Transzendentalphilosophie. Während
\authorcite{Leibniz:Meditationesdecognitioneveritateetideis1999} behauptet, dass die reale
Möglichkeit auf der Grundlage einer vollständigen Analyse eines Begriffs und der
darauf fundierten Einsicht in seine Widerspruchsfreiheit erkannt werden
kann,\footnote{\cite[Vgl.][589]{Leibniz:Meditationesdecognitioneveritateetideis1999}.}
fordert \name[Immanuel]{Kant}: Eine reale Möglichkeit ist nur, was Gegenstand in
einer möglichen Erfahrung sein kann; und Gegenstand einer möglichen Erfahrung
ist, was den formalen Bestimmungen entspricht, die uns die transzendentale
Ästhetik und die transzendentale Logik aufdecken. Insofern ist die reale
Möglichkeit an die (formalen) Bedingungen unserer sinnlichen Wahrnehmung
gebunden, an Raum, Zeit und die Kategorien sowie Grundsätze des reinen
Verstandes. Wir erkennen sie \emph{a priori}.\footnote{Natürlich
können wir außerdem auch aufgrund einer tatsächlichen sinnlichen Wahrnehmung auf
die Möglichkeit des bereits als wirklich erkannten schließen.}

Das Wirkliche hat darüber hinaus tatsächlich eine Verbindung zu den
\singlequote{materialen} Bedingungen der Erfahrung, die \name[Immanuel]{Kant} hier \emph{Empfindung}
nennt und die offensichtlich auf unsere Rezeptivität verweisen:
\begin{quote}
Das Postulat, die \ori{Wirklichkeit} der Dinge zu erkennen, fordert
\ori{Wahrnehmung}, mithin Empfindung, deren man sich bewußt ist, zwar nicht
eben unmittelbar, von dem Gegenstande selbst, dessen Dasein erkannt werden
soll, aber doch Zusammenhang desselben mit irgend einer wirklichen
Wahrnehmung, nach den Analogien der Erfahrung, welche alle reale Verknüpfung
in einer Erfahrung überhaupt darlegen.\footnote{\cite[][B
272]{Kant:KritikderreinenVernunft2003}, \cite[][III:
189.23--28]{Kant:GesammelteWerke1900ff.}.}
\end{quote}
Zum einen erkennen wir etwas als wirklich, wenn wir es sinnlich wahrnehmen. Den
Baum vor dem Fenster, den Lärm der Straße oder den Duft des Apfelkuchens
erkennen wir auf diese Art als wirklich. Zum anderen erkennen wir als wirklich,
was wir zwar nicht direkt sinnlich wahrnehmen können, was aber in einer kausalen
Verbindung zu etwas von uns sinnlich wahrgenommenem steht. Atome, Moleküle,
Elektronen und viele andere Dinge erkennen wir in diesem Sinne zwar nicht durch
direkte sinnliche Wahrnehmung als wirklich, aber doch indirekt durch
beobachtbare Wirkungen. Wir können sagen, dass wir indirekt die Wirklichkeit
erkennen, wenn wir \emph{inferentielles} Wissen generieren; direkt erkennen wir
dann, wenn es sich um eine nicht-inferentielle Erkenntnis handelt.\footnote{Ob
wir Teilchen in der Nebelkammer nun direkt oder indirekt als wirklich erkennen, solche Fragen sollen hier offen bleiben.
Ein ähnlicher Topos findet sich bei
\authorfullcite{Wolff:Discursuspraeliminarisdephilosophiaingenere1996} (siehe
Anm.~\ref{Anmerkung:CognitioArcanaCognitioCommunis} auf
S.~\pageref{Anmerkung:CognitioArcanaCognitioCommunis}). Ebenso setze ich mich
hier nicht mit der Frage auseinander, ob es ausreicht, den Begriff der
Wirklichkeit über eine \emph{tatsächliche} kausale Verbindung zum erkennenden
Subjekt zu erläutern, oder ob stärker berücksichtigt werden muss, dass auch
Dinge wirklich sein können, die in keiner kausalen Verbindung zu uns stehen
(etwa Ereignisse, die jetzt in sehr großer Entfernung von uns geschehen).
\name[Immanuel]{Kant} spricht jedenfalls nicht davon, dass nur wirklich
\emph{ist}, was uns affiziert, sondern dass wir nur dieses als wirklich
\emph{erkennen}.} Was wir so indirekt erkennen, das erkennen wir vermittelt über
Merkmale, die als Erkenntnisgründe dienen. Aber dies setzt doch voraus, dass es
etwas gibt, das wir unmittelbar erkennen. Und diese unmittelbare -- in diesem
Sinne \singlequote{intuitive} -- Erkenntnis kann nur sinnlich sein. Wir erkennen
die Wirklichkeit nicht aus reiner Vernunft.

Die Unterscheidung von Möglichkeit und Wirklichkeit setzt voraus, dass es mit
Sinnlichkeit und Verstand zwei Stämme menschlicher Erkenntnis gibt.
Der Verstand zeichnet für die formalen Bedingungen der Erkenntnis verantwortlich, die
Sinnlichkeit für die materialen. Die Unterscheidung von Wirklichkeit und
Möglichkeit artikuliert gerade die Differenz zwischen diesen Quellen, insofern
etwas möglich ist, sobald die formalen Bedingungen erfüllt sind. Über die
Erfüllung der materialen Bedingungen ist damit noch nichts ausgemacht. Wäre unser
Verstand anschauend, könnte er die materialen Bedingungen unserer Erkenntnis
selbst garantieren. Damit artikulieren die Kategorien der Modalität
keine Eigenschaften der Gegenstände selbst, sondern ihr Verhältnis zu den beiden
\singlequote{Grundquellen unseres Gemüts}. Das ist die besondere Beziehung, die
sie zu der besonderen Beschaffenheit unseres Erkenntnisvermögens haben.

\authorfullcite{Foerster:Die25JahrederPhilosophie2011} stellt die Unterscheidung
von Möglichkeit und Wirklichkeit als Folge der Besonderheit unserer
\emph{Anschauung} und nicht unseres Verstandes
dar.\footnote{\cite[Vgl.][153]{Foerster:Die25JahrederPhilosophie2011}.} Nach
meiner Interpretation ist es hingegen irrelevant, ob wir etwas der Besonderheit
unseres Verstandes oder unserer Anschauung zuschreiben, denn intellektuelle
Anschauungen sind Erkenntnisse eines anschauenden Verstandes. Wichtiger
ist daher \authorcite{Foerster:Die25JahrederPhilosophie2011}s zweiter Schritt,
in welchem er verschiedene Arten intellektueller Anschauung differenziert, denen
unterschiedliche Arten eines anschauenden Verstandes zuzuordnen wären.
Der Einheit von Möglichkeit und Wirklichkeit entspräche eine produktive
Anschauung und dieser wiederum ein Verstand, der als Weltursache zu denken wäre.
Aber in der \titel{Kritik der reinen Vernunft} gibt es eben auch die Konzeption
einer im Vergleich zur unsrigen alternativen Anschauung, mit deren Hilfe die Unterscheidung von Dingen an sich
und Erscheinungen artikuliert wird. Dass wir die Dinge nur als Erscheinungen
erkennen und dass wir zwischen Möglichkeit und Wirklichkeit der Dinge in unserer
Erfahrungserkenntnis unterscheiden, das sind aber allem Anschein nach
unterschiedliche Themen. Und so ließe sich vermuten, dass auch
unterschiedliche Formen unserer Endlichkeit darin zur Sprache kommen.

Nach \authorcite{Foerster:Die25JahrederPhilosophie2011} ist es die
\emph{produktive} intellektuelle Anschauung, die in der Deduktion der Kategorien
bemüht wird und für die in den \singlequote{Postulaten des empirischen Denkens
überhaupt} und in \S~76 der \titel{Kritik der Urteilskraft} Wirklichkeit und
Möglichkeit in eins fallen \emph{respective} diese Unterscheidung verschwindet.
Eine andere Form der intellektuellen Anschauung werde hingegen im Anhang zur
transzendentalen Analytik der \titel{Kritik der reinen Vernunft} und in \S~77
der \titel{Kritik der Urteilskraft} herangezogen, welche nicht den Bedingungen
unserer Sinnlichkeit unterworfen sei und daher die Dinge an sich
erkenne, \emph{ohne} dass sie die Gegenstände ihres Erkennens
hervorbringt, also produktiv ist.\footnote{Siehe oben, Kapitel
\ref{subsection:EineMoeglicheVieldeutigkeit}.}

Ich möchte im folgenden darlegen, dass es sich bei der Anschauung eines
intuitiven Verstandes generell um eine produktive Anschauung handelt und dass die
Konzeption einer nicht produktiven, sondern \singlequote{übersinnlichen}
intellektuellen Anschauung gänzlich unverständlich ist. Unsere Anschauung heißt
sinnlich, \enquote{weil sie nicht ursprünglich, d.\,i. eine solche ist, durch
die selbst das Dasein des Objektes der Anschauung gegeben wird {\punkt}, sondern
von dem Dasein des Objektes abhängig, mithin nur dadurch, daß die
Vorstellungsfähigkeit des Subjekts durch dasselbe affiziert wird, möglich
ist.}\footnote{\cite[][B 72]{Kant:KritikderreinenVernunft2003}, \cite[][III:
72.24--28]{Kant:GesammelteWerke1900ff.}.} Eine Anschauung heißt also
ursprünglich, wenn das \emph{Dasein} des Gegenstandes durch sie gegeben wird.
Dies heißt allem Anschein nach, dass eine Anschauung als ursprünglich gilt, wenn
sie \emph{produktiv} ist, also den Gegenstand hervorbringt. Deswegen kommt eine
solche Anschauung auch nur Gott, niemals aber einem endlichen Wesen zu:
\begin{quote}
Es ist auch nicht nötig, daß wir die Anschauungsart in Raum und Zeit auf die
Sinnlichkeit des Menschen einschränken; es mag sein, daß alles endliche denkende
Wesen hierin mit dem Menschen notwendig übereinkommen müsse, (wiewohl wir dieses
nicht entscheiden können,) so hört sie um dieser Allgemeingültigkeit willen doch
nicht auf Sinnlichkeit zu sein, eben darum, weil sie abgeleitet (intuitus
derivatus), nicht ursprünglich (intuitus originarius), mithin nicht
intellektuelle Anschauung ist, als welche aus dem eben angeführten Grunde allein
dem Urwesen, niemals aber einem, seinem Dasein sowohl als seiner Anschauung nach
(die sein Dasein in Beziehung auf gegebene Objekte bestimmt), abhängigen Wesen
zuzukommen scheint[.]\footnote{\cite[][B 72]{Kant:KritikderreinenVernunft2003},
\cite[][III: 72.29--73.2]{Kant:GesammelteWerke1900ff.}. Schon in seiner
Inauguraldissertation unterscheidet er in ähnlicher Manier unsere Anschauung als
leidend von einer urbildlichen göttlichen Anschauung: \enquote{\ori{Intuitus}
nempe mentis nostrae semper est \ori{passivus}; adeoque eatenus tantum, quatenus
aliquid sensus nostros afficere potest, possibilis. Divinus autem intuitus, qui
obiectorum est principium, non principiatum, cum sit independens, est archetypus
et propterea perfecte intellectualis}
\mkbibparens{\cite[][\S~10]{Kant:Demundisensibilisatqueintelligibilisformaetprincipiis1968},
\cite[][II: 396.31--397.4]{Kant:GesammelteWerke1900ff.}}.}
\end{quote}
Nach diesem Zitat sind unterschiedliche Formen sinnlicher Anschauung möglich,
die von der Form, die uns Menschen eigen ist, abweichen können. Aber eine
Anschauung, die nicht sinnlich ist, sei doch nur als ursprüngliche Anschauung
denkbar, die ausschließlich Gott zugeschrieben werden könne.
\name[Immanuel]{Kant} sagt damit, dass es nur eine einzige Konzeption einer
nicht-sinnlichen Anschauung geben könne: die ursprüngliche, also produktive
Anschauung Gottes.


Doch zunächst ist zu konzidieren, dass es tatsächlich Textbelege für die
gegenteilige Interpretation gibt. Worauf
\authorcite{Foerster:Die25JahrederPhilosophie2011} sich bezieht ist die
nicht-sinnliche Anschauung, die laut \titel{Phaenomena und Noumena}-Kapitel am
Begriff eines \emph{Noumenon} in positiver Bedeutung beteiligt ist:
\begin{quote}
Wenn wir unter Noumenon ein Ding verstehen, \ori{so fern es nicht Objekt
unserer sinnlichen Anschauung ist}, indem wir von unserer Anschauungsart
desselben abstrahieren; so ist dieses ein Noumenon im \ori{negativen} Verstande.
Verstehen wir aber darunter ein Objekt einer \ori{nicht-sinnlichen Anschauung},
so nehmen wir eine besondere Anschauungsart an, nämlich die intellektuelle, die
aber nicht die unsrige ist, von welcher wir auch die Möglichkeit nicht einsehen
können, und das wäre das Noumenon in \ori{positiver}
Bedeutung.\footnote{\cite[][B 307]{Kant:KritikderreinenVernunft2003},
\cite[][III: 209.32--210.2]{Kant:GesammelteWerke1900ff.}.}
\end{quote}
Die Lehre von der Sinnlichkeit, die die Erscheinungen von den Dingen an sich
abgrenzt, rekurriert nun lediglich auf \emph{Noumena} in negativer
Bedeutung.\footnote{\cite[Vgl.][B 307--309]{Kant:KritikderreinenVernunft2003},
\cite[][III: 210.3--34]{Kant:GesammelteWerke1900ff.}.} Um nun auch nur zu
beweisen, dass ein solches \emph{Noumenon} in positiver Bedeutung \emph{möglich} ist,
bedürfte es einer Anschauung, die nicht sinnlich und \emph{a fortiori}
intellektuell ist. Eine solche Anschauung scheint nicht produktiv sein zu
müssen; sie gäbe einem zugehörigen (anschauenden) Verstand lediglich ein
Mannigfaltiges, ohne dafür der Rezeptivität zu bedürfen.


\name[Immanuel]{Kant} spricht selbst von
anderen denkbaren Formen der Anschauung, auf die ebenso Kategorien anwendbar
wären, insofern sie uns ein Mannigfaltiges darböten, das erst zu synthetisieren
wäre.\footnote{\cite[Vgl.][\S~23]{Kant:KritikderreinenVernunft2003},
\cite[][III: 118.7--10]{Kant:GesammelteWerke1900ff.}.} Eine solche Anschauung
scheint daher ebenso wie unsere menschliche Anschauung ein Mannigfaltiges zu
geben und keineswegs produktiv zu sein. Allerdings handelt es sich nach seiner
Auskunft dabei eben um eine andere Form \emph{sinnlicher} Anschauung. Zu Beginn
der transzendentalen Ästhetik findet sich jedoch eine Stelle, nach der in einer
intellektuellen Anschauung Gegenstände (nicht-sinnlich) \emph{gegeben} werden
können:
\begin{quote}
Auf welche Art und durch welche Mittel sich auch immer eine Erkenntnis auf
Gegenstände beziehen mag, so ist doch diejenige, wodurch sie sich auf dieselbe
unmittelbar bezieht, und worauf alles Denken als Mittel abzweckt, die
\ori{Anschauung}. Diese findet aber nur statt, so fern uns der Gegenstand
gegeben wird; dieses aber ist wiederum, uns Menschen wenigstens, nur dadurch
möglich, daß er das Gemüt auf gewisse Weise
affiziere.\footnote{\cite[][B 33]{Kant:KritikderreinenVernunft2003},
\cite[][III: 49.6--11]{Kant:GesammelteWerke1900ff.}. Der Einschub \enquote{uns
Menschen wenigstens} ist ein Zusatz der Auflage von 1787.}
\end{quote}
Dieses Zitat enthält zwei wichtige Aussagen: (i) Dass uns durch sie ein
Gegenstand \emph{gegeben} wird, gehört zur Anschauung \emph{per se} (also unabhängig
davon, ob es sich um eine sinnliche oder eine intellektuelle Anschauung
handelt). (ii)  Uns Menschen kann ein Gegenstand nur dadurch gegeben werden,
dass er uns affiziert, also nur mittels unserer Sinnlichkeit. Demnach ist denkbar, dass
einem Wesen ein Gegenstand durch intellektuelle Anschauung gegeben wird. Also --
so scheint es -- kann es eine intellektuelle Anschauung geben, die ich oben
\singlequote{übersinnlich} nannte, also eine solche, durch die ein Gegenstand
nicht hervorgebracht, sondern auf nicht-sinnliche Weise gegeben wird.


Dass eine Anschauung sinnlich ist, heißt, dass sie dadurch entsteht, dass
unser Erkenntnisvermögen von etwas affiziert wird. Während im Fall der
ursprünglichen Anschauung der Gegenstand der Anschauung seinem Dasein nach von
der Anschauung abhängt, hängt die sinnliche Anschauung von dem durch sie
repräsentierten Gegenstand ab. Die wichtige Frage ist nun, was mit einer
Anschauung gemeint sein soll, die weder sinnlich, noch produktiv, sondern
\singlequote{übersinnlich} ist, und ob es einen Begriff einer solchen
übersinnlichen Anschauung im Werk \name[Immanuel]{Kant}s
gibt.\footnote{Dass es ein drittes neben sinnlicher und übersinnlicher
  Anschauung gibt, behauptet auch
  \authorfullcite{Baum:DeduktionundBeweisinKantsTranszendentalphilosophie1986}, 
der nicht wie \authorcite{Foerster:Die25JahrederPhilosophie2011} und
\authorcite{Prien:KantsLogikderBegriffe2006} zwischen verschiedenen Formen
intellektueller Anschauung differenziert, sondern diese mit der produktiven
Anschauung identifiziert und der \singlequote{übersinnlichen} Anschauung
gegenüberstellt -- freilich ohne zu sagen, wie eine solche Anschauung denkbar
sein soll: \enquote{Diese Anschauung ist bei uns Menschen nicht intellektuell,
könnte aber immer noch entweder sinnlich oder auf andere als auf intellektuelle
Weise nicht-sinnlich sein, denn es besteht kein Widerspruch zwischen der Leugnung
einer intellektuellen Anschauung und der gegen sinnliche und nicht-sinnliche
Anschauung indifferenten Verstandesdefinition, weil zwischen sinnlicher und
intellektueller Anschauung kein kontradiktorischer, sondern ein konträrer
Gegensatz vorliegt}
\parencite[][83--85]{Baum:DeduktionundBeweisinKantsTranszendentalphilosophie1986}.
Mir scheinen \enquote{sinnlich} und \enquote{intellektuell} kontradiktorische
Gegensätze zu bezeichnen, insofern sie den beiden Erkenntnisquellen
\emph{Rezeptivität} und \emph{Spontaneität} entsprechen, wie ich in Kapitel
\ref{subsubsection:BegriffderDiskursivitaet} weiter ausgeführt habe.} 
Rezeptive Vermögen nennt \name[Immanuel]{Kant} Sinnlichkeit, unabhängig davon,
ob sie \singlequote{unserer} Sinnlichkeit
entsprechen.\footnote{\cite[Vgl.][B 75]{Kant:KritikderreinenVernunft2003},
\cite[][III: 75.5--7]{Kant:GesammelteWerke1900ff.}, und ausführlicher oben in
Kapitel \ref{subsection:VerstandundRezeptivitaet}.} Vielleicht ist ein Wesen denkbar,
das ebenso über Sinnlichkeit verfügt, deren Form aber nicht der unsrigen
entspricht und nicht auf Zeit und Raum beruht.\footnote{Dafür, dass Zeit und
Raum die Formen unserer Anschauung sind, können keine weiteren Gründe angegeben
werden, es ist folglich kontingent
\mkbibparens{\cite[vgl.][\S~21]{Kant:KritikderreinenVernunft2003}, \cite[][III:
116.23--29]{Kant:GesammelteWerke1900ff.}}.} Zu beachten ist, dass eine
Anschauung in \name[Immanuel]{Kant}s Sprachgebrauch auch
dann als sinnlich gilt, wenn sie nicht den Formen \emph{unserer} Sinnlichkeit
(Zeit und Raum) unterliegt. Es reicht, dass das Mannigfaltige dadurch gegeben
wird, dass es uns in irgendeiner Art und Weise \emph{affiziert}. Was es
heißen soll, dass es weder sinnlich gegeben, noch von uns hervorgebracht wird,
scheint mir mehr als fraglich zu sein. Es müsste heißen, dass weder
die Anschauung von dem Dasein des Gegenstandes noch dessen Dasein von
der Anschauung abhängt. Dennoch soll es freilich -- so ist wenigstens
anzunehmen -- eine notwendige Verbindung zwischen beiden geben. Auch nach
\authorfullcite{Duesing:NaturteleologieundMetaphysikbeiKantundHegel1990}
folgt daher aus der Spontaneität einer (intellektuellen) Anschauung ihre
Produktivität.\footnote{\cite[Vgl.][144]{Duesing:NaturteleologieundMetaphysikbeiKantundHegel1990}.}
\name[Immanuel]{Kant} selbst sagt, es sei kaum einsehbar, wie sich
Vorstellungen anders auf Gegenstände beziehen können, als dadurch, dass (a) der
Gegenstand die Vorstellung möglich macht, indem er das Subjekt affiziert, oder
(b) die Vorstellung den Gegenstand möglich macht, indem sie ihn
(b\textsubscript{1}) wie im Falle von Handlungen hervorbringt oder
(b\textsubscript{2}) indem sie ihn \emph{als Gegenstand} möglich macht, wie die die Kategorien
leisten.\footnote{\cite[Vgl.][\S~14]{Kant:KritikderreinenVernunft2003},
\cite[][III: 104.5--17]{Kant:GesammelteWerke1900ff.}. Siehe auch den Brief an
Markus \name[Marcus]{Herz} vom 21. Februar 1772
\parencite[][X: 124\,f.]{Kant:GesammelteWerke1900ff.} sowie meine Überlegungen
in Kap. \ref{subsection:MetaphysikundAutonomie} ab S.
\pageref{AutonomiedesVerstandes}.} Wenn wir den Sonderfall (b\textsubscript{2})
einmal außen vor lassen, bleibt nur, dass eine Anschauung sich entweder sinnlich
oder produktiv auf einen Gegenstand beziehen kann. In einer Anmerkung in seinem
Handexemplar der \titel{Metaphysica}
\authorcite{Baumgarten:Metaphysica---Metaphysik2011}s bemerkt \name[Immanuel]{Kant} ebenfalls, dass
nur Gott ein intuitiver Verstand zugeschrieben werden könne, weil dessen
Erkenntnisart nicht anders als produktiv gedacht werden
könne.\footnote{\enquote{Es ist schwerlich zu begreifen, wie ein anderer
intuitiver Verstand statt finden solte als der gottliche. Denn der erkennet in
sich als Urgrunde (und archetypo) aller Dinge Moglichkeit; aber endliche Wesen
können nicht aus sich selbst andere Dinge erkennen, weil sie nicht ihre Urheber
sind, es sey denn die bloße Erscheinungen, die sie a priori erkennen könen. 	  	  	 
Daher können wir die Dinge an sich selbst nur in Gott
erkennen} \mkbibparens{\cite[][\nopp 6048]{Kant:Reflexionen1900ff.},
\cite[][XVII: 433.16--21]{Kant:GesammelteWerke1900ff.}}. Nach
\name[Erich]{Adickes} stammt sie aus den Jahren 1783--1784. Damit lässt sich
natürlich anknüpfend an \authorcite{Foerster:Die25JahrederPhilosophie2011} auch
hier einwenden, dass \name[Immanuel]{Kant} sich die relevanten Unterschiede eben
erst später -- zwischen der zweiten Auflage der \titel{Kritik der reinen
Vernunft} 1787 und der \titel{Kritik der Urteilskraft} 1790 --  klar machen
konnte \parencite[vgl.][\pno~177,
Anm.]{Foerster:DieBedeutungvonSS7677deremphKritikderUrteilskraftfuerdieEntwicklungdernachkantischenPhilosophieTeil12002}.}

An der zitierten Stelle spricht \name[Immanuel]{Kant} davon,
dass \emph{jede} Anschauung nur stattfinden kann, insofern ihr Gegenstand
gegeben wird. Dies trifft also auch für die produktive Anschauung
zu. Damit dies nicht widersprüchlich ist, muss \name[Immanuel]{Kant}
auch dort davon sprechen, dass der Gegenstand gegeben wird, wo er hervorgebracht
wird. Der Gegenstand kann also dadurch gegeben werden, dass er von dem
erkennenden (unendlichen) Subjekt hervorgebracht wird, oder dadurch, dass er ein
erkennendes (endliches) Subjekt affiziert. In dieser Deutung entfiele wiederum
die Notwendigkeit, eine Doppeldeutigkeit des Begriffs \enquote{intellektuelle
Anschauung} anzunehmen. Die Argumente, die für die Annahme einer
Doppeldeutigkeit des Ausdrucks \enquote{intellektuelle Anschauung} sprechen,
lassen sich somit entkräften, während etliche Gründe exegetischer wie
philologischer Art für die Annahme sprechen, dass es nur einen Sinn von
\enquote{intellektuelle Anschauung} gibt und dass eine intellektuelle Anschauung
in diesem Sinne immer produktiv ist.

\subsection{Intuitiver Verstand und das
Synthetisch-Allgemeine}\label{subsection:IntuitiverVerstandunddasSynthetischAllgemeine}
Die Diskursivität des Verstandes und die Sinnlichkeit unserer Anschauung wurden
mit einem intuitiven Verstand und einer intellektuellen Anschauung verglichen, um zu
zeigen, dass die Unterscheidung von Möglichkeit und Wirklichkeit nur für
endliche Wesen gilt (Kapitel
\ref{subsubsection:UnterscheidungvonDenkenundErkennen}). Im nun zu besprechenden
Fall möchte \name[Immanuel]{Kant} zeigen, dass es der besonderen Beschaffenheit
unseres Erkenntnisvermögens geschuldet ist, dass wir Naturprodukte als Wirkungen
einer nach Zwecken handelnden Kausalität denken müssen. Er stellt dabei unseren diskursiven
einem als möglich angenommenen intuitiven Verstand gegenüber. Ziel der
Gegenüberstellung ist es zu zeigen, dass die Notwendigkeit, Naturprodukte als
Produkte einer nach Zwecken und Endursachen wirkenden Kausalität anzusehen,
lediglich der Endlichkeit unseres Verstandes geschuldet ist und nicht in den
Dingen selbst liegt.\footnote{\cite[][\S~77]{Kant:KritikderUrteilskraft2009},
\cite[][V: 408.2--13]{Kant:GesammelteWerke1900ff.}.} Außerdem verweist
\name[Immanuel]{Kant} auf einen
\singlequote{anderen} Verstand, weil die Besonderheit unseres Verstandes
es nötig mache, \enquote{den obersten Grund dazu in einem ursprünglichen Verstande als
Weltursache zu suchen.}\footnote{\cite[][\S~77]{Kant:KritikderUrteilskraft2009},
\cite[][V: 410.10--11]{Kant:GesammelteWerke1900ff.}.} Die Konzeption eines
anderen Verstandes kommt hier also in zwei verschiedenen Funktionen vor,\footnote{Zur Doppelfunktion des Verstandes siehe bspw.
\cite[][68]{Duesing:DieTeleologieinKantsWeltbegriff1968}, und oben
Anm. \ref{FussnoteDoppelfunktion} auf S. \pageref{FussnoteDoppelfunktion}.} von denen primär die
erste interessiert.\footnote{Die zweite Funktion beschreibt den
\singlequote{anderen} Verstand als produktiv und stellt definitiv keinerlei
Abweichung von der Interpretation in Kap.
\ref{subsection:DiskursiverVerstandundsinnlicheAnschauung} dar.}

Für die folgenden Rekonstruktionen sind einige Vorbemerkungen nötig:
Argumentation und Darstellung in \S~77 sind von vielen Vorannahmen abhängig, für
die \name[Immanuel]{Kant} in den vorherigen Abschnitten der \titel{Analytik der
teleologischen Urteilskraft} Beweise vorzulegen beansprucht. Die dort
entwickelten Positionen sind bekanntlich höchst umstritten. Dies liegt zumindest teilweise in den
Fortschritten begründet, die insbesondere die Biologie in den an
\name[Immanuel]{Kant} anschließenden beiden Jahrhunderten machte. Ich kann hier
weder den gesamten systematischen Rahmen nachzeichnen, noch eine Verteidigung
und Rechtfertigung der kantischen Position vornehmen. Ich werde beispielsweise
nicht fragen, ob \name[Charles]{Darwin} der \singlequote{\name[Isaac]{Newton}
des Grashalms} ist. Mir geht es ausschließlich um die Beantwortung der Frage, ob
es sich bei der Darstellung der Besonderheit unseres Verstandes in \S~77 um
dieselbe Besonderheit handelt, die bereits als Endlichkeit unseres
Erkenntnisvermögens ausgemacht wurde.

Zu Beginn des \S~77 der \titel{Kritik der Urteilskraft} stellt
\name[Immanuel]{Kant} fest, dass \enquote{die Ursache der Möglichkeit} des
\enquote{Begriff[s] eines Naturzwecks {\punkt} nur in der Idee liegen
kann}\footnote{\cite[][\S~77]{Kant:KritikderUrteilskraft2009}, \cite[][V:
405.9--12]{Kant:GesammelteWerke1900ff.}.}, wenngleich das
Naturprodukt selbst doch in der Natur gegeben sei; darin unterscheide sich die
Idee eines Naturzwecks von anderen Ideen.
Nach \name[Immanuel]{Kant} ist etwas ein Naturzweck, wenn es von sich selbst
sowohl Ursache als auch Wirkung
ist.\footnote{\cite[Vgl.][\S~64]{Kant:KritikderUrteilskraft2009},
\cite[][V: 370.35--37]{Kant:GesammelteWerke1900ff.}. \name[Immanuel]{Kant}
bezeichnet diese Angabe zunächst als \singlequote{vorläufig}.}
Und der bisherige Textverlauf präsentierte bereits die Annahme, dass der Begriff
eines organisierten Wesens, das als Naturzweck betrachtet wird, nicht
konstitutiv, sondern lediglich regulativ
ist und auf der besonderen Beschaffenheit unseres Erkenntnisvermögens
beruht.\footnote{\enquote{Es ist doch etwas ganz anderes, ob ich sage: die
Erzeugung gewisser Dinge der Natur, oder auch der gesamten Natur, ist nur durch
eine Ursache, die sich nach Absichten zum Handeln bestimmt, möglich, oder: ich
kann \ori{nach der eigentümlichen Beschaffenheit meiner Erkenntnisvermögen}
über die Möglichkeit jener Dinge und ihre Erzeugung nicht anders urteilen, als
wenn ich mir zu dieser eine Ursache, die nach Absichten wirkt, mithin ein Wesen
denke, welches nach der Analogie mit der Kausalität eines Verstandes produktiv
ist} \mkbibparens{\cite[][\S~75]{Kant:KritikderUrteilskraft2009},
\cite[][V: 397.31--398.3]{Kant:GesammelteWerke1900ff.}}.
\cite[Vgl.][\pno~385\,f.]{Guyer:FromNaturetoMorality2001}.
Siehe dazu auch \cite{Stolzenberg:OrganismusundUrteilskraft2001}.}
Darin stimmt die Idee eines Naturzwecks mit den Ideen einer
unbedingten Notwendigkeit des Urgrundes der Natur und einer unbedingten
Kausalität überein. Doch es besteht ein wichtiger Unterschied: Die Idee eines
Naturzwecks scheint ein konstitutives Prinzip zu sein, insofern Naturprodukte
doch in der Erfahrung gegeben sind. Dieser Unterschied habe seinen Ursprung
darin, dass die Idee ein Vernunftprinzip für die reflektierende Urteilskraft --
für \enquote{die Anwendung eines Verstandes überhaupt auf mögliche Gegenstände
der Erfahrung} -- sei und somit \enquote{eine Eigentümlichkeit \ori{unseres}
(menschlichen) Verstandes in Ansehung der Urteilskraft, in der Reflexion
derselben über Dinge der
Natur}\footnote{\cite[][\S~77]{Kant:KritikderUrteilskraft2009}, \cite[][V:
405.25--27]{Kant:GesammelteWerke1900ff.}.} betreffe. Wenn dies aber stimme
(wenn die Idee eines Naturzwecks der Beschaffenheit unseres Verstandes
geschuldet ist), dann müsse \enquote{die Idee von einem anderen möglichen
Verstande als dem menschlichen zum Grunde
liegen}\footnote{\cite[][\S~77]{Kant:KritikderUrteilskraft2009},
\cite[][V: 405.27--28]{Kant:GesammelteWerke1900ff.}.} und dann sei zu fragen, was
denn die Besonderheit unseres Verstandes ist und worin er sich von dem gedachten
Vergleichsverstand unterscheidet.

Nun sagt \name[Immanuel]{Kant}, wir müssten eine \emph{Zufälligkeit} der
Beschaffenheit unseres Verstandes aufsuchen. Und diese
fänden wir \enquote{ganz natürlich in dem \ori{Besonderen}, welches die
Urteilskraft unter das \ori{Allgemeine} der Verstandesbegriffe bringen soll; denn durch das Allgemeine
\ori{unseres} (menschlichen) Verstandes ist das Besondere nicht
bestimmt}\footnote{\cite[][\S~77]{Kant:KritikderUrteilskraft2009},
\cite[][V: 406.11--14]{Kant:GesammelteWerke1900ff.}.}. Weil unser Verstand ein
\emph{diskursiver} Verstand ist, sei es für ihn
zufällig, was er in der Natur an Besonderem vorfindet. Die Verstandesbegriffe,
unter die das Besondere der Natur zu bringen ist, sind die Kategorien; in Bezug
auf diese verhält sich unsere Urteilskraft bestimmend. Da ein endlicher Verstand
aber entsprechend \emph{a priori} zwar das Mögliche, nicht aber das Wirkliche
erkennen kann, sondern dafür auf die Sinne angewiesen ist, ist doch alles
Wirkliche in der Natur aus der Sicht des endlichen Verstandes
zufällig.

Bis hierher handelt es sich um \emph{dieselbe} Besonderheit unseres
Verstandes, die auch schon im vorigen Kapitel
\ref{subsubsection:UnterscheidungvonDenkenundErkennen} beschrieben
wurde. Ein intuitiver Verstand, der selbst anschaute, müsste nicht auf
die Sinne rekurrieren, da ihm das Besondere durch die eigene Tätigkeit
gegeben wäre -- er wäre \enquote{ein Vermögen einer \ori{völligen}
  Spontaneität der Anschauung}\footnote{\cite[][\S~77]{Kant:KritikderUrteilskraft2009},
  \cite[][V: 406.21--22]{Kant:GesammelteWerke1900ff.}.}. Es gäbe für
ihn keine Differenz von Möglichkeit und Wirklichkeit und \emph{a
  fortiori} keine Zufälligkeit. Wenn \name[Immanuel]{Kant} sagt, dass
\emph{unser} Verstand die Besonderheit an sich habe, \enquote{daß er
  in seinem Erkenntnisse z.\,B. der Ursache eines 
Produkts, vom \ori{Analytisch-Allgemeinen} (von Begriffen) zum Besonderen (der
gegebenen empirischen Anschauung) gehen}\footnote{\cite[][\S~77]{Kant:KritikderUrteilskraft2009},
\cite[][V: 407.13--16]{Kant:GesammelteWerke1900ff.}.} müsse, dann drückt dies also
zunächst keine neue Besonderheit, sondern die uns längst bekannte Endlichkeit unseres oberen
Erkenntnisvermögens aus. Bei den angesprochenen Begriffen handelt es sich um die
Kategorien, in dem genannten Beispiel geht es also um die Kategorie der Ursache. Von diesen
Begriffen geht der Verstand aus, um sinnliche Wahrnehmungen mittels der Urteilskraft darunter
zu subsumieren. Anders verführe ein intuitiver Verstand, der nicht von
allgemeinen Begriffen, sondern von Anschauungen ausginge:
\begin{quote}
Nun können wir uns aber auch einen Verstand denken, der, weil er nicht wie der
unsrige diskursiv, sondern intuitiv ist, vom \ori{Synthetisch-Allgemeinen} (der
Anschauung eines Ganzen als eines solchen) zum Besonderen geht, d.\,i. vom
Ganzen zu den Teilen; der also und dessen Vorstellung des Ganzen die
\ori{Zufälligkeit} der Verbindung der Teile nicht in sich enthält, um eine
bestimmte Form des Ganzen möglich zu machen, die unser Verstand bedarf, welcher
von den Teilen als allgemein gedachten Gründen zu verschiedenen darunter zu
subsumierenden möglichen Formen als Folgen fortgehen
muss.\footnote{\cite[][\S~77]{Kant:KritikderUrteilskraft2009},
\cite[][V: 407.19--28]{Kant:GesammelteWerke1900ff.}.}
\end{quote}
Die Anschauungen des intuitiven Verstandes wären -- zumindest nach meiner Lesart
-- keine sinnlichen, sondern intellektuelle Anschauungen, also Ausdruck der
Spontaneität unseres Erkenntnisvermögens, welches das Besondere aus sich selbst
heraus zu generieren verstünde. Für ein solches \enquote{Vermögen einer
\ori{völligen Spontaneität der
Anschauung}}\footnote{\cite[][\S~77]{Kant:KritikderUrteilskraft2009},
\cite[][V: 406.21--22]{Kant:GesammelteWerke1900ff.}.} gäbe es keinen Unterschied
zwischen den Modalitäten der Möglichkeit, Wirklichkeit und Notwendigkeit und
\emph{a fortiori} keine Zufälligkeit. Letztere ist nur Ausdruck der Tatsache,
dass unser Erkennen auf zwei grundlegenden Erkenntnisquellen beruht und das
Besondere nur durch die sinnliche Anschauung gegeben werden kann.

Diese Lesart bestätigt sich auch  bei Betrachtung der Redeweise von analytisch-
und synthetisch-allgemeinem. Der diskursive Verstand geht von Begriffen, der
intuitive Verstand von Anschauungen aus zum Besonderen. Die Begriffe des
diskursiven Verstandes sind analytisch-allgemein, die Anschauungen des
intuitiven Verstandes sind synthetisch-allgemein. Doch was bedeuten die
Ausdrücke \enquote{ana\-ly\-tisch-all\-ge\-mein} und
\enquote{syn\-the\-tisch-all\-ge\-mein}? In der {\jaeschelogik} ist hierzu zu lesen:
\begin{quote}
Allgemeine Regeln sind entweder \ori{analytisch} oder \ori{synthetisch}
allgemein. \ori{Jene} abstrahieren von den Verschiedenheiten; \ori{diese}
attendieren auf die Unterschiede und bestimmen folglich doch auch in Ansehung
ihrer.\footnote{\cite[][\S~21]{Kant:ImmanuelKantsLogik1977},
\cite[][IX: 102.31--33]{Kant:GesammelteWerke1900ff.}. Die Vorlage zu dieser
Textstelle in \name[Immanuel]{Kant}s Ausgabe des
\authorcite{Meier:AuszugausderVernunftlehre1752}schen Logikbuchs lautet:
\enquote{Allgemeine Regeln sind entweder
\ori{analytisch allgemein}: indem sie von den Verschiedenheiten
abstrahieren, oder \ori{synthetisch allgemein}: und diese attendieren
auf die Unterschiede, bestimmen doch auch in Ansehung ihrer}
\mkbibparens{\cite[][\nopp 3086]{Kant:Reflexionen1900ff.},
\cite[][XVI: 651.5--8]{Kant:GesammelteWerke1900ff.}}.}
\end{quote}
Begriffe sind allgemeine Regeln; und dass sie analytisch-allgemein sind, heißt,
dass sie von von den Verschiedenheiten der unter sie fallenden Gegenstände
abstrahieren. Der Begriff kann als Regel aufgefasst
werden, insofern er Merkmale enthält, unter die alle Gegenstände fallen, die
unter den Begriff fallen. Wenn Begriffe Merkmalskomplexionen sind, dann müssen wir
in entsprechenden Urteilen nicht über die Begriffe hinausgehen, um ihre Wahrheit
einzusehen. Es handelt sich daher um \emph{analytische} Urteile (was erklärt,
warum \name[Immanuel]{Kant} von analytisch-allgemeinen Regeln spricht). Der
Begriff Junggeselle hat als Merkmal beispielsweise \enquote{unverheiratet} --
dass Junggesellen unverheiratet sind ist ein analytisches Urteil. Merkmale wie
dieses treffen auf jeden Junggesellen zu, in ihrer Hinsicht sind sie alle
gleich. Die Verschiedenheiten unter den Junggesellen kommen unter diesen
Merkmalen nicht vor. Begriffe -- sagt \name[Immanuel]{Kant} in der \titel{Kritik
der reinen Vernunft} -- stehen unter dem Grundsatz der Bestimmbarkeit, ein
Begriff ist aber \enquote{in Ansehung dessen, was in ihm selbst nicht enthalten
ist, unbestimmt}\footnote{\cite[][B 599]{Kant:KritikderreinenVernunft2003},
\cite[][III: 385.18--19]{Kant:GesammelteWerke1900ff.}.}. Eine
synthetisch-allgemein Regel müsste alle Gegenstände, auf die sie anwendbar ist,
auch in ihren Besonderheiten bestimmen. Sie entspräche dem Grundsatz der durchgängigen
Bestimmung, unter dem Gegenstände nach Auskunft der \titel{Kritik der reinen
Vernunft} stehen.\footnote{\cite[Vgl.][B 599]{Kant:KritikderreinenVernunft2003},
\cite[][III: 385.25--28]{Kant:GesammelteWerke1900ff.}.} Eine
synthetisch-allgemeine Regel wäre also eine solche Regel, die alle in ihren
Anwendungsbereich fallenden Gegenstände vollständig bestimmt.

Während analytische Allgemeinheit durch Begriffe zustande
kommt,\footnote{\enquote{Je einfacher ein obiect gedacht wird, desto eher ist
analytische allgemeinheit zufolge eines Begrifs moglich}
\mkbibparens{\cite[][\nopp 3086]{Kant:Reflexionen1900ff.},
\cite[][XVI: 651.8--9]{Kant:GesammelteWerke1900ff.}}. Siehe auch die
Parallelstelle in \cite[][\S~21]{Kant:ImmanuelKantsLogik1977},
\cite[][IX: 102.33--103.2]{Kant:GesammelteWerke1900ff.}.} ist die Möglichkeit
synthetischer Allgemeinheit fraglich. Da es sich nicht um Begriffe handeln kann,
schlägt \authorfullcite{Nuzzo:KantandtheUnityofReason2005} vor, das
Synthetisch-Allgemeine als Anschauung etwa in den reinen Anschauungsformen Raum
und Zeit zu sehen. Während alle Gegenstände \emph{unter} den reinen
Verstandesbegriffen enthalten sind, sind sie zugleich \emph{in} Raum und Zeit
enthalten.\footnote{\cite[Vgl.][351]{Nuzzo:KantandtheUnityofReason2005}.} Dies
wiederum harmoniert mit einer anderen Textstelle:
In einer Anmerkung in seinem Exemplar der \titel{Metaphysica}
\authorcite{Baumgarten:Metaphysica---Metaphysik2011}s notiert
\name[Immanuel]{Kant}, dass zwischen einer \singlequote{intuitiven
Allgemeinheit} und einer \singlequote{diskursiven Allgemeinheit} zu
unterscheiden sei und dass das Besondere \singlequote{\emph{unter}} dem
dis\-kur\-siv-All\-ge\-mei\-nen, aber \singlequote{\emph{in}} dem
in\-tui\-tiv-All\-ge\-mei\-nen enthalten sei.\footnote{\enquote{Die intuitive
Allgemeinheit ist von der discursiven zu unterscheiden.
In der letzteren ist das besondere nicht im allgemeinen, sondern unter ihm 	  	  	 
enthalten, wohl aber in der
ersteren} \mkbibparens{\cite[][\nopp 6178]{Kant:Reflexionen1900ff.},
\cite[][XVIII: 481.4--6]{Kant:GesammelteWerke1900ff.}.}.}
Die Relationen \enquote{ist enthalten in} und \enquote{ist enthalten unter}
bestimmen Inhalt und Umfang von
Begriffen;\footnote{Vgl. \cite[][87]{Stuhlmann-Laeisz:KantsLogik1976}. Siehe
etwa \cite[][\S~7]{Kant:ImmanuelKantsLogik1977},
\cite[][IX: 95.27--30]{Kant:GesammelteWerke1900ff.}.} den Inhalt stellen
seine Merkmale dar, den Umfang all diejenigen Vorstellungen, von denen er
korrekt ausgesagt wird.\footnote{Nach dieser Redeweise sind Anschauungen $A^1, A^2$ und Begriffe $P^1, P^2$
\emph{unter} einem Begriff $P$ enthalten, wenn gilt:
$A^1$ ist $P$, $A^2$ ist $P$, $P^1$ ist $P$ und $P^2$ ist $P$. Umgekehrt ist $P$
in diesem Fall \emph{in} den Vorstellungen $A^1, A^2, P^1, P^2$ enthalten (Er
ist ein \emph{Merkmal} dieser Vorstellungen).} Wir betrachten als Beispiel den
Begriff des Hundes; \emph{unter} diesem Begriff finden sich sich etliche weitere
Vorstellungen wie die Begriffe des Schäferhundes, des Dackels und des Pudels,
aber auch Anschauungen wie die von Bello oder
Snoopy.\footnote{\singlequote{Unter} einem Begriff enthalten sind also
Anschauungen und Begriffe, nicht nur Begriffe -- wie in der traditionellen Logik
des 18. Jahrhunderts -- und nicht Gegenstände -- wie in der Logik ab
\authorcite{Frege:DieGrundlagenderArithmetik1988}
\parencite[vgl.][\pno~87\,f.]{Stuhlmann-Laeisz:KantsLogik1976}.} Es handelt sich
hier um das Besondere, das unter dem Begriff des Hundes enthalten ist. Der
Begriff ist in dieser Hinsicht diskursiv-allgemein.

Damit lässt sich auch die Gegenüberstellung von ana\-ly\-tisch-all\-ge\-mei\-nem
und syn\-the\-tisch-all\-ge\-mei\-nem als Ausgangspunkt des diskursiven
\emph{respective} intuitiven Verstandes im von mir angenommenen Paradigma des endlichen Verstandes
interpretieren. Warum sollte es einen Grund geben zu behaupten, dass hier eine
ganz andere Konzeption eines \singlequote{anderen} Verstandes zugrunde liege, die
nicht einmal fordern muss, dass dieser andere Verstand kein endlicher Verstand
ist?

Nach \authorcite{Foerster:Die25JahrederPhilosophie2011}\footnote{\authorcite{Foerster:Die25JahrederPhilosophie2011}
schreibt die folgenden Überlegungen ihrem Ursprung nach \name[Johann Wolfgang
von]{Goethe} zu, scheint sie aber auch selbst für zutreffend zu halten. Ich
werde mich nicht mit der Frage befassen, inwieweit es sich um eine authentische
\name[Johann Wolfgang von]{Goethe}-Interpretation handelt, sondern
die Überlegungen \authorcite{Foerster:Die25JahrederPhilosophie2011} zuschreiben.} scheitert
\name[Immanuel]{Kant} bei dem Versuch, das Wachstum und die Metamorphosen von
Organismen zu beschreiben. Denn Begriffe seien in dessen Theorie durch Merkmale
bestimmt, die auf alle unter den Begriff fallenden Gegenstände zutreffen und ihn
damit gegen andere abgrenzen. Im Ergebnis handle es sich dann aber um statische
Begriffe, die den Veränderungen nicht gerecht werden, die die unter sie
fallenden Gegenstände erleiden. Um metamorphosierende Organismen zu verstehen,
bräuchten wir Begriffe, die sich quasi mit den Gegenständen
wandeln.\footnote{\enquote{Das heißt, ich muss eine Möglichkeit finden, den
Begriff selbst so beweglich und veränderlich zu machen, dass er die Entwicklung
seines Gegenstandes mitvollziehen kann. Genauer gesagt, muss ich das Denken so
in die Anschauung versenken, den Begriff, mit dem der erste Zustand gedacht
wird, so plastisch oder flüssig machen, dass er sich mit dem metamorphosierenden
Gegenstand entwickelt}
\parencite[][183]{Foerster:DieBedeutungvonSS7677deremphKritikderUrteilskraftfuerdieEntwicklungdernachkantischenPhilosophieTeil12002}.}
Dies lässt sich an einem Beispiel illustrieren:
Der Begriff des Schmetterlings bezeichnet ein Insekt, dass sich in mehreren
Stadien -- Ei, Raupe, Puppe, Imago -- entwickelt, deren Übergänge wir als
Metamorphosen beschreiben können. Offensichtlich bestehen aber keinerlei
sichtbare Gemeinsamkeiten zwischen diesen verschiedenen Stadien: Die Raupe hat
mir dem Imago so wenig Ähnlichkeit wie mit dem Ei, aus dem sie schlüpfte, oder der
Puppe, die sie wird. Dennoch handelt es sich in jedem Stadium um denselben
Organismus, der in allen seinen Stadien unter den Begriff des Schmetterlings
fällt, wenngleich es keine gemeinsamen Merkmale zu geben scheint. Es gibt
\emph{prima facie} keine Begriffe $P$, $Q$\ldots , für die gilt, dass jeder
Gegenstand, der unter den Begriff des Schmetterlings fällt, \emph{eo ipso} unter
die Begriffe $P$, $Q$\ldots fällt. \name[Immanuel]{Kant}s Theorie des Begriffs, der zufolge Begriffe
Merkmalskomplexionen sind und aus Teilbegriffen bestehen, scheint auf Lebewesen,
die Metamorphosen durchmachen, nicht anwendbar zu sein.

In Wahrheit aber ist \name[Immanuel]{Kant}s Begriffstheorie natürlich
auch auf metamorphosierende Lebewesen anwendbar: Die Tatsache dieser
Metamorphosen ist einfach selbst als Merkmal des jeweiligen Begriffs
aufzufassen. Diese Metamorphosen wiederum werden durch Merkmale des Begriffs
\enquote{Schmetterling} bezeichnet; der Begriff selbst \emph{handelt von}
Veränderungen, aber er unterliegt ihnen freilich nicht selbst. Begriffe
beschreiben die Eigenschaften des von ihnen bezeichneten, aber sie teilen diese
Eigenschaften nicht. Und entsprechend bestehen Begriffe aus Merkmalen,
Gegenstände hingegen aus Teilen.

Es ist auch behauptet worden, dass \name[Immanuel]{Kant} das Verhältnis von
Begriffen und Anschauungen mit dem Verhältnis von Ganzem und Teil durcheinander
bringe und seine Gegenüberstellung daher fehlerhaft
sei, insofern sie eine begriffliche mit einer
physikalischen Relation konfundiere.\footnote{Vgl.
\cite[][62--67]{Rang:ZweckmaessigkeitZweckursaechlichkeitundGanzheitlichkeitinderorganischenNatur1993}.}
\authorfullcite{Quarfood:DiscursivityandTranscendentalIdealism2012} wendet
dagegen ein, dass der diskursive Verstand seinen Gegenstand
durch sukzessive Synthesis des in der Anschauung gegebenen Mannigfaltigen erst herstelle.\footnote{\enquote{However, if the
discursive understanding has to build its cognized objects by successive syntheses of data
given in intuition, this at least suggests that the syntheses will determine an
object little by little.}
\parencite[][147]{Quarfood:DiscursivityandTranscendentalIdealism2012}} Doch
\authorcite{Quarfood:DiscursivityandTranscendentalIdealism2012}s Rekonstruktion
ist unbefriedigend, schon weil die Theorie von der Synthesis des
Mannigfaltigen in der Anschauung, so wie sie hier angewandt wird, nicht
zufrieden stellt. Nach seiner Lesart möchte \name[Immanuel]{Kant} sagen, dass
wir von den Teilen eines Organismus zum Ganzen fortschreiten, weil unsere
Sinnlichkeit zunächst die Teile darbietet, die wir dann durch die Synthesis des
Verstandes erst zu einem Ganzen zusammensetzen. Aber \emph{de facto} sehen wir
nicht erst Beine, Ohren, Nase und andere Körperteile, die wir dann zu einem
Hamster zusammensetzen, sondern wir sehen den Hamster, den wir erst durch eine
gedankliche Leistung in seine Teile \singlequote{zerlegen} können.

Warum spricht \name[Immanuel]{Kant} davon, dass der intuitive Verstand im
Unterschied zum diskursiven Denken vom Ganzen zu den Teilen gehe? Inwiefern muss
unser Verstand \enquote{von den Teilen als allgemein gedachten Gründen
zu verschiedenen darunter zu subsumierenden möglichen Formen als
Folgen}\footnote{\cite[][\S~77]{Kant:KritikderUrteilskraft2009},
\cite[][V: 407.26--27]{Kant:GesammelteWerke1900ff.}.} fortschreiten?
In welchem Verhältnis steht das Verhältnis von Begriffen und Anschauungen zum
Verhältnis von Teilen und Ganzem? Diese Fragen führen auf das Problem, dass in
\name[Immanuel]{Kant}s Ausführungen nicht leicht zu erkennen ist, inwiefern die
Eigenschaft unseres Verstandes, sich in Erklärungen nach mechanischen Kausalgesetzen zu
richten, mit der Eigenschaft in Verbindung steht, ein Vermögen der Begriffe und
nicht der Anschauungen zu sein. Es ist mehrfach behauptet worden, dass hier gar
kein Zusammenhang
besteht.\footnote{\cite[Vgl.][153--159]{McLaughlin:KantsKritikderteleologischenUrteilskraft1989}.
Nach \authorcite{McLaughlin:KantsKritikderteleologischenUrteilskraft1989} stellt
\name[Immanuel]{Kant} \enquote{die mechanistische Eigentümlichkeit unseres
Verstandes als ein Faktum hin und unternimmt keinen Versuch zu erklären, worin
sie besteht und begründet sei}
\parencite[][157]{McLaughlin:KantsKritikderteleologischenUrteilskraft1989}.
Seinen Ursprung habe dieses Faktum in der \singlequote{analytisch-synthetischen}
oder \singlequote{resolutiv-kompositiven} Methode der Naturwissenschaft des 17.
und 18. Jahrhunderts
\parencite[vgl.][\pno~157\,f.]{McLaughlin:KantsKritikderteleologischenUrteilskraft1989}.}
Dennoch gilt es zu beachten, dass \name[Immanuel]{Kant} in diesem Abschnitt
stets auf eben diese Eigenschaft unseres Verstandes rekurriert, ein Vermögen des
Denkens zu sein, das selbst nicht anzuschauen vermag und für welches das
Besondere zufällig ist.

Die Überlegungen zur Diskursivität unseres Verstandes in der \titel{Kritik der
Urteilskraft} stehen im Kontext der Dialektik der teleologischen Urteilskraft
und sollen helfen, die Antinomie aufzulösen, die durch den Gegensatz von
mechanischer und teleologischer Erklärungsart ausgelöst wird.\footnote{Vgl.
\cite[][146]{McLaughlin:KantsKritikderteleologischenUrteilskraft1989}.} Die
Darstellung eines anderen Verstandes ist als Vorbereitung der
\enquote{Vereinigung des Prinzips des allgemeinen Mechanismus der Materie mit
dem teleologischen in der Technik der
Natur}\footnote{So lautet der Titel von
\S~78 der \cite{Kant:KritikderUrteilskraft2009}
\parencite[][V: 410.13--15]{Kant:GesammelteWerke1900ff.}.} in \S~78 zu
verstehen, nicht als eigenständige Abhandlung. Letztlich löst
\name[Immanuel]{Kant} die Antinomie dadurch, dass er zeigt, dass sie aus der
Sicht eines intuitiven Verstandes nicht
besteht.\footnote{Vgl. \cite[][351]{Nuzzo:KantandtheUnityofReason2005}.} Aus
unserer Sicht -- so lautet die zentrale These -- lassen sich Organismen nur
unter Rückgriff auf eine zweckmäßig wirkende Ursache verstehen. Um zu zeigen,
dass dies aber nicht beweist, dass sie nur auf diese Art möglich sind (was
bewiese, dass sie tatsächlich Wirkungen einer absichtlich handelnden Ursache
sind), rekurriert \name[Immanuel]{Kant} auf die mögliche Konzeption
eines \singlequote{anderen} Verstandes, der dieser Notwendigkeit nicht unterläge.
\begin{quote}
[M]an kann an einem Dinge, welches wir als Naturzweck beurteilen müssen (einem
organisierten Wesen), zwar alle bekannten und noch zu entdeckenden Gesetze der
mechanischen Erzeugung versuchen und auch hoffen dürfen, damit guten Fortgang zu
haben, niemals aber der Berufung auf einen davon ganz unterschiedenen
Erzeugungsgrund, nämlich der Kausalität durch Zwecke, für die Möglichkeit eines
solchen Produkts überhoben sein, und schlechterdings kann keine menschliche
Vernunft (auch keine endliche, die der Qualität nach der unsrigen ähnlich wäre,
sie aber dem Grade nach noch so sehr überstiege) die Erzeugung auch nur eines
Gräschens aus bloß mechanischen Ursachen zu verstehen
hoffen.\footnote{\cite[][\S~77]{Kant:KritikderUrteilskraft2009},
\cite[][V: 409.27--37]{Kant:GesammelteWerke1900ff.}.}
\end{quote}
Nur wegen der Beschaffenheit unseren menschlichen Erkenntnisvermögens sei es
daher notwendig, einen sich auf Zwecke berufenden Grund für die äußeren
Gegenstände im übersinnlichen Substrat der Natur und den obersten Grund
in einem \emph{intellectus archetypus} zu
suchen.\footnote{Vgl. \cite[][\S~77]{Kant:KritikderUrteilskraft2009},
\cite[][V: 410-3--11]{Kant:GesammelteWerke1900ff.}.}

Ein Organismus besteht aus Teilen, deren Anordnung zweckmäßig
ist. Die einzelnen Teile verhalten sich in einer Weise zueinander und zum
Ganzen, dass sie sich sowohl wechselseitig als auch das Ganze erhalten und
reproduzieren.\footnote{\cite[Vgl.][\S~64]{Kant:KritikderUrteilskraft2009},
\cite[][V: 371.13--372.11]{Kant:GesammelteWerke1900ff.}.}
\begin{quote}
Wenn wir nun ein Ganzes der Materie seiner Form nach als ein Produkt der Teile
und ihrer Kräfte und Vermögen, sich von selbst zu verbinden {\punkt} betrachten,
so stellen wir uns eine mechanische Erzeugungsart desselben vor: Aber es kommt
auf solche Art kein Begriff von einem Ganzen als Zweck heraus, dessen innere
Möglichkeit durchaus die Idee von einem Ganzen voraussetzt, von der selbst die
Beschaffenheit und Wirkungsart der Teile abhängt, wie wir uns doch einen
organisierten Körper vorstellen
müssen.\footnote{\cite[][\S~77]{Kant:KritikderUrteilskraft2009},
\cite[][V: 408.24--31]{Kant:GesammelteWerke1900ff.}.}
\end{quote}
Unser diskursiver Verstand kann die Entstehung des Ganzen und seine
Eigenschaften nur als Wirkungen der Beschaffenheit und Anordnung der Teile
erklären. Wir können die Wirkungsweise einer Uhr beispielsweise verstehen, wenn
wir verstehen, wie die einzelnen Teile in ihr angeordnet sind und
zusammenwirken. So erklären wir ihr Funktionieren auf mechanische Art. Die
Anordnung der Teile wiederum verstehen wir nur, wenn wir sie als Ergebnis der
Herstellung nach einem vorher gemachten Plan (einer \singlequote{Idee}) ansehen.
Auf diese Weise erklären wir das Zustandekommen der Uhr teleologisch (was
solchen Artefakten adäquat ist). Auch bei Organismen liegt es nahe,
die Eigenschaften des Ganzen mechanisch aus Beschaffenheit und Anordnung der
Teile zu verstehen, aber die Anordnung der Teile selbst scheint doch von dem
Ganzen abzuhängen und einem Plan zu entsprechen. Deshalb scheint es notwendig zu
sein, sie in Analogie zu künstlich hergestellten Gegenständen zu denken, obwohl
wir keinen Beleg dafür haben, dieses Urteil als objektiv gültig zu betrachten.

Die Vorstellung, dass das Ganze die Ursache der Teile ist, sei aus der Sicht
unseres diskursiven Verstandes ein
Widerspruch.\footnote{\cite[Vgl.][\S~77]{Kant:KritikderUrteilskraft2009},
\cite[][V: 407.34--37]{Kant:GesammelteWerke1900ff.}.} Das heißt zunächst, dass
wir nur zwei Optionen bei der Erklärung organisierter Wesen haben: Wir können
einerseits mechanisch erklären, warum etwas auf eine bestimmte Weise beschaffen
ist oder bestimmte Eigenschaften hat, indem wir dies auf äußere Ursachen oder
die Wirkung seiner Teile zurückführen. Andererseits können wir seine Form
teleologisch erklären, insofern wir die Anordnung und Beschaffenheit der Teile
als Wirkung einer Vorstellung auffassen. Wir können eben die Eigenschaften einer
Uhr als Wirkung des Zusammenspiels der Teile verstehen oder den Aufbau der Uhr
als Wirkung des zweckmäßigen Handelns eines Uhrmachers. Aber wir können die
Anordnung der Teile nicht als Wirkung des Ganzen der Uhr verstehen. Und dies
wiederum liege an der Diskursivität unseres Verstandes. Der intuitive
Verstand betrachte das Ganze des Organismus als \enquote{den Grund der
Möglichkeit der Verknüpfung der
Teile}\footnote{\cite[][\S~77]{Kant:KritikderUrteilskraft2009},
\cite[][V: 407.36]{Kant:GesammelteWerke1900ff.}.}, doch dies sei aus der Sicht
des diskursiven Verstandes ein Widerspruch. Unser endlicher Verstand könne sich nur
denken, dass \enquote{die \ori{Vorstellung} eines Ganzen den Grund der Möglichkeit der Form
desselben und der dazu gehörigen Verknüpfung der Teile
enthalte.}\footnote{\cite[][\S~77]{Kant:KritikderUrteilskraft2009},
\cite[][V: 407.37--408.2]{Kant:GesammelteWerke1900ff.}.} Eine solche Vorstellung
ist aber ein \emph{Zweck} und \emph{a fortiori} können wir uns Naturprodukte lediglich in
Analogie zu zweckgerichteter Produktion vorstellen.

Nach \authorcite{Foerster:Die25JahrederPhilosophie2011} ist
\name[Immanuel]{Kant}s Exposition eines Verstandes, der nicht vom
Ana\-ly\-tisch-All\-ge\-mei\-nen (von Begriffen), sondern vom
Synthetisch-Allgemeinen (von der Anschauung eines Ganzen) ausgeht, nicht mit der
produktiven intellektuellen Anschauung, für die Möglichkeit und Wirklichkeit
zusammenfallen, oder mit einem intuitiven Verstand, der die Gegenstände des
Erkennens selbst hervorbringt, sondern mit
\authorcite{Spinoza:EthikingeometrischerOrdnungdargestellt2007}s Konzeption
einer \emph{scientia intuitiva} in Verbindung zu bringen, die \enquote{alle
Eigenschaften [ihres] Gegenstandes aus dessen Wesenheit herzuleiten
erlaubt}\footcite[][187]{Foerster:DieBedeutungvonSS7677deremphKritikderUrteilskraftfuerdieEntwicklungdernachkantischenPhilosophieTeil12002},
insofern sie im \enquote{Besonderen zugleich das Allgemeine} sieht --
\enquote{Anschauung und Begriff fallen
zusammen.}\footcite[][188]{Foerster:DieBedeutungvonSS7677deremphKritikderUrteilskraftfuerdieEntwicklungdernachkantischenPhilosophieTeil12002}
Der Anknüpfungspunkt besteht gerade darin, dass wir versuchen wollen, uns die
Möglichkeit der Teile als vom Ganzen abhängig vorzustellen, insofern der
intuitive Verstand von dem Wesen des Gegenstandes ausgeht.

\authorcite{Spinoza:EthikingeometrischerOrdnungdargestellt2007} erwähnt die
\emph{scientia intuitiva} oder Erkenntnis der dritten Gattung (\emph{tertii
generis cognitio}) in einem \emph{scholium} zur 40. \emph{propositio} des
zweiten Teils der \titel{Ethica}. Sie gehe von der adäquaten Idee der Wesenheit
(\emph{essentia}) göttlicher Attribute aus und gelange dann zur ebenfalls
adäquaten Erkenntnis der Wesenheit einzelner Dinge.\footnote{\enquote{Atque
hoc cognoscendi genus procedit ab adaequata idea essentiae formalis quorundam
Dei attributorum ad adaequatam cognitionem essentiae rerum}
\parencite[][2p40s2]{Spinoza:EthikingeometrischerOrdnungdargestellt2007}. Diese
Gattung wird außerdem im \titel{Tractatus de intellectus emendatione}
beschrieben
\parencite[vgl.][II: 10.20--21, 11.13--19]{Spinoza:SpinozaOpera1972}.} Zur
Erläuterung stützt sich
\authorcite{Spinoza:EthikingeometrischerOrdnungdargestellt2007} vornehmlich auf
Beispiele aus der Mathematik. So lautet sein bekanntes Beispiel aus der
\titel{Ethica}: Es seien drei Zahlen $a$, $b$ und $c$ gegeben und dazu eine
vierte Zahl $d$ aufzufinden, so dass gilt: $b:a = d:c$. Zur Lösung ist $c$ mit
$b$ zu multiplizieren und das Ergebnis wiederum durch $a$ zu dividieren.
Nun gebe es verschiedene Arten, wie man dies wissen kann. Wir können es etwa von
anderen gehört haben oder wir
haben es bei einfachen Zahlen ausprobiert und dann verallgemeinert. Dies nennt
\authorcite{Spinoza:EthikingeometrischerOrdnungdargestellt2007} Erkenntnis der
ersten Gattung oder auch Meinung (\emph{opinio}) oder Vorstellung
(\emph{imaginatio}). Wir können aber auch einen mathematischen Beweis
nachvollzogen und Vernunfterkenntnis (\emph{ratio}) erworben haben, die er
Erkenntnis der zweiten Gattung nennt. Eine Erkenntnis der dritten Gattung läge
hingegen dann vor, wenn wir den Zusammenhang der ersten beiden Zahlen mit einem
Blick (\emph{uno intuitu}) erfassen könnten, um dann auf die vierte Zahl zu
schließen, wie uns dies bei ganz einfachen Zahlen möglich
sei.\footnote{\enquote{Ex.\,gr. datis numeris 1, 2, 3 nemo non videt quartum
numerum proportionalem esse 6, atque hoc multo clarius, quia ex ipsa ratione,
quam primum ad secundum habere uno intuitu videmus, ipsum quartum concludimus}
\parencite[][2p40s2]{Spinoza:EthikingeometrischerOrdnungdargestellt2007}.
\authorfullcite{Matheron:SpinozaandEuclideanArithmetic1986} betont zu Recht,
dass nach \authorcite{Spinoza:EthikingeometrischerOrdnungdargestellt2007}s
\titel{Ethica} das Verhältnis der ersten beiden Zahlen \emph{uno intuito}
erfasst werde, wir dann aber darauf \emph{schließen}, welche Zahl die vierte sein muss
\parencite[vgl.][\pno~144\,f.]{Matheron:SpinozaandEuclideanArithmetic1986}.}

Welches Merkmal grenzt die \emph{scientia intuitiva} von den anderen
Erkenntnisgattungen ab? Nach
\authorfullcite{Bartuschat:SpinozasTheoriedesMenschen1992} ist es die Tatsache,
dass es sich um eine rationale Erkenntnis von \emph{Einzelnem} handelt, während
die Vernunft (\emph{ratio}) als zweite Gattung nur Allgemeines
erkenne.\footnote{\enquote{Die Erkenntnis der Essenz von einzelnem und damit
Gottes im einzelnen ist die scientia intuitiva. Sie ist mit dem Konzept Gottes
genau dann verbunden, wenn Gott als ein Wesen konzipiert ist, das die Ursache
von Individuellem ist, das in seiner Endlichkeit von der Unendlichkeit Gottes
verschieden ist und deshalb sein Verursachtsein an sich selber erweisen muß}
\parencite[][121]{Bartuschat:SpinozasTheoriedesMenschen1992}.} Nach
\authorfullcite{Roed:SpinozasIdeederScientiaintuitivaunddieSpinozanischeWissenschaftskonzeption1977}
ist \enquote{die intuitive Einsicht primär Erkenntnis Gottes und sekundär
Erkenntnis endlicher Dinge in Abhängigkeit von der Erkenntnis
Gottes}\footcite[][\pno~497\,f.]{Roed:SpinozasIdeederScientiaintuitivaunddieSpinozanischeWissenschaftskonzeption1977}
und damit \enquote{Wissen vom absoluten Ganzen in der Einheit seiner
Momente}\footcite[][498]{Roed:SpinozasIdeederScientiaintuitivaunddieSpinozanischeWissenschaftskonzeption1977}.
Dabei habe der Begriff der \emph{scientia intuitiva} jedoch verschiedene
Bedeutungskomponenten, die nicht alle mit dieser Deutung in Einklang zu bringen
seien.\footcite[Vgl.][498]{Roed:SpinozasIdeederScientiaintuitivaunddieSpinozanischeWissenschaftskonzeption1977}
\authorfullcite{Foerster:Die25JahrederPhilosophie2011} wiederum verweist darauf,
dass die Vernunfterkenntnis die Dinge aus äußeren Ursachen erkenne, während die
\emph{scientia intuitiva} die Dinge erkenne, insofern sie Ursachen ihrer selbst
sind.\footnote{\cite[Siehe][91]{Foerster:Die25JahrederPhilosophie2011}:
\enquote{Bereits in seiner Frühschrift \ori{Abhandlung über die Berichtigung des Verstandes} hatte er
darauf insistiert, dass zur adäquaten Erkenntnis einer Sache diese entweder bloß
durch ihre Wesenheit oder durch ihre nächste Ursache begriffen werden müsse:
wenn eine Sache an und für sich besteht oder Ursache ihrer selbst ist, so muss
sie bloß durch ihre Wesenheit erkannt werden; wenn sie zu ihrem Dasein aber eine
Ursache braucht, dann muss sie durch ihre nächste Ursache erkannt werden (TIE
§92). Soll die Erkenntnis die Form eines Systems haben, dann muss freilich mit
dem der Anfang gemacht werden, dessen Begriff den Begriff keiner anderen Sache
voraussetzt.}} Und dies entspricht dem Begriff des Naturzwecks
bei \name[Immanuel]{Kant}. Der intuitive Verstand erkennt die Dinge als
Naturzwecke, die von sich selbst Ursache und Wirkung sind, ohne ihnen eine
äußere, nach Zwecken handelnde Ursache zuschreiben zu
müssen. \name[Immanuel]{Kant} wiederum behauptet, dass uns endlichen
Subjekten ein solches Vorgehen nicht möglich sei. \emph{Wir} können
die Eigenschaften eines Organismus und die Anordnung seiner Teile
nicht dadurch erklären, dass wir auf sein Wesen rekurrieren und so das
Ganze -- \emph{respective} dessen Essenz -- zur Ursache der Organisation der Teile machen.

Nach \authorcite{Foerster:Die25JahrederPhilosophie2011} besteht der wesentliche
Grund, warum \name[Immanuel]{Kant} die von
\authorcite{Spinoza:EthikingeometrischerOrdnungdargestellt2007} behauptete
Möglichkeit der \emph{scientia intuitiva} verneint, darin, dass er nicht wie
dieser von der Mathematik, sondern von einem göttlichen Verstand
ausgehe.\footnote{Vgl. \cite[][255]{Foerster:Die25JahrederPhilosophie2011}.}
Die Idee eines intuitiven Verstandes schwebe ihm ja selbst vor, er sehe nur
nicht, wie eine entsprechende Art zu erkennen möglich sein könnte. Auch \authorcite{Spinoza:EthikingeometrischerOrdnungdargestellt2007} bedauert, dass er so
wenige Beispiele für die \emph{scientia intuitiva}
finde.\footnote{\cite[Vgl.][II: 11.18--19]{Spinoza:SpinozaOpera1972}.} In der
\titel{Ethica} finden sich kaum Beispiele für das Vorliegen der
dritten Erkenntnisgattung. Stattdessen wird behauptet, dass wir über
solche intuitiven Erkenntnisse verfügen, weil wir eine adäquate Idee
Gottes besitzen und aus dieser Idee schlussfolgern können.
\name[Immanuel]{Kant} bestreitet gerade dies, und zwar vor dem Hintergrund der
Überlegungen zum Verhältnis von Anschauung und Begriff, wie sie oben betrachtet
wurden.\footnote{Siehe Kapitel
\ref{subsubsection:UnterscheidungvonDenkenundErkennen}.} Gerade weil uns als
endlichen Wesen die Unterscheidung von Möglichkeit und Wirklichkeit aufgezwungen
sei, könnten wir ein solches Wesen, wie es
\authorcite{Spinoza:EthikingeometrischerOrdnungdargestellt2007} als die eine
Substanz konzipiert, nicht denken und seine Existenz nicht gemäß dem
\singlequote{ontologischen} Gottesbeweis
einsehen.\footnote{\cite[Vgl.][\S~76]{Kant:KritikderUrteilskraft2009},
\cite[][V: 402.18--32]{Kant:GesammelteWerke1900ff.}.}

\authorcite{Foerster:Die25JahrederPhilosophie2011} zufolge hat erst \name[Johann
Wolfgang von]{Goethe} gesehen, wie die Idee der \emph{scientia intuitiva} von
der Mathematik auf die Naturerkenntnis ausgeweitet werden kann.\footnote{Vgl.
\cite[][188]{Foerster:DieBedeutungvonSS7677deremphKritikderUrteilskraftfuerdieEntwicklungdernachkantischenPhilosophieTeil12002}.}
Dabei bleibe diese Erkenntnisart esoterisch: Sie sei nur durch Übung zu erlangen
und darum wenigen Menschen vorbehalten.\footnote{\enquote{Dass eine solche
Verbindung von diskursivem und intuitivem Denken nur unter Anstrengungen und als
Ergebnis wiederholter Übung möglich ist, war Goethe durchaus klar}
\parencite[][185]{Foerster:DieBedeutungvonSS7677deremphKritikderUrteilskraftfuerdieEntwicklungdernachkantischenPhilosophieTeil12002}.}
\name[Immanuel]{Kant}s Fehler sei es gewesen, die Möglichkeit einer solchen --
doch immerhin denkbaren -- Erkenntnisart grundlos abzuleugnen, statt zumindest
den Versuch dazu zu unternehmen.\footnote{\enquote{Es dürfte klar sein, dass sich ein begründetes Urteil über die Möglichkeit oder
Unmöglichkeit eines solchen Verfahrens (und damit über die Möglichkeit eines
intuitiven Verstandes) nicht fällen lässt, ohne dass diese Schritte erprobt und
nachvollzogen werden. Ein ausschließlich diskursives Denken, das aus sich selbst
heraus seine Alternativenlosigkeit glaubt wissen zu können, erweist seine
philosophische Naivität gerade dadurch, dass es dogmatisch selbst noch hinter
die Kantische Forderung zurückfällt, eine Alternative zu unserem gegenwärtigen
Erkenntnisvermögen zumindest versuchsweise zu denken, um es nicht für das einzig
mögliche zu halten}
\parencite[][187]{Foerster:DieBedeutungvonSS7677deremphKritikderUrteilskraftfuerdieEntwicklungdernachkantischenPhilosophieTeil12002}.
Es ist mehrfach argumentiert worden, \name[Immanuel]{Kant} zeige doch selbst
die Möglichkeit eines \singlequote{anderen} Erkennens auf, indem er -- speziell
in den \S\S~76 und 77 der \titel{Kritik der Urteilskraft} ein solches
konzipiert, bevor er es grundlos bzw. aufgrund nicht explizit gemachter
empirischer Beobachtung verwirft. Ein solcher Vorwurf findet sich etwa bei
\authorcite{Hegel:GesammelteWerke} in \titel{Glauben und Wissen}
\parencite[vgl.][IV: 335,9..11, 338.35--343.17]{Hegel:GesammelteWerke}.}
Um intuitiv zu erkennen sei zunächst durch eine kontinuierlich Reihe von
Beobachtungen ein Ganzes in den Blick zu bekommen, von welchem ausgehend ein intuitiver Verstand erst zu
den Teilen gehen könne. Im Anschluss daran müsse man sich die Sache durch Denken
aneignen. Und dazu bedürfe es einer Verbindung von anschauender und diskursiver
Erkenntnis. Was soll das aber heißen?
\authorcite{Foerster:Die25JahrederPhilosophie2011} schreibt:
\begin{quote}
Ich kann die Pflanze nicht anders zeichnen, als Stück für Stück und
nacheinander. Genauso kann ich scheinbar deren Entwicklung nicht anders denken
als diskursiv und nacheinander. Aber so entwickelt sich kein Organismus. Er
wächst in allen Teilen zugleich. {\punkt} Die Pflanze bildet sich in allen ihren
Teilen zugleich. Um diese Prozesse nachzuvollziehen, muss ich in Gedanken also
auch an allen ihren Stellen zugleich sein; mit anderen Worten: das Denken muss
intuitiv, d.\,h. anschauend werden. Der Gedanke eines gleichzeitigen Ganzen von
Teilen und der einer Abfolge von Veränderungen der Teile muss ein einzelner,
selbst lebendiger Gedanke
werden.\footnote{\cite[][\pno~184\,f.]{Foerster:DieBedeutungvonSS7677deremphKritikderUrteilskraftfuerdieEntwicklungdernachkantischenPhilosophieTeil12002}.}
\end{quote}
Doch diese Überlegung ist nicht verlockender als die Überlegung zur Lebendigkeit
von Begriffen: Ebenso wie ein Begriff die Metamorphosen seines Gegenstands durch
Merkmale beinhaltet, ohne dass er selbst metamorphosieren müsste, kann ein Gedanke von
der sukzessiven Entwicklung eines Gegenstandes handeln, ohne sich dabei selbst
zu verändern.\footnote{\authorcite{Foerster:Die25JahrederPhilosophie2011} verwendet hier m.\,E.
einen anderen Diskursivitätsbegriff als denjenigen, den ich in Kapitel
\ref{subsection:DiskursiverVerstandundsinnlicheAnschauung} herausstellte. Ein
Denken ist danach diskursiv, wenn es Schritt für Schritt vorgeht (was der
Bedeutung des \enquote{\emph{discurrere}} natürlich nahe kommt). Intuitiv wäre
hingegen ein Denken, das das Ganze simultan in den Blick bekommt. Dies wiederum
erinnert eher an den Begriff der \emph{cognitio intuitiva} (im Kontrast zu
einer \emph{cognitio symbolica}) bei
\textcite[vgl.][585--588]{Leibniz:Meditationesdecognitioneveritateetideis1999}, als an
\authorcite{Spinoza:SpinozaOpera1972} und \name[Immanuel]{Kant}.}

\name[Immanuel]{Kant} betont auch in \S~77 der \titel{Kritik der Urteilskraft}
die Beschaffenheit unseres Verstandes als eines Vermögens der
\emph{Begriffe}\footnote{\enquote{Unser Verstand ist ein Vermögen der Begriffe,
d.\,i. ein diskursiver Verstand}
\mkbibparens{\cite[][\S~77]{Kant:KritikderUrteilskraft2009}, \cite[][V:
406.16--17]{Kant:GesammelteWerke1900ff.}}.} und die Tatsache, dass für einen
solchen \emph{diskursiven} Verstand die Verbindung der Teile eines Ganzen
zufällig sein muss. Es ist zunächst dieselbe
Zufälligkeit, die uns bei den besonderen Gesetzen der Natur begegnet, die auch
in der Konstitution lebender Wesen vorliegt. Ein intuitiver Verstand, für den die
Unterscheidung von Möglichkeit und Wirklichkeit nicht besteht, denkt nicht den
Grund für die Wirklichkeit eines Gegenstandes als äußere Ursache, weil der
Gegenstand mit seiner Möglichkeit bereits als wirklich gegeben wäre. Unser
diskursiver Verstand, der Möglichkeit und Wirklichkeit unterscheidet, kann sich
die Möglichkeit eines Gegenstandes nicht als dessen innere hinreichende Ursache
vorstellen, sondern bedarf der Annahme einer äußeren Ursache. Der intuitive
Verstand hingegen ginge genau deshalb vom Ganzen zu den Teilen, weil für ihn mit
der Möglichkeit des Ganzen dessen Wirklichkeit und damit auch die Wirklichkeit
seiner Teile und damit deren Anordnung gegeben wäre. Demnach ist es dieselbe Besonderheit
unseres Verstandes, die schon für die Unterscheidung von Möglichkeit und Wirklichkeit
verantwortlich zeichnete, die auch die Besonderheit in Ansehung der
reflektierenden Urteilskraft hervorbringt, Naturprodukte nur in Analogie zu
Zwecken denken zu können.

\subsection{Die Endlichkeit des
Willens}\label{subsubsection:DieEndlichkeitdesWillens}
Ebenso wie die Begriffe der Möglichkeit und Wirklichkeit nach den Überlegungen
aus Kapitel \ref{subsubsection:UnterscheidungvonDenkenundErkennen} und der
Begriff eines Naturzwecks nach Kapitel
\ref{subsection:IntuitiverVerstandunddasSynthetischAllgemeine} nur Bedeutung für
einen endlichen, diskursiven Verstand haben, weil sie die Trennung von Begriff
und Anschauung, von Verstand und Sinnlichkeit voraussetzen, für einen denkbaren
\singlequote{anderen}, nämlich intuitiven Verstand jedoch völlig ohne Bedeutung
wären, so liege es an \enquote{der subjektiven
Beschaffenheit unseres praktischen Vermögens {\punkt}, daß die moralischen
Gesetze als Gebote} \enquote{vorgestellt werden
müssen}\footnote{\cite[][\S~76]{Kant:KritikderUrteilskraft2009}, \cite[][V:
403.30--32]{Kant:GesammelteWerke1900ff.}.}. Für einen unendlichen Willen sind
keine Gebote und Imperative denkbar, denn es fehlt die Möglichkeit des Handelns
wider die Vernunft. Unser Wille ist endlich, insofern wir von den Gesetzen der
Autonomie abweichen können. Nur ein endlicher Wille unterliegt
Verbindlichkeiten.\footnote{\enquote{Der Wille, dessen Maximen notwendig mit den Gesetzen der Autonomie
zusammenstimmen, ist ein \ori{heiliger}, schlechterdings guter Wille. Die
Abhängigkeit eines nicht schlechterdings guten Willens vom Prinzip der Autonomie
(die moralische Nötigung) ist \ori{Verbindlichkeit}. Diese kann also auf ein
heiliges Wesen nicht bezogen werden} \mkbibparens{\cite[][BA
86]{Kant:GrundlegungzurMetaphysikderSitten1965}, \cite[][IV:
439.28--33]{Kant:GesammelteWerke1900ff.}}.}
Für einen heiligen Willen kann es keine Verbindlichkeiten und damit
keine Imperative geben, denn es fehlt die Möglichkeit des
Zuwiderhandelns.\footnote{\enquote{Ein vollkommen guter Wille würde also eben sowohl unter objektiven Gesetzen
(des Guten) stehen, aber nicht dadurch als zu gesetzmäßigen Handlungen
\ori{genötigt} vorgestellt werden können, weil er von selbst, nach seiner
subjektiven Beschaffenheit, nur durch die Vorstellung des Guten bestimmt werden
kann. Daher gelten für den \ori{göttlichen} und überhaupt für einen
\ori{heiligen} Willen keine Imperativen; das \ori{Sollen} ist hier am unrechten
Orte, weil das \ori{Wollen} schon von selbst mit dem Gesetz notwendig einstimmig
ist} \mkbibparens{\cite[][BA 39]{Kant:GrundlegungzurMetaphysikderSitten1965},
\cite[][IV: 414.1--8]{Kant:GesammelteWerke1900ff.}}. Siehe auch
\cite[][A 57]{Kant:KritikderpraktischenVernunft1974},
\cite[][V: 32.15--21]{Kant:GesammelteWerke1900ff.}: Das moralische Gesetz
\enquote{schränkt sich also nicht bloß auf Menschen ein, sondern geht auf alle endlichen Wesen, die Vernunft und Willen
haben, ja schließt sogar das unendliche Wesen, als oberste Intelligenz, mit ein.
Im ersteren Falle aber hat das Gesetz die Form eines Imperativs, weil man an
jenem zwar als vernünftigem Wesen einen reinen, aber als mit Bedürfnissen und
sinnlichen Bewegursachen affiziertem Wesen keinen heiligen Willen, d.\,i. einen
solchen, der keiner dem moralischen Gesetze widerstreitenden Maxime fähig wäre,
voraussetzen kann.}}
Deswegen ist das moralische Gesetz für endliche Wesen ein Gesetz der Pflicht,
das mit Nötigung und dem Gefühl der Achtung verbunden ist, für einen unendlichen
Willen hingegen ist es ein Gesetz der Heiligkeit.\footnote{\cite[][A
146]{Kant:KritikderpraktischenVernunft1974}, \cite[][V:
82.8--12]{Kant:GesammelteWerke1900ff.}. Außerdem kommen nur einem endlichen Willen Neigungen und Interessen zu: \enquote{Die Abhängigkeit des
Begehrungsvermögens von Empfindungen heißt Neigung, und diese beweist also jederzeit ein Bedürfnis. Die Abhängigkeit eines zufällig
bestimmbaren Willens aber von Prinzipien der Vernunft heißt ein Interesse.
Dieses findet also nur bei einem abhängigen Willen statt, der nicht von selbst
jederzeit der Vernunft gemäß ist; beim göttlichen Willen kann man sich kein
Interesse gedenken} \mkbibparens{\cite[][BA
38]{Kant:GrundlegungzurMetaphysikderSitten1965}, \cite[][IV:
413.26--31]{Kant:GesammelteWerke1900ff.}}.}

Dies ist die Charakterisierung der Endlichkeit unseres Willens. Die Frage, die nun zu beantworten ist, lautet:
Handelt es sich auch der Endlichkeit des menschlichen Willens um eine Ausprägung oder Folge der
Endlichkeit unseres Verstandes? Oder steht sie getrennt neben dieser
Endlichkeit? Letzteres scheint naheliegend zu sein, insofern die Endlichkeit eines konativen
statt eines kognitiven Vermögens angesprochen ist. Unterschiede
\name[Immanuel]{Kant} zwischen Verstand und Willen als zwei unterschiedlichen
Vermögen, dann läge es nahe davon auszugehen, dass beide Ausprägungen unserer Endlichkeit unverbunden
nebeneinander stehen. Geht man hingegen -- wie
\authorcite{Spinoza:EthikingeometrischerOrdnungdargestellt2007} gegen
\authorcite{Descartes:OeuvresdeDescartes1983} betont\footnote{Zum Verhältnis von
Wille und Verstand bei \authorcite{Descartes:OeuvresdeDescartes1983} siehe oben,
S. \pageref{Absatz:DescarteszuEndlichkeitundVerhaeltnisvonWilleundVerstand}\,f.
Spinoza kritisiert dies in der \titel{Ethica}:
\enquote{Voluntas, et intellectus unum, et idem sunt} \parencite[][\nopp
2p49c]{Spinoza:EthikingeometrischerOrdnungdargestellt2007}.} -- davon aus, dass
der Wille selbst ein intellektuelles Vermögen ist, dann gilt es, den möglichen
Zusammenhang genauer in's Auge zu fassen.

Die Eigenschaft des moralischen Gesetzes, uns in der Form von Imperativen zu
begegnen, beschreibt die Endlichkeit unserer Vernunft, insofern sie praktische
Vernunft ist. Vernunft (im engeren Sinne) ist zunächst das Vermögen, zu
schließen. Vernunft ist praktisch, wenn sie sich nicht in Ableitung von
Erkenntnissen äußert, sondern von Handlungen -- die praktische Vernunft heißt
darum auch \enquote{Wille}.\footnote{\enquote{Ein jedes Ding der Natur wirkt nach Gesetzen. Nur ein vernünftiges Wesen hat
  das Vermögen, \ori{nach der Vorstellung} der Gesetze, d.\,i. nach Prinzipien, zu
  handeln, oder einen \ori{Willen}. Da zur Ableitung der Handlungen von Gesetzen
  \ori{Vernunft} erfodert wird, so ist der Wille nichts anders, als praktische
  Vernunft} \mkbibparens{\cite[][BA 36]{Kant:GrundlegungzurMetaphysikderSitten1965},
  \cite[][IV: 412.26--30]{Kant:GesammelteWerke1900ff.}}.}
Wir können also gleichbedeutend von der Endlichkeit des Willens und der
praktischen Vernunft sprechen, denn der Wille ist \emph{nichts anderes} als
praktische Vernunft. Wie aus dem Wortlaut dieses Zitats deutlich wird, äußert
sich die praktische Vernunft oder der Wille in praktischen Syllogismen. Ein
Syllogismus ist praktisch, wenn er als Konklusion keine Aussage hat -- auch
keine imperativische, die zur Handlung auffordert --, sondern eine
Handlung. Als Prinzip ist beispielsweise die Maxime gegeben \enquote{Ich will
stets ehrlich sein.}.
Sie fungiert als Obersatz in einem Syllogismus und erlaubt in einer Situation,
in der ich etwas gefragt werde, den Schluss auf die
\emph{Handlung}, in der ich die Wahrheit sage (oder die Antwort verweigere,
damit aber zumindest nichts falsches sage).
Die Konklusion besteht wohlgemerkt nicht in einer Erkenntnis der Art \enquote{Ich sollte jetzt
das-und-das sagen!}, sondern direkt in der entsprechenden
Handlung.\footnote{Dem widersprechen jedoch einige Textbelege, an denen
\name[Immanuel]{Kant} entgegen dem Wortlaut der \titel{Grundlegung} als
Resultat der praktischen Vernunft keine Handlungen, sondern Erkenntnisse
bezüglich eines Sollens ansieht. Siehe z.\,B.
\cite[][B 661]{Kant:KritikderreinenVernunft2003},
\cite[][III: 421.17--19]{Kant:GesammelteWerke1900ff.}:
\enquote{Ich begnüge mich hier, die theoretische Erkenntnis durch eine solche zu
erklären, wodurch ich erkenne, was da \ori{ist}, die praktische aber, dadurch
ich mir vorstelle, was da \ori{sein soll}.} Siehe zu dieser Frage
\cite{Engstrom:KantsDistinctionbetweenTheoreticalandPracticalKnowledge2002}.}
Der Wille oder die praktische Vernunft ist somit ein Vermögen, Handlungen als
Ausdruck allgemeiner Grundsätze auszuführen.

\begin{comment}
Zunächst scheint unser Wille die Diskursivität des Verstandes und die
Sinnlichkeit der Anschauung gerade nicht zu teilen. Es ließe sich gar vermuten,
dass die praktische Vernunft der intellektuellen Anschauung oder dem intuitiven
Verstand entspricht, insofern sie die Gegenstände ihrer Erkenntnis -- die
ausgeführten Handlungen -- selbst hervorbringt. Nun ist -- wie
\authorfullcite{Engstrom:KantsDistinctionbetweenTheoreticalandPracticalKnowledge2002}
anführt --  auch die Erkenntnis der praktischen Vernunft eine Handlung des
\emph{diskursiven}
Verstandes.\footnote{\cite[Vgl.][\pno~59\,f.:]{Engstrom:KantsDistinctionbetweenTheoreticalandPracticalKnowledge2002}
\enquote{practical knowledge is productive of its object not only with respect to the latter's form, but even with respect to its existence. Now this description of how practical knowledge
is related to its object might appear to make practical knowledge
indistinguishable from intellectual intuition, the divine cognition with which
human discursive cognition is contrasted, and to which Kant tacitly alludes in
the passage from §14. But the distinction between these two can be duly
maintained if we bear in mind that practical knowledge, like theoretical
knowledge, is the product of discursive intellect, whose knowledge always
proceeds from concepts, or general representations, rather than from intuitions,
or singular representations, and further that, as a result, in the case of
practical knowledge, the production of the object is always the arrengement, in
accordance with a general form, of presupposed matter, whereas, in the case of
intellectual intuition, there is no matter requisite as a condition under which
the production is possible.}}
\end{comment}

Der Wille ist Bestandteil des oberen Erkenntnisvermögens und somit dessen
Endlichkeit unterworfen. In der \titel{Kritik der Urteilskraft} heißt es
entsprechend, dass die Endlichkeit des Willens aufzuheben hieße, \enquote{die Vernunft
ohne Sinnlichkeit (als subjektiver Bedingung ihrer Anwendung auf Gegenstände
der Natur)}\footnote{\cite[][\S~76]{Kant:KritikderUrteilskraft2009},
\cite[][V: 403.34--36]{Kant:GesammelteWerke1900ff.}.} und damit als Ursache in
einer intelligiblen Welt zu betrachten. Unser Wille ist endlich, weil
unsere Vernunft nur mit unserer Sinnlichkeit praktische Vernunft sein
kann. Dem stimmt auch
\authorfullcite{Foerster:Die25JahrederPhilosophie2011} zu:
\enquote{Für eine Vernunft, die ohne Sinnlichkeit als subjektive Bedingung ihrer
Anwendung wirksam sein könnte, fiele dieser Unterschied fort. Der Gegensatz
{\punkt} ist also nur gültig für ein praktisches Vernunftwesen, das zugleich
sinnlich ist und dessen Kausalität mit derjenigen der Sinnenwelt nicht
zusammenfällt.}\footnote{\cite[][151]{Foerster:Die25JahrederPhilosophie2011}.}
Da die Vernunft (im engeren Sinne) Teil des oberen Erkenntnisvermögens -- des
Verstandes als des Vermögens der Spontaneität -- ist und ihre Endlichkeit darin
besteht, auf Sinnlichkeit angewiesen zu sein, handelt es sich bei der
Endlichkeit unseres Willens also um eine Ausformung der Endlichkeit des
Verstandes.

\section{Zusammenfassung und Ausblick}
In Kapitel \ref{section:KantalsliberalerAufklaerer} erarbeitete ich einen
Aufklärungsbegriff, der die intellektuelle Freiheit und
Unabhängigkeit fokussiert und sich gegen Passivität und Abhängigkeit wendet.
Ihm zufolge sind wir zwar in unserem Verstandesgebrauch davon abhängig, dass wir uns
in einer Gemeinschaft mit anderen befinden. Aber wir verstehen uns innerhalb
dieser Gemeinschaft doch als gleichwertige Mitglieder, die jedes Urteil selbst
\mbox{(mit-)} kontrollieren, statt es passiv in den eigenen
Überzeugungsvorrat aufzunehmen. Selbstdenken ist zunächst abstrakt als
Selbsttätigkeit (Spontaneität) beschrieben und dem passiven Rezipieren entgegen
gestellt.

In diesem \ref{chapter:endlichkeitmenschlichendenkens}. Kapitel zeigte sich
aber, dass unsere Selbsttätigkeit in jedem Fall von Rezeptivität abhängig ist.
Dies macht gerade die Endlichkeit des Menschen aus:
dass er bei jeder kognitiven Operation von seiner Rezeptivität abhängig bleibt.
Insofern besteht hier ein klarer Gegensatz von Aufklärung und Endlichkeit, den
es im folgenden aufzulösen gilt. Es geht darum zu zeigen, wie angesichts der
vielfältigen Abhängigkeiten, denen wir als endliche Wesen unterliegen, dennoch
sinnvoll von intellektueller Selbständigkeit und Unabhängigkeit gesprochen
werden kann. In welchen Formen können uns Erkenntnisse \singlequote{gegeben}
werden, ohne dass dies einen Verlust an Mündigkeit mit sich bringt?

