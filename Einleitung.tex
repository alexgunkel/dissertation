\section{Endlichkeit, Aufklärung und
Kantianismus}\label{section:EndlichkeitundKantianismus}
Herbert \name[Herbert]{Schnädelbach} bemüht sich seit geraumer Zeit, einen Kantianismus,
der sich zugleich als Philosophie der \emph{Endlichkeit} und der \emph{Aufklärung} versteht,
als Gebot der Stunde auszuweisen, um ihn gegen einen neuen Hegelianismus in Stellung zu
bringen.\footnote{\cite[Vgl.][]{Schnaedelbach:HegelsErben2006,Schnaedelbach:WirKantianer2005,Schnaedelbach:WarumHegel?1999},
sowie die Diskussion in: \cite{Stekeler-Weithofer:StreitumHegel2000}.} Dabei
bleibt der Zusammenhang zwischen Endlichkeit und Aufklärung jedoch offen. Es ist
lediglich klar, \emph{dass} \name[Herbert]{Schnädelbach} eine Verbindung sieht. Denn wenn
er sich auf \name[Immanuel]{Kant} als den Philosophen der Endlichkeit
bezieht, so sieht er  in ihm den Philosophen, an den es im
Interesse einer Fortführung der Aufklärung anzuschließen
gelte.\footcite[Vgl.][836-7]{Schnaedelbach:WirKantianer2005} Dabei hält er die
Ergebnisse und Voraussetzungen der Transzendentalphilosophie für längst obsolet, aber
dennoch für Bestandteile einer einheitlichen um Aufklärung bemühten
Philosophie.\footnote{\cite[Vgl.][]{Schnaedelbach:WirKantianer2005}.} Als
Philosoph der Endlichkeit soll \name[Immanuel]{Kant} zugleich die Aufklärung und die
moderne Kultur verkörpern. Aufklärung und moderne Kultur sind ihm sich
gegenseitig stützende oder bedingende Themen, die \name[Immanuel]{Kant}s Philosophie
\emph{als Ganze} charakterisieren.

Doch weder die Auffassung, dass \name{Kant}s Philosophie als Ganze in ein
Projekt der Aufklärung integriert ist, noch die, dass \name[Immanuel]{Kant}s
Aufklärungsverständnis mit der Endlichkeit des Menschen harmoniert, ist
selbstverständlich. Beispielsweise beschreibt Heinrich \name[Heinrich]{Rickert},
auf dessen Schrift \titel{\name[Immanuel]{Kant} als Philosoph der modernen
Kultur} sich \name[Herbert]{Schnädelbach} wesentlich bezieht,
\name[Immanuel]{Kant} als \emph{Überwinder} der Aufklärung, der \enquote{die
Schranken \ori{aller} Philosophie, die es vor ihm in Europa gab, mit Rücksicht auf die Kulturprobleme
prinzipiell
durchbrach.}\footnote{\cite[][140]{Rickert:KantalsPhilosophdermodernenKultur1924}.}
Kern der Überwindung der Aufklärung sei dabei die \titel{Kritik der reinen
Vernunft}, welche durch die Differenzierung von Erkenntnisbereichen die moderne,
nach-aufklärerische Kultur begründet habe.\footnote{\cite[Vgl.][141]{Rickert:KantalsPhilosophdermodernenKultur1924}:
\enquote{Kant hat als erster Denker in Europa die \ori{allgemeinsten theoretischen
Grundlagen} geschaffen, die wissenschaftliche Antworten auf spezifisch moderne
Kulturprobleme überhaupt \ori{möglich} machen, und insbesondere läßt sich dartun: sein
Denken, wie es sich in seinen drei großen Kritiken darstellt, ist in dem Sinn
\enquote{kritisch}, d.\,h. \ori{scheidend} und \ori{Grenzen ziehend} gewesen, daß es dadurch
im Prinzip dem Prozeß der \ori{Verselbständigung} und \ori{Differenzierung} der Kultur
entspricht, wie er sich seit dem Beginn der Neuzeit faktisch vollzogen, aber in
der Philosophie vor Kant noch keinen theoretischen Ausdruck gefunden hatte.}}
Dagegen wiederum sieht \name[Heinrich]{Rickert}s Zeitgenosse Kurt
\name[Kurt]{Sternberg} gerade in von \name[Heinrich]{Rickert} herausgestellten
Momenten von \name[Immanuel]{Kant}s kritischer Philosophie die Aufklärung
bewahrt.\footcite[Vgl.][]{Sternberg:AufklaerungKlassizismusundRomantikbeiKant1931}
Auch Norbert \name[Norbert]{Hinske} -- um wieder auf einen aktuellen Autor, aber
ebenso dezidierten \name[Immanuel]{Kant}kenner zu verweisen -- sieht Aufklärung
in der Einsicht
in die Endlichkeit fundiert:
\begin{quote}
  Philosophie der Aufklärung ist demgemäß ihrem eigenen Impuls zufolge
  Philosophie der endlichen Vernunft. Sie lebt und denkt aus der Einsicht, daß
  das Ganze, die Totalität der Wahrheit, auf die Vernunft aus ist, dem Menschen
  nicht gegeben, sondern nur aufgegeben ist. Sie steht im Horizont des
  Unbedingten, aber sie befindet sich nicht in seinem Besitz. Sie leidet an
  dieser ihrer Endlichkeit, aber sie lügt sich nicht über sie hinweg. Eben das
  aber gibt ihr zugleich auch die Offenheit, die Vernunft des Anderen: die
  Vernunft \emph{jedes} Anderen, als ein Stück der allgemeinen Vernunft zu begreifen
  und
  ernstzunehmen.\footnote{\cite[][38]{Hinske:KantalsHerausforderungandieGegenwart1980}.}
\end{quote}
Unsere Endlichkeit, die uns jederzeit vom Besitz der (ganzen) Wahrheit trenne,
sei Grundlage des Pluralismus und der Einsicht in
die notwendige Meinungsvielfalt und Toleranz für den
Andersdenkenden.\footnote{Das Wort \enquote{Toleranz} ist hier nicht ohne
Schwierigkeit, zumal \name[Immanuel]{Kant} \enquote{den hochmütigen Namen der
\ori{Toleranz} von sich ablehnt}
(\cite[][A~491]{Kant:BeantwortungderFrage:WasistAufklaerung?1977}, \cite[][VIII:
40.30--31]{Kant:GesammelteWerke1900ff.}). Siehe hierzu
\cite{Weidemann:VonenquotebisweilenunvermeidlicherGeringschaetzung2010}.} Für
\name[Herbert]{Schnädelbach} wie für \name[Norbert]{Hinske} ist es die aus der Einsicht in die
Endlichkeit der Vernunft resultierende Forderung des Pluralismus, die
\name[Immanuel]{Kant} zum nach wie vor aktuellen Proponenten der Aufklärung mache.

Jürgen \name[Jürgen]{Engfer} hingegen sieht in der \enquote{Einsicht in die
Endlichkeit {\punkt} des Menschen} ein Grundproblem, welches der Aufklärung gerade
entgegenstehe. Es sei \name[Christian]{Thomasius} gewesen, der den dahinter stehenden
Konflikt zweier Traditionen erstmals virulent werden ließ: Auf der einen Seite stehe eine
aufklärerische Tradition, die dem Menschen Selbstaufklärung und Selbstbestimmung
abverlange. Auf der anderen Seite aber stehe eine protestantische Tradition, die
dem Menschen als endlichem Wesen gerade die Fähigkeit hierzu
abspreche.\footnote{\cite[Vgl.][36]{Engfer:ChristianThomasius1989}.} Das
Ergebnis sei ein anhaltender Konflikt zwischen Aufklärung und Endlichkeit:
\begin{quote}
  Wenn Thomasius daher -- noch einmal: zu Recht -- als der erste Begründer der
Aufklärung in Deutschland angesehen wird, dann gehört er zu jener Klasse von Begründern
einer Tradition, die ihren Nachfolgern nicht eine Lösung, sondern ein Problem
hinterlassen: in diesem Fall das Problem, wie die Einsicht in die Endlichkeit
und Fehlbarkeit des Menschen, die Thomasius am Ende so stark betont, mit dem
Anspruch auf Selbstaufklärung und Selbstbestimmung zu vereinbaren sei, mit dem
er beginnt.\footcite[36]{Engfer:ChristianThomasius1989}
\end{quote}
Die These von einem inneren Zusammenhang zwischen der Einsicht in die
Endlichkeit und Aufklärungsdenken wird also durchaus bestritten --  bis hin zur
Behauptung eines Widerstreits zwischen beiden.


\name[Immanuel]{Kant} selbst sah sich zweifellos als Aufklärer, das belegen
seine Aufsätze in der Berlinischen Monatsschrift.\footnote{Dagegen behauptet
\authorfullcite{Beyerhaus:KantsenquoteProgrammderAufklaerungausdemJahre17841921},
dass sogar \name[Immanuel]{Kant}s Schrift über die Frage \enquote{Was ist
Aufklärung?} bereits \enquote{in ihrer Tendenz \ori{gegen} die Aufklärung
gerichtet} sei
\parencite[][12]{Beyerhaus:KantsenquoteProgrammderAufklaerungausdemJahre17841921},
weil es ihr an Radikalität mangele und \name[Immanuel]{Kant} sich zu sehr in
Loyalität zum preußischen Staat begebe.} Leider sagt er wenig dazu, wie
sich seine kritische Philosophie zu Begriff und Programm der Aufklärung verhält.
Möglicherweise hat er ja gerade mit seiner Vernunftkritik den Rahmen der
Aufklärung, wie er sie in seinen kleineren Schriften vertritt, unwissentlich
verlassen und einer neuen Epoche den Weg bereitet. Diese Frage ist nicht einfach
mit Textbelegen zu beantworten, da \name[Immanuel]{Kant} nicht hinreichend explizite
Angaben zu dem Zusammenhang von Aufklärungsprogrammatik und Vernunftkritik
macht. Die Frage wendet sich direkt dem \emph{systematischen} Zusammenhang
zweier Gedankenzüge in \name[Immanuel]{Kant}s Denken zu.

Ob es einen systematischen Zusammenhang zwischen Aufklärung und Endlichkeit
gibt, und falls ja, welchen, das bleibt also zu erörtern. Viel hängt daran, was
man unter \enquote{Aufklärung} und unter \enquote{Endlichkeit} versteht und
welche Charakteristica der Endlichkeit und vor allem der Aufklärung man jeweils
als wesentlich und welche man als nebensächlich ansieht.

\name[Immanuel]{Kant} selbst vermeidet die Verwendung des Begriffspaars
\enquote{endlich}/\enquote{unendlich} bei der Charakterisierung unserer
intellektuellen Fähigkeiten respective als Beschreibung der Differenz zwischen
menschlichem und göttlichem
Denken.\footnote{\cite[Vgl.][]{Kant:DereinzigmoeglicheBeweisgrundvomDaseinGottes1977} \cite[][II: 154.4--9]{Kant:GesammelteWerke1900ff.}, dagegen aber
\cite{Kant:KritikderreinenVernunft2003}, \cite[][III:
72.29--73.4]{Kant:GesammelteWerke1900ff.},
\cite{Kant:KritikderpraktischenVernunft1974}, \cite[][V:
32.35--33.5]{Kant:GesammelteWerke1900ff.} und
\cite[][\S~87]{Kant:KritikderUrteilskraft2009}, \cite[][V:
450.10--12]{Kant:GesammelteWerke1900ff.}.} Dennoch bleibt die
Endlichkeitsbehauptung unter anderen Namen präsent:
Er spricht beispielsweise von unserem diskursiven -- im Unterschied zu einem anschauenden oder intuitiven
-- Verstand, unterscheidet einen \emph{intellectus ectypus} von einem \emph{intellectus
archetypus} oder beschreibt unsere Anschauung als sinnlich und nicht
intellektuell. Wer \name[Immanuel]{Kant} als Philosophen der Endlichkeit
betrachtet, bezieht sich auf die Begrenzung des Wissens auf den Bereich
möglicher Erfahrung und die Anerkennung eines Bereichs des Denkbaren, welcher
jenseits der Grenzen unserer Erfahrung liegt. Wenn hier von Endlichkeit und
Unendlichkeit gesprochen wird, so ist damit in erster Linie die Endlichkeit
respective Unendlichkeit des (menschlichen respective göttlichen) \emph{Denkens}
gemeint. Mit dieser wird freilich leicht eine andere Endlichkeit oder
Unendlichkeit in Verbindung
gebracht.\footnote{Eine klassische Äußerungen findet sich bei
\authorfullcite{Descartes:OeuvresdeDescartes1983}, der als selbstverständlich
ansieht, dass er als endliches Wesen das Unendliche -- gemeint ist Gott --
nicht begreifen (\singlequote{\emph{comprehendere}}) könne:
\enquote{} \mkbibparens{\cite{Descartes:OeuvresdeDescartes1983},
\cite{Descartes:OeuvresdeDescartes1983}}.} Es gilt als Ergebnis der
Vernunftkritik, dass der Mensch mit seinen kognitiven Fähigkeiten nicht über den Bereich des Endlichen hinaus gelange und das Unendliche ihm stets verschlossen bleibe: Menschliches Denken sei endliches Denken und der Anspruch, das Unendliche zu erkennen, der Traum einer hinfälligen Metaphysik.
Geträumt habe ihn \authorcite{Hegel:GesammelteWerke} -- so lautet die Fortsetzung dieser Erzählung,
welche in \name[Immanuel]{Kant} und \authorcite{Hegel:GesammelteWerke} die beiden Antipoden erblickt, die jedes
philosophische Selbstverständnis in der deutschsprachigen Philosophie als
Bezugspunkte dominieren.\footnote{Vgl.
\cite{Habermas:WegederDetranszendentalisierung2004,Schnaedelbach:WirKantianer2005},
kritisch dazu \cite[][11--24]{Pleines:VonKantzuHegel2007}.} Zwischen beiden
Seiten kann es nun zu einer Auseinandersetzung um die Frage kommen, ob wir denn tatsächlich jederzeit
und in allem endlich sind und welche Funktion der Gedanke eines Unendlichen
einnimmt. Gibt es vielleicht doch ein Erkennen des Unendlichen?
\authorcite{Hegel:GesammelteWerke}ianer mögen so etwas behaupten wollen, insofern sie die Aufgabe der
Philosophie darin sehen.%
\footnote{\label{anmerkung:hegelsunendlichkeitinderdiskussion}Zu \authorcite{Hegel:GesammelteWerke}
selbst siehe Anmerkung \ref{Anmerkung:hegelsforderungnachunendlichkeit} auf
Seite \pageref{Anmerkung:hegelsforderungnachunendlichkeit}. Für den Anschluss an
\authorcite{Hegel:GesammelteWerke}
\cite[siehe
bspw.][]{Hutter:HegelsPhilosophiedesGeistes2007,Menegoni:DasEndlicheunddasUnendlicheinHegelsDenken2004,Pinkard:AgencyFinitudeandIdealism2003}.
Siehe auch die mit \textcite{Hartmann:Hegel1972} beginnende Diskussion um
metaphysische und nichtmetaphysische Hegelauslegungen. Auf der einen Seite gibt
es Befürworter einer \distanz{Philosophie der Unendlichkeit}.
\cite[Siehe bspw.][104]{Jaeschke:DieUnendlichkeitderSubjektivitaet2004}:
\enquote{Von \enquote{Subjektivität}, vom Begriff oder vom Ich zu reden, heißt
immer schon, vom \enquote{unendlichen Subjekt} zu reden.} Dabei vertritt
\authorcite{Jaeschke:DasabsoluteWissen2004} eine Lesart, die ähnlich der von Terry \name{Pinkard}
(\cite[vgl.][\pno~261f.]{Pinkard:HegelsPhenomenology1994})
versucht, \authorcite{Hegel:GesammelteWerke}s Rede von Unendlichkeit ernstzunehmen, ohne darunter etwas \enquote{Mysteriöses} zu
verstehen, \enquote{sondern diejenige Gestalt des Geistes, in der er sich auf sich
selber richtet, reflexiv wird, um sich zu erkennen}
\parencite[][197]{Jaeschke:DasabsoluteWissen2004}. Dass \enquote{unendlich} bei
\authorcite{Hegel:GesammelteWerke} \enquote{reflexiv} oder \enquote{selbstbezüglich} (\enquote{self-relating}) bedeute, ist
mehrfach behauptet worden \parencite[vgl.\ z.\,B.][]{Davis:HegelsIdealism2012}.
Dagegen gibt es Protest aus unterschiedlichen Richtungen. Jürgen \name{Habermas} wendet
ein, solche Interpretationen seien gerade deshalb nicht mit \authorcite{Hegel:GesammelteWerke} zu
vereinbaren, weil sie selbst endlich Züge haben
\parencite[vgl.][209--219]{Habermas:WegederDetranszendentalisierung2004}. In
der Nähe der Romantik wird \authorcite{Hegel:GesammelteWerke}s Philosophie von
\authorfullcite{Pippin:KantischeTragoedieundHegelscheRomantik2004} verortet.
\name[Immanuel]{Kant}s Philosophie sei \enquote{die Epitome eines tragischen Standpunktes} und
\enquote{eine Philosophie der Endlichkeit.} Dagegen scheine \authorcite{Hegel:GesammelteWerke}s
Philosophie \enquote{ein glückliches Ende zu haben und zur Romantik zu gehören. Die
Philosophie der Endlichkeit [sei] überwunden, und eine Philosophie der
Unendlichkeit [werde] verfochten}
\parencite[][117]{Pippin:KantischeTragoedieundHegelscheRomantik2004}.
Allerdings hält er \authorcite{Hegel:GesammelteWerke}s These zur Unendlichkeit und zu einem
unendlichem Denken für weitgehend irrelevant
(\cite[vgl.][128]{Pippin:KantischeTragoedieundHegelscheRomantik2004}). Auch
\authorfullcite{Fulda:HegelsBegriffdesabsolutenGeistes2001} wendet sich gegen eine deflationäre Lesart
\authorcite{Hegel:GesammelteWerke}s, sieht dessen Forderung nach dem Erkennen des Unendlichen aber als berechtigt an, ohne jedoch unsere
eigene Endlichkeit anzutasten
\parencite[vgl.][171--175]{Fulda:HegelsBegriffdesabsolutenGeistes2001}. Gegen
all diese Lesarten behauptet \authorfullcite{Stekeler-Weithofer:TheQuestionofSystem2006}, dass gerade
\authorcite{Hegel:GesammelteWerke} \emph{der} Philosoph der Endlichkeit sei
\mkbibparens{\cite[vgl.][98]{Stekeler-Weithofer:TheQuestionofSystem2006},
\cite[ähnlich][]{Chiereghin:WozuHegelineinemZeitalterderEndlichkeit?1998}}.
Dagegen wendet sich \cite[][]{Philipsen:NichtsalsKontexte2000}.
Jens \name[Jens]{Halfwassen} sieht in
\authorcite{Hegel:GesammelteWerke}s Unendlichkeitsbegriff, mit dem dieser an neuplatonische
Philosopheme anschließe, die negative Theologie überwunden
\parencite[vgl.][30,~78--83]{Halfwassen:HegelundderspaetantikeNeuplatonismus1999}.
Christian \name[Christian]{Iber} urteilt, \enquote{[m]it der Theorie
der Unendlichkeit versuch[e] \authorcite{Hegel:GesammelteWerke}[,] einen \ori{Beweis} für die Position
des Vernunftidealismus zu führen}
\parencite[][155]{Iber:SubjektivitaetVernunftundihreKritik1999} und entwickle
\enquote{das Grundgerüst seiner Philosophie im ganzen}
\parencite[][156]{Iber:SubjektivitaetVernunftundihreKritik1999}.
} Die Betonung unserer Endlichkeit und der Unerkennbarkeit des Unendlichen
gelten hingegen als Charakteristika des \name[Immanuel]{Kant}ianismus und eines
\distanz{kritischen Weges}. An der Endlichkeit unseres Denkens scheitere die
Fortsetzung der Metaphysik\footnote{Siehe hierzu aus hegelianischer Perspektive
\cite{Bartuschat:HegelsneueMetaphysik2007}.} und an ihr hänge das befreiende
Potenzial einer aufgeklärten, d.\,h.\  pluralistischen, profanen und reflexiv
gewordenen Moderne.%
\footnote{\cite[Vgl.][]{Schnaedelbach:WirKantianer2005}. Nach \name[Herbert]{Schnädelbach}
zeichnet sich die Modernität \name[Immanuel]{Kant}s durch die Momente
vollständiger Reflexivität, Profanität und Pluralität aus
\parencite[vgl.][838--843]{Schnaedelbach:WirKantianer2005}.
\authorfullcite{Jaeschke:DasabsoluteWissen2004} zufolge finden sich Reflexivität
und Profanität (oder Säkularität) hingegen eher bei
\authorcite{Hegel:GesammelteWerke} und ausgerechnet in dem Begriff (wahrer)
Unendlichkeit umgesetzt \parencite[vgl.][]{Jaeschke:DasabsoluteWissen2004}.}

In den letzten Jahren zeigte sich eine Wiederentdeckung
\authorcite{Hegel:GesammelteWerke}s innerhalb der zeitgenössischen
Philosophie\footnote{\cite[Vgl.][]{McDowell:MindandWorld1994,McDowell:HavingtheWorldinView2009,McDowell:Self-DeterminingSubjectivityandExternalConstraint2009,McDowell:HegelsIdealismasRadicalizationofKant2008,Brandom:MakingItExplicit1994,Brandom:TalesoftheMightyDead2002}.
Für den deutschsprachigen Raum siehe
\cite{Stekeler-Weithofer:HegelsAnalytischePhilosophie1992,Grau:EinKreisvonKreisen2001,Quante:HegelsBegriffderHandlung1993,Halbig:ObjektivesDenken2002}.
\cite[Siehe auch den Überblick in][]{Quante:AbsolutesDenken1996}.}, die an die
\name[Immanuel]{Kant}-Renaissance des 20.\ Jahrhunderts
anknüpft.\footnote{\cite[Siehe][]{Sellars:SienceandMetaphysics1968,Strawson:TheBoundsofSense1975}.
\name[Wilfrid]{Sellars} beschreibt -- so berichtet \name[Richard]{Rorty} -- seine eigene
Philosophie als den Versuch \enquote{to usher analytic philosophy out of its
\name[David]{Hume}an and into its \name[Immanuel]{Kant}ian stage}
\parencite[][3]{Sellars:EmpiricismandthePhilosophyofMind1997}.
\authorcite{Brandom:MakingItExplicit1994} möchte von dort aus zu \authorcite{Hegel:GesammelteWerke}
weiter \mkbibparens{vgl. \name[Richard]{Rorty}s Bemerkung in seiner Einleitung
in \cite[][\pno~8f.]{Sellars:EmpiricismandthePhilosophyofMind1997}}.} Diese
Wiederkehr eines oft
totgeglaubten\footnote{\cite[Siehe][]{Schnaedelbach:WarumHegel?1999}, und die
Diskussion in \cite{Stekeler-Weithofer:StreitumHegel2000}.} Autors machte
Erwartungen eines \enquote{Kehraus mit
\authorcite{Hegel:GesammelteWerke}}\footnote{So lautet die Überschrift über die
letzten beiden Vorlesungen in
\cite[][293--357]{Tugendhat:SelbstbewusstseinundSelbstbestimmung1979}.}
zunichte, denen zufolge seine Philosophie nur noch von historischem Interesse
sei. Herbert \name[Herbert]{Schnädelbach} verteidigt dagegen den \name[Immanuel]{Kant}ianismus mit
der Behauptung, es gebe nur \enquote{\enquote{halbe} \authorcite{Hegel:GesammelteWerke}ianer, deren
andere Hälfte \index{Kant, Immanuel}kantianisch
ist}\footnote{\cite[][835]{Schnaedelbach:WirKantianer2005}.}. \authorcite{Hegel:GesammelteWerke}ianismus
scheint nur möglich zu sein, wenn er sich von den ungeheuren Anmaßungen
distanziert, die den Rahmen \noname{Hegel}hegelschen Denkens bilden, und
stattdessen Einsichten rekonstruiert, die \authorcite{Hegel:GesammelteWerke} zu Detailfragen
anbietet.\footnote{\cite[Vgl.][]{Sandkaulen:SystemundSystemkritik2006,Halbig:Einleitung2004,Loewith:AktualitaetundInaktualitaetHegels1973},
dagegen: \cite{Stekeler-Weithofer:TheQuestionofSystem2006}. Gegen ein Unendliches im Sinne des
Ganzen oder Systems der Philosophie wenden sich
\cite{Schnaedelbach:SystemundGeschichte2003,Schneiders:VomWeltweisenzumGottverdammten1988}.} Die in
\authorcite{Hegel:GesammelteWerke}s Philosophie zentrale Forderung einer Überwindung der Endlichkeit%
\footnote{\label{Anmerkung:hegelsforderungnachunendlichkeit}Den prägnantesten
Ausdruck findet diese in der \titel{Wissenschaft der Logik}
\mkbibparens{\cite[vgl.][]{Hegel:WissenschaftderLogikI1986},
\cite[][XXI: 142.15--21]{Hegel:GesammelteWerke}}.} bleibt scheinbar
notwendig unerfüllt.


\phantomsection\label{paragraph:schnaedelbachhegeladornohorkheimer} Jedoch
behauptet \authorcite{Hegel:GesammelteWerke}, \name[Immanuel]{Kant} selbst sei
auf die Möglichkeit eines unendlichen Denkens
gestoßen.\footnote{\cite[Vgl.][]{Hegel:GlaubenundWissenoderdieReflexionsphilosophiederSubjektivitaetinderVollstaendigkeitihrerFormenalsKantischeJacobischeundFichteschePhilosophie1968},
\cite[][IV: 341.7--8, 341.24--34, 344.30--345.1]{Hegel:GesammelteWerke}.} Für
Autoren wie Herbert \name[Herbert]{Schnädelbach} und Martin
\name[Martin]{Heidegger} hingegen ist die Endlichkeit fester Anker kantischer
Philosophie und damit eine eindeutige Differenz zu
\authorcite{Hegel:GesammelteWerke} gegeben. Für \name[Martin]{Heidegger} ist die
Menschlichkeit der Vernunft gerade ihre
Endlichkeit.\footnote{\cite[Vgl.][21]{Heidegger:KantunddasProblemderMetaphysik1965}.}
Schnädelbach sieht die Grundlage für das Festhalten an der Endlichkeit nicht in
den Ergebnissen der Vernunftkritik, sondern in Kants Grundausrichtung als
Aufklärer und \enquote{klassische[r] Philosoph der
Moderne}\footnote{\cite[][838]{Schnaedelbach:WirKantianer2005} im Anschluss an
\cite{Rickert:KantalsPhilosophdermodernenKultur1924}.}; denn für den, der an
diesem Projekt festhält, stehe die Endlichkeit des Menschen von Beginn an fest.
Er scheint dabei durch die Arbeiten von
\authorfullcite{Horkheimer:DialektikderAufklaerung1997} geprägt zu sein,
wenngleich er deren Grundintention in der \titel{Dialektik der Aufklärung} nicht
teilt, sondern geradezu in ihr Gegenteil umkehrt. Während
\name[Herbert]{Schnädelbach} sich explizit auf die Seite der Aufklärung stellt,
machen \authorcite{Horkheimer:DialektikderAufklaerung1997} sie für den Verfall
der westlichen Zivilisation verantwortlich.\footnote{Max \name[Max]{Horkheimer}
hat sich von diesem Urteil später mit aller Schärfe distanziert. Er schreibt:
\enquote{Mir scheint es nicht darauf anzukommen, zur \name[Immanuel]{Kant}ischen
Philosophie zurückzuführen -- der Naukantianismus ist wahrlich überholt --, es
gilt vielmehr, ihre Wahrheit durchsichtig zu machen, in bescheidenem Maß dabei
zu helfen, daß die Aufklärung, die in Deutschland ihr höchstes theoretisches
Bewußtsein durch \name[Immanuel]{Kant} empfing, erst einmal wirklich aufgenommen
wird} \parencite[][206]{Horkheimer:KantsPhilosophieunddieAufklaerung1967},
sowie: \enquote{Keiner war dem Fanatismus der durch Manipulation gelenkten
Massen mehr abhold als der Philosoph der Autonomie. Zur Aufklärung gehört die Gesinnung, die
den Fortschritt der Menschheit an der Entfaltung der geistigen Kräfte der
einzelnen mißt und ihn zugleich zur Verpflichtung eines jeden macht.}
\parencite[][214]{Horkheimer:KantsPhilosophieunddieAufklaerung1967}.
Siehe hierzu zusammenfassend \cite[15--17]{Recki:KantunddieAufklaerung2006}.}
\name[Herbert]{Schnädelbach} übernimmt von ihnen aber die Diagnose, wonach die
Endlichkeit des Denkens das grundlegende Merkmal der Aufklärung ausmache.
\authorcite{Horkheimer:DialektikderAufklaerung1997} sprechen nicht
\emph{expressis verbis} von der \emph{Endlichkeit} des Denkens, sondern von
\enquote{fortschreitendem Denken}\footnote{Z.\,B. \cite[][19,
41]{Horkheimer:DialektikderAufklaerung1997}.} und \enquote{diskursive[r]
Logik}\footnote{Z.\,B. \cite[][26,
30]{Horkheimer:DialektikderAufklaerung1997}.}. So
beginnt die \titel{Dialektik der Aufklärung} mit den Worten: \enquote{Seit je
hat Aufklärung im umfassenden Sinn fortschreitenden Denkens das Ziel verfolgt,
von den Menschen die Furcht zu nehmen und sie als Herren
einzusetzen.}\footcite[][19]{Horkheimer:DialektikderAufklaerung1997} Beachtet man, dass
\enquote{fortschreitend} die Übersetzung von \enquote{diskursiv}
ist\footnote{Hierauf wie auf die Bedeutung, die
\authorcite{Horkheimer:DialektikderAufklaerung1997} der Diskursivität des
Denkens als \emph{definiens} der Aufklärung beimessen, wies mich Dietrich
\name[Dietrich]{Schotte} hin.} und die Diskursivität des Denkens nach
\name[Immanuel]{Kant} dessen Endlichkeit ausmacht, dann erhellt dennoch, dass in
der \titel{Dialektik der Aufklärung} die Endlichkeit des Denkens zum definierenden
Merkmal der Aufklärung erklärt wird. Und während der emanzipatorische Anspruch
der Aufklärung sicherlich lobenswert ist, so wird der vermeintlichen Verengung
auf das endliche Denken unter Rückgriff auf \authorcite{Hegel:GesammelteWerke}s
Bild von einem notwendigen Übergang der Aufklärung in den
\enquote{Schrecken}\footnote{\cite[Vgl.][533]{Hegel:PhaenomenologiedesGeistes1980},
\cite[][IX: 316.10--11]{Hegel:GesammelteWerke}: \enquote{Die absolute Freyheit
und der Schrecken.}} die Schuld an vielen Übeln und zentralen Katastrophen der
Moderne zugeschrieben -- von der Jakobinerherrschaft bis zur Shoah.
Aufklärung wollte den Menschen durch die Kraft des Denkens zum Herrscher über
die Natur machen; doch durch ihre Konzeption des Denkens als eines diskursiven,
in Allgemeinbegriffen sich bewegenden vermöge sie den Menschen selbst nur noch
hinsichtlich seiner Nützlichkeit zu sehen -- mit fürchterlichen Folgen für die
menschliche
Zivilisation.\footnote{\cite[Vgl.][29]{Horkheimer:DialektikderAufklaerung1997}:
\enquote{Die Abstraktion, das Werkzeug der Aufklärung, verhält sich zu ihren
Objekten wie das Schicksal, dessen Begriff sie ausmerzt: als Liquidation. Unter
der nivellierenden Herrschaft des Abstrakten, die alles in der Natur zum
Wiederholbaren macht, und der Industrie, für die sie es zurichtet, wurden
schließlich die Befreiten selbst zu jenem \singlequote{Trupp}, den
\authorcite{Hegel:GesammelteWerke} als das Resultat der Aufklärung bezeichnet
hat.}}



Wo nun Aufklärung als historisches Phänomen dieser einfachen Darstellung bei
\authorcite{Horkheimer:DialektikderAufklaerung1997} widerspricht (was sie
erwartungsgemäß recht häufig tut), machen diese von einem einfachen Schema Gebrauch: Die Dialektik beschreibt die konsequente Aufklärung,
wohingegen die Realität fast nur inkonsequente, gemäßigte Realisierungen zeige,
da viele Autoren vor den Konsequenzen ihres Denkens zuschreckten und in
Doppeldeutigkeiten Zuflucht
suchten.\footnote{\cite[Vgl.
z.\,B.][\pno~102f.:]{Horkheimer:DialektikderAufklaerung1997}
\enquote{\name[Immanuel]{Kant}s Begriffe sind doppelsinnig. Vernunft als das
transzendentale überindividuelle Ich enthält die Idee eines freien
Zusammenlebens der Menschen, in dem sie zum allgemeinen Subjekt sich
organisieren und den Widerstreit zwischen der reinen und empirischen Vernunft in
der bewußten Solidarität des Ganzen aufheben. {\punkt} Zugleich jedoch bildet
Vernunft die Instanz des kalkulierenden Denkens, das die Welt für die Zwecke der
Selbsterhaltung zurichtet und keine anderen Funktionen kennt als die der
Präparierung des Gegenstandes aus bloßem Sinnenmaterial zum Material der
Unterjochung.}} Auf diese Weise gelingt es wenigstens scheinbar, das
vielgestaltige Phänomen der Aufklärung in ein so simples Schema zu pressen. Hier
liegt ein offensichtlicher Kritikpunkt an jeder Rede von einer Dialektik der
Aufklärung: Von \singlequote{\emph{der} (einen) Aufklärung} zu sprechen kann
einem solch breiten geistesgeschichtlichen Phänomen gar nicht gerecht werden. Es
ist gar nicht a priori auszumachen, dass es einen einheitlichen Begriff der
Aufklärungsphilosophie gebe, dass dieser sich durch ein einziges Merkmal
charakterisieren lasse und dass es sich dabei um die \index{Kant,
Immanuel}kantische Differenzierung von Denken und Anschauung -- die
\singlequote{Endlichkeit} (\name[Herbert]{Schnädelbach}) oder \singlequote{Diskursivität}
(\authorcite{Horkheimer:DialektikderAufklaerung1997}) des Denkens -- handle.

Die groben Linien ihrer Aufklärungskritik entnehmen \authorcite{Horkheimer:DialektikderAufklaerung1997} der
Darstellung \authorcite{Hegel:GesammelteWerke}s in der \titel{Phänomenologie des
Geistes}\footnote{\cite[Vgl.][486--547]{Hegel:PhaenomenologiedesGeistes1980},
\cite[][IX: 292.11--323.21]{Hegel:GesammelteWerke}.}, aber auch schon in
\titel{Glauben und Wissen}\footnote{Vgl.
\cite{Hegel:GlaubenundWissenoderdieReflexionsphilosophiederSubjektivitaetinderVollstaendigkeitihrerFormenalsKantischeJacobischeundFichteschePhilosophie1968},
\cite[][IV: 315.5--324.33]{Hegel:GesammelteWerke}.} und in der Einleitung der
\titel{Grundlinien zur Philosophie des
Rechts}\footnote{\cite[Vgl.][\S\S~5--21]{Hegel:GrundlinienderPhilosophiedesRechts1970},
\cite[][XIV,1: 32.11--42.6]{Hegel:GesammelteWerke}. Hier spricht
\authorcite{Hegel:GesammelteWerke} nicht \emph{expressis verbis} von Aufklärung,
thematisiert aber diegleichen Zusammenhänge wie in den anderen genannten
Abschnitten.}. Im Unterschied zu \name[Herbert]{Schnädelbach} sehen
\authorcite{Horkheimer:DialektikderAufklaerung1997} in
\authorcite{Hegel:GesammelteWerke}s Philosophie jedoch gerade nicht die
\emph{Überwindung} der Endlichkeit des Denkens, sondern deren
\enquote{Apotheose}.\footcite[Vgl.][41]{Horkheimer:DialektikderAufklaerung1997}
\authorcite{Hegel:GesammelteWerke} selbst wiederum scheint mir nicht die These
zu vertreten, dass Aufklärung das Denken nur als endliches in den Blick bekomme.
Vielmehr scheint er in der Aufklärung (und dort inbegriffen auch bei
\name[Immanuel]{Kant}) ein Beispiel für das zu sehen, was er
\singlequote{schlechte Unendlichkeit} nennt: das einander Gegenüberstehen von
Endlichkeit und Unendlichkeit.\footnote{Siehe hierzu die Einleitung in
\titel{Glauben und Wissen}:
\cite[][3--21]{Hegel:GlaubenundWissenoderdieReflexionsphilosophiederSubjektivitaetinderVollstaendigkeitihrerFormenalsKantischeJacobischeundFichteschePhilosophie1968},
\cite[][IV: 315--324]{Hegel:GesammelteWerke}.} Wenngleich die zur Diskussion
gestellte Diagnose einer Dialektik der Aufklärung von
\authorcite{Hegel:GesammelteWerke} stammt, so doch nicht
\name[Herbert]{Schnädelbach}s Beschreibung der Philosophie der Aufklärung als
einer Philosophie der Endlichkeit. Denn \authorcite{Hegel:GesammelteWerke}
vindiziert der Aufklärung mitnichten eine Beschränkung auf
Endliches.\footnote{Die Behauptung der Aufklärung, eine Philosophie der
Endlichkeit zu vertreten, wäre dann eher Ausdruck einer Täuschung der Aufklärung
über sich selbst; vgl. \cite{Hegel:PhaenomenologiedesGeistes1980}, \cite[][IX:
306.28--29]{Hegel:GesammelteWerke}: \enquote{Die Aufklärung {\punkt} ist eben so
wenig über sich selbst aufgeklärt.}} Zwar fasse sie im Erkennen die Dinge der
Anschauung (Hain, Hölzer, Brotteig\dots), auch dann wenn sie Gegenstände
religiöser Verehrung sind, allesamt als endliche
auf\footnote{\cite{Hegel:GlaubenundWissenoderdieReflexionsphilosophiederSubjektivitaetinderVollstaendigkeitihrerFormenalsKantischeJacobischeundFichteschePhilosophie1968},
\cite[][IV: 316.10--317.15]{Hegel:GesammelteWerke}; siehe auch
\cite[][501]{Hegel:PhaenomenologiedesGeistes1980}, \cite[][IX:
300.2--7]{Hegel:GesammelteWerke}: \enquote{Sie sagt hiernach über den Glauben,
daß sein absolutes Wesen ein Steinstück, ein Holzblock sey, der Augen habe und
nicht sehe, oder auch etwas Brodteig, der auf dem Acker gewachsen, von Menschen
verwandelt darauf zurückgeschickt werde; -- oder nach welchen Weisen sonst der
Glauben, das Wesen anthopomorphisiere, sich gegenständlich und vorstellig
mache.}}, worin sie das Wesen der Religion
verkenne\footnote{\cite[Vgl.][\pno~501f.]{Hegel:PhaenomenologiedesGeistes1980},
\cite[][IX: 300.8--18]{Hegel:GesammelteWerke}.}.
Doch gehe sie im Denken durchaus über diese Anschauung hinaus -- ebenso wie der Glaube, von dem sie sich in Wahrheit kaum
unterscheide.\footnote{\cite[Vgl.][]{Hegel:PhaenomenologiedesGeistes1980},
\cite[][IX: ]{Hegel:GesammelteWerke}: \enquote{}??????????} Aufklärung ziele wie
ihr Gegenspieler -- der Glaube -- auf das Unendliche, welches sie aber falsch,
nämlich in Entgegensetzung zum Endlichen auffasse. Das Unendliche der Aufklärung
sei daher das schlechte, nicht das wahre Unendliche. Und weil es als schlechtes
Unendliches inhaltsleer bleibe, sei der Sieg der Aufklärung letztlich nur ein
äußerlicher; denn die Inhalte bestimme weiterhin die vermeintlich besiegte
Religion.

Die Gleichung \singlequote{Aufklärungsphilosophie $=$ Philosophie der Endlichkeit} operiert
auf beiden Seiten mit einem unklaren Begriff: Weder was Aufklärung noch was
Endlichkeit (des Denkens) ist, darf als verständlich vorausgesetzt werden. Es
lässt sich gar nicht oft genug betonen, dass es \emph{die} Aufklärung als
homogenes Gebilde nie gegeben hat und nie geben
konnte.\footnote{Siehe dazu
\cite{Fulda:GabesenquotedieAufklaerung2013}.\luecke{Jonathan Clark und Dan
Edelstein fehlen}} Selbst die feineren Gruppierungen von Autoren zu einer
schottischen, einer französischen und einer deutschen Aufklärung sind als Orientierungen mit äußerster Vorsicht zu betrachten. Ebenso ist nicht leicht anzugeben, worin die Endlichkeit des Menschen besteht und was Inhalt der Forderung einer Überwindung der
Endlichkeit sein kann. Um Bescheidenheitsappelle, wie sie dem
\name[Immanuel]{Kant}ianismus bei \name[Herbert]{Schnädelbach} und anderen
zugrunde liegen, nicht zur leeren Worthülse werden zu lassen, müsste ihre
Bedeutung aber erst expliziert werden. Denn was macht etwa unser Denken zu einem
endlichen? Ist es die Fallibilität, die im 20.\ Jahrhundert so oft betont wurde?
Ist es die Tatsache, dass wir nicht alles wissen können? Dass es vielleicht
Bereiche gibt, in denen wir gar nichts wissen können? Was behauptete jemand, der
ein Denken postulierte, das über das Endliche hinausgeht? Behauptete er, alles
zu wissen oder immun gegen Irrtümer zu sein? Bedeutet Endlichkeit im Handeln,
dass wir mitunter Unmoralisches wollen oder tun? Dass wir von Natur aus sündhaft
seien? Oder dass wir in der Ausführung unserer Handlungen mitunter scheitern?
Dass wir auf Mithilfe angewiesen und unsere Kräfte begrenzt sind?


In der philosophischen Tradition wurde Endlichkeit als
\textit{Begrenztheit}\footnote{Klassisch hierfür ist \authorcite{Descartes:DiscoursdelaMethode2011}, der
unseren Verstand (intellectus) als endlich charakterisiert, insofern er nur
einen Teil dessen erkennt, worüber wir urteilen können
\parencite[vgl.][VII:~56.9-58.25]{Descartes:OeuvresdeDescartes1983}. Eine
explizite Definition des Endlichkeitsbegriffs im Sinne der Begrenztheit, die sich nicht
nur auf das Denken bezieht, gibt
\textcite[][\nopp 1d2]{Spinoza:EthikingeometrischerOrdnungdargestellt2007}:
\enquote{Ea res dicitur in suo genere finita, quae alia ejusdem naturae terminari
potest.} Daraus folgt schließlich, dass die Substanz, als das, \enquote{quod in se
est et per se concipitur} \parencite[][\nopp
1d3]{Spinoza:EthikingeometrischerOrdnungdargestellt2007}, unendlich ist
\parencite[vgl.][\nopp 1p8]{Spinoza:EthikingeometrischerOrdnungdargestellt2007},
während alle endlichen Dinge in der einen unendlichen Substanz sind
\parencite[vgl.][\nopp
1p15]{Spinoza:EthikingeometrischerOrdnungdargestellt2007}.} (in seinen
praktischen und kognitiven Fähigkeiten, räumlich, zeitlich etc.), als
\textit{Abhängigkeit}\footnote{Diese wird in der christlichen Lehre von der
Geschöpflichkeit des Menschen am ausführlichsten thematisiert.} (von Anderen,
von gesellschaftlichen Praxisformen, von der Natur, der Welt, von Gott etc.) und
auf manche andere Weise\footnote{Von einigem Gewicht sind folgende
Verwendungen des Begriffs der \emph{Un}endlichkeit: (i) V.\,a.\ die theologische
Tradition kennt den Ausdruck \enquote{unendlich} auch als Kennzeichen für die
Zuschreibung von Prädikaten \emph{via eminentia}. Gott sei unendlich gütig heißt
demnach, ihm komme das Prädikat der Güte zu, aber auf eine eminente Weise,
d.\,h.\ in einem ganz anderen, \distanz{höheren} Sinne, als Güte uns Menschen
zukommt. (ii) Die neuzeitliche Logik spricht von \emph{unendlichen Urteilen}
(iudicia infinita, indefinita, limitativa), welche oberflächlich bejahende
Urteile sind, deren Prädikat jedoch eine Verneinung enthält und daher nicht
angibt, was das Subjekt ist, sondern bloß, was es nicht ist. Während dies in der
allgemeinen Logik vernachlässigt werden kann, interessiert es innerhalb einer
transzendentalen Logik
\mkbibparens{\cite[vgl.][B~97f.]{Kant:KritikderreinenVernunft2003}, \cite[][III:
88.3--32]{Kant:GesammelteWerke1900ff.}; vgl. außerdem
\cite[][A~160--162]{Kant:ImmanuelKantsLogik1977}, \cite[][IX:
103.23--104.24]{Kant:GesammelteWerke1900ff.}, und dessen Ursprung in %
\cite[][\nopp 3065]{Kant:Reflexionen1900ff.}, \cite[][XVI:
639.2--5]{Kant:GesammelteWerke1900ff.}, als Anmerkung zu
\textcite[][\S~294]{Meier:AuszugausderVernunftlehre1752}}. (iii) Dann heißt
\enquote{unendlich} auch soviel wie \enquote{unbestimmt}, was die Richtung weiterverfolgt. (iv) Bei
einigen Autoren wird \enquote{Unendlichkeit} oder \enquote{das Unendliche} auch als
Ausdruck für die Wahrheit oder das Wahre verwendet. (v) Eine weitere Bedeutung
ist \enquote{Wirklichkeit} (actualitas).} beschrieben. Aber eine solche Angabe
macht uns unsere Endlichkeit noch nicht verständlich, sondern spannt den
Horizont möglicher Antworten auf, ohne diese systematisch zu verbinden oder gar
zu sagen, was Wesen und Grund unserer Endlichkeit ist. Gerade heute scheint der
Begriff eines qualitativen oder metaphysischen Unendlichen\footnote{Die
Unterscheidung von mathematischem und qualitativem Unendlichen ist uns primär
von \authorcite{Hegel:GesammelteWerke} geläufig, in Wahrheit aber schon zuvor
verbreitet gewesen.
\textcite[][\S~248]{Baumgarten:Metaphysica---Metaphysik2011} bezeichnet das
mathematisch Unendliche als indefinitum und infinitum imaginarium. Auch
\name[Immanuel]{Kant} übernimmt die Unterscheidung
\mkbibparens{\cite[vgl.][]{Kant:DereinzigmoeglicheBeweisgrundvomDaseinGottes1977}
\cite[][II: 154.4--9]{Kant:GesammelteWerke1900ff.}}. Bei der Unterscheidung von
menschlichem und göttlichem Denken interessiert gerade das metaphysische oder
qualitative, nicht das mathematische oder quantitative Unendliche. In \titel{Was
heißt: sich im Denken orientieren?} spricht \name[Immanuel]{Kant} hingegen Gott
\enquote{\ori{Unendlichkeit} der Größe nach zur Unterscheidung von allem Geschöpfe} zu
\mkbibparens{\cite[][A 322]{Kant:Washeisst:SichimDenkenorientieren?1977},
\cite[][VIII: 142.30]{Kant:GesammelteWerke1900ff.}} Dort wird jedoch nicht der
göttliche Verstand beschrieben, sondern begründet, warum Gott in unserer
Wahrnehmung nicht als solcher identifizierbar sei.} und damit der Begriff
unserer eigenen Endlichkeit jede Verständlichkeit verloren zu haben. Verschärft wird dies dadurch, dass der mathematische Unendlichkeitsbegriff leicht mit dem metaphysischen konfundiert wird. Eine Klärung des Endlichkeitsbegriffs, wie er
als philosophischer Terminus im 18. Jahrhundert kursierte, scheint mir dabei
auch systematisch, also über jedes historische Interesses hinaus geeignet zu
sein, die Frage nach unserer Endlichkeit wieder zu einem klar umrissenen Problem
der Philosophie werden zu lassen.

\name[Immanuel]{Kant} selbst thematisiert die Besonderheiten unseres endlichen Denkens
mehrfach. In der \titel{Kritik der reinen Vernunft} betont er die Diskursivität
unseres Verstandes durch Abgrenzung gegen einen anschauenden Verstand\footnote{\cite[Vgl.][]{Kant:KritikderreinenVernunft2003},
\cite[][III: 112.20--33]{Kant:GesammelteWerke1900ff.}; ob der in der
\cite{Kant:KritikderreinenVernunft2003} \parencite[][III: 456.37]{Kant:GesammelteWerke1900ff.}
angesprochene und als gesetzgebende Vernunft bezeichnete intellectus archetypus mit dem gleichnamigen
Vermögen in der KU identisch ist, ist fraglich. In
\cite[][\S~21]{Kant:KritikderreinenVernunft2003}, \cite[][III:
116.23--29]{Kant:GesammelteWerke1900ff.}, spricht er außerdem noch von der
Besonderheit unseres Verstandes, nur durch Kategorien und gerade durch
diejenigen, die wir haben, erkennen zu können.} und die Besonderheit unserer
Anschauung als sinnlicher und nicht intellektueller%
\footnote{\cite[Vgl.][]{Kant:KritikderreinenVernunft2003}, \cite[][III:
24.16--21, 72.29--73.2, 210.21--34, 225.35]{Kant:GesammelteWerke1900ff.}.
\cite[In][III: 72f.]{Kant:GesammelteWerke1900ff.} identifiziert die
intellektuelle Anschauung mit der Anschauung eines nicht-endlichen Wesens.}. In der praktischen
Philosophie kontrastiert er unseren endlichen einem heiligen Willen, für den es
keine Imperative gebe.\footnote{\cite[Vgl.][]{Kant:KritikderpraktischenVernunft1974}, \cite[][V:
32.16--33.5, 79.27--35]{Kant:GesammelteWerke1900ff.}.} Und in der Kritik der
Urteilskraft fasst er die Endlichkeit der theoretischen und der praktischen
Vernunft zusammen und verbindet sie mit einer weiteren Form unserer Endlichkeit
-- der teleologischen Darstellung der (lebendigen) Natur.%
\footnote{\cite[Vgl.][\S~77]{Kant:KritikderUrteilskraft2009}, \cite[][V:
405--410.11]{Kant:GesammelteWerke1900ff.}.} Gerade diese Passagen sind es,
welche den Anknüpfungspunkt für die Entwicklung hin zum \distanz{objektiven} und
\distanz{absoluten} Idealismus darstellen.\footnote{Vgl.\
\cite{Foerster:DieBedeutungvonSS7677deremphKritikderUrteilskraftfuerdieEntwicklungdernachkantischenPhilosophieTeil12002,Foerster:DieBedeutungvonSS7677deremphKritikderUrteilskraftfuerdieEntwicklungdernachkantischenPhilosophieTeil22002,Engfer:MenschlicheVernunft2002}.}
Die genannte Charakterisierung der kantischen Endlichkeitsbehauptung wird sich
bei ihrer Analyse als problematisch erweisen. 
Durchaus verbreitet ist die Lesart, wonach das
Unendliche und Unbedingte \distanz{hinter} den Dingen der Welt feststehe und von
unserem endlichen Denken nur geahnt und vermutet, nicht aber erkannt werden
könne, oder als allgemeine Menschenvernunft, unmittelbare Einsicht oder Individualität
\distanz{in uns} vorliege.\footnote{Wegen dieser doppelten Möglichkeit bilden
die \enquote{Reflexionsphilosophien} nach \authorcite{Hegel:GesammelteWerke} bei
\name[Immanuel]{Kant},
\authorfullcite{Jacobi:UeberdieLehredesSpinozainBriefenandenHerrnMosesMendelssohn2000}
und \authorcite{Fichte:DieBestimmungdesMenschen1800} wiederum Gegensätze unter
sich
\mkbibparens{\cite[vgl.][]{Hegel:GlaubenundWissenoderdieReflexionsphilosophiederSubjektivitaetinderVollstaendigkeitihrerFormenalsKantischeJacobischeundFichteschePhilosophie1968},
\cite[][IV: 321.1--18]{Hegel:GesammelteWerke}}.} Dies klingt bescheiden, liberal
und pluralistisch, aber es widerstreitet dem Aufklärungsprogramm, weil es nicht
zu Mündigkeit führt, sondern zu Beliebigkeit. Es ist aber weder Kants Position,
noch ist es aus der Perspektive einer an Kant orientierten, aufklärerischen
Philosophie wünschenswert anzunehmen, es gebe den Bereich des Erkennbaren und
daneben ein unerkennbares Reich des Absoluten, Unendlichen, bloß Denkbaren. Die
Irreführung liegt darin, die \emph{interne Differenzierung} unseres Wissens mit
einer \emph{externen Begrenzung} zu verwechseln. Freilich ist diese Irreführung
in \name[Immanuel]{Kant}s Darstellung angelegt, aber sie vereitelte, wäre sie zwingend,
jedes sinnvolle Verständnis seiner Aufklärungsprogrammatik.\footnote{Das
Verhältnis von Denken und Erkennen ist daher gar nicht so einfach zu bestimmen,
mag auch mancher befürchten, dass man schnell in \enquote{eine sophistische
Deformation genau derjenigen Differenz} gerate, \enquote{die die Endlichkeit unserer
Vernunft anzeigt} \parencite[][843]{Schnaedelbach:WirKantianer2005}.} Denn die
interne Differenzierung lässt sich in der Tat nur aus der Perspektive der
Aufklärungsphilosophie verstehen.



% \subsection{Fragestellung}\label{paragraph:fragestellung}
Die zentrale Frage dieser Arbeit lautet: Inwiefern ist \name[Immanuel]{Kant} der Philosoph
der Endlichkeit? Teilfragen sind: Wie ist bei \name[Immanuel]{Kant} unsere Endlichkeit
bestimmt? Wie ergibt sie sich? Haben Endlichkeit im Erkennen und Endlichkeit im
Handeln eine gemeinsame Wurzel? Welche Funktion übernimmt der Gedanke eines
unendlichen Denkens? Welche Konsequenzen ergeben sich für die Möglichkeit von
Metaphysik? Ist die Endlichkeitsbehauptung tatsächlich Grundlage einer
aufgeklärten Moderne? Welche Probleme ergeben sich aus dieser Vorstellung von
Endlichkeit?

% \subsection{Thesen der Arbeit}\label{paragraph:thesenderarbeit}
Man muss zunächst sehen, dass die Frage nach der Endlichkeit des Menschen mit
seiner Autonomie und Selbständigkeit im Handeln wie auch im Denken hervortritt.
Nicht die Fehlbarkeit oder Reichweite unseres Denkens und Erkennens
stehen am Anfang einer Bestimmung des Endlichkeitsbegriffs, sondern unsere
Abhängigkeit. Und dies hat Folgen dafür, wie die Vernunftkritik die Ansprüche
von Aufklärung und Philosophie zu vermitteln versucht:
Um der Aufklärungsprogrammatik \name[Immanuel]{Kant}s gerecht zu werden, darf
man die Vernunftkritik nicht im Sinne einer externen Begrenzung, sondern muss
sie als eine interne Differenzierung unseres Wissens
verstehen.\footnote{\cite[Zur Relevanz und Aktualität einer solchen Position
siehe][8--11]{Abel:KnowledgeResearch2012}.} Sonst verwechselte man Selbstdenken
mit Beliebigkeit und vereitelte jeden Versuch, die Aporie liberaler Aufklärung
aufzulösen: Einerseits soll der mündige Mensch selbst denken, statt sich von
Anderen leiten zu lassen, andererseits ist aber jeder in seinem Denken von
Anderen, von Traditionen, Gebräuchen und Überlieferungen abhängig.%
\footnote{\cite[Vgl.][]{Foucault:DieOrdnungdesDiskurses1998}, der die restringierenden
Grundlagen des je eigenen Denkens primär als einschränkende Mächte schildert.
Dieselbe Grundlage hat das \distanz{\name[Ernst-Wolfgang]{Böckenförde}-Diktum}, eine liberale
Gesellschaft sei von Voraussetzungen abhängig, die sie nicht selbst garantieren
könne
\parencite[vgl.][60]{Boeckenfoerde:DieEntstehungdesStaatesalsVorgangderSaekularisation1976};
siehe \cite{Habermas:VorpolitischeGrundlagendesdemokratischenRechtsstaates?2005,Habermas:DialektikderSaekularisierung2005,Habermas:VorpolitischeGrundlagendesdemokratischenRechtsstaates?2005}.
In dieselbe Kerbe schlägt jede Kritik an einem aufklärerischen Individualismus
von \name[Jean-Jacques]{Rousseau} über \authorcite{Hegel:GesammelteWerke} und
\name[Karl]{Marx} bis hin zu modernen Kommunitaristen 
\parencite[siehe
bspw.][]{MacIntyre:WhoseJustice?WhichRationality?1988,Taylor:SourcesoftheSelf1989}.}
Denn Denken gibt es nur als kompetente Teilnahme an Formen des Gebens und
Forderns von Gründen. Dies beschreibt eine, vielleicht die entscheidende
Dimension unserer Endlichkeit. Die Aufklärungsaporie ergibt sich aus der
Spannung zwischen der Autonomieforderung und der Endlichkeit des Menschen. Sie
aufzulösen hieße, unsere Endlichkeit in einer Weise zu erläutern, die zugleich
die Möglichkeit der Autonomie endlicher Wesen aufzeigt.

Unendlichkeit ist seit \authorfullcite{Fichte:DieBestimmungdesMenschen1800} der
Ausdruck für die Spontaneität einer autonomen Vernunft, dem Leitbild der
Aufklärungsprogrammatik\footnote{\cite[Vgl.][]{Pinkard:AgencyFinitudeandIdealism2003}.},
und dies ist der Anknüpfungspunkt nicht nur \authorcite{Fichte:DieBestimmungdesMenschen1800}s, sondern auch
\authorcite{Hegel:GesammelteWerke}s in \titel{Glauben und Wissen}.\footnote{Der Zusammenhang mit
Fragen der Freiheit und Mündigkeit wird innerhalb des Einleitungsaufsatzes in
das Kritische Journal noch deutlicher, wo es um die Erarbeitung der
\distanz{Idee der Philosophie} geht \parencite[vgl.][IV:
117--128]{Hegel:GesammelteWerke}. Dass es \authorcite{Hegel:GesammelteWerke}
bei der Explikation des Begriffs wahrer Unendlichkeit zumindest \emph{auch} um einen Begriff von
Freiheit geht, der die Naivitäten des \index{Kant, Immanuel}kantischen Begriffs hinter
sich lässt, erfahren wir in der Rechtsphilosophie \parencite[vgl.][XIV,1:
31--49]{Hegel:GesammelteWerke}.} Somit kennzeichnet die Frage nach
\distanz{wahrer} Unendlichkeit das Problem eines vernünftigen Verständnisses
freien Denkens.\footnote{Bei \authorcite{Hegel:GesammelteWerke} wird die Unendlichkeit an mehreren
Stellen zum zentralen Thema der Untersuchung. Die Wissenschaft der Logik
thematisiert sie in der Lehre vom Sein zunächst als qualitative und dann als
quantitative (mathematische) Unendlichkeit \parencite[vgl.][XXI: 124--143,
218--309]{Hegel:GesammelteWerke}. In der Philosophie des Geistes markiert die
Unendlichkeit den Übergang zum dritten Teil, dem absoluten Geist, und fundiert
dabei die Rechtsphilosophie \parencite[vgl.][XX:
379.5--19, 383.10--385.3, 476.4--477.37, 479.16--18]{Hegel:GesammelteWerke}. 
Dabei ist die Verwendungsweise des Unendlichkeitsbegriff der Rechtsphilosophie, wie
sie in den \S\S~4--27 der Grundlinien entwickelt werden \parencite[vgl.][XIV,1:
31.8--45.2]{Hegel:GesammelteWerke}, diejenige, die für das hier verhandelte
Thema von größter Relevanz ist. Daneben liefert v.\,a. die Entwicklung des
Begriffs der wahren Unendlichkeit in Form des Fürsichseins als qualitative
Unendlichkeit in der Seinslogik begrifflich-strukturelle Grundlagen
\parencite[vgl.][XXI: 144.3--4]{Hegel:GesammelteWerke}.
Das mathematische Unendliche hebt \authorcite{Hegel:GesammelteWerke} besonders wegen seiner
Vorbildfunktion hervor \parencite[vgl.][XXI: 237.4--5]{Hegel:GesammelteWerke}.}
Wie ist es möglich, dass ein Erkenntnisvermögen durch etwas beschränkt wird, ohne
dabei den Charakter der Spontaneität zu verlieren? Wie kann ein Vermögen der
Spontaneität beschränkt werden? Wie kann der Verstand durch die Sinne beschränkt
werden? Kann die Vernunft durch Traditionen und Gebräuche beschränkt werden und
zugleich ihren Charakter als Spontaneität bewahren? Legt man die generelle
Abhängigkeit unseres Denkens von vorgefundenen Praxisformen zugrunde, so lässt
sich auch von der anderen Seite aus fragen: Wie ist es möglich, Autonomie zu
explizieren, ohne sie als epistemischen und kognitiven Individualismus
misszuverstehen? Gerade dies stellt den Hintergrund der Rede von der Endlichkeit
des menschlichen Denkens dar. Ein Ziel dieser Arbeit ist es, genau diese
Fragerichtungen auch als solche \name[Immanuel]{Kant}s zu rekonstruieren und zu
zeigen, dass Endlichkeit gerade in diesem Sinne Aufklärung zum Problem werden
lässt.


\section{Inhaltliches Vorgehen}
Ich beginne mit der Einordnung \name[Immanuel]{Kant}s in das Projekt der
Aufklärung, indem ich charakterisiere, wie \name[Immanuel]{Kant} das Projekt der
Aufklärung konzipiert (Kapitel \ref{section:KantalsliberalerAufklaerer}). Es
wird sich dabei (am Ende von Kapitel \ref{subsection:SelbstdenkenbeiKant}) als
wichtigste Frage ergeben, welche das weitere Vorgehen bestimmt: Wie kann unser
Denken und Erkennen erstens autonom (Kapitel
\ref{section:autonomieunddaszeugnisanderer}) und zweitens authentisch (Kapitel
\ref{section:AufklaerungundRationalismus}) sein? Dabei wird bereits der
systematische Rahmen der Artikulation der Endlichkeit menschlichen Denkens
aufgedeckt und auf entsprechende Formen unserer Endlichkeit verwiesen:
Menschliches Denken ist endlich, insofern seine Überführung in Handlungen durch
antogonistische Neigungen gehemmt wird (so ein Ergebnis von Kapitel
\ref{section:KantalsliberalerAufklaerer}) In diesem Punkt wird Aufklärung zur
Aufgabe für eine endliche Vernunft (und deren Endlichkeit zu einem Hindernis der
Aufklärung) nicht insofern unsere Endlichkeit uns am Erwerb bestimmter
Einsichten hinderte, sondern insofern wir wegen unserer Endlichkeit in ihrer
praktischen \emph{Umsetzung} scheitern. Außerdem ist menschliches Denken
endlich, insofern ihm empirische Inhalte \emph{gegeben} werden. Dies kann durch
die Sinne oder andere Menschen geschehen (testimoniales Wissen, Kapitel
\ref{section:autonomieunddaszeugnisanderer}). Drittens ist es endlich, insofern
es auch dort von anderen abhängig ist, wo seine Inhalte solche \emph{a priori}
sind (Kapitel \ref{section:AufklaerungundRationalismus}).

Es ergeben sich schließlich drei Aspekte unserer Endlichkeit, deren Relevanz als
Hindernisse von Mündigkeit und Selbstbestimmung in den Kapiteln
\ref{section:KantalsliberalerAufklaerer} bis
\ref{section:AufklaerungundRationalismus} augezeigt wird:
\begin{enumerate}
\item Wir sind endlich, insofern der Realisierung des als vernünftig
Eingesehenen antagonistische Neigungen entgegenstehen (Kapitel \ref{section:KantalsliberalerAufklaerer}).
\item Wir sind endlich, insofern wir nicht alle Inhalte aus unserem Denken
generieren können, sondern auf Informationen von außen angewiesen sind. Diese
Form der Endlichkeit kommt insbesondere bei \name[Immanuel]{Kant}s Behandlung
des Themas \enquote{testimoniales Wissen} zum Tragen
(Kapitel \ref{section:autonomieunddaszeugnisanderer}).
\item Wir sind endlich, insofern wir selbst dort, wo uns vernünftige Einsichten
möglich sind, ohne auf Informationen von außen angewiesen zu sein, doch von
anderen -- von Gesprächspartnern, Lehrern und Traditionen -- abhängig sind
(Kapitel \ref{section:AufklaerungundRationalismus}).
\end{enumerate}

Ist es uns aber trotz dieser Aspekte unserer Endlichkeit möglich, Metaphysik zu
betreiben? Diese wird mitunter gerade mit einem (erst zu bestimmenden)
\singlequote{Unendlichen} assoziiert. In Kapitel \ref{Kapitel5Metaphysik} werde
ich aufzeigen, welche Rolle der Metaphysik in dem in den vorherigen Kapiteln
entwickelten Aufklärungsprojekt zukommt und welche Form sie darin annimmt. Nach
einer verbreiteten Auffassung -- die durch Kantianer wie
\authorcite{Schnaedelbach:WirKantianer2005} verbreitet wird, an der aber auch
\name[Immanuel]{Kant} selbst seinen Anteil hat\footnote{Man sehe etwa die
Anmerkungen in \cite[][A xi]{Kant:KritikderreinenVernunft2003},
\cite[][IV: 9.26--39]{Kant:GesammelteWerke1900ff.}.} -- endet die Zeit der
Metaphysik mit der Aufklärung, weil diese durch die Einsicht in unsere Endlichkeit geprägt sei.
Dieses Bild gilt es partiell zu korrigieren. \name[Immanuel]{Kant}s Metaphysikbegriff wird
sich dabei in der Tat als durch das Zusammenspiel von Aufklärungsverständnis und
Einsicht in unsere Endlichkeit geprägt erweisen.

Kapitel \ref{chapter:endlichkeitmenschlichendenkens} schließlich thematisiert
die Endlichkeit des menschlichen Denkens als eigenständiges Thema, um solchen
Fragen nachzugehen wie: Welche Formen der Endlichkeit haben sich ergeben? Wie
verhalten sich diese Formen zueinander? Gibt es eine grundlegende Form der
Endlichkeit, von der die anderen Formen abhängen? Inwiefern ist
\name[Immanuel]{Kant} nun \singlequote{der} Philosoph der Endlichkeit?

