

Unsere Endlichkeit steht der Forderung nach Selbstdenken entgegen,
insofern wir das Besondere gar nicht durch eigenes Nachdenken bestimmen können, sondern auf gegebene
Erkenntnisse angewiesen bleiben, und insofern wir in der vernünftigen
Ausrichtung unserer Handlungen stets mit der Abhängigkeit unseres Willens von
Neigungen als Hindernissen konfrontiert werden. Unsere Endlichkeit macht
Aufklärung aber nicht unmöglich, sondern erst notwendig. Ein nicht-endliches Wesen bedürfte nicht der Aufklärung, der nur
dasjenige Wesen unterworfen werden kann, dessen Vernunfttätigkeit von seiner
Sinnlichkeit abhängig ist, da diese seinem diskursiven Denken überhaupt erst
Inhalte bereitstellt. Der Verstand eines endlichen Wesens ist diskursiv,
insofern er nur mittelbaren Bezug zu den Gegenständen seines Denkens aufweist:
Denken ist Erkennen durch Begriffe; und Begriffe sind Vorstellungen, die sich
nur vermittelst allgemeiner Merkmale auf ihre Gegenstände beziehen.
Vorstellungen, die einer solchen Vermittlung nicht bedürfen, sondern sich
unmittelbar auf Gegenstände beziehen, nennt \name[Immanuel]{Kant}
\enquote{Anschauungen}. Ohne Vorstellungen, die sich unvermittelt auf
Gegenstände beziehen, könnte es aber gar keinen Gegenstandsbezug geben, unser
Denken wäre ein \enquote{frictionless spinning in the
void}\footnote{\cite[][11]{McDowell:MindandWorld1994}.}, wie
\authorfullcite{McDowell:MindandWorld1994} schreibt.


Ein unendliches Wesen müsste in seinem Denken nicht zwischen
rezeptivem und spontanem Erkenntnisvermögen unterscheiden: Sein Erkennen wäre
gänzlich spontan und seine Vernunft fände kein Hindernis an den eigenen
sinnlichen Antrieben. Damit wäre ein solches Wesen in einer Weise
selbstbestimmt, aktiv und \singlequote{spontan}, die sich von unserer Art des
Denkens und Erkennens nicht graduell, sondern qualitativ unterscheidet. Wir
endliche Wesen hingegen, deren oberes Erkenntnisvermögen diskursiv ist, sind stets der Gefahr ausgesetzt,
unsere Aktivität und Spontaneität hintanzustellen und uns der Passivität
hinzugeben. Dem entgegenzuwirken heißt Aufklärung.

Kant bestimmt Aufklärung als den Ausgang des Menschen aus selbst verschuldeter
Unmündigkeit und damit als Haltung, immer und überall selbst zu denken und die
eigene Vernunft als obersten Maßstab der Wahrheit seiner Überzeugungen gelten zu
lassen. Dabei nimmt er jedoch die Warnung
\authorcite{Wolff:Discursuspraeliminarisdephilosophiaingenere1996}s vor einem
Abgleiten in Flachheit und Beliebigkeit durch eine falsch verstandene Aufklärung
auf. Es kommen dabei zwei unterschiedliche Aufklärungsverständnisse in den
Blick: Selbstdenken kann zum einen als Resultat eines \emph{Entschlusses}, zum
anderen als Ausdruck einer \emph{Fähigkeit} verstanden werden.
\authorcite{Wolff:Discursuspraeliminarisdephilosophiaingenere1996}s Einsicht
ist: Nur durch die Ausbildung eigener intellektueller Kompetenzen lässt sich
Selbstdenken erreichen. \name[Immanuel]{Kant} vertritt nun allem Anschein nach
die Gegenposition, auch indem er sich von
\authorcite{Wolff:Discursuspraeliminarisdephilosophiaingenere1996}s
Methodenideal -- \Revision{der mathematischen Methode} -- explizit abgrenzt. Tatsächlich
verwirft \name[Immanuel]{Kant} den Gedanken, intellektuelle Kompetenz durch
bestimmte Standards oder eine bestimmte Methode beschreiben zu können. Er
verfällt jedoch nicht der naiven Vorstellung, Selbstdenken sei schlicht das
Resultat eines Entschlusses, der keiner weiteren Kompetenzen bedürfte.
Stattdessen bindet er sie zurück an die Ausbildung und Ausübung intellektueller
Kompetenz in Gemeinschaft und im Austausch mit anderen. Der aufgeklärt Denkende
ist daher nicht der isoliert von anderen jedes Urteil nur vor sich selbst
rechtfertigende Individualist, sondern derjenige, der sich nach dem Modell
einer republikanischen Vernunft selbst als gleichberechtigtes Mitglied einer
Gemeinschaft von Denkenden begreift. Die eigene Vernunft ist zu verstehen als
der je eigene \emph{Gebrauch} der \emph{allgemeinen} Menschenvernunft.

Aufklärung verlangt von uns nicht, alles zu wissen, sondern das zu
wissen, was für unser alltägliches Handeln wesentlich ist. \singlequote{Wesentliche}
Erkenntnisse sind wiederum solche, die sich auf die \singlequote{Bestimmung des
Menschen} -- ein zentrales Schlagwort der zweiten Hälfte der Aufklärung --
beziehen und den \emph{Weltbegriff der Philosophie} ausmachen. Ein solches
Wissen nennt Kant \enquote{pragmatisch} und versteht darunter
handlungsrelevantes Wissen, also einerseits Wissen darum, wie wir handeln
sollen, -- welches sich in den verschiedenen Arten von Imperativen (Regeln der
Geschicklichkeit, Ratschläge der Klugheit und Gebote der Sittlichkeit)
artikuliert -- sowie andererseits Wissen über die Situation, in der wir Menschen
uns als Menschen in der Welt befinden (die \emph{conditio humana}). Die
Herausforderung besteht dabei nicht darin, solches Wissen erst selbst zu
erwerben; gewöhnlich verfügen wir längst über dieses Wissen. Worum wir uns
hingegen erst bemühen müssen ist, das vorhandene Wissen in unserem Handeln
wirksam werden zu lassen.

Dennoch bleibt die Frage bestehen: Kann Mündigkeit vereinbar sein mit dem Erwerb
von Wissen durch die Auskunft Anderer? Selbstdenken wird verstanden als Selbstbestimmung im Denken, als
Versuch, unabhängig von Traditionen und Autoritäten unser Denken selbst zu
verantworten. Der Unaufgeklärte verzichtet darauf, selbst zu denken -- so
\name[Immanuel]{Kant}s \emph{prima facie} problematisches Postulat --, wenn er
Bücher kaufen kann, die das \enquote{verdrießliche Geschäft} des Denkens für
ihn übernehmen.\footnote{\cite[][A
482]{Kant:BeantwortungderFrage:WasistAufklaerung?1977}, \cite[][VIII:
35.13--16]{Kant:GesammelteWerke1900ff.}.} Wenn wir denken, dann tun wir jedoch
oft gut daran, uns nicht auf unsere eigenen oft schlecht ausgebildeten
Fähigkeiten und unsere eigenen mitunter schlecht fundieren Einsichten zu
verlassen, sondern auf das Wissen von solchen, die es besser wissen als wir. Wir
fragen Menschen, die sich aufgrund ihrer Ausbildung besser mit der Materie
auskennen (Experten), oder solche, die einen Sachverhalt besser beurteilen
können als wir, weil sie sich zu einer bestimmten Zeit an einem bestimmten
epistemisch günstigen Ort aufgehalten haben (Zeugen). Bereits im 18.
Jahrhundert waren sich Philosophen wie Kant der Tatsache bewusst, dass der
Erwerb von Wissen keine Leistung von Einzelkämpfern, sondern Resultat
gemeinschaftlicher Erkenntnisbemühungen ist. So besäßen wir gar
nicht in nennenswertem Umfang Wissen, wenn wir nicht über sogenanntes testimoniales
Wissen verfügten, also über solche Überzeugungen, die wir lediglich auf die
Autorität anderer Personen hin für wahr halten, insofern diese uns entsprechende
Informationen zur Verfügung stellen und uns an ihrem Wissen teilhaben lassen.
Die Notwendigkeit solchen Wissens scheint aber die Forderung der Aufklärung, in
allen Belangen selbst zu denken, \emph{ad absurdum} zu führen, auch dann, wenn
wir den Bereich dessen, was wir selbst kompetent beurteilen müssen, auf
pragmatisches Wissen einschränken.

Seit \authorcite{Descartes:OeuvresdeDescartes1983}, der einen der
folgenreichsten Angriffe auf die \singlequote{Büchergelehrsamkeit} führte,
bemühen sich Philosophen um eine sinnvolle Darlegung der Möglichkeit
testimonialen Wissens. \authorcite{Descartes:OeuvresdeDescartes1983}' Argumente
weisen in zwei unterschiedliche Richtungen: Zum einen argumentiert er gegen die
Verlässlichkeit der Auskünfte, die wir von anderen erhalten. Zum anderen
behauptet er, durch das Lesen von Büchern erwürben wir doch keine Wissenschaft,
sondern lediglich \singlequote{historische Kenntnisse}. Von weitaus größerer
Relevanz für das Projekt der Aufklärung ist die zweite Argumentationsrichtung,
weil sie die Entwicklung eigener intellektueller Kompetenzen betrifft. Wir sind
nicht unaufgeklärt, wenn wir auf Grundlage einer Mitteilung etwas für wahr
halten, was \emph{de facto} falsch ist, sondern dann, wenn wir uns passiv
verhalten und bloß rezipieren, was unser aktives Mitdenken und die Entwicklung
eigener intellektueller Kompetenzen verlangt.


Die Forderung, selbst die Verantwortung für die eigenen Überzeugungen zu
übernehmen, betrifft zunächst dasjenige Wissen, welches wir nicht aus der
Erfahrung, sondern aus dem Gebrauch unserer Vernunft haben: Mathematik und
Philosophie. Da die Mathematik wegen der Möglichkeit der Darstellung ihrer
Inhalte in der reinen Anschauung leichter adäquat gelehrt und gelernt werden
kann, bleibt als schwierige Herausforderung die adäquate Aneignung
philosophischen Wissens, also rationaler Erkenntnis aus Begriffen. Ein solches
rationales Wissen aus Begriffen nennt \name[Immanuel]{Kant} auch
\emph{Metaphysik}.\footnote{Die Bestimmungen des Selbstdenkens als der Forderung, selbständig metaphysische
Erkenntnisse zu generieren, harmonieren mit Kants Explikation des Begriffs des
Aberglaubens wie auch des Vorurteils. Vorurteile sind vorläufige, unbegründete
Urteile, insofern sie als Prinzipien unseres Schließens wirksam sind. Um
Vorurteile zu vermeiden müssten wir zunächst erkennen, welche Prinzipien unseren
Urteilen zugrunde liegen, und uns aktiv mit den begrifflichen Grundlagen unseres
Denkens auseinander
setzen \mkbibparens{\cite[vgl.][A~116\,f.,]{Kant:ImmanuelKantsLogik1977}
\cite[][IX: 75.24--76.12]{Kant:GesammelteWerke1900ff.}}. Vorurteilsbehaftet
bleibt derjenige, der die begrifflichen Grundlagen seines Denkens so hinnimmt, wie sie ihm gegeben wurden.
Aberglaube ist nach Kant das Vorurteil, die Welt so zu repräsentieren, als wäre
sie nicht notwendig den Gesetzen unseres Verstandes unterworfen. Es handelt sich
also nicht einfach um empirisch falsche Urteile, sondern um Vorurteile, da sie
auf einem falschen metaphysischen Grundsatz beruhen. Dem Aberglauben steht daher
die metaphysische Einsicht entgegen, dass alles, was geschieht, nach objektiven
Gesetzen geschieht. Deswegen ist er eine Form der Heteronomie im Denken: Wer
abergläubisch ist, der bewertet die gegebene Information höher als sein eigenes
Denken. Dabei spielt es keine Rolle, ob die Information von den eigenen Sinnen
zu kommen scheint oder von einem vermeintlichen Experten oder Zeugen.} Und damit bezieht sich die Forderung des
\enquote{Sapere aude!} zunächst und vor allem auf metaphysische Erkenntnisse, was aus der Sicht
manches Aufklärungsverständnisses überraschen dürfte. Nun also soll Aufklärung
\emph{nur} durch Metaphysik möglich sein, freilich durch eine Metaphysik, die
sich als \singlequote{kritische} Metaphysik von der Metaphysik seiner Vorgänger
unterscheidet. Doch das Grundproblem dieser Arbeit tritt nun neu hervor: Ist
Metaphysik aus der Perspektive endlicher Wesen überhaupt möglich?

Nach \name[Immanuel]{Kant} zeichnen sich obere Erkenntnisvermögen dadurch aus,
dass sie metaphysische Erkenntnisse generieren und dadurch Prinzipien enthalten,
die all unser Erkennen bestimmen. Zwar sind wir endlich und können daher nur
durch Rückgriff auf Sinnlichkeit gehaltvolle Gedanken haben. Aber dieses Denken,
welches nur durch seinen Bezug auf Erfahrung gehaltvoll ist, ist doch zugleich
autonom, d.\,h. es wird durch Grundsätze bestimmt, die selbst nicht der
Erfahrung entlehnt, auf diese aber notwendigerweise anwendbar sind.
Eine \emph{Ethics of Belief} im Sinne \name[Immanuel]{Kant}s und seines
Aufklärungsprogramms hat daher zur Kernforderung, nicht das passiv zu
übernehmen, was unser eigenes Denken regieren sollte. Der Grundsatz, den
obersten Probierstein der Wahrheit in seiner \emph{eigenen} Vernunft zu suchen,
den \name[Immanuel]{Kant} als Definition des Begriffs des Selbstdenkens anführt,
verweist in diesem Sinne auf die Autonomie der Erkenntnisvermögen. Wer
Erkenntnisse, die er \emph{a priori} (oder \singlequote{\emph{ex principiis}})
erwerben könnte, passiv aufnimmt, der lässt sich überreden statt überzeugen.

\Revision[Heidemann]{Damit ist die Frage nach einer \emph{ethics of belief} bei
\name[Immanuel]{Kant} im Grunde auch abschließend beantwortet. Selbstverständlich ließen sich viele
weitere und weitaus konretere Regeln benennen, die der mündige und aufgeklärte
Mensch befolgen sollte. Aber diese Regeln stellen keine Konkretisierung des
Begriffs Mündigkeit dar. Denn letztere besteht niemals in der bloßen Befolgung
solcher Regeln, die vielmehr selbst zum Thema selbständigen Philosophierens zu
machen sind. Mündig ist also nicht, wer epistemischen Regeln der Aufklärung
folgt, sondern wer die eigenen epistemischen Grundsätze reflektiert und
selbständig artikulieren, bewerten, verteidigen und gegebenenfalls revidieren
kann.}

\begin{comment}
Gerade in Fragen der Religion kommt zum Vorschein, wie pikant die Ergebnisse der
Vernunftkritik aus aufklärerischer Perspektive sind. Die Unmöglichkeit objektiv
gültiger Urteile bezüglich der Existenz Gottes befördert intellektuelle Freiheit
in Religionssachen entgegen dem ersten Anschein gerade nicht, sondern stellt
eine Gefährdung derselben dar. Der aufgeklärte Mensch benötigt einen allgemeinen
Maßstab der Vernunft, sonst bleibt ihm nur die Urteilsenthaltung. Um diesen Maßstab
zu entwickeln bedarf es der Theorie eines praktischen Vernunftglaubens. Diese
Konzeption eines Vernunftglaubens ist gerade Ausdruck der Spannung zwischen
Aufklärung und ihrer Forderung nach Selbstdenken und Mündigkeit auf der einen
und unserer Endlichkeit und der ihr geschuldeten Begrenzung unseres
Erkenntnisbereichs auf der anderen Seite.
\end{comment}