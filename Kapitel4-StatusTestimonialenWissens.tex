Die Forderung nach einem kritischen Umgang mit der je
eigenen geistigen Tradition und nach Unabhängigkeit von geistigen Autoritäten
bei gleichzeitigem Bewusstsein der eigenen Abhängigkeit von Anderen prägt weite
Teile der Aufklärung in Deutschland.
Zeugnis dessen ist der in Vorurteilstheorien der Aufklärung verbreitete
Dualismus von Vorurteilen der Autorität und Vorurteilen aus
\singlequote{Übereilung} oder
Selbstüberschätzung.\footnote{\cite[Vgl.][245--246]{Albrecht:ChristianThomasius1999}.}
Diese Bipolarität wird auch in der {\jaeschelogik}
gelehrt, wenn \name[Immanuel]{Kant} die \emph{Vorurteile} in solche
\emph{des Ansehens} -- einer Person, der Menge oder des Zeitalters (des
Altertums oder der \singlequote{Neuigkeit}) -- und solche \emph{aus Eigenliebe}
-- den \enquote{logische[n] Egoismus} --
unterteilt.\footnote{\cite[Vgl.][A~119--125]{Kant:ImmanuelKantsLogik1977},
\cite[][IX: 77.26--80.32]{Kant:GesammelteWerke1900ff.}. Die Grundlage findet
\name[Gottlob Benjamin]{Jäsche} in den Reflexionen 2563--2582 , \cite[][XVI:
417.12--427.4]{Kant:GesammelteWerke1900ff.}.} Und dieselbe Polarität besteht in
der \titel{Kritik der Urteilskraft} zwischen der Maxime des Selbstdenkens und
der Maxime der \enquote{erweiterten Denkungsart}, sein Urteil stets an dem
Urteil anderer zu messen.\footnote{Siehe hierzu Kapitel
\ref{section:sensuscommunis}.}


Dieses Spannungsverhältnis wirft unmittelbar die Frage auf, wie mit dem
Wissen und den Informationen anderer umzugehen ist, wenn man weder seine eigene
Urteilsfähigkeit überschätzen, noch sich leichtfertig der Autorität anderer
ergeben möchte. Es verwundert daher nicht, in der Literatur der Neuzeit seit
\authorcite{Descartes:OeuvresdeDescartes1983} mannigfache Thematisierungen des Problems testimonialen
Wissens zu finden. Oliver \authorcite{Scholz:DasZeugnisanderer2001} spricht zu Recht entgegen dem Mainstream nicht von einer neuen
Fragerichtung der Erkenntnistheorie\footnote{Dieser Mainstream findet sich noch
in:
\cite[][529--531]{Grundmann:AnalytischeEinfuehrungindieErkenntnistheorie2008},
sowie \cite[][46]{Wilholt:SozialeErkenntnistheorie2007}.}, sondern \enquote{von
einer Renaissance oder Wiedergeburt der erkenntnistheoretischen Diskussion der
Testimonialerkenntnis \punkt , da es im 17. und 18. Jahrhundert bereits eine
sehr extensive und intensive Diskussion gegeben
hat.}\footnote{\cite[][354]{Scholz:DasZeugnisanderer2001}.} Die Tatsache, dass
Überlegungen zu testimonialem Wissen aus der deutschen Aufklärungsphilosophie
innerhalb der aktuellen Debatten (die sich in historischen Rückverweisen fast
ausschließlich auf John \name[John]{Locke}, David \name[David]{Hume} und Thomas
\name[Thomas]{Reid} beziehen) unberücksichtigt bleiben, täuscht leicht darüber hinweg, wie ausführlich gerade
dort darüber diskutiert wurde, wann wer welche Erkenntnisse auf das Zeugnis
eines anderen hin übernehmen darf.


Tatsächlich sind gerade die Themen der Sozialen Erkenntnistheorie -- wie
\authorfullcite{Goldman:Experts:WhichOnesShouldYouTrust?2001} schreibt -- \enquote{practically quite
pressing.}\footnote{\cite[][85]{Goldman:Experts:WhichOnesShouldYouTrust?2001}. 
Besonders alltagsnah sei die Herausforderung, zu bestimmen \enquote{how
lay-persons should evaluate the testimony of experts and decide which of two or
more rival experts is most credible. It is of practical importance because in a
complex, highly specialized world people are constantly confronted with
situations in which, as comparative novices (or even ignoramuses), they must
turn to putative experts for intellectual guidance or assistance}
\parencite[][85]{Goldman:Experts:WhichOnesShouldYouTrust?2001}.} Denn bezüglich
der Fragen danach, welchen Informationen wir vertrauen sollen und wer zu Recht
als Experte gilt, stellt die moderne Informationsgesellschaft uns, die vom
Anspruch nach Mündigkeit geprägten Nachfahren der Aufklärung, tatsächlich jeden
Tag vor handfeste erkenntnistheoretische Probleme. Und so sehr sich die
Situation auch durch moderne Informationstechnologien wie das Internet
verschärft haben mag, ist sie doch strukturell mit der Situation im
18.~Jahrhundert vergleichbar. Solche \enquote{pragmatischen} Fragestellungen
passen daher wie wenige andere in die neuzeitlich-aufklärerische Tradition
\singlequote{praktischer Logiken}.\footnote{Siehe hierzu oben
Anm.~\ref{Fussnote:PraktischeLogiken}, S.~\pageref{Fussnote:PraktischeLogiken}.}
Bevor auf deren Inhalt eingegangen wird, ist noch einiges zum historischen
Hintergrund und zur allgemeinen Philosophiegeschichtsschreibung anzumerken. Das
vorherrschende Geschichtsbild bezüglich testimonialen Wissens fokussiert auf
David \name[David]{Hume} und Thomas \name[Thomas]{Reid} sowie, als
individualistischen Ausgangspunkt, der zumindest teilweise überwunden wird,
Ren{\'e} \authorcite{Descartes:OeuvresdeDescartes1983}. Bevor ich also auf Vertreter der deutschsprachigen
Aufklärung und speziell auf \name[Immanuel]{Kant} zu sprechen komme, soll dieser historische Hintergrund zumindest kurz skizziert
werden. Dies wird insbesondere den Problemhorizont erkennbar werden lassen, der
den systematischen Rahmen für die Entwicklung der Konzeption \name[Immanuel]{Kant}s zu
testimonialem Wissen darstellt. Den Anfang machen jedoch einige
systematisch-begriffliche Überlegungen zum Themenfeld, die bei der
Charakterisierung der zu behandelnden Autoren unverzichtbar sind.


\section[Vorbemerkungen zur Sozialen Erkenntnistheorie]{Systematische und
begriffliche Vorbemerkungen zur
Sozialen
Erkenntnistheorie}\label{section:SystematischeundbegrifflicheVorbemerkungen}


In Kapitel \ref{chapter:AufklaerungundWissenschaft} war zu sehen, dass die
Konfrontation der Aufklärungsforderung nach Mündigkeit und Selbstdenken mit der
epistemischen Situation des wissenschaftlichen Laien -- und ein solcher ist
jeder von uns zumindest bezüglich der meisten Disziplinen -- von
\name[Immanuel]{Kant} dadurch entschärft wird, dass er bestimmte Bereiche des Wissens und Denkens
akzentuiert. Moral und Klugheit sind die Kernkompetenzen des mündigen Menschen,
der zu seiner Mündigkeit noch Wissen um die \emph{conditio humana} benötigt,
nicht aber in Wissenschaften bewandert sein muss. In Fragen der Moral
und Klugheit benötigen wir darüber hinaus keine spezielle Methodenausbildung,
sondern müssen unser vorhandenes Wissen zur Anwendung bringen und unsere
Urteilskraft im intellektuellen Austausch mit anderen entwickeln. Selbstdenken ist -- so zeigte Kapitel
\ref{section:KantalsliberalerAufklaerer} -- eine genuin soziale Angelegenheit.
Es verweist nicht auf die autarke Wissensgewinnung des einsamen epistemischen
Heroen, der allein im Hausrock am Kamin das Wissen der Welt rekonstruiert,
sondern auf die aktive Teilnahme an einer als republikanisch verstandenen
Vernunft. Der Selbstdenker versteht sich und handelt als gleichberechtigter
Teilnehmer der \singlequote{\emph{r{e}publique des lettres}}.

Nun bringt jedoch die Tatsache, dass jeder von uns in den meisten Bereichen Laie
ist, mit sich, dass wir in der Regel gar nicht als gleichberechtigte Teilnehmer
eines intellektuellen Austausches auftreten können. Wir erlangen viel Wissen
dadurch, dass wir von anderen erfahren, was der Fall ist. Wir lesen über das
Geschehen in der Welt in einer Zeitung; wir informieren uns über
wissenschaftliche Erkenntnisse in Fachzeitschriften oder
populärwissenschaftlichen Magazinen; wir fragen Freunde und Bekannte, die sich
mit etwas besser auskennen als wir. Und gerade das Wissen, das unserer Klugheit
zugrunde liegt, entstammt zu einem erheblichen Teil nicht eigener Erfahrung,
sondern der Erziehung, die uns zuteil wurde, als wir selbst Kinder waren. All
dies sind Formen des Wissenserwerbs, in denen wir nicht mit gleichberechtigter
Stimme auftreten, in denen wir aber auch nicht sagen, dass sie unsere Mündigkeit
gefährden.

Zu Beginn möchte ich eine naheliegende, aber dennoch unbefriedigende Antwort auf
diese Frage zurückweisen. In Kapitel \ref{section:KantalsliberalerAufklaerer}
sagte ich, dass die Forderung der Aufklärung nach Selbstbestimmung ihr Augenmerk
vor allem auf handlungsorientierende Erkenntnisse in Ethik und Religion sowie
Fragen der Klugheit richtet. Wir sollen -- so ließe sich nun mutmaßen -- bei
Fragen, auf die naturwissenschaftliche Antworten schon vorhanden (oder doch
zumindest möglich) sind, auf Experten vertrauen, nicht aber in Fragen der Moral
und Religion sowie der Ausrichtung des eigenen Lebens auf Glückseligkeit hin.
Hierfür sind zwei Gründe denkbar:
\begin{nummerierung}
\item Der erste denkbare Grund wäre, dass eine wissenschaftliche Behandlung von
Moral und Religion und sicheres Wissen in diesen Bereichen \emph{niemandem}
möglich seien. Einen solchen Gedankengang erwähnte ich oben zu
\name[Christian]{Thomasius} bezüglich der Privatheit religiöser Überzeugung.\footnote{Siehe
S.~\pageref{ThomasiusZuPrivatheitreligioesenBekenntnisses}.} Wenn \name[Immanuel]{Kant}
nun behauptet, das Wissen begrenzt zu haben, um dem Glauben Platz zu verschaffen
und den Einfluss philosophischer \singlequote{Schulen} zu
reduzieren,\footnote{\cite[Vgl.][B
xxx--xxxii]{Kant:KritikderreinenVernunft2003}, \cite[][III:
19.5--37]{Kant:GesammelteWerke1900ff.}.} dann lässt sich dies \emph{prima facie}
ebenso interpretieren. Jedoch scheint mir eine solche Interpretation nicht
sonderlich fruchtbar zu sein. Denn zunächst ist offensichtlich, dass eine solche
Position gerade die Moral betreffend nicht \name[Immanuel]{Kant}s Ansicht ist.
Bei Fragen danach, wie wir handeln sollen, gelte vielmehr, dass \emph{jeder} sicheres Wissen besitzen
kann.\footnote{Siehe hierzu die Überlegungen zum
\enquote{moralisch-epistemischen Optimismus} auf den Seiten
\pageref{Abschnitt:moralischepistemischerOptimismus}--\pageref{Abschnitt:moralischepistemischerOptimismus-Ende}.}
Wenngleich dies bezüglich \emph{religiöser} Überzeugungen nicht ganz so leicht
einzusehen ist, postuliert \name[Immanuel]{Kant} doch zumindest, dass wir in
Fragen des \emph{moralisch} Richtigen \emph{alle} kompetent urteilen können (mit
Ausnahme vielleicht von kleinen Kindern und Menschen mit besonderen geistigen
Anomalien). Doch auch in Fragen der Religion distanziert sich
\name[Immanuel]{Kant} von der Vorstellung, ein aufgeklärter Glaube sei ein solcher, der dem jeweiligen
individuellen Belieben anheim gestellt und keinem intersubjektiv gültigen
Maßstab der Vernünftigkeit unterworfen sei. \name[Immanuel]{Kant} bewertet den
Religionsglauben (im Gegensatz zum Kirchenglauben) auf der Grundlage der
\emph{praktischen} statt der \emph{theoretischen} Vernunft, aber er bewertet ihn
damit tatsächlich an einem Maßstab der \emph{Vernunft}. Am ehesten ließe sich
eine Forderung nach \emph{eigenem} Urteilen mangels fundierten Wissens aus
\index{Kant, Immanuel}kantischer Sicht heraus noch im Bereich der \emph{Klugheit} --
speziell der \emph{Privat}klugheit -- vertreten.\footnote{Siehe hierzu Kapitel
\ref{subsection:aufklaerungundpraxis}.} Wenngleich es allgemein (für jeden
Menschen) gültige Regeln in Bezug auf das Erreichen eigenen Glücks geben mag,
sei dieses doch von so vielen Faktoren und insbesondere von biographischen
Zufälligkeiten abhängig, dass es sich doch immer nur um unsichere und
unverbindliche \enquote{Ratschläge} handle.

Hier spielt möglicherweise auch eine Rolle, dass Privatklugheit sich auf
Eigenschaften des Subjekts bezieht, zu denen dieses einen besonderen
\singlequote{privilegierten Zugang} zu haben scheint. Um zu wissen, was mich
glücklich macht, muss ich bei meinen Handlungen auf meine eigenes Gefühl der
Lust oder Unlust achten; und zu meinen Gefühlen der Lust und Unlust stehe ich
selbst in einer besonderen Relation, die mein Urteil diesbezüglich privilegiert.
Die Frage, ob ich lieber mit Freunden ausgehe, Sport treibe oder den Abend vor
dem Fernseher verbringe, kann ich nicht delegieren, weil außer mir niemand die
Antwort kennen kann. Denn dazu muss man wissen, welche Gefühle diese Tätigkeiten
in mir auslösen; und dazu habe wiederum nur ich Zugang. Es liegt -- sollte diese
Theorie eines privilegierten Zugangs wahr sein\footnote{Es gibt durchaus Gründe,
um mindestens die Allgemeinheit einer solchen Behauptung zu bestreiten. Man
beachte etwa das Phänomen der Selbsttäuschung: Wir können uns durchaus darin
täuschen, was uns selbst Freude bereitet; bspw.
wenn mit unseren tatsächlichen Neigungen ein Bild davon konfligiert, wie wir
gerne wären. Jemand mag sich selbst sagen, er möge eine aktive
Freizeitgestaltung, obwohl er sie stets als stressig erlebt und lieber zuhause
auf dem Sofa sitzt. Andere können eine solche Selbsttäuschung mitunter
durchschauen.} -- eine epistemische Asymmetrie in der Natur der Sache.
Allerdings wäre dieser Ansatz nur auf der Grundlage eines Intuitionismus oder
Emotivismus in der Metaethik auf Fragen der Moral
übertragbar und fällt für alle Bereiche außerhalb der Privatklugheit somit als
Erklärungsansatz innerhalb des von \name[Immanuel]{Kant} gesteckten Rahmens aus.
Da er nicht mehr als die Privatklugheit betreffen kann und bei der Explikation
des Begriffs der Mündigkeit keine weitere Funktion übernehmen wird, genügt hier
dieser allgemeine Verweis, ohne dass ich damit behaupten wollte, die Theorie
des privilegierten Zugangs überzeuge oder entspreche den Ansichten
\name[Immanuel]{Kant}s.

\item Der zweite denkbare Grund resultiert aus der gegenteiligen Annahme, dass
\emph{jeder} sicheres Wissen in diesen Bereichen erwerben könne. Eine solche
Annahme harmoniert besser mit \name[Immanuel]{Kant}s moralisch-epistemischen
Optimismus.\footnote{Siehe oben, S.
\pageref{Fussnote:moralischepistemischerOptimismus}.} Und auch diese Annahme
spricht für unsere Forderung nach Selbstdenken in den entsprechenden Bereichen.
Denn sowohl dann, wenn jeder das nötige Wissen hat, wie auch dann, wenn niemand
darüber verfügt, entfällt jede Grundlage für eine Berufung auf Experten, die nur
sinnvoll scheint, wo epistemische Asymmetrien vorliegen, also dort, wo
verschiedene Menschen unterschiedlich kompetent urteilen können. Das Dilemma von
Mündigkeit und Abhängigkeit erweist sich hier im Grunde als ein solches von
Mündigkeit als Forderung der Aufklärung und epistemischer Asymmetrie als dem
bestimmenden Merkmal der modernen arbeitsteiligen Wissenschaftsorganisation.
Epistemische Asymmetrie liegt dabei nicht \emph{per se} vor, sondern immer in
Bezug auf bestimmte Themengebiete und Kompetenzbereiche. Nur dort, wo eine
solche Asymmetrie vorliegt und wir uns auf der nicht-privilegierten Seite
befinden, scheint der Forderung der Aufklärung nach Selbstdenken nicht genügt zu
werden. Denn es ist nicht verständlich, warum wir uns auf unser eigenes Urteil
in Fällen verlassen sollen, in denen wir unsere epistemische Situation dadurch
signifikant verschlechtern.
\end{nummerierung}

Zur Steigerung der Plausibilität lassen sich Kombinationen und Variationen aus
beiden Gründen anführen. So ließe sich vorschlagen, den ersten Grund in Bezug
auf religiöse Überzeugungen und den zweiten Grund in Bezug auf Moral und Ethik
vorzubringen. Wir sollen dann in Fragen der Religion unseren eigenen
Überzeugungen folgen, weil objektives Wissen nicht möglich und uns infolge
dessen niemand epistemisch überlegen ist, und in moralischen Fragen, weil wir
selbst kompetent beurteilen können, was moralisch richtig ist.\footnote{Wir
können die Trennlinie auch anders verlaufend denken, bspw.~zwischen einem
vernünftigen Kern der Religion, über den jeder kompetent urteilen kann
(Vernunftreligion) und zusätzliche Komponenten, die sich jedem kompetenten
Urteil entziehen (Kirchenglauben). Da ich den Lösungsvorschlag in Gänze
zurückweise, sind solche Fragen hier nicht relevant.} Oder wir reklamieren den
ersten Grund für Fragen der Klugheit und den zweiten Grund für Fragen der Ethik,
Moral und Religion. Es ist diese Konstellation, die den Aussagen \name[Immanuel]{Kant}s am
nächsten zu kommen scheint. Denn während er sich sehr skeptisch bezüglich der
Möglichkeit einer allgemeinen Klugheitslehre äußert, spricht er jedem die
Fähigkeit zu, kompetent über Moral und Religion zu urteilen, wenn denn nur
\enquote{Satzungen und
Formeln}\footnote{\cite[][A~483]{Kant:BeantwortungderFrage:WasistAufklaerung?1977},
\cite[][VIII: 36.8--10]{Kant:GesammelteWerke1900ff.}: \enquote{Satzungen und
Formeln, diese mechanischen Werkzeuge eines vernünftigen Gebrauchs oder vielmehr Mißbrauchs
seiner Naturgaben, sind die Fußschellen einer immerwährenden Unmündigkeit.}},
also beispielsweise die Bindung durch Dogmen eines kirchlichen Lehramtes, überwunden sind.

Ein solcher Vorschlag versucht, das Dilemma von Mündigkeit und epistemischer
Abhängigkeit dadurch aufzulösen, dass Mündigkeit und Abhängigkeit jeweils eigene
epistemische Bereiche zugewiesen werden. Abhängig und auf Experten angewiesen
sind wir bei Angelegenheiten der Wissenschaft, weil dort epistemische
Asymmetrien vorliegen; mündig sein sollen wir bei Fragen der Moral/Ethik und
Religion sowie vielleicht in einigen (aber möglicherweise nicht in allen)
Fragen der (Privat\mbox{-)} Klugheit, wo keine solchen Asymmetrien vorliegen,
sondern die Ausgangsbedingungen gleich oder annähernd gleich sind. Welche Auswirkungen beispielsweise bestimmte Stoffe auf den
menschlichen Organismus haben, können andere oft besser beurteilen als ich
selbst, wenn sie die entsprechende Expertise haben. Ob meine Handlung mir
selbst Freude bereitet, das muss ich hingegen selbst beurteilen; denn wer sollte
hier mehr wissen als ich selbst?


Dieser Vorschlag wirkt verlockend, aber er erreicht -- wie noch zu sehen sein
wird -- sein Ziel nicht. Er liefert uns weder einen Begriff von Mündigkeit und
Selbstdenken, noch hilft er uns in unserem \name[Immanuel]{Kant}verständnis.
Zunächst lässt er sich nicht mit dem Wortlaut der \index{Kant,
Immanuel}kantischen Schriften in Einklang bringen, denn \name[Immanuel]{Kant}
sagt nicht, wir sollten uns bei bestimmten Fragen aus unserer Unmündigkeit
herausarbeiten und den Mut aufbringen, unseren eigenen Verstand zu gebrauchen.
Stattdessen betont er, wir sollen dies \emph{immer} tun. Aufklärung sei die
Maxime, \emph{jederzeit} selbst zu denken,\footnote{\cite[Vgl.][A
329]{Kant:Washeisst:SichimDenkenorientieren?1977},
\cite[][VIII: 146.30--31]{Kant:GesammelteWerke1900ff.}.} und die Maxime einer \emph{niemals}
passiven, also \emph{immer} aktiven Vernunft zeichne das vorurteilsfreie Denken
aus\footnote{Vgl. \cite[][\S~40]{Kant:KritikderUrteilskraft2009}, \cite[][V:
294.20]{Kant:GesammelteWerke1900ff.}.}. Aufklärung heiße dann auch, \enquote{mit
seiner Vernunft nicht passiv, sondern \myemph{jederzeit} sich selbst
gesetzgebend zu sein}\footnote{\cite[][\S~40]{Kant:KritikderUrteilskraft2009},
\cite[][V:
294.30--31]{Kant:GesammelteWerke1900ff.}.}. Unter den Vorurteilen ist wiederum
der Aberglaube das schlimmste, welcher darin bestehe, die Natur als den Gesetzen
unseres Verstandes nicht unterworfen zu
denken.\footnote{\cite[Vgl.][\S~40]{Kant:KritikderUrteilskraft2009}, \cite[][V:
294.22--24]{Kant:GesammelteWerke1900ff.}.} Aberglaube drückt sich zum Beispiel
im Glauben an die Wirkung von Gebeten, an die Existenz von Feen und
Hexen oder an Horoskope aus. Das sind alles aber keine moralischen Fehlurteile und auch
keine Urteile, die sich allein aufgrund der praktischen Vernunft korrigieren
lassen. Stattdessen handelt es sich wenigstens zu einem beachtlichen Teil um
Fehlurteile der theoretischen Vernunft, deren Korrektur wir den modernen
Wissenschaften zuschreiben, die uns vor das Dilemma von Mündigkeit und
epistemischer Asymmetrie stellen.\footnote{Auch \name[Immanuel]{Kant} sieht
hier Fehlurteile der theoretischen Vernunft vorliegen; aber er schreibt ihre
Korrektur nicht der empirischen Forschung, sondern der Metaphysik zu; siehe
dazu unten Kapitel \ref{subsection:MetaphysikundAutonomie}.}

Unsere Begriffe von Aufklärung und Mündigkeit haben sehr viel damit zu tun, wie
jemand mit seiner epistemisch nachteiligen Position gegenüber Experten in
unserer modernen Wissensgesellschaft umgeht. Deshalb befriedigt der Vorschlag,
Selbstdenken und Mündigkeit auf Bereiche zu beschränken, in denen es kein
Expertentum geben kann, auch systematisch nicht. Es fiele uns zu Recht schwer,
jemanden als aufgeklärt zu bezeichnen, der von Geozentrismus, Phlogistontheorie
oder einem Erdalter von etwa 7000 Jahren überzeugt ist, weil er selbsternannten
\singlequote{Experten} auf diesem Gebiet folgt oder solche Theorien selbst für
irgendwie \singlequote{plausibler} hält. Gerade Verfechter religiös
inspirierter Gedankengebäude wie des \emph{Intelligent Design} berufen sich
gerne auf das Selbstdenken und eine vermeintlich größere
\singlequote{Plausibilität} solch einfacher Erklärungen gegenüber den
komplexeren und dem Alltagsdenken entfernteren wissenschaftlichen Theorien. Sie
bringen damit -- gänzlich gegen die Intention der Aufklärung -- den
\singlequote{gemeinen} (hier im Sinne von \emph{vulgaris}, nicht von
\emph{communis}) Menschenverstand mit seinen Vorurteilen gegen die
wissenschaftliche Vernunft in Stellung.


Ob im Gegenzug mündig und aufgeklärt ist, wer zwar den \singlequote{richtigen}
Experten (Menschen mit tatsächlicher wissenschaftlicher Expertise auf dem
betreffenden Gebiet), diesen aber ebenso \emph{alles} glaubt, darf zumindest
bezweifelt werden. Zur Mündigkeit gehört wesentlich auch der \emph{kritische}
Umgang mit unserer Wissenschafts- und Informationsgesellschaft. Es sollte also
möglich sein, beide Aspekte so zu vereinen, dass Abhängigkeit und Mündigkeit bei
denselben Überzeugungen \emph{zugleich} vorliegen können. Sollte dem nicht so
sein, stellte dies die Aufklärung vor ein Problem, aus dem es kein Entrinnen
gäbe. Wir müssten ihre Grundintention dann entweder in der Gewinnung und in der
Verbreitung wahrer wissenschaftlicher Ergebnisse sehen, könnten aber ihre
Forderung nach Mündigkeit und Selbstdenken nicht mehr integrieren. Oder wir
sähen in der Forderung nach Mündigkeit und Selbstdenken eben die Grundintention;
dann aber ginge Aufklärung nicht mehr mit der neuzeitlichen
Wissen\-schafts\-ent\-wick\-lung einher, sonder stünde ihr diametral entgegen.
Als Zwischenfazit ergibt sich Folgendes: \emph{Das Grundproblem jeder
Aufklärungsphilosophie betrifft den \emph{gemeinsamen}, aber dennoch individuell
\emph{mündigen} Umgang mit Wissen und Erkenntnis gerade angesichts epistemischer
Asymmetrien.}



% \subsection{Rekonstruktion testimonialen Wissens: Reduktionismus, Skeptizismus
% und Credulismus}\label{subsection:ReduktionismusSkeptizismusCredulismus}
Einer verbreiteten philosophiehistorischen Auffassung zufolge befassen sich
Autoren der Aufklärung, die sich erkenntnistheoretischen Fragen widmen,
bevorzugt mit dem Problem, wie ein auf sich allein gestelltes Subjekt ohne
Interaktion mit anderen zu Wissen gelangt.\footnote{Siehe etwa
\cite[][passim]{Grundmann:DietraditionelleErkenntnistheorieundihreHerausforderer2001}.}
Stimmte dies, so versäumte es die Aufklärungsphilosophie, ihr eigenes
Grundproblem überhaupt zu fassen zu bekommen, geschweige denn zu lösen. Sie
steht vor der Aufgabe, sich mit dem
\emph{gemeinsamen} Umgang mit Wissen und Erkenntnis auseinanderzusetzen. Fragen
des Umgangs mit Wissen und Erkenntnissen, die wir durch Interaktion mit Anderen
haben, werden heute unter dem Schlagwort \enquote{\emph{Social Epistemology}} oder
\enquote{Soziale Erkenntnistheorie} diskutiert.\footnote{Einen Überblick über
diese Disziplin geben Thomas
\textcite[vgl.][529--541]{Grundmann:AnalytischeEinfuehrungindieErkenntnistheorie2008}
und Torsten \textcite[vgl.][]{Wilholt:SozialeErkenntnistheorie2007}. Klassische
Beiträge finden sich in den Sammelbänden von
\textcite[vgl.][]{Schmitt:SocializingEpistemology1994a} und von
\textcite[vgl.][]{Matilal:KnowingfromWords1994} sowie wiederum bei
\textcite[vgl.][]{Schmitt:SpecialIssue:SocialEpistemology1987}. Klassische
Arbeiten sind außerdem die Monographien von Alvin Ira
\textcite[siehe][]{Goldman:KnowledgeinaSocialWorld1999} und Tony
\textcite[][]{Coady:Testimony1992}.} Es fragt sich somit, ob es in der
Aufklärungsphilosophie bereits Vorläufer der heutigen \emph{Sozialen
Erkenntnistheorie} gibt. Wenn es entsprechende Überlegungen nicht geben sollte,
wäre dies für die Bewertung der Reflexion der Aufklärer über ihr eigenes Projekt
fatal. Innerhalb dieser Teildisziplin der Epistemologie werden in der Regel zwei
Arten des Eingebundenseins unseres Wissens und Erkennens in eine epistemische
Gemeinschaft diskutiert:
\phantomsection\label{Abschnitte:ZweiThemenSozialerErkenntnistheorie}
\begin{nummerierung}
\item Zum einen übernehmen wir oft Erkenntnisse von anderen, die diese bereits
besitzen. Wissen, welches wir von anderen haben, also etwas, was uns erzählt
wurde, was wir in einem Buch gelesen oder im Radio gehört haben, nennen wir
heute \emph{testimoniales Wissen}. Zu \name[Immanuel]{Kant}s Zeiten war hierfür
der Ausdruck \enquote{Glaube} (fides) oder -- eindeutiger --
\enquote{historischer Glaube} üblich, der aber einige terminologische
Schwierigkeiten mit sich bringt, weswegen ich hier der Eindeutigkeit halber auf
den moderneren \emph{terminus technicus} \enquote{testimoniales Wissen}
zurückgreife. Analog zu testimonialem Wissen werde ich dann auch von
testimonialen Überzeugungen, dem testimonialen Wissenserwerb und Ähnlichem
sprechen. Es ist wichtig zu beachten, dass die Mitteilung durch einen
anderen\footnote{Ich verwende die Ausdrücke \enquote{Mitteilung},
\enquote{mitteilen} und \enquote{mitteilbar} in der Regel anders als
\name[Immanuel]{Kant}, der dabei nur manchmal an (genuin) testimoniales Wissen denkt,
insgesamt aber einen weiteren Begriff verwendet. Es gibt auch Formen der
Mitteilung\textsubscript{Kant}, die nicht auf testimoniales Wissen
verweisen (siehe dazu Kapitel
\ref{subsubsection:EndlichesundUnendlichesErkennen}).} und das Vertrauen in
denjenigen hier \emph{konstitutiven} Charakter trägt. Wie Elizabeth
\authorcite{Anscombe:Intention2000} betont, ist das Phänomen, dass wir
\emph{jemandem glauben}, grundlegend unterschieden von Situationen, in denen wir
\emph{etwas glauben}, nachdem oder auch weil uns jemand dieses (oder etwas
anderes) erzählt hat, ohne dass das Vertrauen in die entsprechende Person
grundlegend wäre.\footcite[Vgl.][4]{Anscombe:WhatIsIttoBelieveSomeone2008} Wenn
uns jemand etwas mitteilt, was wir ohnehin wussten oder wofür wir unabhängige
Erkenntnisgründe haben, oder wenn uns jemand auf etwas aufmerksam macht, was wir
sonst nicht bemerkt hätten, liegt kein testimoniales Wissen vor, sondern nur
dann, wenn wir etwas \emph{auf die Autorität eines anderen hin} für wahr
halten.\footnote{Siehe zu diesem Punkt
\cite[][398--400]{Hawley:TestimonyandKnowingHow2010}.} Ich werde von
\enquote{genuin testimonialem Wissen} sprechen, wenn ich diesen Punkt betonen
möchte, setze es aber auch bei dem Begriff \enquote{testimoniales Wissen}
voraus. Testimoniales Wissen ist Wissen, das auf einem Wissens\emph{transfer}
beruht, dem eine epistemische Asymmetrie zugrunde liegt. Nur wenn jemand
anderes etwas weiß, was ich nur durch Berufung auf die Mitteilung als
gerechtfertigt ansehen kann, vertraue ich auf eine Mitteilung und gründe meine
eigene Überzeugung darauf. Diese Asymmetrie ist die notwendige -- wenngleich
noch nicht hinreichende -- \emph{Bedingung} testimonialen Wissens.

Es gibt nun zwei Personengruppen, von denen wir testimoniales Wissen bekommen:
Einerseits \emph{Experten}, also Menschen, die sich auf einem Themengebiet besser auskennen
als ihre Mitmenschen, etwa weil sie aufgrund ihre Ausbildung spezielle
epistemische Fähigkeiten und Kenntnisse besitzen, und \emph{Zeugen}, die zwar keine
besonderen Kenntnisse oder Fähigkeiten besitzen, aber etwas erfahren haben, was
uns noch unbekannt ist. Experten besitzen in der Regel erweiterte Kompetenzen,
Zeugen befanden sich in besonderen epistemischen Gelegenheiten; beides
privilegiert bestimmte Subjekte gegenüber anderen und zieht eine epistemische
Asymmetrie nach sich. Als Oberbegriff über beide Gruppen werde ich hier den
Ausdruck \enquote{\emph{Informant}} verwenden; Informanten sind also in diesem
Zusammenhang Menschen, die sich in epistemisch privilegierten Positionen
befinden, weil sie als Zeugen oder Experten etwas wissen, was wir (als nicht
privilegiert) nur unter Rückgriff auf ihr Wissen zu unserem eigenen Wissen
machen können. Im allgemeinen lasse ich offen, ob wir testimoniales Wissen
erwerben \emph{können} oder über eine bestimmte Person tatsächlich erwerben und
ob der Wissenstransfer legitim ist oder wäre oder die Kriterien für Mündigkeit
verletzt; auch können sowohl Einzelpersonen als auch Personengruppen als
Informanten bezeichnet werden. Und wenngleich der Ausdruck \enquote{testimonial}
sprachlich auf den Zeugen (und nicht auf den Experten) verweist, werde ich ihn
hier in Bezug auf Wissen auf der Grundlage beider Personengruppen verwenden.
Testimoniales Wissen ist dann also Wissen, welches jemand durch Rückgriff auf
einen Informanten erworben hat.

\item Zum anderen findet die Generierung neuen Wissens in aller Regel in der
Form kooperativer Forschung statt. Dies schließt an die grundlegenden
Überlegungen zu Selbstdenken und Vernunft in Kapitel
\ref{section:sensuscommunis} an.
Wissenschaft ist kein Unterfangen genialer Einzelkämpfer, die je individuell ihr
Fach bereichern und sich dann bloß hinterher ihre neuen Erkenntnisse
mitteilen.\footnote{Siehe hierzu
\cite[][49--52]{Wilholt:SozialeErkenntnistheorie2007}.} Erkennen -- zumal
wissenschaftliches -- beruht auf der Teilnahme an gemeinsamen
Erkenntnisprozessen, dem Erlernen von gemeinsam Methoden richtigen Erkennens,
der gemeinsamen Kontrolle der Ergebnisse und nicht zuletzt dem vorgängigen
Erwerb einer gemeinsamen Sprache. Subjekte sind möglicherweise auch dort
aufeinander angewiesen, wo keine epistemische Asymmetrie vorliegt. Ich spreche
von Intersubjektivität, um solche
Phänomene symmetrischer epistemischer Abhängigkeiten zu bezeichnen. Dabei
stellen asymmetrische epistemische Abhängigkeiten für das Projekt
intellektueller Selbständigkeit und Mündigkeit eine größere Gefahr dar
als die Abhängigkeit von anderen, denen gegenüber sich das Subjekt als
ebenbürtig verstehen kann. Wir wir gesehen haben, ist Intersubjektivität nicht
nur mit Selbstdenken und Mündigkeit vereinbar, sondern sogar eine Voraussetzung
kompetenten eigenen Urteilens, wenn es nur in der Form selbständiger Teilnahme,
also ohne Asymmetrien und Hierarchien geschieht.\footnote{Siehe Kapitel
\ref{section:sensuscommunis}.}
\end{nummerierung}

Wir gehen normalerweise davon aus, \emph{dass} es möglich ist, testimoniales
Wissen zu haben. Diese Annahme der Möglichkeit testimonialen
Wissens ist so fest in unserem Denken verankert, dass vorgeschlagen wurde, darin
eine Grundbedingung für jede angemessene philosophische Explikation des
Wissensbegriffs zu
sehen.\footnote{\phantomsection\label{Fussnote:WissensbegriffunddasZeugnisanderer}\cite[Siehe][57--58,
63]{Fricker:TheEpistemologyofTestimony1987}. \enquote{[W]e must
take the principle that \ori{knowledge can be spread through language-use} as a
constraint on our theorising about what knowledge is: it is a condition of
adequacy on an account of knowledge that it have this consequence.}
(\cite[][\pno~57\,f.]{Fricker:TheEpistemologyofTestimony1987}.) Siehe auch
\cite[][198]{McDowell:KnowledgebyHearsay1994}, wo die Möglichkeit aber nur
Bedingung eines adäquaten Begriffs testimonialen Wissens darstellt.
\enquote{[I]f a knowledgeable speaker gives intelligible expression to his
knowledge, it may become available at second hand to those who understand what
he says} (\cite[][417]{McDowell:KnowledgebyHearsay1994}).} Mir scheint eine
solche Vorgabe für den Wissensbegriff korrekt zu sein -- aus Gründen, die ich
gleich vorlegen werde. Zunächst sei bemerkt, dass mit der Einsicht, \emph{dass}
testimoniales Wissen möglich ist, noch nicht gesagt ist, \emph{wie} es möglich
ist. Handelt es sich bei testimonialem Wissen um eine eigenständige
Erkenntnisquelle neben Sinnlichkeit und Verstand? Gibt es ein
genuines Erkenntnisprinzip, das uns sagt, dass Mitteilungen anderer zumindest
prima facie zu trauen sei? Oder haben wir testimoniales Wissen, weil wir
Erfahrungswissen davon haben, dass andere bestimmte Aussagen tätigen und dass
von solchen Behauptungen in bestimmten Situationen auf das Vorliegen
entsprechender Sachverhalte geschlossen werden kann? Handelt es sich bei
testimonialem Wissen also um inferentielle Erkenntnisse, die auf einem logischen
Schluss beruhen?

Der Begriff einer \emph{Erkenntnisquelle} oder Quelle des Wissens (\emph{fountain of
knowledge}) findet sich bei \authorfullcite{Locke:TheWorksofJohnLocke1963}, der im \titel{Essay
Concerning Human Understanding} schreibt, diese Quellen lägen gänzlich in der
Erfahrung -- der äußeren Wahrnehmung (\emph{sensation}) einerseits und der
inneren Selbstwahrnehmung (\emph{reflexion}) andererseits.\footnote{\enquote{Our
observation employed either about external sensible objects, or about the
internal operations of our minds, perceived and reflected on by ourselves, is
that which supplies our understandings with all the material of thinking.
These two are the fountains of knowledge, from whence all the ideas we have,
or can naturally have, do spring} \mkbibparens{\cite[][Buch II, Kap.
I, \S~2]{Locke:AnEssayConcerningHumanUnderstanding1963}, in: \cite[][I:
82\,f.]{Locke:TheWorksofJohnLocke1963}}.} Den Erkenntnisquellen entspringen aber
keine Begründungen für Urteile, sondern das \emph{Material} des Denkens, also
die Inhalte, aus denen wir unsere Urteile bilden. Es sind die Vorstellungen
(\emph{ideas}), die den beiden Erkenntnisquellen entspringen, und unseren Geist,
der zunächst eine \emph{tabula rasa} (\enquote{white paper, void of all
characters, without any ideas}\footnote{\cite[][Buch II, Kap.
I, \S~2]{Locke:AnEssayConcerningHumanUnderstanding1963}, in: \cite[][I:
82\,f.]{Locke:TheWorksofJohnLocke1963}.}) sei, mit Inhalten des Denkens
versorgen, aus denen er dann Urteile bilden und Wissen generieren kann.

\name[Immanuel]{Kant} an äußert sich in der
ersten Auflage der \titel{Kritik der reinen Vernunft} sehr ausführlich zu
Erkenntnisquellen. Dabei benennt er drei konkrete Quellen -- Sinn,
Einbildungskraft und Apperzeption:
\begin{quote}
Es sind drei subjektive Erkenntnisquellen, worauf die Möglichkeit einer
Erfahrung überhaupt, und Erkenntnis der Gegenstände derselben beruht:
\ori{Sinn}, \ori{Einbildungskraft} und \ori{Apperzeption}; jede derselben kann
als empirisch, nämlich in der Anwendung auf gegebene Erscheinungen betrachtet
werden, alle aber sind auch Elemente oder Grundlagen a priori, welche selbst
diesen empirischen Gebrauch möglich machen. Der \ori{Sinn} stellt die
Erscheinungen empirisch in der \ori{Wahrnehmung} vor, die \ori{Einbildungskraft}
in der \ori{Assoziation} (und Reproduktion), die \ori{Apperzeption} in dem
\ori{empirischen Bewußtsein} der Identität dieser reproduktiven Vorstellungen
mit den Erscheinungen, dadurch sie gegeben waren, mithin in der
\ori{Rekognition}.\footnote{\cite[][A 115]{Kant:KritikderreinenVernunft2003},
\cite[][IV: 86.16--27]{Kant:GesammelteWerke1900ff.}.}
\end{quote}
Dass Sinnlichkeit und Verstand (Apperzeption\footnote{\enquote{\ori{Die
Einheit der Apperzeption in Beziehung auf die Synthesis der Einbildungskraft}
ist der \ori{Verstand}, und eben dieselbe Einheit, beziehungsweise auf die
\ori{transzendentale Synthesis} der Einbildungskraft, der \ori{reine Verstand}}
\mkbibparens{\cite[][A 119]{Kant:KritikderreinenVernunft2003},
\cite[][IV: 88.22--25]{Kant:GesammelteWerke1900ff.}}. \enquote{Und so ist die
synthetische Einheit der Apperzeption der höchste Punkt, an dem man allen Verstandesgebrauch, selbst die ganze Logik, und, nach ihr, die Transzendental-Philosophie heften muß, ja dieses Vermögen ist der Verstand selbst} \mkbibparens{\cite[][B 134]{Kant:KritikderreinenVernunft2003}, \cite[][III: 109.35--39]{Kant:GesammelteWerke1900ff.}}.}) als Erkenntnisquellen angesprochen werden, verwundert nicht weiter.
Aber auch die Einbildungskraft (als produktive wie reproduktive) ist nach dieser Auskunft eine Erkenntnisquelle. Diese
Darstellung von drei Erkenntnisquellen findet sich in der ersten Auflage im
Rahmen der Deduktion der reinen Verstandesbegriffe und folgt der dort zu
findenden Vorgehensweise. So folgen der \enquote{Synthesis der Apprehension in
der Anschauung}\footnote{\cite[][A 98]{Kant:KritikderreinenVernunft2003},
\cite[][IV: 77.2]{Kant:GesammelteWerke1900ff.}.} die \enquote{Synthesis der
Reproduktion in der Einbildung}\footnote{\cite[][A 100]{Kant:KritikderreinenVernunft2003},
\cite[][IV: 77.32]{Kant:GesammelteWerke1900ff.}.} und schließlich die
\enquote{Synthesis der Rekognition im Begriffe}\footnote{\cite[][A
103]{Kant:KritikderreinenVernunft2003},
\cite[][IV: 79.15]{Kant:GesammelteWerke1900ff.}.}.
In die zweite Auflage wurde diese Darstellung zwar nicht übernommen, es findet
sich dort nur der Verweis darauf, dass (innerer) Sinn, Einbildungskraft und
Apperzeption  Quellen von Vorstellungen \emph{a priori}
enthalten,\footnote{\cite[Vgl.][B 194]{Kant:KritikderreinenVernunft2003},
\cite[][III: 144.5--11]{Kant:GesammelteWerke1900ff.}.} und die Redeweise von
Sinnlichkeit und Verstand als den \enquote{zwei Grundquellen des
Gemüts}\footnote{\cite[][B 74]{Kant:KritikderreinenVernunft2003},
\cite[][III: 74.9]{Kant:GesammelteWerke1900ff.}.}, die in der Einleitung auch
als \enquote{Stämme der menschlichen
Erkenntnis}\footnote{\cite[][B 29]{Kant:KritikderreinenVernunft2003},
\cite[][III: 46.7]{Kant:GesammelteWerke1900ff.}.} bezeichnet
werden.\footnote{Siehe dazu auch oben, Kap. \ref{subsection:DiskursiverVerstandundsinnlicheAnschauung} dieser Arbeit.} Allerdings scheint
mir die Darstellung, wonach Sinnlichkeit, Einbildungskraft und Apperzeption Erkenntnisquellen sind,
für die hier interessierenden Fragen nicht hilfreich zu sein. Denn
\name[Immanuel]{Kant} bezieht sich hier in der \titel{Analytik der Begriffe}
nicht auf Erkenntnisse als \emph{Urteile}, sondern auf
Erkenntnisse im Sinne von bewussten objektiven Vorstellungen -- also auf
\emph{Begriffe} und \emph{Anschauungen} --, um zu zeigen, das reine
Verstandesbegriffe auch bei Anschauungen involviert sind.\footnote{Siehe zur
Ambiguität von \enquote{Erkenntnis} oben, Anm.
\ref{Anmerkung:ErkenntnisInZweierleiSinn} auf S.
\pageref{Anmerkung:ErkenntnisInZweierleiSinn}.} Sinn, Einbildungskraft und
Apperzeption können nur \emph{zusammen} Vorstellungen hervorbringen, denen
objektive Realität zukommt.


Auch an anderen Stellen spricht er von Erkenntnisquellen -- oder auch
von Quellen der Metaphysik\footnote{\cite[Vgl.][\S~2]{Kant:ProlegomenazueinerjedenkuenftigenMetaphysikdiealsWissenschaftwirdauftretenkoennen1977},
\cite[][IV: 265.6--266.8]{Kant:GesammelteWerke1900ff.}.} oder metaphysischer
Urteile\footnote{\cite[Vgl.][\S~3]{Kant:ProlegomenazueinerjedenkuenftigenMetaphysikdiealsWissenschaftwirdauftretenkoennen1977},
\cite[][IV: 270.9]{Kant:GesammelteWerke1900ff.}.}, der reinen Mathematik und
Naturwissenschaft\footnote{\cite[Vgl.][\S~5]{Kant:ProlegomenazueinerjedenkuenftigenMetaphysikdiealsWissenschaftwirdauftretenkoennen1977},
\cite[][IV: 280.10]{Kant:GesammelteWerke1900ff.}.}, der
Vernunft\footnote{\cite[Vgl.][A
3]{Kant:ProlegomenazueinerjedenkuenftigenMetaphysikdiealsWissenschaftwirdauftretenkoennen1977},
\cite[][IV: 255.8]{Kant:GesammelteWerke1900ff.}.}, von
Begriffen\footnote{\cite[Vgl.][A
62]{Kant:ProlegomenazueinerjedenkuenftigenMetaphysikdiealsWissenschaftwirdauftretenkoennen1977},
\cite[][IV: 288.22]{Kant:GesammelteWerke1900ff.}.} (des Raumes und der
Zeit\footnote{\cite[Vgl.][B 119\,f.,]{Kant:KritikderreinenVernunft2003}
\cite[][III: 101.13--15]{Kant:GesammelteWerke1900ff.}.}), der Urteile darüber,
was gerecht ist\footnote{\cite[Vgl.][BA 32]{Kant:DieMetaphysikderSitten1977Rechtslehre},
\cite[][VI: 229.18--230.6]{Kant:GesammelteWerke1900ff.}.} --, aber nirgends gibt
er so explizit die verschiedenen Quellen an. Einer ausführlicheren Diskussion der
Quellen einer Erkenntnis ist der erste Paragraph der \titel{Prolegomena}
gewidmet. \name[Immanuel]{Kant} unterscheidet dort zunächst Objekt, Quelle und
Art einer Erkenntnis und sagt, anhand dieser drei ließen sich Wissenschaften
einteilen. Der Art nach unterteilen sich Erkenntnisse in analytische und
synthetische,\footnote{\cite[Vgl.][\S~2]{Kant:ProlegomenazueinerjedenkuenftigenMetaphysikdiealsWissenschaftwirdauftretenkoennen1977},
\cite[][IV: ]{Kant:GesammelteWerke1900ff.}.} der Quelle nach -- dies sagt er in
den \titel{Prolegomena} jedoch nicht explizit -- offenbar in Erkenntnisse
\emph{a priori} und Erkenntnisse \emph{a posteriori}. In der \titel{Kritik der reinen
Vernunft} schreibt er: \enquote{Man nennt solche \ori{Erkenntnisse a priori},
und unterscheidet sie von den \ori{empirischen}, die ihre Quellen a
posteriori, nämlich in der Erfahrung haben.}\footnote{\cite[][B
2]{Kant:KritikderreinenVernunft2003}, \cite[][III:
28.4--6]{Kant:GesammelteWerke1900ff.}. Siehe auch
\cite[][B 35]{Kant:KritikderreinenVernunft2003},
\cite[][III: 50.35, 51.22]{Kant:GesammelteWerke1900ff.}.} Als empirische
Quellen spricht der die äußere und innere Erfahrung an, und offenbar gibt es auch Quellen in der reinen Vernunft und dem reinen Verstande.\footnote{\cite[Vgl.][\S~1]{Kant:KritikderreinenVernunft2003},
\cite[][IV: 265.6--266.8]{Kant:GesammelteWerke1900ff.}.} \name[Immanuel]{Kant}
spricht zwar weder die Erfahrung noch die Vernunft als Erkenntnisquellen an,
verortet diese Quellen aber \emph{in} der Erfahrung und \emph{in} der Vernunft,
so dass es von geringerer Bedeutung zu sein scheint, die Quellen selbst zu
individualisieren, als sie in diesem Sinne zuzuordnen. \name[Immanuel]{Kant}
nutzt das Wort \enquote{Quelle} häufig, im Zusammenhang von
\enquote{Erkenntnisquelle} oder \enquote{Quelle von Erkenntnissen} ist jedoch
zwischen 1781 und 1787 eine Verschiebung festzustellen. Während 1781 Sinn,
Einbildungskraft und Apperzeption als Erkenntnisquellen bezeichnet werden, weil
sie in Zusammenwirkung Vorstellungen mit objektiver Realität ermöglichen,
gebraucht er später diesen Begriff eher im Sinne des Ursprungs von
Rechtfertigungen von Urteilen, die \emph{a priori} oder \emph{a posteriori} sein
können. Dennoch bleibt die Redeweise von Quellen von Begriffen bestehen, wird
aber ebenso auf die Unterscheidung \emph{a priori}/\emph{a posteriori}
bezogen.\footnote{Siehe z.\,B. \cite[][B 57]{Kant:KritikderreinenVernunft2003},
\cite[][III: 63.37--64.2]{Kant:GesammelteWerke1900ff.}.}

Dass \name[Immanuel]{Kant} also einerseits die (äußere und innere) Erfahrung
und andererseits die Vernunft, nicht aber die Auskunft anderer als
Erkenntnisquellen anspricht, deutet \emph{prima facie} auf eine
individualistische Position hin. Doch es ist offensichtlich, dass wir auch aus
den Mitteilungen anderer Wissen erwerben. Wie lässt sich diese Möglichkeit
testimonialen Wissens rekonstruieren, wenn wir Erfahrung und Vernunft als
Erkenntnisquellen zulassen?

Hilfreich zur besseren systematischen Übersicht scheinen mir die begrifflichen
Grundlagen zu sein, die \authorfullcite{Fricker:AgainstGullibility1994}
entwickelt. Sie wirft die Frage auf, ob es sich bei der Mitteilung von
Erkenntnissen um einen \enquote{epistemic link} handelt, der den Empfänger mit allen nötigen Prämissen versorgt, um in
einer Überzeugung gerechtfertigt zu sein (\enquote{primary epistemic link}), oder
ob dem Empfänger dazu zusätzliche Prämissen bekannt sein müssen
(\enquote{secondary epistemic link}). Sehen, dass
$p$ der Fall ist, ist ein Beispiel für einen \emph{primary epistemic link} zu
der Überzeugung $p$. Wenn ich hingegen sehe, dass die Straße nass ist, dann ist
dies nur ein \emph{secondary epistemic link} zu der Überzeugung, dass es
geregnet hat, denn zur Begründung benötige ich die weitere Prämisse, dass Regen
die übliche Art und Weise ist, wie Straßen nass werden.

Die Frage nach dem Status testimonialen Wissens lässt sich nun folgendermaßen
formulieren: Wenn mir jemand mitteilt, dass $p$, habe ich dann bereits alle
nötigen Prämissen, um gerechtfertigt glauben zu können, dass $p$ (\emph{primary
epistemic link}), oder benötige ich weitere Prämissen wie die, dass die
angeführte Autorität vertrauenswürdig ist und immer die Wahrheit sagt (\emph{secondary epistemic
link})? \name[Elizabeth]{Fricker} behauptet, dass Mitteilungen nur einen \emph{secondary
epistemic link} zu dem jeweiligen testimonialen Wissen
darstellen.\footnote{\cite[Vgl.][]{Fricker:TheEpistemologyofTestimony1987}.} Wir
benötigen nach \name[Elizabeth]{Fricker}s Vorstellung also weitere Prämissen, um durch
Mitteilungen Wissen zu erlangen.
Erfährt man beispielsweise von einem Passanten, in welcher Richtung sich die
Kathedrale befindet, so reicht dies nach dieser Konzeption für sich genommen
nicht als Begründung um zu wissen, wo sich die Kathedrale befindet. Nur wenn man weiter (begründet?) überzeugt ist, dass
unser Informant selbst \emph{weiß}, wo die Kathedrale ist (weil er vielleicht
Einheimischer ist), und uns nicht belügt (weil es vielleicht eine
anthropologische Tatsache ist, dass Menschen meist ehrlich antworten), können
wir darauf \emph{schließen}, wo die Kathedrale ist.\footnote{Das Beispiel
stammt von \textcite[vgl.][\pno~197\,f.]{McDowell:KnowledgebyHearsay1994}.}



Es ergeben sich drei mögliche Positionen, je nachdem, wie ein Philosoph darüber
denkt, ob es sich bei testimonialen Erkenntnissen um einen \emph{primary
epistemic link} oder einen \emph{secondary epistemic link} handelt und ob eine
Rechtfertigung von testimonialem Wissen als eines \emph{secondary epistemic
link} funktionieren kann\footnote{Die intensive Diskussion der Frage
testimonialen Wissens hat freilich in den letzten Jahren weitere Misch- und
Zwischenpositionen hervorgebracht, z.\,B. der Versuch von
\authorfullcite{Lackey:ItTakesTwotoTango2006} eines dritten Weges jenseits
von Credulismus und Reduktionismus
\parencite[vgl.][]{Lackey:ItTakesTwotoTango2006}. Dennoch scheinen mir die hier
angeführten \singlequote{klassischen} Positionen das Feld am besten zu
strukturieren. Versuche wie der von \authorcite{Lackey:ItTakesTwotoTango2006}
sind letztlich nur vor diesem Hintergrund verständlich.}:
\begin{nummerierung}
\item Der \singlequote{Credulismus} oder \singlequote{defaultism} behauptet,
dass eine gültige Regel besagt, Mitteilungen seien \emph{prima facie}
hinreichende Begründungen für
Wissen.\footnote{\phantomsection\label{Fussnote:BegriffdesCredulismus}\authorfullcite{Wilholt:SozialeErkenntnistheorie2007}
verwendet die Ausdrücke \enquote{credulism} und \enquote{defaultism} für
Positionen, die behaupten, dass wir bei Mitteilungen zumindest \emph{prima
facie} berechtigt sind, ihnen zu vertrauen
\parencite[vgl.][48]{Wilholt:SozialeErkenntnistheorie2007}.} Credulisten sehen
den Erwerb testimonialen Wissens also als \emph{primary epistemic link} an.
\item Vertreter eines \singlequote{testimonialen Reduktionismus} hingegen
behaupten, dass es sich bei testimonialem Wissen um einen \emph{secondary epistemic link}
handelt: Wir erwerben nur dann Wissen durch die Mitteilung anderer, wenn wir
für die Überzeugung, dass uns wahre Erkenntnisse mitgeteilt werden, Gründe
haben, die von dieser Mitteilung selbst unabhängig
sind.\footnote{\cite[Vgl.][533]{Grundmann:AnalytischeEinfuehrungindieErkenntnistheorie2008}:
\enquote{Der Reduktionismus besagt, dass ein Hörer (oder Leser) nur dann eine
gerechtfertigte Überzeugung oder Wissen von dem, was der Sprecher (oder Autor)
behauptet, erwirbt, wenn er gerechtfertigt glaubt, dass der Sprecher (oder
Autor) zuverlässig und aufrichtig in dem ist, was er sagt (oder schreibt).} Nach
Torsten \textcite[][48]{Wilholt:SozialeErkenntnistheorie2007} vertritt der
testimoniale Reduktionismus die Auffassung, \enquote{in einer
mitteilungsbasierten Überzeugung könne nur gerechtfertigt sein, wer auch von der
Verlässlichkeit der fraglichen Mitteilung überzeugt sei und für diese
Überzeugung wiederum Rechtfertigungsgründe besitze, die sich letztlich nicht auf
Mitteilung anderer stützen.} Nach Oliver
\textcite[][358]{Scholz:DasZeugnisanderer2001} zeichnet sich der testimoniale
Reduktionismus durch folgende Annahme aus:
\enquote{Um das Zeugnis anderer als Quelle von gerechtfertigter Meinung und
Wissen zu vindizieren, ist eine Zurückführung auf andere epistemische Quellen erforderlich.}}
\item Der \singlequote{testimoniale
Skeptizismus} fordert uns schließlich auf, testimoniales Wissen gänzlich zu
verwerfen; danach irren wir uns, wenn wir glauben, wir seien berechtigt, unsere
testimonialen Überzeugungen als Wissen anzusehen. Er behauptet somit, dass
testimoniales Wissen nicht im Sinne eines \emph{primary epistemic link}
verstanden werden kann und dass eine Rechtfertigung als \emph{secondary
epistemic link} nicht funktioniert.
\end{nummerierung}

\begin{comment}
\subsection{Externalismus und Internalismus in der
\emph{Social Epistemology}}\label{subsubsection:DieSystematischePerspektive}
Die Möglichkeiten, auf die skeptische Herausforderung bezüglich der Möglichkeit
testimonialen Wissens -- den testimonialen Skeptizismus -- zu reagieren,
unterscheiden sich nicht grundlegend von unseren Möglichkeiten, auf einen
allgemeinen Wissensskeptizismus zu reagieren. Es scheint zunächst so zu sein,
dass wir nicht alle Möglichkeiten explizit ausschließen können, wonach unsere
Überzeugungen sich als falsch oder ihre epistemische Grundlage sich als
ungenügend herausstellen könnte. Wir können bei genuin testimonialen
Überzeugungen die Möglichkeit, belogen zu werden, vielleicht nicht restlos
eliminieren. Gerade dies macht den testimonialen Skeptizismus im Gegensatz zu
einem allgemeinen Skeptizismus attraktiv. Die Argumente, mit denen
sich ein universeller Skeptizismus mit Bezug auf Wissen motivieren lässt, das wir je
individuell generieren, muss auf ausgefallene Szenarien wie \authorcite{Descartes:OeuvresdeDescartes1983}'
bösen Dämon oder \authorcite{Putnam:TheMeaningofglqMeaninggrq1975}s Gehirn im
Tank rekurrieren, die uns nicht als \emph{reale} Möglichkeiten, sondern nur als
philosophische Gedankenspiele begegnen. Aber die Möglichkeiten, die dazu führen,
auch bei einfachen und alltäglichen Beispielen testimonialen Wissens in die Irre
geführt zu werden, sind reale Möglichkeiten.
Und so scheint es, dass sich die Möglichkeit testimonialen Wissens mit einem
einigermaßen starken Wissensbegriff nicht
verträgt.\footnote{\authorfullcite{Scholz:AutonomieangesichtsepistemischerAbhaengigkeiten2001}
unterstellt \name[Immanuel]{Kant} entsprechend die Ansicht, dass \enquote{die
illusorischen Ideale epistemischer Autonomie, die einen großen Teil der
neuzeitlichen Erkenntnistheorie geprägt haben, aufgegeben werden} müssten
\parencite[][839]{Scholz:AutonomieangesichtsepistemischerAbhaengigkeiten2001},
gerade weil er ihm attestiert, Mitteilungen als legitime Wissensquelle korrekt
einzuschätzen. Es gehe also darum, den Wissensbegriff, der traditionell viel zu
stark zu sein scheint, weil er eine \emph{Garantie} der Wahrheit unserer
Überzeugung verlangt, der epistemischen Realität anzupassen.
\authorfullcite{Kern:QuellendesWissens2006} spricht hier von
\enquote{Ermäßigung}, \cite[vgl.][112]{Kern:QuellendesWissens2006}:
\enquote{Positionen der Ermäßigung wollen sagen: Das Subjekt muß nicht alle
Umstände ausschließen, sondern nur die relevanten.} Ob die bloße Abschwächung
des Wissensbegriffs, die die Möglichkeit des Irrtums auch bei Wissen konzidiert
und die man gemeinhin \enquote{Fallibilismus} nennt, weiterhilft, mag zunächst dahingestellt sein.}
Aber können wir diese Möglichkeit in vielen oder zumindest manchen Fällen als
unbegründet verwerfen? Und falls dies so sein sollte: Lässt sich konkretisieren,
in \emph{welchen} Fällen wir Irrtumsszenarien als irrelevant ignorieren
dürfen?

Ich werde in diesem Kapitel erstens zu zeigen versuchen, dass die Forderung der
Aufklärung nach Mündigkeit internalistische epistemische Regeln verlangt, also
solche epistemischen Regeln, bei denen für das erkennende Subjekt selbst
ersichtlich ist, ob ihre Anwendungsbedingungen erfüllt sind. Externalistische
epistemische Regeln hingegen enthalten eine Bedingung, bei der für das Subjekt nicht
ersichtlich ist, ob sie tatsächlich erfüllt ist. Zweitens soll darauf
hingewiesen werden, dass aus diesem epistemischen Internalismus kein
epistemischer Individualismus resultiert.
Die Unvorsichtigkeit hinter der Behauptung, Aufklärung führe unmittelbar zu
einer individualistischen Erkenntnistheorie, liegt in dem Übergang von einem
Internalismus zu einem Individualismus. Doch bevor ich dies erläutere, werde ich
mit Hilfe von \authorfullcite{Lewis:ElusiveKnowledge1996}' Liste epistemischer
Regeln versuchen, den Unterschied zwischen internalistischen und
externalistischen epistemischen Regeln zu verdeutlichen.

Wenn wir etwas wissen, dann haben wir nicht nur eine wahre Überzeugung und auch
nicht nur eine Überzeugung, zu der wir eine Begründung haben, sondern eine
Überzeugung, die so gut begründet ist, dass nicht sinnvoll angenommen werden
kann, dass wir uns irren. Andernfalls haben wir bloß eine Vermutung. Auch
Vermutungen können gut begründet sein, aber ihre Begründung schließt die
Möglichkeit des Irrtums nicht aus; darin unterscheiden sie sich von Wissen. Wir
vermuten etwa, dass es bald regnet, weil wir sehen, dass Wolken aufziehen. Da es
aber auch möglich ist, dass die Wolken in eine andere Richtung ziehen oder ohne
Regen über uns hinweg ziehen, vermuten wir nur und wissen nicht.

Epistemische Regeln im Sinne von \authorcite{Lewis:ElusiveKnowledge1996} geben
an, welche möglichen Szenarien zu beachten sind und welche ignoriert werden können, wenn wir
Wissensansprüche bewerten. Er hat -- nicht exklusiv zum Thema testimonialen
Wissens, sondern allgemein zur Untersuchung des Wissensbegriffs -- eine Liste
von insgesamt sechs Regeln erstellt, die angeben, wann wir Möglichkeiten als
irrelevant betrachten dürfen.\footnote{\cite[Vgl. zum
Folgenden][554--560]{Lewis:ElusiveKnowledge1996}.} Ich liste sie hier zur
Übersicht auf, ohne damit behaupten zu wollen, dass alle sechs Regeln
überzeugen. Es handelt sich um Regeln, die systematisch interessieren und
aktuell diskutiert werden und die hier vor allem zur Illustration der
Überlegungen zu Internalismus und Externalismus dienen.
Wie ich zugleich deutlich machen möchte, lassen sie sich alle leicht auf den
Fall testimonialen Wissens anwenden. \authorcite{Lewis:ElusiveKnowledge1996}'
Regeln lauten (in geänderter Reihenfolge):
\begin{nummerierung}
\item \emph{Rule of Actuality}. Wir müssen stets diejenige Möglichkeit
berücksichtigen, die \emph{verwirklicht} ist. Wenn Jasmin glaubt, ein Haus zu
sehen, aber \emph{de facto} vor einer Hausattrappe steht, kann sie nicht wissen,
dass vor ihr ein Haus steht. Und ebenso wenig kann sie wissen, dass vor ihr ein
Haus steht, wenn dort tatsächlich eines steht, sie aber zeitgleich halluziniert;
denn die Halluzinationen sind ein Grund, die Möglichkeit des Irrtums in Betracht
zu ziehen, auch wenn sie selbst um diese Halluzinationen nicht weiß. Wenn Peter
gerade lügt oder sich selbst täuscht oder wenn Peter ein notorischer Lügner
oder im Erkennen bestimmter Dinge sehr unzuverlässig
ist\footnote{\phantomsection\label{Anmerkung:PetersOrientierungsschwierigkeiten}Man
denke an den Fall, dass Jasmin als Touristin in einer fremden Stadt Peter, den sie nicht kennt, aber für einen Einheimischen hält, nach dem Weg zur Kathedrale
fragt. Peter habe ein so schlechtes Orientierungsvermögen, dass niemand, der
ihn kennt, auf seine Ratschläge hören würde. In diesem Fall weise er Jasmin aber
in die richtige Richtung. \emph{Weiß} Jasmin, die Peters Orientierungsprobleme
nicht kennt, nun, wie sie zur Kathedrale gelangt? Ich denke, man wird diese
Frage verneinen müssen.}, dann handelt es sich um eine Irrtumsmöglichkeit, die
nicht ignoriert werden darf. Und Jasmin kann unabhängig davon, ob sie selbst
weiß, dass Peter lügt oder sich täuscht, kein testimoniales Wissen von Peter
erhalten.\footnote{Dies gilt natürlich auch in dem Fall, dass Peter tatsächlich
die Wahrheit sagt, aber im Erkenntnisprozess ein solcher Fehler enthalten ist,
wie wir ihn von den
\authorcite{Gettier:IsJuftifiedTrueBeliefKnowledge?1963}-Beispielen kennen.
Wenn Peter bspw. Jasmin in die Irre führen möchte, indem er ihr die falsche
Richtung weist, aufgrund seiner mangelnden Ortskenntnis Jasmin aber doch in die
richtige Richtung schickt, dann kann Jasmin nicht auf der Grundlage von Peters
Mitteilung \emph{wissen}, welche Richtung die richtige ist. Möglicherweise
könnte aus der in Anm.~\ref{Anmerkung:PetersOrientierungsschwierigkeiten}
geschilderte Fall hier als Beispiel dienen.}
\item \emph{Rule of Belief}. Jede Möglichkeit muss berücksichtigt werden, von
der das Subjekt selbst überzeugt ist oder von der es vernünftigerweise überzeugt
sein sollte, weil sie sich aus einer anderen Überzeugung ergibt.
Wenn Jasmin die Überzeugung hat, zu halluzinieren, (oder überzeugt ist, ein
starkes Halluzinogen eingenommen zu haben), dann muss sie die Möglichkeit in
Betracht ziehen, sich in ihrer Wahrnehmung zu täuschen. Und wenn Jasmin
\emph{glaubt}, dass Peter lügt, dann kann sie auf der Grundlage seiner
Auskunft kein testimoniales Wissen erlangen, auch dann nicht, wenn Peter nach
bestem Wissen und Gewissen die Wahrheit sagt. Das gleiche gilt, wenn Jasmin offensichtliche Gründe hat
anzunehmen, dass Peter ein unzuverlässiger Informant ist, ohne deshalb
misstrauisch zu werden. Jasmin könnte beispielsweise wissen, dass Peter eine
Amsel nicht von einer Krähe unterscheiden kann, ihm aber dennoch glauben, dass
eine Amsel im Garten sitzt, weil sie gerade nicht daran denkt, dass Peter
hierbei kein zuverlässiger Informant ist. Sie sollte dann vernünftigerweise von
der Möglichkeit überzeugt sein, dass Peter sich täuscht, weil sich diese
Überzeugung aus einer anderen Überzeugung ergibt, die sie hat.
\item \emph{Rule of Reliablity}. Wenn wir Informationen durch Prozesse erlangen,
die normalerweise zuverlässig ablaufen, dann dürfen wir die
Möglichkeit ihres Versagens außer Acht lassen. Zumindest gilt dies so lange, wie
keine Gegengründe (\enquote{\emph{defeaters}}) gegen ihre Zuverlässigkeit vorliegen.
Unsere visuelle Wahrnehmung gehört sicherlich zu diesen zuverlässigen Prozessen;
ihr dürfen wir vertrauen, solange nichts an der aktuellen Situation gegen ihre
Zuverlässigkeit spricht. Das Zeugnis anderer (\enquote{testimony}) wird von
\authorcite{Lewis:ElusiveKnowledge1996} selbst als ein solcher zuverlässiger
Prozess angeführt, der uns in der Regel verlässlich mit wahren Informationen
versorgt.\footcite[Vgl.][558]{Lewis:ElusiveKnowledge1996} Jasmin kann niemals
mit Gewissheit ausschließen, dass Peter lügt. (Andernfalls erwürbe sie kein
genuin testimoniales Wissen.) Aber sie darf diese Möglichkeit vielleicht
ignorieren, solange kein Grund zum Misstrauen vorliegt. Und zwar darf sie dies,
wenn Peters Mitteilungen (oder Mitteilungen im
allgemeinen\footnote{Offensichtlich besteht hier die Schwierigkeit zu
bestimmen, welche konkreten Vorgänge zu der relevanten Art von Prozessen
zusammenzufassen sind, deren Zuverlässigkeit hier zugrunde zu legen ist.
Gehört Peters Mitteilung, dass $p$, zu den legitimen Informationsvorgängen,
weil Mitteilungen generell zuverlässige Prozesse darstellen? Sind
speziell Peters Mitteilungen die bestimmende Referenzklasse, deren
Zuverlässigkeit zu bewerten ist? Oder sind vielleicht Mitteilung bezüglich
bestimmter Themen zuverlässig, bezüglich bestimmter anderer Themen jedoch
nicht? All dies scheint ein entsprechender Ansatz, der die \emph{Rule of
Reliability} in der Erläuterung testimonialen Wissens zugrunde legt, weiter
klären zu müssen.}) normalerweise zuverlässige Informationsquellen darstellen
und nur in Ausnahmefällen zu Fehlurteilen führen.
\end{nummerierung}

Bei dieser Auswahl handelt es sich vermutlich um die bekanntesten epistemischen
Regeln. Die anderen Regeln sind weniger einschlägig und vielleicht handelt es
sich bei näherer Analyse lediglich um Spezial- oder Anwendungsfälle der ersten
drei Regeln, wie \authorcite{Lewis:ElusiveKnowledge1996}
selbst einräumt.\footnote{\cite[Vgl.][559]{Lewis:ElusiveKnowledge1996}.
\authorfullcite{Kern:QuellendesWissens2006} reduziert ihre Diskussion der Regeln
von \name{Lewis} sogar auf eine einzige, die \emph{Rule of Reliablity}, da diese
die zentrale epistemische Regel des Externalismus artikuliere.
\cite[Siehe][120]{Kern:QuellendesWissens2006}.} Ich stelle sie hier der Vollständigkeit wegen mit vor:
\begin{enumerate}[resume]
\item \emph{Rule of Resemblance}. Wenn eine Möglichkeit sehr stark an eine
andere Möglichkeit erinnert, die wir nicht ignorieren dürfen, dann darf sie auch
nicht ignoriert werden. Wenn wir nach der Einnahme eines bestimmten Medikaments
unserer visuellen Wahrnehmung misstrauen und keinen relevanten Unterschied
zwischen unserer visuellen und unserer auditiven Wahrnehmung kennen, dann
müssen wir es für möglich erachten, dass wir uns in dem täuschen, was wir zu
hören glauben. Und wenn Jasmin die Möglichkeit, dass Peter lügt, für real hält,
weil sie Peter für einen notorischen Lügner hält, und Klaus und Peter sich in
ihrem Verhalten kaum voneinander unterscheiden, dann hilft es ihr möglicherweise
nichts, wenn Klaus ihr eine Auskunft gibt und nicht Peter.
\item \emph{Rule of Conservatism}. Wir dürfen all diejenigen Möglichkeiten
ignorieren, von denen bekannt ist, dass sie für gewöhnlich bewusst ignoriert
werden dürfen. Wenn jeder die Möglichkeit, dass Peter lügt, als lächerlich
abtut, dann dürfen wir diese Möglichkeit ebenfalls ignorieren. Und wenn Peter
weit und breit Anerkennung als aufmerksamer Beobachter erfährt,
brauchen wir seinen diesbezüglichen Kompetenzen gegenüber keinerlei Zweifel
haben.
\item \emph{Rule of Attention}. Wenn wir eine Irrtumsmöglichkeit selbst gerade
im Blick haben, dann gehört sie zu den relevanten Möglichkeiten. Aber sie gehört
nur zu den relevanten Möglichkeiten aus der Perspektive desjenigen, der auf sie
achtet. Wenn \emph{wir} also darüber nachdenken, dass Peter lügen könnte, dann
müssen \emph{wir} diese Möglichkeit berücksichtigen und dann können \emph{wir}
über Peters Mitteilungen kein Wissen erwerben (und möglicherweise können
\emph{wir} Jasmin \emph{aus unserer Sicht} kein Wissen zuschreiben).
Jasmin hingegen, die über diese Irrtumsmöglichkeiten gerade nicht nachdenkt, mag sie vernachlässigen
dürfen, insofern keine andere Regel vorschreibt, sie zu beachten. Diese Regel
führt letztlich dazu, dass wir, solange wir Erkenntnistheorie betreiben und
skeptische Szenarien entwickeln, jegliche Wissensansprüche zurückweisen müssen,
bis wir die skeptischen Szenarien widerlegt haben.
\end{enumerate}
Es könnte also sein, dass die eine oder andere dieser Regeln nur einen
Spezialfall einer anderen Regel artikuliert. Aber dieser Aspekt von
\authorcite{Lewis:ElusiveKnowledge1996}' Systematik braucht uns hier nicht
weiter zu beschäftigen. Wichtiger ist folgende Einteilung, die ich hier am
Beispiel der drei ersten Regeln expliziere:


Die \emph{Rule of Actuality} und die \emph{Rule of Reliability} artikulieren
\emph{externalistische}
Annahmen,\footnote{\authorcite{Lewis:ElusiveKnowledge1996} selbst nennt die
\emph{Rule of Actuality} externalistisch
\parencite[vgl.][554]{Lewis:ElusiveKnowledge1996}, aber es ist klar, dass gerade
die \emph{Rule of Reliabilism} eine Grundaussage des epistemischen Externalismus
artikuliert \parencite[vgl.][120]{Kern:QuellendesWissens2006}.} denn sie nehmen
auf etwas Bezug, das dem Subjekt nicht notwendigerweise zugänglich sein muss.
Die \emph{Rule of Actuality} gebietet uns die Berücksichtigung von
Möglichkeiten, die \emph{de facto} realisiert sind, unabhängig davon, ob wir um
ihr Vorliegen wissen. Ebenso erlaubt uns die \emph{Rule of Reliability}, die
Möglichkeit des Versagen \emph{de facto} zuverlässiger Prozesse so lange zu
ignorieren, bis uns Gründe bekannt werden, die explizit dafür sprechen, dass
eine solche Möglichkeit verwirklicht ist. Ihre Anwendung hängt explizit nicht
davon ab, dass wir um die Zuverlässigkeit der Prozesse wissen. Die
\emph{Rule of Belief} charakterisiert hingegen internalistische Positionen, denn
sie macht unsere Zustimmung zu Aussagen abhängig von Erkenntnissen, die dem
Subjekt selbst verfügbar sind. Sie fordert die Berücksichtigung von
Möglichkeiten, für deren Verwirklichung uns Gründe bekannt sind, unabhängig
davon, ob die Möglichkeiten \emph{de facto} realisiert sind.

Eine Eigenschaft externalistischer Erlaubnisregeln wird noch wichtig werden: Sie
gelten stets nur \emph{prima facie}; sie verlieren ihre Kraft, wenn es Anzeichen dafür gibt, dass
die entsprechende Grund\-an\-nahme in einem Fall falsch ist, wenn beispielsweise
etwas dafür spricht, dass ein in der Regel zuverlässiger Prozess nicht standardmäßig
abläuft. Man spricht in diesem Fall von einem \enquote{\emph{defeater}}. Ein
\emph{defeater} ist also das Wissen um einen Umstand, bei dessen Vorliegen die
Anwendung einer Erlaubnisregel unzulässig ist.

Betrachten wir zunächst die externalistische Verbotsregel, die \emph{Rule of
Actuality}. Sie sagt, dass tatsächlich vorliegende Störungen im
Informationsfluss nicht ignoriert werden dürfen, und schließt damit die
klassischen \authorcite{Gettier:IsJuftifiedTrueBeliefKnowledge?1963}-Beispiele, die sich selbstverständlich auch
im Bereich testimonialer Erkenntnisse konstruieren lassen, aus. Man kann so
reagieren, dass man die Bedingungen für das Vorliegen von testimonialem Wissen
von äußeren Faktoren abhängig macht und mit Thomas \name[Thomas]{Grundmann} etwa
Folgendes proklamiert:
\begin{quote}
  Das Zeugnis anderer ist dann eine zuverlässige
  Quelle, wenn der Informant aufrichtig ist und seine Äußerungen seinerseits auf
  zuverlässige Quellen stützt. Sobald diese Bedingungen objektiv erfüllt sind,
  kann das Zeugnis anderer Wissen oder gerechtfertigte Überzeugungen liefern,
  auch wenn der Rezipient davon kein explizites Wissen
  besitzt.\footnote{\cite[][537]{Grundmann:AnalytischeEinfuehrungindieErkenntnistheorie2008}.}
\end{quote}
Wir sind dann gerechtfertigt in unserer Überzeugung und besitzen testimoniales
Wissen, wenn unser Informant eine zuverlässige Wissensquelle ist, die in
unserem Fall tatsächlich über Wissen verfügt und uns dieses
mitteilt.\footnote{\cite[Vgl.][]{Grundmann:DietraditionelleErkenntnistheorieundihreHerausforderer2001}
sowie
\cite[][529--541]{Grundmann:AnalytischeEinfuehrungindieErkenntnistheorie2008}.}
Nehmen wir beispielsweise an, Peter weiß tatsächlich, dass es regnet, und Jasmin
hat keine Gründe, an Peters Glaubwürdigkeit zu zweifeln, und kommt infolge dessen zu
der Überzeugung, dass es regnet. Wenn Peter tatsächlich weiß, dass es regnet und
tatsächlich zuverlässig die Wahrheit sagt, dann weiß nun auch Jasmin, dass es
regnet. Wenn Peter aber \emph{de facto} gar nicht weiß, dass es regnet, oder
wenn Peter \emph{de facto} keine zuverlässige Informationsquelle ist, dann
erwirbt Jasmin kein testimoniales Wissen -- auch dann nicht, wenn es tatsächlich
regnet. Somit entzieht sich unserer eigenen Kenntnis, ob es sich bei unseren
eigenen Überzeugungen tatsächlich um testimoniales Wissen handelt. Ob eine
Rechtfertigung vorliegt \emph{respective} ob es sich um eine hinreichende
Rechtfertigung für Wissen handelt, hängt also von uns unbekannten
Kontextfaktoren
ab.\footnote{\cite[Vgl.][532]{Grundmann:AnalytischeEinfuehrungindieErkenntnistheorie2008}:
\enquote{Solange man {\punkt} annimmt, dass eine gerechtfertigte Überzeugung
(oder Wissen) nur zustande kommt, wenn man selbst die Gründe kennt, die die
Wahrheit der Überzeugung zumindest wahrscheinlich machen, dann fehlen genau
diese Gründe im Fall der Information durch Dritte. {\punkt} Hinter der Ablehnung
des Zeugnisses anderer als Wissens- und Rechtfertigungsquelle steht also der
erkenntnistheoretische Zugangsinternalismus.}}

Ein Vorteil in der Einbeziehung externalistischer Regeln liegt -- so die
Schlussfolgerung des Externalismus -- darin, dass Positionen, die wie diejenige
\name[Thomas]{Reid}s oder der lokale Reduktionismus unserer epistemischen
Situation Rechnung tragen, damit mühelos funktionieren.
Externalistische Regeln erlauben uns, mit der Tatsache umzugehen, dass wir
testimoniales Wissen erlangen und erlangen müssen, \emph{bevor} wir in der Lage
sind, die Zuverlässigkeit der Quellen zu beurteilen, und beschreiben diesen
Prozess des Wissenserwerbs doch zugleich als vernünftig und legitim. Wir können
testimoniales Wissen über Informanten erlangen, bevor wir ihre Zuverlässigkeit
oder die Zuverlässigkeit von Informanten im allgemeinen einschätzen können, weil
es sich \emph{de facto} um zuverlässige Informationsquellen handelt. Ihr größter
Vorteil liegt freilich darin, dass wir mit ihrer Hilfe erläutern können, wie wir
uns in Situationen befinden können, in denen kein Wissen vorliegt und die sich
dennoch aus unserer Perspektive in keiner Hinsicht von Situationen
unterscheiden, in denen Wissen vorliegt. Situationen, in denen wir alles richtig
gemacht zu haben scheinen und dennoch irren,\footnote{Es kann hier offen
bleiben, ob es denkbar ist, dass wir keinen Fehler begehen und dennoch falsche
Überzeugungen haben. Man könnte entgegnen, dass wir in jedem Fall eine Regel
falsch angewandt haben, wenn ein Irrtum resultiert, auch wenn uns dies nicht
bewusst ist und vielleicht jeder andere an unserer Stelle denselben Fehler
begangen hätte.} oder die
\authorcite{Gettier:IsJuftifiedTrueBeliefKnowledge?1963}-Beispiele illustrieren dies.

Die externalistischen Überlegungen haben sicherlich eine gewisse Plausibilität,
insofern es um den Begriff des Wissens geht. Aber taugen sie auch zur Explikation des Mündigkeitsbegriffs?
Aus der Perspektive einer am Ausgang des Menschen aus seiner selbst verschuldeten
Unmündigkeit orientierten Aufklärung bleibt die externalistische Lösung
unbefriedigend: Wenn uns unbekannt ist, ob die Anwendungsbedingung einer
epistemischen Regel erfüllt ist, dann ist schwer zu sehen, wie damit ein
kritischer Umgang mit Mitteilungen anderer von einem unkritischen abgegrenzt
werden soll. Denn es scheint weitgehend unserem \singlequote{epistemischen
Glück}\footnote{Dieser Ausdruck hat sich in Anlehnung an einen Ausdruck von
Bernard \textcite[vgl.][]{Williams:MoralLuck1981} verbreitet; siehe z.\,B.
\cite{Engel:IsEpistemicLuckCompatiblewithKnowledge1992}.} anheim gestellt, ob wir
in unserem Vertrauen in die Mitteilung richtig handeln oder nicht. Insofern
Aufklärung den Ausgang des Menschen aus seiner selbst verschuldeten Unmündigkeit
fordert, kann sie nicht auf externalistische Regeln verweisen, die
letztlich unabhängig von einen epistemisch verantwortungsvollen oder
verantwortungslosen Umgang des Einzelnen mit seinen Informationsquellen
lediglich festlegen, wann \emph{Wissen} vorliegt.


Die Frage, ob Wissen vorliegt, ist eine andere Frage als die, ob jemand
sich als mündig erweist. Und Mündigkeit ist wiederum nach \name[Immanuel]{Kant}s
Auffassung etwas, das man sich im Laufe der Jahre erst erarbeiten muss. Auch der
Unmündige verfügt bereits über Wissen. Umgekehrt mag es sogar möglich sein,
Mündigkeit und Irrtum zu verbinden. Wenn Aufklärung in dem Ausgang des Menschen
aus seiner selbst verschuldeten Unmündigkeit besteht und eine veränderte
\enquote{Denkungsart} erfordert, es ihr also nicht primär um wahre Inhalte,
sondern um die Freiheit des Denkens geht, dann könnte auch der aufgeklärteste
Mensch noch falsche Überzeugungen haben können.\footnote{Siehe hierzu auch die
Überlegungen in Kapitel \ref{Zitat:Lessing:EineDuplik}, die einer Äußerung
\authorcite{Lessing:EineDuplik1897}s kulminieren, die ich auf Seite
\pageref{Zitat:Lessing:EineDuplik} zitiere.} Wir irren auch dort, wo wir unseren
eigenen Verstand gebrauchen und unsere Vernunft zum obersten \enquote{Probirstein} der Wahrheit erheben.

Wenngleich es also möglich sein mag, Wissen zu erwerben, ohne mündig zu sein, so
besteht das Ziel der Aufklärung doch zweifelsfrei im mündigen Wissenserwerb. Und
dies wiederum muss heißen, dass der Wissenserwerb sich an internalistischen
epistemischen Regeln orientiert. Die Mutmaßung, wonach Aufklärungsdenken dem
Individualismus verpflichtet zu sein scheint, weil es sich notwendig gegen eine externalistische Position wenden
muss, habe ich weiter oben bereits angesprochen.\footnote{Siehe
S.~\pageref{Absatz:AufklaerungundZugangsInternalismus}--\pageref{Absatz:AufklaerungundZugangsInternalismus-ENDE}.}
Aufklärung und die Forderung nach epistemischer Autonomie forderten einen
epistemischen Internalismus und dieser ziehe wiederum einen epistemischen
Individualismus nach sich. Es ist klar, dass dieser Überlegung widersprochen
werden muss, wenn Aufklärung eine ernsthafte Option sein soll. Dabei scheint mir
aber nicht die Verbindung von Aufklärung und Internalismus zurückzuweisen zu
sein, sondern die Behauptung, der Internalismus ziehe den Individualismus
nach sich. Aufklärungsphilosophie ist \emph{per se} dem Internalismus
verpflichtet, denn sie erwartet von dem mündigen Subjekt die \emph{eigene}
Kontrolle der Gründe für die je eigenen Überzeugungen.

\end{comment}


\section{Individualistische Ansätze in der Philosophie der
Neuzeit}\label{subsection:TestimonialesWissenUndMuendigkeit}
Die Frage nach der Möglichkeit testimonialen Wissens wird oft vor dem
Hintergrund eines angenommenen erkenntnistheoretischen Individualismus innerhalb der Philosophie
der Neuzeit diskutiert,\footnote{Ich behaupte damit nicht, dass der
Individualismus ein prägendes Element oder gar Grundlage der neuzeitlichen Philosophie sei.
Vielmehr werde ich darauf verweisen, dass weite Teile der
Aufklärungsphilosophie den (erkenntnistheoretischen) Individualismus explizit
zurückweisen. Einen ersten Teil dieser Zurückweisung lieferte bereits Kapitel
\ref{section:sensuscommunis}.} als dessen Gewährsmann
\authorfullcite{Descartes:OeuvresdeDescartes1983}
gilt.\footnote{\cite[Vgl.][\pno~407f.]{Welbourne:IsHumeReallyaReductivist2002}:
\enquote{Perhabs the epistemic individualism inculcated by
\authorcite{Descartes:OeuvresdeDescartes1983} has caused philosophers either to
neglect or to despise testimony.}
\authorfullcite{Grundmann:AnalytischeEinfuehrungindieErkenntnistheorie2008}
nennt hingegen \singlename{Platon} und \name[John]{Locke} als Beispiele dieses
Individualismus (\cite[vgl.][\pno~530f.]{Grundmann:AnalytischeEinfuehrungindieErkenntnistheorie2008}).}
Der Individualismus muss bestreiten, dass testimoniales Wissen im Sinne eines
\emph{primary epistemic link} möglich ist. Das heißt, er muss den Credulismus
zurückweisen und sich zwischen den Positionen des testimonialen Reduktionismus
und des testimonialen Skeptizismus entscheiden. Wir werden mit Thomas
\name[Thomas]{Reid} sehen, wie unattraktiv beide Positionen sind (Abschnitt
\ref{subsubsection:ThomasReid}). Hier soll aber zunächst erläutert werden, wie
für sie argumentiert werden kann. Bevor ich auf den testimonialen Reduktionismus
eingehe, für den David \name[David]{Hume} Pate steht (Abschnitt
\ref{subsubsection:DavidHume}), sollen daher nun skeptische Argumente von
\authorcite{Descartes:OeuvresdeDescartes1983} Erwähnung finden.




\subsection{Ren{\'e} Descartes' testimonialer
Skeptizismus}\label{subsection:DescartesKritikantestimonialemWissen}
Während ich vorhin mit \name[Immanuel]{Kant} die Vorteile epistemischer Arbeitsteilung
anhand der Analogie zu Handel und Gewerbe aufzeigte\footnote{Siehe oben Kapitel
\ref{Abschnitt:EpistemischeArbeitsteilung}.},
\Revision[Pelletier]{beginnt} \authorcite{Descartes:OeuvresdeDescartes1983}
\Revision[Pelletier]{seine Überlegungen mit der Betonung problematischer
Aspekte des Vertrauens auf das mitgeteilte Wissen Anderer}.
Er behauptet im \titel{Discours de la M{\'e}thode}, \enquote{daß Werke, die aus
mehreren Teilen zusammengesetzt sind und von der Hand verschiedener Meister
gefertigt wurden, häufig nicht dieselbe Vollkommenheit aufweisen wie die, an
denen nur ein einzelner gearbeitet hat.}\footnote{\cite[][VI:
11]{Descartes:OeuvresdeDescartes1983}.} Und er vergleicht die Gewinnung
wissenschaftlicher Erkenntnis dann auch nicht mit der effizienteren
arbeitsteiligen Produktion von Gütern in Fabriken und Manufakturen, sondern mit
der Arbeit an Werken der Kunst und Architektur. Diese seien vollkommener, wenn
einer allein den Plan macht. Arbeiten hingegen viele Menschen nach ihren jeweils
eigenen Vorstellungen an Teilen des Werkes, ohne durch einen gemeinsamen, den
jeweiligen Vorstellungen der einzelnen Ausführenden vorausgehenden
ursprünglichen Plan zusammengehalten zu werden, dann gebe es kein
wohl proportioniertes Ganzes, sondern ein unproportioniertes Zufallsprodukt, das
an die verwinkelten Straßen und Wege einer alten und über Jahrhunderte
gewachsenen Stadt erinnere, wo statt vernünftiger Absichten bloß der Zufall
regiere.\footnote{\cite[Vgl.][VI: 11--13]{Descartes:OeuvresdeDescartes1983}.}

Diese Passagen artikulieren zunächst kein klar erkennbares Argument; und in
unserer heutigen Zeit neigt das ästhetische Empfinden vielleicht mehr als das von
\authorcite{Descartes:OeuvresdeDescartes1983} und seinen Zeitgenossen dazu, die natürliche und
ungleichförmige Gewachsenheit, wie wir sie in historischen Stadtkernen
exemplifiziert finden, mindestens ebenso, wenn nicht sogar höher zu schätzen als
das Regelmäßige, Systematische und Geplante, das uns in Neubaugebieten,
Satellitenstädten und Trabantensiedlungen entgegen schlägt. Freilich ist dieser
ästhetische Reiz verwinkelter und unübersichtlicher Gassen nicht das, was wir in
wissenschaftlichen Erkenntnissen suchen. Hier geht es uns um
Transparenz und Übersichtlichkeit. Aber dass kooperative Forschung hier
unterlegen sei, diesen Irrtum widerlegt die Erfahrung der
Wissenschaftsgeschichte der vergangenen Jahrhunderte. Selbst den Anschein eines
überzeugenden Arguments kann diese Analogie mit Stadtbildern daher nicht mehr
erwecken; sie macht lediglich \authorcite{Descartes:OeuvresdeDescartes1983}' Grundhaltung anschaulich, die
sich gegen das \singlequote{Studium der Büchergelehrsamkeit}
(\singlequote{l'{\'e}tude des lettres}) älterer Bildungsvorstellungen richtet,
diesem aber nicht das moderne Modell kooperativen Forschens, sondern die
Vorstellung eines zurückgezogenen Denkers
entgegenstellt.\footnote{\cite[Vgl.][VI: 9]{Descartes:OeuvresdeDescartes1983}.}

In den \titel{Regulae ad directionem
ingenii}, die ihm als Vorlage für den \titel{Discours de la M{\'e}thode}
dienten, entwickelt \authorcite{Descartes:OeuvresdeDescartes1983} eine sachhaltigere und systematisch
augefeiltere Argumentation zur Zurückweisung testimonialer Erkenntnisse, die in
drei Stufen vorgeht, die sich auf den Mangel an Deutlichkeit mitgeteilter
Erkenntnisse (\ref{Abschnitt:DescartesunddieUeberinterpretation}), tatsächlich
und möglicherweise widerstreitende Meinungen
(\ref{Abschnitt:DescartesundderWiderstreitderMeinungen}), und die Differenz von
Wissenschaft und bloß historischer Kenntnis
(\ref{Abschnitt:DescartesundhistorischeKenntnisse}) beziehen\footnote{Von
diesen wird jedoch nur das \ref{Abschnitt:DescartesundderWiderstreitderMeinungen}. Argument
in die weitere Argumentation des \titel{Discours} übernommen.}:
\begin{nummerierung}
 \item\phantomsection\label{Abschnitt:DescartesunddieUeberinterpretation}
 Zunächst sei mitgeteilten Informationen zu misstrauen, weil die Informanten oft
 selbst leichtgläubig und unvorsichtig und teils sogar missgünstig seien. Dabei sieht
 \authorcite{Descartes:OeuvresdeDescartes1983} durch Eitelkeit und Geltungsdrang begründete
 Anreizstrukturen, die der Genese und Weitergabe sicheren Wissens entgegenstehen
 und Übertreibungen provozieren.\footnote{\cite[Vgl.][X:
 367.5--23]{Descartes:OeuvresdeDescartes1983}.} Ein Autor sei durch eigenen
 Geltungsdrang stets versucht, seine Ergebnisse überzuinterpretieren und ihnen
 durch schwer verständliches Vokabular den Anschein größeren intellektuellen
 Gewichts oder besonderer Wissenschaftlichkeit zu geben, statt bloß das wirklich
 Erreichte in nüchterner Sprache ohne Übertreibung vorzutragen.
 Der Witz der Argumentation besteht darin, auf stets vorhandene Anreizstrukturen
 und Versuchungen aufmerksam zu machen, denen wohl jeder Autor zumindest
 gelegentlich erliegt und die jedem Adressaten einen ersten Grund liefern, an
 der Bonität seiner Informationsquelle zu zweifeln.
 Dieser Grund ist dabei gänzlich unabhängig von besonderen Umständen wie den
 Charaktereigenschaften der jeweiligen Person und liegt daher universell vor.
 Nicht zwingend handelt es sich um eine bewusste Täuschung, deren Möglichkeit
 unterstellt wird; es könnte auch eine unbewusste -- wenngleich möglicherweise
 schuldhafte -- Unvorsichtigkeit vorliegen, die mit einer \emph{Selbst}täuschung
 des Autors einhergeht. Das Argument zielt darauf, unser Vertrauen in die 
 \emph{Redlichkeit und Vorsichtigkeit} oder das
 \emph{\singlequote{intellektuelle Verantwortungsbewusstsein} der Autoren} zu
 unterminieren, ohne dabei ein besonders pessimistisches Menschenbild zugrunde
 legen zu müssen.
 
 Dass jemand ein zuverlässiger Informant ist, setzt zuallererst voraus, dass er
 selbst zuverlässig und vorurteilsfrei zu erkennen vermag, damit wir nicht seine Irrtümer und
 Vorurteile übernehmen. Als allgemeines Kriterium für Zuverlässigkeit und
 Vorurteilsfreiheit nennt \authorcite{Descartes:OeuvresdeDescartes1983} die Klarheit und Deutlichkeit einer
 Erkenntnis.\footnote{\enquote{[V]ideor pro regul{\^a} generali posse statuere,
 illud omne esse verum, quod valde clare {\&} distincte percipio}
 \mkbibparens{\cite[][VII:
 35.14--15]{Descartes:OeuvresdeDescartes1983}, siehe auch
 \cite[][VIII:
 16.18--17.10]{Descartes:OeuvresdeDescartes1983}}. Berühmt wurde die
 Unterscheidung durch die weitere Ausarbeitung bei
 \textcite[vgl.][]{Leibniz:Meditationesdecognitioneveritateetideis1999}.}
 \emph{Klar} nennt \authorcite{Descartes:OeuvresdeDescartes1983} eine Erkenntnis (hier:
 \emph{perceptio}), die uns gegenwärtig (\emph{praesens}) und zugänglich
 (\emph{aperta}) ist. Als Beispiel für eine klare Erkenntnis nennt
 \authorcite{Descartes:OeuvresdeDescartes1983} in den \titel{Principia
 philosophiae} einen starken
 Schmerz\footnote{\cite[Vgl.][VIII:
 22.10--11]{Descartes:OeuvresdeDescartes1983}.} und in den \titel{Meditationes
 de prima philosophia} die Erkenntnis, dass ich denke und dass ich 
 bin\footnote{\cite[Vgl.][VII:  35.3--15]{Descartes:OeuvresdeDescartes1983}.}.
 \emph{Deutlich} ist wiederum eine klare Erkenntnis, die darüber hinaus nichts
 enthält, was nicht klar ist (und zwar, \emph{weil} sie deutlich
 ist).\footnote{\enquote{Claram voco illam
 [perceptionem; A.\,G.], quae menti attendenti praesens {\&} aperta est:
 sicut ea clar{\`e} {\`a} nobis videri dicimus, qu{\ae}, oculo intuenti
 pr{\ae}sentia, satis fortiter {\&} apert{\`e} illum movent. Distinctam autem
 illam, quae, c{\`u}m clara sit, ab omnibus aliis ita sejuncta est {\&}
 praecisa, ut nihil plan{\`e} aliud, qu{\`a}m quod clarum est, in se contineat}
 \mkbibparens{\cite[][VIII: 22.3--9]{Descartes:OeuvresdeDescartes1983}}.} So ist die
 Wahrnehmung des Schmerzes nur dann klar, wenn sie nicht mit Annahmen über die Natur und Ursache des Schmerzes
 vermischt ist. Und das \enquote{cogito, sum} stellt nur dann eine deutliche
 Erkenntnis dar, wenn wir nicht glauben, es folge neben unserer Existenz als
 \emph{res cogitans} auch unsere körperliche Existenz als \emph{res extensa}.
 
 \authorcite{Descartes:OeuvresdeDescartes1983} behauptet, dass die meisten Menschen nie in ihrem Leben etwas
 klar und deutlich
 erkennen.\footnote{\cite[Vgl.][VIII: 22]{Descartes:OeuvresdeDescartes1983}.}
 Nun ist es alltäglich, dass sich unserer Aufmerksamkeit Dinge klar darstellen; jedes Kind erkennt nach
 \authorcite{Descartes:OeuvresdeDescartes1983} Dinge klar. Was fehlt, ist die Deutlichkeit unserer
 Erkenntnisse. Wer etwas klar, aber nicht deutlich erkennt, der erkennt eine
 Wahrheit mit Gewissheit, aber mischt Überzeugungen mit hinein, die in dem, was
 er erkennen konnte, nicht enthalten sind, und für die daher eine entsprechende
 Evidenz fehlt. Wir überschätzen in diesem Fall unsere Erkenntnis. Und wenn die
 meisten Menschen vielleicht vieles klar, aber kaum etwas deutlich erkennen,
 dann ist zu erwarten, dass ihre Mitteilungen Vorurteile enthalten, also
 Behauptungen, die durch das klar Erkannte nicht gedeckt sind. Eben dies liegt
 auch \authorcite{Descartes:OeuvresdeDescartes1983}' Argument zugrunde, Autoren neigten oft dazu, ihre
 eigenen Erkenntnisse zu überschätzen.\footnote{Siehe oben,
 S.~\pageref{Abschnitt:DescartesunddieUeberinterpretation}.}
 
 Es geht bei diesem Argument um die Verletzung epistemischer Pflichten bei dem
 ursprünglichen Erwerb einer Erkenntnis durch den Informanten selbst (durch
 Wahrnehmung, Vernunfttätigkeit oder auch wiederum durch die Mitteilung anderer)
 ebenso wie um die Verletzung seiner Aufrichtigkeitspflicht bei der Weitergabe
 des Wissens. \authorcite{Descartes:OeuvresdeDescartes1983} muss hierbei nicht behaupten, dass die
 Verletzung solcher Pflichten die Regel und nicht die Ausnahme sei. Ebenso wie sein
 Traumargument in den \titel{Meditationes de prima philosophia} nur auf der
 Prämisse aufbaut, dass wir \emph{manchmal} träumen, ohne ein Kriterium zur
 Abgrenzung von Traum und Wirklichkeit zu
 haben,\footnote{\phantomsection\label{Anmerkung:KriteriumzurUnterscheidungTraumWirklichkeitbeiDescartes}\cite[Vgl.][VII:
 19.8--23]{Descartes:OeuvresdeDescartes1983}. In der sechsten Meditation nennt
 er freilich als ein solches Kriterium die Kohärenz eines Erlebens mit unseren
 anderen (v.\,a~früheren) Erkenntnissen: \enquote{nunc enim
 adverto permagnum inter utrumque ese discrimen, in eo qu{\`o}d nunquam
 insomnia cum reliquis omnibus actionibus vitae a memori{\^a} conjugantur, ut
 ea quae vigilanti occurrunt} \mkbibparens{\cite[][VII: 89.21--25]{Descartes:OeuvresdeDescartes1983}}.} reicht hier die
 Prämisse, dass Autoren wenigstens \emph{manchmal} unredlich oder in ihrem eigenen Erkenntnisgewinn
 unvorsichtig sind, ohne dass wir aus unserer Perspektive darüber urteilen
 können, ob sie es in einem konkreten Fall sind. Wie das Argument des bösen
 Dämons in den \titel{Meditationes} zeigt, genügt sogar die bloße
 Denkmöglichkeit eines Irrtumsszenarios, um Wissensansprüche zu unterminieren.
 
 \item\phantomsection\label{Abschnitt:DescartesundderWiderstreitderMeinungen}
 Selbst wenn wir keinen Grund hätten, anderen Menschen Unredlichkeit oder
 Unvorsichtigkeit zu unterstellen, bliebe doch ein gewichtiger Grund zum Zweifel
 an testimonialem Wissen bestehen: Menschen vertreten \emph{widersprechende Meinungen}. Wie nach
 ihm David \name[David]{Hume} und Immanuel \name[Immanuel]{Kant}\footnote{Vgl.
 \cite[][3]{Hume:ATreatiseofHumenNature2007}; \cite[][A
 ix]{Kant:KritikderreinenVernunft2003}, \cite[][IV:
 8.4--9]{Kant:GesammelteWerke1900ff.}; \cite[][B
 vii--ix]{Kant:KritikderreinenVernunft2003}, \cite[][III:
 7.2--8.25]{Kant:GesammelteWerke1900ff.}.} sieht \authorcite{Descartes:OeuvresdeDescartes1983} durch die
 Verbreitung sich widersprechender Meinungen die Autorität von Wissenschaften
 untergraben, wobei \name[Immanuel]{Kant} und \name[David]{Hume} dadurch insbesondere das Ansehen
 der Metaphysik gefährdet sehen. Die Uneinheitlichkeit des für wahr
 Gehaltenen mag auf manchen Gebieten stärker und auf anderen weniger stark
 ausgeprägt sein, ja innerhalb der einen oder anderen Disziplin gibt es
 vielleicht gar keine offensichtlichen Differenzen oder es gibt nur eine sehr kleine
 Minderheit, die einer überwältigenden Mehrheit gegenübersteht.
 
 \authorcite{Descartes:OeuvresdeDescartes1983} sieht wiederum in der \emph{Möglichkeit} des
 Widerspruchs das Ansehen \emph{jedes} tradierten Wissens unterminiert, auch
 dort, wo faktisch keine große Gruppe an Abweichlern von einer verbreiteten
 Auffassung existiert. Das Ansehen einer Meinung kann schließlich nicht von der
 \emph{Anzahl} derer abhängen, die sie vertreten:
 Stimmenmehrheit ist kein Argument bei der Erforschung der Wahrheit. Man
 bedenke, dass die Vertreter aller wissenschaftlichen Revolutionen -- vom
 Heliozentrismus bis zur Relativitätstheorie -- zunächst eine kleine Minderheit
 bildeten.\footnote{Nach
 \authorfullcite{Kuhn:DieStrukturwissenschaftlicherRevolutionen1967}
 \enquote{werden die wissenschaftlichen Revolutionen durch ein wachsendes, doch
 {\punkt} oft auf eine kleine Untergruppe der wissenschaftlichen Gemeinschaft
 beschränktes Gefühl eingeleitet, daß ein existierendes Paradigma aufgehört hat,
 bei der Erforschung eines Aspekts der Natur {\punkt} in adäquater Weise zu
 funktionieren}
 \parencite[][104]{Kuhn:DieStrukturwissenschaftlicherRevolutionen1967}.}
 
 Hierin scheint \authorcite{Descartes:OeuvresdeDescartes1983} sein stärkstes Argument zu
 sehen, denn im \titel{Discours de la M{\'e}thode} wird diese Diagnose zum
 entscheidenden Argument gegen die leichtfertige Berufung auf testimoniales
 Wissen: Durch die Tatsache, dass viele einander widerstreitende Meinungen
 existieren, werde das Selbstdenken quasi zur unfreiwilligen Notwendigkeit. Denn
 selbst \emph{wenn} wir willens wären, unsere epistemische Selbständigkeit
 aufzugeben, müssten wir uns doch zumindest entscheiden, \emph{wessen} Meinung
 wir als die unsrige übernehmen. Aber ohne eigenes Wissen scheint es
 aussichtslos zu sein, eine begründete Entscheidung zugunsten bestimmter
 Informanten und gegen andere zu
 fällen.\footnote{\enquote{[I]e ne pouuois
 choisir personne dont les opinions me semblassent deuoir estre
 prefer{\'e}e a celles des autres, {\&} ie me trouuay comme
 contraint d'entreprendre moymesme de me conduire}
 \mkbibparens{\cite[][VI: 16.26--29]{Descartes:OeuvresdeDescartes1983}}.}
 Jedenfalls beweist der Dissens zwischen Informanten, dass falsche Auskünfte wirklich und wir daher im
 Falle testimonialer Überzeugungen tatsächlich mit der Möglichkeit der
 Fehlinformation konfrontiert sind. Ob unser Informant seinerseits
 epistemische Pflichten verletzt hat, kann dabei offen bleiben.
 
 \item\phantomsection\label{Abschnitt:DescartesundhistorischeKenntnisse} Obwohl
 \authorcite{Descartes:OeuvresdeDescartes1983} damit den Zweifel an der Möglichkeit testimonialen Wissens
 hinreichend begründet zu haben glaubt, liefert er ein drittes Argument.
 Selbst unter der kontrafaktischen Annahme der Zuverlässigkeit sämtlicher
 Mitteilungen, der Einhelligkeit unter Informanten und der absoluten Redlichkeit
 der Überbringer des Wissens, bliebe der Erwerb von Wissen aus Büchern
 unbefriedigend (und dasselbe gilt wohl auch für Wissen, das auf anderem Wege,
 aber immer noch aus zweiter Hand erworben wurde); denn wir erwürben dadurch
 keine Wissenschaft, sondern \singlequote{historische Kenntnisse} von
 Wissenschaften.\footnote{\enquote{[I]ta enim non scientias videremus didicisse,
 sed historias} \mkbibparens{\cite[][X:
 367.22--23]{Descartes:OeuvresdeDescartes1983}}.} Als Beispiel eignet sich die
 Mathematik als Disziplin, die all denen, die sich am ihrem
 Methodenideal orientieren, wegen der Gewissheit und Klarheit ihrer
 Erkenntnisse und der Einhelligkeit der Mathematiker untereinander als Paradigma
 von Wissenschaftlichkeit gilt.\footnote{Aber auch \name[Immanuel]{Kant}, der
 eine Orientierung an der Mathematik für die Philosophie zurückweist, sieht die
 Mathematik privilegiert; \cite[vgl.][B~x]{Kant:KritikderreinenVernunft2003},
 \cite[][III:  9.7--9]{Kant:GesammelteWerke1900ff.}. Siehe dazu auch
 unten, Kap.~\ref{subsubsection:EndlichesundUnendlichesErkennen}.} Hier können wir annehmen, dass die
 ersten beiden Kritikpunkte an testimonialem Wissen zu vernachlässigen sind. Und doch
 findet \authorcite{Descartes:OeuvresdeDescartes1983} einen gewichtigen Grund, die
 \singlequote{Büchergelehrsamkeit} auch -- vielleicht sogar gerade -- innerhalb der Mathematik
 zurückzuweisen: Wir mögen Lehrbücher der Mathematik zurate ziehen, deren Inhalt
 fehlerfrei ist und deren Autoren keinerlei Absicht haben, uns zu betrügen. So
 wären wir doch keine Mathematiker, wenn wir lediglich
 deren Inhalte korrekt reproduzieren könnten. Mathematiker wären wir erst, wenn
 wir dieselben \emph{Kompetenzen} erworben haben, die bei der Generierung der in den
 Lehrbüchern enthaltenen Erkenntnisse angewandt wurden. Und dazu muss man die
 Beweise \emph{selbst} führen und das Wissen \emph{selbst} erzeugen
 (oder dies zumindest \emph{können}). Testimoniales Wissen widerstreitet der
 Forderung nach Methodenkompetenz, die stures Auswendiglernen nicht als
 Verdienst anerkennt.
 
 Es ist diese dritte Überlegung, die in der Philosophie der
 Aufklärung\footnote{Siehe unten, Kapitel
 \ref{subsection:BewertungvonInformationennachihrerART}.} bis hin zu
 \name[Immanuel]{Kant}\footnote{Siehe Kap.
 \ref{section:MuendigkeitundPhilosophie}.} und sogar darüber hinaus\footnote{So
 schreibt \authorcite{Hegel:GesammelteWerke} in der \titel{Phänomenologie des
 Geistes}:
 \enquote{Was die \ori{mathematischen} Wahrheiten betrift, so würde noch weniger
 [als bei historischen Wahrheiten; A.\,G.] der für einen Geometer gehalten
 werden, der die Theoreme \singlename{Euklid}s \ori{auswendig} wüßte, ohne ihre
 Beweise, ohne sie, wie man im Gegensatze sich ausdrücken könne, \ori{inwendig}
 zu wissen.} \mkbibparens{\cite[][IX: 31.35--32.1]{Hegel:GesammelteWerke}}.} ihr Potential entfalten wird.
 Eine solche Aversion gegen reines Reproduzieren-können gelernter Inhalte und eine entsprechende Forderung
 nach der Ausbildung methodischer Kompetenzen bestimmt später die Wirkung von Christian
 \name[Christian]{Thomasius} als eines universitären
 Lehrers.\footnote{\cite[Vgl.][241--243]{Albrecht:ChristianThomasius1999}.}
 In der Aufklärungsphilosophie im deutschsprachigen Raum von \name[Christian]{Thomasius} über
 \authorcite{Wolff:Psychologiaempirica1968} bis hin zu \name[Immanuel]{Kant} wird gerade dieses Argument den zentralen
 Gesichtspunkt der Diskussion um testimoniales Wissen bestimmen. Deswegen werde
 ich in den Kapiteln \ref{subsection:BewertungvonInformationennachihrerART} und
 \ref{section:MuendigkeitundPhilosophie} noch ausführlicher darauf eingehen.
 \authorcite{Descartes:OeuvresdeDescartes1983} hingegen übernimmt diese Argumentation nicht in den \titel{Discours de la M{\'e}thode}, was dafür spricht, dass er ihr keine entscheidende Bedeutung oder wenig Überzeugungskraft
 beimisst. Und soweit ich sehe, spielt diese Überlegung auch in der aktuellen
 Diskussion in der Sozialen Erkenntnistheorie keine Rolle, ebenso wenig wie bei
 ihren Bezugsautoren \name[David]{Hume} und \name[Thomas]{Reid}. Deren Thema ist im Anschluss
 an \authorcite{Descartes:OeuvresdeDescartes1983}' Hauptargumentationsstrang die Frage nach der
 \emph{Gewissheit} testimonialer Erkenntnisse: Lassen sich alle Zweifel an der Wahrheit von
 überbrachten Informationen ausräumen? Und inwieweit ist dies nötig, um sie als
 \enquote{Wissen} bezeichnen zu können?
\end{nummerierung}


Es handelt sich bei \authorcite{Descartes:OeuvresdeDescartes1983}' Argumenten um drei Arten von
Vorbehalten gegenüber testimonialem Wissen, die voneinander weitgehend
unabhängig sind -- wobei die ersten beiden Argumente die Gemeinsamkeit haben,
dass sie die Möglichkeit der Fehlinformation zu erweisen suchen (wenngleich auf
unterschiedlichen Wegen), während das dritte Argument diese Möglichkeit gar
nicht voraussetzt. Alle drei Argumente intendieren zumindest \emph{prima facie}
die Zurückweisung \emph{jeglichen} testimonialen Wissens und nicht nur die
Zurückweisung eines eingrenzbaren Bereichs desselben. Sollten sie gültig sein,
so etablieren sie einen umfassenden testimonialen Skeptizismus.

Wie auch immer man diese Argumente nun bewerten mag, so ist
sich \authorcite{Descartes:OeuvresdeDescartes1983} doch der Folgen eines gänzlichen Verzichts auf
testimoniales Wissen bewusst, denn dieses kommt in jedem menschlichen Denken
notwendiger Weise in erheblichem Umfang vor: Schließlich haben wir alle als
Kinder unser Wissen von anderen übernehmen müssen, ohne auch nur die Chance zur
eigenen bewussten Kontrolle unseres Wissenserwerbs zu haben.
\footnote{\cite[Vgl.][VI: 13.1--12]{Descartes:OeuvresdeDescartes1983}:
\enquote{Et ainsi encore ie pensay que, pource que nous auons tous est{\'e} enfants auant que
  d'estre hommes, {\&} qu'il nous a fallu long temps estre gouuernez
  par nos appetis {\&} nos Precepteurs, qui estoient souuent
  contraires les vns aux autres, {\&} qui, ny les vns ny les autres, ne nous
  conseilloient peutestre pas tousiours le meilleur, il est presqu'impossible
  que nos jugements soient si purs, ny si solides qu'ils auroient est{\'e}, si
  nous auions eu l'vsage entier de nostre raison d{\`e}s le point de nostre
  naissance, {\&} que nous n'eussions iamais est{\'e} conduit que par elle.}}
Hier liegt das Hauptproblem des erkenntnistheoretischen Individualismus: Das
Vertrauen in das Wissen anderer steht zumindest hinsichtlich der zeitlichen
Entwicklung am \emph{Anfang} unseres Erkennens. Es ist keine Art des
Wissenserwerbs, die wir erst spät auf der Grundlage anderer Arten erwerben,
sondern gerade die erste Art, Wissen zu erwerben. \authorcite{Descartes:OeuvresdeDescartes1983} denkt wohl
noch nicht daran, dass bereits mit dem Erwerb einer Sprache Wissen über die Welt
erworben wird.\footnote{In diesem Sinne zeigt Peter
F.~\textcite{Strawson:KnowingfromWords1994}, dass unsere epistemische Praxis mit
Wissen aus zweiter Hand anfangen muss und ohne dieses gar nicht möglich wäre. Es
handelt sich dabei in der Regel nicht um die explizite Weitergabe von Wissen
durch Mitteilung, sondern um die implizite Weitergabe in der Vermittlung von
\enquote{cognitive-practical ways of proceeding into which we were initiated
when we learned our language} (\cite[][415]{McDowell:KnowledgebyHearsay1994}).}
Aber er erkennt doch, dass unser erstes Wissen oder zumindest der größte Teil
unseres anfänglich erworbenen Wissens nicht durch \emph{eigenes Nachdenken} und
eigene \emph{bewusste} Wahrnehmung (oder auch nur \emph{bewusstes} Erfassen
mitgeteilter Informationen) zustande kommt, weil unsere Fähigkeit zum bewussten
und kritischen Erkenntnisgewinn erst ausgebildet werden muss. Deshalb sei die Übernahme fremder Ansichten, wie sie notgedrungen am
Anfang eines jeden intellektuellen Lebensweges steht, zunächst ein unkritisch
hingenommener Prozess.\footnote{Dies entspricht freilich nur der Darstellung in
den \titel{Regulae ad directionem ingenii} und dem \titel{Discours de la
M{\'e}thode}; in den \titel{Principia philosophiae} argumentiert
\authorcite{Descartes:OeuvresdeDescartes1983} über das Zeugnis der
\emph{Sinne}, dem wir anfangs unkritisch folgten: \enquote{Quoniam infantes nati sumus, {\&}
varia de rebus sensibilibus judicia pri{\`u}s tulimus, qu{\`a}m integrum
nostr{\ae} rationis usum haberemus, multis praejudiciis {\`a} veri cognitione
avertimur} \mkbibparens{\cite[][VIII:
5.5--8]{Descartes:OeuvresdeDescartes1983}}. In
den \titel{Meditationes} lässt er zunächst offen, woher seine aus frühester
Jugend hingenommenen Vorurteile stammen \mkbibparens{vgl.
\cite[][VII: 17.2--10]{Descartes:OeuvresdeDescartes1983}}. Im weiteren Verlauf
argumentiert er dann aber lediglich gegen die Verlässlichkeit von Wahrnehmung
und Vernunftgebrauch, die Verlässlichkeit testimonialer Erkenntnis wird gar
nicht erst eigens thematisiert.} Wenn
\authorcite{Thomasius:ChristianThomasiuseroeffnetDerStudirendenJugendzuLeipzigineinemDiscoursWelcherGestaltmandenenFrantzoseningemeinemLebenundWandelnachahmensolle?1994}
seine Vorurteilstheorie mit der Feststellung beginnt, dass der Mensch als Kind
erst vieles lernen müsse, was er notgedrungen ungeprüft zu übernehmen und damit
als Vorurteil anzunehmen habe, dann schließt er damit genau an
\authorcite{Descartes:OeuvresdeDescartes1983}' Vorstellungen
an.\footnote{\cite[Vgl.][104]{Schneiders:AufklaerungundVorurteilskritik1983}.}


Der erkenntnistheoretische Individualismus taugt also nicht als Beschreibung
unserer normalen intellektuellen Entwicklung. Den Erwerb unserer ersten
Erkenntnisse können wir gerade nicht bewusst kontrollieren. Der Ausweg wiederum
kann nach \authorcite{Descartes:OeuvresdeDescartes1983} nur darin bestehen, später im Leben einmal alles
Gelernte als falsch anzusehen und sich zu fragen, welche Erkenntnisse sich dem
isolierten Denker als unbezweifelbar wahr
erweisen.\footnote{\enquote{\dots\unkern quibus [pr{\ae}judiciis; A.\,G.] non
aliter videmus posse liberari, qu{\`a}m si semel in vita de iis omnibus
studeamus dubitare, in quibus vel minimam incertitudinis suspicionem reperiemus}
\mkbibparens{\cite[][VIII: 5.8--11]{Descartes:OeuvresdeDescartes1983}}.
\enquote{Animadverti jam ante aliquot annos qu{\`a}m multa, ineunte aetate, falsa pro veris admiserim, {\&}
qu{\`a}m dubia sint quaecunque istis postea superextruxi, ac proinde funditus
omnia semel in vit{\^a} esse evertenda, atque a primis fundamentis denuo
inchoandum, si quid aliquando firmum {\&} mansurum cupiam in scientiis
stabilire} \mkbibparens{\cite[][VII: 17.2--8]{Descartes:OeuvresdeDescartes1983}}.} Da
unser Wissen mindestens im Normalfall auf testimonialem Wissen
aufbaut, kann ein Verzicht auf dieses Wissen nur unter der Bedingung realisiert
werden, den eigenen Wissensbestand grundlegend neu aufzubauen und dazu erst
einmal \emph{alles} vorhandene Wissen zu verwerfen. \authorcite{Descartes:OeuvresdeDescartes1983} betrachtet
die natürliche Reihenfolge, der zufolge testimoniales Wissen am Anfang steht und die Grundlage
unseres Wissensbestandes bildet, als kontingent: Es sei
möglich, das eigene Denken als erste Quelle des Wissens zu nutzen und darauf
erst die anderen Arten des Wissenserwerbs -- eigene sinnliche Erfahrung und Information
durch andere -- aufzubauen. Freilich gelingt dies nur demjenigen, der einige
Übung im Gebrauch der eigenen Vernunft hat und zudem die Motive des
Individualismus einsieht. Als Option für Kinder, ihre geistige Entwicklung
sogleich auf individualistischer Grundlage zu beginnen, taugt es nicht.
Die Möglichkeit, durch Mitteilung anderer informiert zu werden, wäre dann erst
vor dem Hintergrund dieses kartesischen Programms neu zu begründen.


\Revision[Pelletier]{Freilich sieht
\authorcite{Descartes:OeuvresdeDescartes1983}, welch gravierende Folgen eine solche Forderung zunächst hat, nur noch diejenigen
Erkenntnisse als hinreichend fundiert anzusehen, die ein einzelner von Grund auf
selbst kontrolliert hat, ohne sich dabei auf genuin testimoniales Wissen zu
stützen. Es erscheint ihm zunächst sogar eher als Anmaßung, ein solches Projekt
auch nur für sich selbst in Angriff zu nehmen.\footnote{\cite[Vgl.][VI:
13.13--16.29]{Descartes:OeuvresdeDescartes1983}.} Er hält es aber nicht für
prinzipiell unmöglich, sondern nur die Umsetzung in großem Rahmen -- als
gesamtgesellschaftliches Projekt oder als Neuausrichtung der Wissenschaft -- für
problematisch oder zumindest
unrealistisch.\footnote{\authorcite{Descartes:OeuvresdeDescartes1983} gibt sich
überzeugt, \enquote{qu'il n'y auroit veritablement point d'apparence qu'vn
particulier fist dessein de reformer un Estat, en y changeant tout d{\'e}s les
fondemens, {\&} en le renuersant pour le redresser; ny mesme aussy de refomer le
corps des sciences, ou l'ordre establi dans les escoles pour les enseigner}
(\cite[][VI: 13.21--26]{Descartes:OeuvresdeDescartes1983}).} Er selbst könne für
sich allein den Gesamtbestand seines Wissens jedoch auf explizit
individualistischer Grundlage rekonstruieren.\footnote{\cite[Vgl.][VI:
13.27--14.1]{Descartes:OeuvresdeDescartes1983}.} Es gelingt
dies, insofern er letztlich zu zeigen vermag, dass unsere herkömmlichen
Erkenntniswege -- die eigene Erfahrung und die Informationen, die wir von
anderen erlangen -- durchaus Wissen
konstituieren.\footnote{\Revision[Pelletier]{Allerdings benötigen wir dafür auch
das Wissen um die rationale Fundierung der Gewissheit solcher Erkenntnisweisen. Deshalb kann nach
\authorcite{Descartes:OeuvresdeDescartes1983} der Atheist kein Wissen erlangen;
schließlich steht ihm der Gottesbeweis nicht zur Verfügung, mittels dessen die
Zuverlässigkeit unserer Erkenntnisquellen erst demonstriert wird
\parencite[vgl.][VII: 141.3--13]{Descartes:OeuvresdeDescartes1983}.}} Inwieweit
ihm dies tatsächlich gelingt, kann hier außen vor bleiben. Der Weg, den er wählt
(der Rekurs auf einen Gottesbeweis), ist für Philosophen wie
\name[Immanuel]{Kant} jedenfalls nicht gangbar.}

Was wir durch den Gebrauch unserer Vernunft \enquote{klar und deutlich}
einsehen, an dessen Wahrheit besteht kein Zweifel -- so
jedenfalls meint \authorcite{Descartes:OeuvresdeDescartes1983} in der dritten
Meditation auf der Grundlage der zuvor bewiesenen Existenz Gottes nachweisen zu
können.\footnote{Nach \authorcite{Descartes:OeuvresdeDescartes1983} sind nur
Intuition und Deduktion zulässige Mittel wissenschaftlichen Erkenntnisgewinns.
Dabei versteht er unter Deduktion dasjenige, was aus dem intuitiv erkannten mit
Notwendigkeit erschlossen wird \mkbibparens{\cite[vgl.][X:
369.18--22]{Descartes:OeuvresdeDescartes1983}}.
Zur Wissenschaft zählt also nichts weiter als die Folgerungshülle intuitiver
Erkenntnis, weswegen der Begriff der Intuition das Fundament darstellt, worunter
er \emph{expressis verbis} nicht wie
\authorcite{Wolff:Discursuspraeliminarisdephilosophiaingenere1996} und andere
die Erfahrungserkenntnis (siehe dazu Kap.
\ref{subsection:VerstandundRezeptivitaet} dieser Arbeit), sondern ausschließlich
ein unmittelbares Begreifen des Geistes:
\enquote{Per \ori{intuitum} intelligo, non fluctuantem sensuum fidem, vel mal{\`e} componentis imaginationis judicium fallax; sed
  mentis pur{\ae} {\&} attent{\ae} tam facilem distinctumque conceptum, vt de
  eo, quod intelligimus, nulla prorsus dubitatio relinquatur; seu, quod idem est, mentis
  pur{\ae} {\&} attent{\ae} non dubium conceptum, qui {\`a} sol{\^a} rationis
  luce nascitur} \mkbibparens{\cite[][X:
  368.13--19]{Descartes:OeuvresdeDescartes1983}}.} \Revision[Pelletier]{Wäre
  dies die einzig legitime Basis unseres Wissens, so ergäbe sich ein
  Verständnis von Selbstdenken, das sich durch seine Radikalität auszeichnete,
  aber auch selbst \emph{ad absurdum} führte: Es handelte sich um eine radikale
  Form des Individualismus, die nichts gelten lassen will als das, was jeder
  einzelne aus seinem eigenen Denken heraus als wahr erkennen kann. Dies scheint
  zunächst jede Form rezeptiv erlangten Wissens und \emph{a fortiori} jede
  Form testimonialem Wissen \emph{per se} zurückzuweisen. Die Legitimität
testimonialen Wissens müsste erst über individualistische Wissensquellen
umständlich rekonstruiert
werden.\footnote{\cite[Vgl.][431]{Stevenson:WhyBelieveWhatPeopleSay?1993}:
\enquote{On \authorcite{Descartes:OeuvresdeDescartes1983}'s new individualistic
approach, testimony can have evidential force only in a very secondary way, if
at all.}}}

\subsection{David Humes testimonialer Reduktionismus}
\label{subsubsection:DavidHume}
Das Problem des radikalen Selbstdenkens liegt darin, dass sich aus reiner
Vernunft das Corpus unseres Wissens nicht rekonstruieren lässt. Niemand ist so
vermessen zu behaupten, wir könnten aus reiner Vernunft
erkennen, dass Rom im Jahre 753 vor unserer Zeitrechnung gegründet wurde, dass Bananen
beim Reifen gelb werden oder dass es draußen in diesem Moment regnet. Die
radikale Konzeption des Selbstdenkens hätte daher eine absurde Revision des
Wissensbegriffs zu Konsequenz. Um den Umfang dessen, was wir gemeinhin und
vernünftiger Weise als unser Wissen bezeichnen, einzufangen, müssen wir also
weitere Wissensquellen neben der Vernunft zulassen. Die nächste als legitim und
zuverlässig anerkannte Wissensquelle ist die je eigene (sinnliche) Erfahrung.
Daraufhin stellt sich dann die Frage, ob wir unseren Wissensbestand bereits mit
diesen zwei Quellen -- Vernunft und Erfahrung -- rekonstruieren können, oder ob auch testimonialem
Wissen eine eigene Wissensquelle zuzusprechen ist.

In der neueren Erkenntnistheorie nimmt die Diskussion testimonialen Wissens
entsprechend dieser Überlegung oft die Frage zu ihrem Ausgangspunkt, ob testimoniales
Wissen als \emph{secondary epistemic link} auf nicht-testimoniales Wissen (also
Wissen aus Vernunft und aus eigener Erfahrung) reduzierbar und die Annahme einer eigenständigen Wissensquelle somit
unnötig ist. Die Position des \enquote{testimonialen Reduktionismus} wird dabei
häufig David \name[David]{Hume} attestiert.\footnote{\cite[Siehe
z.\,B.][532]{Grundmann:AnalytischeEinfuehrungindieErkenntnistheorie2008}, sowie
die Ausführungen von Axel
\textcite[vgl.][]{Gelfert:HumeonTestimonyRevisited2010}, der sich selbst
kritisch zu diesem \enquote{received view} verhält: \enquote{In contemporary
discussions of the epistemic status of testimony-based beliefs, David Hume is
typically cast in the role of \enquote{global reductionist}, who demands that
each of us must have first-hand, non-testimonial evidence of the reliability of
(relevant reference classes of) testimony, before accepting any new instance of
it} \parencite[][60]{Gelfert:HumeonTestimonyRevisited2010}.} Dieser glaube, dass
testimoniales Wissen sich aus je eigenem Wahrnehmungswissen und weiteren Prämissen wie der Verlässlichkeit entsprechender
Zeugnisse, welche wir wiederum aus eigenen Erfahrungen erworben haben,
ergibt.\footnote{\cite[Vgl.][46,
48]{Wilholt:SozialeErkenntnistheorie2007}, \cite[][144]{Audi:Epistemology1998}.
Dass dies eine gute Charakterisierung \name[David]{Hume}s ist, bezweifelt
\textcite{Gelfert:HumeonTestimonyRevisited2010}.} Es ist aber bereits fraglich, ob \name[David]{Hume}
überhaupt eine Theorie testimonialen Wissens
vertritt.\footnote{\cite[Vgl.][410]{Welbourne:IsHumeReallyaReductivist2002}:
\enquote{\name[David]{Hume} has no \ori{theory} of testimony, properly so-called; hence
not even a bad theory.}
\cite[Ebenso][303]{Faulkner:DavidHumesReductionistEpistemologyofTestimony1998}:
\enquote{\name[David]{Hume} does \ori{not} give a theory of testimony.}} Wenn
überhaupt, dann muss der Leser sie erst aus seinen Ausführungen des zehnten
Abschnitts des \titel{Enquiry Concerning Human Understanding} heraus
rekonstruieren.
Nach Michael \authorfullcite{Root:HumeontheVirtuesofTestimony2001} vertritt \name[David]{Hume} nur im
\titel{Enquiry} einen testimonialen Reduktionismus, nicht aber im
\titel{Treatise}.\footnote{\cite[Vgl.][]{Root:HumeontheVirtuesofTestimony2001}.}
\authorfullcite{Gelfert:HumeonTestimonyRevisited2010} behauptet, dass \name[David]{Hume} eher einem
\emph{lokalen} Reduktionismus nahe komme.\footnote{\cite[Vgl.][73]{Gelfert:HumeonTestimonyRevisited2010}.}
Während der \singlequote{globale testimoniale Reduktionismus} behauptet, wir
könnten unser gesamtes Wissen auf der ausschließlichen Grundlage
nicht-testimonialen Wissens rekonstruieren, fordert der \singlequote{lokale
testimoniale Reduktionismus} für \emph{einzelne} Fälle von Mitteilungen, die
Glaubwürdigkeit der Information auch unter Rückgriff auf bereits vorhandenes
Wissen zu belegen, wobei dieses Wissen auch auf Mitteilungen beruhen
darf.\footnote{\cite[Vgl. hierzu][]{Fricker:AgainstGullibility1994}.} Ein
lokaler Reduktionismus akzeptiert, dass unser Wissenserwerb auf irreduzibel testimonialem Wissen
beruht, fordert aber, dass wir -- möglicherweise ab einem gewissen Alter,
welches wir dann als Eintritt in die Mündigkeit beschreiben könnten -- neues
testimoniales Wissen nur erwerben können, indem wir fundierte Erkenntnisse (etwa über die Glaubwürdigkeit von Mitteilungen über bestimmte Themen) als Prämissen zugrunde legen. Schließlich
sei es auch nach \name[David]{Hume} nicht notwendig, testimoniales Wissen auf Wissen
erster Hand zu reduzieren, um unser Vertrauen in Mitteilungen zu
regulieren.\footnote{\cite[Vgl.][73]{Gelfert:HumeonTestimonyRevisited2010}:
\enquote{Testimony, thus, needs to be regulated by the same kinds of maxims as
experience in general. But regulating our reliance on testimony in ways that
accord with reason and experience, does not require reducing testimonial
knowledge to first-hand experience.}} Auch \authorfullcite{Root:HumeontheVirtuesofTestimony2001}
bestreitet, dass \name[David]{Hume} einem globalen Reduktionismus verpflichtet sei. Denn sowohl bei der
Einschätzung der Glaubwürdigkeit eines Informanten, als auch bei der
Einschätzung der \mbox{(Un-)} Wahrscheinlichkeit des Behaupteten lasse sich auch
nach \name[David]{Hume} auf testimoniales Wissen
zurückgreifen.\footnote{\cite[Vgl.][30]{Root:HumeontheVirtuesofTestimony2001}:
\enquote{Hume’s account of testimony is individualistic in one sense. B must
weigh, for himself, A’s credibility against the implausibility that p before
believing that p based on A’s testimony. But Hume does not maintain that B must
rely entirely on his own evidence in estimating A’s credibility or p’s
implausibility. On Hume’s account, B’s backward-looking evidence that A is a
credible witness can include the testimony of other witnesses and his belief
about the prior probability that p can be based on the testimony of other
witnesses as well.}} \authorfullcite{Faulkner:DavidHumesReductionistEpistemologyofTestimony1998} behauptet, dass sich
\name[David]{Hume} bei eingehender Betrachtung als der \emph{einzige} wirkliche Vertreter eines
testimonialen Reduktionismus erweise. Aber auch er bestreitet die Korrektheit
der Standardinterpretation, schließlich sei die Erfahrungsbasis, die \name[David]{Hume}
für testimoniales Wissen in Anschlag bringt, nicht so beschaffen, dass wir aus
verschiedenen Erfahrungen auf die Glaubwürdigkeit von Informationen und
Informanten schließen müssten. Vielmehr nähmen wir die Glaubwürdigkeit direkt
wahr -- unter anderem in unserer
Selbstwahrnehmung.\footnote{\cite[Vgl.][305]{Faulkner:DavidHumesReductionistEpistemologyofTestimony1998}:
\enquote{Hume does \ori{not} state that we \ori{infer} the credibility of
testimony from our past observations of the conjunctions between testimony and
the testified facts, \ori{but} that we judge there will be such a conjunction
because \ori{we observe the veracity of testimony}. That is, we can make a
\ori{direct}, rather than inferential, judgement of the credibility of
testimony.}} Die Erfahrungsbasis, die \name[David]{Hume} beschreibt, besteht
nicht ausschließlich in Paaren von Berichtswahrnehmungen und zugehörigen
Tatsachenwahrnehmungen, sondern in anthropologischen Tatsachen, die die
Wahrhaftigkeit fördern.
Jeder weiß -- auch von sich selbst --, dass Menschen zur Wahrheit tendieren und
mit (einem Gefühl der) Scham auf aufgedeckte eigene Lügen
reagieren.\footnote{\enquote{Were not the memory tenacious to a certain degree;
had not men
commonly an inclination to truth and a principle of probity; were they not
sensible to shame, when detected in a falsehood: Were not these, I say,
discovered by \ori{experience} to be qualities, inherent in human nature, we
should never repose the least confidence in human testimony}
\parencite[][90]{Hume:AnEnquiryConcerningHumanUnderstanding1964}.} Es sind
solche Überlegungen, die die Vermutung nähren, die Rechtfertigungsgrundlage liege
nach \name[David]{Hume} nicht -- oder wenigstens nicht nur -- in der
bisherigen Erfahrung mit fremden
Informationen.\footnote{Eine weitere Deutungsmöglichkeit liefert
\authorfullcite{Root:HumeontheVirtuesofTestimony2001}, der die Rolle von
Konventionen und gesellschaftlich geforderten Tugenden verweist:
\enquote{First, B can rely on his past observations of witnesses (a
backward-looking reason). He can estimate A’s credibility based on his past
experience that witnesses like A speak the truth. Second, B can rely on
conventions (a forward-looking reason).
He can estimate A’s credibility based on the fact that A and B are members of
an epistemic community in which giving credible testimony or being a credible
witness is an artificial virtue, a conventional solution to a problem of human
coordination, viz., one member revising his beliefs to match the reasonable
beliefs of others} \parencite[][19]{Root:HumeontheVirtuesofTestimony2001}.}

Belege für die Interpretation \name[David]{Hume}s als eines testimonialen
Reduktionisten finden sich im zehnten Abschnitt des \titel{Enquiry Concerning Human Understanding}.
Dieser Abschnitt \titel{Of Miracles} ist nicht einfach als eigenständiger Essay
innerhalb des \titel{Enquiry} zu betrachten, der einen religionskritischen
Nebenschauplatz eröffnet. In erster Linie untersucht er eine zentrale
Fragestellung des \titel{Enquiry}, die sich direkt aus dem vierten Abschnitt
\titel{Sceptical Doubts concerning the Operations of the Understanding} ergibt.
Es sei von Interesse herauszufinden, was die Natur jener Evidenz ist, durch die
wir auch dort, wo das gegenwärtige Zeugnis unsere Sinne und unsere Erinnerung
nicht zulangt, erkennen, was es gibt und was der Fall ist.\footnote{\enquote{It
may, therefore, be a subject worthy of curiosity, to enquire what is the nature
of that evidence, which assures us of any real existence and matter of fact,
beyond the present testimony of our senses, or the records of our memory}
\parencite[][23]{Hume:AnEnquiryConcerningHumanUnderstanding1964}.}


Für weitgehend unproblematisch hält \name[David]{Hume} unseren Wissenserwerb im Bereich
der \emph{relations of ideas} und derjenigen Tatsachen (\emph{matters of
fact}), von denen wir durch eigene sinnliche Wahrnehmung oder durch die
Erinnerung an frühere eigene sinnliche Wahrnehmung wissen. Der
größte Teil unseres Wissens beruht jedoch nicht auf diesen wenigen Quellen:
Sonst wüssten wir tatsächlich neben mathematischen Wahrheiten bloß darüber
Bescheid, was uns direkt umgibt oder umgab. Ließe unser Wissensbegriff nur als Wissen
zu, was auf eigener Überlegung oder eigener Wahrnehmung beruht, so hätte dies
wiederum absurde Konsequenzen: Nur der Mörder selbst oder ein direkter
Augenzeuge kennte die Wahrheit über seine Tat; ein polizeilicher Ermittler oder
ein Richter könnte niemals wissen, was geschah, insofern er nicht zufällig
selbst Augenzeuge (oder der Mörder) ist.

\name[David]{Hume} legt sich bereits im vierten Abschnitt des \titel{Enquiry
Concerning Human Understanding} fest, dass es sich um ein Kausalitätsverhältnis
handeln muss, auf dessen Grundlage wir auf das Bestehen von Sachverhalten
schließen, die wir nicht selbst wahrnehmen oder wahrgenommen
haben,\footnote{\enquote{All reasonings concerning matter of fact seem to be
founded on the relation of \ori{Cause and Effect}. By means of that relation
alone we can go beyond the evidence of our memory and senses}
\parencite[][24]{Hume:AnEnquiryConcerningHumanUnderstanding1964}.} also auch im
Falle solcher Sachverhalte, die uns berichtet werden. Wenn wir keine direkte
Kenntnis einer Tatsache haben, sondern nur indirekte Kenntnis, so ist unser
Wissen Ergebnis einer Schlussfolgerung (\enquote{reasoning}). Als Beispiel einer
Tatsache, von der zu wissen das Ergebnis einer Schlussfolgerung ist, führt
\name[David]{Hume} den Fall an, dass jemand von dem Aufenthalt eines Freundes in
Frankreich weiß, weil er einen Brief von ihm erhalten hat, in dem er ihm dies
mitteilt. Dies könnte ein klarer Fall testimonialen Wissens sein (es wäre immer
noch ein Fall testimonialen Wissens -- wenngleich kein so
klarer Fall --, wenn als Grundlage des Wissens nicht eine Mitteilung im Wortlaut
des Briefes dient, sondern etwa der Poststempel); jedenfalls ist testimoniales Wissen ein Paradebeispiel für das
Wissen um Tatsachen, die unseren Sinnen weder zugänglich sind noch zugänglich
waren und auf die wir nach \name[David]{Hume}s Darstellung im vierten Abschnitt
des \titel{Enquiry} daher auf dem Wege einer Schlussfolgerung gelangen. Dazu
müssen wir -- so behauptet \name[David]{Hume} -- die Kette von
Wissensweitergaben rekonstruieren, die von dem berichteten Sachverhalt über den
oder die ersten Augenzeugen und möglicherweise einen, mehrere oder viele Mittler
(die deutschsprachige Aufklärungsphilosophie spricht von
\enquote{Ohrenzeugen}\footnote{Siehe
Anm.~\ref{Anmerkung:AugenzeugenundOhrenzeugen} auf
S.~\pageref{Anmerkung:AugenzeugenundOhrenzeugen}.}) bis zu uns führt.


\name[David]{Hume} scheint sich Geschichtsschreibung folgendermaßen
vorzustellen: Wir können Wissen erwerben, indem wir uns auf das Wissen eines
Informanten stützen, insofern wir Grund zu der Annahme haben, dass unser
Informant auch tatsächlich entsprechendes Wissen besitzt. Erwarb er dieses
Wissen seinerseits wieder von einem anderen Informanten, benötigen wir wiederum
die Annahme, dass dieser Informant wirkliches Wissen besitzt, entweder von einem
weiteren Informanten oder aus der eigenen Wahrnehmung. An irgendeiner Stelle
muss unsere Kette von Informanten bei jemandem ankommen, der das Wissen nicht
von jemand anderem hat, sondern aus eigener Anschauung.\footnote{\enquote{We learn the events of former ages from history; but then we must peruse the
  volumes, in which this instruction is contained, and thence carry up our
  inferences from one testimony to another, till we arrive at the eye-witnesses
  and spectators of these distant
  events} \parencite[][39]{Hume:AnEnquiryConcerningHumanUnderstanding1964}.} Nun sei für jeden
solchen Übergang eine eigene Schlussfolgerung nötig (was im Falle von Berichten
über Ereignisse, die lange vergangen sind, mitunter auch recht lange
Schlussfolgerungsketten ergäbe). Wenn $A$ beispielsweise als Anfang der
Informationskette wahrnimmt, dass $p$, dann ist $A$ hinreichend gerechtfertigt
in der Überzeugung, dass $p$, und \emph{weiß} daher, dass $p$. Wenn $A$ nun $B$
dieses Wissen mitteilt, dann kann auch $B$ in ihrer gewonnenen Überzeugung
hinreichend gerechtfertigt sein und somit \emph{wissen}, dass $p$. Aber dazu
müsse $B$ eben darauf \emph{schließen}, dass $p$. Als Prämissen für diesen
Schluss stehe ihr zum einen die eigene Wahrnehmung zur Verfügung, dass $A$ sagt,
dass $p$; jeder Schluss auf entfernte Tatsachen beginne bei einer direkten
Wahrnehmung oder
Erinnerung\footnote{\cite[Vgl.][39]{Hume:AnEnquiryConcerningHumanUnderstanding1964}:
\enquote{[O]ur conclusion from experience carry us beyond our memory and senses,
and assure us of matters of fact, which happened in the most distant places and
most remote ages; yet some fact must always be present to the senses or memory,
from which we may first proceed in drawing these conclusions.}}.
$B$ benötigt aber weiter einen Obersatz, demzufolge sie von $A$s Behaupten, dass
$p$, auf die Wahrheit von $p$ schließen darf. Und dieser Obersatz beschreibt
nach \name[David]{Hume} einen allgemeinen Ursache-Wirkungs-Zusammenhang, der
von einem berichteten Sachverhalt hin zur Aussage unseres Informanten reichen
muss. Und wenn es sich um eine längere Informationskette handelt, benötigen wir
eine entsprechend lange Kette von Obersätzen, die zusammen einen komplexen
Kausalzusammenhang von dem Ereignis über sämtliche Informanten bis zu uns
beschreiben.


Wie können Obersätze von Schlüssen, die den Übergang von der Erkenntnis, dass
jemand sagt, etwas sei der Fall, zu der Erkenntnis, eben dies sei der Fall,
begründet sein? Nach \name[David]{Hume} haben wir Zutrauen in menschliches
Zeugnis, weil wir aus Erfahrung von der Aufrichtigkeit menschlicher Berichte
wissen. Er behauptet, unsere Gewissheit entstamme lediglich unserer Beobachtung
der Wahrhaftigkeit menschlichen Zeugnisses und der üblichen Übereinstimmung der
Tatsachen mit den Berichten von Zeugen.\footnote{\enquote{It will be sufficient
to observe, that our assurance in any argument of this kind is derived from no
other principle than our observation of the veracity of human testimony, and of
the usual conformity of facts to the reports of witnesses}
\parencite[][90]{Hume:AnEnquiryConcerningHumanUnderstanding1964}.} Zwischen
einer Äußerung und einem Sachverhalt, den die Äußerung beschreibt, besteht ein
kausaler, regelmäßig wahrnehmbarer Zusammenhang. Es folgt also auf die
Beobachtung eines Sachverhalts regelmäßig die Beobachtung der Äußerung oder
umgekehrt auf die Beobachtung der Äußerung die Beobachtung des Sachverhalts. Wir
stellen auf diese Weise eine kausale Beziehung zwischen beiden fest, die so
geartet ist, dass die Äußerung kausale Folge (Wirkung) des Sachverhalts
(Ursache) ist. Die Äußerung ist dadurch ein Zeichen für einen Sachverhalt
ähnlich wie Krankheitssymptome Zeichen einer bestimmten Infektion sind, als
deren Wirkung sie auftreten.\footnote{Zu dieser Standardlesart und ihrer
Verbreitung in der jüngeren Literatur der Sozialen Erkenntnistheorie vergleiche
man die Übersicht von Axel
\textcite[vgl.][61--63]{Gelfert:HumeonTestimonyRevisited2010}.}


Dabei bleibt zunächst offen, ob unsere Erfahrungsbasis den jeweiligen
Informanten selbst betrifft oder eine Gruppe von
Informanten.\footnote{\cite[Vgl.][20]{Root:HumeontheVirtuesofTestimony2001}:
\enquote{$B$’s estimate of $A$’s credibility can be based on either (3) or (4):
(3) $B$ observes that $A$’s testimony has been conjoined with the truth and
infers that her testimony usually conforms to the truth; (4) $B$ observes that
the testimony of witnesses like $A$ has been conjoined with the truth and infers
that her testimony usually conformes to the truth.}} Während aber die erste
Option uns in vielen Fällen in die Situation bringt, gar keine
Bewertungsgrundlage zu haben, weil uns die jeweiligen Informanten zum ersten Mal
zur Verfügung stehen, ermöglicht uns die zweite Option, die
Zuverlässigkeit von Informanten zu bewerten, mit denen wir bisher noch keinerlei
Erfahrung gemacht
haben.\footnote{\cite[Vgl.][20]{Root:HumeontheVirtuesofTestimony2001}:
\enquote{Because witnesses can be alike, can form a natural kind, B may be able
to predict that A will give honest testimony, even if B has never observed A’s
testimony.}} Man denke hier etwa an die Einheimische, die wir als Touristen in
einer fremden Gegend nach dem Weg fragen. Wir haben in aller Regel noch keine
Erfahrungen mit ihr selbst als Informantin machen können, wohl aber mit der
Zuverlässigkeit von Einheimischen im allgemeinen. \name[David]{Hume} sagt allgemein
über Ursache-Wirkungs-Zusammenhänge, wir erwarteten, dass auf Ereignisse, die
einander \singlequote{ähnlich} sind, auch wiederum einander ähnliche Ereignisse
folgen.\footnote{\cite[Vgl.][63]{Hume:AnEnquiryConcerningHumanUnderstanding1964}:
\enquote{we may define a cause to be \ori{an object, followed by another, and
where all the objects, similar to the first, are followed by objects similar
to the second.}}} Worin jedoch die Ähnlichkeit besteht, bleibt gerade auch im
Falle von Informanten und Informationen offen. Es bleibt also die Schwierigkeit
bestehen, wie sich die relevanten Bezugsgruppen
zusammensetzen.\footnote{\cite[Vgl.
hierzu][83--85]{Coady:Testimony1992}. Nach Michael
\textcite[vgl.][20]{Root:HumeontheVirtuesofTestimony2001} muss \name[David]{Hume}
voraussetzen, dass es \singlequote{natürliche Arten} (\enquote{\emph{natural
kinds}}) von Informanten gibt. Es handelt sich freilich um einen Spezialfall des
allgemeinen Problems, wie die \singlequote{ähnlichen} Ereignisse in \name[David]{Hume}s
Definition des Kausalitätsbegriffs zu bestimmen sind.}

Der Schluss von einer Mitteilung auf eine bestehende Tatsache (\enquote{\emph{matter
of fact}}), die sich unserer unmittelbaren Wahrnehmung und unserer Erinnerung
verschließt, sei eine von mehreren Möglichkeiten des Schließens auf der
Grundlage eines
Ursache-Wirkungs-Zusammenhangs.\footnote{\cite[Vgl.][23--24]{Hume:AnEnquiryConcerningHumanUnderstanding1964}.}
Wir entdecken also einen konstanten Zusammenhang zwischen menschlichen
Behauptungen und dem Bestehen der behaupteten Sachverhalte, noch \emph{bevor}
wir damit beginnen, etwas auf die Behauptung anderer hin für wahr zu halten.
Erst \emph{nachdem} uns dieser Zusammenhang durch vielfältige Erfahrungen
bekannt werden konnte, beginnen wir, aus den Behauptungen anderer auf die
entsprechenden Sachverhalte zu schließen. Ob der Ausdruck \enquote{Kausalität}
den gesuchten Zusammenhang gut beschreibt, sei zwar fraglich (wobei \name[David]{Hume}
dies offensichtlich bejaht, aber bezweifelt, dass dies ohne weiteres
konsensfähig ist). Von größerer Bedeutung sei aber, dass die Evidenz einer
Überzeugung im Falle testimonialen Wissens auf einer Schlussfolgerung beruhe, die einer
Grundlage bedürfe. Und weil diese Grundlage nicht a priori sei, könne es sich
nur um die Erfahrung eines regelmäßigen Zusammenhangs
handeln.\footnote{\enquote{The reason, why we place any credit in witnesses and historians, is
not derived from any \ori{connexion}, which we perceive \ori{a priori},
between testimony and reality, but because we are accustomed to find a
conformity between them} \parencite[][\pno~91f.]{Hume:AnEnquiryConcerningHumanUnderstanding1964}.}

Wo keine solche Grundlage vorliegt und wir dennoch auf einen
Sachverhalt schließen, genau dort -- so scheint es zunächst -- liegt
Leichtgläubigkeit vor, denn ein weiser Mensch richtet sich in seinen
Überzeugungen nach den vorliegenden
Evidenzen.\footnote{\cite[Vgl.][89]{Hume:AnEnquiryConcerningHumanUnderstanding1964}:
\enquote{A wise man {\punkt} proportions his belief to the evidence.}} Doch
\name[David]{Hume}s Punkt ist ein anderer: Leichtgläubigkeit
(\enquote{\emph{credulity}}) sei die häufigste Schwäche der Menschen. Obwohl
Erfahrung mit Zeugnissen und der Wahrhaftigkeit die einzige Grundlage für
unser Vertrauen in die Mitteilungen von Geschehnissen sei, richteten wir uns
doch in der Bewertung von Mitteilungen nicht nach unseren Erfahrungen, sondern
neigten dazu, alles Mögliche zu glauben. Wir glaubten Berichte über
Geistererscheinungen, Zauberei und Wunder, so sehr ihnen auch die täglichen
Erfahrungen und Beobachtungen widersprechen.\footnote{\enquote{No weakness of
human nature is more universal and conspicuous than what we commonly call
credulity, or a too easy faith in the testimony of others; and this weakness is
also very naturally accounted for from the influence of resemblance. When we
receive any matter of fact upon human testimony, our faith arises from the very
same origin as our inferences from causes to effects, and from effects to
causes; nor is there any thing but our experience of the governing principles of
human nature, which can give us assurance of the veracity of men. But tho’
experience be the true standard of this, as well as of all other judgments, we
seldom regulate ourselves entirely by it; but have a remarkable propensity to
believe whatever is reported, even concerning apparitions, enchantments, and
prodigies, however contrary to daily experience and observation}
\parencite[][78]{Hume:ATreatiseofHumenNature2007}.}


Wichtig ist hier, dass \name[David]{Hume} sich gegen
Leichtgläubigkeit bei solchen Behauptungen wendet, die der Alltagserfahrung
widersprechen. Wenn \name[David]{Hume} die \singlequote{Leichtgläubigkeit}
(\enquote{credulity}) vieler Menschen geißelt, muss dies weder ein generelle
Ablehnung testimonialen Wissens nach sich ziehen, noch auch nur eine
Relativierung der Gewissheit testimonialen Wissens. Vielmehr attackiert er eine
übertriebene Bereitschaft, den Berichten anderer auch dort Glauben zu schenken,
wo jemand gute Gründe hat, dies nicht zu
tun.\footnote{\cite[Vgl.][420]{Welbourne:IsHumeReallyaReductivist2002}:
\enquote{\name[David]{Hume} (unlike \name[Thomas]{Reid}) always uses the word
\enquote{credulity} in a bad sense; for him, as for us, it is the name of an
intellectual vice, and does not refer, as it does in \name[Thomas]{Reid}, to the
general propensity to believe what one is told.}}


Gute Gründe, einer Mitteilung \emph{keinen} Glauben zu schenken, haben wir
\name[David]{Hume} zufolge insbesondere dort, wo jemand etwas sehr
Unwahrscheinliches, Wunderbares oder gar Absurdes
behauptet.\footnote{\cite[Vgl.][88--94]{Hume:AnEnquiryConcerningHumanUnderstanding1964}.}
Bei genauerem Hinsehen zeigt sich, dass es \name[David]{Hume} gerade bei seiner
allgemeinen Kritik an Wunderberichten nicht um die Evidenz \emph{für} eine
Überzeugung geht, die sich nach dem Informanten und dessen Eigenschaften
richtet, sondern um die entgegengesetzte Evidenz \emph{gegen} diese Überzeugung,
die mit dem Inhalt -- oder der \emph{Art} des Inhalts -- der Information zu tun
hat.\footnote{\cite[Vgl.][68]{Gelfert:HumeonTestimonyRevisited2010}:
\enquote{Hume rules out miraculous testimony on the basis of specific features
of the content of the testimony in question, rather than on the basis of
failures of the institution of testimony as such.}} Deswegen kann \name[David]{Hume}
seine Kritik an testimonialem Wissen auch auf eine bestimmte Klasse von Fällen
(die Wunderberichte) beschränken, während er gewöhnliche Fälle unangetastet
lässt.\footnote{\cite[Vgl.][74]{Gelfert:HumeonTestimonyRevisited2010}.} Erzählt
uns jemand, dass John F. Kennedy von den Toten auferstanden, Tante Ernas
unheilbare Krankheit geheilt oder Onkel Philipp über Wasser gelaufen sei,
spricht genug gegen die Glaubwürdigkeit des Berichts, um ihm unsere Zustimmung
zu verweigern. Dies gilt selbst dann, wenn die Glaubwürdigkeit der Autorität,
die uns berichtet, sonst über allen Zweifel erhaben
ist.\footnote{\cite[Vgl.][92]{Hume:AnEnquiryConcerningHumanUnderstanding1964}.}
Doch gerade in den kuriosesten Fällen zeige sich eine völlig irrationale
Bereitschaft, den Berichten zu vertrauen; und diese Bereitschaft resultiere aus
unseren Affekten der Überraschung (\emph{surprize}) und des Staunens
(\emph{wonder}).\footnote{\cite[Vgl.][95]{Hume:AnEnquiryConcerningHumanUnderstanding1964}:
\enquote{The passion of \ori{surprize} and \ori{wonder}, arising from miracles,
being an agreeable emotion, gives a sensible tendency towards the belief of
those events, from which it is derived.}} Sie sei im Bereich des Religiösen
besonders
auffällig.\footnote{\cite[Vgl.][95]{Hume:AnEnquiryConcerningHumanUnderstanding1964}:
\enquote{But if the spirit of religion join itself the love of wonder, there is
an end of common sense; and human testimony, in these circumstances, loses all
pretensions to authority.}} Religiöse Menschen glauben so abstruse Behauptungen
wie dass ein Toter wieder lebendig wurde oder dass ein Mensch über Wasser lief und Wasser
in Wein verwandelte. Leichtgläubig ist also im allgemeinen nicht derjenige, der
eine Mitteilung ohne Evidenz ihrer Glaubwürdigkeit zur Grundlage seiner
Überzeugungen macht -- denn für die Glaubwürdigkeit von Mitteilungen haben
wir genügend (nach \name[David]{Hume}: empirische) Evidenz --, sondern derjenige, der
die entgegenstehende Evidenz gegen die Wahrheit des Berichteten nicht würdigt.
Allerdings kann diese gegensätzliche Evidenz nur dadurch die Glaubwürdigkeit
einer Mitteilung unterminieren, weil sie von derselben Art ist. Deswegen ist die
Explikation der Grundlagen testimonialen Wissens so wichtig für
\authorcite{Hume:AnEnquiryConcerningHumanUnderstanding1964}. Es steht Erfahrung gegen
Erfahrung und das Ergebnis ist ein Zustand des Nichtwissens, weil sich die
Evidenzen auf beiden Seiten gegenseitig aufheben.


\name[David]{Hume}s Ziel im zehnten Abschnitt des \titel{Enquiry} besteht darin,
die Unmöglichkeit testimonialen Wissens von \emph{Wundern} nachzuweisen, wobei er
unter einem \enquote{Wunder} die Verletzung eines allgemeinen Naturgesetzes
(durch den Willen einer Gottheit oder ähnliches) versteht. Im Rahmen dieser
Kritik an Wunderberichten beansprucht er zu zeigen, dass unser Wissen von
Naturgesetzen, die zu verletzen das \emph{Definiens} des Wunderbegriffs
ausmacht\footnote{\cite[Vgl.][\pno~93,
Anm.:]{Hume:AnEnquiryConcerningHumanUnderstanding1964} \enquote{A miracle may be
accurately defined, \ori{a transgression of a law of nature by a particular
volition of the Deity, or by the interposition of some invisible agent.}} Im
Haupttext des zehnten Abschnitts hebt er ausschließlich auf die Verletzung eines
Naturgesetzes ab: \enquote{A miracle is a violation of the laws of nature}
(\cite[][93]{Hume:AnEnquiryConcerningHumanUnderstanding1964}).}, und
testimoniales Wissen dieselbe Grundlage und darum auch dieselben Grade an
Gewissheit haben, weswegen sie ihre jeweilige Gewissheit auch gegenseitig
aufheben, wenn sie sich widersprechen.


Widerspricht also ein menschliches Zeugnis einem allgemeinen Naturgesetz, dann
stehen sich zwei einander widersprechende Erkenntnisse gegenüber, die beide
die gleiche Grundlage haben: unsere Erfahrung. Da wir aber keinen Grund haben,
dem einen mehr als dem anderen zu trauen, und menschliches Zeugnis kein höheres
Ansehen genießt als das Zeugnis unserer Sinne, können wir nur schließen, dass
den Berichten über Wunder weniger Gewissheit zukommt als anderen Mitteilungen,
da unsere alltägliche Erfahrung gegen die Wahrheit des Berichts spricht. Im
Falle religiöser Schriften, wo uns nicht ein direkter Augenzeuge ein Wunder
berichtet, sondern eine lange Mitteilungskette vorliegt, dort komme es gar zu
gänzlichen Verlust an Gewissheit. Denn wenn $A$ sieht, dass $p$, und dies $B$
mitteilt, dann weiß möglicherweise auch $B$, dass $p$ (wenn sie hinreichende
Gründe hat, $A$s Auskunft zu vertrauen). Und wenn $B$ dieses Wissen wiederum an
$C$ weitergibt, erlangt auch $C$ dasselbe Wissen, dass auch $A$ und $B$ haben --
vorausgesetzt sie hat hinreichende Gründe anzunehmen, dass $B$ vertrauenswürdig
ist und selbst wiederum hinreichende Gründe hat zu glauben, dass $p$. Aber die
Zuverlässigkeit der Gründe oder die Evidenz (\emph{evidence}) von $p$ könne bei
jedem Übergang nur abnehmen, niemals größer werden. Dies werde ich das
\enquote{\label{DasHumeschePrinzip}\name[David]{Hume}sche Prinzip}
nennen.\footnote{\name[Immanuel]{Kant} spricht in diesem Fall von
\enquote{subordinirten Zeugen} und konstatiert ebenso eine abnehmende
Glaubwürdigkeit bei Zunahme der Länge der Mitteilungskette. Umgekehrt führe eine
größere Anzahl von Zeugen, die unabhängig voneinander, aber übereinstimmend
dasselbe berichten, zu größerer Glaubwürdigkeit: \enquote{In der reihe der
subordinirten Zeugen nimmt die historische Glaubwürdigkeit ab; in der Reihe der
coordinirten Zeugen nimmt sie zu. Die Reihe der coordinirten Zeugnisse heißt das
öffentliche Gerüchte, der subordinirten Zeugen hingegen eine mündliche
Ueberlieferung. Wenn in den coordinirten Zeugnisse der Augenzeuge unbekannt ist;
so heißts eine gemeine Sage}
\mkbibparens{\cite[][]{Kant:LogikPhilippi1966}, \cite[][XXIV:
450.23--28]{Kant:GesammelteWerke1900ff.}}} (Dieses Prinzip ist in seiner Allgemeinheit
falsch, wie später zu zeigen sein wird.\footnote{Siehe unten, Kapitel
\ref{AbschnittzuCrusiusundKritischemJournalismus} ab
S.~\pageref{AbschnittzuCrusiusundKritischemJournalismus}.}) $C$ hat also in
jedem Fall weniger gute Gründe zu glauben, dass $p$, als $B$. Hat $C$ nun
anderweitige Gründe für die Annahme, dass $\lnot p$, dann überragen diese Gründe
leicht die Gründe aus der Kette von Mitteilungen, die sie für die Überzeugung
hat, dass $p$. Handelt es sich bei $p$ um ein Wunder, dann haben wir aus unserer
Erfahrung direkte Evidenz dafür, dass $\lnot p$, die mindestens der Evidenz
entspricht, die wir für die Verlässlichkeit eines einzelnen Übergangs von einem
Mittelzeugen auf den nächsten haben.



In den meisten Fällen ist es eine völlig richtige Reaktion auf die Mitteilungen
anderer, dass wir unsere Überzeugungen nach ihren Äußerungen richten.
Schließlich gehört die Mitteilung anderer zu den wichtigsten, nützlichsten und
notwendigsten Arten, Wissen über Tatsachen zu
erlangen.\footnote{\cite[Vgl.][90]{Hume:AnEnquiryConcerningHumanUnderstanding1964}:
\enquote{[T]here is no species of reasoning more common, more useful, and even
more necessary to human life, than that which is derived from the testimony of
men, and the reports of eye-witnesses and spectators.}} \name[David]{Hume} vertritt
keine revisionistische Konzeption, die einen großen oder gar den größten Teil
dessen, was wir für testimoniales \emph{Wissen} halten, verwirft, sondern kritisiert
spezielle Fälle von Informationen. Je nachdem welche Art von Bericht vorliegt
und wer diesen Bericht abgibt, könne sogar von einem Beweis und nicht nur von
Wahrscheinlichkeit gesprochen werden.\footnote{\enquote{And as the evidence,
derived from witnesses and human testimony, is founded on past experience, so it
varies with the experience, and is regarded either as a \ori{proof} or a
\ori{probability}, according as the conjunction between any particular kind of
report and any kind of object has been found to be constant or variable}
\parencite[][91]{Hume:AnEnquiryConcerningHumanUnderstanding1964}.
\cite[Vgl.][]{Gelfert:HumeonTestimonyRevisited2010}.} Muss man
\name[David]{Hume} nun einen testimonialen Reduktionismus unterstellen, um sein
Argument gegen die Glaubwürdigkeit von Wunderberichten zu rekonstruieren?


Aber auch wenn nicht die Evidenz \emph{für} die Wahrheit einer Mitteilung im
Fokus der Kritik \name[David]{Hume}s steht, sondern die gegensätzliche Evidenz
\emph{gegen} ihre Glaubwürdigkeit, bleibt das Vertrauen auf die Mitteilung
anderer in seinem Bild doch stets rechtfertigungsbedürftig. Daher ist die
Bedeutung der Erfahrung auch für die Überzeugungskraft alltäglicher Mitteilungen
nicht von der Hand zu weisen. Denn solange wir keine Gründe haben, die
\emph{für} die Glaubwürdigkeit von bestimmten Mitteilungen oder Mitteilungen im
allgemeinen sprechen, haben wir auch keinerlei Gründe, etwas für wahr zu halten,
was uns erzählt wird. Und vernünftigerweise werden wir unsere Zustimmung
verweigern, wo wir keine Gründe haben. \name[David]{Hume} sagt, dass wir in
Alltagssituationen über solche Gründe tatsächlich in einem Maße verfügen, dass
wir in der Regel Wissen durch die Mitteilungen anderer akquirieren. Gewiss
neigen wir dazu, Informationen aus zweiter Hand viel zu leichtfertig zu
vertrauen, insofern wir die Gegengründe nicht hinreichend beachten. Aber
zunächst sei unser Vertrauen durch vorangegangene Erfahrungen zumindest
\emph{prima facie} berechtigt. Während \authorcite{Descartes:OeuvresdeDescartes1983} nur Gründe
aufzählt, die \emph{gegen} die Zuverlässigkeit von Informationen sprechen (und
auf das Thema testimoniales Wissen nicht wieder rekurriert), behauptet
\name[David]{Hume}, dass uns aus unserer Erfahrung viele Gründe \emph{für} die
Verlässlichkeit von Informationen vorlägen.

\section{Nicht-individualistische Ansätze}
\label{section:NichtindividualistischeAnsaetze}
\subsection{Thomas Reids Credulismus}
\label{subsubsection:ThomasReid}
Nach der Standardinterpretation \name[David]{Hume}s behauptet dieser, dass wir
erfahrungsabhängiges Wissen um die Zuverlässigkeit menschlichen Zeugnisses
benötigen, \emph{bevor} wir erstmalig durch Berichte anderer Wissen erwerben
können. Doch die Überlegung, dass wir erst Erfahrungen mit der Verlässlichkeit
menschlicher Zeugnisse machen müssen, um hinterher auf dieser Grundlage von
einer Aussage, die uns gegenüber gemacht wird, auf das Bestehen eines durch
diese Aussage beschriebenen Sachverhalts schließen zu können, erweist sich bei
näherer Betrachtung schlicht als absurd.\footnote{\cite[Vgl.][3]{Anscombe:WhatIsIttoBelieveSomeone2008}:
\enquote{The view needs only to be stated to be promptly rejected. It was always
absurd, and the mystery is how \name[David]{Hume} could ever have entertained it.}}
Thomas \name[Thomas]{Reid} weist darauf hin, dass wir nach diesem Modell als Kinder
zunächst misstrauisch sein müssten -- schließlich fehlt uns zunächst jede
Evidenz \emph{für} die Zuverlässigkeit testimonialen Wissens --,
um mit zunehmendem Alter immer unkritischer zu werden, nachdem wir immer mehr
positive Belege in unserer Erfahrung sammeln
konnten.\footnote{\cite[Vgl.][\pno~194\,f.:]{Reid:AnInquiryIntotheHumanMindonthePrinciplesofCommonSense1997}
\enquote{Children, on this supposition, would be absolutely incredulous; and
therefor absolutely incapable of instruction: those who had little knowledge of
human life, and of the manners and characters of men, would be in next degree
incredulous: and the most credulous men would be those of greatest experience,
and of the deepest penetration; because, in many cases, they would be able to
find good reasons for believing testimony, which the weak and the ignorant could
not discover.}} Im Gegenteil dazu sind offensichtlich gerade Kinder besonders
gutgläubig, während Menschen mit zunehmendem Alter eine kritischere Einstellung
zu testimonialem Wissen (zumindest in bestimmten Fällen) erwerben.



\name[Thomas]{Reid} artikuliert damit eine Einsicht, die später
\authorcite{Wittgenstein:UeberGewissheit1977} in \titel{Über Gewißheit} stark macht:
\enquote{Das Kind lernt, indem es dem Erwachsenen glaubt. Der Zweifel kommt \ori{nach} dem
Glauben.}\footnote{\cite[][\S~160]{Wittgenstein:UeberGewissheit1977}.} Erst spät
lernen wir dann zwischen glaubwürdigen und unglaubwürdigen Informanten zu
unterscheiden,\footnote{\cite[Vgl.][\S~143]{Wittgenstein:UeberGewissheit1977}:
\enquote{Ein Kind lernt viel später, daß es glaubwürdige und unglaubwürdige
Erzähler gibt, als es Fakten lernt, die ihm erzählt werden.}} auch wenn wir
dadurch zuvor vieles gelernt haben, was sich später möglicherweise als falsch
herausstellt\footnote{\cite[Vgl.][\S~161]{Wittgenstein:UeberGewissheit1977}:
\enquote{Ich habe eine Unmenge gelernt und es auf die Autorität von Menschen
angenommen, und dann manches durch eigene Erfahrung bestätigt oder entkräftet
gefunden.} Siehe auch \cite[][\S\S~159,
162, 170]{Wittgenstein:UeberGewissheit1977}. Die Ähnlichkeit der Darstellungen
\name[Thomas]{Reid}s und \name[Ludwig]{Wittgenstein}s erkennt
\textcite[][434]{Stevenson:WhyBelieveWhatPeopleSay?1993}.}.
Also sollte eher das \emph{Misstrauen} die Folge unserer Erfahrungen sein und
nicht das \emph{Vertrauen}, welches die natürliche Haltung des
Menschen am Beginn seines Lebens darstellt.

Thomas \name[Thomas]{Reid} erkennt also die Absurdität einer reduktionistischen
Position, wenn diese als Beschreibung unser intellektuellen Entwicklung
verstanden wird, wie sie tatsächlich vonstatten geht. Dabei
ist er vielleicht der erste Autor, der explizit den testimonialen Reduktionismus
verwirft und einen Credulismus, also eine nicht-individualistische Position
bezüglich testimonialem Wissen vertritt.\footnote{Nach
\authorfullcite{Grundmann:AnalytischeEinfuehrungindieErkenntnistheorie2008} ist \name[Thomas]{Reid} der erste Vertreter einer
\emph{externalistischen} Position und deshalb sei seine Theorie testimonialen Wissens derjenigen \name[David]{Hume}s
überlegen;
\cite[vgl.][537]{Grundmann:AnalytischeEinfuehrungindieErkenntnistheorie2008}.
Jedoch begründet
\authorcite{Grundmann:AnalytischeEinfuehrungindieErkenntnistheorie2008} diese
Einschätzung nicht weiter. Siehe zu dieser Einordnung auch
\cite{Woudenberg:ThomasReidbetweenExternalismandInternalism2013}.} Nach
\authorcite{Reid:EssaysontheIntellectualPowersofMan2002} gibt es zwei Arten
geistiger Tätigkeiten (\enquote{operations of mind}), von denen die einen von
einzelnen Subjekten ausgeführt und individualistisch verstanden werden können,
während die anderen eines Zusammenspiels mit anderen bedürfen und irreduzibel
sozial
sind.\footnote{\cite[Vgl.][68]{Reid:EssaysontheIntellectualPowersofMan2002}:
\enquote{Some operations of our minds, from their very nature, are \ori{social},
others are \ori{solitary}.} Weiter heißt es: \enquote{By the first, I understand
such operations as necessarily suppose an intercourse with some other
intelligent being.}} \authorcite{Reid:EssaysontheIntellectualPowersofMan2002}s
bemerkenswerte Behauptung ist, dass es aussichtslos sei, geistige Tätigkeiten,
die nur als soziale Tätigkeiten denkbar sind -- die \emph{social operations of
mind} wie Mitteilen, Versprechen, Fragen oder Befehlen --, in Elemente auflösen
zu wollen, die individualistisch verstanden werden können -- \emph{solitary
operations of mind}. Die
philosophische Tradition habe zwar stets versucht, sie in Termini
individualistisch verstandener Tätigkeiten zu analysieren; dabei zeige sich aber
nur, dass allgemein völlig Verständliches mysteriös
werde.\footnote{\cite[Vgl.][68]{Reid:EssaysontheIntellectualPowersofMan2002}:
\enquote{These acts of mind are perfectly understood by every man of common
understanding; but, when philosophers attempt to bring them within the pale of
ther divisions, by analysing them, they find inexplicable mysteries, and even
contradictions, in them.}}

\begin{comment}
Gerade in der Analytischen Philosophie ist bis heute die Tendenz verbreitet,
gemeinsames Handeln und kollektive Intentionalität als (komplexes) Gefüge jeweils
individueller Handlungen und Intentionen zu analysieren.
\footnote{\authorfullcite{Tuomela:We-intentions1988} fachten eine Diskussion um den
ontologischen Status kollektiver Intentionalität an, in der
\authorfullcite{Searle:CollectiveIntentionsandActions1992} die
anti-reduktionistische Gegenposition vertrat, wonach kollektive Intentionalität
ein nicht-reduzierbares und natürliches Phänomen
darstellt; \cite[vgl.][]{Tuomela:We-intentions1988} sowie die neueren
Ausführungen in \cite{Tuomela:We-IntentionsRevisited2005}, in denen Raimo
\authorcite{Tuomela:We-IntentionsRevisited2005} auf Einwände eingeht.
\cite[Vgl.][]{Searle:CollectiveIntentionsandActions1992}.
Siehe auch die von \textcite{Tuomela:We-IntentionsRevisited2005} angeführte und
diskutierte Literatur.}\end{comment}
Dabei legt \authorcite{Reid:EssaysontheIntellectualPowersofMan2002} sein
Augenmerk auf solche Verstandeshandlungen, deren Ausführung nur durch das
gemeinsame Handeln (\enquote{\emph{intercourse}}) mit anderen vernünftigen Wesen
möglich (und zwar \emph{logisch} möglich) sei.  Dabei liegt eine
\emph{gemeinsame} Tätigkeit vor und nicht das gleichzeitige Ausführen
individueller Tätigkeiten wie in dem Fall, in dem zwei Passanten zufällig
denselben Weg durch eine Stadt zurücklegen, ohne dieses Handeln als gemeinsames
verstehen zu
können.\footnote{\authorfullcite{Smith:DerWohlstandderNationen1993} scheint sich
das Gruppenverhalten bei Tieren so vorzustellen, dass es sich nur scheinbar um
gemeinsames Verhalten handelt, wenn bspw. ein Rudel Wölfe ein Reh jagt. Nach ihm
sollten wir diese Situation so auffassen, dass mehrere Wölfe unabhängig
voneinander jeweils ein Reh jagen und es sich dabei zufällig um ein und dasselbe
Reh handelt \parencite[vgl.][16]{Smith:DerWohlstandderNationen1993}. So abwegig
diese Beschreibungsform auch sein mag, sie verdeutlicht den hier anvisierten
Unterschied.} Nach \authorcite{Reid:EssaysontheIntellectualPowersofMan2002}
liegt ein solcher Unterschied zwischen individuellem und gemeinsamem Handeln
gerade auch bei Urteilen auf der einen und Mitteilungen auf der anderen Seite
vor. Mitteilen sei eine gänzlich andere Tätigkeit als Urteilen. Wenn ein Zeuge
einem Richter eine Auskunft gibt, dann drücke diese nicht sein Urteil
(\enquote{\emph{judgment}}), sondern eine Mitteilung
(\enquote{\emph{testimony}}) aus. Fragen wir hingegen jemanden nach seiner
Überzeugungen in einer Angelegenheit der Wissenschaft, dann drückt seine Antwort
ein Urteil (\enquote{\emph{judgment}}) aus, ist aber keine Mitteilung
(\enquote{\emph{testimony}}).\footnote{\enquote{Affirmation and denial is very often the expression of testimony, which is a
  different act of the mind, and ought to be distinguished from judgment.
  
  A judge asks of a witness what he knows of such a matter to which he was an eye
  or ear witness. He answers, by affirming or denying something. But his answer
  does not express his judgment; it is his testimony. Again, I ask a man his
  opinion in a matter of science or of criticism. His answer is not testimony; it
  is the expression of his
  judgment.}
  \parencite[][\pno~406\,f.]{Reid:EssaysontheIntellectualPowersofMan2002}.}

Urteil und Mitteilung unterscheiden sich nicht nur dadurch, dass eine Mitteilung
jederzeit in Anwesenheit anderer \emph{vernehmbar artikuliert} sein
muss\footnote{\cite[Vgl.][407]{Reid:EssaysontheIntellectualPowersofMan2002}:
\enquote{Testimony is a social act, and it is essential to it to be expressed by
words or signs. A tacit testimony is a contradiction: But there is no
contradiction in a tacit judgment; it is complete without being expressed.}},
sondern vor allem auch darin, dass es sich nur bei falschen Mitteilungen um
\emph{Lügen} handelt, nicht aber bei falschen
Urteilen\footnote{\cite[Vgl.][407]{Reid:EssaysontheIntellectualPowersofMan2002}:
\enquote{In testimony a man pledges his veracity for what he affirms; so that a
false testimony is a lie: But a wrong judgment is not a lie; it is only an
error.}}. Mitteilungen
ähneln Versprechen, die Verpflichtungen begründen, während Urteile in Analogie
zu bloßen Absichtsbekundungen zu verstehen sind, mit denen keine Verpflichtung
einhergeht und die daher auch keine Rechte auf der Seite des Hörers
begründen.\footnote{Einen ähnlichen Unterschied artikuliert
\authorfullcite{Austin:OtherMinds1979} bezüglich der Formulierungen \enquote{Ich
\emph{weiß}, dass $p$} und \enquote{Ich bin \emph{mir sicher}, dass $p$}. Im
ersten Fall gibt der Sprecher dem Hörer die Erlaubnis, etwas auf seine (des
Sprechers) Verantwortung hin unter seine Überzeugungen aufzunehmen und auch
weiterzugeben -- es handelt sich um eine Mitteilung im hier besprochenen Sinn.
Im zweiten Fall hingegen artikuliert der Sprecher nur ein eigenes Urteil, dass
nicht als Mitteilung aufzufassen ist und welches der Hörer auch nur auf seine
eigene (des Hörers) Verantwortung hin übernehmen und weitergeben darf;
\cite[vgl.][98--103]{Austin:OtherMinds1979}. Einen Zusammenhang zwischen
\name[Thomas]{Reid}s Theorie sozialer Verstandestätigkeiten und
\authorcite{Austin:OtherMinds1979}s Überlegungen zur Sprechakttheorie sehen
\textcite[vgl.][]{Schuhmann:ElementsofspeechacttheoryintheworkofThomasReid1990}.}
Versprechen wiederum sind die paradigmatischen Fälle genuin sozialer
intellektueller Tätigkeiten, die nur als ursprünglich soziale Phänomene
verständlich sind und die Vorlage zu Diskussion anderer Formen sozialer Verstandestätigkeiten
liefern.\footnote{\cite[Vgl.][]{Coady:ReidandtheSocialOperationsofMind2004}.}


Bei dem Versuch, soziale Phänomene wie Versprechen, Absprachen oder Verträge auf
individualistische Handlungen zurückzuführen, kommt uns der
Verpflichtungscharakter solcher Phänomene abhanden. Und dann wird mysteriös, wie
es sein kann, dass uns unser individuelles Handeln (das Äußern eines
Versprechens) mit einer Verbindlichkeit belegen kann. Etwas Ähnliches passiere
-- so \authorcite{Reid:EssaysontheIntellectualPowersofMan2002} -- auch bei
Mitteilungen: Der genuin soziale Charakter derselben lasse sich nicht erläutern,
indem man eine Mitteilung als ein (privates) Urteil \singlequote{plus $X$}
analysiert. Wir müssen die \emph{social operations of mind} \name[Thomas]{Reid}
zufolge als ursprünglich und natürlich
ansehen\footnote{\cite[Vgl.][69]{Reid:EssaysontheIntellectualPowersofMan2002}:
\enquote{The Author of our being intended us to be social beings, and has, for
that end, given us social intellectual powers, as well as social affections.
Both are original parts of our constitution, and the exertions of both no less
natural than the exertions of those powers that are solitary and selfish.}} und
ihnen als solchen mehr Aufmerksamkeit widmen. Es sei ein Fehler, dass
Philosophen sich bei der Untersuchung menschlichen Denkens stets auf die
Betrachtung des einzelnen Denkers in Situationen der Einsamkeit
fokussieren.\footnote{\cite[Vgl.][70]{Reid:EssaysontheIntellectualPowersofMan2002}:
\enquote{Why have speculative men laboured so anxiously to analyse our solitary
operations, and given so little attention to the social? I know no other reason
but this, that, in the divisions that have been made of the mind’s operations,
the social have been omitted, and thereby thrown behind the curtain.}} Diesen
Fehler macht offensichtlich
\authorcite{Hume:AnEnquiryConcerningHumanUnderstanding1964}, wenn er den Erwerb
von testimonialem Wissen so rekonstruiert, dass die Äußerungen unserer
Informanten analog zu Krankheitssymptomen oder Fossilien verstanden werden, von
denen wir auf eine Krankheitsursache oder den früheren Zustand der Fauna
schließen. Hier gerät der soziale Aspekt des Mitteilens völlig aus dem Blick.

Nun mag es sein, dass \name[Thomas]{Reid}s Thematisierung der sozialen Grundlagen
unseres Denkens der Komplexität dieses Zusammenhangs nicht abschließend gerecht
wird. Insbesondere mag man
bezweifeln, dass es möglich ist, so scharf zwischen individuellen und sozialen
Verstandestätigkeiten zu unterscheiden, wie \name[Thomas]{Reid} dies für richtig
hält.\footnote{\cite[Vgl.][197]{Coady:ReidandtheSocialOperationsofMind2004}:
\enquote{The basic problem is that Reid’s contrasting of the solitary and the social
operations draws him unwittingly into an equally sharp contrast between things
that in reality overlap.}} Vielleicht sollten wir selbst die Annahme in Zweifel
ziehen, dass es überhaupt Verstandestätigkeiten gibt, die als ursprünglich
individualistisch zu verstehen sind. \authorcite{Coady:ReidandtheSocialOperationsofMind2004} etwa
bezweifelt, dass \name[Thomas]{Reid} umfassenden Einblick hatte, wie weitreichend unsere
eigene Urteilsfähigkeit durch den vorherigen Einfluss anderer bedingt
ist.\footnote{\cite[Vgl.][197--201]{Coady:ReidandtheSocialOperationsofMind2004}.
Möglicherweise stellt \name[Immanuel]{Kant}s Position mit ihrer Betonung der
sozialen Grundlagen des je eigenen mündigen Urteilens (siehe Kap.
\ref{section:sensuscommunis}) hier einen entscheidenden Fortschritt dar.} Wie
auch immer dies zu bewerten ist, so sollte doch der Fortschritt gegenüber
\authorcite{Hume:AnEnquiryConcerningHumanUnderstanding1964} in Bezug auf
testimoniales Wissen beachtet werden: Während
\authorcite{Hume:AnEnquiryConcerningHumanUnderstanding1964} ohne weiteres davon
ausgeht, dass die eigene direkte Wahrnehmung und die eigene Erinnerung völlig
verschieden seien von dem Erkenntnisgewinn durch Mitteilungen -- ersteres ist
\singlequote{unmittelbar}, zweiteres beruht auf einer Schlussfolgerung --, sieht
\authorcite{Reid:EssaysontheIntellectualPowersofMan2002} keinen nennenswerten
Unterschied zwischen beiden. In seiner Schrift
\titel{An Inquiry Into the Human Mind on the Principles of Common Sense}
behauptet \name[Thomas]{Reid} eine enge Analogie zwischen dem Zeugnis anderer Menschen
und dem Zeugnis der Sinne, die es angeraten sein lasse, beides zugleich
abzuhandeln.\footnote{\cite[Vgl.][190]{Reid:AnInquiryIntotheHumanMindonthePrinciplesofCommonSense1997}:
\enquote{[S]o remarkable is the analogy between these two, and the analogy
between the principles of the mind which are subservient to the one and those
which are subservient to the other, that, without further apology, we shall
consider them together.}}


Nach \name[Thomas]{Reid} handelt es sich bei dem Erwerb von Informationen durch
andere um eine selbständige, aber mit der eigenen Erfahrung auf eine Stufe gestellte
Wissensquelle. Sie bedarf daher auch keiner Rechtfertigung auf der Grundlage von
Erfahrung, sondern ist von sich auch bereits legitim. Unser testimonialer
Wissenserwerb beruht auf zwei eigenständigen, ursprünglichen Prinzipien, die
keiner vorgängigen Rechtfertigung bedürfen:
\begin{nummerierung}
\item Das \emph{Prinzip der Aufrichtigkeit} (\enquote{principle of
veracity}\footnote{\cite[][194]{Reid:AnInquiryIntotheHumanMindonthePrinciplesofCommonSense1997}.})
besagt, dass wir dazu tendieren, die Wahrheit zu sagen und die Zeichen unserer
Sprache zur Bezeichnung unserer wirklichen Überzeugungen zu
gebrauchen.\footnote{\enquote{The first of these principles is, a propensity
to speak the truth, and to use the signs of language, so as to convey our real
sentiments}
\parencite[][193]{Reid:AnInquiryIntotheHumanMindonthePrinciplesofCommonSense1997}.
In den 21 Jahre später erschienen \titel{Essays on the Intellectual Powers of
Man} heißt es: \enquote{Another first principle I take to be, That certain
features of the countenance, sounds of the voice, and gestures of the body,
indicate certain thoughts and dispositions of mind}
\parencite[][484]{Reid:EssaysontheIntellectualPowersofMan2002}.}
\item Das \emph{Prinzip der \singlequote{Gutgläubigkeit}} (\enquote{principle of
credulity}\footcite[][194]{Reid:AnInquiryIntotheHumanMindonthePrinciplesofCommonSense1997})
besagt, dass wir Vertrauen in die Aufrichtigkeit der anderen haben und glauben,
was sie uns
erzählen.\footnote{\enquote{Another original principle implanted in us by the
Supreme Being, is a disposition to confide in the veracity of others, and to
believe what they tell us}
\parencite[][194]{Reid:AnInquiryIntotheHumanMindonthePrinciplesofCommonSense1997}.
Analog in den \titel{Essays on the Intellectual Powers of Man}:
\enquote{Another first principle appears to me to be, That there is a certain
regard due to human testimony in matters of fact, and even to human authority
in matters of opinion}
\parencite[][487]{Reid:EssaysontheIntellectualPowersofMan2002}.}
\end{nummerierung}
\name[Thomas]{Reid} attackiert nicht nur den testimonialen Reduktionismus in der Form,
wonach dieser behauptet, es gebe eine Erfahrungsgrundlage für unser Vertrauen in
die Glaubwürdigkeit anderer, sondern auch die Position, die auf individualistischer Grundlage Gründe für unsere
eigene Aufrichtigkeit entwickeln zu müssen meint. Wir müssen nicht erst Gründe
dafür kennenlernen, die Wahrheit zu sagen (etwa soziale Sanktionsmechanismen
gegenüber Lügnern), sondern tendieren von Natur aus dazu. Gewiss handelt es sich
bei den beiden Prinzipien um zwei Seiten derselben Medaille, wobei das Prinzip
der Gutgläubigkeit dennoch hier das einschlägigere ist.


Das Vertrauen in Behauptungen anderer ist gegenüber dem Misstrauen, das erst
über einen Lernprozess erworben wird, ursprünglich. Wir folgen dem \singlequote{Prinzip der
Gutgläubigkeit}, bevor wir auch nur in der Lage sind, Überlegungen zur
Verlässlichkeit menschlichen Zeugnisses
anzustellen.\footnote{\cite[Vgl.][487]{Reid:EssaysontheIntellectualPowersofMan2002}:
\enquote{Before we are capable of reasoning about testimony or authority, there
are many things which it concerns us to know, for which we can have no other
evidence.}} Erst in der Folge, wenn unsere Fähigkeiten entwickelt sind, können
wir Gründe für unsere Erkenntnisprinzipien bezüglich testimonialem Wissen
finden.\footnote{\cite[Vgl.][488]{Reid:EssaysontheIntellectualPowersofMan2002}:
\enquote{[W]hen our faculties ripen, we find reason to check that propensity to
yield to testimony and to authority, which was so necessary and so natural in
the first period of life.}} Erwachsenen Menschen mit gereiften intellektuellen
Fähigkeiten ist es überhaupt erst möglich, Gründe für die Verlässlichkeit
menschlichen Zeugnisses in einer Weise vorzubringen, wie sie \name[David]{Hume}
bespricht. Hinzufügen ließe sich noch, dass gerade Überlegungen zu Charakter und
Interesselosigkeit unserer Informanten, denen großes Gewicht bei der Abwägung
zukommt, ob eine Information glaubhaft ist, oft selbst auf testimoniales Wissen
bezüglich dieser Informanten angewiesen sind.\footnote{So
\textcite[vgl.][196]{Coady:ReidandtheSocialOperationsofMind2004} gegen
\name[Thomas]{Reid}.} Ohne ursprüngliches Vertrauen in die Aufrichtigkeit
anderer käme unser Denken gar nicht in
Gang.\footnote{\cite[Vgl.][193]{Coady:ReidandtheSocialOperationsofMind2004}:
\enquote{The quest for knowledge begins with dependence upon the word of others
rather than validating that dependence some way down the path as a secondary
supplement to individual knowledge. This \singlequote{beginning} is historical
ans also epistemically normative.}} Ebenso ist das wahrheitsgemäße Aussagen ursprünglich,
die Lüge hingegen etwas, was wir erst spät erlernen und wofür wir tatsächlich
erst Gründe kennenlernen müssen. Beide Prinzipien sind natürlich auch
voneinander unabhängig. Wir haben nicht deshalb Vertrauen in die Mitteilungen
anderer, weil wir von einem allgemeinen Prinzip der Aufrichtigkeit wissen,
sondern schon lange zuvor, wenn wir über solche Prinzipien noch gar nicht
nachgedacht haben.

\name[Thomas]{Reid} beschreibt hier nach eigenem Verständnis zunächst Prinzipien, nach
denen Menschen \emph{de facto} denken, also solche, die wir tatsächlich
anwenden, ohne sie zuvor begründet zu haben. Es handelt sich um Behauptungen in
Betreff der \emph{tatsächlichen} individuellen Entwicklung menschlichen Denkens, die
gleichwohl nicht zufällig so und nicht anders verlaufe. Heißt dies dann, dass
\name[David]{Hume} zwar in Bezug auf die \emph{tatsächliche} Entwicklung menschlichen
Denkens Unrecht hat, aber ein an \authorcite{Descartes:OeuvresdeDescartes1983} angelehntes Programm der
individualistischen \emph{Rekonstruktion} dennoch durchführbar sein könnte?
Können wir den Prinzipien, denen wir \emph{de facto} folgen, solche
entgegenstellen, denen wir \emph{idealiter} folgen \emph{sollten}?

\name[Thomas]{Reid} konzidiert zumindest, dass Menschen Überlegungen anstellen
können, wie \name[David]{Hume} sie generell fordert, wenn auch erst nachdem sie zunächst dem
Prinzip der Gutgläubigkeit entsprechend Informationen erworben haben.
Möglicherweise vertritt er zwar keinen globalen, wohl aber einen lokalen
Reduktionismus, der folgendes behauptet: Es sei zwar unmöglich, unseren gesamten
Wissensbestand auf individualistischer Grundlage zu rekonstruieren; wohl aber
könnten wir, die wir als erwachsene Menschen mit reifer Vernunft schon viel über
unsere Mitmenschen wissen (und sei es aus dem Zeugnis anderer), nun bei jeder
Mitteilung überlegen, welche allgemeinen Gründe für die Glaubwürdigkeit
sprechen. Dabei sei es uns erlaubt, auf psychologisches Wissen zurückzugreifen,
das wir nicht selbst erarbeitet haben, sondern welches wir von
anderen mitgeteilt bekamen. (Wir könnten es aus
\authorcite{Kant:GesammelteWerke1900ff.}s Anthropologievorlesungen erworben
haben.)

\authorfullcite{Fricker:AgainstGullibility1994}, die bekannteste Vertreterin
eines lokalen Reduktionismus, bezeichnet den Credulismus als Leichtgläubigkeit
(\enquote{gullibility})\footnote{\cite[Vgl.][145]{Fricker:AgainstGullibility1994}:
\enquote{The thesis I advocate in opposition to a PR thesis [der These von einem
präsumtiven epistemischen Recht; A.\,G.], is that a hearer should always engage
in some assessment of the speaker for trustworthiness. To believe what is
asserted without doing so is to believe blindly, uncritically.
This is gullibility.}}, der wegen der Realitätsferne eines \emph{globalen}
Reduktionismus eben durch die Alternative eines solchen \emph{lokalen}
Reduktionismus zu begegnen sei.\footnote{\cite[Vgl.][136]{Fricker:AgainstGullibility1994}: \enquote{My
issue is the local reductionist question: whether, within a subject's coherent
system of beliefs and inferential practices {\punkt}, beliefs from testimony can
be exhibited as justified in virtue of very general patterns of inference and
justification; or if a normative epistemic principle special to testimony must
be invoked to vindicate them and explain their status as knowledge.}}
\name[Elizabeth]{Fricker} behauptet, dass nur ein lokaler Reduktionismus, der
im Falle des Erwerbs von Informationen Evidenz für die Glaubwürdigkeit des
jeweils Mitteilenden einfordert -- sonst könnte aus Sicht des lokalen
Reduktionismus der Informationserwerb nicht als mündig beschrieben werden --,
Ausdruck einer kritischen Haltung sein
könne.\footnote{\cite[Vgl.][]{Fricker:AgainstGullibility1994}.} Wenn $A$, um
sich von der Wahrheit der Aussage $B$'s zu überzeugen, in \emph{diesem} Sinne
den eigenen Verstand gebrauchen soll, so muss er aus dem Hören, was $B$ sagt,
und seinen früheren Erfahrungen mit $B$'s Auskünften (oder mit Aussagen
vergleichbarer Zeugen) auf die Wahrheit der Aussage \emph{schließen}. Wir hätten
-- wäre ein solcher Schluss möglich -- einen Begriff von Mündigkeit, der auf der
einen Seite alle unsere Erkenntnisse der Forderung nach Selbständigkeit
unterwirft und auf der anderen Seite die Möglichkeit eröffnet, testimoniales
Wissen zu haben. Man kann nun aber bezweifeln, dass es möglich ist, mit einer
solchen Strategie testimoniales Wissen in einem nennenswerten Umfang zu
begründen.\footnote{Siehe hierzu die klassische Argumentation in
\cite{Anscombe:HumeandJuliusCaesar1973}.} Auch \name[Elizabeth]{Fricker} hält es
für vergeblich, unser Wissen auf einer vollständig individualistischen Grundlage
zu rekonstruieren. Aber gegeben unseren gesamten Wissensvorrat, der sich zu
großen Teilen aus genuin testimonialem Wissen speist, lasse sich doch für
einzelne Fälle von Mitteilungen eine lokale Reduktion leisten. Die Basis für
eine solche Reduktion bildet unser gesamtes Hintergrundwissen über das Verhalten
unserer Mitmenschen, all das, was wir (Volks-) Psychologie oder
\singlequote{Menschenkenntnis} nennen können. Auch \name[Immanuel]{Kant}
beispielsweise behandelt solches Wissen unter dem Titel der
\enquote{Weltklugheit} in der pragmatischen Anthropologie und schreibt ihm eine
wichtige Funktion im Rahmen der Aufklärung zu, wie wir oben
sahen.\footnote{Siehe Kapitel \ref{subsection:DieBestimmungdesMenschen},
v.\,a.~S.~\pageref{Absatz:Weltklugheit}--\pageref{Absatz:Weltklugheit-ENDE}.}

Elizabeth \name[Elizabeth]{Fricker} propagiert den lokalen Reduktionismus als einzig
verbleibende Möglichkeit einer sozialen Erkenntnistheorie, die die Möglichkeit
testimonialen Wissens mit dem Streben nach Mündigkeit vereint. Demnach wäre das
Prinzip der Gutgläubigkeit im Kindesalter unumgänglich, aber im weiteren Verlauf
immer mehr zurückzudrängen, bis wir irgendwann gänzlich auf seine Anwendung bei
dem Erwerb neuen Wissens verzichten könnten, wenngleich unser Wissen natürlich
weiterhin von früher erworbenem genuin testimonialen Wissen abhängig bliebe, für
das wir keine lokalen Reduktionen vorliegen haben. Sowohl aus der Sicht des
Credulismus, als auch aus der des lokalen Reduktionismus hängt es von unserer
Abwägung von Gründen für und wider die Glaubwürdigkeit des Informanten ab, ob
unser Umgang mit testimonialem Wissen kritisch oder unkritisch ist. Und der
Vorteil beider Positionen ist damit, dass sie uns ermöglichen, die Forderung
nach einem kritischen Umgang mit testimonialem Wissen durch eine Vielzahl
epistemischer Regeln zu konkretisieren.
\name[Thomas]{Reid} merkt an, dass ein kritisch denkender Mensch zunehmend
seinen Erwerb testimonialen Wissens durch Überlegungen wie die von
\name[David]{Hume} angeführten begleite. Schließlich liefert uns die Erfahrung
nicht nur Belege der Unzuverlässigkeit von Mitteilungen, sondern gerade auch die
von \name[David]{Hume} angeführten Belege ihrer
Verlässlichkeit\footnote{\cite[Vgl.][488]{Reid:EssaysontheIntellectualPowersofMan2002}:
\enquote{But when our faculties ripen, we find reason to check that propensity
to yield to testimony and to authority, which was so necessary and so natural in
the first period of life.}}; nur geschieht dies nicht, \emph{bevor} wir erstmals
auf Zeugnisse rekurrieren.
\authorcite{Reid:EssaysontheIntellectualPowersofMan2002} behauptet, dass wir mit
zunehmendem Alter immer mehr Wissen auf Weisen erwerben, die nicht auf dem
Vertrauen in andere beruhen. Aber dennoch könne es niemals zu einem
völligen Verzicht auf testimoniales Wissen kommen. Denn auch wenn die Vernunft zunehmend
weniger auf die Hilfe anderer rekurrieren müsse, sei doch jeder durch seine
gesamte Lebensspanne hindurch darauf angewiesen, durch Mitteilungen anderer
Erkenntnisse zu erwerben, die durch den \singlequote{eigenen Vernunftgebrauch}
nicht erworben werden können.\footnote{\enquote{Reason hath likewise her
infancy, when she must be carried in arms: then she leans entirely upon
authority, by natural instinct, as if she was conscious of her own weakness; and
without this support, she becomes vertiginous. When brought to maturity by
proper culture, she begins to feel her own strength, and leans less upon the
reason of others; she learns to suspect testimony in some cases, and to
disbelieve it in others; and sets bounds to that authority to which she was at
first entirely subject. But still, to the end of life, she finds a necessity of
borrowing light from testimony, where she has none within herself, and of
leaning in some degree upon the reason of others, where she is conscious of her
own imbecillity}
\parencite[][195.15--25]{Reid:AnInquiryIntotheHumanMindonthePrinciplesofCommonSense1997}.}
Einen Zustand völliger kognitiver Autarkie erreichen wir niemals, weder im Sinne
des globalen, noch im Sinne des lokalen Reduktionismus.

Das Problem des erkenntnistheoretischen Individualismus -- die dauerhafte
Angewiesenheit auf testimoniales Wissen -- ist unausweichlich. Wie
\name[Thomas]{Reid} erkennt, können wir nicht einfach beschließen, nur noch auf
den \emph{eigenen} Vernunftgebrauch zu setzen und auf jedes testimoniale Wissen zu
verzichten. Wir können ihm zufolge lediglich \emph{graduell} mündiger werden,
indem wir -- ausgehend von unserer ursprünglichen umfassenden Abhängigkeit von anderen --
zunehmend mehr Erkenntnisse \emph{selbst} kontrollieren. Das heißt einerseits,
dass wir Wissen gar nicht mehr in demselben Umfang als testimoniales
Wissen erwerben, sondern durch eigene Erfahrung und eigenen Vernunftgebrauch.
Dass dies aber nur einen Teil unseres Wissenserwerbs betreffen kann, wird klar,
wenn man versucht, sich etwa unsere Rechtspraxis ohne testimoniales Wissen, also
das Gerichtswesen ohne Zeugenaussagen vorzustellen. Testimoniales Wissen ist
kein geringer, sondern der quantitativ vielleicht größte Teil unseres
Wissens.\footnote{\cite[Vgl.][557]{Reid:EssaysontheIntellectualPowersofMan2002},
wo \name[Thomas]{Reid} in Bezug auf die verschiedenen Arten wahrscheinlichen (im
Gegensatz zu demonstrativem) Wissen schreibt: \enquote{The first kind is that of
human testimony, upon which the greatest part of human knowledge is built.}} Und
es ist illusorisch, diesen Teil vollständig zu ersetzen. Noch immer werden wir
darauf angewiesen sein, einen großen Teil unseres Wissens als testimoniales
Wissen zu erwerben. Und entsprechend verheerend wirkte sich ein Verzicht auf
testimoniales Wissen aus, welches wir \emph{ohne} nicht-testimoniale
Rechtfertigung, also allein auf Grundlage des Prinzips der Gutgläubigkeit
gewinnen.\footnote{\cite[Vgl.][194]{Reid:AnInquiryIntotheHumanMindonthePrinciplesofCommonSense1997}:
\enquote{It is evident, that, in the matter of testimony, the balance of human
judgment is by nature inclined to the side of belief; and turns to that side of
itself, when there is nothing put into the oposite scale. If it was not so, no
proposition that is uttered in discourse would be believed, until it was
examined and tried by reason; and most men would be unable to find reasons for
believing the thousandth part of what is told them. Such distrust and
incredulity would deprive us of the greatest benefits of society, and place us
in a worse condition than that of savages.}} Andererseits heißt dies, dass wir
im Falle testimonialen Wissens Gründe für und gegen die Glaubwürdigkeit einer
Mitteilung kennen, die sich auch, aber nicht nur auf die Glaubwürdigkeit eines
Informanten
stützen\footnote{\cite[Vgl.][558]{Reid:EssaysontheIntellectualPowersofMan2002}:
\enquote{The belief we give to testimony in many cases is not solely grounded
upon the veracity of the testifier. In a single testimony, we consider the
motives a man might have to falsity. If there be no appearances of any such
motive, much more if there be motives on the other side, his testimony has
weight independent of his moral character. If the testimony be circumstantial,
we consider how far the circumstances agree together, and with things that are
known. It is so very difficult to fabricate a story, which cannot be detected by
a judicious examination of the circumstances, that it acquires evidence, by
being able to bear such a trial. There is an art in detecting false evidence in
judicial proceedings, well known to able judges and barristers; so that I
believe few false witnesses leave the bar without suspicion of their guilt.}}.
Die Bewertung von Mitteilungen gehört zu den wichtigsten Kompetenzen, die jeder von uns erwirbt
und die zentrale Funktionen im gesellschaftlichen Zusammenleben
erfüllen.\footnote{\cite[Vgl.][\pno~557f.]{Reid:EssaysontheIntellectualPowersofMan2002}:
\enquote{The faith of history depends upon it, as well as the judgment of solemn
tribunals, with regard to mens acquired rights, and with regard to their guilt
or innocence when they are charged with crimes. A great part of the business of
the Judge, of Counsel at the bar, of the Historian, the Critic, and the
Antiquitarian, is to canvass and weigh this kind of evidence; and no man can act
with common prudence in the ordinary occurrences of life, who has not some
competent judgment of
it.}} Die Basis hierfür bilden freilich das von anderen erworbene Wissen und die
Kompetenzen, die wir durch Unterweisung erworben haben. Hier scheint
Mündigkeit eine Grenze zu haben; den schlechthin mündigen Menschen gibt es
danach nicht.

\subsection{Christian August Crusius' Präsumtionstheorie}
\label{subsubsection:ChristianAugustCrusius}
Die Position \authorfullcite{Reid:EssaysontheIntellectualPowersofMan2002}s ist
komfortabler als die Position des Individualismus. Allerdings hat sie einen
ernsthaften Nachteil: Wir können Wissen
aus den Mitteilungen anderer erhalten, weil diese Mitteilungen \emph{de facto}
eine zuverlässige Wissensquelle sind. Wir wissen nicht zuvor um die
Zuverlässigkeit dieser Wissensquelle -- denn woher sollten wir dies zuvor
erfahren --, sondern handeln unbewusst nach Prinzipien, die unser Erkennen
leiten. Diese Prinzipien sind von uns nicht erkannt, sondern
\singlequote{angeboren}; und so erwerben wir Wissen auf der Grundlage von
Grundsätzen, die uns selten bewusst sind und die selbst keine Rechtfertigung
besitzen. Aus einer Perspektive, wie sie uns als Ausgangspunkt der
Aufklärung interessiert, ist es hingegen nicht
sinnvoll, unsere epistemische Situation nach Kriterien zu bewerten, deren
Verletzt- oder Erfülltsein das Subjekt nicht kontrollieren kann. Angeborene
Grundsätze helfen uns generell nicht weiter, schon weil sie allzu sehr nach
Vorurteilen aussehen; wir wollen bewusst \emph{vernünftigen} Grundsätzen folgen.
Eben dies ist ja die Forderung \name[Immanuel]{Kant}s, wenn dieser das
Selbstdenken über die Kontrolle der Vernünftigkeit der Grundsätze des Denkens
und Erkennens bestimmt.\footnote{Siehe oben, Kap.
\ref{subsection:SelbstdenkenbeiKant}.} Sich mit dem vorliegen unvermeidbarer
angeborener epistemischer Prinzipien zu beruhigen, erweckt den Anschein eines
Aufrufs zu Bequemlichkeit und Unmündigkeit.

Der Credulismus muss jedoch nicht unkritisch sein, insofern wir im Laufe der
Zeit lernen, wann wir Gründe haben, einer Information zu misstrauen. Es gehört
zu den wichtigen Kompetenzen, die jeder von uns im Laufe seines Lebens mehr oder
minder stark ausprägt, die Bonität von Informationen und Informanten zu
bewerten. Nur handelt es sich dabei um eine Kompetenz, die sich erst im Laufe
der Jahre und auch auf der Grundlage testimonialen Wissens ausprägt und die
nicht schon dem ersten Erwerb testimonialen Wissens zugrunde liegt. Sie
\emph{konstituiert} nicht testimoniales Wissen, aber sie \emph{reguliert} es.
Wir könnten etwa mit \authorfullcite{Austin:OtherMinds1979} sagen, dass die
Übernahme von Überzeugungen die \emph{default}-Position ist, die wir erst in
Frage stellen, wenn \emph{berechtigte} Zweifel auftauchen:
\begin{quote}
Natürlicherweise sind wir vernünftig: Wir sagen nicht, wir wüssten etwas (aus
zweiter Hand), wenn es irgendeinen konkreten Grund gibt, das Zeugnis
anzuzweifeln. Aber es muss \ori{irgendeinen} Grund geben. Es ist fundamental für
Unterhaltungen (wie für andere Dinge), dass wir berechtigt sind, anderen zu
vertrauen, außer insofern es einen konkreten Grund für Misstrauen gibt.
Menschen zu glauben, ihr Zeugnis anzunehmen, das ist der, oder ein wichtiger,
Sinn von Unterhaltungen.\footnote{\enquote{Naturally, we are judicious: we
don't say we know (at second hand) if there is any special reason to doubt the testimony: but there has to be
\ori{some} reason. It is fundamental in talking (as in other matters) that we
are entitled to trust others, except in so far as there is some concrete reason
to distrust them. Believing persons, accepting testimony, is the, or one main,
point of talking} \parencite[][82]{Austin:OtherMinds1979}.}
\end{quote}
\authorcite{Austin:OtherMinds1979} behauptet, dass wir generell nicht alle Zweifel
ausschließen können, aber auch nicht ausschließen müssen, um Wissen zu erwerben, und dass
dies für alle Arten von Wissen gilt, testimoniales wie
Wahrnehmungswissen.\footnote{\cite[Vgl.][84]{Austin:OtherMinds1979}:
\enquote{Enough is enough: it does not mean everything. Enough means enough to
show that (within reason, and for present intents and purposes) it
\singlequote{can’t} be anything else, there is no room for an alternative,
competing, description of it. It does \ori{not} mean, for example, enough to
show it isn’t a \ori{stuffed} goldfinch.}} Nach \authorcite{Austin:OtherMinds1979} haben wir
ein präsumtives epistemisches Recht, das unser Vertrauen in die jeweilige Autorität
zunächst legitimiert, auch wenn uns keine Gründe für ein solches Vertrauen
vorliegen.\footnote{Der Ausdruck \enquote{präsumtives epistemisches Recht}
(\enquote{\emph{presumtive epistemic right}}) stammt von Elizabeth
\textcite[][140]{Fricker:AgainstGullibility1994}.} Diese Präsumtion der
Glaubwürdigkeit bildet somit das zentrale (und irreduzible) Prinzip, welches den
Erwerb testimonialen Wissens ermöglicht. Ich hatte für diese Position bereits
den Ausdruck \enquote{Credulismus} eingeführt.\footnote{Siehe oben,
S.~\pageref{Fussnote:BegriffdesCredulismus}, Fußnote
\ref{Fussnote:BegriffdesCredulismus}.}

Es kann gute Gründe dafür geben, misstrauisch zu sein. Aber Vertrauen oder die
Annahme der Zuverlässigkeit einer Information und eines Informanten bleibt die
\emph{default}-Position. Deswegen bietet sich hier der Begriff der Präsumtion
an: Ähnlich wie bei juristischen Präsumtionen -- etwa der Unschuldsvermutung --
handelt es sich nicht um einen inferentiellen Freifahrtschein (auf die
Glaubwürdigkeit), sondern um eine Bestimmung, wo die Beweislast zu Beginn liegt.
Der Vorteil einer Präsumtion gegenüber dem Verweis auf angeborene Grundsätze
liegt auf der Hand: Es handelt sich um einen Grundsatz, der uns sagt, was zu
glauben \emph{vernünftig} ist und dessen vernünftige Anwendung wir selbst
kontrollieren können. Eine Präsumtionstheorie passt also zu einer
internalistischen Position.


Explizit von einer \enquote{Präsumtion} zugunsten der Information oder des Informanten spricht bereits
\authorfullcite{Crusius:Anweisungvernuenftigzuleben1744} im \titel{Weg zur
Gewißheit und Zuverläßigkeit der menschlichen Erkenntniß}, einem Werk, welches
bereits einige Jahre vor \name[Thomas]{Reid}s einschlägigen Schriften publiziert
wurde.\footnote{\phantomsection\label{Anmerkung:KantundCrusiusPraesuppositionsTheorie}Vgl.
zu den folgenden Ausführungen die Überlegungen
\authorcite{Crusius:Anweisungvernuenftigzuleben1744}' zu testimonialem Wissen in
\cite[][\S\S~605--627]{Crusius:WegzurGewissheitundZuverlaessigkeitdermenschlichenErkenntniss1965}.
Ob \name[Immanuel]{Kant} diese Überlegungen bekannt waren, lässt sich nicht
zweifelsfrei ermitteln. Einerseits schien er selbst nicht über ein Exemplar
dieses Werkes von \authorcite{Crusius:Anweisungvernuenftigzuleben1744} zu
verfügen \mkbibparens{Es findet sich zumindest kein entsprechender Eintrag in
\cite{Warda:ImmanuelKantsBuecher1922}.}, andererseits findet sich in seinen
Anmerkungen in \authorfullcite{Meier:Vernunftlehre1752}s \titel{Auszug aus der
Vernunftlehre} eine Passage, die vermuten lässt, dass er
\authorcite{Crusius:Anweisungvernuenftigzuleben1744}' Überlegungen
kannte \mkbibparens{\cite[vgl.][\nopp 2589]{Kant:Reflexionen1900ff.};
\cite[][XVI: 430.4--431.8]{Kant:GesammelteWerke1900ff.}. \name[Immanuel]{Kant}
spricht dort im Kontext \enquote{historische Wahrscheinlichkeit} von der Präsumtion, dass
sich Unwahrheiten selbst verraten, und von der Bewertung der inneren Möglichkeit
und Wahrscheinlichkeit des Berichteten. Wir werden gerade hierin die zentralen
Überlegungen von \authorcite{Crusius:Anweisungvernuenftigzuleben1744}
ausmachen.}.} \Revision{Crusius ist gewiss nicht der erste Autor,
der von Präsumtionen spricht. Sein Lehrer
\authorfullcite{Hoffmann:Vernunft-Lehre1737} thematisiert die
\enquote{Präsumtions-Wahrscheinlichkeit} sehr ausführlich in seiner
\titel{Vernunft-Lehre} von 1737.\footnote{\Revision{\cite[Vgl.][1124--1144]{Hoffmann:Vernunft-Lehre1737}.
Dabei verweist \authorcite{Hoffmann:Vernunft-Lehre1737} darauf, dass bereits vor ihm
Autoren auf Präsumtionen verwiesen haben
\parencite[vgl.][Vorrede, unpaginiert]{Hoffmann:Vernunft-Lehre1737}.}}}
\authorcite{Crusius:Anweisungvernuenftigzuleben1744}' zentrale Aussage zu
testimonialem Wissen lautet: \enquote{Ein jedwedes Zeugniß hat eine historische
Beweiskraft, vermöge des Grundes, daß man die Wahrhaftigkeit bey einer Erzehlung
als den natürlichen Zustand präsumiren
müsse}\footnote{\Cite[][\S~617]{Crusius:WegzurGewissheitundZuverlaessigkeitdermenschlichenErkenntniss1965}.}.
Da \authorcite{Crusius:Anweisungvernuenftigzuleben1744} auch andere Präsumtionen
kennt, können wir der Eindeutigkeit halber hier mit
\authorcite{Crusius:Anweisungvernuenftigzuleben1744} selbst von einer
\enquote{historischen Präsumtion}\footnote{\Cite[][\S~617]{Crusius:WegzurGewissheitundZuverlaessigkeitdermenschlichenErkenntniss1965}.}
oder von der \emph{Bonitätspräsumtion} sprechen.
Diese besagt also, dass wir im Falle einer Information so lange davon auszugehen
haben, dass ihr tatsächlich Wissen und Aufrichtigkeit auf der Seite des
Informanten zugrunde liegt, wie uns keine Gründe bekannt sind, die die Präsumtion außer Kraft
setzen.\footnote{\cite[Vgl.][\S~606]{Crusius:WegzurGewissheitundZuverlaessigkeitdermenschlichenErkenntniss1965}:
\enquote{Ohne besondere Ursachen, welche die Präsumtion entkräften, oder das
Gegen\-theil erweisen, präsumiret man nicht, daß eine Erzehlung ohne Grund
sey.} Ähnlich schreibt \name[Immanuel]{Kant}: \enquote{Man hält eine Erzehlung
ohne Grund, wenn die Art, wie der Erzehlende es hat verstehn könen, schwer
einzusehen ist. Man hält sie aber darin nicht vor ungegründet, wenn man keine
Ursachen weiß} \mkbibparens{\cite[][\nopp 2589]{Kant:Reflexionen1900ff.};
\cite[][XVI: 430.7--10]{Kant:GesammelteWerke1900ff.}}.} Zwar sagt
\authorcite{Crusius:Anweisungvernuenftigzuleben1744}, dass bei testimonialem Wissen nur Wahrscheinlichkeit zu erreichen sei, die er \enquote{historische Wahrscheinlichkeit} nennt und die von
Demonstrationen zu unterscheiden sei; denn \emph{beweisen} lasse sich nicht, ob
es sich bei einer Information um eine in Wissen des Informanten fundierte
Nachricht handelt oder um eine vorsätzlich oder selbst fehlerhaft angenommene
Erdichtung.\footnote{\cite[Vgl.][\S~605]{Crusius:WegzurGewissheitundZuverlaessigkeitdermenschlichenErkenntniss1965}.}
Aber ein Blick auf alltägliche Erfahrungen\footnote{Bei diesen alltäglichen
Erfahrungen handelt es sich gewiss nicht um solche im Sinne \name[David]{Hume}s. Wir
machen keine alltäglichen Erfahrungen, auf deren Grundlage wir erst auf die
Bonität von Informationen und Informanten schließen. Vielmehr geht es um die
alltägliche Erfahrung, dass Informationen diese Bonität ohne weitere auf den
Einzelfall abgestimmte Legitimation bereits von uns zugesprochen bekommen.}
zeige bereits, dass es sich doch um \enquote{moralische Gewißheit} handeln
könne\footnote{\cite[Vgl.][\S~605]{Crusius:WegzurGewissheitundZuverlaessigkeitdermenschlichenErkenntniss1965}:
\enquote{Daß eine historische Wahrscheinlichkeit sey, und dieselbe sich auch in
unzehligen Fällen gar leicht in eine moralische Gewißheit verwandele, lehret so
gleich die innerliche Empfindung in den Exempeln, welche uns in dem menschlichen
Leben vorkommen.}}, was dem Status entspricht, der unserer empirischen
Erkenntnis generell zukommt.\footnote{Nach \authorcite{Meier:Vernunftlehre1752}
ist die \singlequote{moralische Gewissheit} charakterisiert durch einen
\enquote{Grad der Wahrscheinlichkeit, welcher in unserm regelmässigen
Verhalten so gut ist, als eine ausführliche Gewissheit}
\mkbibparens{\cite[][\pno~48\,f.,]{Meier:AuszugausderVernunftlehre1752}
\cite[][XVI: 432.23--24]{Kant:GesammelteWerke1900ff.}}.} Dass testimoniales
Wissen nicht wie mathematisches Wissen in Beweisen fundiert ist, tut seiner Validität keinen Abbruch.


Nun ergibt sich die Notwendigkeit, eine begründete Abgrenzung
zwischen Präsumtionen und Vorurteilen (sowie
\singlequote{Hypothesen}\footnote{Ausführlich setzt sich damit Christian
\authorcite{Wolff:Psychologiaempirica1968} auseinander; \cite[vgl.][\S\S~125--129]{Wolff:Discursuspraeliminarisdephilosophiaingenere1996}.}) zu
erarbeiten. Viele Philosophen in der Zeit der Aufklärung waren sich der Tatsache
bewusst, dass unser Denken auf grundlegenden Annahmen beruht. Sie erkannten aber auch, dass es kein
Ausweg ist, dem Begriff \enquote{Vorurteil} seine pejorative Konnotation zu
streiten und Vorurteile \emph{toto genere} zu akzeptieren. Vielmehr gilt es,
legitime Grundannahmen von (illegitimen) Vorurteilen (auch sprachlich) zu
unterscheiden. Denn während Vorurteile generell illegitime Urteile darstellen,
sollen Präsumtionen und vorläufige Urteile zumindest bei korrekter oder
\singlequote{kritischer} Anwendung legitim sein. Der Begriff der
Präsumtion ist also aus Sicht der Aufklärung schon wegen der Notwendigkeit
einschlägig, legitime von illegitimen vorläufig angenommenen
Grundsätzen (Präsumtionen von Vorurteilen) zu
unterscheiden.\footnote{\cite[Vgl.][44]{Scholz:VerstehenundRationalitaet1999}:
\enquote{Präsumtionen als vernünftige vorläufige Unterstellungen sind
strikt von den irrationalen Vorurteilen zu unterscheiden, die die Aufklärer
zurecht bekämpft haben}.}

Aus \authorcite{Crusius:Anweisungvernuenftigzuleben1744}' Sicht handelt es sich
im Falle der Bonitätspräsumtion nicht um ein Vorurteil, sondern um eine Präsumtion,
weil sie zwei Eigenschaften vereint:
\begin{nummerierung}
\item \emph{Die Bonitätspräsumtion ist als solche vernünftig.} Es wäre hingegen
gerade unvernünftig, einem Informanten zu misstrauen, wenn sich kein besonderer
Grund zu diesem Misstrauen findet. Als Gründe für die Vernünftigkeit der
Bonitätspräsumtion nennt \authorcite{Crusius:Anweisungvernuenftigzuleben1744}
die Tatsachen, dass Menschen generell ein natürliches Streben nach Wahrheit
haben und dass \enquote{zumahl in einer mercklichen Zusammensetzung, sich
Erdichtung, Unwahrheit und Affectation selbst verräth}\footnote{\cite[][\S~606]{Crusius:WegzurGewissheitundZuverlaessigkeitdermenschlichenErkenntniss1965}.}.
Bedeutsam sind sicherlich auch die Konsequenzen, die sich aus einem Verzicht auf
genuin testimoniales Wissen ergäben.

\item \emph{Die Anwendung der Bonitätspräsumtion ist im Einzelfall
kritisierbar.} Auch \authorcite{Crusius:Anweisungvernuenftigzuleben1744} kennt
bereits, was in der heutigen Diskussion \enquote{\emph{defeater}} genannt wird:
\enquote{daß ein Zeugniß schon seine Glaubwürdigkeit verliere, wenn sich der
Grund der allgemeinen historischen Präsumtion darauf nicht
schickt}\footnote{\Cite[][\S~617]{Crusius:WegzurGewissheitundZuverlaessigkeitdermenschlichenErkenntniss1965}.}.
Die Bonitätspräsumtion stellt ein vorläufiges Urteil dar, aber nicht in dem
Sinne, dass sich die Präsumtion selbst jederzeit als falsch herausstellen
könnte, sondern in dem Sinne, dass jederzeit zur Disposition steht, ob sie in
einem vorliegenden Einzelfall greift. Eine Zurückweisung der Anwendung auf einen
Einzelfall lässt die Geltung der Präsumtion selbst unangetastet. Die Anwendung
der Präsumtion zugunsten einer bestimmten Information könne daher
\enquote{entweder entkräftet, oder gar durch Gegengründe ausdrücklich widerleget
werden}\footnote{\Cite[][\S~606]{Crusius:WegzurGewissheitundZuverlaessigkeitdermenschlichenErkenntniss1965}.}.
Ein mündiger Umgang mit testimonialem Wissen zeigt sich somit in der korrekten
\emph{Anwendung} der Bonitätspräsumtion auf vorliegende Fälle.
\enquote{\ori{Keine Präsumtion gilt weiter, als wieferne sich auch ihr
Beweisgrund in dem vorhandenen Exempel appliciren lässet}}\footnote{\Cite[][\S~407]{Crusius:WegzurGewissheitundZuverlaessigkeitdermenschlichenErkenntniss1965}.}.
Das heißt hier, dass die Präsumtion der Glaubwürdigkeit des Informanten dann
nicht anzuwenden ist, wenn es Gründe gibt, den natürlichen Zustand der
Wahrhaftigkeit nicht vorauszusetzen. Fälle, in denen sie nicht zur Anwendung
kommen sollte, werden durch -- wie man heute sagt -- \emph{defeaters} bestimmt.
Solche \emph{defeaters} beweisen (oder begründen) nicht das Gegenteil, aber sie
unterminieren die Rechtfertigung einer Behauptung. Dadurch kann der an sich
bereits vernünftige Erwerb testimonialen Wissens zugleich ein kritischer und der
Empfänger einer Information ein mündiger Rezipient werden.
\end{nummerierung}


Um zu erkennen, welcher Art die \emph{defeater} sind, die zu berücksichtigen
sind, müssen wir uns zunächst fragen, welche Dinge überhaupt als
Informationsquelle genutzt werden können. Es erstaunt dabei die Ausführlichkeit,
mit der er über naheliegende Quellen wie mündliche und schriftliche Mitteilungen
hinausgeht:
\begin{quote}
  Die Data zu einer historischen Wahrscheinlichkeit können sehr vielerley seyn.
  Es gehören dazu nicht nur Geschichtbücher und Urkunden, ausführliche und
  beyläufige Zeugnisse und Erzehlungen, sondern auch Denckmale, Müntzen, Bilder,
  Aufschriften, Ueberbleibsale und hinterlassene Folgen, ja auch die innerliche
  Beschaffenheit der erzehlten Sache gehöret mit darzu, ob sie nemlich leicht
  möglich oder an sich vermuthlich ist, oder
  nicht.\footnote{\Cite[][\S~608]{Crusius:WegzurGewissheitundZuverlaessigkeitdermenschlichenErkenntniss1965}.}
\end{quote}
Dies sind die Ausgangspunkte, die zunächst als solche zu würdigen sind. Es ließe
sich hier einwenden, dass die Liste zu weit greift, indem sie Überbleibsel,
Folgen und die innere Beschaffenheit der Sache selbst mit einbezieht.
\authorcite{Crusius:Anweisungvernuenftigzuleben1744} scheint damit den Bereich testimonialen Wissens zu verlassen,
denn wer Fossilien findet und daraus Rückschlüsse auf vergangene Geschehnisse
zieht, der macht etwas ganz anderes als derjenige, der eine Mitteilung erhält
und testimoniales Wissen erwirbt. Dies war die Lehre, die im letzten Kapitel aus
\authorcite{Reid:EssaysontheIntellectualPowersofMan2002}s Unterscheidung von
\emph{solitary} und \emph{social operations of mind} gezogen werden
konnte. Wenn wir auch manchmal bei Fossilien und ähnlichen
Hinweisen auf frühere Geschehnisse von \singlequote{Zeugnissen} (früherer Zeiten) sprechen, so ist dies doch mit der Grundlage testimonialen Wissens nicht zu verwechseln.

Man beachte jedoch, wie \authorcite{Crusius:Anweisungvernuenftigzuleben1744}
diese \enquote{Data} in seiner Darstellung behandelt. Sie werden gerade nicht
einfach mit Informationen aus zweiter Hand und Erzählungen gleichgestellt, sondern in einer
Weise in die Überlegungen einbezogen, die an einen anderen Aspekt der
Darstellung \authorcite{Hume:ATreatiseofHumenNature2007}s erinnert:
\begin{quote}
  Was die Beweiskraft der \ori{Datorum} bey historischen Beweisen anlanget, so
  ist dieselbe theils in der Existenz und Beschaffenheit der Zeugnisse, d.\,i.
  der bezeugten Erfahrung anderer; theils in der Beschaffenheit der Sache selbst,
  die erzehlet wird,
  anzutreffen.\footnote{\Cite[][\S~609]{Crusius:WegzurGewissheitundZuverlaessigkeitdermenschlichenErkenntniss1965}.
  Siehe auch \name[Immanuel]{Kant}s Überlegungen in
  \cite[][\nopp 2589]{Kant:Reflexionen1900ff.},
  \cite[][XVI: 430.14--15]{Kant:GesammelteWerke1900ff.}: \enquote{Also aus der
  Beschaffenheit der Sache und der Beschaffenheit der Zeugen (vielheit, deren einer es nicht von den Andern hat).} }
\end{quote}
Ebenso wie \authorcite{Hume:ATreatiseofHumenNature2007} achtet
\authorcite{Crusius:Anweisungvernuenftigzuleben1744} nicht nur auf den
Informanten und dessen Kompetenz und Glaubwürdigkeit, sondern ebenso darauf, ob
es sich ihrem Inhalt nach um eine glaubwürdige Erzählung handelt. Hier trennt
\authorcite{Crusius:Anweisungvernuenftigzuleben1744} diejenigen
Ausgangspunkte, die genuin testimoniales Wissen begründen, von sonstigen
Ausgangspunkten. Welche Bedeutung letzteren zukommt, wird an folgender Stelle
deutlich:
\begin{quote}
  Man giebt dabey theils auf die Möglichkeit; theils auf die Wahrscheinlichkeit
  Achtung, welche der Sache auch schon zukommt, wiefern man auf die Zeugnisse,
  wodurch sie weiter bestätiget werden soll, noch nicht Acht
  hat.\footnote{\Cite[][\S~609]{Crusius:WegzurGewissheitundZuverlaessigkeitdermenschlichenErkenntniss1965}.}
\end{quote}
Es ist im Grunde nie der Fall, dass die Mitteilung durch andere die einzige
Informationsquelle ist, die uns zur Verfügung steht. Oft hat ein
Ereignis, das uns berichtet wird, weitere Spuren hinterlassen, wie etwa die
Hinterlassenschaften einer bedeutenden Schlacht. Fast immer können wir
einschätzen, wie wahrscheinlich es überhaupt ist, dass etwas so geschah, wie es
uns jemand berichtet. Und selbst wo wir über keine weiteren Informationen
verfügen, kann dies aussagekräftig sein; beispielsweise wenn uns ein
historisches Ereignis berichtet wird, welches -- seltsamer Weise -- keinerlei
Spuren hinterlassen hat. Solche Merkwürdigkeiten sprechen gegen die Annahme, es
habe tatsächlich stattgefunden. Vor allem aber können wir unabhängig von der
Mitteilung einschätzen, ob das, was uns berichtet wird, in sich schlüssig und
allgemein möglich oder sogar wahrscheinlich ist, oder ob die Erzählung etwas
sehr Außergewöhnliches\footnote{Hierzu auch \cite[][\nopp
2589]{Kant:Reflexionen1900ff.}, \cite[][XVI:
430.18]{Kant:GesammelteWerke1900ff.}:
\enquote{Seltene Dinge haben eine unwarscheinlichkeit. e. g. Große Armeen.}},
Unwahrscheinliches oder gar Unmögliches und in sich widersprüchliches behauptet.
Möglicherweise widerspricht sich unser Informant oder er behauptet etwas, was
bekannten Naturgesetzen widerspricht.

Dies war die Pointe der Argumentation \name[David]{Hume}s gegen die
Glaubwürdigkeit von Wunderberichten: Bei jeder Information gilt es nicht nur,
die Informanten zu bewerten, sondern auch den Informationsgehalt anhand
desjenigen Wissens, über welches wir bereits unabhängig von diesen bestimmten
Informanten verfügen.\footnote{Dabei verfügen wir über dieses Wissen nicht zwingend
unabhängig von Informanten überhaupt. Wenn wir eine Information bspw. mit
unserem allgemeinen physikalischen Wissen konfrontieren, dann handelt es sich
dabei in der Regel um Wissen, welches wir selbst aus zweiter Hand haben.} Zwar
mögen wir über Einzelfälle als solche kein vorgängiges Wissen haben, jedoch
liegt uns immer allgemeines Wissen über entsprechende Geschehnisse vor. Um
\name[David]{Hume}s Beispiel zu verwenden: Wenn uns berichtet wird, jemand sei
erst tot gewesen, später jedoch wieder erwacht, dann benötigen wir kein
testimoniales Wissen mit gegenteiligem Inhalt. Wir können unseren Informanten
bereits aufgrund unseres allgemeinen Wissens über die Irreduzibilität des Todes
der Lüge oder des Irrtums bezichtigen. Einen solchen
Umstand können wir insofern zu den \emph{defeaters} zählen, als durch ihn die
Bonitätspräsumtion hinfällig wird -- er hebt die Evidenz auf, die der
Mitteilung als solcher zukommt. Es reicht nicht, um das Gegenteil der bezeugten
Sache zu erweisen, aber es ist doch genug, um Zweifel an der Glaubwürdigkeit der
Information zu erregen. Und genau diese Funktion können wir denjenigen
\enquote{Data} zusprechen, die selbst nicht geeignet sind, testimoniales Wissen
zu begründen, weil sie nicht von der Art einer Mitteilung sind.

Dennoch stellt
\authorcite{Crusius:WegzurGewissheitundZuverlaessigkeitdermenschlichenErkenntniss1965}'
Theorie einen wichtigen Fortschritt gegenüber
\authorcite{Hume:AnEnquiryConcerningHumanUnderstanding1964}s Position dar: Der
Begriff der Präsumtion erlaubt es, mithilfe der \singlequote{defeater} kritische
Momente zu integrieren, ohne annehmen zu müssen, testimoniales Wissen habe die
gleiche Grundlage wie unsere sonstigen Erfahrungserkenntnisse. Ebenso wie
\authorcite{Reid:EssaysontheIntellectualPowersofMan2002} vertritt er eine
credulistische Position, der zufolge wir keine zusätzlichen
rechtfertigungsbedürftigen Prämissen benötigen, um von einer Mitteilung auf die
Wahrheit des Mitgeteilten zu schließen. Aber während
\authorcite{Reid:EssaysontheIntellectualPowersofMan2002} von einem angeborenen
Prinzip spricht, nach dem wir \emph{de facto} handeln, und damit die Frage nach
der Vernünftigkeit außer Acht lässt, geht es
\authorcite{Crusius:WegzurGewissheitundZuverlaessigkeitdermenschlichenErkenntniss1965}
gerade darum, unseren Umgang mit testimonialem Wissen als prinzipiengeleitet und
vernünftig auszuweisen. Freilich kann auch er nicht beweisen, dass allgemein
gilt: Wenn Personen mit der Eigenschaft $XYZ$ uns mitteilen, dass $p$, dann ist
$p$ wahr. Aber er kann erstens begründen, dass es vernünftig ist, zunächst von
der Wahrheit einer Information auszugehen, und zeigt zweitens, dass ein solches
Vorgehen nicht leichtgläubig sein muss.

\section{Zusammenfassung und Ausblick}
Ich habe in diesem Kapitel zwei Arten von Theorien testimonialen Wissens
unterschieden und dazu paradigmatische Darlegungen aus der Philosophie der
Neuzeit skizziert. Individualistische Ansätze behaupten, dass Mitteilungen von
anderen nicht als \emph{primary epistemic link} verstanden werden dürfen, der
uns ohne Umschweife mit Wissen versorgt. Wir können entweder die Möglichkeit
testimonialen Wissens \emph{toto genere} zurückweisen und einen testimonialen
Skeptizismus vertreten, oder aber im Sinne eines testimonialen Reduktionismus
behaupten, dass solche Mitteilungen einen \emph{secondary epistemic link}
darstellen, also nur unter Rekurs auf weitere, begründungsbedürftige Prämissen
Wisen generieren können. \authorfullcite{Descartes:OeuvresdeDescartes1983} und
David \name[David]{Hume} waren Vertreter eines solchen Individualismus.
Nichtindividualistische Ansätze wiederum sehen Mitteilungen als \emph{primary
epistemic link} an, der uns direkt mit Wissen versorgt, ohne dass es dafür einer
weiteren Begründung bedürfte. Der klassische Autor ist
\authorfullcite{Reid:EssaysontheIntellectualPowersofMan2002}; aber mit
\authorfullcite{Crusius:WegzurGewissheitundZuverlaessigkeitdermenschlichenErkenntniss1965}
und seiner Präsumtionstheorie vertritt  auch ein Vertreter der deutschsprachigen
Aufklärung einen solchen Ansatz. Demnach können wir Mitteilungen anderer ohne
Umschweife als Grundlage eigenen Wissenserwerbs ansehen, solange keine Gründe
vorliegen, diese in Zweifel zu ziehen. Im folgenden
\ref{chapter:MuendigerErwerbTestimonialenWissens}. Kapitel soll nun genauer
untersucht werden, was einen solchen Wissenserwerb aus der Sicht von Vertretern
der Aufklärung im Deutschland des 18. Jahrhunderts zu einem
\singlequote{kritischen} Umgang mit testimonialem Wissen macht.
